\section{序言}

本书仅供学习交流。中文版源自鲁羊翻译,未作修改;词汇注解为笔者所加。

本书单词量较我国初中英语课程标准(2022年)所要求的1600词,多出1000生词,适合
拥有高中及以上英语水平的读者阅读。如有不懂,可查词汇表,即使略过、参考中文翻
译也可。英语学习不应是一字一记、一板一眼的学习;条条大路通罗马,大胆体验、尽
情享受英语学习的历程,体验语言自身的魅力吧。

亲爱的读者们,能否发现中文版、词汇注解中的错误呢?开动你们的脑筋噢,可与我邮
件联系。


\clearpage

\section{海明威生平}

以下内容整合自网络,或有不实信息。

欧内斯特·米勒·海明威(英语:Ernest Miller Hemingway,\doulos{ˈhɛmɪŋweɪ};
1899年7月21日—1961年7月2日),美国人、古巴记者和作家,他是20世纪最著名的小
说家之一。

海明威出生于美国伊利诺伊州芝加哥郊区的奥克帕克的一个医生家庭,父亲热衷于户外
活动,尤其是打猎和钓鱼;母亲爱好音乐和艺术,虔信宗教。

据传,海明威刚出生时被母亲当作女孩培养,父亲则希望他走向“阳刚”道路。他10岁
时的生日礼物是父亲给他的一杆猎枪;11岁学抽雪茄;12岁学打猎;14岁就参加各种拳
击比赛;16岁因打猎苍鹭上法庭;18岁因拳击造成左眼视力下降无法参军,成为实习记
者。19岁成为红十字会战地服务团的司机,赶赴第一次世界大战意大利战场,不久后被
炮弹爆炸产生的大量碎片击伤。29岁时其父亲因经济(经济大萧条时期)和健康状况自
杀。41岁时,海明威被苏联内务人民委员会(NKVD,即克格勃的前身)招募,代号为
“阿尔戈”(Argo)。此外,海明威还曾与其他情报机构建立联系,包括美国战略情报
局(OSS,即中央情报局的前身)。他的情报工作并无建树。

40岁至61岁时,海明威在古巴定居,并称自己为“普通的古巴人”。1950年圣诞节后不
久,海明威在古巴哈瓦那郊区的别墅瞭望山庄,(Finca Vigía)动笔写作《老人与海》;
1951年2月23日就完成了初稿,前后仅用时八周,仅结局就修改39次;1952年出版本书。
该书改编自真实故事,剧情讲述一名古巴老渔夫圣地亚哥(Santiago)与一条大马林鱼
的缠斗。该书改编自海明威在1935年得知的真实故事,这也是美国作家欧内斯特·海明
威生前最后一部主要作品。

冰山理论是海明威提出的一个概念,它认为一个故事的真正深度和意义隐藏在表面之下,
如同冰山在海平面之上的部分只有其八分之一,而八分之七隐藏在海平面之下;有力的故
事讲述需要省略明确可经验的细节。在他看来,没有说出来的话和说出来的话一样重要。
它常表现为电报似的行文、背景的省略、充满留白的对话和冷静克制、不带情感色彩的
结尾。这些特点尤其突出地体现在《老人与海》及其他短篇小说中。

他一生中的感情错综复杂,先后结过四次婚,是美国“迷惘的一代”作家中的代表人物,
作品中对生活、世界、社会都表现出了迷茫和彷徨。

冷战时期,因其与苏联和古巴的关系,受到过FBI监视。1961年7月2日,身患抑郁症、
被害妄想症的海明威在家中用双管猎枪自杀,享年62岁。
