\begin{paracol}{2}
\clearpage
\part{ONE}\label{one}

\switchcolumn

\part*{第一部分}

\switchcolumn*

\section{I}

\switchcolumn

\section*{一}\label{ux4e00}

\switchcolumn*

It was a bright cold day in April, and the clocks were striking
thirteen. Winston Smith, his chin nuzzled into his breast in an effort
to escape the vile wind, slipped quickly through the glass doors of
Victory Mansions, though not quickly enough to prevent a swirl of gritty
dust from entering along with him.

\switchcolumn

四月的一天,晴朗而寒冷,时钟敲过十三下,温斯顿•史密斯缩着身子溜进胜利大厦的玻璃门,他动作迅速,却还是慢了一些,风吹起沙土跟着他钻进了屋。

\switchcolumn*

The hallway smelt of boiled cabbage and old rag mats. At one end of it a
colored poster, too large for indoor display, had been tacked to the wall.
It depicted simply an enormous face, more than a meter wide: the face of a
man of about forty-five, with a heavy black mustache and ruggedly handsome
features. Winston made for the stairs. It was no use trying the lift. Even
at the best of times it was seldom working, and at present the electric
current was cut off during daylight hours. It was part of the economy drive
in preparation for Hate Week. The flat was seven flights up, and Winston,
who was thirty-nine and had a varicose ulcer above his right ankle, went
slowly, resting several times on the way. On each landing, opposite the lift
shaft, the poster with the enormous face gazed from the wall. It was one of
those pictures which are so contrived that the eyes follow you about when
you move. \textsc{Big Brother Is Watching You}, the caption beneath it ran.

\switchcolumn

门厅里弥漫着煮白菜和旧垫子的味道,门厅尽头有一张并不适合放在室内的过于巨大的彩色宣传画。宣传画上只有一个一米多宽的庞大面孔。那是一张四十五岁左右的男人的脸,留着浓密的黑色胡须,粗犷而英俊。温斯顿走向楼梯,电梯是坐不上了,就算在最好的情况下,它也很少会开。为了配合仇恨周的节约运动,整个白天都停止供电。温斯顿的公寓在七层,三十九岁的他右脚踝上有一处因静脉曲张引起的溃疡,他只能慢慢地走,不时还要停下来歇一会儿。每层楼梯的平台处都在正对着电梯的墙上贴上了那张巨大的宣传画,画中人似乎在凝视着你,他的视线跟着人一起移动。而画的下方有这样一行字:老大哥在看着你。

\switchcolumn*

Inside the flat a fruity voice was reading out a list of figures which
had something to do with the production of pig iron. The voice came from
an oblong metal plaque like a dulled mirror which formed part of the
surface of the right-hand wall. Winston turned a switch and the voice
sank somewhat, though the words were still distinguishable. The
instrument (the telescreen, it was called) could be dimmed, but there
was no way of shutting it off completely. He moved over to the window: a
smallish, frail figure, the meagerness of his body merely emphasized by
the blue overalls which were the uniform of the Party. His hair was very
fair, his face naturally sanguine, his skin roughened by coarse soap and
blunt razor blades and the cold of the winter that had just ended.

\switchcolumn

公寓里,一个洪亮的声音正在念着和钢铁生产有关的数字。声音来自镶嵌在右墙上的一块类似毛玻璃镜子的长方形金属板中。温斯顿把声音调低,这个装置(被称为电屏)只能将声音调低,不能完全关上。温斯顿走到窗前,蓝色的党员制服让他瘦小的身材看起来愈发单薄。他的发色很浅,面庞红润,皮肤因长期使用劣质肥皂和钝剃须刀变得粗糙。寒冷的冬天才刚刚结束。

\switchcolumn*

Outside, even through the shut window pane, the world looked cold. Down
in the street little eddies of wind were whirling dust and torn paper
into spirals, and though the sun was shining and the sky a harsh blue,
there seemed to be no color in anything except the posters that were
plastered everywhere. The black-mustachio\textquotesingle d face gazed
down from every commanding corner. There was one on the house front
immediately opposite. \textsc{Big Brother Is Watching You}, the caption said,
while the dark eyes looked deep into Winston\textquotesingle s own. Down
at street level another poster, torn at one corner, flapped fitfully in
the wind, alternately covering and uncovering the single word INGSOC. In
the far distance a helicopter skimmed down between the roofs, hovered
for an instant like a bluebottle, and darted away again with a curving
flight. It was the Police Patrol, snooping into people\textquotesingle s
windows. The patrols did not matter, however. Only the Thought Police
mattered.

\switchcolumn

就算隔着紧闭的窗户,外面看上去仍然很冷。街道上,风裹挟着尘土和碎纸呼啸飞旋。虽然阳光明媚,天色湛蓝,所有的一切仍仿佛蜕去了色彩,除了那些随处可见的宣传画。那张留着黑色胡子的脸从各个角落里居高临下地看着。正对面的房子就贴了这样的宣传画,画上的人正用黑色的眼睛盯着温斯顿,``老大哥在看着你''。一张宣传画掉到了地上,它一角损破,被风吹动,发出啪啪的声响,一个单词时隐时现:英社。远远地,一架直升机在屋顶间飞过,像苍蝇一样在空中盘旋,又转了个弯飞走了。那是警察巡逻队正在窥视人们的窗户。但巡逻队并不让人畏惧,让人畏惧的是思想警察。

\switchcolumn*

Behind Winston\textquotesingle s back the voice from the telescreen was
still babbling away about pig iron and the overfulfillment of the Ninth
Three-Year Plan. The telescreen received and transmitted simultaneously.
Any sound that Winston made, above the level of a very low whisper,
would be picked up by it; moreover, so long as he remained within the
field of vision which the metal plaque commanded, he could be seen as
well as heard. There was of course no way of knowing whether you were
being watched at any given moment. How often, or on what system, the
Thought Police plugged in on any individual wire was guesswork. It was
even conceivable that they watched everybody all the time. But at any
rate they could plug in your wire whenever they wanted to. You had to
live---did live, from habit that became instinct---in the assumption
that every sound you made was overheard, and, except in darkness, every
movement scrutinized.

\switchcolumn

在温斯顿的身后,电屏还在没完没了地播放着关于钢铁产量及超额完成三个九年计划的报告。电屏可以同时接收和发送信息,温斯顿发出的任何声响只要比轻轻低语稍大一点,都将被电屏接收。而他不只会被听到,还能被看到,只要他待在这块金属板的特定范围内。当然,你没法知道某时某刻你的一言一行是否处在监视之下。你只能想象思想警察会以怎样的频率、怎样的标准接通某个人的线路,他们很有可能一刻不停地监视着所有人的线路。可以肯定,只要他们想,他们可以在任何时候接上你的线路。你不得不在这样的情况下生活——这就是生活,从习惯变成本能——设想一下,你发出的所有声音都将被监听,除非在黑暗中,你的每个举动都将被监视。

\switchcolumn*

Winston kept his back turned to the telescreen. It was safer; though, as
he well knew, even a back can be revealing. A kilometer away the
Ministry of Truth, his place of work, towered vast and white above the
grimy landscape. This, he thought with a sort of vague distaste---this
was London, chief city of Airstrip One, itself the third most populous
of the provinces of Oceania. He tried to squeeze out some childhood
memory that should tell him whether London had always been quite like
this. Were there always these vistas of rotting nineteenth-century
houses, their sides shored up with balks of timber, their windows
patched with cardboard and their roofs with corrugated iron, their crazy
garden walls sagging in all directions? And the bombed sites where the
plaster dust swirled in the air and the willow herb straggled over the
heaps of rubble; and the places where the bombs had cleared a larger
patch and there had sprung up sordid colonies of wooden dwellings like
chicken houses? But it was no use, he could not remember: nothing
remained of his childhood except a series of bright-lit tableaux,
occurring against no background and mostly unintelligible.

\switchcolumn

温斯顿保持着背对电屏的姿势,这样相对安全。但他清楚,就算背对电屏仍有可能泄露信息。一公里外就是他工作的地方——真理部,那是一幢高大的白色建筑,高高地耸立在一片脏兮兮的区域内。这里,他有些厌恶地想着——这就是伦敦,大洋国第三人口大省一号空降带的主要城市。他努力在记忆中搜寻关于童年的蛛丝马迹,难道伦敦一直是这样吗?到处都是19世纪的破房子,墙的侧面靠木架子支撑,窗户上钉着硬纸板,屋顶上铺着皱巴巴的铁皮,就连花园的围墙也已经东倒西歪?还有那些经历了空袭的地方,灰尘飞舞,野花在残砖断瓦间生长,许多像鸡舍一样肮脏的木板房突然出现在爆炸后的空地上!但他的努力是徒劳的,他什么也想不起来,除了一片明亮的、背景难辨的、让人无法理解的画面,他的童年记忆里什么都没有。

\switchcolumn*

The Ministry of Truth---Minitrue, in Newspeak\footnote{Newspeak was the
  official language of Oceania. For an account of its structure and
  etymology, see Appendix.}---was startlingly different from any other
object in sight. It was an enormous pyramidal structure of glittering white
concrete, soaring up, terrace after terrace, three hundred meters into the
air. From where Winston stood it was just possible to read, picked out on
its white face in elegant lettering, the three slogans of the Party:

\switchcolumn

真理部——用新话来说就是``真部''——和人们看到的所有建筑不同。它是幢金字塔结构的庞然大物,白色的水泥闪闪发亮。它一层一层地交叠着,高高耸起,足有三百多米。站在温斯顿所在的地方,刚好可以看到被人用漂亮字体写在白色墙壁上的三条党的标语:

\switchcolumn*

\headline{WAR IS PEACE\\
FREEDOM IS SLAVERY\\
IGNORANCE IS STRENGTH.}

\switchcolumn

\headline{战争即和平\\
  自由即奴役\\
无知即力量}

\switchcolumn*

The Ministry of Truth contained, it was said, three thousand rooms above
ground level, and corresponding ramifications below. Scattered about
London there were just three other buildings of similar appearance and
size. So completely did they dwarf the surrounding architecture that
from the roof of Victory Mansions you could see all four of them
simultaneously. They were the homes of the four Ministries between which
the entire apparatus of government was divided: the Ministry of Truth,
which concerned itself with news, entertainment, education, and the fine
arts; the Ministry of Peace, which concerned itself with war; the
Ministry of Love, which maintained law and order; and the Ministry of
Plenty, which was responsible for economic affairs. Their names, in
Newspeak: Minitrue, Minipax, Miniluv, and Miniplenty.

\switchcolumn

据说,真理部光是在地面上就有三千个房间,在地面下还有相应的附属建筑。整个伦敦只有三座外形和规模与之类似的大楼。这些大楼将它们周围的建筑映衬得渺小不堪。站在胜利大厦的顶部,可以同时看到这四座大楼,即四个部的所在地,政府的所有职能都分布在这四个部中。

\switchcolumn*

The Ministry of Love was the really frightening one. There were no
windows in it at all. Winston had never been inside the Ministry of
Love, nor within half a kilometer of it. It was a place impossible to
enter except on official business, and then only by penetrating through
a maze of barbed-wire entanglements, steel doors, and hidden machine-gun
nests. Even the streets leading up to its outer barriers were roamed by
gorilla-faced guards in black uniforms, armed with jointed truncheons.

\switchcolumn

真理部负责新闻、娱乐、教育、艺术;和平部负责战争;仁爱部负责法律和秩序;富足部负责经济。这四个部的名字若用新话来讲就是``真部''、``和部''、``爱部''和``富部''。其中,仁爱部是最令人战栗的一个,整幢大楼没有一扇窗户。温斯顿从来没有去过仁爱部,甚至不曾踏入距离它半公里的区域。这是一个非公莫入的地方,即使要进去,也要穿过遍布着铁丝网、设立着铁门和机关枪的道路。而在仁爱部外面的大街上,还有穿着黑色制服、带着警棍的警察,他们在街上巡逻,个个面目可怖。

\switchcolumn*

Winston turned round abruptly. He had set his features into the
expression of quiet optimism which it was advisable to wear when facing
the telescreen. He crossed the room into the tiny kitchen. By leaving
the Ministry at this time of day he had sacrificed his lunch in the
canteen, and he was aware that there was no food in the kitchen except a
hunk of dark-colored bread which had got to be saved for
tomorrow\textquotesingle s breakfast. He took down from the shelf a
bottle of colorless liquid with a plain white label marked \textsc{VICTORY
GIN}. It gave off a sickly, oily smell, as of Chinese rice-spirit.
Winston poured out nearly a teacupful, nerved himself for a shock, and
gulped it down like a dose of medicine.

\switchcolumn

温斯顿突然转过身来,换上一副心满意足的快乐表情,这是面对电屏时的明智之举。他穿过房间来到小厨房。在这个时间离开真理部,他无法在食堂吃午饭。而他也清楚,他的厨房里只有一块要留到明天早上作早餐的黑面包。于是,他从架子上拿起一个贴着白色标签的瓶子,瓶子里装着无色的液体:胜利杜松子酒。这种酒散发着令人恶心的汽油味,类似中国的白酒。温斯顿差不多倒了一茶杯,然后鼓足勇气,像喝药那样灌了下去。

\switchcolumn*

Instantly his face turned scarlet and the water ran out of his eyes. The
stuff was like nitric acid, and moreover, in swallowing it one had the
sensation of being hit on the back of the head with a rubber club. The
next moment, however, the burning in his belly died down and the world
began to look more cheerful. He took a cigarette from a crumpled packet
marked \textsc{VICTORY CIGARETTES} and incautiously held it upright, whereupon
the tobacco fell out onto the floor. With the next he was more
successful. He went back to the living room and sat down at a small
table that stood to the left of the telescreen. From the table drawer he
took out a penholder, a bottle of ink, and a thick, quarto-sized blank
book with a red back and a marbled cover.

\switchcolumn


他的脸立刻红了起来,泪水从眼角渗出。这东西的味道好像硝酸,喝的时候让人觉得后
脑勺仿佛被橡皮棍狠狠地敲了一下。不过,过了一会儿,当胃里的灼烧感退去,世界看
起来可爱多了。温斯顿从印有``胜利香烟''字样的小匣子中抽出一根烟,不小心将它拿
竖了,烟丝掉到地上,他只好再抽出一根,这次成功了。他回到起居室,在电屏左边的
小桌子前坐下,从抽屉里拿出一根笔杆、一瓶墨水和一个四开大小的空白本子。本子的
封皮是红色的,上面还压有大理石花纹。

\switchcolumn*

For some reason the telescreen in the living room was in an unusual
position. Instead of being placed, as was normal, in the end wall, where
it could command the whole room, it was in the longer wall, opposite the
window. To one side of it there was a shallow alcove in which Winston
was now sitting, and which, when the flats were built, had probably been
intended to hold bookshelves. By sitting in the alcove, and keeping well
back, Winston was able to remain outside the range of the telescreen, so
far as sight went. He could be heard, of course, but so long as he
stayed in his present position he could not be seen. It was partly the
unusual geography of the room that had suggested to him the thing that
he was now about to do.

\switchcolumn

通常电屏会被安装在房间最里面的墙上,方便监视整个房间,但不知为什么,温斯顿起居室的电屏却被装在了正对着窗户的较长的那面墙上。电屏的一边有个凹进去的地方,大概是建房子时计划留出来放书柜的,温斯顿就坐在这里,他尽可能地向后靠过去,让自己待在电屏的监视范围外。当然,他的声音还是可以被听到,但只要他待在现在这个位置,他就可以不被看到。这房子与众不同的格局让他想到了这点。

\switchcolumn*

But it had also been suggested by the book that he had just taken out of
the drawer. It was a peculiarly beautiful book. Its smooth creamy paper,
a little yellowed by age, was of a kind that had not been manufactured
for at least forty years past. He could guess, however, that the book
was much older than that. He had seen it lying in the window of a frowsy
little junk shop in a slummy quarter of the town (just what quarter he
did not now remember) and had been stricken immediately by an
overwhelming desire to possess it. Party members were supposed not to go
into ordinary shops (``dealing on the free market,'' it was called), but
the rule was not strictly kept, because there were various things such
as shoelaces and razor blades which it was impossible to get hold of in
any other way. He had given a quick glance up and down the street and
then had slipped inside and bought the book for two dollars fifty. At
the time he was not conscious of wanting it for any particular purpose.
He had carried it guiltily home in his brief case. Even with nothing
written in it, it was a compromising possession.

\switchcolumn

而他之所以要这样做还和他刚刚拿出的那个本子有关。这是个漂亮的本子。纸质细致而光滑,由于年代久远的关系——这种纸至少有四十年没再生产了——纸张有些泛黄。但他猜测这本子的年代很可能比四十年还要长。他是在一间散发着臭味的小旧货店的橱窗里看到它的。这家店就坐落在一个破败的街区里(他也记不清具体是哪个区)。看到它的第一眼,他就遏制不住要拥有它的欲望。虽然党员不允许进入普通商店(即``自由市场的买卖''),但这个规定并没有被严格执行,因为有太多东西,比如鞋带和刮胡刀的刀片,不到这里就无法买到。他迅速地向街道两旁看了几眼,然后钻到店里,花了两块五毛钱将它买下,他还没想到要拿它来做什么。他将它收在公文包里,怀着惴惴不安的心情回了家,就算本子上一字未写,也可能引起怀疑。

\switchcolumn*

The thing that he was about to do was to open a diary. This was not
illegal (nothing was illegal, since there were no longer any laws), but
if detected it was reasonably certain that it would be punished by
death, or at least by twenty-five years in a forced-labor camp. Winston
fitted a nib into the penholder and sucked it to get the grease off. The
pen was an archaic instrument, seldom used even for signatures, and he
had procured one, furtively and with some difficulty, simply because of
a feeling that the beautiful creamy paper deserved to be written on with
a real nib instead of being scratched with an ink pencil. Actually he
was not used to writing by hand. Apart from very short notes, it was
usual to dictate everything into the speakwrite, which was of course
impossible for his present purpose. He dipped the pen into the ink and
then faltered for just a second. A tremor had gone through his bowels.
To mark the paper was the decisive act. In small clumsy letters he
wrote:

\switchcolumn

他准备写日记,这件事并不违法(没有什么事会违法,因为不再有法律),但一旦被发现却有可能被处以死刑,或者至少被关到劳动营里服上二十五年徒刑。温斯顿将钢笔尖装到笔杆上,又在笔尖上舔了舔,清掉了上面的油。钢笔已经是古旧的东西,就算是签名也很少会用到,他私下里花了很大力气才弄到一支,而这样做仅仅因为他觉得只有用钢笔写字才配得上那漂亮、细致的纸张,他不能用蘸水笔在这样的纸上划抹。事实上,他并不习惯用手写字,除了一些便签,一般情况下他都是对着语音记录器口述事情,而他眼下要做的事当然不能用语音记录器。他拿起笔,在墨水里蘸了蘸,又迟疑了片刻。一种震颤的感觉充斥了他的身体,在纸上写上标题有着决定性的意义。他用纤小笨拙的字体写道:

\switchcolumn*


\begin{quotation}
  April 4th, 1984.
\end{quotation}

\switchcolumn

\begin{quotation}
一九八四年四月四日
\end{quotation}

\switchcolumn*

He sat back. A sense of complete helplessness had descended upon him. To
begin with, he did not know with any certainty that this \emph{was} 1984. It
must be round about that date, since he was fairly sure that his age was
thirty-nine, and he believed that he had been born in 1944 or 1945; but it
was never possible nowadays to pin down any date within a year or two.

\switchcolumn

他的身子向后靠去,整个人都被无助感吞噬。首先,他不能确定今年是不是1984年,他能肯定的是他今年三十九岁,并且他知道自己是在1944年或1945年出生。但是现在要想准确地确认年份,没有一两年的误差是不可能的。

\switchcolumn*

For whom, it suddenly occurred to him to wonder, was he writing this
diary? For the future, for the unborn. His mind hovered for a moment
round the doubtful date on the page, and then fetched up with a bump
against the Newspeak word \emph{doublethink}. For the first time the
magnitude of what he had undertaken came home to him. How could you
communicate with the future? It was of its nature impossible. Either the
future would resemble the present, in which case it would not listen to
him, or it would be different from it, and his predicament would be
meaningless.

\switchcolumn

突然,他想起一个问题,他究竟为了什么写日记。为未来?为那些尚未出世的人?有那么一会儿,他的心思都放在了那令人生疑的年份上。他猛地想起新话中的一个名词``双重思想''。这是他第一次意识到自己正在做一件艰巨的事。人要如何和未来交流?这是不可能的。未来若和现在类似,未来不会听他说话;未来若和现在不同,他的预言就失去了意义。

\switchcolumn*

For some time he sat gazing stupidly at the paper. The telescreen had
changed over to strident military music. It was curious that he seemed
not merely to have lost the power of expressing himself, but even to
have forgotten what it was that he had originally intended to say. For
weeks past he had been making ready for this moment, and it had never
crossed his mind that anything would be needed except courage. The
actual writing would be easy. All he had to do was to transfer to paper
the interminable restless monologue that had been running inside his
head, literally for years. At this moment, however, even the monologue
had dried up. Moreover, his varicose ulcer had begun itching unbearably.
He dared not scratch it, because if he did so it always became inflamed.
The seconds were ticking by. He was conscious of nothing except the
blankness of the page in front of him, the itching of the skin above his
ankle, the blaring of the music, and a slight booziness caused by the
gin.

\switchcolumn

有那么一会儿,他对着纸发呆。电屏里播着刺耳的军乐。奇怪的是,他看上去不只失去了表达自我的勇气,还忘记了他要表达的事情。过去的几个星期里,他一直在为这一刻做准备,他没有想过除了勇气他还需要什么。其实,写东西很容易,他只需要将多年以来一直盘桓在脑海里的那些没完没了的、焦躁的内心独白放到纸上就可以了。但在这一刻,这内心独白竟枯竭了。更何况脚上那静脉曲张引起的溃疡开始发作,痒得他难以忍受,他又不敢去抓,一抓就会红肿发炎。时间一点点地过去,面对着空空白纸,他只能感觉到脚上的瘙痒,电屏音乐的刺耳以及酒后的醉意。

\switchcolumn*

Suddenly he began writing in sheer panic, only imperfectly aware of what
he was setting down. His small but childish handwriting straggled up and
down the page, shedding first its capital letters and finally even its
full stops:

\switchcolumn

突然,他慌张地写了起来,并不清楚自己究竟写了什么。他用孩子般幼小的字体在纸上肆意书写,先是忽略了大写字母,最后竟连标点也略过了:

\switchcolumn*

\begin{quotation}
April 4th, 1984. Last night to the flicks. All war films. One very
good one of a ship full of refugees being bombed somewhere in the
Mediterranean. Audience much amused by shots of a great huge fat man
trying to swim away with a helicopter after him. first you saw him
wallowing along in the water like a porpoise, then you saw him through
the helicopters gunsights, then he was full of holes and the sea round
him turned pink and he sank as suddenly as though the holes had let in
the water. audience shouting with laughter when he sank. then you saw a
lifeboat full of children with a helicopter hovering over it. there was
a middle-aged woman might have been a jewess sitting up in the bow with
a little boy about three years old in her arms. little boy screaming
with fright and hiding his head between her breasts as if he was trying
to burrow right into her and the woman putting her arms round him and
comforting him although she was blue with fright herself. all the time
covering him up as much as possible as if she thought her arms could
keep the bullets off him. then the helicopter planted a 20 kilo bomb in
among them terrific flash and the boat went all to matchwood. then there
was a wonderful shot of a childs arm going up up up right up into the
air a helicopter with a camera in its nose must have followed it up and
there was a lot of applause from the party seats but a woman down in the
prole part of the house suddenly started kicking up a fuss and shouting
they didnt oughter of showed it not in front of kids they didnt it aint
right not in front of kids it aint until the police turned her turned
her out i dont suppose anything happened to her nobody cares what the
proles say typical prole reaction they never---
\end{quotation}

\switchcolumn

\begin{quotation}
1984年4月4日。昨晚看了电影,都是战争片。其中一部非常好看,一艘载着难民的轮船在地中海的某个地方遇到轰炸。看着胖男人在大海中拼命游泳试图逃脱追赶他的直升机,观众们非常开心。一开始,这男人就像一头海豚在波涛中起伏,直升机上的人通过瞄准器发现了他,紧接着他的身上便布满枪眼,周围的海水也被他的血染成了粉红色,他的身体突然下沉,就好像海水从枪眼里灌了进去。在他下沉的时候,观众们哄堂大笑。之后人们看到一条满是孩子的救生艇待在盘旋着的直升机的下方。一个似乎是犹太裔的中年女人抱着三岁大的小男孩坐在船头。孩子吓坏了,号啕大哭,把头深深地埋进她的怀里,就好像要钻到她身体里,她则安慰着他,用双手将他环住,可她自己也因为恐惧面色发青。她以为她可以用手臂挡住子弹,保护她的孩子。直升机飞过来投下一颗二十公斤的炸弹,伴随着一道剧烈的光芒,小艇变成碎片。接下来的镜头非常清晰:孩子的手臂被炸得高高的,直升机前的摄影机一直追着它在拍。从党员的座位区传来一片掌声,群众区里却突然站出来一个女人,那女人嚷嚷着说这电影不应该放给孩子看,他们做得不对。直到警察赶来将她押了出去。我不觉得她会出什么事,没有人会关心群众说什么,群众的典型反应就是他们从来不\ldots\ldots{}
\end{quotation}

\switchcolumn*

Winston stopped writing, partly because he was suffering from cramp. He
did not know what had made him pour out this stream of rubbish. But the
curious thing was that while he was doing so a totally different memory
had clarified itself in his mind, to the point where he almost felt
equal to writing it down. It was, he now realized, because of this other
incident that he had suddenly decided to come home and begin the diary
today.

\switchcolumn

温斯顿停下笔,一方面因为他感到肌肉在抽搐,他不知道为什么从他笔端倾泻的是这些垃圾。而奇怪的是,他忽然清楚地回想起一件毫不相关的事,他觉得自己完全有能力将这件事写出来。他想起来正是这件事让他突然萌生了回家写日记的念头。

\switchcolumn*

It had happened that morning at the Ministry, if anything so nebulous
could be said to happen.

\switchcolumn

假使这模模糊糊的事确实发生过,那它就发生在这天上午,在部里。

\switchcolumn*

It was nearly eleven hundred, and in the Records Department, where
Winston worked, they were dragging the chairs out of the cubicles and
grouping them in the center of the hall, opposite the big telescreen, in
preparation for the Two Minutes Hate. Winston was just taking his place
in one of the middle rows when two people whom he knew by sight, but had
never spoken to, came unexpectedly into the room. One of them was a girl
whom he often passed in the corridors. He did not know her name, but he
knew that she worked in the Fiction Department. Presumably---since he
had sometimes seen her with oily hands and carrying a spanner---she had
some mechanical job on one of the novel-writing machines. She was a
bold-looking girl of about twenty-seven, with thick dark hair, a
freckled face, and swift, athletic movements. A narrow scarlet sash,
emblem of the Junior Anti-Sex League, was wound several times round the
waist of her overalls, just tightly enough to bring out the shapeliness
of her hips. Winston had disliked her from the very first moment of
seeing her. He knew the reason. It was because of the atmosphere of
hockey fields and cold baths and community hikes and general
clean-mindedness which she managed to carry about with her. He disliked
nearly all women, and especially the young and pretty ones. It was
always the women, and above all the young ones, who were the most
bigoted adherents of the Party, the swallowers of slogans, the amateur
spies and nosers-out of unorthodoxy. But this particular girl gave him
the impression of being more dangerous than most. Once when they passed
in the corridor she had given him a quick sidelong glance which seemed
to pierce right into him and for a moment had filled him with black
terror. The idea had even crossed his mind that she might be an agent of
the Thought Police. That, it was true, was very unlikely. Still, he
continued to feel a peculiar uneasiness, which had fear mixed up in it
as well as hostility, whenever she was anywhere near him.

\switchcolumn

将近11点温斯顿工作的记录司为了给两分钟仇恨会做准备,大家纷纷将椅子从隔间里往外拉,直拉到大厅中央,正对着大型电屏。温斯顿刚要坐在中间一排的某个位置上,两个和他有点头之交的人出人意料地走了过来。其中一个是个女孩,经常和他在走廊里擦肩而过,他叫不出她的名字,只知道她可能在小说司工作,她经常满手油污地拿着扳手——她负责为某个正在写长篇小说的部长维修写作机。她看上去胆子很大,大约二十七岁,有着一头浓密的黑发和长满雀斑的脸,她行动敏捷,就像运动员一样。她的腰间系着一条窄窄的鲜红色的饰带,那是青少年反性同盟的标志,饰带在她的工作服上绕了几圈,缠得刚刚好,巧妙地衬托出她臀部的线条。第一眼看到她,温斯顿就心生厌恶,他知道这是为什么。因为他努力制造出类似曲棍球场、冷水浴、集体远足那样排除杂念的氛围。几乎所有女人他都厌恶,特别是年轻貌美的。女人,尤其是年轻女人,总是盲目地追随着党,她们不假思索地接受党的口号,无偿地侦察那些异端思想。他觉得这个女孩比别的女人更加危险。一次,他们在走廊里相遇,她斜着眼睛迅速地打量了他一番,那眼神仿佛刺透了他的身体,瞬间将黑色的恐惧注入其中。他闪过一个念头:或许,她是思想警察。尽管这种可能性很小,接近她,仍会让他不舒服,而这不适感既包含敌意也包含恐惧。

\switchcolumn*

The other person was a man named O\textquotesingle Brien, a member of
the Inner Party and holder of some post so important and remote that
Winston had only a dim idea of its nature. A momentary hush passed over
the group of people round the chairs as they saw the black overalls of
an Inner Party member approaching. O\textquotesingle Brien was a large,
burly man with a thick neck and a coarse, humorous, brutal face. In
spite of his formidable appearance he had a certain charm of manner. He
had a trick of resettling his spectacles on his nose which was curiously
disarming---in some indefinable way, curiously civilized. It was a
gesture which, if anyone had still thought in such terms, might have
recalled an eighteenth-century nobleman offering his snuffbox. Winston
had seen O\textquotesingle Brien perhaps a dozen times in almost as many
years. He felt deeply drawn to him, and not solely because he was
intrigued by the contrast between
O\textquotesingle Brien\textquotesingle s urbane manner and his
prizefighter\textquotesingle s physique. Much more it was because of a
secretly held belief---or perhaps not even a belief, merely a
hope---that O\textquotesingle Brien\textquotesingle s political
orthodoxy was not perfect. Something in his face suggested it
irresistibly. And again, perhaps it was not even unorthodoxy that was
written in his face, but simply intelligence. But at any rate he had the
appearance of being a person that you could talk to, if somehow you
could cheat the telescreen and get him alone. Winston had never made the
smallest effort to verify this guess; indeed, there was no way of doing
so. At this moment O\textquotesingle Brien glanced at his wristwatch,
saw that it was nearly eleven hundred, and evidently decided to stay in
the Records Department until the Two Minutes Hate was over. He took a
chair in the same row as Winston, a couple of places away. A small,
sandy-haired woman who worked in the next cubicle to Winston was between
them. The girl with dark hair was sitting immediately behind.

\switchcolumn

另外一个是个男人,名叫奥布兰,是内党党员,有着重要且隐蔽的职务。温斯顿对他的
职务只模模糊糊地知道个大概。看着这个身穿黑色制服的内党党员走过来,人群迅速安
静下来。奥布兰是个大个子,体格壮实,他脖子短粗,面容粗糙、冷酷又幽默。尽管模
样令人畏惧,举止却很有魅力。他有个小动作——推一推鼻梁上的眼镜,不知道为什么,
这能让人放下对他的戒备——很难说清也很奇怪,这动作让人觉得他很文雅。若一直想
下去,人们会联想到18世纪的贵族用鼻烟壶款待宾客。许多年来,温斯顿只见过奥布兰
十几次,对他非常感兴趣,这不只因为奥布兰优雅的举止和拳击手般的外表反差强烈,
而是温斯顿有个秘密的信念——也许还算不上信念,他多少希望奥布兰在政治方面不会
那么正统。奥布兰的表情让人不由自主地得出这个结论。也许,他脸上呈现的不是``非
正统'',而仅仅是``聪明''。但不管怎样,看着他的外表,你会萌生躲开电屏,和他单
独聊聊的念头。温斯顿从来没有耗费半点力气证实这个猜测,事实上,他也没法去证实。
奥布兰低头看了看手表,快11点了,显然他决定待在记录司直到两分钟仇恨会结束。他
搬来一把椅子,坐在温斯顿那排,两人之间只隔了两个位子,夹在他们中间的是个淡茶
色头发的身材矮小的女人,她就在温斯顿隔壁的办公室工作。至于那黑头发的女孩则刚
好坐在他们的后面。

\switchcolumn*

The next moment a hideous, grinding screech, as of some monstrous
machine running without oil, burst from the big telescreen at the end of
the room. It was a noise that set one\textquotesingle s teeth on edge
and bristled the hair at the back of one\textquotesingle s neck. The
Hate had started.

\switchcolumn

很快,大厅尽头的电屏里发出了可怕的演讲声,那声音非常难听,好像一台正在运作的大型机器在没有加油的情况下发出的噪音。这声音让人咬牙切齿,寒毛直立。仇恨会开始了。

\switchcolumn*

As usual, the face of Emmanuel Goldstein, the Enemy of the People, had
flashed onto the screen. There were hisses here and there among the
audience. The little sandy-haired woman gave a squeak of mingled fear
and disgust. Goldstein was the renegade and backslider who once, long
ago (how long ago, nobody quite remembered), had been one of the leading
figures of the Party, almost on a level with Big Brother himself, and
then had engaged in counter-revolutionary activities, had been condemned
to death, and had mysteriously escaped and disappeared. The program of
the Two Minutes Hate varied from day to day, but there was none in which
Goldstein was not the principal figure. He was the primal traitor, the
earliest defiler of the Party\textquotesingle s purity. All subsequent
crimes against the Party, all treacheries, acts of sabotage, heresies,
deviations, sprang directly out of his teaching. Somewhere or other he
was still alive and hatching his conspiracies: perhaps somewhere beyond
the sea, under the protection of his foreign paymasters; perhaps
even---so it was occasionally rumored---in some hiding place in Oceania
itself.

\switchcolumn

像往常一样,屏幕上出现了人民公敌埃曼纽尔•高德斯坦因的脸。观众纷纷发出嘘声,淡茶色头发的女人还尖叫起来,叫声中混杂着恐惧与憎恶。高德斯坦因是叛徒,是堕落的人。很久以前(没有人记得究竟有多久)他是党的领导者,几乎和老大哥平起平坐。然后他参加了反革命活动,被判处死刑,但他又神秘脱逃,销声匿迹。两分钟仇恨会的内容每天都不一样,但每次都把高德斯坦因当做重中之重。他是头号叛徒,是最早玷污党纯洁性的人。后来的一切反党行为、一切破坏行动、一切异端思想和违规越轨的行为都是他教唆的。他仍然活着,仍在策划阴谋,他也许在国外,躲在外国人的庇护下,也许在国内,藏匿在某个地方——时不时就会传出这样的谣言。

\switchcolumn*

Winston\textquotesingle s diaphragm was constricted. He could never see
the face of Goldstein without a painful mixture of emotions. It was a
lean Jewish face, with a great fuzzy aureole of white hair and a small
goatee beard---a clever face, and yet somehow inherently despicable,
with a kind of senile silliness in the long thin nose near the end of
which a pair of spectacles was perched. It resembled the face of a
sheep, and the voice, too, had a sheeplike quality. Goldstein was
delivering his usual venomous attack upon the doctrines of the
Party---an attack so exaggerated and perverse that a child should have
been able to see through it, and yet just plausible enough to fill one
with an alarmed feeling that other people, less level-headed than
oneself, might be taken in by it. He was abusing Big Brother, he was
denouncing the dictatorship of the Party, he was demanding the immediate
conclusion of peace with Eurasia, he was advocating freedom of speech,
freedom of the press, freedom of assembly, freedom of thought, he was
crying hysterically that the revolution had been betrayed---and all this
in rapid polysyllabic speech which was a sort of parody of the habitual
style of the orators of the Party, and even contained Newspeak words:
more Newspeak words, indeed, than any Party member would normally use in
real life. And all the while, lest one should be in any doubt as to the
reality which Goldstein\textquotesingle s specious claptrap covered,
behind his head on the telescreen there marched the endless columns of
the Eurasian army---row after row of solid-looking men with
expressionless Asiatic faces, who swam up to the surface of the screen
and vanished, to be replaced by others exactly similar. The dull,
rhythmic tramp of the soldiers\textquotesingle{} boots formed the
background to Goldstein\textquotesingle s bleating voice.

\switchcolumn

温斯顿觉得自己的心脏正在缩紧。每当看到高德斯坦因的脸,他就心情复杂,这让他非常痛苦。高德斯坦因长着一张瘦削的犹太人的面孔,白发蓬松,还蓄着一小撮山羊胡——这是张聪明人的脸,但这脸又惹人生厌,鼻子又细又长,鼻梁上架着眼镜,看上去既衰老又愚蠢。他的脸让人想起绵羊,甚至连他的声音也和绵羊相似。高德斯坦因对党进行攻击,他夸大其词,蛮不讲理,言辞恶毒,这套把戏就连孩子都能看穿,可听上去又有几分道理,人们不得不提高警惕。若头脑不够清醒,很容易就被他蛊惑。高德斯坦因辱骂老大哥,抨击党独裁,要求立即和欧亚国和谈,还要求言论自由、新闻自由、集会自由、思想自由。他情绪激动地叫喊着,说革命遭到了背叛——在讲这些话时,他语速极快,多使用多音节词。这分明是对党的演说风格的拙劣模仿,他甚至使用了新话——毫无疑问,比任何党员在日常生活中讲到的新话都要多。在他说话的时候,为了防止有人被他似乎有些道理的花言巧语蒙蔽,电屏上,他的脑袋后面可以看到数不清的欧亚国士兵正排着纵队前进——这些身材壮硕的士兵都长着典型的亚洲人的脸,都面无表情,他们一队接一队地走着,绵绵不断地出现在电屏上,每个人看起来都一样。他们用沉重又富有节奏感的脚步声衬托着高德斯坦因的那类似绵羊的喊叫。

\switchcolumn*

Before the Hate had proceeded for thirty seconds, uncontrollable
exclamations of rage were breaking out from half the people in the room.
The self-satisfied sheeplike face on the screen, and the terrifying
power of the Eurasian army behind it, were too much to be borne;
besides, the sight or even the thought of Goldstein produced fear and
anger automatically. He was an object of hatred more constant than
either Eurasia or Eastasia, since when Oceania was at war with one of
these powers it was generally at peace with the other. But what was
strange was that although Goldstein was hated and despised by everybody,
although every day, and a thousand times a day, on platforms, on the
telescreen, in newspapers, in books, his theories were refuted, smashed,
ridiculed, held up to the general gaze for the pitiful rubbish that they
were---in spite of all this, his influence never seemed to grow less.
Always there were fresh dupes waiting to be seduced by him. A day never
passed when spies and saboteurs acting under his directions were not
unmasked by the Thought Police. He was the commander of a vast shadowy
army, an underground network of conspirators dedicated to the overthrow
of the State. The Brotherhood, its name was supposed to be. There were
also whispered stories of a terrible book, a compendium of all the
heresies, of which Goldstein was the author and which circulated
clandestinely here and there. It was a book without a title. People
referred to it, if at all, simply as \emph{the book}. But one knew of
such things only through vague rumors. Neither the Brotherhood nor
\emph{the book} was a subject that any ordinary Party member would
mention if there was a way of avoiding it.

\switchcolumn

仇恨会只进行了半分钟,房间里就有一半人按捺不住愤怒发出咆哮。那扬扬得意的绵羊脸以及这绵羊脸后欧亚国震慑人心的力量都让人无法忍受。另一方面,只要看到高德斯坦因的脸,或者仅仅是想象一下他的样子,人们就不禁恐惧、愤怒。相比欧亚国或东亚国,他更频繁地成为人们仇恨的对象,因为无论大洋国和这两国中的哪一个开战,通常,它都要和另一国保持和平。但奇怪的是,虽然所有人都仇恨、鄙视高德斯坦因,虽然他的理论每天甚至一天上千次地在讲台上、在电屏内、在报纸里、在书本中遭到批驳、攻击、嘲弄,被当做毫无价值的垃圾。他的影响似乎从未减弱过,总会有一些新的什么人上当受骗,每天思想警察都会抓到受他指使的破坏分子和间谍。他是影子部队的领导者,这支部队由试图推翻政府的阴谋者构成,是个庞大的地下网络,即传说中的``兄弟会''。此外,还有一种秘密的说法称有一本收录了各种异端邪说的可怕的书在暗中散发。这本书由高德斯坦因撰写,没有名字,人们提起它时只会说``那本书''。不过,这些都是道听途说,一般的党员都尽可能不去提兄弟会和那本书。

\switchcolumn*

In its second minute the Hate rose to a frenzy. People were leaping up
and down in their places and shouting at the tops of their voices in an
effort to drown the maddening bleating voice that came from the screen.
The little sandy-haired woman had toned bright pink, and her mouth was
opening and shutting like that of a landed fish. Even
O\textquotesingle Brien\textquotesingle s heavy face was flushed. He was
sitting very straight in his chair, his powerful chest swelling and
quivering as though he were standing up to the assault of a wave. The
dark-haired girl behind Winston had begun crying out ``Swine! Swine!
Swine!'' and suddenly she picked up a heavy Newspeak dictionary and flung
it at the screen. It struck Goldstein\textquotesingle s nose and bounced
off; the voice continued inexorably. In a lucid moment Winston found
that he was shouting with the others and kicking his heel violently
against the rung of his chair. The horrible thing about the Two Minutes
Hate was not that one was obliged to act a part, but that it was
impossible to avoid joining in. Within thirty seconds any pretense was
always unnecessary. A hideous ecstasy of fear and vindictiveness, a
desire to kill, to torture, to smash faces in with a sledge hammer,
seemed to flow through the whole group of people like an electric
current, turning one even against one\textquotesingle s will into a
grimacing, screaming lunatic. And yet the rage that one felt was an
abstract, undirected emotion which could be switched from one object to
another like the flame of a blowlamp. Thus, at one moment
Winston\textquotesingle s hatred was not turned against Goldstein at
all, but, on the contrary, against Big Brother, the Party, and the
Thought Police; and at such moments his heart went out to the lonely,
derided heretic on the screen, sole guardian of truth and sanity in a
world of lies. And yet the very next instant he was at one with the
people about him, and all that was said of Goldstein seemed to him to be
true. At those moments his secret loathing of Big Brother changed into
adoration, and Big Brother seemed to tower up, an invincible, fearless
protector, standing like a rock against the hordes of Asia, and
Goldstein, in spite of his isolation, his helplessness, and the doubt
that hung about his very existence, seemed like some sinister enchanter,
capable by the mere power of his voice of wrecking the structure of
civilization.

\switchcolumn

仇恨会进行到两分钟时,人们陷入狂热。他们跳起来,在座位上尽情叫喊,试图盖过电屏里传来的发狂一般的羊叫声。那个浅茶色头发的矮个子女人涨红了脸,嘴像离开水的鱼一张一合。就连奥布兰也满面通红。他坐在椅子上,身体笔直,本就厚实的胸膛膨胀起来,像被电击一般颤抖不已。在温斯顿身后,黑头发的女人大喊着``猪猡!猪猡!猪猡!''还突然捡起厚厚的新话词典向电屏砸去。词典击中高德斯坦因的鼻子,又弹了起来,他的讲话没有受到丝毫影响,仍在继续。某个瞬间,温斯顿意识到自己和周围的人并无两样,他跟着大家一起叫喊,还用脚后跟使劲去踢椅子腿。这正是两分钟仇恨会的可怕之处,没有人被强迫着参加,但是没有人能避开它不参加。不出三十秒,人们就会抛开矜持。混杂着恐惧感和报复欲的快感,对杀戮、虐待、用大铁锤痛殴他人面部的渴望如电流一般通过人群,促使你背离自己的本意,变成一个面容扭曲、高声叫喊的疯子。但是,人体察到的这种愤怒只是一种抽象的、盲目的情绪,类似喷灯的火焰,可以从一个对象转移到另一个对象。因此,有那么一会儿,温斯顿不再仇恨高德斯坦因,而正相反,这仇恨之情转移到老大哥、党和思想警察上。在这一刻,他的心倒向了电屏上这个孤独的、被嘲笑的异端分子——他在这个充斥着谎言的世界里捍卫着真理和理智。但是,没过多久他又和周围的人融在一起,认为那些攻击高德斯坦因的话语都是正确的。此时,他对老大哥的厌恶又变成了崇拜,老大哥的形象愈发高大,他成了一个所向无敌、无所畏惧的保护者,宛若一块高高矗立的、与亚洲人对抗的巨石。而高德斯坦因虽然孤立无援,虽然连是否存在都尚存疑问,但他看上去就像个阴险的巫师,只需运用话语的力量就能让文明的结构毁坏殆尽。

\switchcolumn*

It was even possible, at moments, to switch one\textquotesingle s hatred
this way or that by a voluntary act. Suddenly, by the sort of violent
effort with which one wrenches one\textquotesingle s head away from the
pillow in a nightmare, Winston succeeded in transferring his hatred from
the face on the screen to the dark-haired girl behind him. Vivid,
beautiful hallucinations flashed through his mind. He would flog her to
death with a rubber truncheon. He would tie her naked to a stake and
shoot her full of arrows like Saint Sebastian. He would ravish her and
cut her throat at the moment of climax. Better than before, moreover, he
realized \emph{why} it was that he hated her. He hated her because she
was young and pretty and sexless, because he wanted to go to bed with
her and would never do so, because round her sweet supple waist, which
seemed to ask you to encircle it with your arm, there was only the
odious scarlet sash, aggressive symbol of chastity.

\switchcolumn

有时,人能将仇恨的目标换来换去。温斯顿在转瞬之间就把对电屏中人的仇恨之情转移到身后的黑发女子身上,其速度之快,其情之猛,就好像噩梦惊醒后猛地把头从枕头的一侧扭到另一侧。他的心中出现了``生动美丽''的幻觉:他会用橡皮棍将她打死,他要把她脱光了绑在木桩上,像处死圣•塞巴斯蒂恩那样,让她乱箭穿心。他还要强奸她,并在高潮时割断她的喉咙。他比从前更清楚,为什么自己那么恨她。他恨她,因为她年轻漂亮却不性感,他想和她上床却永远不会如意。她柔软的腰肢似乎在诱惑你伸出手臂搂住她,但上面围着的却是令人讨厌的红色腰带,一个咄咄逼人的贞节的象征。

\switchcolumn*

The Hate rose to its climax. The voice of Goldstein had become an actual
sheep\textquotesingle s bleat, and for an instant the face changed into
that of a sheep. Then the sheep-face melted into the figure of a
Eurasian soldier who seemed to be advancing, huge and terrible, his
submachine gun roaring, and seeming to spring out of the surface of the
screen, so that some of the people in the front row actually flinched
backwards in their seats. But in the same moment, drawing a deep sigh of
relief from everybody, the hostile figure melted into the face of Big
Brother, black-haired, black mustachio\textquotesingle d, full of power
and mysterious calm, and so vast that it almost filled up the screen.
Nobody heard what Big Brother was saying. It was merely a few words of
encouragement, the sort of words that are uttered in the din of battle,
not distinguishable individually but restoring confidence by the fact of
being spoken. Then the face of Big Brother faded away again, and instead
the three slogans of the Party stood out in bold capitals:

\switchcolumn

仇恨达到顶点,高德斯坦因的声音真的变成了羊叫,而他的脸也一度变成羊脸。然后这羊脸又和欧亚国士兵融为一体。那士兵身材高大,样貌慑人,手中的冲锋枪正在咆哮,他似乎在冲锋,整个人好像要从电屏里跳出来。这吓坏了坐在第一排的人,一些人甚至开始向后靠去。但就在这一刻,电屏上,欧亚国的士兵变成了老大哥,他巨大的脸占满了整个屏幕——黑头发、黑胡子,面孔充满力量,安详沉稳。所有人都安下心来,没有人听见老大哥究竟说了什么。他仅有的鼓舞人心的话语被淹没在吵闹的战斗声里,难以听清,但他讲话本身就可以让人们恢复信心。在这之后,老大哥的脸隐去,电屏上出现用黑色粗体的大写字母写的三句标语:

\switchcolumn*

\headline{WAR IS PEACE\\
FREEDOM IS SLAVERY\\
IGNORANCE IS STRENGTH}

\switchcolumn

\headline{战争即和平\\
  自由即奴役\\
无知即力量}

\switchcolumn*

But the face of Big Brother seemed to persist for several seconds on the
screen, as though the impact that it had made on
everyone\textquotesingle s eyeballs were too vivid to wear off
immediately. The little sandy-haired woman had flung herself forward
over the back of the chair in front of her. With a tremulous murmur that
sounded like ``My Savior!'' she extended her arms toward the screen. Then
she buried her face in her hands. It was apparent that she was uttering
a prayer.

\switchcolumn

然而,老大哥的脸又在电屏上停留了几秒钟,也许它给人们的眼睛留下了太深的印象,以至于不能立即消失。浅茶色头发的女人扑在身前的椅子背上,颤抖着轻声呼喊:``我的拯救者。''她向屏幕张开双手,接着又用双手捧起自己的脸。显然,她在做祷告。

\switchcolumn*

At this moment the entire group of people broke into a deep, slow,
rhythmical chant of ``B-B!... B-B!... B-B!'' over and over again, very
slowly, with a long pause between the first ``B'' and the second---a
heavy, murmurous sound, somehow curiously savage, in the background of
which one seemed to hear the stamp of naked feet and the throbbing of
tom-toms. For perhaps as much as thirty seconds they kept it up. It was
a refrain that was often heard in moments of overwhelming emotion.
Partly it was a sort of hymn to the wisdom and majesty of Big Brother,
but still more it was an act of self-hypnosis, a deliberate drowning of
consciousness by means of rhythmic noise. Winston\textquotesingle s
entrails seemed to grow cold. In the Two Minutes Hate he could not help
sharing in the general delirium, but this subhuman chanting of ``B-B!...
B-B!'' always filled him with horror. Of course he chanted with the rest:
it was impossible to do otherwise. To dissemble your feelings, to
control your face, to do what everyone else was doing, was an
instinctive reaction. But there was a space of a couple of seconds
during which the expression in his eyes might conceivably have betrayed
him. And it was exactly at this moment that the significant thing
happened---if, indeed, it did happen.

\switchcolumn

这时,所有在场者都慢慢地、有节奏地、一遍又一遍地用低沉的声音喊
着:``B---B!\ldots\ldots B---B!''他们喊得很慢,第一个B和第二个B之间有长长
的间隔。这低沉又模糊的声音很奇怪地带着野蛮的气息。在这样的背景下,人好像听到
了赤脚踩地的声音和咚咚的鼓声。喊叫大约持续了30秒。在那些情绪盖过理智的时刻,
人们常可以听到这样的叫声。在某种程度上,它是对老大哥英明和权威的赞颂,但更多
的,它是一种自我催眠,人们故意用有节奏的声音来麻痹自己的意识。温斯顿由内而外
地感到寒冷。两分钟仇恨会上,他无法控制住自己,他和大家一起陷入狂热。这非人的
叫喊声``B---B!\ldots\ldots B---B!''让他惊恐。当然,他也和大家一起呼喊,他不
可能不这么做。这是人的本能:隐藏自己的真实情感,控制自己的表情,别人怎么做你
就怎么做。但有那么几秒钟,他的眼神可能出卖了他,就在这一刻,颇具意义的事情发
生了——如果它的确发生过。

\switchcolumn*

Momentarily he caught O\textquotesingle Brien\textquotesingle s eye.
O\textquotesingle Brien had stood up. He had taken off his spectacles
and was in the act of resettling them on his nose with his
characteristic gesture. But there was a fraction of a second when their
eyes met, and for as long as it took to happen Winston knew---yes, he
\emph{knew}!---that O\textquotesingle Brien was thinking the same thing
as himself. An unmistakable message had passed. It was as though their
two minds had opened and the thoughts were flowing from one into the
other through their eyes. ``I am with you,'' O\textquotesingle Brien
seemed to be saying to him. ``I know precisely what you are feeling. I
know all about your contempt, your hatred, your disgust. But
don\textquotesingle t worry, I am on your side!'' And then the flash of
intelligence was gone, and O\textquotesingle Brien\textquotesingle s
face was as inscrutable as everybody else\textquotesingle s.

\switchcolumn

那一瞬间,他捕捉到奥布兰的目光。当时奥布兰站着,正准备用他特有的姿势将之前摘下的眼镜戴上。就在他们四目相交的不到一秒的时间里,温斯顿明白了——是的,他明白!——奥布兰和自己正思考着相同的事。一个不容置疑的信息传递过来,就好像两个人的大脑都敞开,彼此的思想借助目光传递到对方那里。``我和你一样,''奥布兰似乎在对着他说。``我完全清楚你的感受,我知道你蔑视什么、仇恨什么、厌恶什么,但别担心,我站在你这边。''之后这心意相通的时刻过去,奥布兰的表情又变得和其他人一样深不可测。

\switchcolumn*

That was all, and he was already uncertain whether it had happened. Such
incidents never had any sequel. All that they did was to keep alive in
him the belief, or hope, that others besides himself were the enemies of
the Party. Perhaps the rumors of vast underground conspiracies were true
after all---perhaps the Brotherhood really existed! It was impossible,
in spite of the endless arrests and confessions and executions, to be
sure that the Brotherhood was not simply a myth. Some days he believed
in it, some days not. There was no evidence, only fleeting glimpses that
might mean anything or nothing: snatches of overheard conversation,
faint scribbles on lavatory walls---once, even, when two strangers met,
a small movement of the hands which had looked as though it might be a
signal of recognition. It was all guesswork: very likely he had imagined
everything. He had gone back to his cubicle without looking at
O\textquotesingle Brien again. The idea of following up their momentary
contact hardly crossed his mind. It would have been inconceivably
dangerous even if he had known how to set about doing it. For a second,
two seconds, they had exchanged an equivocal glance, and that was the
end of the story. But even that was a memorable event, in the locked
loneliness in which one had to live.

\switchcolumn

过程就是这样,温斯顿开始怀疑事情是否真的发生过。这样的事情不会有后续,所有这些只不过让他保持信念、怀抱希望——除了他自己,还有人是党的敌人。也许关于庞大的地下网络的传言是真的,也许兄弟会真的存在。尽管诸如逮捕、招供、处决类的事一直在发生,但人们仍然没法确定兄弟会存在,有时温斯顿觉得它有,有时又觉得没有。没有任何证据,除了那些捕风捉影的东西,它们可能谕示着什么,也可能什么都不是:无意中听到的谈话,厕所墙上胡乱涂写的模糊语句,甚至陌生人相遇时所做的也许是``暗号''的小动作。统统这些全是猜测,很有可能都出自他的臆想。他又回到他的隔间去,没有再看到奥布兰。他几乎没想过继续和奥布兰那样接触。这太危险了,就算他知道怎样做,他也不能去做。他们只不过在一两秒的时间里混混沌沌地交换了目光,这就是事情的全部。而即便如此,对生活在自我隔绝中的孤独的人来说,它依然值得铭记。

\switchcolumn*

Winston roused himself and sat up straighter. He let out a belch. The
gin was rising from his stomach.

\switchcolumn

温斯顿坐直了身子,打了个嗝,杜松子酒的味道从胃里泛了上来。

\switchcolumn*

His eyes refocused on the page. He discovered that while he sat
helplessly musing he had also been writing, as though by automatic
action. And it was no longer the same cramped awkward handwriting as
before. His pen had slid voluptuously over the smooth paper, printing in
large neat capitals---

\switchcolumn

他注视着本子,发现自己在胡思乱想的时候一直在写东西,就像自动动作。并且字迹也不同于以往的潦草。他的钢笔在光滑的纸上用大写字母整整齐齐地写着:

\switchcolumn*

\headline{DOWN WITH BIG BROTHER\\
DOWN WITH BIG BROTHER\\
DOWN WITH BIG BROTHER\\
DOWN WITH BIG BROTHER\\
DOWN WITH BIG BROTHER
}

\switchcolumn

\headline{
打倒老大哥\\
打倒老大哥\\
打倒老大哥\\
打倒老大哥\\
打倒老大哥}

\switchcolumn*

over and over again, filling half a page.

\switchcolumn

一遍又一遍,写满了半张纸。

\switchcolumn*

He could not help feeling a twinge of panic. It was absurd, since the
writing of those particular words was not more dangerous than the
initial act of opening the diary; but for a moment he was tempted to
tear out the spoiled pages and abandon the enterprise altogether.

\switchcolumn

他不由得一阵恐慌。其实毫无必要,因为这些字并不比写日记一事更加危险,有那么一
会儿,他想把写了字的纸撕掉,就此放弃记日记。

\switchcolumn*

He did not do so, however, because he knew that it was useless. Whether
he wrote DOWN WITH BIG BROTHER, or whether he refrained from writing it,
made no difference. Whether he went on with the diary, or whether he did
not go on with it, made no difference. The Thought Police would get him
just the same. He had committed---would still have committed, even if he
had never set pen to paper---the essential crime that contained all
others in itself. Thoughtcrime, they called it. Thoughtcrime was not a
thing that could be concealed forever. You might dodge successfully for
a while, even for years, but sooner or later they were bound to get you.

\switchcolumn

但他没有,他知道这毫无用处。不管他是不是写``打倒老大哥'',不管他是不是继续写
日记,都没有什么不同。思想警察仍会抓住他。他已经触犯了——就算他没用笔写在纸
上,他仍犯了——包含其他一切罪行的根本性的大罪,他们称它``思想罪''。思想罪无
处遁形,你可能成功地躲上一阵儿,甚至几年,但他们早晚会抓到你。

\switchcolumn*

It was always at night---the arrests invariably happened at night. The
sudden jerk out of sleep, the rough hand shaking your shoulder, the
lights glaring in your eyes, the ring of hard faces round the bed. In
the vast majority of cases there was no trial, no report of the arrest.
People simply disappeared, always during the night. Your name was
removed from the registers, every record of everything you had ever done
was wiped out, your one-time existence was denied and then forgotten.
You were abolished, annihilated: \emph{vaporized} was the usual word.

\switchcolumn

总是在夜里——抓捕总发生在夜里。他们突然将睡梦中的你惊醒,一面粗暴地抓着你的肩膀,一面用灯光直射你的眼睛,你的床边围绕着一堆凶恶的面孔。绝大多数情况下,不会有审讯,也不会有报道。总是在夜里人们消失了。你的名字将从登记簿上移除,你的所有记录都将被清除,你的存在遭到否定,接着你就被遗忘。你被除掉了,被消灭了,人们通常将之称为``被蒸发''。

\switchcolumn*

For a moment he was seized by a kind of hysteria. He began writing in a
hurried untidy scrawl:

\switchcolumn

有那么一会儿,他变得歇斯底里,开始在纸上胡乱地涂写:

\switchcolumn*

\begin{quotation}
theyll shoot me i dont care theyll shoot me in the back of the
neck i dont care down with big brother they always shoot you in the back
of the neck i dont care down with big brother---
\end{quotation}

\switchcolumn

他们会枪毙我我不在乎,他们会在我的脖子后面开枪我不害怕打倒老大哥,他们总是在你的脖子后面开枪我不害怕打倒老大哥。

\switchcolumn*

He sat back in his chair, slightly ashamed of himself, and laid down the
pen. The next moment he started violently. There was a knocking at the
door.

\switchcolumn

他向椅背靠去,稍稍觉得惭愧,放下了笔。突然他一惊:有人在敲门 。

\switchcolumn*

Already! He sat as still as a mouse, in the futile hope that whoever it
was might go away after a single attempt. But no, the knocking was
repeated. The worst thing of all would be to delay. His heart was
thumping like a drum, but his face, from long habit, was probably
expressionless. He got up and moved heavily toward the door.

\switchcolumn

已经来了!他像老鼠似的一动不动,徒劳地祈祷不管是谁敲一会儿就离开。但是没有,敲门声又响了起来。不开门最糟糕了,他的心像鼓一样怦怦地跳着,不过,习惯成自然,他也许仍面无表情。他站起来,脚步沉重地向门走去。

\switchcolumn*


\section{II}\label{ii}

\switchcolumn

\section*{二}\label{ux4e8c}

\switchcolumn*

As he put his hand to the doorknob Winston saw that he had left the
diary open on the table. DOWN WITH BIG BROTHER was written all over it,
in letters almost big enough to be legible across the room. It was an
inconceivably stupid thing to have done. But, he realized, even in his
panic he had not wanted to smudge the creamy paper by shutting the book
while the ink was wet.

\switchcolumn

握住门把手的瞬间,温斯顿发现日记还摊在桌子上,上面尽是``打倒老大哥'',字体大到站在房间的另一端都能看得一清二楚。这简直愚蠢透顶!但他知道,这是因为哪怕身处慌乱,自己也不想让未干的墨迹弄脏洁白的纸。

\switchcolumn*

He drew in his breath and opened the door. Instantly a warm wave of
relief flowed through him. A colorless, crushed-looking woman, with
wispy hair and a lined face, was standing outside.

\switchcolumn

他吸了口气,将门打开,顿时放下心来,感到一阵温暖。门外站着的是一个面容憔悴的女人,她脸色苍白,头发稀疏,皱纹满面。

\switchcolumn*

``Oh, comrade,'' she began in a dreary, whining sort of voice, ``I thought
I heard you come in. Do you think you could come across and have a look
at our kitchen sink? It\textquotesingle s got blocked up and---''

\switchcolumn

``哦,同志,''她用沉闷的声音说,``我听到你进门的声音,你可以帮我看下厨房的水池吗?它好像堵住了。''

\switchcolumn*

It was Mrs. Parsons, the wife of a neighbor on the same floor. (``Mrs.''
was a word somewhat discountenanced by the Party---you were supposed to
call everyone ``comrade''---but with some women one used it
instinctively.) She was a woman of about thirty, but looking much older.
One had the impression that there was dust in the creases of her face.
Winston followed her down the passage. These amateur repair jobs were an
almost daily irritation. Victory Mansions were old flats, built in 1930
or thereabouts, and were falling to pieces. The plaster flaked
constantly from ceilings and walls, the pipes burst in every hard frost,
the roof leaked whenever there was snow, the heating system was usually
running at half steam when it was not closed down altogether from
motives of economy. Repairs, except what you could do for yourself, had
to be sanctioned by remote committees which were liable to hold up even
the mending of a window pane for two years.

\switchcolumn

原来是帕森斯太太,同层楼某个邻居的妻子。(党不大赞成使用``太太''这一称呼,认为对所有人都应称``同志'',但人们仍然会对某些女人使用这个词)她大概三十岁,看上去比实际年龄老得多,脸上的皱纹仿佛嵌着灰尘一般。温斯顿跟着她向走廊走去。这样的修理工作每天都有,让人心烦。胜利大厦是始建于1903年的老房子,已经摇摇欲坠。天花板和墙上的灰泥频繁剥落,天气一冷水管就开裂,一下雪屋顶就漏水,即使不是为了节约而将暖气完全关闭,也只会提供一半的热量。维修之事除非自己动手,否则必须经过高高在上的委员会的批准,可就连换玻璃这样的小事,委员会也会拖上两年才解决。

\switchcolumn*

``Of course it\textquotesingle s only because Tom isn\textquotesingle t
home,'' said Mrs. Parsons vaguely.

\switchcolumn

``汤姆刚好不在家。''帕森斯太太含含糊糊地说。

\switchcolumn*

The Parsons\textquotesingle s flat was bigger than
Winston\textquotesingle s, and dingy in a different way. Everything had
a battered, trampled-on look, as though the place had just been visited
by some large violent animal. Games impedimenta---hockey sticks, boxing
gloves, a burst football, a pair of sweaty shorts turned inside
out---lay all over the floor, and on the table there was a litter of
dirty dishes and dog-eared exercise books. On the walls were scarlet
banners of the Youth League and the Spies, and a full-sized poster of
Big Brother. There was the usual boiled-cabbage smell, common to the
whole building, but it was shot through by a sharper reek of sweat,
which---one knew this at the first sniff, though it was hard to say
how---was the sweat of some person not present at the moment. In another
room someone with a comb and a piece of toilet paper was trying to keep
tune with the military music which was still issuing from the
telescreen.

\switchcolumn

帕森斯的屋子比温斯顿的大一点,是另一种形式的阴暗。所有东西都像被殴打过一样,就好像刚才有猛兽闯了进来。地板上散落着各种体育用品,曲棍球棒、拳击手套、破足球、翻过来的汗渍斑斑的短裤。桌上放着脏碟子和折了角的练习本,墙上贴着青年团和侦察队的红旗以及巨大的老大哥的画像。房间像这幢大楼的其他地方一样弥漫着煮白菜的气味,不仅如此,还有浓烈的汗臭味。随便是谁一闻就闻得出来,但不知为什么,这气味来自一个当时并不在场的人。而在另一个房间里,有人在吹用梳子和厕纸做成的喇叭,试图和上电屏里传出的军乐声。

\switchcolumn*

``It\textquotesingle s the children,'' said Mrs. Parsons, casting a
half-apprehensive glance at the door. ``They haven\textquotesingle t been
out today. And of course---''

\switchcolumn

``是孩子们。''帕森斯太太忧虑地向那扇房门看去,``今天他们没出去,当然——''

\switchcolumn*

She had a habit of breaking off her sentences in the middle. The kitchen
sink was full nearly to the brim with filthy greenish water which smelt
worse than ever of cabbage. Winston knelt down and examined the
angle-joint of the pipe. He hated using his hands, and he hated bending
down, which was always liable to start him coughing. Mrs. Parsons looked
on helplessly.

\switchcolumn

她有个习惯,话只说一半。厨房的水池里溢满了绿色的脏水,味道比煮白菜还难闻。温斯顿跪下来检查水管的拐弯处,他不愿意动手,也不愿意弯下身子,这会诱发他的咳嗽。帕森斯太太什么忙都帮不上,站在一旁观看。

\switchcolumn*

``Of course if Tom was home he\textquotesingle d put it right in a
moment,'' she said. ``He loves anything like that. He\textquotesingle s
ever so good with his hands, Tom is.''

\switchcolumn

``当然,如果汤姆在家,一会儿就能修好。''她说,``他喜欢做这个,他的手很巧,汤姆就是。''

\switchcolumn*

Parsons was Winston\textquotesingle s fellow employee at the Ministry of
Truth. He was a fattish but active man of paralyzing stupidity, a mass
of imbecile enthusiasms---one of those completely unquestioning, devoted
drudges on whom, more even than on the Thought Police, the stability of
the Party depended. At thirty-five he had just been unwillingly evicted
from the Youth League, and before graduating into the Youth League he
had managed to stay on in the Spies for a year beyond the statutory age.
At the Ministry he was employed in some subordinate post for which
intelligence was not required, but on the other hand he was a leading
figure on the Sports Committee and all the other committees engaged in
organizing community hikes, spontaneous demonstrations, savings
campaigns, and voluntary activities generally. He would inform you with
quiet pride, between whiffs of his pipe, that he had put in an
appearance at the Community Center every evening for the past four
years. An overpowering smell of sweat, a sort of unconscious testimony
to the strenuousness of his life, followed him about wherever he went,
and even remained behind him after he had gone.

\switchcolumn

帕森斯是温斯顿在真理部的同事。他胖胖的,人有点儿蠢,带着一腔愚钝的热情,在各方面都很活跃。他是那种不问为什么,有献身精神又劳劳碌碌的人,党要凭借他们维持稳定,他们的作用甚至超过了思想警察。帕森斯三十五岁,就在不久前刚刚不情愿地脱离了青年团,而在进入青年团之前,他曾不顾规定在少年侦察队多待了一年。他在真理部的职务是附属性的,对智力没有要求。但另一方面,他同时还在体育委员会和其他组织负责领导集体远足、自发游行、节约运动和义务劳动等活动。他会在抽烟斗的工夫用平静的语气颇为自豪地告诉你,在过去的四年里,他每天晚上都会去集体活动中心。而无论他走到哪里,他都会把汗味带过去,那汗味倒成了他精力充沛的证据。

\switchcolumn*

``Have you got a spanner?'' said Winston, fiddling with the nut on the
angle-joint.

\switchcolumn

``你有扳手吗?''温斯顿说,他正摸着水管接口处的螺帽。

\switchcolumn*

``A spanner,'' said Mrs. Parsons, immediately becoming invertebrate. ``I
don\textquotesingle t know, I\textquotesingle m sure. Perhaps the
children---''

\switchcolumn

``扳手,''帕森斯太太的声音有些犹疑,``不知道,说不定,也许孩子们——''

\switchcolumn*

There was a trampling of boots and another blast on the comb as the
children charged into the living room. Mrs. Parsons brought the spanner.
Winston let out the water and disgustedly removed the clot of human hair
that had blocked up the pipe. He cleaned his fingers as best he could in
the cold water from the tap and went back into the other room.

\switchcolumn

在一阵脚步和喇叭声后,孩子们冲进客厅。帕森斯太太拿来了扳手。温斯顿放掉脏水,忍着恶心将堵塞水管的一团头发取出。他用冰冷的自来水洗干净他的手,回到另一个房间。

\switchcolumn*

``Up with your hands!'' yelled a savage voice.

\switchcolumn

``举起手来!''一个粗鲁的声音喊道。

\switchcolumn*

A handsome, tough-looking boy of nine had popped up from behind the
table and was menacing him with a toy automatic pistol, while his small
sister, about two years younger, made the same gesture with a fragment
of wood. Both of them were dressed in the blue shorts, gray shirts, and
red neckerchiefs which were the uniform of the Spies. Winston raised his
hands above his head, but with an uneasy feeling, so vicious was the
boy\textquotesingle s demeanor, that it was not altogether a game.

\switchcolumn

有个九岁的男孩从桌子后面跳了出来,他很漂亮也很凶狠,正用一支玩具手枪对着他。比他小两岁的妹妹也用木棍对着温斯顿。两个孩子都穿着蓝色的短裤和灰色的衬衫,都戴着红领巾。这是侦察队的制服。温斯顿把手举过头,心神不宁,男孩的样子如此凶狠,不完全是在游戏。

\switchcolumn*

``You\textquotesingle re a traitor!'' yelled the boy.
``You\textquotesingle re a thought-criminal! You\textquotesingle re a
Eurasian spy! I\textquotesingle ll shoot you, I\textquotesingle ll
vaporize you, I\textquotesingle ll send you to the salt mines!''

\switchcolumn

``你这个叛徒!''男孩喊着,``你这个思想犯!你是欧亚国的间谍!我要枪毙你!我要消灭你!我要把你送到盐矿去!''

\switchcolumn*

Suddenly they were both leaping round him, shouting ``Traitor!'' and
``Thought-criminal!'', the little girl imitating her brother in every
movement. It was somehow slightly frightening, like the gamboling of
tiger cubs which will soon grow up into man-eaters. There was a sort of
calculating ferocity in the boy\textquotesingle s eye, a quite evident
desire to hit or kick Winston and a consciousness of being very nearly
big enough to do so. It was a good job it was not a real pistol he was
holding, Winston thought.

\switchcolumn

两个孩子突然围着他跳了起来。``叛徒!''\,``思想犯!''小女孩完全在模仿他的哥哥。这多少有些令人害怕,他们好像两只小虎崽,用不了多久就会成为吃人的野兽。男孩子眼神里写着狡猾和残忍,流露出要踢打温斯顿的意图,他清楚自己很快就会长到可以这样做的年纪了。温斯顿只能庆幸男孩手里拿的不是真枪。

\switchcolumn*

Mrs. Parsons\textquotesingle s eyes flitted nervously from Winston to
the children, and back again. In the better light of the living room he
noticed with interest that there actually \emph{was} dust in the creases
of her face.

\switchcolumn

帕森斯太太不安地看着孩子们,起居室的光线非常好,温斯顿发现她的皱纹里果真嵌着灰尘。

\switchcolumn*

``They do get so noisy,'' she said. ``They\textquotesingle re disappointed
because they couldn\textquotesingle t go to see the hanging,
that\textquotesingle s what it is. I\textquotesingle m too busy to take
them, and Tom won\textquotesingle t be back from work in time.''

\switchcolumn

``他们真能闹,''她说,``没能看绞刑,他们很失望,所以才这样闹。我太忙了,没时间带他们去,汤姆下班又晚。''

\switchcolumn*

``Why can\textquotesingle t we go and see the hanging?'' roared the boy in
his huge voice.

\switchcolumn

``为什么不能去看绞刑!''小男孩大声问。

\switchcolumn*

``Want to see the hanging! Want to see the hanging!'' chanted the little
girl, still capering round.

\switchcolumn

``要看绞刑!要看绞刑!''小女孩一边叫着,一边蹦来蹦去。

\switchcolumn*

Some Eurasian prisoners, guilty of war crimes, were to be hanged in the
Park that evening, Winston remembered. This happened about once a month,
and was a popular spectacle. Children always clamored to be taken to see
it. He took his leave of Mrs. Parsons and made for the door. But he had
not gone six steps down the passage when something hit the back of his
neck an agonizingly painful blow. It was as though a red-hot wire had
been jabbed into him. He spun round just in time to see Mrs. Parsons
dragging her son back into the doorway while the boy pocketed a
catapult.

\switchcolumn

温斯顿想起来,今天晚上公园里要对几个犯了战争罪的欧亚国罪犯执行绞刑。这种事每个月都有一次,大家都喜欢看。小孩子总是吵着嚷着让大人带他们去。他向帕森斯太太道完别就向门口走去。但他没走几步就被人用什么东西在脖子上重重打了一下,顿时他的脖子就像被烧红的铁丝刺进去那样,疼痛难忍。他转过身,看到帕森斯太太正抓着儿子往屋里拖,那男孩则把一个弹弓往口袋里塞。

\switchcolumn*

``Goldstein!'' bellowed the boy as the door closed on him. But what most
struck Winston was the look of helpless fright on the
woman\textquotesingle s grayish face.

\switchcolumn

``高德斯坦因!''在屋门关上的刹那,男孩喊道。温斯顿惊讶地发现帕森斯太太既无奈又恐惧。

\switchcolumn*

Back in the flat he stepped quickly past the telescreen and sat down at
the table again, still rubbing his neck. The music from the telescreen
had stopped. Instead, a clipped military voice was reading out, with a
sort of brutal relish, a description of the armaments of the new
Floating Fortress which had just been anchored between Iceland and the
Faroe Islands.

\switchcolumn

回到自己的公寓后,他快步走向电屏,摸了摸脖子,在桌子旁坐下。电屏已经停止播放音乐,一名军方人士正一字一句地念着关于冰岛和法罗群岛间设置的新式浮动堡垒的事,这个堡垒不久前刚刚建成。

\switchcolumn*

With those children, he thought, that wretched woman must lead a life of
terror. Another year, two years, and they would be watching her night
and day for symptoms of unorthodoxy. Nearly all children nowadays were
horrible. What was worst of all was that by means of such organizations
as the Spies they were systematically turned into ungovernable little
savages, and yet this produced in them no tendency whatever to rebel
against the discipline of the Party. On the contrary, they adored the
Party and everything connected with it. The songs, the processions, the
banners, the hiking, the drilling with dummy rifles, the yelling of
slogans, the worship of Big Brother---it was all a sort of glorious game
to them. All their ferocity was turned outwards, against the enemies of
the State, against foreigners, traitors, saboteurs, thought-criminals.
It was almost normal for people over thirty to be frightened of their
own children. And with good reason, for hardly a week passed in which
the \emph{Times} did not carry a paragraph describing how some
eavesdropping little sneak---``child hero'' was the phrase generally
used---had overheard some compromising remark and denounced his parents
to the Thought Police.

\switchcolumn

温斯顿心想,那女人一定为她的孩子担惊受怕。再过两年,他们就会没日没夜地监视她。几乎所有孩子都是可怕的。最糟糕的是侦察队已经将他们培养成肆意妄为的家伙,但同时他们又不会有任何违抗党的控制意向。恰恰相反,他们崇尚和党有关的一切。他们唱歌、列队前进、打起旗帜、远足、用木制步枪进行操练、高喊口号、崇拜老大哥——这在他们看来光荣而有趣。他们凶残的本性被激发出来,用在国家的敌人、外国人、叛徒、思想犯身上。超过三十岁的人普遍害怕自己的孩子。差不多每个星期《泰晤士报》都会看到关于偷听父母谈话的小暗探的报道——通常会称之为``小英雄''——偷听父母的危害性言论,然后向思想警察报告。

\switchcolumn*

The sting of the catapult bullet had worn off. He picked up his pen
half-heartedly, wondering whether he could find something more to write
in the diary. Suddenly he began thinking of O\textquotesingle Brien
again.

\switchcolumn

弹弓造成的疼痛消退了。他漫不经心地拿起笔,思考是不是还要在日记上写些什么。突然,他又想起奥布兰。

\switchcolumn*

Years ago---how long was it? Seven years it must be---he had dreamed
that he was walking through a pitch-dark room. And someone sitting to
one side of him had said as he passed: ``We shall meet in the place where
there is no darkness.'' It was said very quietly, almost casually---a
statement, not a command. He had walked on without pausing. What was
curious was that at the time, in the dream, the words had not made much
impression on him. It was only later and by degrees that they had seemed
to take on significance. He could not now remember whether it was before
or after having the dream that he had seen O\textquotesingle Brien for
the first time; nor could he remember when he had first identified the
voice as O\textquotesingle Brien\textquotesingle s. But at any rate the
identification existed. It was O\textquotesingle Brien who had spoken to
him out of the dark.

\switchcolumn

究竟有多久了?大约七年前,他曾作过一个梦,梦到自己穿过漆黑的房间。当他走过时,有个人在他身侧说:``我们将在没有黑暗的地方相见。''声音很平静,不是命令。他继续向前走。奇怪的是,当时,梦中的这句话并没有给他留下多深的印象。直到后来这句话才渐渐有了意义。他记不清第一次见到奥布兰是在做梦之前还是在做梦之后。他也记不清自己什么时候意识到说这句话的是奥布兰。但不管怎样,他确信,在黑暗中和他说话的就是奥布兰。

\switchcolumn*

Winston had never been able to feel sure---even after this
morning\textquotesingle s flash of the eyes it was still impossible to
be sure---whether O\textquotesingle Brien was a friend or an enemy. Nor
did it even seem to matter greatly. There was a link of understanding
between them more important than affection or partisanship. ``We shall
meet in the place where there is no darkness,'' he had said. Winston did
not know what it meant, only that in some way or another it would come
true.

\switchcolumn

温斯顿一直无法确定奥布兰是敌是友,即便这天上午他注意到他闪烁的眼神。但这似乎并不重要。他们心意相通,这比友谊或同志感情更加重要。他说``我们将在没有黑暗的地方相见''。温斯顿不知道这话的含义,只单纯觉得它一定会通过某种方式实现。

\switchcolumn*

The voice from the telescreen paused. A trumpet call, clear and
beautiful, floated into the stagnant air. The voice continued raspingly:

\switchcolumn

电屏里的讲话声暂停下来,一声清亮的号响打破了沉寂。接着,刺耳的讲话声又出现了:

\switchcolumn*

``Attention! Your attention, please! A newsflash has this moment arrived
from the Malabar front. Our forces in South India have won a glorious
victory. I am authorized to say that the action we are now reporting may
well bring the war within measurable distance of its end. Here is the
newsflash---''

\switchcolumn

``注意!请大家注意!现在播放从马拉巴阡县发来的急电。我军在南印度取得了辉煌的胜利,战争即将结束。急电如下——''

\switchcolumn*

Bad news coming, thought Winston. And sure enough, following on a gory
description of the annihilation of a Eurasian army, with stupendous
figures of killed and prisoners, came the announcement that, as from
next week, the chocolate ration would be reduced from thirty grams to
twenty.

\switchcolumn

坏消息来了,温斯顿想。果然,在描绘完欧亚国部队被惨烈歼灭的情形以及列举完一堆关于杀敌、俘虏的数字后,电屏里宣布从下星期开始巧克力的供应量由每天30克削减到每天20克。

\switchcolumn*

Winston belched again. The gin was wearing off, leaving a deflated
feeling. The telescreen---perhaps to celebrate the victory, perhaps to
drown the memory of the lost chocolate---crashed into ``Oceania,
\textquotesingle tis for thee.'' You were supposed to stand to attention.
However, in his present position he was invisible.

\switchcolumn

温斯顿又打了一个嗝,酒劲几乎完全消退了,只留下泄气的感觉。也许为了庆祝胜利,
也许为了让人们忘掉削减巧克力供应量的消息,电屏里传来铿锵有力的曲子《为了大洋
国》。照规矩,在这个时候他应该立正。但电屏看不到他现在待的位置。

\switchcolumn*

``Oceania, \textquotesingle tis for thee'' gave way to lighter music.
Winston walked over to the window, keeping his back to the telescreen.
The day was still cold and clear. Somewhere far away a rocket bomb
exploded with a dull, reverberating roar. About twenty or thirty of them
a week were falling on London at present.

\switchcolumn

在这首曲子后,电屏播起了轻音乐。温斯顿走到窗口,仍用后背对着电屏。天气依然那
么寒冷、晴朗。远处,一枚火箭弹爆炸了,发出震耳欲聋的爆炸声。在伦敦,这种火箭
弹一星期要落下二三十枚。

\switchcolumn*

Down in the street the wind flapped the torn poster to and fro, and the
word INGSOC fitfully appeared and vanished. Ingsoc. The sacred
principles of Ingsoc. Newspeak, doublethink, the mutability of the past.
He felt as though he were wandering in the forests of the sea bottom,
lost in a monstrous world where he himself was the monster. He was
alone. The past was dead, the future was unimaginable. What certainty
had he that a single human creature now living was on his side? And what
way of knowing that the dominion of the Party would not endure \emph{for
ever}? Like an answer, the three slogans on the white face of the
Ministry of Truth came back at him:

\switchcolumn

楼下的街道上,那张破损的宣传画被风吹得哗哗作响,``英社''二字时隐时现。英社,英社的神圣原则。新话,双重思想,变幻莫测的过去。他觉得自己正徜徉在海底森林里,他在这畸形的世界里迷失了,化身成怪兽。他独自一人。过去已经死了,未来不能想象。他如何确定究竟谁和他站在一起?他如何知道党的统治不会永远继续下去?真理部那白色墙壁上的标语再次引起他的注意,就像这些问题的答案:

\switchcolumn*

\headline{WAR IS PEACE\\
FREEDOM IS SLAVERY\\
IGNORANCE IS STRENGTH
}

\switchcolumn

\headline{战争即和平\\
  自由即奴役\\
无知即力量}

\switchcolumn*

He took a twenty-five-cent piece out of his pocket. There, too, in tiny
clear lettering, the same slogans were inscribed, and on the other face
of the coin the head of Big Brother. Even from the coin the eyes pursued
you. On coins, on stamps, on the covers of books, on banners, on
posters, and on the wrapping of a cigarette packet---everywhere. Always
the eyes watching you and the voice enveloping you. Asleep or awake,
working or eating, indoors or out of doors, in the bath or in bed---no
escape. Nothing was your own except the few cubic centimeters inside
your skull.

\switchcolumn

他从口袋里掏出一枚二角五分的硬币,硬币上也刻着这三句标语,字小而清晰。硬币的另一面是老大哥的头像。即使在硬币上,他的眼睛也紧盯着你。硬币上、邮票上、书籍的封面上、旗子上、烟盒上——他无处不在,而他的眼睛总是盯着你,他的声音总是环绕着你。不管你是睡着还是醒着,在工作还是在吃饭,在室内还是在室外,在浴室里还是在床上——你无处躲避。除了你脑袋里的几立方厘米,没有什么东西属于你。

\switchcolumn*

The sun had shifted round, and the myriad windows of the Ministry of
Truth, with the light no longer shining on them, looked grim as the
loopholes of a fortress. His heart quailed before the enormous pyramidal
shape. It was too strong, it could not be stormed. A thousand rocket
bombs would not batter it down. He wondered again for whom he was
writing the diary. For the future, for the past---for an age that might
be imaginary. And in front of him there lay not death but annihilation.
The diary would be reduced to ashes and himself to vapor. Only the
Thought Police would read what he had written, before they wiped it out
of existence and out of memory. How could you make appeal to the future
when not a trace of you, not even an anonymous word scribbled on a piece
of paper, could physically survive?

\switchcolumn

太阳西斜,真理部的窗户因为没有阳光照射就像堡垒上的枪眼一样冰冷。在这金字塔状的庞然大物前,他感到恐惧。它太强大了,不可能被攻克,一千枚火箭弹也无法将它摧毁。他再次想起那个问题,他为谁写这日记。为未来,为过去——为了想象中的时代。可等待着他的不是死而是消灭。日记会被烧毁,他也将``消失''。只有思想警察能读到他写的东西,然后他们又会将关于它的记忆清除。当你存在的痕迹,哪怕你随意写在纸上的没有姓名的字句都被清除得一干二净,你要如何向未来呼喊呢?

\switchcolumn*

The telescreen struck fourteen. He must leave in ten minutes. He had to
be back at work by fourteen-thirty.

\switchcolumn

电屏里的钟响了十四下。他必须在十分钟内离开,他要在14点30分上班。

\switchcolumn*

Curiously, the chiming of the hour seeming to have put new heart into
him. He was a lonely ghost uttering a truth that nobody would ever hear.
But so long as he uttered it, in some obscure way the continuity was not
broken. It was not by making yourself heard but by staying sane that you
carried on the human heritage. He went back to the table, dipped his
pen, and wrote:

\switchcolumn

奇怪的是,钟声让他振奋。他是孤独的幽灵,他说了谁也听不到的真理。只要他说出来,保持清醒和理智,你就承袭了人类的传统。他返回桌边,蘸了蘸墨水,写道:

\switchcolumn*

\begin{quotation}
To the future or to the past, to a time when thought is free, when
men are different from one another and do not live alone---to a time
when truth exists and what is done cannot be undone:\par
From the age of uniformity, from the age of solitude, from the age
of Big Brother, from the age of doublethink---greetings!
\end{quotation}

\switchcolumn

\begin{quotation}
为未来或过去,为思想自由、张扬个性且不孤独的时代——为真实的,不会抹杀清除过往之事的时代致敬!\par
从千篇一律的时代,从孤独的时代,从老大哥的时代,从双重思想的时代——致敬!
\end{quotation}

\switchcolumn*

He was already dead, he reflected. It seemed to him that it was only
now, when he had begun to be able to formulate his thoughts, that he had
taken the decisive step. The consequences of every act are included in
the act itself. He wrote:

\switchcolumn

他想,他已经死了。对他而言,只有理清了自己的思绪,才算迈出决定性的一步。行动的后果就蕴涵在行动本身中。他写道:

\switchcolumn*

\begin{quotation}
Thoughtcrime does not entail death: thoughtcrime IS death.
\end{quotation}

\switchcolumn

\begin{quotation}
思想罪不会让人死,思想罪本身就是死。
\end{quotation}

\switchcolumn*

Now that he had recognized himself as a dead man it became important to
stay alive as long as possible. Two fingers of his right hand were
inkstained. It was exactly the kind of detail that might betray you.
Some nosing zealot in the Ministry (a woman, probably; someone like the
little sandy-haired woman or the dark-haired girl from the Fiction
Department) might start wondering why he had been writing during the
lunch interval, why he had used an old-fashioned pen, \emph{what} he had
been writing---and then drop a hint in the appropriate quarter. He went
to the bathroom and carefully scrubbed the ink away with the gritty
dark-brown soap which rasped your skin like sandpaper and was therefore
well adapted for this purpose.

\switchcolumn

现在,既然他意识到自己已经死去,那么尽可能活得长便至关重要。他右手的两个指头沾上了墨水。没错,这样的小细节也会暴露你。部里随便一个爱管闲事的人(可能是个女人,比如浅茶色头发的女人和黑色头发的女孩)也许会去打探为什么他会在中午吃饭时写东西,为什么他用的是老式钢笔,而他又在写些什么——然后向相关部门暗示一下。他到浴室里用一块粗糙的深褐色肥皂洗去墨迹。这肥皂擦在皮肤上就像用砂纸磨东西,很适合拿来清洗墨迹。

\switchcolumn*

He put the diary away in the drawer. It was quite useless to think of
hiding it, but he could at least make sure whether or not its existence
had been discovered. A hair laid across the page-ends was too obvious.
With the tip of his finger he picked up an identifiable grain of whitish
dust and deposited it on the corner of the cover, where it was bound to
be shaken off if the book was moved.

\switchcolumn

他把日记放进抽屉,把它藏起来是不可能的。不过他至少可以确定它是否被人发现。往里面夹头发太明显了,所以他用指尖蘸了一颗不容易被发现的白色灰尘,放在日记本的封面上。若有人动了本子,它就会掉下来。

\switchcolumn*


\section{III}\label{iii}

\switchcolumn

\section*{三}\label{ux4e09}

\switchcolumn*

Winston was dreaming of his mother.

\switchcolumn

温斯顿梦见了他的母亲。

\switchcolumn*

He must, he thought, have been ten or eleven years old when his mother
had disappeared. She was a tall, statuesque, rather silent woman with
slow movements and magnificent fair hair. His father he remembered more
vaguely as dark and thin, dressed always in neat dark clothes (Winston
remembered especially the very thin soles of his
father\textquotesingle s shoes) and wearing spectacles. The two of them
must evidently have been swallowed up in one of the first great purges
of the Fifties.

\switchcolumn

母亲失踪时,他不是十岁就是十一岁。她身材高大,轮廓优美,沉默寡言,动作缓慢,还有一头浓密美丽的金发。至于他的父亲,他就印象模糊了。只依稀记得他又黑又瘦,总是穿着整洁的深色衣服(温斯顿尤其记得父亲的鞋跟很薄),戴着一副眼镜。他们是在五十年代的第一次大清洗中``消失''的。

\switchcolumn*

At this moment his mother was sitting in some place deep down beneath
him, with his young sister in her arms. He did not remember his sister
at all, except as a tiny, feeble baby, always silent, with large,
watchful eyes. Both of them were looking up at him. They were down in
some subterranean place---the bottom of a well, for instance, or a very
deep grave---but it was a place which, already far below him, was itself
moving downwards. They were in the saloon of a sinking ship, looking up
at him through the darkening water. There was still air in the saloon,
they could still see him and he them, but all the while they were
sinking down, down into the green waters which in another moment must
hide them from sight forever. He was out in the light and air while they
were being sucked down to death, and they were down there \emph{because}
he was up here. He knew it and they knew it, and he could see the
knowledge in their faces. There was no reproach either in their faces or
in their hearts, only the knowledge that they must die in order that he
might remain alive, and that this was part of the unavoidable order of
things.

\switchcolumn

在梦中,他的母亲在距离他很远的一个很深的地方坐着,怀里抱着他的妹妹。他几乎记不起妹妹了,只记得她瘦小羸弱,非常安静,有一双机警的大眼睛。她们待在一个类似井底、墓穴的地方,一边抬着脑袋看着他,一边慢慢下沉。她们在一艘沉船的大厅里,透过漆黑的海水仰望他。大厅里有空气,他们都能看到彼此,她们在绿色的海水中下沉,很快就被淹没了。他在的地方有光,有空气,她们却被死亡卷走。她们之所以会在下面,是因为他在上面。他们都清楚这点。从她们的表情上,他看不到她们对他的责备,为了让他活下去,她们必须死,无可避免。

\switchcolumn*

He could not remember what had happened, but he knew in his dream that
in some way the lives of his mother and his sister had been sacrificed
to his own. It was one of those dreams which, while retaining the
characteristic dream scenery, are a continuation of
one\textquotesingle s intellectual life, and in which one becomes aware
of facts and ideas which still seem new and valuable after one is awake.
The thing that now suddenly struck Winston was that his
mother\textquotesingle s death, nearly thirty years ago, had been tragic
and sorrowful in a way that was no longer possible. Tragedy, he
perceived, belonged to the ancient time, to a time when there were still
privacy, love, and friendship, and when the members of a family stood by
one another without needing to know the reason. His
mother\textquotesingle s memory tore at his heart because she had died
loving him, when he was too young and selfish to love her in return, and
because somehow, he did not remember how, she had sacrificed herself to
a conception of loyalty that was private and unalterable. Such things,
he saw, could not happen today. Today there were fear, hatred, and pain,
but no dignity of emotion, no deep or complex sorrows. All this he
seemed to see in the large eyes of his mother and his sister, looking up
at him through the green water, hundreds of fathoms down and still
sinking.

\switchcolumn

他不记得发生了什么。但在梦里,他明白,从某种角度讲,母亲和妹妹是为了他牺牲的。
有时在梦里,人仍然能够进行思考。人在梦里意识到的事情,醒后再看,仍然意义匪浅。
母亲去世快三十年了,温斯顿突然发现她的死是那么悲惨,这样的悲剧现在已经不存在
了。他想悲剧只发生在遥远的过去,在那个时代,仍然存在着个人私事、存在着爱和友
谊。在那个时代,一家人要相互支撑而无须问为什么。关于母亲的回忆深深刺痛了他的
心,她为爱他而死,可当时年幼自私的他却不清楚要如何来回报这种爱。不知道为什么,
他记不清具体的情况,母亲出于对忠诚的信念牺牲了自己,而那忠诚只属于个人,不可
改变。这样的事情在今天已不可能发生,今天世上充斥着恐惧、仇恨、痛苦,却没有情
感的尊严,没有深深的、复杂的哀痛。他从母亲和妹妹那大睁着的眼里看到这一切,她
们在绿色的水里仰望他,她们在几百英尺下,继续下沉。

\switchcolumn*

Suddenly he was standing on short springy turf, on a summer evening when
the slanting rays of the sun gilded the ground. The landscape that he
was looking at recurred so often in his dreams that he was never fully
certain whether or not he had seen it in the real world. In his waking
thoughts he called it the Golden Country. It was an old, rabbit-bitten
pasture, with a foot track wandering across it and a molehill here and
there. In the ragged hedge on the opposite side of the field the boughs
of the elm trees were swaying very faintly in the breeze, their leaves
just stirring in dense masses like women\textquotesingle s hair.
Somewhere near at hand, though out of sight, there was a clear,
slow-moving stream where dace were swimming in the pools under the
willow trees.

\switchcolumn

突然,一个夏日的傍晚,他站在了松软的草地上,夕阳的斜晖将土地染成了金黄色。这景象经常出现在他的梦里,以至于他不能确定此情此景是否真的存在于现实中。从睡梦中醒来,他把这叫做黄金乡。那里有一大片被野兔啃过的老草场,草场中间有一条被人踩出来的小径,上面到处可见鼹鼠的洞。草场对面是参参差差的树丛,榆树的枝条在微风中轻轻摇晃,它抖动着的茂密的树叶就像女人的长发。而在不远处还有一条清澈的小溪在缓缓流淌,溪中有鱼在游弋。

\switchcolumn*

The girl with dark hair was coming toward him across the field. With
what seemed a single movement she tore off her clothes and flung them
disdainfully aside. Her body was white and smooth, but it aroused no
desire in him; indeed, he barely looked at it. What overwhelmed him in
that instant was admiration for the gesture with which she had thrown
her clothes aside. With its grace and carelessness it seemed to
annihilate a whole culture, a whole system of thought, as though Big
Brother and the Party and the Thought Police could all be swept into
nothingness by a single splendid movement of the arm. That too was a
gesture belonging to the ancient time. Winston woke up with the word
``Shakespeare'' on his lips.

\switchcolumn

那个黑头发的女孩正穿过草场向他走来,猝不及防地脱掉了衣服,高傲地将它们丢到一边。她的身体光洁白皙,却挑不起他的欲望,事实上,他几乎没怎么看她。她扔掉衣服的姿态令他敬佩。她的动作中混杂着优雅和满不在乎的意味,就好像摧毁了整个文化、思想的体系,就好像不经意地挥一挥手,就能将老大哥、党、思想警察都扫荡干净。这动作同样属于遥远的过去,温斯顿喃喃地念着``莎士比亚''的名字,醒了过来。

\switchcolumn*

The telescreen was giving forth an ear-splitting whistle which continued
on the same note for thirty seconds. It was nought seven fifteen,
getting-up time for office workers. Winston wrenched his body out of
bed---naked, for a member of the Outer Party received only three
thousand clothing coupons annually, and a suit of pajamas was six
hundred---and seized a dingy singlet and a pair of shorts that were
lying across a chair. The Physical Jerks would begin in three minutes.
The next moment he was doubled up by a violent coughing fit which nearly
always attacked him soon after waking up. It emptied his lungs so
completely that he could only begin breathing again by lying on his back
and taking a series of deep gasps. His veins had swelled with the effort
of the cough, and the varicose ulcer had started itching.

\switchcolumn

电屏里传来了刺耳的哨音,一直持续了三十秒。此时是早上7点15分,这是办公室工作人
员起床的时间。温斯顿艰难地从床上爬起来——他浑身赤裸,作为外党党员他每年只有
三千张布票,一套睡衣就要花去六百张——他拎起挂在椅子上的背心和短裤,背心已经
褪了颜色。再过三分钟就是体操时间。他忽然剧烈地咳嗽起来,咳得直不起身,每天起
床,他都会咳上一阵,把肺咳清。之后,他仰着身子躺到床上,深深地喘几口气,恢复
了呼吸。由于咳得过于用力,那静脉曲张形成的溃疡又痒了起来。

\switchcolumn*

``Thirty to forty group!'' yapped a piercing female voice. ``Thirty to
forty group! Take your places, please. Thirties to forties!''

\switchcolumn

``三十岁到四十岁的一组!''一个尖利的女声响起。``三十岁到四十岁的一组!请大家站好,三十岁到四十岁的!''

\switchcolumn*

Winston sprang to attention in front of the telescreen, upon which the
image of a youngish woman, scrawny but muscular, dressed in tunic and
gym shoes, had already appeared.

\switchcolumn

温斯顿立即翻身下床,在电屏前站好。电屏上有一个年轻的女人,骨瘦如柴却肌肉发达,身着束腰外衣和运动鞋。

\switchcolumn*

``Arms bending and stretching!'' she rapped out. ``Take your time by me.
\emph{One}, two, three, four! \emph{One}, two, three, four! Come on,
comrades, put a bit of life into it! \emph{One}, two, three, four!
\emph{One}, two, three, four!...''

\switchcolumn

``屈伸手臂!''她喊道,``请跟我一起做。一、二、三、四!一、二、三、四!同志们,精神些!一、二、三、四!一、二、三、四!\ldots\ldots''

\switchcolumn*

The pain of the coughing fit had not quite driven out of
Winston\textquotesingle s mind the impression made by his dream, and the
rhythmic movements of the exercise restored it somewhat. As he
mechanically shot his arms back and forth, wearing on his face the look
of grim enjoyment which was considered proper during the Physical Jerks,
he was struggling to think his way backward into the dim period of his
early childhood. It was extraordinarily difficult. Beyond the late
Fifties everything faded. When there were no external records that you
could refer to, even the outline of your own life lost its sharpness.
You remembered huge events which had quite probably not happened, you
remembered the detail of incidents without being able to recapture their
atmosphere, and there were long blank periods to which you could assign
nothing. Everything had been different then. Even the names of
countries, and their shapes on the map, had been different. Airstrip
One, for instance, had not been so called in those days: it had been
called England or Britain, though London, he felt fairly certain, had
always been called London.

\switchcolumn

咳嗽发作造成的痛感还没有驱走梦留下的印象,而这富有节奏感的运动又帮助温斯顿强化了梦的记忆。他一面机械地挥动着手臂,做出与做操相应的愉悦表情,一面回忆童年的情景。这非常困难,关于五十年代晚期的记忆已然褪去,找不到可供参考的资料,就连生活都变得模糊混沌。记忆中的重大事件也许根本就没发生过,就算你记住事情的细节,你也没法重塑那氛围。况且,还有相当长的一段时期是空白的,你根本想不起发生了什么。那时的一切都和现在不同。国家的名字、地图上的形状,都和现在大不一样。比如那时一号空降场叫英格兰、不列颠。不过伦敦倒一直都叫伦敦,对此温斯顿很有把握。

\switchcolumn*

Winston could not definitely remember a time when his country had not
been at war, but it was evident that there had been a fairly long
interval of peace during his childhood, because one of his early
memories was of an air raid which appeared to take everyone by surprise.
Perhaps it was the time when the atomic bomb had fallen on Colchester.
He did not remember the raid itself, but he did remember his
father\textquotesingle s hand clutching his own as they hurried down,
down, down into some place deep in the earth, round and round a spiral
suitcase which rang under his feet and which finally wearied his legs
that he began whimpering and they had to stop and rest. His mother, in
her slow dreamy way, was following a long way behind them. She was
carrying his baby sister---or perhaps it was only a bundle of blankets
that she was carrying: he was not certain whether his sister had been
born then. Finally they had emerged into a noisy, crowded place which he
had realized to be a Tube station.

\switchcolumn

什么时候打仗,温斯顿记不清了。但在他童年时,倒是有很长一段时间都很和平。因为他记得,某次空袭让大家措手不及,或许就是科尔彻斯特被原子弹袭击的那次。关于空袭本身,他没有记忆,但他能忆起父亲如何紧抓着他的手,带他前往一个位于地下的、很深的地方。他们沿着螺旋状的楼梯不停地走,一直走到他两腿发软开始哭闹。他的母亲失魂落魄地,慢慢地跟在后面,怀里抱着他的妹妹:他不能确定抱在她怀里的究竟是他的妹妹还是几条毯子,不能确定那个时候他的妹妹是否已经降生。他们最后来到一个喧闹拥挤的地方,一个地铁站。

\switchcolumn*

There were people sitting all over the stone-flagged floor, and other
people, packed tightly together, were sitting on metal bunks, one above
the other. Winston and his mother and father found themselves a place on
the floor, and near them an old man and an old woman were sitting side
by side on a bunk. The old man had on a decent dark suit and a black
cloth cap pushed back from very white hair; his face was scarlet and his
eyes were blue and full of tears. He reeked of gin. It seemed to breathe
out of his skin in place of sweat, and one could have fancied that the
tears welling from his eyes were pure gin. But though slightly drunk he
was also suffering under some grief that was genuine and unbearable. In
his childish way Winston grasped that some terrible thing, something
that was beyond forgiveness and could never be remedied, had just
happened. It also seemed to him that he knew what it was. Someone whom
the old man loved, a little granddaughter perhaps, had been killed.
Every few minutes the old man kept repeating:

\switchcolumn

地铁站的石板地上坐满了人,铁制铺位上也全都是人。温斯顿和家人找到一块空地,在他们身旁一个老头儿和老太太并肩坐着。老头儿穿着黑色套装,十分得体,后脑上还戴着黑布帽子,他头发花白,满脸通红,蓝色的眼睛里溢满泪水。他身上散发出浓烈的酒气,就好像从他皮肤中流出的不是汗而是酒,这忍不住让人猜想他眼睛里流出来的也是酒。他虽然醉了,却非常悲伤。温斯顿用他那幼小的心灵体会他的痛楚,一定发生了可怕的事,那一定是件不能被原谅又无法挽回的事。他觉得他知道这件事。老头儿深爱的——也许是他的小孙女,被杀死了。每隔几分钟老头儿就会说:

\switchcolumn*

``We didn\textquotesingle t ought to \textquotesingle ave trusted
\textquotesingle em. I said so, Ma, didn\textquotesingle t I?
That\textquotesingle s what come of trusting \textquotesingle em. I said
so all along. We didn\textquotesingle t ought to \textquotesingle ave
trusted the buggers.''

\switchcolumn

``我们不应该相信他们。我说过的,孩子他妈,是不是?这就是相信的结果。我一直这
么说,我们不应该相信那些同性恋。''

\switchcolumn*

But which buggers they didn\textquotesingle t ought to have trusted
Winston could not now remember.

\switchcolumn

他们究竟不该相信谁?温斯顿忘记了。

\switchcolumn*

Since about that time, war had been literally continuous, though
strictly speaking it had not always been the same war. For several
months during his childhood there had been confused street fighting in
London itself, some of which he remembered vividly. But to trace out the
history of the whole period, to say who was fighting whom at any given
moment, would have been utterly impossible, since no written record, and
no spoken word, ever made mention of any other alignment than the
existing one. At this moment, for example, in 1984 (if it was 1984),
Oceania was at war with Eurasia and in alliance with Eastasia. In no
public or private utterance was it ever admitted that the three powers
had at any time been grouped along different lines. Actually, as Winston
well knew, it was only four years since Oceania had been at war with
Eastasia and in alliance with Eurasia. But that was merely a piece of
furtive knowledge which he happened to possess because his memory was
not satisfactorily under control. Officially the change of partners had
never happened. Oceania was at war with Eurasia: therefore Oceania had
always been at war with Eurasia. The enemy of the moment always
represented absolute evil, and it followed that any past or future
agreement with him was impossible.

\switchcolumn

自那之后战争不断。不过,严格地说,不是同一场战争。在他小时候,有几个月伦敦发生了巷战。有些巷战他印象深刻,可要叙述整个过程,说出某一时间交战的双方是谁,那就做不到了。因为没有任何相关资料、没有任何有关此事的只言片语,除了目前的国家联盟外也没有提到其他什么联盟。而现在,举个例子,1984年(如果真是1984年),大洋国和东亚国结盟,一起对抗欧亚国。但不管是在公开场合还是在私下的谈话里,都未承认这三大国曾有过不同的结盟。而温斯顿很清楚,仅仅4年前,因为和东亚国交火,大洋国就曾和欧亚国结盟。只是这仅仅是由于记忆力失控才侥幸记住的片断。按照官方的说法,盟友关系从未发生过变化。大洋国和欧亚国交战,大洋国一直在和欧亚国交战。眼下的敌人是绝对邪恶的象征,这意味着无论过去还是未来,都不可能和它站在同一边。

\switchcolumn*

The frightening thing, he reflected for the ten thousandth time as he
forced his shoulders painfully backward (with hands on hips, they were
gyrating their bodies from the waist, an exercise that was supposed to
be good for the back muscles)---the frightening thing was that it might
all be true. If the Party could thrust its hand into the past and say of
this or that event, \emph{it never happened}---that, surely, was more
terrifying than mere torture and death.

\switchcolumn

他的肩膀尽可能地向后仰去(他用手托住屁股,做上半身的转体,据说这对背部肌肉有益)——打仗的事也许是真的,如果党能够控制过去,说某件事从未发生过——那不是比拷打和死刑更可怕吗?

\switchcolumn*

The Party said that Oceania had never been in alliance with Eurasia. He,
Winston Smith, knew that Oceania had been in alliance with Eurasia as
short a time as four years ago. But where did that knowledge exist? Only
in his own consciousness, which in any case must soon be annihilated.
And if all others accepted the lie which the Party imposed---if all
records told the same tale---then the lie passed into history and became
truth. ``Who controls the past,'' ran the Party slogan, ``controls the
future: who controls the present controls the past.'' And yet the past,
though of its nature alterable, never had been altered. Whatever was
true now was true from everlasting to everlasting. It was quite simple.
All that was needed was an unending series of victories over your own
memory. ``Reality control,'' they called it; in Newspeak, ``doublethink.''

\switchcolumn

竟说大洋国从来没和欧亚国结过盟。他,温斯顿却记得就在四年之前,大洋国还曾和欧亚国结盟。但这个常识立足何处呢?这常识只存在于他的思想里,而他的思想很快就会被消灭。如果其他人都接受了党强加的谎言——如果所有记录都这样说——那么这个谎言就会被载入史册成为真理。党有一句口号:``谁能控制过去谁就能控制未来;谁能控制现在谁就能控制过去。''而过去,尽管它的性质可以被篡改,但它却从未被改变过。什么东西现在是正确的,那它永远都是正确的。这非常简单,人只需要不停地、反复地战胜记忆。他们把这叫做``现实控制'',用新话来说就是``双重思想''。

\switchcolumn*

``Stand easy!'' barked the instructress, a little more genially.

\switchcolumn

``稍息!''女教练喊,语气温和了一些。

\switchcolumn*

Winston sank his arms to his sides and slowly refilled his lungs with
air. His mind slid away into the labyrinthine world of doublethink. To
know and not to know, to be conscious of complete truthfulness while
telling carefully constructed lies, to hold simultaneously two opinions
which canceled out, knowing them to be contradictory and believing in
both of them, to use logic against logic, to repudiate morality while
laying claim to it, to believe that democracy was impossible and that
the Party was the guardian of democracy, to forget whatever it was
necessary to forget, then to draw it back into memory again at the
moment when it was needed, and then promptly to forget it again, and
above all, to apply the same process to the process itself---that was
the ultimate subtlety: consciously to induce unconsciousness, and then,
once again, to become unconscious of the act of hypnosis you had just
performed. Even to understand the word ``doublethink'' involved the use of
doublethink.

\switchcolumn

温斯顿放下手臂,缓缓地吸了口气。他陷落在双重思想的迷宫里。知还是不知,知晓全部情况却故意编造谎言,同时持有两种相互矛盾的观点。一方面明知二者的矛盾之处,一方面又对二者都确信无疑。用逻辑来反对逻辑,在否定道德的同时拥护道德,在声称不可能有民主的同时又声称党是民主的捍卫者。忘掉那些应当忘掉的事,再在需要的时候回忆起它们,然后再忘掉,最重要的是,要用同样的方法处理这过程本身。这简直太妙了:有意识地进入无意识,然后继续,让人意识不到自己对自己进行了催眠。要了解``双重思想'',就要使用双重思想。

\switchcolumn*

The instructress had called them to attention again. ``And now
let\textquotesingle s see which of us can touch our toes!'' she said
enthusiastically. ``Right over from the hips, please, comrades.
\emph{One}-two! \emph{One}-two!...''

\switchcolumn

女教练命令大家立正。``现在看看谁能摸到自己的脚趾!''她热情洋溢,``弯腰,同志们,开始,一——二!一——二!\ldots\ldots''

\switchcolumn*

Winston loathed this exercise, which sent shooting pains all the way
from his heels to his buttocks and often ended by bringing on another
coughing fit. The half-pleasant quality went out of his meditations. The
past, he reflected, had not merely been altered, it had been actually
destroyed. For how could you establish even the most obvious fact when
there existed no record outside your own memory? He tried to remember in
what year he had first heard mention of Big Brother. He thought it must
have been at some time in the Sixties, but it was impossible to be
certain. In the Party histories, of course, Big Brother figured as the
leader and guardian of the Revolution since its very earliest days. His
exploits had been gradually pushed backwards in time until already they
extended into the fabulous world of the Forties and the Thirties, when
the capitalists in their strange cylindrical hats still rode through the
streets of London in great gleaming motor cars or horse carriages with
glass sides. There was no knowing how much of this legend was true and
how much invented. Winston could not even remember at what date the
Party itself had come into existence. He did not believe he had ever
heard the word Ingsoc before 1960, but it was possible that in its
Oldspeak form---``English Socialism,'' that is to say---it had been
current earlier. Everything melted into mist. Sometimes, indeed, you
could put your finger on a definite lie. It was not true, for example,
as was claimed in the Party history books, that the Party had invented
airplanes. He remembered airplanes since his earliest childhood. But you
could prove nothing. There was never any evidence. Just once in his
whole life he had held in his hands unmistakable documentary proof of
the falsification of a historical fact. And on that occasion---

\switchcolumn

温斯顿最讨厌这节操,一阵剧烈的痛感从他的脚踝传到屁股,让他咳嗽起来,他从沉思中获得的快感消失殆尽。被修改的过去实际上被毁掉了。假使过去只存在于你的记忆中,除此之外别无证据,你拿什么证明哪怕是显而易见的事实呢?他试图想起他第一次听到老大哥这名字是在哪一年。大概是在六十年代,但无法确定。在党史里,老大哥在革命一开始就是领导者。他的功绩可以一直追溯到四十年代和三十年代。那时,资本家还戴着样子古怪的圆形礼帽,乘着锃亮的豪华汽车,或坐着带玻璃窗的马车穿梭在伦敦的街道上。没人知道这说法有几分真,几分假。温斯顿甚至记不起党建立的具体时间,在六十年代之前他似乎没有听说过``英社''一词,但它的旧话形态,即``英国社会主义'',可能在六十年代之前就出现了。每件事都面目模糊。有时,你很清楚哪些是谎言,比如党史中说飞机是党发明的,这不可能。因为他很小就知道飞机了。但你无法证明这点,没有任何证据。他人生之中只有一次掌握了确凿证据,证明某个历史事实是伪造的,那是——

\switchcolumn*

``Smith!'' screamed the shrewish voice from the telescreen. ``6079 Smith W!
Yes, \emph{you}! Bend lower, please! You can do better than that.
You\textquotesingle re not trying. Lower, please!
\emph{That\textquotesingle s} better, comrade. Now stand at ease, the
whole squad, and watch me.''

\switchcolumn

``史密斯!''电屏里传来吼叫,``6079号温斯顿•史密斯!没错,就是你!再弯低些!你能做得更好。你没尽力,再低一些!好多了,同志!现在,全队稍息,看着我。''

\switchcolumn*

A sudden hot sweat had broken out all over Winston\textquotesingle s
body. His face remained completely inscrutable. Never show dismay! Never
show resentment! A single flicker of the eyes could give you away. He
stood watching while the instructress raised her arms above her head
and---one could not say gracefully, but with remarkable neatness and
efficiency---bent over and tucked the first joint of her fingers under
her toes.

\switchcolumn

温斯顿突然大汗淋漓。他的表情让人捉摸不定,永远不要流露出不高兴的神色!千万不能表现出不满!转瞬间的一个眼神就会暴露自己。他站在那里看女教练举起双臂,她的姿势说不上优美,却很干净利落。她弯下腰,将手指的第一关节垫在了脚下。

\switchcolumn*

``\emph{There}, comrades! \emph{That\textquotesingle s} how I want to see
you doing it. Watch me again. I\textquotesingle m thirty-nine and
I\textquotesingle ve had four children. Now look.'' She bent over again.
``You see \emph{my} knees aren\textquotesingle t bent. You can all do it
if you want to,'' she added as she straightened herself up. ``Anyone under
forty-five is perfectly capable of touching his toes. We
don\textquotesingle t all have the privilege of fighting in the front
line, but at least we can all keep fit. Remember our boys on the Malabar
front! And the sailors in the Floating Fortresses! Just think what
\emph{they} have to put up with. Now try again. That\textquotesingle s
better, comrade, that\textquotesingle s \emph{much} better,'' she added
encouragingly as Winston, with a violent lunge, succeeded in touching
his toes with knees unbent, for the first time in several years.

\switchcolumn

``就这样,同志们,我希望看到你们所有人都这样做。再看我做一遍。我三十九岁了,有四个孩子。可是瞧!''她再次弯下腰。``你们看,我的膝盖没有弯曲,如果你想你也可以做到。''她直起身子。``四十五岁以下的人都能碰到脚趾。不是所有人都能上前线打仗,可至少要保持身体健康。想想那些在马拉巴前线的战士!想想水上堡垒的水兵!想想他们都经历了什么。再来一次。好多了,同志,好多了。''她看着温斯顿,鼓励他。而他正用力向前弯下身体,膝盖笔直,双手碰触脚尖。数年来,他第一次做到这个程度。

\switchcolumn*


\section{IV}\label{iv}

\switchcolumn

\section*{四}\label{ux56db}

\switchcolumn*

With the deep, unconscious sigh which not even the nearness of the
telescreen could prevent him from uttering when his
day\textquotesingle s work started, Winston pulled the speakwrite toward
him, blew the dust from its mouthpiece, and put on his spectacles. Then
he unrolled and clipped together four small cylinders of paper which had
already flopped out of the pneumatic tube on the right-hand side of his
desk.

\switchcolumn

温斯顿叹了口气,就算坐在电屏旁他也不能控制自己不在每天工作伊始叹气。他拉出语音记录器,吹掉话筒上的尘土,戴上眼镜,然后把从办公桌右侧气力输送管送出的四小卷纸铺开,夹在一起。

\switchcolumn*

In the walls of the cubicle there were three orifices. To the right of
the speakwrite, a small pneumatic tube for written messages; to the
left, a larger one for newspapers; and in the side wall, within easy
reach of Winston\textquotesingle s arm, a large oblong slit protected by
a wire grating. This last was for the disposal of waste paper. Similar
slits existed in thousands or tens of thousands throughout the building,
not only in every room but at short intervals in every corridor. For
some reason they were nicknamed memory holes. When one knew that any
document was due for destruction, or even when one saw a scrap of waste
paper lying about, it was an automatic action to lift the flap of the
nearest memory hole and drop it in, whereupon it would be whirled away
on a current of warm air to the enormous furnaces which were hidden
somewhere in the recesses of the building.

\switchcolumn

他隔间的墙上有三个小洞。在语音记录器右边的是传达书面指示的气力输送管;左边略大些的用来送报纸,而位于侧墙上温斯顿触手可及的用来处理废报纸的椭圆形洞口则被铁丝网罩住。像这样的洞口在大楼里有成千上万,到处都是,不只房间里有,就连走廊上也每隔一段距离就设置一个。出于某种原因,人们叫它记忆洞。当你要销毁某样文件时,当你看到废纸时,你能随手掀起离你最近的洞盖子将它们扔进去,它们会被一股温暖的气流卷走,卷到大楼隐蔽处的大型锅炉里。

\switchcolumn*

Winston examined the four slips of paper which he had unrolled. Each
contained a message of only one or two lines, in the abbreviated
jargon---not actually Newspeak, but consisting largely of Newspeak
words---which was used in the Ministry for internal purposes. They ran:

\switchcolumn

温斯顿察看着那四张展开来的卷纸,每张纸上都有一两句指示,非新话,但包含大量新话词汇,这是内部行话的缩写。它们是:

\switchcolumn*

\begin{quotation}
times 17.3.84 bb speech malreported africa rectify

\switchcolumn

泰晤士报 17.3.84 bb 讲话误报非洲矫正

\switchcolumn*

times 19.12.83 forecasts 3 yp 4th quarter 83 misprints verify current
issue

\switchcolumn

泰晤士报 19.12.83 预报三年计划四季度83处误印核实现刊

\switchcolumn*

times 14.2.84 miniplenty malquoted chocolate rectify

\switchcolumn

泰晤士报 14.2.84 bb 富部误引巧克力数据修改

\switchcolumn*

times 3.12.83 reporting bb dayorder doubleplusungood refs unpersons
rewrite fullwise upsub antefiling.
\end{quotation}

\switchcolumn

泰晤士报 3.12.83 bb 关于双倍增加不好的指示提到非人全部重写存档前上交

\switchcolumn*

With a faint feeling of satisfaction Winston laid the fourth message
aside. It was an intricate and responsible job and had better be dealt
with last. The other three were routine matters, though the second one
would probably mean some tedious wading through lists of figures.

\switchcolumn

温斯顿感到一丝得意,他将第四项指示放到一旁,这件事非常复杂,且意义重大,最好留到最后再做。另外三则是例行公事,虽然第二件需要查阅大量数据,非常乏味。

\switchcolumn*

Winston dialed ``back numbers'' on the telescreen and called for the
appropriate issues of the \emph{Times}, which slid out of the pneumatic
tube after only a few minutes\textquotesingle{} delay. The messages he
had received referred to articles or news items which for one reason or
another it was thought necessary to alter, or, as the official phrase
had it, to rectify. For example, it appeared from the \emph{Times} of
the seventeenth of March that Big Brother, in his speech of the previous
day, had predicted that the South Indian front would remain quiet but
that a Eurasian offensive would shortly be launched in North Africa. As
it happened, the Eurasian Higher Command had launched its offensive in
South India and left North Africa alone. It was therefore necessary to
rewrite a paragraph of Big Brother\textquotesingle s speech in such a
way as to make him predict the thing that had actually happened. Or
again, the \emph{Times} of the nineteenth of December had published the
official forecasts of the output of various classes of consumption goods
in the fourth quarter of 1983, which was also the sixth quarter of the
Ninth Three-Year Plan. Today\textquotesingle s issue contained a
statement of the actual output, from which it appeared that the
forecasts were in every instance grossly wrong.
Winston\textquotesingle s job was to rectify the original figures by
making them agree with the later ones. As for the third message, it
referred to a very simple error which could be set right in a couple of
minutes. As short a time ago as February, the Ministry of Plenty had
issued a promise (a ``categorical pledge'' were the official words) that
there would be no reduction of the chocolate ration during 1984.
Actually, as Winston was aware, the chocolate ration was to be reduced
from thirty grams to twenty at the end of the present week. All that was
needed was to substitute for the original promise a warning that it
would probably be necessary to reduce the ration at some time in April.

\switchcolumn

温斯顿拨了电屏上的``过刊''号码,要来了相关日期的《泰晤士报》,没几分钟,报纸
就被输送过来。他收到的指示都和文章或新闻有关,出于某种原因,按照官方说法,它
们必须被核正。举个例子,3月17日《泰晤士报》中提到,老大哥在前一天的讲话中预言
南印度前线不会发生什么事情,欧亚国不久便会对北非发动进攻。事实上,欧亚国最高
司令部没有理会北非,反而攻打了南印度。这就需要人们对老大哥的预言进行修改,使
之符合现实。再比如,12月19日的《泰晤士报》公布了1983年第四季度,即第九个三年
计划的第六季官方对各类消费品产量的预测。而今天一期的《泰晤士报》刊登了实际产
量。两相对比,之前的每项数字都错得离谱。对温斯顿来说,他的工作就是要改正之前
的数字让它们和后来的相应。至于第三项指示,针对的是几分钟就能改好的小错误。就
在二月份,富部曾许诺(用官方的话说就是``绝对保证'')1984年内巧克力的供应量不
会减少。实际上,温斯顿知道这星期一过完巧克力的供应量就要从三十克削减到二十克。
因此,他要将官方的诺言换成一句提醒式的话语``四月份可能会减少供应量''。

\switchcolumn*

As soon as Winston had dealt with each of the messages, he clipped his
speakwritten corrections to the appropriate copy of the \emph{Times} and
pushed them into the pneumatic tube. Then, with a movement which was as
nearly as possible unconscious, he crumpled up the original message and
any notes that he himself had made, and dropped them into the memory
hole to be devoured by the flames.

\switchcolumn

每完成一项指示,温斯顿就要把语音记录器记下的更正字条附在相应的《泰晤士报》上,
一起送入气力输送管。然后他要尽可能自然地将发给他的指示和写好的草稿揉成一团,
扔进记忆洞,让它们被烈火焚毁。

\switchcolumn*

What happened in the unseen labyrinth to which the pneumatic tubes led,
he did not know in detail, but he did know in general terms. As soon as
all the corrections which happened to be necessary in any particular
number of the \emph{Times} had been assembled and collated, that number
would be reprinted, the original copy destroyed, and the corrected copy
placed on the files in its stead. This process of continuous alteration
was applied not only to newspapers, but to books, periodicals,
pamphlets, posters, leaflets, films, sound tracks, cartoons,
photographs---to every kind of literature or documentation which might
conceivably hold any political or ideological significance. Day by day
and almost minute by minute the past was brought up to date. In this way
every prediction made by the Party could be shown by documentary
evidence to have been correct; nor was any item of news, or any
expression of opinion, which conflicted with the needs of the moment,
ever allowed to remain on record. All history was a palimpsest, scraped
clean and reinscribed exactly as often as was necessary. In no case
would it have been possible, once the deed was done, to prove that any
falsification had taken place. The largest section of the Records
Department, far larger than the one in which Winston worked, consisted
simply of persons whose duty it was to track down and collect all copies
of books, newspapers, and other documents which had been superseded and
were due for destruction. A number of the \emph{Times} which might,
because of changes in political alignment, or mistaken prophecies
uttered by Big Brother, have been rewritten a dozen times still stood on
the files bearing its original date, and no other copy existed to
contradict it. Books, also, were recalled and rewritten again and again,
and were invariably reissued without any admission that any alteration
had been made. Even the written instructions which Winston received, and
which he invariably got rid of as soon as he had dealt with them, never
stated or implied that an act of forgery was to be committed; always the
reference was to slips, errors, misprints, or misquotations which it was
necessary to put right in the interests of accuracy.

\switchcolumn

他不知道气力输送管会通向哪里,那里是看不见的迷宫。但他了解大致的情况。任何一
期《泰晤士报》在被核正修改后,都将被重新印刷并存档,原有的则会被销毁。而这种
情况不单适用于报纸,还适用于书籍、期刊、小册子、宣传画、传单、电影、录音、漫
画、照片——一切可能承载政治思想或意识形态的文字、文件。过去时时刻刻都在被修
改,被要求和现况相符。如此,党的所有预言都是正确的,任何违背当前需要的东西都
不允许留有记录,历史就像那种可以被反复重写的本子,只要需要,就可以涂涂抹抹。
而一旦这项工作完成,人们就无法证明历史被伪造过。在记录司最庞大的处里,人们的
主要工作就是回收、销毁一切不合时宜的书报文件,这个处要比温斯顿所在的大得多。
因为政治联盟的变化和老大哥的预言谬误,一些《泰晤士报》可能已被改写了十多次,
它们仍按照原来的日期进行存档,至于原有的、与之冲突的版本则不会被留下。书籍也
是一样,不止一次地被收回来重写,再发行时还拒绝承认做过修改。就连温斯顿处理后
便销毁的书面指示也从未明言或暗示要伪造什么,它总是使用笔误、错误、误印或误
引。

\switchcolumn*

But actually, he thought as he readjusted the Ministry of
Plenty\textquotesingle s figures, it was not even forgery. It was merely
the substitution of one piece of nonsense for another. Most of the
material that you were dealing with had no connection with anything in
the real world, not even the kind of connection that is contained in a
direct lie. Statistics were just as much a fantasy in their original
version as in their rectified version. A great deal of the time you were
expected to make them up out of your head. For example, the Ministry of
Plenty\textquotesingle s forecast had estimated the output of boots for
the quarter at a hundred and forty-five million pairs. The actual output
was given as sixty-two millions. Winston, however, in rewriting the
forecast, marked the figure down to fifty-seven millions, so as to allow
for the usual claim that the quota had been overfulfilled. In any case,
sixty-two millions was no nearer the truth than fifty-seven millions, or
than a hundred and forty-five millions. Very likely no boots had been
produced at all. Likelier still, nobody knew how many had been produced,
much less cared. All one knew was that every quarter astronomical
numbers of boots were produced on paper, while perhaps half the
population of Oceania went barefoot. And so it was with every class of
recorded fact, great or small. Everything faded away into a shadow-world
in which, finally, even the date of the year had become uncertain.

\switchcolumn

温斯顿一边修改富部的数字一边想,这连伪造都算不上,这不过是用一个谎言代替另一个谎言。人所处理的绝大部分材料都和现实世界毫不相关,甚至连显而易见的谎言与现实间的联系都没有。无论是修改前还是修改后,这些统计数字都荒诞不经。大部分时候,它们都是人凭空想象的。比如,富部预计该季度鞋子产量将达到一亿四千五百万双,而实际上只有六千二百万双。于是,温斯顿修改了预计的数字,将它减少为五千七百万,这样就可以说超额完成了任务。不过,六千二百万既不比一亿四千五百万真实,也不比五千七百万真实。实际上很可能一双鞋子都没生产。而更有可能的是没有人知道究竟生产了多少鞋子,也没有人关心这事。人们知道的是,每季度书面上都生产了天文数字般的鞋子,但大洋国里却有近一半人是光着脚的。所有被记录下来的事情无论大小皆是如此。每件事情都处在影子世界中,到最后,人们连今年是哪一年都说不清了。

\switchcolumn*

Winston glanced across the hall. In the corresponding cubicle on the
other side a small, precise-looking, dark-chinned man named Tillotson
was working steadily away, with a folded newspaper on his knee and his
mouth very close to the mouthpiece of the speakwrite. He had the air of
trying to keep what he was saying a secret between himself and the
telescreen. He looked up, and his spectacles darted a hostile flash in
Winston\textquotesingle s direction.

\switchcolumn

温斯顿看了看大厅,一个名叫狄洛森的小个子就坐在大厅的另一端,温斯顿的对面。他下巴微黑,看上去精明谨慎,正不慌不忙地工作着。他的膝盖上放着一摞报纸,嘴巴靠近语音记录器,好像除了电屏,并不想让别人听到他的讲话。他抬起头,眼镜反了光,似乎充满敌意。

\switchcolumn*

Winston hardly knew Tillotson, and had no idea what work he was employed
on. People in the Records Department did not readily talk about their
jobs. In the long, windowless hall, with its double row of cubicles and
its endless rustle of papers and hum of voices murmuring into
speakwrites, there were quite a dozen people whom Winston did not even
know by name, though he daily saw them hurrying to and fro in the
corridors or gesticulating in the Two Minutes Hate. He knew that in the
cubicle next to him the little woman with sandy hair toiled day in, day
out, simply at tracking down and deleting from the press the names of
people who had been vaporized and were therefore considered never to
have existed. There was a certain fitness in this, since her own husband
had been vaporized a couple of years earlier. And a few cubicles away a
mild, ineffectual, dreamy creature named Ampleforth, with very hairy
ears and a surprising talent for juggling with rhymes and meters, was
engaged in producing garbled versions---definitive texts, they were
called---of poems which had become ideologically offensive but which for
one reason or another were to be retained in the anthologies. And this
hall, with its fifty workers or thereabouts, was only one sub-section, a
single cell, as it were, in the huge complexity of the Records
Department. Beyond, above, below, were other swarms of workers engaged
in an unimaginable multitude of jobs. There were the huge printing shops
with their sub-editors, their typography experts, and their elaborately
equipped studios for the faking of photographs. There was the
teleprograms section with its engineers, its producers, and its teams of
actors specially chosen for their skill in imitating voices. There were
the armies of reference clerks whose job was simply to draw up lists of
books and periodicals which were due for recall. There were the vast
repositories where the corrected documents were stored, and the hidden
furnaces where the original copies were destroyed. And somewhere or
other, quite anonymous, there were the directing brains who coordinated
the whole effort and laid down the lines of policy which made it
necessary that this fragment of the past should be preserved, that one
falsified, and the other rubbed out of existence.

\switchcolumn

温斯顿并不了解狄洛森,不知道他负责什么工作。记录司的人都不喜欢谈论自己的工作。
在没有窗户的长长的大厅两边是一个个小的办公隔间,翻动纸张发出的声音和对语音记
录器讲话的嗡嗡声没完没了。其中一些人尽管每天都出现在走廊里,每天在两分钟仇恨
会上挥舞双手,温斯顿却连名字也叫不上来。他知道在他隔壁,浅茶色头发的女人一天
到晚忙碌不停,她的工作仅仅是在报纸上查找那些被蒸发的人们的名字,并将它们删去。
这件事很适合她,就在几年前,她的丈夫才被蒸发掉。而在不远处的另一个小隔间,是
他的同事安普福斯,他性格温和,有点窝囊,总是心不在焉,耳朵上还长着密密的毛。
可他在运用押韵、格律上却有着惊人的天赋。他的工作就是修改那些违背官方意识形态
却又不得不保留的诗歌——他们称之为``限定版''。整个大厅有五十多个工作人员,这
些人还只是处下的一个科,即记录司这庞杂机构的一个小细胞。在记录司里,从上至下,
工作多得难以想象。在规模庞大的印刷车间里,有编辑、有校排、有制造假照片的设备
精密的暗房。在电屏节目处,有工程师、有制片和各种擅长模仿他人声音的演员。此外,
还有数目众多的资料核查员,忙着开列需要被回收的书籍、期刊的清单,还有不知姓名
的领导们在制定政策,决定什么保留、什么伪造、什么销毁。司里有用来存放被修改的
文本的大型仓库和用来焚毁文本原件的隐蔽的锅炉。

\switchcolumn*

And the Records Department, after all, was itself only a single branch
of the Ministry of Truth, whose primary job was not to reconstruct the
past but to supply the citizens of Oceania with newspapers, films,
textbooks, telescreen programs, plays, novels---with every conceivable
kind of information, instruction, or entertainment, from a statue to a
slogan, from a lyric poem to a biological treatise, and from a
child\textquotesingle s spelling book to a Newspeak dictionary. And the
Ministry had not only to supply the multifarious needs of the Party, but
also to repeat the whole operation at a lower level for the benefit of
the proletariat. There was a whole chain of separate departments dealing
with proletarian literature, music, drama, and entertainment generally.
Here were produced rubbishy newspapers containing almost nothing except
sport, crime, and astrology, sensational five-cent novelettes, films
oozing with sex, and sentimental songs which were composed entirely by
mechanical means on a special kind of kaleidoscope known as a
versificator. There was even a whole sub-section---\emph{Pornosec}, it
was called in Newspeak---engaged in producing the lowest kind of
pornography, which was sent out in sealed packets and which no Party
member, other than those who worked on it, was permitted to look at.

\switchcolumn

记录司不过是真理部的一个部门,事实上真理部的主要工作并不是改写历史,而是为大洋国的公民提供报纸、电影、教材、电屏节目、戏剧、小说——从雕像到标语,从抒情诗到生物论文,从小孩的拼写书籍到新话词典——所有能够想到的信息、教育、娱乐。真理部不单要满足党的各种需求,还要制造一些低层次的东西满足无产阶级的需要。因此,它特地单辟了一些部门,生产除了体育、占星、罪案外别无他物的垃圾报纸以及五分钱一本的、内容刺激的小说和色情电影、伤感音乐——这些音乐都是用一种被称作谱曲机的特殊的搅拌机机械式地做出来的。不仅如此,它还有一个部门专门负责创作低级色情的文学——新话叫色情科。这些文学被密封发行,除了相关工作人员谁也不许看,就连党员也被禁止浏览。

\switchcolumn*

Three messages had slid out of the pneumatic tube while Winston was
working; but they were simple matters, and he had disposed of them
before the Two Minutes Hate interrupted him. When the Hate was over he
returned to his cubicle, took the Newspeak dictionary from the shelf,
pushed the speakwrite to one side, cleaned his spectacles, and settled
down to his main job of the morning.

\switchcolumn

又有三条指示从气力输送口传出来,所幸都是一些简单的事务。在两分钟仇恨会开始前,温斯顿就将它们处理完毕。仇恨会后,他回到他的小隔间,从书架上取下新话词典,把语音记录器推到一边。他擦了擦眼镜,开始处理当天上午的主要工作。

\switchcolumn*

Winston\textquotesingle s greatest pleasure in life was in his work.
Most of it was a tedious routine, but included in it there were also
jobs so difficult and intricate that you could lose yourself in them as
in the depths of a mathematical problem---delicate pieces of forgery in
which you had nothing to guide you except your knowledge of the
principles of Ingsoc and your estimate of what the Party wanted you to
say. Winston was good at this kind of thing. On occasion he had even
been entrusted with the rectification of the \emph{Times} leading
articles, which were written entirely in Newspeak. He unrolled the
message that he had set aside earlier. It ran:

\switchcolumn

温斯顿生活中的最大乐趣便是工作。他的大部分工作都是常规性的,沉闷单调,但偶尔也会有非常困难的事情,会让你像解数学题一样忘掉自己,沉浸其中——那是些微妙复杂的伪造工作,你只能凭借对英社原则的了解和对党意图的揣测来完成。温斯顿颇擅长此事,有一次,他甚至受命用新话改写《泰晤士报》的头版文章。他将之前放在一旁的那份指示打开:

\switchcolumn*

\begin{quotation}
times 3.12.83 reporting bb dayorder doubleplusungood refs unpersons
rewrite fullwise upsub antefiling.
\end{quotation}

\switchcolumn

泰晤士报 3.12.83 bb 关于双倍增加不好的指示提到非人全部重写存档前上交。

\switchcolumn*

In Oldspeak (or standard English) this might be rendered:

\switchcolumn

用老话(或标准英语)即:

\switchcolumn*

\begin{quotation}
The reporting of Big Brother\textquotesingle s Order for the Day in the
\emph{Times} of December 3rd 1983 is extremely unsatisfactory and makes
references to nonexistent persons. Rewrite it in full and submit your
draft to higher authority before filing.
\end{quotation}

\switchcolumn

1983年12月3日《泰晤士报》所载的关于老大哥所下指示的报道非常不妥,其中提到了不存在的人。全部重写,并在存档前交由上级过目。

\switchcolumn*

Winston read through the offending article. Big
Brother\textquotesingle s Order for the Day, it seemed, had been chiefly
devoted to praising the work of an organization known as FFCC, which
supplied cigarettes and other comforts to the sailors in the Floating
Fortresses. A certain Comrade Withers, a prominent member of the Inner
Party, had been singled out for special mention and awarded a
decoration, the Order of Conspicuous Merit, Second Class.

\switchcolumn

温斯顿将这篇问题报道重新阅读了一遍。原来老大哥在里面表扬了一个名为FFCC的机构,
该机构的主要职责是为水上堡垒的士兵提供香烟等物资的供给。在这篇报道中,内党要
人威瑟斯受到了特别嘉奖,得到了一枚二等卓越勋章。

\switchcolumn*

Three months later FFCC had suddenly been dissolved with no reasons
given. One could assume that Withers and his associates were now in
disgrace, but there had been no report of the matter in the press or on
the telescreen. That was to be expected, since it was unusual for
political offenders to be put on trial or even publicly denounced. The
great purges involving thousands of people, with public trials of
traitors and thought-criminals who made abject confession of their
crimes and were afterwards executed, were special showpieces not
occurring oftener than once in a couple of years. More commonly, people
who had incurred the displeasure of the Party simply disappeared and
were never heard of again. One never had the smallest clue as to what
had happened to them. In some cases they might not even be dead. Perhaps
thirty people personally known to Winston, not counting his parents, had
disappeared at one time or another.

\switchcolumn

谁料三个月后FFCC突然被解散,原因不明。尽管报纸和电屏都没有报道这件事,但可以
肯定威瑟斯和他的同僚们失宠了。通常,政治犯不会被公开审判,也不会被公开批判。
倒是在牵扯众多的大清洗中,叛徒和思想犯才会受到公判,他们可怜兮兮地坦白认罪,
然后被处以死刑。但这几年才有一次。大多时候,令党头不满的人会悄无声息地消失,
再也没有人见到他们,也不会有人知道在他们身上发生了什么,也许他们中的一些并没
有死。在温斯顿认识的人里,有三十多人就这样消失了,其中还不包括他的父母。

\switchcolumn*

Winston stroked his nose gently with a paper clip. In the cubicle across
the way Comrade Tillotson was still crouching secretively over his
speakwrite. He raised his head for a moment: again the hostile
spectacle-flash. Winston wondered whether Comrade Tillotson was engaged
on the same job as himself. It was perfectly possible. So tricky a piece
of work would never be entrusted to a single person; on the other hand,
to turn it over to a committee would be to admit openly that an act of
fabrication was taking place. Very likely as many as a dozen people were
now working away on rival versions of what Big Brother had actually
said. And presently some master brain in the Inner Party would select
this version or that, would re-edit it and set in motion the complex
processes of cross-referencing that would be required, and then the
chosen lie would pass into the permanent records and become truth.

\switchcolumn

温斯顿用纸夹轻轻擦了下鼻子,在他对面,狄洛森仍在对着语音记录器说话,他抬起头,
眼镜上又出现一道敌意的光。温斯顿不知道狄洛森的工作是不是和自己的一样,这样麻
烦的工作不可能完全交给一个人,但另一方面,若把这事交给一个委员会去做就等于公
开承认了伪造。因此,很有可能,修改老大哥讲话的工作由十几个人共同负责,他们要
将各自的修改版上交,由内党的智囊首领选出一个加以编辑、核对。待一切进行完毕,
这份被选中的谎言就会被载入档案,成为真理。

\switchcolumn*

Winston did not know why Withers had been disgraced. Perhaps it was for
corruption or incompetence. Perhaps Big Brother was merely getting rid
of a too-popular subordinate. Perhaps Withers or someone close to him
had been suspected of heretical tendencies. Or perhaps---what was
likeliest of all---the thing had simply happened because purges and
vaporizations were a necessary part of the mechanics of government. The
only real clue lay in the words ``refs unpersons,'' which indicated that
Withers was already dead. You could not invariably assume this to be the
case when people were arrested. Sometimes they were released and allowed
to remain at liberty for as much as a year or two years before being
executed. Very occasionally some person whom you had believed dead long
since would make a ghostly reappearance at some public trial where he
would implicate hundreds of others by his testimony before vanishing,
this time forever. Withers, however, was already an \emph{unperson}. He
did not exist; he had never existed. Winston decided that it would not
be enough simply to reverse the tendency of Big
Brother\textquotesingle s speech. It was better to make it deal with
something totally unconnected with its original subject.

\switchcolumn

温斯顿不清楚威瑟斯失宠的原因,也许因为腐败,也许因为失职,也许因为老大哥要铲
除过于受人爱戴的下属,也许因为他和某个异议人士走得太近。也许——最有可能的理
由是——清洗和蒸发都已是政府机制的必要组成。指示里提到``不存在的人'',这是和
威瑟斯下落有关的唯一一条线索,它说明威瑟斯已经死了。不是所有人被逮捕都可以这
样推测,有时人们会被释放,享受一两年的自由,然后才被处死。还有时,你认为已经
死了的人会像鬼魂一样突然出现在公审大会上,他做的供词还会牵连数百个人,然后他
才永远地销声匿迹。但这次不同,威瑟斯是``不存在的人'',这意味着他从未存在过。
这让温斯顿觉得单单修改老大哥所讲话语的倾向性远远不够,最好是将它改成和原有话
题毫无关联的事。

\switchcolumn*

He might turn the speech into the usual denunciation of traitors and
thought-criminals, but that was a little too obvious, while to invent a
victory at the front, or some triumph of overproduction in the Ninth
Three-Year Plan, might complicate the records too much. What was needed
was a piece of pure fantasy. Suddenly there sprang into his mind,
ready-made as it were, the image of a certain Comrade Ogilvy, who had
recently died in battle, in heroic circumstances. There were occasions
when Big Brother devoted his Order for the Day to commemorating some
humble, rank-and-file Party member whose life and death he held up as an
example worthy to be followed. Today he should commemorate Comrade
Ogilvy. It was true that there was no such person as Comrade Ogilvy, but
a few lines of print and a couple of faked photographs would soon bring
him into existence.

\switchcolumn

他可以将讲话改成常见的对叛徒和思想犯的斥责,但这太明显了。而如果他杜撰一场胜仗,抑或是第九个三年计划的超额完成,又过于复杂。最好的办法是完完全全凭空虚构。忽然,他的脑子里出现了奥吉维尔同志的形象,他刚刚在战场上牺牲,是个现成的例子。老大哥不时就会表扬低级别的普通党员,让他们的生死成为他人效仿的对象。今天老大哥应该纪念下奥吉维尔同志。实际上并不存在什么奥吉维尔,但这不要紧,只要印几行字,放上几张伪造的照片,他就``存在''了。

\switchcolumn*

Winston thought for a moment, then pulled the speakwrite toward him and
began dictating in Big Brothers familiar style: a style at once military
and pedantic, and, because of a trick of asking questions and then
promptly answering them (``What lessons do we learn from this fact,
comrades? The lessons---which is also one of the fundamental principles
of Ingsoc---that,'' etc., etc.), easy to imitate.

\switchcolumn

温斯顿思考片刻,将语音记录器拉出来,模仿起老大哥的讲话腔调,这腔调包含着军人和学者两种截然不同的风格,同时还有自己的套路——自问自答。(同志们,我们从这件事上得到了怎样的教训?教训,这是英社的基本原则,这\ldots\ldots 等等,等等)这很容易模仿。

\switchcolumn*

At the age of three Comrade Ogilvy had refused all toys except a drum, a
submachine gun, and a model helicopter. At six---a year early, by a
special relaxation of the rules---he had joined the Spies; at nine he
had been a troop leader. At eleven he had denounced his uncle to the
Thought Police after overhearing a conversation which appeared to him to
have criminal tendencies. At seventeen he had been a district organizer
of the Junior Anti-Sex League. At nineteen he had designed a hand
grenade which had been adopted by the Ministry of Peace and which, at
its first trial, had killed thirty-one Eurasian prisoners in one burst.
At twenty-three he had perished in action. Pursued by enemy jet planes
while flying over the Indian Ocean with important despatches, he had
weighted his body with his machine gun and leapt out of the helicopter
into deep water, despatches and all---an end, said Big Brother, which it
was impossible to contemplate without feelings of envy. Big Brother
added a few remarks on the purity and single-mindedness of Comrade
Ogilvy\textquotesingle s life. He was a total abstainer and a nonsmoker,
had no recreations except a daily hour in the gymnasium, and had taken a
vow of celibacy, believing marriage and the care of a family to be
incompatible with a twenty-four-hour-a-day devotion to duty. He had no
subjects of conversation except the principles of Ingsoc, and no aim in
life except the defeat of the Eurasian enemy and the hunting-down of
spies, saboteurs, thought-criminals, and traitors generally.

\switchcolumn

奥吉维尔同志在三岁时除了鼓、轻机枪、直升机模型外,什么玩具都不要。六岁时就参加了侦察队,到九岁便当上队长,十一岁时他就将偷听到的他叔叔所讲的有犯罪倾向的话报告给思想警察。十七岁时他成为少年反性同盟的区域组织者。十九岁时,他设计的手榴弹被和平部采用,第一次试验便炸死了三十一个欧亚国俘虏。他二十三岁时在战场上牺牲。当时他正带着重要文件飞行在印度洋上空,敌人的喷气式飞机发现了他,紧追不舍。他将机枪捆在身上,跳出飞机,带着文件没入大海。老大哥认为这样的结局令人羡慕。老大哥还就奥吉维尔同志这纯洁无瑕又忠诚可鉴的一生发表了感慨。他不抽烟,不喝酒,除了每天在健身房里锻炼的一个小时外,没有任何娱乐。他发誓要孤独一生,因为在他看来,婚姻和家庭会让人无法将一天二十四小时全部投入到工作中。除了英社的原则,他从不谈论其他话题,除了打击抓捕间谍、破坏分子、思想犯和叛国贼,他的生活没有其他目标。

\switchcolumn*

Winston debated with himself whether to award Comrade Ogilvy the Order
of Conspicuous Merit; in the end he decided against it because of the
unnecessary cross-referencing that it would entail.

\switchcolumn

温斯顿想了很久,要不要授予奥吉维尔同志卓越勋章,最后还是决定不授予,这样就少了很多核查工作。

\switchcolumn*

Once again he glanced at his rival in the opposite cubicle. Something
seemed to tell him with certainty that Tillotson was busy on the same
job as himself. There was no way of knowing whose version would finally
be adopted, but he felt a profound conviction that it would be his own.
Comrade Ogilvy, unimagined an hour ago, was now a fact. It struck him as
curious that you could create dead men but not living ones. Comrade
Ogilvy, who had never existed in the present, now existed in the past,
and when once the act of forgery was forgotten, he would exist just as
authentically, and upon the same evidence, as Charlemagne or Julius
Caesar.

\switchcolumn

他又看了看他对面隔间里的竞争对手,好像有什么东西告诉他,狄洛森正在做着和他一样的事情。他不知道最后会采纳谁的工作成果,但他觉得一定会是自己的。一个小时前还不存在什么奥吉维尔同志,现在他却成了事实。他有种奇妙的感觉,他造得出死人,却造不出活人。在现实中从未存在过的奥吉维尔同志,现在却存在于过去,当伪造之事被遗忘,他就会像查理大帝或恺撒大帝那样成为真实的存在,且有证据可以证明。

\switchcolumn*


\section{V}\label{v}

\switchcolumn

\section*{五}\label{ux4e94}

\switchcolumn*

In the low-ceilinged canteen, deep under ground, the lunch queue jerked
slowly forward. The room was already very full and deafeningly noisy.
From the grille at the counter the steam of stew came pouring forth,
with a sour metallic smell which did not quite overcome the fumes of
Victory Gin. On the far side of the room there was a small bar, a mere
hole in the wall, where gin could be bought at ten cents the large nip.

\switchcolumn

食堂设在深深的地下,天花板很低,领午餐的队伍慢慢地挪动着。这里到处是人,吵闹嘈杂。炖菜的腾腾蒸气从餐台的铁栏处钻出,泛着金属的酸味,它没能将杜松子酒的气味压住。在食堂的另一端有个小酒吧,小到仿若开在墙上的洞,只要一角钱就能买到一大杯杜松子酒。

\switchcolumn*

``Just the man I was looking for,'' said a voice at
Winston\textquotesingle s back.

\switchcolumn

``正在找你!''温斯顿背后传来一个人的声音。

\switchcolumn*

He turned round. It was his friend Syme, who worked in the Research
Department. Perhaps ``friend'' was not exactly the right word. You did not
have friends nowadays, you had comrades; but there were some comrades
whose society was pleasanter than that of others. Syme was a
philologist, a specialist in Newspeak. Indeed, he was one of the
enormous team of experts now engaged in compiling the Eleventh Edition
of the Newspeak dictionary. He was a tiny creature, smaller than
Winston, with dark hair and large, protuberant eyes, at once mournful
and derisive, which seemed to search your face closely while he was
speaking to you.

\switchcolumn

他转过身,原来是在研究司上班的朋友赛姆。也许在眼下这个世界称其为``朋友''并不妥当,如今人们没有朋友,只有同志。不过和一些同志交往会比和另外一些更愉快些。赛姆是语言学家、新话学家,是编纂新话词典的众多专家中的一个。他的身材很小,比温斯顿还小,他的头发是黑色的,眼睛大大地突起,神情既悲伤又有几分嘲讽。和人讲话时,他习惯盯着人的脸,那双大眼睛仿佛在人的脸上搜寻着什么。

\switchcolumn*

``I wanted to ask you whether you\textquotesingle d got any razor
blades,'' he said.

\switchcolumn

``你有刀片吗?''他说。

\switchcolumn*

``Not one!'' said Winston with a sort of guilty haste.
``I\textquotesingle ve tried all over the place. They
don\textquotesingle t exist any longer.''

\switchcolumn

``一片也没有了。''温斯顿有点心虚,``我找遍了,都用完了。''

\switchcolumn*

Everyone kept asking you for razor blades. Actually he had two unused
ones which he was hoarding up. There had been a famine of them for
months past. At any given moment there was some necessary article which
the Party shops were unable to supply. Sometimes it was buttons,
sometimes it was darning wool, sometimes it was shoelaces; at present it
was razor blades. You could only get hold of them, if at all, by
scrounging more or less furtively on the ``free'' market.

\switchcolumn

每个人都跑过来管你要刀片。实际上,温斯顿攒了两个刀片。过去几个月刀片一直短缺。在任何时候都会有一些必需品是党的商店里供不应求的。有时是扣子,有时是线,有时是鞋带,现在是刀片。人们只能偷偷摸摸地去``自由''市场购买。

\switchcolumn*

``I\textquotesingle ve been using the same blade for six weeks,'' he added
untruthfully.

\switchcolumn

``我的刀片已经用了六个星期了。''他又加了句假话。

\switchcolumn*

The queue gave another jerk forward. As they halted he turned and faced
Syme again. Each of them took a greasy metal tray from a pile at the
edge of the counter.

\switchcolumn

队伍向前挪动了一点,人们停下时,他回头看着赛姆。俩人都从堆放在餐台上的油乎乎的盘子中取出了一个。

\switchcolumn*

``Did you go and see the prisoners hanged yesterday?'' said Syme.

\switchcolumn

``昨天看绞刑了吗?''赛姆问。

\switchcolumn*

``I was working,'' said Winston indifferently. ``I shall see it on the
flicks, I suppose.''

\switchcolumn

``有工作要做,''温斯顿淡淡地说,``我可以在电影上看。''

\switchcolumn*

``A very inadequate substitute,'' said Syme.

\switchcolumn

``那可差远了。''赛姆说。

\switchcolumn*

His mocking eyes roved over Winston\textquotesingle s face. ``I know
you,'' the eyes seemed to say, ``I see through you. I know very well why
you didn\textquotesingle t go to see those prisoners hanged.'' In an
intellectual way, Syme was venomously orthodox. He would talk with a
disagreeable gloating satisfaction of helicopter raids on enemy
villages, the trials and confessions of thought-criminals, the
executions in the cellars of the Ministry of Love. Talking to him was
largely a matter of getting him away from such subjects and entangling
him, if possible, in the technicalities of Newspeak, on which he was
authoritative and interesting. Winston turned his head a little aside to
avoid the scrutiny of the large dark eyes.

\switchcolumn

他用充满嘲弄意味的目光打量着温斯顿。``我了解你。''那眼神好像在说:``我已经看穿了你,我知道你为什么不去看绞刑。''作为一个知识分子,赛姆又正统又恶毒。他会幸灾乐祸地谈论直升机如何袭击敌人的村庄,谈论思想犯如何被审讯,如何招供,如何在仁爱部的地下室里遭受处决。这让人非常不快。和他说话,总要想办法岔开话题,如果可能,最好将话题引到关于新话的技术性问题上,他是这方面的权威,且兴趣浓厚。温斯顿将头扭到一边以便躲开他黑色的大眼睛。

\switchcolumn*

``It was a good hanging,'' said Syme reminiscently. ``I think it spoils it
when they tie their feet together. I like to see them kicking. And above
all, at the end, the tongue sticking right out, and blue---a quite
bright blue. That\textquotesingle s the detail that appeals to me.''

\switchcolumn

``那绞刑很棒,''赛姆回忆,``就是把他们的脚绑起来不大好。我喜欢看他们的脚在空中乱踢。最重要的是,最后,他们的舌头会伸出来,颜色非常青。这些细节特别吸引我。''

\switchcolumn*

``Nex\textquotesingle, please!'' yelled the white-aproned prole with the
ladle.

\switchcolumn

``下一个!''一个系着白围裙,拿着勺子的人喊道。

\switchcolumn*

Winston and Syme pushed their trays beneath the grille. Onto each was
dumped swiftly the regulation lunch---a metal pannikin of pinkish-gray
stew, a hunk of bread, a cube of cheese, a mug of milkless Victory
Coffee, and one saccharine tablet.

\switchcolumn

温斯顿和赛姆将他们的盘子放到餐台的铁栏下,食堂的工作人员立即为他们盛好午饭——一盒灰粉色的炖菜,一块面包,一块干酪,一杯不加奶的胜利咖啡和一片糖精。

\switchcolumn*

``There\textquotesingle s a table over there, under that telescreen,''
said Syme. ``Let\textquotesingle s pick up a gin on the way.''

\switchcolumn

``电屏下面有张空桌,''赛姆说,``我们顺便买点儿酒。''

\switchcolumn*

The gin was served out to them in handleless china mugs. They threaded
their way across the crowded room and unpacked their trays onto the
metal-topped table, on one corner of which someone had left a pool of
stew, a filthy liquid mess that had the appearance of vomit. Winston
took up his mug of gin, paused for an instant to collect his nerve, and
gulped the oily-tasting stuff down. When he had winked the tears out of
his eyes he suddenly discovered that he was hungry. He began swallowing
spoonfuls of the stew, which, in among its general sloppiness, had cubes
of spongy pinkish stuff which was probably a preparation of meat.
Neither of them spoke again till they had emptied their pannikins. From
the table at Winston\textquotesingle s left, a little behind his back,
someone was talking rapidly and continuously, a harsh gabble almost like
the quacking of a duck, which pierced the general uproar of the room.

\switchcolumn

他们拿着装有杜松子酒的马克杯穿过拥挤的人群,来到了空桌前,把盘子放在铁制的桌面上。不知什么人在桌子的一角弄洒了菜,就像吐出来的一样让人恶心。温斯顿拿起酒杯愣了一会儿,然后鼓足勇气,将这带着油味的东西吞了下去。当眼泪流出来时,他感觉到饥饿,便一勺一勺地吃起炖菜。菜炖得一塌糊涂,里面有些软塌塌的粉红色的东西,好像是肉。在把餐盒里的炖菜吃光前,他们谁也没有说话。而在温斯顿左边,一个声音又粗又哑的人像鸭子一样说个不停,在人声喧闹的餐厅里尤其刺耳。

\switchcolumn*

``How is the dictionary getting on?'' said Winston, raising his voice to
overcome the noise.

\switchcolumn

``词典编得怎么样了?''温斯顿大声说,试图压过餐厅里的喧哗。

\switchcolumn*

``Slowly,'' said Syme. ``I\textquotesingle m on the adjectives.
It\textquotesingle s fascinating.''

\switchcolumn

``很慢,''赛姆答,``我负责形容词,很有趣。''

\switchcolumn*

He had brightened up immediately at the mention of Newspeak. He pushed
his pannikin aside, took up his hunk of bread in one delicate hand and
his cheese in the other, and leaned across the table so as to be able to
speak without shouting.

\switchcolumn

说到新话,赛姆的精神就来了。他推开餐盒,用细长的手指拿起面包和干酪,因为不想大声喊话,他的身体向前倾斜。

\switchcolumn*

``The Eleventh Edition is the definitive edition,'' he said.
``We\textquotesingle re getting the language into its final shape---the
shape it\textquotesingle s going to have when nobody speaks anything
else. When we\textquotesingle ve finished with it, people like you will
have to learn it all over again. You think, I dare say, that our chief
job is inventing new words. But not a bit of it! We\textquotesingle re
destroying words---scores of them, hundreds of them, every day.
We\textquotesingle re cutting the language down to the bone. The
Eleventh Edition won\textquotesingle t contain a single word that will
become obsolete before the year 2050.''

\switchcolumn

``第十一版是定稿。''他说,``我们要搞定语言的最终形态——也就是说,除了这种语言,人们不能再说其他形式的语言。等这工作一完,像你这样的人就要重新开始学。我敢说,你一定以为我们的工作是创造新词。不,完全不是。我们在消灭单词,几十几百地消灭,每天都是这样。我们让语言只剩下一副骨头。2050年前过时的词,十一版中一个都没有。''

\switchcolumn*

He bit hungrily into his bread and swallowed a couple of mouthfuls, then
continued speaking, with a sort of pedant\textquotesingle s passion. His
thin dark face had become animated, his eyes had lost their mocking
expression and grown almost dreamy.

\switchcolumn

他大口大口地吃着面包,然后带着学究式的热情又说了起来。他又黑又瘦的脸庞光彩焕发,那嘲弄的目光消失了,取而代之以梦呓般的神情。

\switchcolumn*

``It\textquotesingle s a beautiful thing, the destruction of words. Of
course the great wastage is in the verbs and adjectives, but there are
hundreds of nouns that can be got rid of as well. It isn\textquotesingle t
only the synonyms; there are also the antonyms. After all, what
justification is there for a word which is simply the opposite of some other
word? A word contains its opposite in itself. Take `good,' for instance. If
you have a word like `good,' what need is there for a word like `bad'?
`Ungood' will do just as well---better, because it\textquotesingle s an
exact opposite, which the other is not. Or again, if you want a stronger
version of `good,' what sense is there in having a whole string of vague
useless words like `excellent' and `splendid' and all the rest of them?
`Plusgood' covers the meaning, or `doubleplusgood' if you want something
stronger still. Of course we use those forms already, but in the final
version of Newspeak there\textquotesingle ll be nothing else. In the end the
whole notion of goodness and badness will be covered by only six words---in
reality, only one word. Don\textquotesingle t you see the beauty of that,
Winston? It was B.B.\textquotesingle s idea originally, of course,'' he
added as an afterthought.

\switchcolumn

``消灭单词真是妙不可言。动词和形容词有很多是多余的,名词也可以去掉好几百个,其中既有同义词也有反义词。总之,如果一个词表达的只是另一个词相反的意思,那它还有什么必要存在下去呢?就拿`好'来说。有了这个字,为什么还需要`坏'字?用`不好'就可以了。这比用`坏'要好,这正好表达了和`好'相反的意思。再比如,你需要一个比`好'语气要强一些的词,为什么要用诸如`精彩'、`出色'等意思含混又没有用处的词呢?`加倍好'就可以了。当然,我们已经在使用这些词了,在新话的最终版里,不会再有其他的词。要表达好和坏的意思只要六个词就够了——实际上只有一个词。温斯顿,你不觉得这很妙吗?这原本是老大哥的意思。''

\switchcolumn*

A sort of vapid eagerness flitted across Winston\textquotesingle s face
at the mention of Big Brother. Nevertheless Syme immediately detected a
certain lack of enthusiasm.

\switchcolumn

听到老大哥,温斯顿的脸上立即现出崇敬的神色,但赛姆还是发现他不够热情。

\switchcolumn*

``You haven\textquotesingle t a real appreciation of Newspeak, Winston,''
he said almost sadly. ``Even when you write it you\textquotesingle re
still thinking in Oldspeak. I\textquotesingle ve read some of those
pieces that you write in the \emph{Times} occasionally.
They\textquotesingle re good enough, but they\textquotesingle re
translations. In your heart you\textquotesingle d prefer to stick to
Oldspeak, with all its vagueness and its useless shades of meaning. You
don\textquotesingle t grasp the beauty of the destruction of words. Do
you know that Newspeak is the only language in the world whose
vocabulary gets smaller every year?''

\switchcolumn

``温斯顿,你还没体会到新话的好处。''他有点失落,``就算你用新话写东西,你还是在用老话想问题。我在《泰晤士报》上读过你的文章,很不错,但它们只是翻译,你仍然喜欢老话,尽管它们词义含混,不实用,差别小。你不知道消灭词汇的好处。新话可是世界上唯一一种词汇越来越少的语言。''

\switchcolumn*

Winston did know that, of course. He smiled, sympathetically he hoped,
not trusting himself to speak. Syme bit off another fragment of the
dark-colored bread, chewed it briefly, and went on:

\switchcolumn

温斯顿当然不能体会,但他还是露出赞同的笑脸。赛姆又咬了口面包,说:

\switchcolumn*

``Don\textquotesingle t you see that the whole aim of Newspeak is to
narrow the range of thought? In the end we shall make thoughtcrime
literally impossible, because there will be no words in which to express
it. Every concept that can ever be needed will be expressed by exactly
\emph{one} word, with its meaning rigidly defined and all its subsidiary
meanings rubbed out and forgotten. Already, in the Eleventh Edition,
we\textquotesingle re not far from that point. But the process will
still be continuing long after you and I are dead. Every year fewer and
fewer words, and the range of consciousness always a little smaller.
Even now, of course, there\textquotesingle s no reason or excuse for
committing thoughtcrime. It\textquotesingle s merely a question of
self-discipline, reality-control. But in the end there
won\textquotesingle t be any need even for that. The Revolution will be
complete when the language is perfect. Newspeak is Ingsoc and Ingsoc is
Newspeak,'' he added with a sort of mystical satisfaction. ``Has it ever
occurred to you, Winston, that by the year 2050, at the very latest, not
a single human being will be alive who could understand such a
conversation as we are having now?''

\switchcolumn

``你不明白新话的目的就是缩小思考范围吗?让每个人都不会再犯思想罪,因为找不到可用来表达的词汇。每一个必要的概念都只能用一个词来表达,这样它的意义就受到限制,它的次要意义就会被消除,被遗忘。第十一版和这个目的相距不远。但在我们死后,这件事还会继续下去,词汇的数量每年都减少,意识的范围也跟着变小。当然,即便在现在,也没有理由去犯思想罪,这是自律和实际控制的事。但最终,没有这个必要。什么时候语言完善了,什么时候革命就完成了。新话就是英社,英社就是新话。''他用一种神秘又满足的语气说,``温斯顿,你想过吗,最晚到2050年,没有哪个活人能听懂我们现在的谈话。''

\switchcolumn*

``Except---'' began Winston doubtfully, and then stopped.

\switchcolumn

``除了\ldots\ldots''温斯顿想说什么,又停住了。

\switchcolumn*

It had been on the tip of his tongue to say ``Except the proles,'' but he
checked himself, not feeling fully certain that this remark was not in
some way unorthodox. Syme, however, had divined what he was about to
say.

\switchcolumn

他想说的是群众,他按捺住自己,不能确定这句话算不算异端,不过赛姆已经猜到他的心思。

\switchcolumn*

``The proles are not human beings,'' he said carelessly. ``By
2050---earlier, probably---all real knowledge of Oldspeak will have
disappeared. The whole literature of the past will have been destroyed.
Chaucer, Shakespeare, Milton, Byron---they\textquotesingle ll exist only in
Newspeak versions, not merely changed into something different, but actually
changed into something contradictory of what they used to be. Even the
literature of the Party will change. Even the slogans will change. How could
you have a slogan like `freedom is slavery' when the concept of freedom has
been abolished? The whole climate of thought will be different. In fact
there will \emph{be} no thought, as we understand it now. Orthodoxy means
not thinking---not needing to think. Orthodoxy is unconsciousness.''

\switchcolumn

``群众不是人。''他说得很轻率,``到2050年,也许更早,所有和老话相关的知识都会消失。过去的所有文学也都要被摧毁,乔叟、莎士比亚、弥尔顿、拜伦——他们的作品将只有新话版本,不但会被改成完全不同的东西,甚至会改成和他们所阐述的意义完全相反的东西。甚至党的书籍、口号都会被修改。自由的概念都消失了,又怎么会有`自由即奴役'?到时候整个思想的氛围都会发生改变。事实上,不再有我们今天说的这种思想,关于思想,正统的含义是——不想、无意识。''

\switchcolumn*

One of these days, thought Winston with sudden deep conviction, Syme
will be vaporized. He is too intelligent. He sees too clearly and speaks
too plainly. The Party does not like such people. One day he will
disappear. It is written in his face.

\switchcolumn

温斯顿突然觉得,总有一天赛姆会被蒸发。他太聪明,看得太透彻又说得太直白。党不会喜欢他。总有一天他会消失,这情形已经写在了他的脸上。

\switchcolumn*

Winston had finished his bread and cheese. He turned a little sideways
in his chair to drink his mug of coffee. At the table on his left the
man with the strident voice was still talking remorselessly away. A
young woman who was perhaps his secretary, and who was sitting with her
back to Winston, was listening to him and seemed to be eagerly agreeing
with everything that he said. From time to time Winston caught some such
remark as ``I think you\textquotesingle re \emph{so} right, I do
\emph{so} agree with you,'' uttered in a youthful and rather silly
feminine voice. But the other voice never stopped for an instant, even
when the girl was speaking. Winston knew the man by sight, though he
knew no more about him than that he held some important post in the
Fiction Department. He was a man of about thirty, with a muscular throat
and a large, mobile mouth. His head was thrown back a little, and
because of the angle at which he was sitting, his spectacles caught the
light and presented to Winston two blank discs instead of eyes. What was
slightly horrible was that from the stream of sound that poured out of
his mouth, it was almost impossible to distinguish a single word. Just
once Winston caught a phrase---``complete and final elimination of
Goldsteinism''---jerked out very rapidly and, as it seemed, all in one
piece, like a line of type cast solid. For the rest it was just a noise,
a quack-quack-quacking. And yet, though you could not actually hear what
the man was saying, you could not be in any doubt about its general
nature. He might be denouncing Goldstein and demanding sterner measures
against thought-criminals and saboteurs, he might be fulminating against
the atrocities of the Eurasian army, he might be praising Big Brother or
the heroes on the Malabar front---it made no difference. Whatever it
was, you could be certain that every word of it was pure orthodoxy, pure
Ingsoc. As he watched the eyeless face with the jaw moving rapidly up
and down, Winston had a curious feeling that this was not a real human
being but some kind of dummy. It was not the man\textquotesingle s brain
that was speaking; it was his larynx. The stuff that was coming out of
him consisted of words, but it was not speech in the true sense: it was
a noise uttered in unconsciousness, like the quacking of a duck.

\switchcolumn

温斯顿吃完面包和干酪,侧了侧身子去喝咖啡。他左边的声音粗哑的人还在没完没了地
说着。一个背对着温斯顿的年轻女人,大概是他的秘书,就坐在那里听他讲,看上去对
他讲的东西颇为赞同。温斯顿间或听到她说:``你是对的,完全同意。''她的声音很年
轻,也很蠢。但那人即使在她说话的时候,也不会停顿下来。温斯顿知道这男人,他在
小说司里担任要职。他大概三十岁,口才了得。他的头微微后仰,由于角度关系,他的
眼镜反着光,温斯顿只能看到两个眼镜片。而他喋喋不休地讲着,你却一个词都听不清
楚。温斯顿只听清一句话:``完全地彻底地消灭高德斯坦因主义。''这话说得飞快,就
像铸成一行的铅字,所有词浑然一体。至于其他的话,听上去就是一片叽叽呱呱的噪声。
不过,你仍然可以了解大致的内容。他很可能是在叱责高德斯坦因,认为要对思想犯和
破坏分子采取更严厉的惩治办法。他也可能是在谴责欧亚国军队,或者歌颂老大哥、马
拉巴阡县的英雄。不管他说的是什么,可以肯定的是,他讲的每个字都绝对正统,绝对
英社。这张没有眼睛的脸和一张一合的嘴巴让温斯顿感觉微妙。他不像是真人,他是假
人。他用喉头说话,不是大脑。他在无意识状态下说出这些话,不能算真正的话,就像
鸭子嘎嘎的叫声。

\switchcolumn*

Syme had fallen silent for a moment, and with the handle of his spoon
was tracing patterns in the puddle of stew. The voice from the other
table quacked rapidly on, easily audible in spite of the surrounding
din.

\switchcolumn

赛姆安静了片刻,拿着汤勺在桌子上的那摊菜上画着什么。尽管餐厅里很吵,仍能听到隔壁桌的男人叽里呱啦的讲话声。

\switchcolumn*

``There is a word in Newspeak,'' said Syme. ``I don\textquotesingle t know
whether you know it: \emph{duckspeak}, to quack like a duck. It is one
of those interesting words that have two contradictory meanings. Applied
to an opponent, it is abuse; applied to someone you agree with, it is
praise.''

\switchcolumn

``新话里有个词,''赛姆说,``我不清楚你是否知道,叫鸭话,就是像鸭子那样嘎嘎地叫。这词很有趣,它有两个截然相反的意思。用在对手身上是骂人,用在自己人身上却是夸奖。''

\switchcolumn*

Unquestionably Syme will be vaporized, Winston thought again. He thought
it with a kind of sadness, although well knowing that Syme despised him
and slightly disliked him, and was fully capable of denouncing him as a
thought-criminal if he saw any reason for doing so. There was something
subtly wrong with Syme. There was something that he lacked: discretion,
aloofness, a sort of saving stupidity. You could not say that he was
unorthodox. He believed in the principles of Ingsoc, he venerated Big
Brother, he rejoiced over victories, he hated heretics, not merely with
sincerity but with a sort of restless zeal, an up-to-dateness of
information, which the ordinary Party member did not approach. Yet a
faint air of disreputability always clung to him. He said things that
would have been better unsaid, he had read too many books, he frequented
the Chestnut Tree Café, haunt of painters and musicians. There was no
law, not even an unwritten law, against frequenting the Chestnut Tree
Café, yet the place was somehow ill-omened. The old, discredited leaders
of the Party had been used to gather there before they were finally
purged. Goldstein himself, it was said, had sometimes been seen there,
years and decades ago. Syme\textquotesingle s fate was not difficult to
foresee. And yet it was a fact that if Syme grasped, even for three
seconds, the nature of his, Winston\textquotesingle s, secret opinions,
he would betray him instantly to the Thought Police. So would anybody
else, for that matter, but Syme more than most. Zeal was not enough.
Orthodoxy was unconsciousness.

\switchcolumn

赛姆一定会消失的,温斯顿想,他有点难过。虽然他知道赛姆看不起自己,不喜欢自己,且只要他认为有理,他就会揭发温斯顿是思想犯。赛姆身上隐约有些问题。他不够谨慎,也不够超脱,不知道隐藏自己的弱点。不能说他不正统。他真诚而热烈地坚信英社原则、敬仰老大哥、憎恶异端,这都是普通党员做不到的。但他却不是能让人放心依靠的人,他总是说些不该说的话。他读的书太多了,总是跑到艺术家聚集的栗树咖啡馆去。没有哪条法律禁止人们去栗树咖啡馆,但那个地方却很危险。一些被清洗的党的领导者之前也很喜欢到那里去,传说很多年前高德斯坦因也去过那里。赛姆的结局不难揣测。但一旦他发现了温斯顿那些秘密的念头,哪怕只有三秒钟,他也会立即向思想警察报告。当然别人也会这样,但赛姆尤其如此。仅有一腔热忱是不够的,正统思想就是无意识。

\switchcolumn*

Syme looked up. ``Here comes Parsons,'' he said.

\switchcolumn

赛姆抬起头,说:``帕森斯来了。''

\switchcolumn*

Something in the tone of his voice seemed to add, ``that bloody fool.''
Parsons, Winston\textquotesingle s fellow tenant at Victory Mansions,
was in fact threading his way across the room---tubby, middle-sized man
with fair hair and a froglike face. At thirty-five he was already
putting on rolls of fat at neck and waistline, but his movements were
brisk and boyish. His whole appearance was that of a little boy grown
large, so much so that although he was wearing the regulation overalls,
it was almost impossible not to think of him as being dressed in the
blue shorts, gray shirt, and red neckerchief of the Spies. In
visualizing him one saw always a picture of dimpled knees and sleeves
rolled back from pudgy forearms. Parsons did, indeed, invariably revert
to shorts when a community hike or any other physical activity gave him
an excuse for doing so. He greeted them both with a cheery ``Hullo,
hullo!'' and sat down at the table, giving off an intense smell of sweat.
Beads of moisture stood out all over his pink face. His powers of
sweating were extraordinary. At the Community Center you could always
tell when he had been playing table tennis by the dampness of the bat
handle. Syme had produced a strip of paper on which there was a long
column of words, and was studying it with an ink pencil between his
fingers.

\switchcolumn

他的声音里似乎有这样一层意味:``他是个讨厌的傻瓜。''帕森斯是温斯顿在胜利大厦
的邻居,他穿过大厅走了过来。他有些胖,身高中等,头发浅黄,有一张神似青蛙的脸。
他不过三十五岁,脖子和腰上就堆着一团团脂肪,但他的动作却像小孩子一样敏捷,他
看上去就好像发育得过猛的小男孩。他虽然穿着一般的制服,却仍让人觉得套在他身上
的是侦察队的蓝短裤、灰衬衫、红领巾。每每想起他,眼前总会浮现出胖胖的膝盖和从
卷起的袖管中露出的短粗手臂。事实上,帕森斯确实如此,只要是参加集体远足或其他
什么体育运动,他就会穿上短裤。他兴高采烈地向温斯顿和赛姆打招呼:``你好!你
好!''在他们的桌子旁坐下,一股浓烈的汗臭味立即弥漫开来。他粉红色的面庞上挂着
汗珠,他出汗的能力令人震惊。在集体活动中心,只要看到湿乎乎的乒乓球拍,就知道
他刚刚打完球。 赛姆拿出一张写有单词的纸,埋头研究。

\switchcolumn*

``Look at him working away in the lunch hour,'' said Parsons, nudging
Winston. ``Keenness, eh? What\textquotesingle s that
you\textquotesingle ve got there, old boy? Something a bit too brainy
for me, I expect. Smith, old boy, I\textquotesingle ll tell you why
I\textquotesingle m chasing you. It\textquotesingle s that sub you
forgot to give me.''

\switchcolumn

``你看他吃饭时还在工作,''帕森斯推了推温斯顿,说,``真积极,哎?老伙计,你在看什么?这对我来说太高深了。史密斯,老伙计,我说说我为什么要找你,你忘记捐款了。''

\switchcolumn*

``Which sub is that?'' said Winston, automatically feeling for money.
About a quarter of one\textquotesingle s salary had to be ear-marked for
voluntary subscriptions, which were so numerous that it was difficult to
keep track of them.

\switchcolumn

``什么捐款?''温斯顿一边问一边掏钱。人们必须将工资的四分之一拿出来捐款,款项多得记不清。

\switchcolumn*

``For Hate Week. You know---the house-by-house fund. I\textquotesingle m
treasurer for our block. We\textquotesingle re making an all-out
effort---going to put on a tremendous show. I tell you, it
won\textquotesingle t be my fault if old Victory Mansions
doesn\textquotesingle t have the biggest outfit of flags in the whole
street. Two dollars you promised me.''

\switchcolumn

``仇恨周捐款,你知道每家都要捐。我是咱们那区的会计。我们可要好好表现,我和你说,如果胜利大厦亮出的旗帜不是那条街上最多的,可别怨我。你说过你要给我两块钱。''

\switchcolumn*

Winston found and handed over two creased and filthy notes, which
Parsons entered in a small notebook, in the neat handwriting of the
illiterate.

\switchcolumn

温斯顿将两张皱巴巴脏兮兮的钞票交给他。帕森斯用文盲才有的那种整齐的字体记在本子上。

\switchcolumn*

``By the way, old boy,'' he said, ``I hear that little beggar of mine let
fly at you with his catapult yesterday. I gave him a good dressing down
for it. In fact I told him I\textquotesingle d take the catapult away if
he does it again.''

\switchcolumn

``另外,老伙计,''他说,``我听说我家的小叫花子昨天拿弹弓打了你,我狠狠地教训了他。我告诉他,如果他再这么做,我就把弹弓没收。''

\switchcolumn*

``I think he was a little upset at not going to the execution,'' said
Winston.

\switchcolumn

``我想可能他没看到绞刑,不高兴了。''温斯顿说。

\switchcolumn*

``Ah, well---what I mean to say, shows the right spirit,
doesn\textquotesingle t it? Mischievous little beggars they are, both of
them, but talk about keenness! All they think about is the Spies, and
the war, of course. D\textquotesingle you know what that little girl of
mine did last Saturday, when her troop was on a hike out Berkhampstead
way? She got two other girls to go with her, slipped off from the hike,
and spent the whole afternoon following a strange man. They kept on his
tail for two hours, right through the woods, and then, when they got
into Amersham, handed him over to the patrols.''

\switchcolumn

``对,这正是我要说的。他的想法是好的,不是吗?他们俩都很淘气,但他们都很积极。他们成天想的不是侦察队就是打仗。你知道上个星期六,我的小女儿在伯克汉姆斯德远足时做了什么吗?她带着另外两个女孩偷偷离开队伍去跟踪一个可疑的陌生人,整整一个下午!她们跟了他两个小时,穿过树林,直到阿默夏姆,她们把那人交给了巡逻队。''

\switchcolumn*

``What did they do that for?'' said Winston, somewhat taken aback. Parsons
went on triumphantly:

\switchcolumn

``为什么?''温斯顿多少有些惊讶,帕森斯继续得意地说:

\switchcolumn*

``My kid made sure he was some kind of enemy agent---might have been
dropped by parachute, for instance. But here\textquotesingle s the
point, old boy. What do you think put her onto him in the first place?
She spotted he was wearing a funny kind of shoes---said
she\textquotesingle d never seen anyone wearing shoes like that before.
So the chances were he was a foreigner. Pretty smart for a nipper of
seven, eh?''

\switchcolumn

``我的孩子认定他是敌人的特务,举个例子,他有可能是空降下来的。问题的关键就在这里,老伙计,你知道是什么让她怀疑他的吗?她发现他的鞋子很奇怪,她从没看到有人穿这样的鞋子。因此,他很可能是外国人。七岁的孩子,多聪明,是不是?''

\switchcolumn*

``What happened to the man?'' said Winston.

\switchcolumn

``后来呢?那人怎样了?''温斯顿问。

\switchcolumn*

``Ah, that I couldn\textquotesingle t say, of course. But I
wouldn\textquotesingle t be altogether surprised if---'' Parsons made the
motion of aiming a rifle, and clicked his tongue for the explosion.

\switchcolumn

``这我就说不出了。不过,我不觉得惊讶,要是——''帕森斯做了一个用步枪瞄准的姿势,嘴里还模仿着射击的咔嚓声。

\switchcolumn*

``Good,'' said Syme abstractedly, without looking up from his strip of
paper.

\switchcolumn

``好。''赛姆敷衍地说,仍埋头看他的小纸条。

\switchcolumn*

``Of course we can\textquotesingle t afford to take chances,'' agreed
Winston dutifully.

\switchcolumn

``我们不能给他们制造机会。''温斯顿老老实实地表示赞同。

\switchcolumn*

``What I mean to say, there is a war on,'' said Parsons.

\switchcolumn

``我的意思是,现在还在打仗。''帕森斯说。这时,他们头顶上的电屏响起一阵号声,就好像要证明帕森斯的观点似的。但这次不是宣布打了胜仗,而是颁布一个公告。

\switchcolumn*

As though in confirmation of this, a trumpet call floated from the
telescreen just above their heads. However, it was not the proclamation
of a military victory this time, but merely an announcement from the
Ministry of Plenty.

\switchcolumn

``同志们!''电屏里传来一个年轻人激昂振奋的声音。``注意,同志们!有个好消息要告诉大家。我们取得了生产大胜利!截止到现在,各类消费品的生产数字表明,过去的一年里,我们的生活水平提高了二十多个百分点。今天上午,大洋国各地都举行了自发的游行,工人们走出工厂、办公室,高举旗帜涌上街头,以表达对老大哥的感激之情,老大哥的英明领导缔造了我们崭新的幸福生活。这里有部分统计数字。食品\ldots\ldots''

\switchcolumn*

``Comrades!'' cried an eager youthful voice. ``Attention, comrades! We have
glorious news for you. We have won the battle for production! Returns
now completed of the output of all classes of consumption goods show
that the standard of living has risen by no less than twenty per cent
over the past year. All over Oceania this morning there were
irrepressible spontaneous demonstrations when workers marched out of
factories and offices and paraded through the streets with banners
voicing their gratitude to Big Brother for the new, happy life which his
wise leadership has bestowed upon us. Here are some of the completed
figures. Foodstuffs---''

\switchcolumn

``我们崭新的幸福生活''出现了很多次,最近,富部很喜欢使用这个词。帕森斯的注意力被号声吸引走。他呆呆地听着,一本正经又目光空洞,由于似懂非懂,他露出厌倦的神情。他不理解这些数字的含义,但他知道它们令人满意。他拿出肮脏的烟斗,烟斗里已经装上了黑黑的烟草。每星期只能得到一百克烟草,很少能将烟斗装满。温斯顿抽的是胜利牌香烟,他小心翼翼地将香烟横着拿在手里。现在他只剩下四根烟,要买新的必须等到明天。他闭上眼睛,不再理会身旁的噪声,专心聆听电屏里的声音。有人为感谢老大哥将巧克力供应量提高到每星期二十克举行了游行。温斯顿心想,昨天才刚刚宣布要将供应量减少到每周二十克,不过二十四小时,他们怎么就完全接受了呢?没错,他们完全接受了这件事。帕森斯就很容易地接受了。他就像牲畜一样愚蠢,隔壁桌上看不见眼睛的家伙也接受了,还情绪狂热,怒气腾腾地要求将那些提到上星期定量是三十克的人揭发出来、逮捕起来、蒸发消灭。赛姆也接受了,他要复杂一些,运用了双重思想。这么说就只有他一人还记得这件事吗?

\switchcolumn*

The phrase ``our new, happy life'' recurred several times. It had been a
favorite of late with the Ministry of Plenty. Parsons, his attention
caught by the trumpet call, sat listening with a sort of gaping
solemnity, a sort of edified boredom. He could not follow the figures,
but he was aware that they were in some way a cause for satisfaction. He
had lugged out a huge and filthy pipe which was already half full of
charred tobacco. With the tobacco ration at a hundred grams a week it
was seldom possible to fill a pipe up to the top. Winston was smoking a
Victory Cigarette which he held carefully horizontal. The new ration did
not start till tomorrow and he had only four cigarettes left. For the
moment he had shut his ears to the remoter noises and was listening to
the stuff that streamed out of the telescreen. It appeared that there
had even been demonstrations to thank Big Brother for raising the
chocolate ration to twenty grams a week. And only yesterday, he
reflected, it had been announced that the ration was to be reduced to
twenty grams a week. Was it possible that they could swallow that, after
only twenty-four hours? Yes, they swallowed it. Parsons swallowed it
easily, with the stupidity of an animal. The eyeless creature at the
other table swallowed it fanatically, passionately, with a furious
desire to track down, denounce, and vaporize anyone who should suggest
that last week the ration had been thirty grams. Syme, too---in some
more complex way, involving doublethink---Syme swallowed it. Was he,
then, \emph{alone} in the possession of a memory?

\switchcolumn

一连串不可思议的数字从电屏里源源不断地涌出。和去年相比,除了疾病、犯罪和精神病外,食物、服装、房屋、家具、饭锅、燃料、轮船、直升机、书、婴儿——所有的一切都增加了。每一年,每一分钟,每个人,每件事,都在飞速发展。温斯顿拿起勺子,像赛姆那样蘸着桌子上那摊菜画了起来。他画了一条线,画出一幅图。他愤愤地寻思着他的物质生活。难道会一直这样下去吗?食物永远是这个味道?他放眼望去,低低矮矮的食堂里挤满了人,人们频繁地蹭着墙壁,把墙蹭成了黑色。破烂的铁制的桌椅排得紧紧密密,以至于人一坐下来,就会碰到别人的手肘。勺柄是弯的,铁盘变了形,马克杯粗糙不堪。每样东西都沾满油污,每一道缝隙都积满灰尘。劣质杜松子酒、劣质咖啡、金属味的炖菜和脏衣服的气味混杂在一起,充斥了整个空间。你有权拥有的东西被骗走了,你的肚子,你的皮肤,都在做着无声的抗议。的确,在他的记忆里,任何东西都和现在没有什么不同。无论何时,食物从未充足过,衣物总有破洞,家具总是破旧,房间里的暖气总也供应不足,地铁一直拥挤,房子从来歪歪斜斜;面包是黑色的,茶叶是稀少的,咖啡是低劣的,香烟是匮乏的——除了人造杜松子酒,没有哪样东西又丰沛又便宜。而随着你上了年纪,情况会越来越糟。但如果因为生活在肮脏贫乏的环境里会让你不快,如果你厌恶这冗长的冬天、破烂的袜子、停开的电梯,厌恶这冰冷的自来水、糙劣的肥皂、漏烟丝的香烟以及那难以下咽的食物。那不是说明,这样的状况是不正常的吗?若非如此,如果你没有关于旧时代的记忆,如果你不知道从前并非如此,你为什么还会觉得这些是难以忍受的?

\switchcolumn*

The fabulous statistics continued to pour out of the telescreen. As
compared with last year there was more food, more clothes, more houses,
more furniture, more cooking pots, more fuel, more ships, more
helicopters, more books, more babies---more of everything except
disease, crime, and insanity. Year by year and minute by minute,
everybody and everything was whizzing rapidly upwards. As Syme had done
earlier, Winston had taken up his spoon and was dabbling in the
pale-colored gravy that dribbled across the table, drawing a long streak
of it out into a pattern. He meditated resentfully on the physical
texture of life. Had it always been like this? Had food always tasted
like this? He looked round the canteen. A low-ceilinged, crowded room,
its walls grimy from the contact of innumerable bodies; battered metal
tables and chairs, placed so close together that you sat with elbows
touching; bent spoons, dented trays, coarse white mugs; all surfaces
greasy, grime in every crack; and a sourish, composite smell of bad gin
and bad coffee and metallic stew and dirty clothes. Always in your
stomach and in your skin there was a sort of protest, a feeling that you
had been cheated of something that you had a right to. It was true that
he had no memories of anything greatly different. In any time that he
could accurately remember, there had never been quite enough to eat, one
had never had socks or underclothes that were not full of holes,
furniture had always been battered and rickety, rooms underheated, tube
trains crowded, houses falling to pieces, bread dark-colored, tea a
rarity, coffee filthy-tasting, cigarettes insufficient---nothing cheap
and plentiful except synthetic gin. And though, of course, it grew worse
as one\textquotesingle s body aged, was it not a sign that this was
\emph{not} the natural order of things, if one\textquotesingle s heart
sickened at the discomfort and dirt and scarcity, the interminable
winters, the stickiness of one\textquotesingle s socks, the lifts that
never worked, the cold water, the gritty soap, the cigarettes that came
to pieces, the food with its strange evil tastes? Why should one feel it
to be intolerable unless one had some kind of ancestral memory that
things had once been different?

\switchcolumn

他再一次环顾了食堂。几乎所有人都面目丑陋,那些没穿蓝色的工作服的也没好看到哪儿去。食堂的另一端,有个身材矮小、神似甲虫的人,一双鼠目东张西望,充满猜疑,他独自一人坐在一旁喝咖啡。温斯顿想,若没有四下观察,人们就很容易相信大部分小伙子都高大魁梧,大部分姑娘都胸脯丰满,大部分人都长着金黄色的头发,都有被太阳晒出来的健康肤色,都生机勃勃,无忧无虑——这正是党塑造的完美体格。但实际上,一号空降场的大多数人都瘦小干黑,病病恹恹。此外,部里到处都是甲虫一般的人。他们过早发胖,四肢粗短,终日忙碌,行动敏捷。他们臃肿的脸上多镶了一双细小的眼睛,还总挂着令人捉摸不透的表情。在党的领导下,这样的人繁殖迅速。

\switchcolumn*

He looked round the canteen again. Nearly everyone was ugly, and would
still have been ugly even if dressed otherwise than in the uniform blue
overalls. On the far side of the room, sitting at a table alone, a
small, curiously beetlelike man was drinking a cup of coffee, his little
eyes darting suspicious glances from side to side. How easy it was,
thought Winston, if you did not look about you, to believe that the
physical type set up by the Party as an ideal---tall muscular youths and
deep-bosomed maidens, blond-haired, vital, sunburnt, carefree---existed
and even predominated. Actually, so far as he could judge, the majority
of people in Airstrip One were small, dark, and ill-favored. It was
curious how that beetlelike type proliferated in the Ministries: little
dumpy men, growing stout very early in life, with short legs, swift
scuttling movements, and fat inscrutable faces with very small eyes. It
was the type that seemed to flourish best under the dominion of the
Party.

\switchcolumn

富部的播报结束了,号声再次响了起来,之后又播起了音乐,声音很小。之前那密集的数字轰炸让帕森斯兴奋不已,他将烟斗从嘴上拿开。

\switchcolumn*

The announcement from the Ministry of Plenty ended on another trumpet
call and gave way to tinny music. Parsons, stirred to vague enthusiasm
by the bombardment of figures, took his pipe out of his mouth.

\switchcolumn

``今年,富部干得不错。''他摇晃着脑袋,说,``顺便说下,史密斯伙计,我猜你也没有刀片借我吧?''

\switchcolumn*

``The Ministry of Plenty\textquotesingle s certainly done a good job this
year,'' he said with a knowing shake of his head. ``By the way, Smith old
boy, I suppose you haven\textquotesingle t got any razor blades you can
let me have?''

\switchcolumn

``一片都没有,''温斯顿说,``我的刀片用了六个星期了。''

\switchcolumn*

``Not one,'' said Winston. ``I\textquotesingle ve been using the same blade
for six weeks myself.''

\switchcolumn

``没关系,我就是问一下,老伙计。''

\switchcolumn*

``Ah, well---just thought I\textquotesingle d ask you, old boy.''

\switchcolumn

``抱歉。''温斯顿说。

\switchcolumn*

``Sorry,'' said Winston.

\switchcolumn

富部播报时,隔壁桌叽叽呱呱的声音暂停了一下,现在它又像刚才一样吵。不知为何,温斯顿想起帕森斯太太,想起她稀疏的头发和嵌着灰尘的脸。不用两年,她的孩子就会向思想警察揭发她,她就会消失不见。赛姆也会消失,温斯顿也会消失,奥布兰同样会消失。但帕森斯不会,那个叽叽呱呱的家伙不会,各个部门里那些庸碌往来的甲虫们更不会。

\switchcolumn*

The quacking voice from the next table, temporarily silenced during the
Ministry\textquotesingle s announcement, had started up again, as loud
as ever. For some reason Winston suddenly found himself thinking of Mrs.
Parsons, with her wispy hair and the dust in the creases of her face.
Within two years those children would be denouncing her to the Thought
Police. Mrs. Parsons would be vaporized. Syme would be vaporized.
Winston would be vaporized. O\textquotesingle Brien would be vaporized.
Parsons, on the other hand, would never be vaporized. The eyeless
creature with the quacking voice would never be vaporized. The little
beetlelike men who scuttled so nimbly through the labyrinthine corridors
of Ministries---they, too, would never be vaporized. And the girl with
dark hair, the girl from the Fiction Department---she would never be
vaporized either. It seemed to him that he knew instinctively who would
survive and who would perish, though just what it was that made for
survival, it was not easy to say.

\switchcolumn

还有那个黑头发的女孩、小说司里的姑娘——她永远都不会被蒸发。本能告诉他谁能活下去,谁会被消灭。但是究竟是什么能让人幸存下来就很难说了。

\switchcolumn*

At this moment he was dragged out of his reverie with a violent jerk.
The girl at the next table had turned partly round and was looking at
him. It was the girl with dark hair. She was looking at him in a
sidelong way, but with curious intensity. The instant that she caught
his eye she looked away again.

\switchcolumn

突然,他回到现实。邻桌的那个女孩正侧着身子看着他,她的目光虽然是斜的却很专注。这让人十分困惑。当她发现温斯顿已经注意到自己,就把眼神挪开了。

\switchcolumn*

The sweat started out on Winston\textquotesingle s backbone. A horrible
pang of terror went through him. It was gone almost at once, but it left
a sort of nagging uneasiness behind. Why was she watching him? Why did
she keep following him about? Unfortunately he could not remember
whether she had already been at that table when he arrived, or had come
there afterwards. But yesterday, at any rate, during the Two Minutes
Hate, she had sat immediately behind him when there was no apparent need
to do so. Quite likely her real object had been to listen to him and
make sure whether he was shouting loudly enough.

\switchcolumn

温斯顿非常惊恐,后背还冒出了汗。但这感觉很快就过去了,只留下一抹不安。她为什么要看他?她为什么跟着他?他想不起她是否在他来之前就坐在那里。昨天的两分钟仇恨会,她坐在他身后,而她完全没必要这样。很有可能,她真正的意图是听听他喊得够不够响。

\switchcolumn*

His earlier thought returned to him: probably she was not actually a
member of the Thought Police, but then it was precisely the amateur spy
who was the greatest danger of all. He did not know how long she had
been looking at him, but perhaps for as much as five minutes, and it was
possible that his features had not been perfectly under control. It was
terribly dangerous to let your thoughts wander when you were in any
public place or within range of a telescreen. The smallest thing could
give you away. A nervous tic, an unconscious look of anxiety, a habit of
muttering to yourself---anything that carried with it the suggestion of
abnormality, of having something to hide. In any case, to wear an
improper expression on your face (to look incredulous when a victory was
announced, for example) was itself a punishable offense. There was even
a word for it in Newspeak: \emph{facecrime}, it was called.

\switchcolumn

之前的想法又浮现在他脑海里:她不一定是思想警察,但就是业余警察才最危险。他不知道她看了他多久,也许五分钟。也许,他并没有控制好他的表情,在公众场合和在电屏跟前放纵思绪非常危险。任何微小的细节,如不自觉地抽搐,不经意地烦恼,自言自语的习惯——只要看起来不正常,都有可能暴露你。那些不恰当的表情,如听胜利公告时将信将疑,本身就是犯罪。关于这个,新话里还有一个名词:脸罪。

\switchcolumn*

The girl had turned her back on him again. Perhaps after all she was not
really following him about; perhaps it was coincidence that she had sat
so close to him two days running. His cigarette had gone out, and he
laid it carefully on the edge of the table. He would finish smoking it
after work, if he could keep the tobacco in it. Quite likely the person
at the next table was a spy of the Thought Police, and quite likely he
would be in the cellars of the Ministry of Love within three days, but a
cigarette end must not be wasted. Syme had folded up his strip of paper
and stowed it away in his pocket. Parsons had begun talking again.

\switchcolumn

女孩又回过头来看他,可能她并非在监视他,可能连续两天都靠近他坐只是巧合。他的烟熄了,为了不让烟丝掉出来,他小心地将它放在桌边。隔壁桌的人可能是思想警察,也许不出三天,温斯顿就会被抓到仁爱部的地下室去。但香烟不能浪费,他可以在下班后继续抽它。赛姆将那张纸条折好,放进口袋。帕森斯又张开了嘴。

\switchcolumn*

``Did I ever tell you, old boy,'' he said, chuckling round the stem of his
pipe, ``about the time when those two nippers of mine set fire to the old
market-woman\textquotesingle s skirt because they saw her wrapping up
sausages in a poster of B.B.? Sneaked up behind her and set fire to it
with a box of matches. Burned her quite badly, I believe. Little
beggars, eh? But keen as mustard! That\textquotesingle s a first-rate
training they give them in the Spies nowadays---better than in my day,
even. What d\textquotesingle you think\textquotesingle s the latest
thing they\textquotesingle ve served them out with? Ear trumpets for
listening through keyholes! My little girl brought one home the other
night---tried it out on our sitting room door, and reckoned she could
hear twice as much as with her ear to the hole. Of course
it\textquotesingle s only a toy, mind you. Still, gives
\textquotesingle em the right idea, eh?''

\switchcolumn

``还没和你说过,老伙计!''他的嘴里还叼着烟斗,``有一次,我的两个孩子在市场上把一个老太婆的裙子给烧了,因为他们看到她用老大哥的画像包香肠。他们偷偷绕到她背后,用了一盒火柴去烧她的裙子,可把她烧得够呛。那两个孩子真是积极啊!现在他们在侦察队里受到了一流的训练,比我当年要好得多。你知道他们发给孩子们什么吗?插在钥匙孔里的窃听器!我的小姑娘那天晚上带回来一个,还插在我们起居室的门上做实验,听到的声音比趴在钥匙孔上大一倍。别看这是个玩具,却能让他们树立正确的思想,是不是?''

\switchcolumn*

At this moment the telescreen let out a piercing whistle. It was the
signal to return to work. All three men sprang to their feet to join in
the struggle round the lifts, and the remaining tobacco fell out of
Winston\textquotesingle s cigarette.

\switchcolumn

电屏里发出尖利的哨音,该去工作了。三个人都站起身争先恐后地去挤电梯,温斯顿香烟里的烟丝掉了出来。

\switchcolumn*


\section{VI}\label{vi}

\switchcolumn

\section*{六}\label{ux516d}

\switchcolumn*

Winston was writing in his diary:

\switchcolumn

温斯顿在日记里这样写道:

\switchcolumn*

\begin{quotation}
It was three years ago. It was on a dark evening, in a narrow side
street near one of the big railway stations. She was standing near a
doorway in the wall, under a street lamp that hardly gave any light. She
had a young face, painted very thick. It was really the paint that
appealed to me, the whiteness of it, like a mask, and the bright red
lips. Party women never paint their faces. There was nobody else in the
street, and no telescreens. She said two dollars. I---
\end{quotation}

\switchcolumn

三年前的一个晚上,天很黑。火车站附近窄窄的街道上,她靠门站着,身旁是晦暗不明的街灯。她很年轻,脸上涂了厚厚的粉。吸引我的正是她的妆容,白白的,好像一个面具。还有她的红唇。女党员从不化妆。街上没有其他的人,也没有电屏,她说两块钱。我——

\switchcolumn*

For the moment it was too difficult to go on. He shut his eyes and
pressed his fingers against them, trying to squeeze out the vision that
kept recurring. He had an almost overwhelming temptation to shout a
string of filthy words at the top of his voice. Or to bang his head
against the wall, to kick over the table and hurl the inkpot through the
window---to do any violent or noisy or painful thing that might black
out the memory that was tormenting him.

\switchcolumn

一时间,他很难写下去。他闭上眼睛,用手指摩挲眼皮,似乎要将那不断出现在脑际的画面给挤出来。他按捺不住思绪,想放开喉咙大声呼喊,骂几句脏话,或者用脑袋撞墙,踢翻桌子,将墨水瓶扔到窗外去。总之,大吵大闹也好,让身体疼痛也罢,他想做点什么忘记那些折磨他的事情。

\switchcolumn*

Your worst enemy, he reflected, was your own nervous system. At any
moment the tension inside you was liable to translate itself into some
visible symptom. He thought of a man whom he had passed in the street a
few weeks back: a quite ordinary-looking man, a Party member, aged
thirty-five or forty, tallish and thin, carrying a brief case. They were
a few meters apart when the left side of the man\textquotesingle s face
was suddenly contorted by a sort of spasm. It happened again just as
they were passing one another: it was only a twitch, a quiver, rapid as
the clicking of a camera shutter, but obviously habitual. He remembered
thinking at the time: that poor devil is done for. And what was
frightening was that the action was quite possibly unconscious. The most
deadly danger of all was talking in your sleep. There was no way of
guarding against that, so far as he could see.

\switchcolumn

人最大的敌人就是自己的神经系统。人内心的紧张随时可能清楚地流露出来。几周前,他在街上遇到一个人,一个党员,外表非常平凡,大约三四十岁,身材瘦长,手里拿着公文包。在二人相距数米时,那人的左颊突然抽搐了一下,两人擦肩而过时,那人又像按相机快门那样迅速地抖了一下。尽管一看就是习惯,温斯顿仍然断定,这可怜的家伙完蛋了。下意识的动作最可怕,防不胜防,而说梦话又是其中最致命的一种。

\switchcolumn*

He drew in his breath and went on writing:

\switchcolumn

温斯顿吸了口气,又继续写道:

\switchcolumn*

\begin{quotation}
I went with her through the doorway and across a backyard into a
basement kitchen. There was a bed against the wall, and a lamp on the
table, turned down very low. She---
\end{quotation}

\switchcolumn

我和她一起进了门,穿过后院,来到地下室的厨房。那里有一张床,靠着墙放着,还有一张桌子,桌上是一盏昏暗的灯。她——

\switchcolumn*

His teeth were set on edge. He would have liked to spit. Simultaneously
with the woman in the basement kitchen he thought of Katharine, his
wife. Winston was married---had been married, at any rate; probably he
still was married, for so far as he knew his wife was not dead. He
seemed to breathe again the warm stuffy odor of the basement kitchen, an
odor compounded of bugs and dirty clothes and villainous cheap scent,
but nevertheless alluring, because no woman of the Party ever used
scent, or could be imagined as doing so. Only the proles used scent. In
his mind the smell of it was inextricably mixed up with fornication.

\switchcolumn

他咬牙切齿,想吐唾沫。和那女人共处时,他想起了他的妻子凯瑟琳。温斯顿结过婚,也许现在,他仍算得上已婚人士,因为他觉得他的妻子仍然活着。他好像又闻到了地下室里那种混杂着臭虫、脏衣服和廉价香水的特殊气味,那里燥热非常又十分诱人。女党员不用香水,没法想象她们用香水的样子。香水是群众用的,在温斯顿心中,香水的气味和私通交融一体,不可分割。

\switchcolumn*

When he had gone with that woman it had been his first lapse in two
years or thereabouts. Consorting with prostitutes was forbidden, of
course, but it was one of those rules that you could occasionally nerve
yourself to break. It was dangerous, but it was not a life-and-death
matter. To be caught with a prostitute might mean five years in a
forced-Plabor camp: not more, if you had committed no other offense. And
it was easy enough, provided that you could avoid being caught in the
act. The poorer quarters swarmed with women who were ready to sell
themselves. Some could even be purchased for a bottle of gin, which the
proles were not supposed to drink. Tacitly the Party was even inclined
to encourage prostitution, as an outlet for instincts which could not be
altogether suppressed. Mere debauchery did not matter very much, so long
as it was furtive and joyless, and only involved the women of a
submerged and despised class. The unforgivable crime was promiscuity
between Party members. But---though this was one of the crimes that the
accused in the great purges invariably confessed to---it was difficult
to imagine any such thing actually happening.

\switchcolumn

两年以来,这是他第一次做堕落的事。嫖妓是禁止的,但有时你却会鼓起勇气反抗这一禁令。这很危险,却说不上生死攸关。若没有其他罪行,嫖妓被抓的只需在劳改营里待上五年。避免被抓现行并不困难。贫民区里到处是准备好出卖自己肉体的女人,有时你只需支付一瓶杜松子酒——群众不允许买这种酒。在私底下,党鼓励卖淫,如此无法被完全压制住的本能就有了发泄途径。只要偷偷地、了无乐趣地与卑微可鄙的下层女人发生关系,一时的放荡无关紧要。只有在党员之间,这种事才不可饶恕。不过很难想象它会在现实中发生,尽管每次大清洗所有被告都承认犯了这样的罪。

\switchcolumn*

The aim of the Party was not merely to prevent men and women from
forming loyalties which it might not be able to control. Its real,
undeclared purpose was to remove all pleasure from the sexual act. Not
love so much as eroticism was the enemy, inside marriage as well as
outside it. All marriages between Party members had to be approved by a
committee appointed for the purpose, and---though the principle was
never clearly stated---permission was always refused if the couple
concerned gave the impression of being physically attracted to one
another. The only recognized purpose of marriage was to beget children
for the service of the Party. Sexual intercourse was to be looked on as
a slightly disgusting minor operation, like having an enema. This again
was never put into plain words, but in an indirect way it was rubbed
into every Party member from childhood onwards. There were even
organizations such as the Junior Anti-Sex League which advocated
complete celibacy for both sexes. All children were to be begotten by
artificial insemination (\emph{artsem}, it was called in Newspeak) and
brought up in public institutions. This, Winston was aware, was not
meant altogether seriously, but somehow it fitted in with the general
ideology of the Party. The Party was trying to kill the sex instinct,
or, if it could not be killed, then to distort it and dirty it. He did
not know why this was so, but it seemed natural that it should be so.
And so far as the women were concerned, the Party\textquotesingle s
efforts were largely successful.

\switchcolumn

党不仅要防止男人和女人结成它难以控制的忠诚同盟,党更要将快感从性行为中剥离。
而无论在婚姻中,还是婚姻外,与其说性欲是敌人,不如说爱欲才是敌人。党员若结婚
则必须经过一个专门的委员会的批准,虽然该委员会从未明示过批准的原则,但假如两
人给人留下了``肉体吸引''的印象,他们的申请就会遭到拒绝。唯一得到承认的婚姻的
目的便是为党生育后代。性交被当做类似灌肠的令人恶心的小手术。关于这点,虽然没
有明确地提出过,却间接地,从孩童时期就向人灌输。不仅如此,还有类似反性同盟这
样的组织,向男男女女宣扬独身生活的好处,并认为所有人都应该采取人工授精(新话
称作`人授')的方式进行生育,生下的孩子则要交由公家抚养。温斯顿明白,这并不意
味着所有这些都要被严格执行,但它们却迎合了党的意识形态。党试图扼杀性的本能,
就算不能完全扼杀,也歪曲它,丑化它。他不知道为什么要这样,但这似乎是自然而然
的事。至少对女人们而言,党在这方面的努力没有白费。

\switchcolumn*

He thought again of Katharine. It must be nine, ten---nearly eleven
years since they had parted. It was curious how seldom he thought of
her. For days at a time he was capable of forgetting that he had ever
been married. They had only been together for about fifteen months. The
Party did not permit divorce, but it rather encouraged separation in
cases where there were no children.

\switchcolumn

他又回忆起凯瑟琳。他们大概有九年,十年——快十一年没在一起了。奇怪的是,他很少想起她。他甚至可以一连几天忘记自己曾经结过婚。他们相处的时间只有大约十五个月。党禁止离婚,但如果没有孩子却鼓励分居。

\switchcolumn*

Katharine was a tall, fair-haired girl, very straight, with splendid
movements. She had a bold, aquiline face, a face that one might have
called noble until one discovered that there was as nearly as possible
nothing behind it. Very early in their married life he had
decided---though perhaps it was only that he knew her more intimately
than he knew most people---that she had without exception the most
stupid, vulgar, empty mind that he had ever encountered. She had not a
thought in her head that was not a slogan, and there was no imbecility,
absolutely none, that she was not capable of swallowing if the Party
handed it out to her. ``The human sound track'' he nicknamed her in his
own mind. Yet he could have endured living with her if it had not been
for just one thing---sex.

\switchcolumn

凯瑟琳有一头浅黄色的头发,身材高挑,举止优雅。她的脸轮廓分明,像鹰一样,若你没有察觉到藏在这张脸背后的空乏,你会觉得她是个高贵的人。和其他人相比他们有更多机会亲密接触,因此刚结婚,他就发现她是他所有认识的人里,最愚蠢、最庸俗、最无知的。她的脑子里只有口号,只有党灌输给她的蠢话,但凡是党提出的,她都悉数接受。他在心里给她起了个外号:人体录音带。不过,若不是因为那件事,他仍可以勉强和她生活下去。那件事就是性。

\switchcolumn*

As soon as he touched her she seemed to wince and stiffen. To embrace
her was like embracing a jointed wooden image. And what was strange was
that even when she was clasping him against her he had the feeling that
she was simultaneously pushing him away with all her strength. The
rigidity of her muscles managed to convey that impression. She would lie
there with shut eyes, neither resisting nor co-operating, but
\emph{submitting}. It was extraordinarily embarrassing and, after a
while, horrible. But even then he could have borne living with her if it
had been agreed that they should remain celibate. But curiously enough
it was Katharine who refused this. They must, she said, produce a child
if they could. So the performance continued to happen, once a week quite
regularly, whenever it was not impossible. She used even to remind him
of it in the morning, as something which had to be done that evening and
which must not be forgotten. She had two names for it. One was ``making a
baby,'' and the other was ``our duty to the Party'' (yes, she had actually
used that phrase). Quite soon he grew to have a feeling of positive
dread when the appointed day came round. But luckily no child appeared,
and in the end she agreed to give up trying, and soon afterwards they
parted.

\switchcolumn

只要碰触到她,她就会躲开,而且全身绷紧。抱着她就像抱着木头人,甚至当她主动拥抱他时,她那绷紧的身体也让他觉得她正用尽浑身力气推开他。她紧闭双眼躺在那里,忍受着一切,不拒绝,也不合作,这让人非常尴尬,而随着时间的推移,这简直令人厌恶。另一方面,如果双方都同意禁欲,他还能和她过下去。问题是凯瑟琳不同意这样做。她说,只要可能,他们就必须生个孩子。因此,他们的性生活非常有规律,每个星期都要来一次,除非在她不可能受孕的时间。她甚至会在早上提醒他,把它当做一件当天晚上必须完成的任务。她称这件事为``制造孩子''、``对党的义务''(她的确这么说)。没过多久,只要规定的日期近了,他就感到恐惧。所幸,他们没能生出孩子。最后她同意放弃。不久,他们就分居了。

\switchcolumn*

Winston sighed inaudibly. He picked up his pen again and wrote:

\switchcolumn

温斯顿默默地叹了口气,拿起笔写道:

\switchcolumn*

\begin{quotation}
She threw herself down on the bed, and at once, without any kind
of preliminary, in the most coarse, horrible way you can imagine, pulled
up her skirt. I---
\end{quotation}

\switchcolumn

``她一头躺到床上,没有任何准备动作,用最粗野、最可怕的方式撩起裙子,我——''

\switchcolumn*

He saw himself standing there in the dim lamplight, with the smell of
bugs and cheap scent in his nostrils, and in his heart a feeling of
defeat and resentment which even at that moment was mixed up with the
thought of Katharine\textquotesingle s white body, frozen forever by the
hypnotic power of the Party. Why did it always have to be like this? Why
could he not have a woman of his own instead of these filthy scuffles at
intervals of years? But a real love affair was an almost unthinkable
event. The women of the Party were all alike. Chastity was as deeply
ingrained in them as Party loyalty. By careful early conditioning, by
games and cold water, by the rubbish that was dinned into them at school
and in the Spies and the Youth League, by lectures, parades, songs,
slogans, and martial music, the natural feeling had been driven out of
them. His reason told him that there must be exceptions, but his heart
did not believe it. They were all impregnable, as the Party intended
that they should be. And what he wanted, more even than to be loved, was
to break down that wall of virtue, even if it were only once in his
whole life. The sexual act, successfully performed, was rebellion.
Desire was thoughtcrime. Even to have awakened Katharine, if he could
have achieved it, would have been like a seduction, although she was his
wife.

\switchcolumn

他看到自己站在昏暗的灯前,鼻子里充满了臭虫和廉价香水的味道,他觉得自己被什么东西击败了,满心怨懑,而此时,这感觉又和对凯瑟琳的想念混杂在一起,她那白皙的肉体在被党催眠后,冻结了。为什么事情总是这个样子?为什么他不能拥有自己的女人以结束这种每隔一两年便不得不去做肮脏下流之事的生活?真正意义上的爱情几乎不能去想象。所有女党员都差不多,禁欲和对党的忠诚一样在她们心中根深蒂固。她们的天性被孩提时代的说教,被游戏和冷水浴,被学校、侦察队、青年团灌输的垃圾,被演讲、游行、歌曲、口号、军乐,清除得干干净净。理智告诉他一定存在例外,但他的内心却不肯相信。她们个个遵循党的要求,个个坚不可摧。对他而言,与其说他希望有女人来爱他,不如说他更想摧毁那道贞洁之墙,哪怕这一生只成功一两次也好。美妙的性活动本身就是反抗。欲望就是思想罪。尽管凯瑟琳是他的妻子,若他唤起她的欲望,就相当于诱奸。

\switchcolumn*

But the rest of the story had got to be written down. He wrote:

\switchcolumn

他要把剩下的故事写完:

\switchcolumn*

\begin{quotation}
I turned up the lamp. When I saw her in the light---
\end{quotation}

\switchcolumn

我拧亮灯,借着灯光看到她——

\switchcolumn*

After the darkness, the feeble light of the paraffin lamp had seemed
very bright. For the first time he could see the woman properly. He had
taken a step toward her and then halted, full of lust and terror. He was
painfully conscious of the risk he had taken in coming here. It was
perfectly possible that the patrols would catch him on the way out; for
that matter they might be waiting outside the door at this moment. If he
went away without even doing what he had come here to do---!

\switchcolumn

在黑暗中待得久了,煤油灯发出的光芒也显得格外明亮。他第一次看清了那个女人的样子。他向她靠近了一步,又停下来,欲望和恐惧占满了他的心。他意识到到这里来有多么危险,这让他十分痛苦。很有可能,他一出去,巡逻队就将他逮捕,也许他们此时已经守在门外了。可如果没达到目的就走——

\switchcolumn*

It had got to be written down, it had got to be confessed. What he had
suddenly seen in the lamplight was that the woman was \emph{old}. The
paint was plastered so thick on her face that it looked as though it
might crack like a cardboard mask. There were streaks of white in her
hair; but the truly dreadful detail was that her mouth had fallen a
little open, revealing nothing except a cavernous blackness. She had no
teeth at all.

\switchcolumn

一定要老老实实地写下来。在灯光下,他看清了她,她上了年纪,她脸上的粉厚得就像快要折断的硬纸面具。她的头发已经泛白,而最可怕的是,她张开的嘴宛若一个黑洞,里面什么都没有,她没有牙。

\switchcolumn*

He wrote hurriedly, in scrabbling handwriting:

\switchcolumn

他慌慌忙忙地写着,字迹潦草不堪:

\switchcolumn*

\begin{quotation}
When I saw her in the light she was quite an old woman, fifty
years old at least. But I went ahead and did it just the same.
\end{quotation}

\switchcolumn

借着灯光,我看清了她。她是个很老的女人,至少有五十岁。可我还是走上前做了那事。

\switchcolumn*

He pressed his fingers against his eyelids again. He had written it down
at last, but it made no difference. The therapy had not worked. The urge
to shout filthy words at the top of his voice was as strong as ever.

\switchcolumn

他用手指压了压眼皮,终于把它写下来了。不过,也没有什么特殊的感觉。这个方法并不管用,想扯开嗓子说脏话的冲动比之前更强烈了。

\switchcolumn*


\section{VII}\label{vii}

\switchcolumn

\section*{七}\label{ux4e03}

\switchcolumn*

\emph{If there is hope} [wrote Winston] \emph{it lies in the proles}.

\switchcolumn

\emph{即使希望存在,}温斯顿写道,\emph{它就在群众身上。}

\switchcolumn*

\sectionbreak

\switchcolumn

\sectionbreak

\switchcolumn*

If there was hope, it \emph{must} lie in the proles, because only there,
in those swarming disregarded masses, eighty-five per cent of the
population of Oceania, could the force to destroy the Party ever be
generated. The Party could not be overthrown from within. Its enemies,
if it had any enemies, had no way of coming together or even of
identifying one another. Even if the legendary Brotherhood existed, as
just possibly it might, it was inconceivable that its members could ever
assemble in larger numbers than twos and threes. Rebellion meant a look
in the eyes, an inflection of the voice; at the most, an occasional
whispered word. But the proles, if only they could somehow become
conscious of their own strength, would have no need to conspire. They
needed only to rise up and shake themselves like a horse shaking off
flies. If they chose they could blow the Party to pieces tomorrow
morning. Surely sooner or later it must occur to them to do it. And
yet---!

\switchcolumn

假使有希望,它一定在群众身上。只有在那些被忽视的芸芸众生之间,在那些占大洋国人口85\%的人们身上,才有可能酝酿出将党摧毁的力量。党不可能从内部崩陷,就算其中真有敌人,他们也无法聚集一起相互确认,传说中的兄弟会也是如此,它无力汇聚众人,只能三三两两地碰个头。对他们来说,反抗意味着一个眼神,一个声音中的变化,至多跑出一点传闻。但是群众不同,只要他们发现自己的力量,他们就没必要躲在暗中进行活动,若他们愿意,他们可以像马驱赶苍蝇那样站出来,到第二天早上,党就会被打得支离破碎。他们迟早会这样做的,不是吗?但——

\switchcolumn*

He remembered how once he had been walking down a crowded street when a
tremendous shout of hundreds of voices---women\textquotesingle s
voices---had burst from a side street a little way ahead. It was a great
formidable cry of anger and despair, a deep loud ``Oh-o-o-o-oh!'' that
went humming on like the reverberation of a bell. His heart had leapt.
It\textquotesingle s started! he had thought. A riot! The proles are
breaking loose at last! When he had reached the spot it was to see a mob
of two or three hundred women crowding round the stalls of a street
market, with faces as tragic as though they had been the doomed
passengers on a sinking ship. But at this moment the general despair
broke down into a multitude of individual quarrels. It appeared that one
of the stalls had been selling tin saucepans. They were wretched, flimsy
things, but cooking pots of any kind were always difficult to get. Now
the supply had unexpectedly given out. The successful women, bumped and
jostled by the rest, were trying to make off with their saucepans while
dozens of others clamored round the stall, accusing the stallkeeper of
favoritism and of having more saucepans somewhere in reserve. There was
a fresh outburst of yells. Two bloated women, one of them with her hair
coming down, had got hold of the same saucepan and were trying to tear
it out of one another\textquotesingle s hands. For a moment they were
both tugging, and then the handle came off. Winston watched them
disgustedly. And yet, just for a moment, what almost frightening power
had sounded in that cry from only a few hundred throats! Why was it that
they could never shout like that about anything that mattered?

\switchcolumn

他想起来,有次他正走在一条拥挤的街道上,突然从前方传来数百人的喧闹声——那是女人的声音,绝望而愤怒,铿锵而低沉,如钟声一般久久回荡:``噢——噢——噢——噢!''
他的心剧烈地跳动起来。开始了!他想。暴乱!群众终于摆脱了羁绊!而抵达出事地点时,他看到的却是两三百个女人围在街边商摊的前面。她们神情凄恻,好像沉船上注定无法获救的乘客。之前那一整片的绝望已经碎裂成七零八落的争吵。原来有个卖劣质铁锅的突然停止出货,而不管什么品质的厨房用品总是很难买到。已经买到锅的女人被人群推搡,想快点离开,更多没买到的则围住摊位,指责摊贩看人下菜碟,囤货不卖。接着,又一阵吵闹声响起,两个胖女人争抢一只锅子,一个人的头发披散开来,锅的把手也被弄掉了。看着她们,温斯顿感觉恶心。就在刚才,数百人的喊叫还凝聚成那样可怕的力量,可她们为什么不为真正重要的事情大吼呢?

\switchcolumn*

He wrote:

\switchcolumn

他写道:

\switchcolumn*

\begin{quotation}
Until they become conscious they will never rebel, and until after
they have rebelled they cannot become conscious.
\end{quotation}

\switchcolumn

\begin{quotation}
不到觉醒的时候,便不会起身反抗;不反抗,就不会有觉醒。
\end{quotation}

\switchcolumn*

That, he reflected, might almost have been a transcription from one of
the Party textbooks. The Party claimed, of course, to have liberated the
proles from bondage. Before the Revolution they had been hideously
oppressed by the capitalists, they had been starved and flogged, women
had been forced to work in the coal mines (women still did work in the
coal mines, as a matter of fact), children had been sold into the
factories at the age of six. But simultaneously, true to the principles
of doublethink, the Party taught that the proles were natural inferiors
who must be kept in subjection, like animals, by the application of a
few simple rules. In reality very little was known about the proles. It
was not necessary to know much. So long as they continued to work and
breed, their other activities were without importance. Left to
themselves, like cattle turned loose upon the plains of Argentina, they
had reverted to a style of life that appeared to be natural to them, a
sort of ancestral pattern. They were born, they grew up in the gutters,
they went to work at twelve, they passed through a brief blossoming
period of beauty and sexual desire, they married at twenty, they were
middle-aged at thirty, they died, for the most part, at sixty. Heavy
physical work, the care of home and children, petty quarrels with
neighbors, films, football, beer, and, above all, gambling filled up the
horizon of their minds. To keep them in control was not difficult. A few
agents of the Thought Police moved always among them, spreading false
rumors and marking down and eliminating the few individuals who were
judged capable of becoming dangerous; but no attempt was made to
indoctrinate them with the ideology of the Party. It was not desirable
that the proles should have strong political feelings. All that was
required of them was a primitive patriotism which could be appealed to
whenever it was necessary to make them accept longer working hours or
shorter rations. And even when they became discontented, as they
sometimes did, their discontent led nowhere, because, being without
general ideas, they could only focus it on petty specific grievances.
The larger evils invariably escaped their notice. The great majority of
proles did not even have telescreens in their homes. Even the civil
police interfered with them very little. There was a vast amount of
criminality in London, a whole world-within-a-world of thieves, bandits,
prostitutes, drug peddlers, and racketeers of every description; but
since it all happened among the proles themselves, it was of no
importance. In all questions of morals they were allowed to follow their
ancestral code. The sexual puritanism of the Party was not imposed upon
them. Promiscuity went unpunished; divorce was permitted. For that
matter, even religious worship would have been permitted if the proles
had shown any sign of needing or wanting it. They were beneath
suspicion. As the Party slogan put it: ``Proles and animals are free.''

\switchcolumn

这句话就像从党的教科书上抄下来的一样。当然,党曾声明,是自己将群众从奴役中解放出来的。革命前,他们被资本家摧残,不仅忍饥挨饿,还要挨打受难,女人被迫到煤矿做工(实际上,现在女人仍在煤矿里做工),小孩六岁大就被卖到工厂里。但是,遵循双重思想的原则,党同时又说,群众天生就是下等人,必须让他们处在``服从''的地位上,他们要像动物一样受制于简单的规定。而实际上,人们对群众知之甚少,也没有必要知道太多。在他们看来,群众的活动除了工作和繁衍,其他无关紧要。他们自生自灭,像阿根廷平原上放养着的牛群。他们似乎过着返璞归真的生活,一种从远古时代流传下来的生活。他们降生,在街头长大。十二岁做工,经过短暂美丽有如鲜花绽放般的青春期后,在二十岁结婚,三十岁衰老,他们中的绝大部分在六十岁时死去。繁重的体力劳动,家人与孩子、与邻居的争吵、电影足球啤酒还有赌博,这就是他们的一切。控制他们并不困难。思想警察的特务在他们中间转来转去,一面散布谣言,一面跟踪并消灭那些有可能变成危险分子的家伙。但却从来没有尝试过向他们灌输党的思想。群众不需要有什么强烈的政治意识,党只要求他们拥有单纯的能够随时唤起的爱国之情,这有利于让他们接受更长的工作时间和更少的劳动所得。有时,他们也会不满,但这不满不会造成什么后果。因为他们不具备抽象思考的能力,他们只会对具体的细枝末节感到不满。那些罪大恶极的事情他们往往视而不见。他们中的大多数家里没有电屏,警察也很少去管他们的事。伦敦的犯罪率非常高,就好像一个充斥着小偷、强盗、妓女、毒贩和骗子的世界,但由于罪行都发生在群众身上,这并不重要。党允许他们按照老规矩处理道德问题,也没有将禁欲主义强加给他们。他们可以乱交而无需担心处罚,离婚也相对容易。若有必要,他们甚至被允许信仰宗教。他们不值得怀疑,正如党的口号说的那样:``群众和动物是自由的。''

\switchcolumn*

Winston reached down and cautiously scra\-tched his varicose ulcer. It had
begun itching again. The thing you invariably came back to was the
impossibility of knowing what life before the Revolution had really been
like. He took out of the drawer a copy of a children\textquotesingle s
history textbook which he had borrowed from Mrs. Parsons, and began
copying a passage into the diary:

\switchcolumn

温斯顿把手伸下去,轻轻挠了挠脚踝上的溃疡,那里又痒起来。说到底还是那个问题,你不可能知道革命前的生活究竟是什么样子。他从抽屉里拿出从帕森斯太太那里借来的、写给小孩子看的历史课本,将其中的一段话抄在了日记本上:

\switchcolumn*

\begin{quotation}
  In the old days [it ran], before the glorious Revolution, London was not
  the beautiful city that we know today. It was a dark, dirty, miserable
  place where hardly anybody had enough to eat and where hundreds and
  thousands of poor people had no boots on their feet and not even a roof to
  sleep under. Children no older than you are had to work twelve hours a day
  for cruel masters, who flogged them with whips if they worked too slowly
  and fed them on nothing but stale breadcrusts and water. But in among all
  this terrible poverty there were just a few great big beautiful houses
  that were lived in by rich men who had as many as thirty servants to look
  after them. These rich men were called capitalists. They were fat, ugly
  men with wicked faces, like the one in the picture on the opposite page.
  You can see that he is dressed in a long black coat which was called a
  frock coat, and a queer, shiny hat shaped like a stovepipe, which was
  called a top hat. This was the uniform of the capitalists, and no one else
  was allowed to wear it. The capitalists owned everything in the world, and
  everyone else was their slave. They owned all the land, all the houses,
  all the factories, and all the money. If anyone disobeyed them they could
  throw him into prison, or they could take his job away and starve him to
  death. When any ordinary person spoke to a capitalist he had to cringe and
  bow to him, and take off his cap and address him as ``Sir.'' The chief of
  all the capitalists was called the King, and---
\end{quotation}

\switchcolumn

\begin{quotation}
在伟大的革命开始前,伦敦并不是今天人们看到的这般美丽。那时的它是个黑暗、肮脏、
充满痛苦的地方,很少有人能吃饱肚子,成千上万的穷人没有鞋子,居无定所。那些孩
子年纪还没有你们大,就不得不去为残暴的老板干活,他们一天要工作十二个小时,动
作稍慢就会遭到鞭打,还只能就着白水吃已经不新鲜的面包皮。但就是在这样贫苦的大
环境下,却还能看到一些高大华丽的房子,它们都属于有钱人,这些人被称作资本家,
拥有数十位仆人。他们个个肥胖丑陋,面容凶狠,参见下页插图。他们穿着黑色的礼服
大衣,戴着像烟囱一样的高帽。这就是他们的制服,其他任何人都不允许穿。世界上的
所有东西都是资本家的,所有其他人都是他们的奴隶。他们拥有一切土地、一切房屋、
一切工厂、一切金钱。谁不听他们的话,谁就会被抓入大牢,或者被剥夺工作,活活饿
死。普通人和资本家说话,要鞠躬行礼,摘下帽子,称他们``先生'',资本家的首领被
称作``国王'',而且——
\end{quotation}

\switchcolumn*

But he knew the rest of the catalogue. There would be mention of the
bishops in their lawn sleeves, the judges in their ermine robes, the
pillory, the stocks, the treadmill, the
cat-o\textquotesingle-nine-tails, the Lord Mayor\textquotesingle s
Banquet, and the practice of kissing the Pope\textquotesingle s toe.
There was also something called the \emph{jus primae noctis}, which
would probably not be mentioned in a textbook for children. It was the
law by which every capitalist had the right to sleep with any woman
working in one of his factories.

\switchcolumn


剩下的内容他都清楚。接下来,会提到穿细麻布法衣的主教,提到穿貂皮法袍的法官,提到脚镣手铐、踏车之刑,提到鞭笞、市长的宴会、亲吻教皇脚尖以及拉丁文里的``初夜权''。也许小孩子的课本里不会有这个。所谓``初夜权''就是依照法律规定,资本家有权和其工厂中的女人睡觉。

\switchcolumn*

How could you tell how much of it was lies? It \emph{might} be true that
the average human being was better off now than he had been before the
Revolution. The only evidence to the contrary was the mute protest in
your own bones, the instinctive feeling that the conditions you lived in
were intolerable and that at some other time they must have been
different. It struck him that the truly characteristic thing about
modern life was not its cruelty and insecurity, but simply its bareness,
its dinginess, its listlessness. Life, if you looked about you, bore no
resemblance not only to the lies that streamed out of the telescreens,
but even to the ideals that the Party was trying to achieve. Great areas
of it, even for a Party member, were neutral and nonpolitical, a matter
of slogging through dreary jobs, fighting for a place on the Tube,
darning a worn-out sock, cadging a saccharine tablet, saving a cigarette
end. The ideal set up by the Party was something huge, terrible, and
glittering---a world of steel and concrete, of monstrous machines and
terrifying weapons---a nation of warriors and fanatics, marching forward
in perfect unity, all thinking the same thoughts and shouting the same
slogans, perpetually working, fighting, triumphing, persecuting---three
hundred million people all with the same face. The reality was decaying,
dingy cities, where underfed people shuffled to and fro in leaky shoes,
in patched-up nineteenth-century houses that smelt always of cabbage and
bad lavatories. He seemed to see a vision of London, vast and ruinous,
city of a million dust bins, and mixed up with it was a picture of Mrs.
Parsons, a woman with lined face and wispy hair, fiddling helplessly
with a blocked wastepipe.

\switchcolumn

你要如何判定其中的哪些是谎言呢?现在,人们的平均生活水平比革命前好了,这可能确是事实。唯一相反的证据来自于你骨子里的无声抗议,这是一种本能,你觉得你无法忍受当下的生活,认为之前的情况一定有所不同。他突然觉得,现代生活真正的特点并不在于它的残酷和缺乏安全感,而在于它的空洞、晦暗和了无生趣。看看现实中的生活,它和电屏中没完没了的谎言毫无共同之处,和党宣称的理想目标也毫无共同之处。哪怕对党员来说,生活也是中性的多,政治性的少,比如努力完成日常事务,在地铁里抢占座位,缝补破袜子,向别人要一片糖精,节省一个烟头\ldots\ldots 而党的理想却大而可怕,闪烁着耀眼的光芒:满世界都是钢筋水泥、大型机器和骇人的武器,到处都是勇猛的战士和狂热的信徒。大家团结一心,齐步前进,所有人都思想一致,所有人都喊着相同的口号,所有人都不知休息地工作着,战斗着,不停地打胜仗,不停地迫害他人——三亿人的脸都一模一样。但现实中的世界,却是城市破败,民皆饥色,人们饿着肚子穿着破鞋住着19世纪建造的几经修补的破房子,房子里还满是煮白菜和厕所的臭味。他好像看到了伦敦,那里庞大破败,有上百万垃圾桶,皱纹满面、头发稀疏的帕森斯太太也在其中,她对堵塞的水管束手无策。

\switchcolumn*

He reached down and scratched his ankle again. Day and night the
telescreens bruised your ears with statistics proving that people today
had more food, more clothes, better houses, better recreations---that
they lived longer, worked shorter hours, were bigger, healthier,
stronger, happier, more intelligent, better educated, than the people of
fifty years ago. Not a word of it could ever be proved or disproved. The
Party claimed, for example, that today forty per cent of adult proles
were literate; before the Revolution, it was said, the number had only
been fifteen per cent. The Party claimed that the infant mortality rate
was now only a hundred and sixty per thousand, whereas before the
Revolution it had been three hundred---and so it went on. It was like a
single equation with two unknowns. It might very well be that literally
every word in the history books, even the things that one accepted
without question, was pure fantasy. For all he knew there might never
have been any such law as the \emph{jus primae noctis}, or any such
creature as a capitalist, or any such garment as a top hat.

\switchcolumn

他又伸手去挠他的脚脖子。电屏日以继夜地往人们的耳朵里塞统计数字,以证明今天的人们有更多的食物、更好的衣物、更结实的房子、更美妙的娱乐活动——人们比五十年前更长寿、更高大、更健壮、更快乐、更聪明,劳动的时间更少,受教育的程度更高。而这其中没有一个词能被证明、被推翻。比如,党说今天在成年群众中,识字的占40\%,在革命前这个数字只有15\%。党说,今天婴儿的死亡率只有160‰,在革命前这个数字高达300‰,就好比包含着两个未知数的等式。历史书上的每个数字,包括那些人们确信无疑的事情,都完全有可能出自虚构。就他所知,可能从来没有``初夜权''这样的法律,从来没有资本家那样的人,从来没有高礼帽那样的服饰。

\switchcolumn*

Everything faded into mist. The past was erased, the erasure was
forgotten, the lie became truth. Just once in his life he had
possessed---\emph{after} the event: that was what counted---concrete,
unmistakable evidence of an act of falsification. He had held it between
his fingers for as long as thirty seconds. In 1973, it must have
been---at any rate, it was at about the time when he and Katharine had
parted. But the really relevant date was seven or eight years earlier.

\switchcolumn

一切都隐遁到迷雾里。过去被清除,清除的行为本身也被遗忘,谎言成为真理。他有生以来只有一次掌握了伪造历史的确凿证据,就在那件事发生之后,这点非常重要。证据在他的手指间停留了三十秒钟之久。1973年,一定是在1973年,无论怎样,它发生在他和凯瑟琳分居的时候。但真正的关键时刻,却比这还要早七八年。

\switchcolumn*

The story really began in the middle Sixties, the period of the great
purges in which the original leaders of the Revolution were wiped out
once and for all. By 1970 none of them was left, except Big Brother
himself. All the rest had by that time been exposed as traitors and
counter-revolutionaries. Goldstein had fled and was hiding, no one knew
where, and of the others, a few had simply disappeared, while the
majority had been executed after spectacular public trials at which they
made confession of their crimes. Among the last survivors were three men
named Jones, Aaronson, and Rutherford. It must have been in 1965 that
these three had been arrested. As often happened, they had vanished for
a year or more, so that one did not know whether they were alive or
dead, and then had suddenly been brought forth to incriminate themselves
in the usual way. They had confessed to intelligence with the enemy (at
that date, too, the enemy was Eurasia), embezzlement of public funds,
the murder of various trusted Party members, intrigues against the
leadership of Big Brother which had started long before the Revolution
happened, and acts of sabotage causing the death of hundreds of
thousands of people. After confessing to these things they had been
pardoned, reinstated in the Party, and given posts which were in fact
sinecures but which sounded important. All three had written long,
abject articles in the \emph{Times}, analyzing the reasons for their
defection and promising to make amends.

\switchcolumn

这件事要从六十年代中期说起,即彻底消灭革命元老的大清洗时期。截至1970年,除老大哥外,其他人都被清除掉了。他们都被当做叛徒、反革命被揭发。高德斯坦因逃走了,藏了起来,没人知道他跑到了哪里;剩下的有一小部分失踪了,大部分在规模宏大的公审大会上坦陈罪行,遭到处决。最后活下来的人中有三个人值得一提,他们是琼斯、阿朗森和鲁瑟夫。他们在1965年被捕,像往常一样消失了一两年,生死未卜,之后又被突然带了出来,并按照惯例供认罪行。他们承认通敌(当时的敌人就是欧亚国),承认盗用公款,承认密谋杀害党的领导并在革命发生前就反对老大哥的领导。他们还称自己导演的破坏活动造成了成百上千人的死亡。不过在坦白了这些罪行后,他们得到了宽大处理,不仅恢复了党籍,还收获了听上去很重要实际很悠闲的职位。他们都在《泰晤士报》上写了长长的检讨书,检讨自己堕落的原因并保证改过自新。

\switchcolumn*

Some time after their release Winston had actually seen all three of
them in the Chestnut Tree Café. He remembered the sort of terrified
fascination with which he had watched them out of the corner of his eye.
They were men far older than himself, relics of the ancient world,
almost the last great figures left over from the heroic early days of
the Party. The glamor of the underground struggle and the civil war
still faintly clung to them. He had the feeling, though already at that
time facts and dates were growing blurry, that he had known their names
years earlier than he had known that of Big Brother. But also they were
out-Plaws, enemies, untouchables, doomed with absolute certainty to
extinction within a year or two. No one who had once fallen into the
hands of the Thought Police ever escaped in the end. They were corpses
waiting to be sent back to the grave.

\switchcolumn

温斯顿曾在栗树咖啡馆见到过他们三个,那是在他们获释后不久。他记得他观察他们,又吃惊又害怕。他们比他年长得多,来自旧世界,是党建立之初那段英雄岁月留下的最后几个大人物。他们身上隐隐地散发着地下斗争和内战时期的风采。尽管事实和年代已经面目模糊,他仍有一种感觉,在知道老大哥前,他就知道他们了。他们是罪犯,是敌人,是不能接触的人,同时可以肯定不出一两年他们就会死掉。落在思想警察手里的,没有一个能逃过这样的结局。他们只是等着进坟墓的行尸走肉罢了。

\switchcolumn*

There was no one at any of the tables nearest to them. It was not wise
even to be seen in the neighborhood of such people. They were sitting in
silence before glasses of the gin flavored with cloves which was the
speciality of the café. Of the three, it was Rutherford whose appearance
had most impressed Winston. Rutherford had once been a famous
caricaturist, whose brutal cartoons had helped to inflame popular
opinion before and during the Revolution. Even now, at long intervals,
his cartoons were appearing in the \emph{Times}. They were simply an
imitation of his earlier manner, and curiously lifeless and
unconvincing. Always they were a rehashing of the ancient themes---slum
tenements, starving children, street battles, capitalists in top
hats---even on the barricades the capitalists still seemed to cling to
their top hats---an endless, hopeless effort to get back into the past.
He was a monstrous man, with a mane of greasy gray hair, his face
pouched and seamed, with protuberant lips. At one time he must have been
immensely strong; now his great body was sagging, sloping, bulging,
falling away in every direction. He seemed to be breaking up before
one\textquotesingle s eyes, like a mountain crumbling.

\switchcolumn

他们周围的位子是空的,靠近他们很不明智。他们安静地坐在桌子旁,对着店里特制的丁香味杜松子酒。三人之中,鲁瑟夫给温斯顿的印象最深。他曾是远近闻名的漫画家,他的讽刺漫画在革命前和革命中都给了人们很大鼓舞,即便在今天,每隔一段时间,《泰晤士报》上就会刊登他的漫画,不过那些漫画只是他早期风格的模仿,既缺乏活力又没有说服力,还总是同一个老主题:贫民窟、饥饿的孩子、巷战、戴着高帽的资本家——甚至在巷战里,资本家也戴着高高的礼帽——他徒劳地尝试重现往日风采。他灰色的头发泛着油光,脸皮松弛,布满伤疤,嘴唇则厚得像黑人。从他高大的身材可以看出,他曾经一定健壮无比,但现在垮下来,发起胀,似乎会向各个方向散开。他就像一座即将倒塌的大山,就要在你的眼前破碎。

\switchcolumn*

It was the lonely hour of fifteen. Winston could not now remember how he
had come to be in the café at such a time. The place was almost empty. A
tinny music was trickling from the telescreens. The three men sat in
their corner almost motionless, never speaking. Uncommanded, the waiter
brought fresh glasses of gin. There was a chessboard on the table beside
them, with the pieces set out, but no game started. And then, for
perhaps half a minute in all, something happened to the telescreens. The
tune that they were playing changed, and the tone of the music changed
too. There came into it---but it was something hard to describe. It was
a peculiar, cracked, braying, jeering note; in his mind Winston called
it a yellow note. And then a voice from the telescreen was singing:

\switchcolumn

当时正值下午三点,人不多。温斯顿已经忘记自己去咖啡馆的理由。咖啡馆里几乎没有客人,电屏上播着轻柔的音乐。那三个人一动不动地坐在角落里,不发一言。服务生主动拿来杜松子酒。就在他们隔壁的桌子上有个棋盘,棋子已经码好,他们也没有去玩。大约一分半钟后,电屏里的音乐突然改变,变成难以形容的、刺耳响亮的歌曲,那声音带着几分嘲讽,温斯顿在心底称之为``黄色警报'',电屏里唱道:

\switchcolumn*

\begin{quotation}
  \noindent ``Under the spreading chestnut tree\\
  I sold you and you sold me:\\
  There lie they, and here lie we\\
  Under the spreading chestnut tree.''
\end{quotation}

\switchcolumn

\begin{quotation}
\noindent 在栗子树的绿茵下,\\
我出卖了你,你出卖了我;\\
他们躺在那儿,我们躺在这儿,\\
就在栗子树的绿荫下。
\end{quotation}

\switchcolumn*

The three men never stirred. But when Winston glanced again at
Rutherford\textquotesingle s ruinous face, he saw that his eyes were
full of tears. And for the first time he noticed, with a kind of inward
shudder, and yet not knowing \emph{at what} he shuddered, that both
Aaronson and Rutherford had broken noses.

\switchcolumn

三个人听了,依然一动不动。温斯顿看了看鲁瑟夫那张已经毁掉的脸,发现他的眼睛里溢满泪水。他第一次注意到阿朗森和鲁瑟夫的鼻梁都断了,这让他胆战心惊,虽然他不知道自己为什么要胆战心惊。

\switchcolumn*

A little later all three were rearrested. It appeared that they had
engaged in fresh conspiracies from the very moment of their release. At
their second trial they confessed to all their old crimes over again,
with a whole string of new ones. They were executed, and their fate was
recorded in the Party histories, a warning to posterity. About five
years after this, in 1973, Winston was unrolling a wad of documents
which had just flopped out of the pneumatic tube onto his desk when he
came on a fragment of paper which had evidently been slipped in among
the others and then forgotten. The instant he had flattened it out he
saw its significance. It was a half-page torn out of the \emph{Times} of
about ten years earlier---the top half of the page, so that it included
the date---and it contained a photograph of the delegates at some Party
function in New York. Prominent in the middle of the group were Jones,
Aaronson, and Rutherford. There was no mistaking them; in any case their
names were in the caption at the bottom.

\switchcolumn

这之后不久,三人又被抓走,说他们在被释放后依然从事阴谋活动。而在第二次受审时,他们不只重新交代了之前所有旧的罪行,还供认了一些新的罪行。这次,他们遭到了处决,他们的结局被写入党史以儆效尤。差不多五年后,即1973年,温斯顿从桌上那一大摞由气力输送管递出的文件中发现了一小片报纸。那显然是不小心夹在中间,忘掉拿走的。他将它打开,登时就意识到它意义重大。那是从十年前的某期《泰晤士报》上撕下来的,是报纸的上半页,注着日期。而在这片报纸上有一张在纽约举行的党大会的照片,照片上位于中间的正是琼斯、阿朗森和鲁瑟夫。一点没错,照片下的说明里还有他们的名字。

\switchcolumn*

The point was that at both trials all three men had confessed that on
that date they had been on Eurasian soil. They had flown from a secret
airfield in Canada to a rendezvous somewhere in Siberia, and had
conferred with members of the Eurasian General Staff, to whom they had
betrayed important military secrets. The date had stuck in
Winston\textquotesingle s memory because it chanced to be Midsummer Day;
but the whole story must be on record in countless other places as well.
There was only one possible conclusion: the confessions were lies.

\switchcolumn

问题是,在两次的审判中,三人招认,那一天他们全部都在欧亚国,正要从加拿大的某个秘密机场往西伯利亚的一处接头地点,与欧亚国总参谋部的人进行会面,将非常重要的军事机密透露给他们。温斯顿对那个日期印象极深,因为那天刚好是仲夏日,但是,在别的什么地方一定也有关于这件事的记录。所以只有一个可能:那些供词是谎言。

\switchcolumn*

Of course, this was not in itself a discovery. Even at that time Winston
had not imagined that the people who were wiped out in the purges had
actually committed the crimes that they were accused of. But this was
concrete evidence; it was a fragment of the abolished past, like a
fossil bone which turns up in the wrong stratum and destroys a
geological theory. It was enough to blow the Party to atoms, if in some
way it could have been published to the world and its significance made
known.

\switchcolumn

当然,这样的事情并不新鲜,就算在当时,温斯顿也不认为那些在大清洗中被清除掉的人真的犯了指控的罪行。但这张报纸却是确凿的证据,是被抹杀了的过去的一个碎片。就好像一枚骨化石突然出现在它不该出现的地层里,致使地质学的某个理论被推翻。假使有办法将它公诸于世,让所有人都知道它的意义,就能让党灰飞烟灭。

\switchcolumn*

He had gone straight on working. As soon as he saw what the photograph
was, and what it meant, he had covered it up with another sheet of
paper. Luckily, when he unrolled it, it had been upside-down from the
point of view of the telescreen.

\switchcolumn

他继续工作,看到这照片并明白它的意义后,他马上用一张纸将它盖住,所幸在打开它时,从电屏里看,它是上下颠倒的。

\switchcolumn*

He took his scribbling pad on his knee and pushed back his chair, so as
to get as far away from the telescreen as possible. To keep your face
expressionless was not difficult, and even your breathing could be
controlled, with an effort; but you could not control the beating of
your heart, and the telescreen was quite delicate enough to pick it up.
He let what he judged to be ten minutes go by, tormented all the while
by the fear that some accident---a sudden draught blowing across his
desk, for instance---would betray him. Then, without uncovering it
again, he dropped the photograph into the memory hole, along with some
other waste papers. Within another minute, perhaps, it would have
crumbled into ashes.

\switchcolumn

他将草稿本放在膝盖上,又把椅子往后拖了拖,尽可能地躲开电屏。控制好面部表情并不困难,只要花点工夫,连呼吸都能控制住,但你控制不了心跳的速度,而电屏却灵敏到连这也能接收到。大约有十分钟,他一直担心会有突发事件暴露自己。比如,突然吹过桌面的一阵风。之后,他没有再打开那张纸,而是将它和其他废纸一起丢到了记忆洞里。也许再过一分钟,它们就会被烧为灰烬。

\switchcolumn*

That was ten---eleven years ago. Today, probably, he would have kept
that photograph. It was curious that the fact of having held it in his
fingers seemed to him to make a difference even now, when the photograph
itself, as well as the event it recorded, was only memory. Was the
Party\textquotesingle s hold upon the past less strong, he wondered,
because a piece of evidence which existed no longer \emph{had once}
existed?

\switchcolumn

这是十年,不,是十一年前的事了。若是在今天,他很可能会将那张照片留下来。奇怪的是,尽管现在这张照片和其他记录在案的事件都只存在于记忆中,他仍然觉得,拿过照片一事意义非浅。他想,这已然不在的证据毕竟曾存在过,党对过去的控制还是那么强吗?

\switchcolumn*

But today, supposing that it could be somehow resurrected from its
ashes, the photograph might not even be evidence. Already, at the time
when he made his discovery, Oceania was no longer at war with Eurasia,
and it must have been to the agents of Eastasia that the three dead men
had betrayed their country. Since then there had been other
changes---two, three, he could not remember how many. Very likely the
confessions had been rewritten and rewritten until the original facts
and dates no longer had the smallest significance. The past not only
changed, but changed continuously. What most afflicted him with the
sense of nightmare was that he had never clearly understood \emph{why}
the huge imposture was undertaken. The immediate advantages of
falsifying the past were obvious, but the ultimate motive was
mysterious. He took up his pen again and wrote:

\switchcolumn

然而,即使今天这照片从灰烬中复原,它也不可能成为证据。他发现照片时,大洋国已不再和欧亚国交战,而这三个死去的人都是东亚国的特务,都背叛了自己的祖国。自那之后,交战的对象已几经变化,究竟变化了两次还是三次,他也记不清了。供词很可能被反复重写,以至于原有的日期和事实真相都失去了意义。过去不仅被篡改,还被不断篡改。他从未仔细想过为什么要如此大张旗鼓地伪造过去,这个问题犹如可怕的噩梦让他饱受折磨。伪造的好处显而易见,但它深层的动机却让人困惑不解。他再次拿起笔,写道:

\switchcolumn*

\begin{quotation}
I understand HOW: I do not understand WHY.
\end{quotation}

\switchcolumn

``我明白怎么做,我不明白为什么。''

\switchcolumn*

He wondered, as he had many times wondered before, whether he himself
was a lunatic. Perhaps a lunatic was simply a minority of one. At one
time it had been a sign of madness to believe that the earth goes round
the sun; today, to believe that the past is unalterable. He might be
\emph{alone} in holding that belief, and if alone, then a lunatic. But
the thought of being a lunatic did not greatly trouble him; the horror
was that he might also be wrong.

\switchcolumn

很多次,他都怀疑自己是不是疯子。也许疯子就是指少数派。曾经人们认为相信地球绕着太阳转是发疯的征兆。但在今天,相信过去不会被篡改也是发疯的征兆。他可能是唯一一个会这样想的人。若真如此,那他就是个疯子。不过,想到这里他并不害怕,让他害怕的是,他的想法可能是错的。

\switchcolumn*

He picked up the children\textquotesingle s history book and looked at
the portrait of Big Brother which formed its frontispiece. The hypnotic
eyes gazed into his own. It was as though some huge force were pressing
down upon you---something that penetrated inside your skull, battering
against your brain, frightening you out of your beliefs, persuading you,
almost, to deny the evidence of your senses. In the end the Party would
announce that two and two made five, and you would have to believe it.
It was inevitable that they should make that claim sooner or later: the
logic of their position demanded it. Not merely the validity of
experience, but the very existence of external reality was tacitly
denied by their philosophy. The heresy of heresies was common sense. And
what was terrifying was not that they would kill you for thinking
otherwise, but that they might be right. For, after all, how do we know
that two and two make four? Or that the force of gravity works? Or that
the past is unchangeable? If both the past and the external world exist
only in the mind, and if the mind itself is controllable---what then?

\switchcolumn

他拣起那本写给小孩子的历史课本,看了看印在卷首的老大哥肖像。老大哥正用那双能将人催眠的眼睛看着他。一股巨大的力量压迫着你,好像有什么东西正刺穿你的头颅,直抵你的大脑,它恐吓你让你放弃信念,差不多要说服你否认你的感觉。最后党说二加二等于五,你也不得不去相信,而他们无可避免地会这样做:他们的地位要求他们这样做。他们的哲学不仅会否认经验的有效性,还会否认现实的存在性。常识是异端中的异端。令人恐惧的不是你不想其所想便杀了你,而是他们有可能是对的。毕竟,我们要如何知道二加二等于四?我们又怎么知道地心引力的存在?怎么知道过去不可改变?若过去和客观世界都只存在于我们的意识里,意识又能被控制——那要如何是好?

\switchcolumn*

But no! His courage seemed suddenly to stiffen of its own accord. The
face of O\textquotesingle Brien, not called up by any obvious
association, had floated into his mind. He knew, with more certainty
than before, that O\textquotesingle Brien was on his side. He was
writing the diary for O\textquotesingle Brien---\emph{to}
O\textquotesingle Brien; it was like an interminable letter which no one
would ever read, but which was addressed to a particular person and took
its color from that fact.

\switchcolumn

但是,不!他的勇气突然迸发出来,没有特意去想,脑海中就浮现出奥布兰的脸。他比从前更加明确地知道,奥布兰就站在他的一边。对,他在为奥布兰写日记,就好像在写一封没有结尾,没人会读,却有特定接收对象的信。而他的文笔也因此变得生动。

\switchcolumn*

The Party told you to reject the evidence of your eyes and ears. It was
their final, most essential command. His heart sank as he thought of the
enormous power arrayed against him, the ease with which any Party
intellectual would overthrow him in debate, the subtle arguments which
he would not be able to understand, much less answer. And yet he was in
the right! They were wrong and he was right. The obvious, the silly, and
the true had got to be defended. Truisms are true, hold on to that! The
solid world exists, its laws do not change. Stones are hard, water is
wet, objects unsupported fall toward the earth\textquotesingle s center.
With the feeling that he was speaking to O\textquotesingle Brien, and
also that he was setting forth an important axiom, he wrote:

\switchcolumn

党告诉你不要相信你的所见所闻,这是他们最终也是最本质的命令。他所面对的力量是何等庞大,党随便一个知识分子都可以轻轻松松将他驳倒,不要说去反驳那些狡猾的论点,光是理解它们,他就无法做到。一想到此,他的心就不禁一沉。但是,他是正确的,他必须去捍卫那些明显的、朴素的、真实的东西!客观世界是存在的,它的规律无可更改。石头硬,水湿,没有支撑的物体会向地心坠落。他有一种感觉,他正在向奥布兰说话,正在阐述一个至关重要的道理,因此,他写道:

\switchcolumn*

\begin{quotation}
Freedom is the freedom to say that two plus two make four. If that
is granted, all else follows.
\end{quotation}

\switchcolumn

\begin{quotation}
  ``自由就是可以说二加二等于四的自由。若它成立,其他一切皆是如此。''
\end{quotation}

\switchcolumn*


\section{VIII}\label{viii}

\switchcolumn

\section*{八}\label{ux516b}

\switchcolumn*

From somewhere at the bottom of a passage the smell of roasting
coffee---real coffee, not Victory Coffee---came floating out into the
street. Winston paused involuntarily. For perhaps two seconds he was
back in the half-forgotten world of his childhood. Then a door banged,
seeming to cut off the smell as abruptly as though it had been a sound.

\switchcolumn

一股咖啡的浓香从过道尽头飘来,直飘到大街上,那是真正的咖啡,不是胜利咖啡。温斯顿不禁停了下来,大概有两秒钟,他仿佛再次回到了几乎被他遗忘的童年时代。那香味就像声音,门砰地响了一声,将它生生截断。

\switchcolumn*

He had walked several kilometers over pavements, and his varicose ulcer
was throbbing. This was the second time in three weeks that he had
missed an evening at the Community Center: a rash act, since you could
be certain that the number of your attendances at the Center were
carefully checked. In principle a Party member had no spare time, and
was never alone except in bed. It was assumed that when he was not
working, eating, or sleeping he would be taking part in some kind of
communal recreations; to do anything that suggested a taste for
solitude, even to go for a walk by yourself, was always slightly
dangerous. There was a word for it in Newspeak: \emph{ownlife}, it was
called, meaning individualism and eccentricity. But this evening as he
came out of the Ministry the balminess of the April air had tempted him.
The sky was a warmer blue than he had seen it that year, and suddenly
the long, noisy evening at the Center, the boring, exhausting games, the
lectures, the creaking camaraderie oiled by gin, had seemed intolerable.
On impulse he had turned away from the bus stop and wandered off into
the labyrinth of London, first south, then east, then north again,
losing himself along unknown streets and hardly bothering in which
direction he was going.

\switchcolumn

他已经沿着街道走了好几公里,脚踝上静脉曲张导致的溃疡又痒了起来。这是他三个星期以来第二次没到集体活动中心去,考虑到去集体中心的次数会被仔细记录,这无疑是莽撞之举。从原则上说,党员没有空闲时间,除了上床睡觉,他永远不会独自一人。在工作、吃饭和睡觉时间之外,他必须参加某项集体活动。独自行动非常危险,包括独自散步。对此,新话还有个专门的名词``个人生活'',它意味着个人主义和孤僻怪异。但是今晚,他一走出真理部的大门就被四月那裹挟着香气的风吸引。天空如此湛蓝,今年头一回他感受到一丝暖意。他突然觉得中心里的夜晚是那样冗长喧闹,劳心耗力的游戏、令人厌烦的讲话,还有靠杜松子酒维系的同志关系,这一切通通都让人难以忍受。冲动之下,他从公共汽车站走开,步行穿过伦敦那迷宫一般的街街巷巷,先向南,再向东,最后又往北,最终迷失在不知名的街道上,随心所欲地走着。

\switchcolumn*

``If there is hope,'' he had written in the diary, ``it lies in the
proles.'' The words kept coming back to him, statement of a mystical
truth and a palpable absurdity. He was somewhere in the vague,
brown-colored slums to the north and east of what had once been Saint
Pancras Station. He was walking up a cobbled street of little two-story
houses with battered doorways which gave straight on the pavement and
which were somehow curiously suggestive of rat holes. There were puddles
of filthy water here and there among the cobbles. In and out of the dark
doorways, and down narrow alleyways that branched off on either side,
people swarmed in astonishing numbers---girls in full bloom, with
crudely lipsticked mouths, and youths who chased the girls, and swollen
waddling women who showed you what the girls would be like in ten years
time, and old bent creatures shuffling along on splayed feet, and ragged
barefooted children who played in the puddles and then scattered at
angry yells from their mothers. Perhaps a quarter of the windows in the
street were broken and boarded up. Most of the people paid no attention
to Winston; a few eyed him with a sort of guarded curiosity. Two
monstrous women with brick-red forearms folded across their aprons were
talking outside a doorway. Winston caught scraps of conversation as he
approached.

\switchcolumn

他曾在日记中写:``假使希望存在,它就在群众身上。''这句话不断出现在他的脑际,
昭示着神秘而荒谬的真理。他来到位于原圣潘克拉斯车站东北处的褐色贫民窟,在一条
用鹅卵石铺就的小路上行走。路的两旁是矮小的双层楼房,它们破破烂烂的大门就立在
人行道的旁边,很奇怪地让人联想起耗子洞。到处都是肮脏的积水,许许多多的人在黑
洞洞的大门里进进出出,在狭窄的小巷里来来往往。女孩们宛若盛开的鲜花,涂着俗艳
的口红,被男孩们追逐。身材臃肿的女人走起路来摇摇晃晃,让你看到女孩们十年后的
样子。还有迈着八字步、身形佝偻、慢慢移动的老人以及打着赤脚、穿着破烂、在污水
坑里玩耍的孩子——他们一听到妈妈的呵斥就四散奔逃。屋子上的玻璃窗差不多有四分
之一都被打破,用木板钉起。大部分人都没有注意到温斯顿,只有几个人好奇又小心地
看着他。两个彪悍的女人叉着砖红色的手臂在门口闲聊,走近她们时,温斯顿听到了她
们的谈话。

\switchcolumn*

```Yes,' I says to \textquotesingle er, `that\textquotesingle s all very
well,' I says. `But if you\textquotesingle d of been in my place
you\textquotesingle d of done the same as what I done. It\textquotesingle s
easy to criticize,' I says, `but you ain\textquotesingle t got the same
problems as what I got.'{}''

\switchcolumn

``{}`是',她说,`这样很好'。`不过,如果你是我,你也会和我一样。评判别人总是很容易,'`但我的麻烦你可没遇到。'{}''

\switchcolumn*

``Ah,'' said the other, ``that\textquotesingle s jest it.
That\textquotesingle s jest where it is.''

\switchcolumn

``啊,''另一个女人说,``就是这样,问题就在这儿。''

\switchcolumn*

The strident voices stopped abruptly. The women studied him in hostile
silence as he went past. But it was not hostility, exactly; merely a
kind of wariness, a momentary stiffening, as at the passing of some
unfamiliar animal. The blue overalls of the Party could not be a common
sight in a street like this. Indeed, it was unwise to be seen in such
places, unless you had definite business there. The patrols might stop
you if you happened to run into them. ``May I see your papers, comrade?
What are you doing here? What time did you leave work? Is this your
usual way home?''---and so on and so forth. Not that there was any rule
against walking home by an unusual route, but it was enough to draw
attention to you if the Thought Police heard about it.

\switchcolumn

两个尖利的声音突然安静下来,两个女人都用敌意的目光打量着正向她们走来的温斯顿。但准确地说那不是敌意,而是警觉,就好像人看到陌生的动物经过,产生暂时的紧张。这条街很少能看到党员的蓝色制服。对温斯顿来说,在这里被人看到并不明智,除非公务在身。若碰上巡逻队,一定会被他们拦住:``能出示下您的证件吗,同志?您在这里做什么?您什么时候下班的?这是您常走的回家的路吗?''诸如此类。并没有什么规定禁止人走其他的路回家,但如果被思想警察知道,就会引起他们的注意。

\switchcolumn*

Suddenly the whole street was in commotion. There were yells of warning
from all sides. People were shooting into the doorways like rabbits. A
young woman leapt out of a doorway a little ahead of Winston, grabbed up
a tiny child playing in a puddle, whipped her apron round it, and leapt
back again, all in one movement. At the same instant a man in a
concertina-like black suit, who had emerged from a side alley, ran
toward Winston, pointing excitedly to the sky.

\switchcolumn

突然整条街都骚动起来,警报声此起彼伏。人们像兔子一样钻进了门,一个年轻的女人从门洞里蹿出来,一把拽过在水坑里玩耍的孩子,用围裙围住,又迅速地蹿了回去,所有动作一气呵成。同时,街道的一边突然出现了一个穿得像手风琴一样的黑衣男子,男子一边向温斯顿跑来,一边紧张兮兮地指着天空。

\switchcolumn*

``Steamer!'' he yelled. ``Look out, guv\textquotesingle nor! Bang
over\textquotesingle ead! Lay down quick!''

\switchcolumn

``汽船!''他大喊:``小心,先生!上面有炸弹,快趴下!''

\switchcolumn*

``Steamer'' was a nickname which, for some reason, the proles applied to
rocket bombs. Winston promptly flung himself on his face. The proles
were nearly always right when they gave you a warning of this kind. They
seemed to possess some kind of instinct which told them several seconds
in advance when a rocket was coming, although the rockets supposedly
traveled faster than sound. Winston clasped his forearms about his head.
There was a roar that seemed to make the pavement heave; a shower of
light objects pattered onto his back. When he stood up he found that he
was covered with fragments of glass from the nearest window.

\switchcolumn

群众不知为什么将火箭弹称作``汽船''。温斯顿立即扑倒。群众给你的警报通常是准确的。他们似乎有种直觉,虽然据说火箭弹的飞行速度已赶超音速,他们还是能在火箭弹袭击的数秒之前感应到它。温斯顿用双手护住头。就在这时,轰隆一声巨响,似乎要将整个街道掀起来,有什么东西雨点般地落到了他的背上,他爬起来一看,原来附近的一扇窗户震碎了,飞溅出玻璃碴儿。

\switchcolumn*

He walked on. The bomb had demolished a group of houses two hundred
meters up the street. A black plume of smoke hung in the sky, and below
it a cloud of plaster dust in which a crowd was already forming round
the ruins. There was a little pile of plaster lying on the pavement
ahead of him, and in the middle of it he could see a bright red streak.
When he got up to it he saw that it was a human hand severed at the
wrist. Apart from the bloody stump, the hand was so completely whitened
as to resemble a plaster cast.

\switchcolumn

他继续走,炸弹将前方两百米处的房屋炸得粉碎,一股浓黑的烟柱高悬天空,而在它下面灰尘形成的云雾腾空而起,人们纷纷涌向废墟。温斯顿身前的街道上堆着一小摊灰泥,一个鲜红色的东西落在那里。走近一看,原来是一只齐腕炸断的手,除了靠近手腕的地方血肉模糊,这只手惨白得好像石膏制品。

\switchcolumn*

He kicked the thing into the gutter, and then, to avoid the crowd,
turned down a side street to the right. Within three or four minutes he
was out of the area which the bomb had affected, and the sordid swarming
life of the streets was going on as though nothing had happened. It was
nearly twenty hours, and the drinking shops which the proles frequented
(``pubs,'' they called them) were choked with customers. From their grimy
swing doors, endlessly opening and shutting, there came forth a smell of
urine, sawdust, and sour beer. In an angle formed by a projecting house
front three men were standing very close together, the middle one of
them holding a folded-up newspaper which the other two were studying
over his shoulders. Even before he was near enough to make out the
expression on their faces, Winston could see absorption in every line of
their bodies. It was obviously some serious piece of news that they were
reading. He was a few paces away from them when suddenly the group broke
up and two of the men were in violent altercation. For a moment they
seemed almost on the point of blows.

\switchcolumn

他将它踢进水沟,然后避开人群,拐入右边的小巷。三四分钟后,他已走出被轰炸的区域。肮脏的大街上人来人往,好像什么事都没发生过。快20点了,群众云集的小酒吧(他们称其为``酒馆''pubs)里客人盈门。黑乎乎的弹簧门不断地打开、关上,骚臭味和陈年啤酒以及碎木屑的气味混在一起从门内飘出来。一幢房子凸出来,形成了一个角落,角落里有三个人,站得很紧。中间的那人拿着折好的报纸,其他两人站在边上看。无须看清他们的表情,从他们的姿势上就能看出他们有多么专注。他们显然正在阅读一则重要新闻。就在温斯顿离他们只有几步远的时候,三个人突然分散开来,其中两人爆发了激烈地争执,有那么一会儿,他们几乎气炸了。

\switchcolumn*

``Can\textquotesingle t you bleeding well listen to what I say? I tell
you no number ending in seven ain\textquotesingle t won for over
fourteen months!''

\switchcolumn

``你他妈的就不能好好听我说话吗?我告诉你,过去十四个月都没有末尾是7的数字赢过!''

\switchcolumn*

``Yes it \textquotesingle as, then!''

\switchcolumn

``不,它赢过!''

\switchcolumn*

``No, it \textquotesingle as not! Back \textquotesingle ome I got the
\textquotesingle ole lot of \textquotesingle em for over two years wrote
down on a piece of paper. I takes \textquotesingle em down
reg\textquotesingle lar as the clock. An\textquotesingle{} I tell you,
no number ending in seven---''

\switchcolumn

``没赢过,从来没赢过!过去两年所有中奖号码我都记纸上了,就在我家放着呢,和表一样准。没有末尾是7的赢过——''

\switchcolumn*

``Yes, a seven \emph{\textquotesingle as} won! I could pretty near tell
you the bleeding number. Four oh seven, it ended in. It were in
February---second week in February.''

\switchcolumn

``不,7当然赢过!我差不多能说出来究竟是他妈的哪个数字。末尾不是4就是7,是在二月——二月的第二个星期。''

\switchcolumn*

``February your grandmother! I got it all down in black and white.
An\textquotesingle{} I tell you, no number---''

\switchcolumn

``二月你奶奶!我白纸黑字记下来的,我告诉你,没有一个号码——''

\switchcolumn*

``Oh, pack it in!'' said the third man.

\switchcolumn

``嚄!住嘴!''第三个人说。

\switchcolumn*

They were talking about the Lottery. Winston looked back when he had
gone thirty meters. They were still arguing, with vivid, passionate
faces. The Lottery, with its weekly pay-out of enormous prizes, was the
one public event to which the proles paid serious attention. It was
probable that there were some millions of proles for whom the Lottery
was the principal if not the only reason for remaining alive. It was
their delight, their folly, their anodyne, their intellectual stimulant.
Where the Lottery was concerned, even people who could barely read and
write seemed capable of intricate calculations and staggering feats of
memory. There was a whole tribe of men who made a living simply by
selling systems, forecasts, and lucky amulets. Winston had nothing to do
with the running of the Lottery, which was managed by the Ministry of
Plenty, but he was aware (indeed everyone in the Party was aware) that
the prizes were largely imaginary. Only small sums were actually paid
out, the winners of the big prizes being nonexistent persons. In the
absence of any real intercommunication between one part of Oceania and
another, this was not difficult to arrange.

\switchcolumn

他们在谈论彩票。温斯顿走了三十米,又回头看了一眼,发现他们还在争吵。群众唯一关注的大事便是每周一次的大抽奖,奖金丰厚。对他们来说,彩票是支撑他们活下去的理由,是他们的乐趣所在、愚蠢所在,是刺激他们大脑的兴奋剂。看到彩票,原本连读写都不大顺利的人也开始进行复杂的运算,甚至连记性都变好了。还有些人干脆靠预测中奖号码、卖中奖秘籍和吉祥物为生。温斯顿和彩票运作无关,那是富部的事情,但他知道(实际上每个党员都知道),奖金在很大程度上是虚构出来的,中大奖的都是不存在的人,只有小奖才会发到中奖者手里。在大洋国地区间的信息交流十分不畅,安排这样的事并不困难。

\switchcolumn*

But if there was hope, it lay in the proles. You had to cling on to
that. When you put it in words it sounded reasonable; it was when you
looked at the human beings passing you on the pavement that it became an
act of faith. The street into which he had turned ran downhill. He had a
feeling that he had been in this neighborhood before, and that there was
a main thoroughfare not far away. From somewhere ahead there came a din
of shouting voices. The street took a sharp turn and then ended in a
flight of steps which led down into a sunken alley where a few
stallkeepers were selling tired-looking vegetables. At this moment
Winston remembered where he was. The alley led out into the main street,
and down the next turning, not five minutes away, was the junk shop
where he had bought the blank book which was now his diary. And in a
small stationer\textquotesingle s shop not far away he had bought his
penholder and his bottle of ink.

\switchcolumn

然而,你必须坚信,假使有希望,它就在群众身上。将这句话记录下来,它合情合理。观察一下街上那些和你擦肩而过的人,它就会变成一种信仰。转弯后,他走上下坡的路,觉得自己之前曾来过这里,往前不久便是主干道,前方传来嘈杂的人声。这条街突然转了方向,走到了头,台阶的尽头是一条低陷的小巷。巷子里有几个小商贩,正在卖打了蔫的蔬菜。温斯顿突然意识到他身在何处,再转一个弯,不用五分钟,就是他买日记本的那家旧货店。而这家店铺旁,还有个小文具店,当初他就是在那里买的笔杆和墨水。

\switchcolumn*

He paused for a moment at the top of the steps. On the opposite side of
the alley there was a dingy little pub whose windows appeared to be
frosted over but in reality were merely coated with dust. A very old
man, bent but active, with white mustaches that bristled forward like
those of a prawn, pushed open the swing door and went in. As Winston
stood watching it occurred to him that the old man, who must be eighty
at the least, had already been middle-aged when the Revolution happened.
He and a few others like him were the last links that now existed with
the vanished world of capitalism. In the Party itself there were not
many people left whose ideas had been formed before the Revolution. The
older generation had mostly been wiped out in the great purges of the
Fifties and Sixties, and the few who survived had long ago been
terrified into complete intellectual surrender. If there was anyone
still alive who could give you a truthful account of conditions in the
early part of the century, it could only be a prole. Suddenly the
passage from the history book that he had copied into his diary came
back into Winston\textquotesingle s mind, and a lunatic impulse took
hold of him. He would go into the pub, he would scrape acquaintance with
that old man and question him. He would say to him: ``Tell me about your
life when you were a boy. What was it like in those days? Were things
better than they are now, or were they worse?''

\switchcolumn

他在台阶上待了一会儿。小巷的对面有一家肮脏的小酒馆,酒馆的窗户上积满了灰尘,看上去就像结了一层霜。一个弯着腰的老人推开酒馆的弹簧门走了进去,他的胡子全白了,像虾须一样翘翘的,但动作依然矫健。温斯顿看着他,想,这老人至少有八十岁,在革命开始时他就已步入中年。他,以及其他一些人,是留下的极少数的能够联结已经消失了的资本主义世界的纽带。在党内,几乎没有谁的思想是在革命前定型的。绝大多数老一辈都在五六十年代的大清洗中被清除了,侥幸活下来的极个别人也已魂飞魄散,彻底在思想上投降。如果有哪个活下来的人能告诉你本世纪早期的真实情况,那他一定在群众中间。突然,他又想起他所摘抄的那段来自历史课本的文章,一个疯狂的念头萌生了,他可以走进酒馆,和那老头说说话,问他一些事情。他想问他:``您小的时候,生活是怎样的呢?和今天一样吗?哪些比现在好,哪些又比现在糟?''

\switchcolumn*

Hurriedly, lest he should have time to become frightened, he descended
the steps and crossed the narrow street. It was madness, of course. As
usual, there was no definite rule against talking to proles and
frequenting their pubs, but it was far too unusual an action to pass
unnoticed. If the patrols appeared he might plead an attack of
faintness, but it was not likely that they would believe him. He pushed
open the door, and a hideous cheesy smell of sour beer hit him in the
face. As he entered, the din of voices dropped to about half its volume.
Behind his back he could feel everyone eyeing his blue overalls. A game
of darts which was going on at the other end of the room interrupted
itself for perhaps as much as thirty seconds. The old man whom he had
followed was standing at the bar, having some kind of altercation with
the barman, a large, stout, hook-nosed young man with enormous forearms.
A knot of others, standing round with glasses in their hands, were
watching the scene.

\switchcolumn

他唯恐自己有时间害怕,所以急匆匆地走下台阶,穿过狭窄的巷子。他肯定是疯了,虽然没有明文规定他不许和群众交谈,不许进入群众的酒馆,但这样的举动很难不引起他人的注意。他想好了,若巡逻队来了,他就说他突然感到头晕,不过他们多半不会相信他。他推开门,一股可怕的劣质酸啤酒的气味扑面而来。他一走进酒馆,里面的说话声就小了一半,他可以感觉到每个人都在盯着他的蓝色制服,屋子里面玩飞镖的人足足停下来半分钟之久。他跟着那个老头来到吧台前,老头正为什么事和酒保争吵。酒保很年轻,人高马大,手臂粗壮。还有几个人端着酒杯看他们吵。

\switchcolumn*

``I arst you civil enough, didn\textquotesingle t I?'' said the old man,
straightening his shoulders pugnaciously. ``You telling me you
ain\textquotesingle t got a pint mug in the \textquotesingle ole
bleeding boozer?''

\switchcolumn

``我已经够礼貌了,不是吗?''老头非常生气地挺着肩膀说,``你的意思是这鬼地方没有他妈一品脱的杯子?''

\switchcolumn*

``And what in hell\textquotesingle s name \emph{is} a pint?'' said the
barman, leaning forward with the tips of his fingers on the counter.

\switchcolumn

``什么他妈一品脱?''酒保的手指撑着柜台,身子向前倾着。

\switchcolumn*

``\textquotesingle Ark at \textquotesingle im! Calls
\textquotesingle isself a barman and don\textquotesingle t know what a
pint is! Why, a pint\textquotesingle s the \textquotesingle alf of a
quart, and there\textquotesingle s four quarts to the gallon.
\textquotesingle Ave to teach you the A, B, C next.''

\switchcolumn

``听我说!一个酒保居然不知道什么是品脱?一品脱是半夸脱,四夸脱是一加仑。下次还得从A、B、C开始教你。''

\switchcolumn*

``Never heard of \textquotesingle em,'' said the barman shortly. ``Liter
and half liter---that\textquotesingle s all we serve.
There\textquotesingle s the glasses on the shelf in front of you.''

\switchcolumn

``从来就没听过这些。''酒保说,``一升,半升,这儿全是这个卖法。杯子就在你前面的架子上。''

\switchcolumn*

``I likes a pint,'' persisted the old man. ``You could \textquotesingle a
drawed me off a pint easy enough. We didn\textquotesingle t
\textquotesingle ave these bleeding liters when I was a young man.''

\switchcolumn

``我就喜欢说一品脱,''老头很固执,``你没那么容易让我不说品脱,我年轻时没有他妈的按升卖这回事。''

\switchcolumn*

``When you were a young man we were all living in the treetops,'' said the
barman, with a glance at the other customers.

\switchcolumn

``你年轻的时候我们还在树上呢。''酒保一边说着,一边看了看其他人。

\switchcolumn*

There was a shout of laughter, and the uneasiness caused by
Winston\textquotesingle s entry seemed to disappear. The old
man\textquotesingle s white-stubbled face had flushed pink. He turned
away, muttering to himself, and bumped into Winston. Winston caught him
gently by the arm.

\switchcolumn

大家哈哈大笑,将温斯顿刚进来时的那种不安的气氛一扫而空。老头布满胡楂儿的脸变得通红。他嘟嘟囔囔地转过身,撞到了温斯顿。温斯顿轻轻搀住他的手臂。

\switchcolumn*

``May I offer you a drink?'' he said.

\switchcolumn

``我可以请您喝一杯吗?''他问。

\switchcolumn*

``You\textquotesingle re a gent,'' said the other, straightening his
shoulders again. He appeared not to have noticed
Winston\textquotesingle s blue overalls. ``Pint!'' he added aggressively
to the barman. ``Pint of wallop.''

\switchcolumn

``真是个绅士,''老头再次挺起肩膀,似乎没有注意到温斯顿的蓝色制服。``品脱!''他向酒保喊,声音里有挑衅的意味。``一品脱猛的!''

\switchcolumn*

The barman swished two half-liters of dark-brown beer into thick glasses
which he had rinsed in a bucket under the counter. Beer was the only
drink you could get in prole pubs. The proles were supposed not to drink
gin, though in practice they could get hold of it easily enough. The
game of darts was in full swing again, and the knot of men at the bar
had begun talking about lottery tickets. Winston\textquotesingle s
presence was forgotten for a moment. There was a deal table under the
window where he and the old man could talk without fear of being
overheard. It was horribly dangerous, but at any rate there was no
telescreen in the room, a point he had made sure of as soon as he came
in.

\switchcolumn

酒保拿出两个厚玻璃杯,在吧台下的水桶里洗了洗,然后在两个杯子里分别倒了半升深棕色啤酒。在群众的酒馆里,你只能找到啤酒。群众不允许喝杜松子酒,不过他们很容易就能买到。人们暂时忘记了温斯顿,重新玩起了飞镖,谈起了彩票。温斯顿和老头在一张靠窗的桌子旁坐下,不用担心旁人听到他们的谈话。虽然这极其危险,但至少屋子里没有电屏,打一进酒馆,他就注意到这点。

\switchcolumn*

``\textquotesingle E could \textquotesingle a drawed me off a pint,''
grumbled the old man as he settled down behind his glass. ``A
\textquotesingle alf liter ain\textquotesingle t enough. It
don\textquotesingle t satisfy. And a \textquotesingle ole
liter\textquotesingle s too much. It starts my bladder running. Let
alone the price.''

\switchcolumn

``别想让我不说品脱,''老头抱怨着坐了下来,酒杯放在他眼前。``才半升,太少了,不够劲儿。一升又太多,会让我的膀胱忙个不停,更别说那价钱。''

\switchcolumn*

``You must have seen great changes since you were a young man,'' said
Winston tentatively.

\switchcolumn

``您年轻时一定和现在有很大变化。''温斯顿试探着说。

\switchcolumn*

The old man\textquotesingle s pale blue eyes moved from the darts board
to the bar, and from the bar to the door of the Gents, as though it were
in the barroom that he expected the changes to have occurred.

\switchcolumn

老头眨了眨淡蓝色的眼睛,目光从飞镖台移到吧台,又从吧台移到男厕所门口,似乎在
这酒吧里寻找着变化。

\switchcolumn*

``The beer was better,'' he said finally. ``And cheaper! When I was a young
man, mild beer---wallop, we used to call it---was fourpence a pint. That
was before the war, of course.''

\switchcolumn

最后,他说:``啤酒更好,也更便宜。我年轻时,管淡啤酒叫汽酒,四便士一品脱。当然这是战前的事了。''

\switchcolumn*

``Which war was that?'' said Winston.

\switchcolumn

``哪次战争?''温斯顿问。

\switchcolumn*

``It\textquotesingle s all wars,'' said the old man vaguely. He took up
his glass, and his shoulders straightened again.
``\textquotesingle Ere\textquotesingle s wishing you the very best of
\textquotesingle ealth!''

\switchcolumn

``所有战争,''老头含含糊糊地说,他拿起杯子,挺了挺肩膀,``祝你身体健康!''

\switchcolumn*

In his lean throat the sharp-pointed Adam\textquotesingle s apple made a
surprisingly rapid up-and-down movement, and the beer vanished. Winston
went to the bar and came back with two more half-liters. The old man
appeared to have forgotten his prejudice against drinking a full liter.

\switchcolumn

他干瘦的喉咙上尖尖的喉结飞快地上下抖动着,啤酒喝光了。温斯顿走到吧台又拿回两杯啤酒,每杯都有半升多。老头好像忘记了他刚刚还在反对喝一整升啤酒。

\switchcolumn*

``You are very much older than I am,'' said Winston. ``You must have been a
grown man before I was born. You can remember what it was like in the
old days, before the Revolution. People of my age don\textquotesingle t
really know anything about those times. We can only read about them in
books, and what it says in the books may not be true. I should like your
opinion on that. The history books say that life before the Revolution
was completely different from what it is now. There was the most
terrible oppression, injustice, poverty---worse than anything we can
imagine. Here in London, the great mass of the people never had enough
to eat from birth to death. Half of them hadn\textquotesingle t even
boots on their feet. They worked twelve hours a day, they left school at
nine, they slept ten in a room. And at the same time there were a very
few people, only a few thousands---the capitalists, they were
called---who were rich and powerful. They owned everything that there
was to own. They lived in great gorgeous houses with thirty servants,
they rode about in motor cars and four-horse carriages, they drank
champagne, they wore top hats---''

\switchcolumn

``您比我年长多了,''温斯顿说,``您在我出生前就是个成年人了,您还记得革命前的那些老日子吗?我这个年纪的人对那时的情况一无所知。我们知道的都是从书上看到的,可书上写的又不一定是真的。我想听听您怎么看。历史书里说,革命前的日子和现在完全不一样。那时的压迫最严重,贫困不公,超出人的想象。就在伦敦,大部分人直到死都没吃过饱饭,他们中有一半以上的人没靴子穿,每天工作十二个小时,九岁就离开学校,十个人挤在一间屋子里。同时,只有很少人,差不多几千个,是资本家。他们有钱有权,每样东西都是他们的财产,他们住着豪华的大房子,有三十多个仆人伺候着。他们坐着汽车、四个轮子的马车,出去喝香槟。他们戴着高高的礼帽——''

\switchcolumn*

The old man brightened suddenly.

\switchcolumn

老头突然眼睛一亮。

\switchcolumn*

``Top \textquotesingle ats!'' he said. ``Funny you should mention
\textquotesingle em. The same thing come into my \textquotesingle ead
only yesterday, I donno why. I was jest thinking, I
ain\textquotesingle t seen a top \textquotesingle at in years. Gorn
right out, they \textquotesingle ave. The last time I wore one was at my
sister-in-law\textquotesingle s funeral. And that was---well, I
couldn\textquotesingle t give you the date, but it must
\textquotesingle a been fifty year ago. Of course it was only
\textquotesingle ired for the occasion, you understand.''

\switchcolumn

``高礼帽!''他说,``真有趣,你会提起这个。我昨天还想到它了呢,不知道为什么,我想很多年都没见过高礼帽了,连影都见不着。最后一次看到它还是在很多年前,在我嫂子的葬礼上。那是在——好吧,我说不上具体的日期。但那怎么也得有五十年了,特地为葬礼租的帽子。''

\switchcolumn*

``It isn\textquotesingle t very important about the top hats,'' said
Winston patiently. ``The point is, these capitalists---they and a few
lawyers and priests and so forth who lived on them---were the lords of
the earth. Everything existed for their benefit. You---the ordinary
people, the workers---were their slaves. They could do what they liked
with you. They could ship you off to Canada like cattle. They could
sleep with your daughters if they chose. They could order you to be
flogged with something called a cat-o\textquotesingle-nine-tails. You
had to take your cap off when you passed them. Every capitalist went
about with a gang of lackeys who---''

\switchcolumn

``高礼帽不重要,''温斯顿说,``重点是那些资本家以及律师、牧师等靠他们生活的人——他们是这世界的主人,所有的一切都对他们有利。你,普通大众、工人——都是他们的奴隶,他们想怎么对你们就怎么对你们,他们可以把你们像运牛那样运到加拿大,若他们乐意,他们还可以和你们的女儿睡觉。他们使唤你们,拿一种叫九尾鞭的东西抽你们。每次你见到他们都不得不脱帽行礼。每个资本家都有一堆仆人——''

\switchcolumn*

The old man brightened again.

\switchcolumn

老头的眼睛又亮了起来。

\switchcolumn*

``Lackeys!'' he said. ``Now there\textquotesingle s a word I
ain\textquotesingle t \textquotesingle eard since ever so long. Lackeys!
That reg\textquotesingle lar takes me back, that does. I recollect---oh,
donkey\textquotesingle s years ago---I used to sometimes go to
\textquotesingle Yde Park of a Sunday afternoon to \textquotesingle ear
the blokes making speeches. Salvation Army, Roman Catholics, Jews,
Indians---all sorts, there was. And there was one bloke---well, I
couldn\textquotesingle t give you \textquotesingle is name, but a real
powerful speaker, \textquotesingle e was. \textquotesingle E
didn\textquotesingle t \textquotesingle alf give it \textquotesingle em!
`Lackeys!' \textquotesingle e says.
`Lackeys of the bourgeoisie! Flunkies of the ruling
class!' Parasites---that was another of them. And
\textquotesingle yenas---\textquotesingle e def\textquotesingle nitely
called \textquotesingle em \textquotesingle yenas. Of course
\textquotesingle e was referring to the Labour Party, you understand.''

\switchcolumn

``仆人!''他说,``现在居然还能听到这个词,我很久没听到它了。仆人!这让我回到了过去。也不知道是多少年前的事了。我常常在星期天的下午去海德公园听演讲,什么救世军、罗马天主教、犹太人、印度人——全都是这些事儿。我不能告诉你他的名字,但他很有魄力。`奴才,'他说,`中产阶级的奴才!统治阶级的走狗!'寄生虫——那是另外一个称呼他们的词,还有豺狼——他一定说过这话。当然,你知道,他说的是工党。''

\switchcolumn*

Winston had the feeling that they were talking at cross purposes.

\switchcolumn

温斯顿觉得两个人各说各话。

\switchcolumn*

``What I really wanted to know was this,'' he said. ``Do you feel that you
have more freedom now than you had in those days? Are you treated more
like a human being? In the old days, the rich people, the people at the
top---''

\switchcolumn

``我想知道的正是这个。''他说,``您有没有觉得现在比过去更自由了?您终于活得像个人了?过去,有钱人,他们高高在上——''

\switchcolumn*

``The \textquotesingle Ouse of Lords,'' put in the old man reminiscently.

\switchcolumn

``贵族院!''老头沉浸在回忆里。

\switchcolumn*

``The House of Lords, if you like. What I am asking is, were these people
able to treat you as an inferior, simply because they were rich and you
were poor? Is it a fact, for instance, that you had to call them
`Sir' and take off your cap when you
passed them?''

\switchcolumn

``贵族院,随您高兴,他们看不起您,仅仅因为他们有钱而您很穷?是这样吗?在那种情况下,您叫他们`先生',看到他们您还得把您的帽子摘下来?''

\switchcolumn*

The old man appeared to think deeply. He drank off about a quarter of
his beer before answering.

\switchcolumn

老头陷入思考,他喝掉了四分之一的啤酒,然后回答:

\switchcolumn*

``Yes,'' he said. ``They liked you to touch your cap to
\textquotesingle em. It showed respect, like. I didn\textquotesingle t
agree with it, myself, but I done it often enough. Had to, as you might
say.''

\switchcolumn

``是的。''他说,``他们喜欢你摸摸帽子表示尊敬。我不赞成这样做,我指我自己,但我也没少这么做。可以这么说,你不得不这样。''

\switchcolumn*

``And was it usual---I\textquotesingle m only quoting what
I\textquotesingle ve read in history books---was it usual for these
people and their servants to push you off the pavement into the gutter?''

\switchcolumn

``它经常发生吗——单说从历史书里看到的——那些人和他们的仆人是不是经常把你从便道上推到水沟里?''

\switchcolumn*

``One of em pushed me once,'' said the old man. ``I recollect it as if it
was yesterday. It was Boat Race night---terrible rowdy they used to get
on Boat Race night---and I bumps into a young bloke on Shaftesbury
Avenue. Quite the gent, \textquotesingle e was---dress shirt, top
\textquotesingle at, black overcoat. \textquotesingle E was kind of
zigzagging across the pavement, and I bumps into \textquotesingle im
accidental-like. \textquotesingle E says, `Why
can\textquotesingle t you look where you\textquotesingle re
going?' \textquotesingle e says. I says,
`Ju think you\textquotesingle ve bought the bleeding
pavement?' \textquotesingle E says,
`I\textquotesingle ll twist your bloody
\textquotesingle ead off if you get fresh with me.' I
says, `You\textquotesingle re drunk.
I\textquotesingle ll give you in charge in \textquotesingle alf a
minute,' I says. \textquotesingle An if
you\textquotesingle ll believe me, \textquotesingle e puts
\textquotesingle is \textquotesingle and on my chest and gives me a
shove as pretty near sent me under the wheels of a bus. Well, I was
young in them days, and I was going to \textquotesingle ave fetched
\textquotesingle im one, only---''

\switchcolumn

``只有一个人推过我,就一次,''老头说,``就像发生在昨天。划船比赛的晚上人们闹得吓人——我在夏福特伯里大街撞到了一个小伙子,他有点儿像绅士,穿着衬衫,黑外套,戴着高礼帽。在人行道上摇摇晃晃地走着,我大概没留神撞到了他。他说:`你就不能看着点儿路吗?'我说:`你以为把他妈的整条路都给买下了?'他说:`你再这么无礼,我就把你脑袋拧下来。'我说:`你醉了,等会儿再收拾你。'我这可不是瞎说。他冲过来用手推我的胸,我差点儿就被推到公交车的轮子下了。我那时很年轻,我正要教训他——''

\switchcolumn*

A sense of helplessness took hold of Winston. The old
man\textquotesingle s memory was nothing but a rubbish heap of details.
One could question him all day without getting any real information. The
Party histories might still be true, after a fashion; they might even be
completely true. He made a last attempt.

\switchcolumn

温斯顿非常无奈,这老头的回忆里只有些垃圾一般的小事。整整一天的时间都拿来问他,也问不出所以然来。党所说的历史可能是真实的,甚至有可能完全是真实的。他要做最后一次尝试。

\switchcolumn*

``Perhaps I have not made myself clear,'' he said. ``What
I\textquotesingle m trying to say is this. You have been alive a very
long time; you lived half your life before the Revolution. In 1925, for
instance, you were already grown up. Would you say, from what you can
remember, that life in 1925 was better than it is now, or worse? If you
could choose, would you prefer to live then or now?''

\switchcolumn

``也许我没讲清楚,''他说,``我的意思是,您活了很长时间,你有一半时间都是在革命前度过的。举个例子,1925年,您已经成年了。您能说说,在您的记忆中,1925年比现在好,还是差?如果您可以选择的话,您更喜欢生活在哪个时期,过去还是现在?
''

\switchcolumn*

The old man looked meditatively at the darts board. He finished up his
beer, more slowly than before. When he spoke it was with a tolerant,
philosophic air, as though the beer had mellowed him.

\switchcolumn

老头默默地看了看飞镖靶,喝光了啤酒,喝的速度比刚才慢了不少。当他再次开口说话时,语调里有一种哲学家般的隐忍,啤酒让他沉静下来。

\switchcolumn*

``I know what you expect me to say,'' he said. ``You expect me to say as
I\textquotesingle d sooner be young again. Most people\textquotesingle d
say they\textquotesingle d sooner be young, if you arst
\textquotesingle em. You got your \textquotesingle ealth and strength
when you\textquotesingle re young. When you get to my time of life you
ain\textquotesingle t never well. I suffer something wicked from my
feet, and my bladder\textquotesingle s jest terrible. Six and seven
times a night it \textquotesingle as me out of bed. On the other
\textquotesingle and there\textquotesingle s great advantages in being a
old man. You ain\textquotesingle t got the same worries. No truck with
women, and that\textquotesingle s a great thing. I ain\textquotesingle t
\textquotesingle ad a woman for near on thirty year, if
you\textquotesingle d credit it. Nor wanted to, what\textquotesingle s
more.''

\switchcolumn

``我知道你希望我说什么。''他说,``你希望我说我很快又会年轻起来,如果你问他们,大多数人都会告诉你,他们最大的愿望就是变得年轻。人年轻时,身体好,有力气。可到了我这岁数,你的身体就没那么好了。我的脚有毛病,我的膀胱也很糟糕,每天晚上都得从床上爬起来六七次。但另一方面,人老了也有好处,你不会再为同一件事操心,也没有女人缠着你。这是件非常棒的事。不管你信不信,我差不多三十年没碰过女人了。我也不想。''

\switchcolumn*

Winston sat back against the window sill. It was no use going on. He was
about to buy some more beer when the old man suddenly got up and
shuffled rapidly into the stinking urinal at the side of the room. The
extra half-liter was already working on him. Winston sat for a minute or
two gazing at his empty glass, and hardly noticed when his feet carried
him out into the street again. Within twenty years at the most, he
reflected, the huge and simple question, ``Was life better before the
Revolution than it is now?'' would have ceased once and for all to be
answerable. But in effect it was unanswerable even now, since the few
scattered survivors from the ancient world were incapable of comparing
one age with another. They remembered a million useless things, a
quarrel with a workmate, a hunt for a lost bicycle pump, the expression
on a long-dead sister\textquotesingle s face, the swirls of dust on a
windy morning seventy years ago; but all the relevant facts were outside
the range of their vision. They were like the ant, which can see small
objects but not large ones. And when memory failed and written records
were falsified---when that happened, the claim of the Party to have
improved the conditions of human life had got to be accepted, because
there did not exist, and never again could exist, any standard against
which it could be tested.

\switchcolumn

温斯顿靠着窗台坐着,没必要再继续了。他正准备再买点儿啤酒时,老头突然站了起来,
拖着脚,快步走进位于酒吧另一端的小便处。多喝的半杯啤酒在他身上起了作用。温斯
顿盯着空杯子,又多坐了一分钟,然后迷迷糊糊地走出了酒馆。他想,最多二十年,
``革命前的生活比现在好吗?''这个最简单也是最关键的问题就永远不会有答案了。事
实上,即使在今天也没有人能回答这个问题。旧时代的幸存者已经丧失了将一个时代和
另一个时代对比的能力。他们记得上百万毫无价值的事,工友间的争吵,丢失的自行车
气筒,死去多时的姐妹,甚至七十年前某个冬天的早晨那将灰尘卷起的旋风,但与之相
关的事实却不在他们的视野之内。他们就像蚂蚁,看得到小的,看不到大的。在这个记
忆不可靠、文字被伪造的时代,人们只能接受党的说法,相信生活水平被提高,因为不
存在任何可以拿来做参照的标准,这标准现在没有,以后也不会有,所有反对的观点都
不能被证实。

\switchcolumn*

At this moment his train of thought stopped abruptly. He halted and
looked up. He was in a narrow street, with a few dark little shops
interspersed among dwelling houses. Immediately above his head there
hung three discolored metal balls which looked as if they had once been
gilded. He seemed to know the place. Of course! He was standing outside
the junk shop where he had bought the diary.

\switchcolumn

他突然停止思考,停下脚步四处张望。他正待在一条窄窄的街道上,街的两旁有不少光
线昏暗的小铺子,和民房混在一起。三个褪了色的金属球就挂在温斯顿头顶上方,看起
来曾镀过金。他好像知道这是哪儿了。没错!这正是他买日记本的那家店,他就站在它
的外面。

\switchcolumn*

A twinge of fear went through him. It had been a sufficiently rash act
to buy the book in the beginning, and he had sworn never to come near
the place again. And yet the instant that he allowed his thoughts to
wander, his feet had brought him back here of their own accord. It was
precisely against suicidal impulses of this kind that he had hoped to
guard himself by opening the diary. At the same time he noticed that
although it was nearly twenty-one hours the shop was still open. With
the feeling that he would be less conspicuous inside than hanging about
on the pavement, he stepped through the doorway. If questioned, he could
plausibly say that he was trying to buy razor blades.

\switchcolumn

他被恐惧击中了。到这里买日记本已经够鲁莽的了,他曾发誓再也不靠近这里。而就在他放纵思绪东想西想时,双脚却将他带回了这里。他之所以要记日记,就是为了警醒自己不要做这类受自杀性冲动驱使的事。同时,他发现尽管时间已接近晚上9点,这家店仍然在营业。相比在外闲逛,待在店子里倒没有那么引人注意。他走进店子,若是有人问起来,他就说他是来买剃须刀片的。

\switchcolumn*

The proprietor had just lighted a hanging oil lamp which gave off an
unclean but friendly smell. He was a man of perhaps sixty, frail and
bowed, with a long, benevolent nose, and mild eyes distorted by thick
spectacles. His hair was almost white, but his eyebrows were bushy and
still black. His spectacles, his gentle, fussy movements, and the fact
that he was wearing an aged jacket of black velvet, gave him a vague air
of intellectuality, as though he had been some kind of literary man, or
perhaps a musician. His voice was soft, as though faded, and his accent
less debased than that of the majority of proles.

\switchcolumn

店主人将悬挂式的油灯点亮,油灯的气味虽然不大干净,却还算好闻。店主人大约六十岁,身体虚弱,弯腰驼背,他的鼻子偏长,看起来人很好,厚厚的眼镜片折射出他的目光,温雅和善。他的头发几乎全白了,可眉毛却依然乌黑浓密。他的眼镜、他轻柔利落的举止以及他黑色的绒质夹克,都为他增添了几分睿智,他看上去就像个文学家、音乐家。他的声音温柔无力,和大部分群众相比,他说话的腔调文雅得多。

\switchcolumn*

``I recognized you on the pavement,'' he said immediately.
``You\textquotesingle re the gentleman that bought the young
lady\textquotesingle s keepsake album. That was a beautiful bit of
paper, that was. Cream laid, it used to be called.
There\textquotesingle s been no paper like that made for---oh, I dare
say fifty years.'' He peered at Winston over the top of his spectacles.
``Is there anything special I can do for you? Or did you just want to
look round?''

\switchcolumn

``您在街上我就认出您了,''他说,``您就是那个买年轻女士的笔记本的先生。那种纸真漂亮,奶油色,以前人都这么叫。现在已经没有这样的纸了,我敢说五十年没生产了。''他的视线穿过镜架上端,``您想买点什么?还是只随便转转?''

\switchcolumn*

``I was passing,'' said Winston vaguely. ``I just looked in. I
don\textquotesingle t want anything in particular.''

\switchcolumn

``我路过这儿,''温斯顿含含混混地说。``没有什么特别要买的东西,我就进来看看。''

\switchcolumn*

``It\textquotesingle s just as well,'' said the other, ``because I
don\textquotesingle t suppose I could have satisfied you.'' He made an
apologetic gesture with his soft-palmed hand. ``You see how it is; an
empty shop, you might say. Between you and me, the antique
trade\textquotesingle s just about finished. No demand any longer, and
no stock either. Furniture, china, glass---it\textquotesingle s all been
broken up by degrees. And of course the metal stuff\textquotesingle s
mostly been melted down. I haven\textquotesingle t seen a brass
candlestick in years.''

\switchcolumn

``也好,''店主说,``反正我也没什么东西能卖给您了。''他摊了摊柔软的手,做了个抱歉的姿势。``您都看到了,可以说,这店已经空了。就咱俩说说,旧货生意算是做到头了,不再有人买,也不再有存货。家具、瓷器、玻璃制品,都慢慢地坏掉。金属制品也大多被收走熔化了,我已经很多年没见过黄铜质地的蜡烛台了。''

\switchcolumn*

The tiny interior of the shop was in fact uncomfortably full, but there
was almost nothing in it of the slightest value. The floor space was
very restricted, because all round the walls were stacked innumerable
dusty picture frames. In the window there were trays of nuts and bolts,
worn-out chisels, penknives with broken blades, tarnished watches that
did not even pretend to be in going order, and other miscellaneous
rubbish. Only on a small table in the corner was there a litter of odds
and ends---lacquered snuffboxes, agate brooches, and the like---which
looked as though they might include something interesting. As Winston
wandered toward the table his eye was caught by a round, smooth thing
that gleamed softly in the lamplight, and he picked it up.

\switchcolumn

店里又挤又小,让人感觉很不舒服,也没有什么有价值的东西。靠墙处堆了很多积满灰尘的相框,让地板的空间愈发有限。橱窗里摆着不少螺钉螺母,一叠叠地摞起来,此外还有磨损严重的刻刀,豁了口的卷笔刀,失去光泽又走不动的表以及其他一些不能用的杂货。只有角落里的小桌子上还有些有趣的小物件,比如涂了漆的鼻烟壶、玛瑙做的胸针。温斯顿朝这桌子走去,一个圆而光滑的东西引起了他的注意。在小油灯的照射下,它散发出柔和的光。他将它拿了起来。

\switchcolumn*

It was a heavy lump of glass, curved on one side, flat on the other,
making almost a hemisphere. There was a peculiar softness, as of
rainwater, in both the color and the texture of the glass. At the heart
of it, magnified by the curved surface, there was a strange, pink,
convoluted object that recalled a rose or a sea anemone.

\switchcolumn

这是一块半球形的厚玻璃,一面是弧形,一面是平的,颜色和质地都像雨水般柔和。在它的中间,还有个粉红色的、形似海藻和玫瑰的东西,被玻璃的弧面放大。

\switchcolumn*

``What is it?'' said Winston, fascinated.

\switchcolumn

``这是什么?''温斯顿问。

\switchcolumn*

``That\textquotesingle s coral, that is,'' said the old man. ``It must have
come from the Indian Ocean. They used to kind of embed it in the glass.
That wasn\textquotesingle t made less than a hundred years ago. More, by
the look of it.''

\switchcolumn

``里面那个是珊瑚,''老头说,``它一定是从印度洋来的。他们把它镶到玻璃里。这东西差不多有一百年了,从样子上看,似乎更久。''

\switchcolumn*

``It\textquotesingle s a beautiful thing,'' said Winston.

\switchcolumn

``它真漂亮。''温斯顿说。

\switchcolumn*

``It is a beautiful thing,'' said the other appreciatively. ``But
there\textquotesingle s not many that\textquotesingle d say so
nowadays.'' He coughed. ``Now, if it so happened that you wanted to buy
it, that\textquotesingle d cost you four dollars. I can remember when a
thing like that would have fetched eight pounds, and eight pounds
was---well, I can\textquotesingle t work it out, but it was a lot of
money. But who cares about genuine antiques nowadays---even the few
that\textquotesingle s left?''

\switchcolumn

``它是很漂亮!''老头欣赏地说。
``现在已经没有什么东西能这么说了。''他咳嗽了一下。
``如果您真想买,就给四块钱吧。我记得,这样的东西从前能卖到八镑。唉,我算不出来,但那真是不少钱。可惜现在还有几个人会注意到老古董呢?况且也没有多少古董留下来。''

\switchcolumn*

Winston immediately paid over the four dollars and slid the coveted
thing into his pocket. What appealed to him about it was not so much its
beauty as the air it seemed to possess of belonging to an age quite
different from the present one. The soft, rainwatery glass was not like
any glass that he had ever seen. The thing was doubly attractive because
of its apparent uselessness, though he could guess that it must once
have been intended as a paperweight. It was very heavy in his pocket,
but fortunately it did not make much of a bulge. It was a queer thing,
even a compromising thing, for a Party member to have in his possession.
Anything old, and for that matter anything beautiful, was always vaguely
suspect. The old man had grown noticeably more cheerful after receiving
the four dollars. Winston realized that he would have accepted three or
even two.

\switchcolumn

温斯顿马上掏出了四块钱买下了它,将它放到口袋里。吸引他的,并不是它的漂亮,而是它带来的那种感觉,它诞生的时代和今天截然不同。这块柔和的,像雨一样的玻璃和他所见过的任何玻璃都不一样。它尤其吸引他的正是它的没有用处,但话说回来,他推测,在过去它一定被人拿来当做镇纸。它让他的口袋沉甸甸的,所幸从外面看不算太鼓。作为一个党员,他不应该拥有它。任何老旧的,美丽的东西都会引人怀疑。老头收到四块钱后高兴多了,温斯顿觉得,就算给他两三块钱,他也会接受。

\switchcolumn*

``There\textquotesingle s another room upstairs that you might care to
take a look at,'' he said. ``There\textquotesingle s not much in it. Just
a few pieces. We\textquotesingle ll do with a light if
we\textquotesingle re going upstairs.''

\switchcolumn

``楼上还有些房间,您可以看看,''他说,``里面没有什么东西,就几样。如果上楼,
就带盏灯。''

\switchcolumn*

He lit another lamp and, with bowed back, led the way slowly up the
steep and worn stairs and along a tiny passage, into a room which did
not give on the street but looked out on a cobbled yard and a forest of
chimney pots. Winston noticed that the furniture was still arranged as
though the room were meant to be lived in. There was a strip of carpet
on the floor, a picture or two on the walls, and a deep, slatternly
armchair drawn up to the fireplace. An old-fashioned glass clock with a
twelve-hour face was ticking away on the mantlepiece. Under the window,
and occupying nearly a quarter of the room, was an enormous bed with the
mattress still on it.

\switchcolumn

他另拿了一盏灯,然后弯着腰,步履缓慢地在前面带路。陡峭破旧的楼梯连接着狭窄的走廊,他们进入一个房间,房间正对着一个铺着鹅卵石的院子和一片树丛。温斯顿注意到房间里的家具,好像一直都有人住在这里。地板上铺着一小块地毯,墙壁上挂着一两幅画,壁炉的旁边还顶着一把脏兮兮的高背扶手椅。一个老式的十二格玻璃面时钟在壁炉上滴滴答答地走着。窗户下,一张带床垫的大床占据了屋内四分之一的面积。

\switchcolumn*

``We lived here till my wife died,'' said the old man half apologetically.
``I\textquotesingle m selling the furniture off by little and little. Now
that\textquotesingle s a beautiful mahogany bed, or at least it would be
if you could get the bugs out of it. But I dare say
you\textquotesingle d find it a little bit cumbersome.''

\switchcolumn

``我妻子去世前,我们就住在这里,''老头说,``我正把家具一点点地卖了,那张床很漂亮,是红木做的,当然至少要把上面的臭虫清干净。不过我猜您可能觉得它有点儿笨重了。''

\switchcolumn*

He was holding the lamp high up, so as to illumine the whole room, and
in the warm dim light the place looked curiously inviting. The thought
flitted through Winston\textquotesingle s mind that it would probably be
quite easy to rent the room for a few dollars a week, if he dared to
take the risk. It was a wild, impossible notion, to be abandoned as soon
as thought of; but the room had awakened in him a sort of nostalgia, a
sort of ancestral memory. It seemed to him that he knew exactly what it
felt like to sit in a room like this, in an armchair beside an open fire
with your feet in the fender and a kettle on the hob, utterly alone,
utterly secure, with nobody watching you, no voice pursuing you, no
sound except the singing of the kettle and the friendly ticking of the
clock.

\switchcolumn

他把灯高高地举起来,好照亮整个房间。在温暖昏暗的灯光下,房间散发出一种古怪的魅力。一个念头闪过温斯顿的大脑,如果他敢冒这个险,租下房子并不困难,每周只要花几块钱。尽管这念头太不现实,以至于刚一萌生他就放弃。但这房子却唤起了他的怀旧之情,唤起了他对往昔的回忆。他很清楚住在这样的房子里会是什么感觉,人坐在扶手椅里,面前是燃烧的炉火,人可以把脚放在挡炉板上,把壶放在铁架上。没有人监视你,也不会有声音来烦你,除了烧水声和钟表的滴答声,再没有别的声音,绝对的独自一人,绝对的安静。

\switchcolumn*

``There\textquotesingle s no telescreen!'' he could not help murmuring.

\switchcolumn

``这儿居然没有电屏!''他不禁说出了声。

\switchcolumn*

``Ah,'' said the old man, ``I never had one of those things. Too expensive.
And I never seemed to feel the need of it, somehow. Now
that\textquotesingle s a nice gateleg table in the corner there. Though
of course you\textquotesingle d have to put new hinges on it if you
wanted to use the flaps.''

\switchcolumn

``啊,''老头说,``我从来就没装过这东西。太贵了。我从来都不觉得自己需要它。角落里的折叠桌不错,不过当然,如果您要用桌上的活板,您得换新的合叶。''

\switchcolumn*

There was a small bookcase in the other corner, and Winston had already
gravitated toward it. It contained nothing but rubbish. The hunting-down
and destruction of books had been done with the same thoroughness in the
prole quarters as everywhere else. It was very unlikely that there
existed anywhere in Oceania a copy of a book printed earlier than 1960.
The old man, still carrying the lamp, was standing in front of a picture
in a rosewood frame which hung on the other side of the fireplace,
opposite the bed.

\switchcolumn

墙角处的小书架吸引了温斯顿,他走过去,发现上面没有什么有价值的东西。就算在群众间图书的查抄和销毁工作也像其他地方一样完成得相当彻底。在大洋国,人们几乎不可能找到1960年以前出版的图书。老头举着灯站在一幅画的前面,画镶在带有蔷薇纹样的画框里,就挂在壁炉旁边,正对着床的墙上。

\switchcolumn*

``Now, if you happen to be interested in old prints at all---'' he began
delicately.

\switchcolumn

``如果您对老版画有兴趣——''他小心地说。

\switchcolumn*

Winston came across to examine the picture. It was a steel engraving of
an oval building with rectangular windows, and a small tower in front.
There was a railing running round the building, and at the rear end
there was what appeared to be a statue. Winston gazed at it for some
moments. It seemed vaguely familiar, though he did not remember the
statue.

\switchcolumn

温斯顿走上前,仔细察看了那幅画。那是一幅钢板版画,画上有个嵌着长方形窗户的椭圆形建筑,建筑前方有个带栏杆的小塔。塔的后边,还有些雕像样的东西。温斯顿盯着画看了一会儿,他好像在哪看到过这建筑,又不记得什么地方有雕像。

\switchcolumn*

``The frame\textquotesingle s fixed to the wall,'' said the old man, ``but
I could unscrew it for you, I dare say.''

\switchcolumn

``画框钉在墙上了,''老头说,``但我可以给您取下来。''

\switchcolumn*

``I know that building,'' said Winston finally. ``It\textquotesingle s a
ruin now. It\textquotesingle s in the middle of the street outside the
Palace of Justice.''

\switchcolumn

``我知道这建筑,''温斯顿说,``它已经被毁掉了,它就在正义宫外面的街道上。''

\switchcolumn*

``That\textquotesingle s right. Outside the Law Courts. It was bombed
in---oh, many years ago. It was a church at one time. St.
Clement\textquotesingle s Dane, its name was.'' He smiled apologetically,
as though conscious of saying something slightly ridiculous, and added:
``\emph{Oranges and lemons, say the bells of St.
Clement\textquotesingle s!}''

\switchcolumn

``对,它就在法院外面。它在很多年前被炸掉了。它曾是一座教堂,圣克莱门特教堂。''他抱歉地笑了笑,似乎意识到自己说了荒谬的话,``橘子和柠檬,圣克莱门特的大钟说。''

\switchcolumn*

``What\textquotesingle s that?'' said Winston.

\switchcolumn

``什么?''温斯顿问。

\switchcolumn*

``Oh---\emph{Oranges and lemons, say the bells of St.
Clement\textquotesingle s}. That was a rhyme we had when I was a little
boy. How it goes on I don\textquotesingle t remember, but I do know it
ended up, \emph{Here comes a candle to light you to bed, Here comes a
chopper to chop off your head}. It was a kind of a dance. They held out
their arms for you to pass under, and when they came to \emph{Here comes
a chopper to chop off your head} they brought their arms down and caught
you. It was just names of churches. All the London churches were in
it---all the principal ones, that is.''

\switchcolumn

``噢,橘子和柠檬,圣克莱门特的大钟说——我小时候经常念的押韵句子。后面的内容记不清了,但我记得结尾:蜡烛照着你睡觉,斧头把你头砍掉。这是一种舞,他们伸起手让你从下面钻过去,当他们唱到斧头把你头砍掉的时候,就用胳膊把你的头夹住。伦敦所有主要教堂,歌谣里都唱到了。''

\switchcolumn*

Winston wondered vaguely to what century the church belonged. It was
always difficult to determine the age of a London building. Anything
large and impressive, if it was reasonably new in appearance, was
automatically claimed as having been built since the Revolution, while
anything that was obviously of earlier date was ascribed to some dim
period called the Middle Ages. The centuries of capitalism were held to
have produced nothing of any value. One could not learn history from
architecture any more than one could learn it from books. Statues,
inscriptions, memorial stones, the names of streets---anything that
might throw light upon the past had been systematically altered.

\switchcolumn

温斯顿并不清楚这些教堂都是哪个世纪的产物,伦敦建筑的建造年代很难确定。所有高大华丽的建筑,只要外表尚新,都被说成革命后建造的,而所有明显的早期建筑都被纳入中世纪建筑的范畴。资本主义被认为在长达几个世纪的时间里没有生产过任何有价值的东西。人们从建筑上了解到的并不比书本上的多。雕像、碑文、纪念牌、街道名——所有能揭示过去的,都被更改了。

\switchcolumn*

``I never knew it had been a church,'' he said.

\switchcolumn

``真想不到它以前是教堂。''温斯顿说。

\switchcolumn*

``There\textquotesingle s a lot of them left, really,'' said the old man,
``though they\textquotesingle ve been put to other uses. Now, how did
that rhyme go? Ah! I\textquotesingle ve got it!

\switchcolumn

``很多教堂都留下来了,真的。''老头说,``不过,它们被用来做别的事了。那首歌怎么唱来着?啊,我想起来了!''

\switchcolumn*

\begin{quotation}
  Oranges and lemons, say the bells of St.
  Clement\textquotesingle s,\\
  You owe me three farthings, say the bells of St. Martin\textquotesingle
  s---
\end{quotation}

\switchcolumn

\begin{quotation}
\noindent 橘子和柠檬,圣克莱门特教堂的大钟说。\\
你欠我三个法寻。圣马丁教堂的大钟说——
\end{quotation}

\switchcolumn*

there, now, that\textquotesingle s as far as I can get. A farthing, that
was a small copper coin, looked something like a cent.''

\switchcolumn

``我只记得这么多了。法寻就是那种小铜板,类似一分钱。''

\switchcolumn*

``Where was St. Martin\textquotesingle s?'' said Winston.

\switchcolumn

``圣马丁教堂在哪儿?''温斯顿问。

\switchcolumn*

``St. Martin\textquotesingle s? That\textquotesingle s still standing.
It\textquotesingle s in Victory Square, alongside the picture gallery. A
building with a kind of a triangular porch and pillars in front, and a
big flight of steps.''

\switchcolumn

``圣马丁教堂?它还在胜利广场,就在画廊那边。它前面有三角形的柱子,台阶又大又高。''

\switchcolumn*

Winston knew the place well. It was a museum used for propaganda
displays of various kinds---scale models of rocket bombs and Floating
Fortresses, waxwork tableaux illustrating enemy atrocities, and the
like.

\switchcolumn

温斯顿对那里非常熟悉。现在,它成了博物馆,展出着各种宣传用的东西,比如火箭弹和水上堡垒的模型,用来表现敌人残酷性的蜡像等等。

\switchcolumn*

``St. Martin\textquotesingle s in the Fields it used to be called,''
supplemented the old man, ``though I don\textquotesingle t recollect any
fields anywhere in those parts.''

\switchcolumn

``过去,圣马丁教堂的名字是田野圣马丁教堂。''老头补充道,``但我不记得那里有什么田野。''

\switchcolumn*

Winston did not buy the picture. It would have been an even more
incongruous possession than the glass paperweight, and impossible to
carry home, unless it were taken out of its frame. But he lingered for
some minutes more, talking to the old man, whose name, he discovered,
was not Weeks---as one might have gathered from the inscription over the
shopfront---but Charrington. Mr. Charrington, it seemed, was a widower
aged sixty-three and had inhabited this shop for thirty years.
Throughout that time he had been intending to alter the name over the
window, but had never quite got to the point of doing it. All the while
that they were talking the half-remembered rhyme kept running through
Winston\textquotesingle s head: \emph{Oranges and lemons, say the bells
of St. Clement\textquotesingle s, You owe me three farthings, say the
bells of St. Martin\textquotesingle s!} It was curious, but when you
said it to yourself you had the illusion of actually hearing bells, the
bells of a lost London that still existed somewhere or other, disguised
and forgotten. From one ghostly steeple after another he seemed to hear
them pealing forth. Yet so far as he could remember he had never in real
life heard church bells ringing.

\switchcolumn

温斯顿没有买下那幅画,和玻璃镇纸相比,它更不适合被买回家,而且也不可能将它带
走,除非把它从画框中取出来。温斯顿在老头这里又多待了几分钟,和他说了些话。根
据店外的题字,人们猜测他叫威克斯,但他不叫这个名字,他叫查林顿,是个鳏夫,今
年六十三岁,在这里住了三十年。三十年来,他一直想把橱窗上的名字改过来,可又一
直没有改。和他交谈时,温斯顿的脑子里总是跑出那首歌谣。``橘子和柠檬。圣克莱门
特教堂的大钟说。你欠我三个法寻。圣马丁教堂的大钟说。''念着念着他好像听到了钟
声,这钟声属于失去的伦敦,那个伦敦仍留在某处,它改变了样貌,被人遗忘。他似乎
听到那钟声从一个又一个鬼影幢幢的尖塔中传来,尽管记忆里,他从未在现实中听到教
堂的钟声。

\switchcolumn*

He got away from Mr. Charrington and went down the stairs alone, so as
not to let the old man see him reconnoitering the street before stepping
out of the door. He had already made up his mind that after a suitable
interval---a month, say---he would take the risk of visiting the shop
again. It was perhaps not more dangerous than shirking an evening at the
Center. The serious piece of folly had been to come back here in the
first place, after buying the diary and without knowing whether the
proprietor of the shop could be trusted. However---!

\switchcolumn

他告别了查林顿先生,一个人下了楼,他不想让老头看到他在出门前察看外面的街道。他决定,每隔一段时间,比如一个月,就冒险到这里看看,这也许要比从活动中心溜走危险得多。买日记本已经很愚蠢了,更别说他又来了一次,而且他还不知道那老头是不是值得信任,但是——

\switchcolumn*

Yes, he thought again, he would come back. He would buy further scraps
of beautiful rubbish. He would buy the engraving of St.
Clement\textquotesingle s Dane, take it out of its frame, and carry it
home concealed under the jacket of his overalls. He would drag the rest
of that poem out of Mr. Charrington\textquotesingle s memory. Even the
lunatic project of renting the room upstairs flashed momentarily through
his mind again. For perhaps five seconds exaltation made him careless,
and he stepped out onto the pavement without so much as a preliminary
glance through the window. He had even started humming to an improvised
tune---

\switchcolumn

是的,他还会回来的。他还会买美丽而无用的东西。他会买下圣克莱门特教堂的版画,他会把它从画框中取出来,藏在制服里带回家。他要把那歌谣完整地从查林顿先生的记忆中挖出来,他甚至要将楼上的房间租下来。这些疯狂的念头一度出现在他的脑际,持续了足有五秒钟,让他兴奋得忘乎所以,他没有隔着橱窗察看就走到了大街上。他还编了个曲子哼起了歌。

\switchcolumn*

\begin{quotation}
\noindent Oranges and lemons, say the bells of St.
Clement\textquotesingle s,\\
You owe me three farthings, say the---\\
\end{quotation}

\switchcolumn

\begin{quotation}
\noindent 橘子和柠檬。圣克莱门特教堂的大钟说。\\
你欠我三个法寻,圣马丁——
\end{quotation}

\switchcolumn*

Suddenly his heart seemed to turn to ice and his bowels to water. A
figure in blue overalls was coming down the pavement, not ten meters
away. It was the girl from the Fiction Department, the girl with dark
hair. The light was failing, but there was no difficulty in recognizing
her. She looked him straight in the face, then walked quickly on as
though she had not seen him.

\switchcolumn


突然,他整个人好像由内而外地冻结了。就在他前面不到十米的地方,一个穿着蓝色制服的人走了过来。是小说司里那个黑发女孩。光线很暗,却不妨碍他认出她。她直视他的脸,然后又迅速走着,好像没有看到他。

\switchcolumn*

For a few seconds Winston was too paralyzed to move. Then he turned to
the right and walked heavily away, not noticing for the moment that he
was going in the wrong direction. At any rate, one question was settled.
There was no doubting any longer that the girl was spying on him. She
must have followed him here, because it was not credible that by pure
chance she should have happened to be walking on the same evening up the
same obscure back street, kilometers distant from any quarter where
Party members lived. It was too great a coincidence. Whether she was
really an agent of the Thought Police, or simply an amateur spy actuated
by officiousness, hardly mattered. It was enough that she was watching
him. Probably she had seen him go into the pub as well.

\switchcolumn

有那么几秒,温斯顿吓得一动不动。之后,他向右转去,步伐沉重地走开了,一时间竟没发现自己走错了路。他思考着要如何解决这个问题。毫无疑问,她正在监视他。她一定跟着他来到这里。他不能相信,在同一个晚上,他们同时出现在和党员所住地区相距甚远的街道上,仅仅是巧合。无论她是思想警察还是热忱过头的业余侦探,都无关紧要。只要她正在监视他,就足够了,也许她还看到他走进了那家小酒馆。

\switchcolumn*

It was an effort to walk. The lump of glass in his pocket banged against
his thigh at each step, and he was half minded to take it out and throw
it away. The worst thing was the pain in his belly. For a couple of
minutes he had the feeling that he would die if he did not reach a
lavatory soon. But there would be no public lavatories in a quarter like
this. Then the spasm passed, leaving a dull ache behind.

\switchcolumn

连走路都变得困难了。每走一步,口袋里的玻璃都要撞击他的大腿,他甚至想把它拿出来扔掉。最糟糕的是,他的肚子疼了起来,他觉得若不能马上找到厕所,他就会死掉。然而附近偏偏一间厕所都没有。过了一会儿,疼劲儿过去了,只留下隐隐的痛感。

\switchcolumn*

The street was a blind alley. Winston halted, stood for several seconds
wondering vaguely what to do, then turned round and began to retrace his
steps. As he turned it occurred to him that the girl had only passed him
three minutes ago and that by running he could probably catch up with
her. He could keep on her track till they were in some quiet place, and
then smash her skull in with a cobblestone. The piece of glass in his
pocket would be heavy enough for the job. But he abandoned the idea
immediately, because even the thought of making any physical effort was
unbearable. He could not run, he could not strike a blow. Besides, she
was young and lusty and would defend herself. He thought also of
hurrying to the Community Center and staying there till the place
closed, so as to establish a partial alibi for the evening. But that too
was impossible. A deadly lassitude had taken hold of him. All he wanted
was to get home quickly and then sit down and be quiet.

\switchcolumn

这是一条死路。温斯顿停下来,站了几秒,不知如何是好。然后他转身沿原路返回。这
时他突然想到,就在三分钟前,他才和那个女孩擦肩而过,若他跑起来,说不定能追上
她。他可以跟着她,一直跟到僻静的地方,然后用石头打碎她的脑袋,口袋里的玻璃足
够重了,完全能拿来一用。但他很快又打消了这个念头,因为单是想象所耗费的力气就
让他难以忍受。他跑不动,也砸不动,她年轻力壮,她会出手还击。他想快点回到活动
中心,并在那儿一直待到关门,然后把这当做晚上哪儿也没去的证据。但这不可能,要
命的疲倦感抓住了他,他只想快点到家,好好静一静。

\switchcolumn*

It was after twenty-two hours when he got back to the flat. The lights
would be switched off at the main at twenty-three thirty. He went into
the kitchen and swallowed nearly a teacupful of Victory Gin. Then he
went to the table in the alcove, sat down, and took the diary out of the
drawer. But he did not open it at once. From the telescreen a brassy
female voice was squalling a patriotic song. He sat staring at the
marbled cover of the book, trying without success to shut the voice out
of his consciousness.

\switchcolumn

当他回到公寓时,已过了晚上10点,到11点公寓的总闸就会关掉。他走进厨房吞下几乎
满满一茶杯的胜利牌杜松子酒,然后回到桌子旁,坐下来,从抽屉里拿出日记本。他没
有立即将日记本打开。电屏里,一个粗哑的女声正唱着爱国歌曲。他坐着,注视着日记
本封面上的大理石花纹,他试图不去听那歌声,但他做不到。

\switchcolumn*

It was at night that they came for you, always at night. The proper
thing was to kill yourself before they got you. Undoubtedly some people
did so. Many of the disappearances were actually suicides. But it needed
desperate courage to kill yourself in a world where firearms, or any
quick and certain poison, were completely unprocurable. He thought with
a kind of astonishment of the biological uselessness of pain and fear,
the treachery of the human body which always freezes into inertia at
exactly the moment when a special effort is needed. He might have
silenced the dark-haired girl if only he had acted quickly enough; but
precisely because of the extremity of his danger he had lost the power
to act. It struck him that in moments of crisis one is never fighting
against an external enemy but always against one\textquotesingle s own
body. Even now, in spite of the gin, the dull ache in his belly made
consecutive thought impossible. And it is the same, he perceived, in all
seemingly heroic or tragic situations. On the battlefield, in the
torture chamber, on a sinking ship, the issues that you are fighting for
are always forgotten, because the body swells up until it fills the
universe, and even when you are not paralyzed by fright or screaming
with pain, life is a moment-to-moment struggle against hunger or cold or
sleeplessness, against a sour stomach or an aching tooth.

\switchcolumn

他们会在夜里逮捕你,总是在夜里。对此正确的做法是,在他们到来之前自杀,有些人确是这么做的,很多失踪实际上都是自杀。但在这世上,人们不可能得到枪支,也不可能得到任何能迅速将人置于死地的毒药,自杀需要莫大的勇气。他意识到一个令人震惊的道理,疼痛和恐惧会带来生理上的无助感,越是到了需要特别用力的时候,身体就越是不听使唤地僵在那里。若他的动作够快,他就能将那个黑发女孩杀死,但正因为他的处境极其危险,他失去了行动的力量。面对危险,人要对付的不是外在的敌人,而是人自己的身体。即使现在喝下杜松子酒,腹部的隐痛仍让他无法条理清楚地进行思考。他突然发现那些看上去英勇、悲壮的事情都不过如此。在战场上,在刑讯室中,在即将沉没的轮船里,你会忘记你所对抗的东西,因为身体将成为最大的问题,就算你没有被吓倒,没有痛苦哀号,生命仍要和饥饿、寒冷、失眠搏斗,要和胃疼、牙疼搏斗。

\switchcolumn*

He opened the diary. It was important to write something down. The woman
on the telescreen had started a new song. Her voice seemed to stick into
his brain like jagged splinters of glass. He tried to think of
O\textquotesingle Brien, for whom, or to whom, the diary was written,
but instead he began thinking of the things that would happen to him
after the Thought Police took him away. It would not matter if they
killed you at once. To be killed was what you expected. But before death
(nobody spoke of such things, yet everybody knew of them) there was the
routine of confession that had to be gone through: the groveling on the
floor and screaming for mercy, the crack of broken bones, the smashed
teeth and bloody clots of hair. Why did you have to endure it, since the
end was always the same? Why was it not possible to cut a few days or
weeks out of your life? Nobody ever escaped detection, and nobody ever
failed to confess. When once you had succumbed to thoughtcrime it was
certain that by a given date you would be dead. Why then did that
horror, which altered nothing, have to lie embedded in future time?

\switchcolumn

他翻开日记本,记下的内容非常重要。电屏里的女人唱起了新的曲子,尖利的声音像玻
璃碴儿一样插进他的大脑。他努力回想奥布兰的样子,这日记为他而记,或者说就是给
他写的。他开始想象在被思想警察带走后,他会有怎样的遭遇。被处死是意料之中,如
果他们没有马上处死他就没关系,但在死前(虽然没人说过,大家却都知道)在认罪的
过程中,他将不可避免地尝尽苦头:他会趴在地板上尖叫哀求,他的骨头将被打断,牙
齿被打掉,头发被鲜血染红。若它们都指向同一个结果,为什么要忍受这些呢?为什么
不把你的生命缩短几天或几个星期呢?没有人能躲过侦查,也没有人能拒不认罪。一旦
你犯下思想罪,你就注定要被处死。为什么他们要做那些恐怖的事儿,又起不到任何警
示作用,难道是为了让未来记住吗?

\switchcolumn*

He tried with a little more success than before to summon up the image
of O\textquotesingle Brien. ``We shall meet in the place where there is
no darkness,'' O\textquotesingle Brien had said to him. He knew what it
meant, or thought he knew. The place where there is no darkness was the
imagined future, which one would never see, but which, by foreknowledge,
one could mystically share in. But with the voice from the telescreen
nagging at his ears he could not follow the train of thought further. He
put a cigarette in his mouth. Half the tobacco promptly fell out on to
his tongue, a bitter dust which was difficult to spit out again. The
face of Big Brother swam into his mind, displacing that of
O\textquotesingle Brien. Just as he had done a few days earlier, he slid
a coin out of his pocket and looked at it. The face gazed up at him,
heavy, calm, protecting, but what kind of smile was hidden beneath the
dark mustache? Like a leaden knell the words came back at him:

\switchcolumn

他又尝试着想象奥布兰的样子,并成功了一点。``我们将在没有黑暗的地方相见。''奥布兰曾这样对他说过,他自认他明白这话的含义。没有黑暗的地方就是想象中的未来,人们永远不会看到。但是,若具备了预知的能力,就能秘密地分享它。电屏里传出的声音正骚扰着他的耳朵,让他无法顺着这个思路畅想下去。他叼起一支烟,有一半烟丝都掉在了他的舌头上,又苦涩又不容易吐出来。在他的意识里,老大哥的脸取代了奥布兰。就像几天前所做的那样,他从口袋里掏出一枚硬币,看着它。硬币上的那张脸注视着他,神情凝重、沉静又充满警惕,而他黑色的八字胡后面藏着的又是怎样的微笑?好像一个沉重的不祥之兆,他又看到了那几句话:

\switchcolumn*

\headline{
  WAR IS PEACE\\
  FREEDOM IS SLAVERY\\
  IGNORANCE IS STRENGTH. }

\switchcolumn

\headline{战争即和平\\
  自由即奴役\\
无知即力量}

\switchcolumn*


\clearpage
\part{TWO}\label{two}

\switchcolumn

\part*{第二部分}

\switchcolumn*

\section{I}

\switchcolumn

\section*{一}\label{ux4e5d}

\switchcolumn*

It was the middle of the morning, and Winston had left his cubicle to go
to the lavatory.

\switchcolumn

上午过去一半,温斯顿离开他的小隔间上厕所。

\switchcolumn*

A solitary figure was coming toward him from the other end of the long,
brightly lit corridor. It was the girl with dark hair. Four days had
gone past since the evening when he had run into her outside the junk
shop. As she came nearer he saw that her right arm was in a sling, not
noticeable at a distance because it was of the same color as her
overalls. Probably she had crushed her hand while swinging round one of
the big kaleidoscopes on which the plots of novels were ``roughed in.'' It
was a common accident in the Fiction Department.

\switchcolumn

长长的走廊里亮堂堂的,从走廊的另一端走来一个人,是那个黑头发的女孩。自旧货店门口遇到她的那个晚上起,已经过去四天。当她走近时,他发现她的右臂吊着绷带,由于绷带的颜色和制服的相同,在远处看不出来。她也许是在操作那台大型搅拌机时弄伤的手,小说的情节雏形就是在这搅拌机里形成的。这种事故在小说司里非常常见。

\switchcolumn*

They were perhaps four meters apart when the girl stumbled and fell
almost flat on her face. A sharp cry of pain was wrung out of her. She
must have fallen right on the injured arm. Winston stopped short. The
girl had risen to her knees. Her face had turned a milky yellow color
against which her mouth stood out redder than ever. Her eyes were fixed
on his, with an appealing expression that looked more like fear than
pain.

\switchcolumn

二人相距大约四米时,女孩摔倒了,她几乎面朝下摔在地上,疼痛让她发出尖叫,一定
是跌到了那条受伤的手臂上。温斯顿立刻停下脚步。女孩已经跪起来,她脸色蜡黄,嘴
唇被衬得更红。她看着他的眼睛,那楚楚可怜的神情与其说是出于疼痛,不如说是出于
害怕。

\switchcolumn*

A curious emotion stirred in Winston\textquotesingle s heart. In front
of him was an enemy who was trying to kill him; in front of him, also,
was a human creature, in pain and perhaps with a broken bone. Already he
had instinctively started forward to help her. In the moment when he had
seen her fall on the bandaged arm, it had been as though he felt the
pain in his own body.

\switchcolumn

一种奇怪的感情涌上温斯顿心头。在他眼前的是敌人,是想杀死他的人,但同时又是一
个受了伤的、正忍受疼痛并有可能骨折的人。他本能地走过去帮助她,当他看到她刚好
跌在缠着绷带的手臂上,他好像也疼了起来。

\switchcolumn*

``You\textquotesingle re hurt?'' he said.

\switchcolumn

``你受伤了吗?''他问。

\switchcolumn*

``It\textquotesingle s nothing. My arm. It\textquotesingle ll be all
right in a second.''

\switchcolumn

``没事。我摔到了手臂,过一会儿就好了。''

\switchcolumn*

She spoke as though her heart were fluttering. She had certainly turned
very pale.

\switchcolumn

她说着,心跳得厉害,脸色明显变得苍白。

\switchcolumn*

``You haven\textquotesingle t broken anything?''

\switchcolumn

``没摔坏哪儿吧?''

\switchcolumn*

``No, I\textquotesingle m all right. It hurt for a moment,
that\textquotesingle s all.''

\switchcolumn

``没,还好,疼一会儿就好了,没关系。''

\switchcolumn*

She held out her free hand to him, and he helped her up. She had
regained some of her color, and appeared very much better.

\switchcolumn

她将那只还能活动的手伸给他,他帮她站了起来,她气色恢复了一些,看起来好多了。

\switchcolumn*

``It\textquotesingle s nothing,'' she repeated shortly. ``I only gave my
wrist a bit of a bang. Thanks, comrade!''

\switchcolumn

``我没事,''她说得很快,``就是手腕摔着了。谢谢,同志!''

\switchcolumn*

And with that she walked on in the direction in which she had been
going, as briskly as though it had really been nothing. The whole
incident could not have taken as much as half a minute. Not to let
one\textquotesingle s feelings appear in one\textquotesingle s face was
a habit that had acquired the status of an instinct, and in any case
they had been standing straight in front of a telescreen when the thing
happened. Nevertheless it had been very difficult not to betray a
momentary surprise, for in the two or three seconds while he was helping
her up the girl had slipped something into his hand. There was no
question that she had done it intentionally. It was something small and
flat. As he passed through the lavatory door he transferred it to his
pocket and felt it with the tips of his fingers. It was a scrap of paper
folded into a square.

\switchcolumn

说完,她就朝着之前的方向走了,她脚步轻盈,好像真的没事儿。整个过程不过半分钟。对温斯顿来说,不让感情呈现在脸上已经是一种本能,况且这事发生时,他们刚好站在电屏前。然而,他还是很难掩饰他的惊异,就在他将她扶起来的两三秒钟里,她迅速将一件东西塞在他手中。毫无疑问,她是故意的。那东西又小又扁。他从厕所门口经过,将它揣进口袋,又用手指摸了摸它。原来是一个折成方形的纸条。

\switchcolumn*

While he stood at the urinal he managed, with a little more fingering,
to get it unfolded. Obviously there must be a message of some kind
written on it. For a moment he was tempted to take it into one of the
water closets and read it at once. But that would be shocking folly, as
he well knew. There was no place where you could be more certain that
the telescreens were watched continuously.

\switchcolumn

他一边上厕所,一边摸捻着将它打开。很明显,里面一定写着某些信息。有那么一瞬,他忍不住要到马桶间里看它。但这太不明智了。正如他所知,没有哪个地方靠得住,电屏无时无刻不在监视人们。

\switchcolumn*

He went back to his cubicle, sat down, threw the fragment of paper
casually among the other papers on the desk, put on his spectacles and
hitched the speakwrite toward him. ``Five minutes,'' he told himself,
``five minutes at the very least!'' His heart bumped in his breast with
frightening loudness. Fortunately the piece of work he was engaged on
was mere routine, the rectification of a long list of figures, not
needing close attention.

\switchcolumn

他回到办公间,坐了下来,将这张纸随随便便地往桌上一放,放到了桌上的一堆纸中。他戴上眼镜,拉出语音记录器,``五分钟,''他对自己说,``至少等五分钟!''他的心怦怦地跳着,发出很大的声音。所幸他的工作只是例行公事,更改一堆数字并不需要耗费太多精力。

\switchcolumn*

Whatever was written on the paper, it must have some kind of political
meaning. So far as he could see there were two possibilities. One, much
the more likely, was that the girl was an agent of the Thought Police,
just as he had feared. He did not know why the Thought Police should
choose to deliver their messages in such a fashion, but perhaps they had
their reasons. The thing that was written on the paper might be a
threat, a summons, an order to commit suicide, a trap of some
description. But there was another, wilder possibility that kept raising
its head, though he tried vainly to suppress it. This was, that the
message did not come from the Thought Police at all, but from some kind
of underground organization. Perhaps the Brotherhood existed after all!
Perhaps the girl was part of it! No doubt the idea was absurd, but it
had sprung into his mind in the very instant of feeling the scrap of
paper in his hand. It was not till a couple of minutes later that the
other, more probable explanation had occurred to him. And even now,
though his intellect told him that the message probably meant
death---still, that was not what he believed, and the unreasonable hope
persisted, and his heart banged, and it was with difficulty that he kept
his voice from trembling as he murmured his figures into the speakwrite.

\switchcolumn

不管怎样,纸条上写的肯定和政治有关。他估计有两种可能,一种的可能性较大,即像他担心的那样,那姑娘真是思想警察。他想不通思想警察为什么要用这种方式传递信息。但也许他们有他们的原因。这纸片也许是威胁,也许是传票,也许是要他自杀的命令,也许是个圈套。而另一种可能虽然荒诞不经却总出现在他的大脑中,他试图将它压下去却徒劳无功。那就是,纸条根本不是思想警察送来的,它来自某个地下组织。或许真的有兄弟会!那女孩就是其中一员!毋庸置疑,这个想法的确荒唐,但纸条一触碰到他,他就萌生了这个念头。直到几分钟后,他才想到了更合理的解释。即使现在,理智告诉他这个信息也许正意味着死亡——他仍然对那个不合理的解释怀抱希望。他的心剧烈地跳着,在对语音记录器叙述数字时,他好容易才控制住自己不让声音颤抖。

\switchcolumn*

He rolled up the completed bundle of work and slid it into the pneumatic
tube. Eight minutes had gone by. He readjusted his spectacles on his
nose, sighed, and drew the next batch of work toward him, with the scrap
of paper on top of it. He flattened it out. On it was written, in a
large unformed handwriting:

\switchcolumn

他将已经完成的工作纸卷起来放进输送管。八分钟过去了。他把眼镜扶正,叹了口气,然后把另一堆工作材料拉到面前。那张纸就在上面,他将它铺平,上面歪歪扭扭地写着几个大字:

\switchcolumn*

\begin{quotation}
I love you.
\end{quotation}

\switchcolumn

\begin{quotation}
我爱你
\end{quotation}

\switchcolumn*


For several seconds he was too stunned even to throw the incriminating
thing into the memory hole. When he did so, although he knew very well
the danger of showing too much interest, he could not resist reading it
once again, just to make sure that the words were really there.

\switchcolumn

他太吃惊了以至于忘记将这个称得上定罪证据的东西扔到记忆洞里。尽管他非常清楚,
表现出太多兴趣相当危险,但在将它扔进记忆洞前,他还是忍不住又看了一遍,想确定
上面是不是真的写着那几个字。

\switchcolumn*

For the rest of the morning it was very difficult to work. What was even
worse than having to focus his mind on a series of niggling jobs was the
need to conceal his agitation from the telescreen. He felt as though a
fire were burning in his belly. Lunch in the hot, crowded, noise-filled
canteen was torment. He had hoped to be alone for a little while during
the lunch hour, but as bad luck would have it the imbecile Parsons
flopped down beside him, the tang of his sweat almost defeating the
tinny smell of stew, and kept up a stream of talk about the preparations
for Hate Week. He was particularly enthusiastic about a papier-mâché
model of Big Brother\textquotesingle s head, two meters wide, which was
being made for the occasion by his daughter\textquotesingle s troop of
Spies. The irritating thing was that in the racket of voices Winston
could hardly hear what Parsons was saying, and was constantly having to
ask for some fatuous remark to be repeated. Just once he caught a
glimpse of the girl, at a table with two other girls at the far end of
the room. She appeared not to have seen him, and he did not look in that
direction again.

\switchcolumn

上午剩下的时间,他已无心工作。要将精力集中在琐碎的工作上本已不易,在电屏前隐
藏自己的情绪就更加困难。他觉得肚子里有火在烧。在吵闹闷热,拥挤得像罐头一样的
食堂里就餐异常痛苦。他原打算吃午饭时一个人待会儿,但他的运气太差了,笨蛋帕森
斯跑过来坐到他身旁,身上的汗味几乎将炖菜的铁皮味盖过,不仅如此,他还喋喋不休
地说着仇恨周的筹备情况,他对女儿用硬纸板做了两米多宽的老大哥头部模型格外兴奋,
这正是他女儿所在的侦察队为仇恨周准备的。让人烦躁的是,在喧闹的人声中,温斯顿
几乎听不见帕森斯讲些什么,他不得不一再要求他重复那些蠢话。他只看见那女孩一次,
她和其他两个女孩坐在食堂的另一端。她们似乎没有看到他,他也不再向她们张望。

\switchcolumn*

The afternoon was more bearable. Immediately after lunch there arrived a
delicate, difficult piece of work which would take several hours and
necessitated putting everything else aside. It consisted in falsifying a
series of production reports of two years ago in such a way as to cast
discredit on a prominent member of the Inner Party who was now under a
cloud. This was the kind of thing that Winston was good at, and for more
than two hours he succeeded in shutting the girl out of his mind
altogether. Then the memory of her face came back, and with it a raging,
intolerable desire to be alone. Until he could be alone it was
impossible to think this new development out. Tonight was one of his
nights at the Community Center. He wolfed another tasteless meal in the
canteen, hurried off to the Center, took part in the solemn foolery of a
``discussion group,'' played two games of table tennis, swallowed several
glasses of gin, and sat for half an hour through a lecture entitled
``Ingsoc in relation to chess.'' His soul writhed with boredom, but for
once he had had no impulse to shirk his evening at the Center. At the
sight of the words \emph{I love you} the desire to stay alive had welled
up in him, and the taking of minor risks suddenly seemed stupid. It was
not till twenty-three hours, when he was home and in bed---in the
darkness, where you were safe even from the telescreen so long as you
kept silent---that he was able to think continuously.

\switchcolumn

下午要好过些。午饭刚过,他就收到了一项复杂的工作,要好几个小时才能完成,他不得不将其他事情暂时搁置。他要修改两年前的一批生产报告,以损害某个受到怀疑的内党要人的名誉。这是温斯顿最擅长的事,在两个多小时的时间里,他都将那女孩置之脑后。但很快,她的面容就又浮现在他的脑中,引起了无法按捺的强烈欲望,他很想单独待上一会儿。他必须独自待着才能将事情理出头绪。今晚他又要到活动中心去,他匆匆忙忙地在食堂里吃过无味的晚餐,然后赶往活动中心参加看似一本正经、实则愚蠢不堪的讨论组会议,他打了两场乒乓球,吞下几杯杜松子酒,又听了半个小时《英社与象棋关系》的讲座。尽管烦得要命,但这是他第一次没有离开的冲动。看到``我爱你''后,他心里充满生存的渴望,为小事冒险愚不可及。直到晚上11点回到家,躺在了床上,他才开始好好思考问题。黑暗中,只要默然不语,就能躲开电屏的监视。

\switchcolumn*

It was a physical problem that had to be solved: how to get in touch
with the girl and arrange a meeting. He did not consider any longer the
possibility that she might be laying some kind of trap for him. He knew
that it was not so, because of her unmistakable agitation when she
handed him the note. Obviously she had been frightened out of her wits,
as well she might be. Nor did the idea of refusing her advances even
cross his mind. Only five nights ago he had contemplated smashing her
skull in with a cobblestone; but that was of no importance. He thought
of her naked, youthful body, as he had seen it in his dream. He had
imagined her a fool like all the rest of them, her head stuffed with
lies and hatred, her belly full of ice. A kind of fever seized him at
the thought that he might lose her, the white youthful body might slip
away from him! What he feared more than anything else was that she would
simply change her mind if he did not get in touch with her quickly. But
the physical difficulty of meeting was enormous. It was like trying to
make a move at chess when you were already mated. Whichever way you
turned, the telescreen faced you. Actually, all the possible ways of
communicating with her had occurred to him within five minutes of
reading the note; but now, with time to think, he went over them one by
one, as though laying out a row of instruments on a table.

\switchcolumn

他要解决一个实际问题:如何和那姑娘保持联系、进行见面。他不再觉得她有设置陷阱,这不可能。当她递给他纸条时,她无疑情绪激动。显然,她吓坏了。他没想过拒绝她的示好。而就在五天前,他还想用石头砸烂她的脑袋。但这没关系,他想象着她赤裸年轻的肉体,一如梦中情景。他原以为她和别人一样脑袋里装满谎言和仇恨,肚子里一副铁石心肠。只要一想到有可能会失去她,他就一阵恐慌,那白皙的肉体很可能会从他手中溜走!而他最担心的,若不能马上联系到她,她也许会改变主意。只是安排见面困难重重。就好比在下象棋时,你已然被将死却仍想再走一步。无论面朝何方,都有电屏对着你。事实上看到那张纸条的五分钟内,他就想尽了所有办法。趁现在还有思考时间,他又一个一个地将它们检查了一遍,就好像把所有工具都摊在桌子上排成一排。

\switchcolumn*

Obviously the kind of encounter that had happened this morning could not
be repeated. If she had worked in the Records Department it might have
been comparatively simple, but he had only a very dim idea whereabouts
in the building the Fiction Department lay, and he had no pretext for
going there. If he had known where she lived, and at what time she left
work, he could have contrived to meet her somewhere on her way home; but
to try to follow her home was not safe, because it would mean loitering
about outside the Ministry, which was bound to be noticed. As for
sending a letter through the mails, it was out of the question. By a
routine that was not even secret, all letters were opened in transit.
Actually, few people ever wrote letters. For the messages that it was
occasionally necessary to send, there were printed postcards with long
lists of phrases, and you struck out the ones that were inapplicable. In
any case he did not know the girl\textquotesingle s name, let alone her
address. Finally he decided that the safest place was the canteen. If he
could get her at a table by herself, somewhere in the middle of the
room, not too near the telescreens, and with a sufficient buzz of
conversation all round---if these conditions endured for, say, thirty
seconds, it might be possible to exchange a few words.

\switchcolumn

显然,今天上午的相遇无法再重来一遍。若她在记录司工作,事情就简单得多。他对大楼里小说司的分布情况印象模糊,他也没有借口到那里去。若他知道她的居住地点、下班时间,他还能想办法在她回家途中和她相遇。但跟在她身后可不安全,在真理部外面晃来晃去一定会引人注意。至于寄信给她,则完全办不到。因为所有信件在邮递时都会被拆开察看已不是秘密。事实上只有很少人还在写信。若必须传递什么消息,人们就用印有文字的明信片,只要将不合适的话划掉就行了。再说,他不知道女孩的名字,更别说她的地址。最后,他认为最安全的地方就是食堂,若能在她独自一人时坐到她的桌旁——这张桌子必须在食堂中间,不能离电屏太近,周围还要很嘈杂——所有这些条件都具备了并持续三十秒,他就能和她说上几句。

\switchcolumn*

For a week after this, life was like a restless dream. On the next day
she did not appear in the canteen until he was leaving it, the whistle
having already blown. Presumably she had been changed onto a later
shift. They passed each other without a glance. On the day after that
she was in the canteen at the usual time, but with three other girls and
immediately under a telescreen. Then for three dreadful days she did not
appear at all. His whole mind and body seemed to be afflicted with an
unbearable sensitivity, a sort of transparency, which made every
movement, every sound, every contact, every word that he had to speak or
listen to, an agony. Even in sleep he could not altogether escape from
her image. He did not touch the diary during those days. If there was
any relief, it was in his work, in which he could sometimes forget
himself for ten minutes at a stretch. He had absolutely no clue as to
what had happened to her. There was no inquiry he could make. She might
have been vaporized, she might have committed suicide, she might have
been transferred to the other end of Oceania---worst and likeliest of
all, she might simply have changed her mind and decided to avoid him.

\switchcolumn

此后的一个星期,生活如同令人焦虑的梦。第二天,她不在食堂,直到他要离开,她才现身。哨声响起。她似乎刚刚换了夜班,他们擦肩而过,没有看对方。第三天,她在老时间出现,却有三个女孩和她在一起,还都坐在电屏下。接着,连续三天她都没有来。他的身心备受煎熬,极度敏锐。他的每个举动、发出的每个声音,进行的每个接触,他说的以及听到的每句话,都无法掩饰,这让他痛苦万分。即使在梦中,他也无法逃开,不能不想她的样子。这些天他都没有碰日记,如果说有什么能让他放松一下,那就是工作,有时,他可以忘记自己一连工作十分钟。他不知道她发生了什么事,一点线索都没有。她可能蒸发了,可能自杀了,可能被调到大洋国的另一端——而最糟糕也最有可能的是,她也许只是改变主意,决定避开他。

\switchcolumn*

The next day she reappeared. Her arm was out of the sling and she had a
band of sticking plaster round her wrist. The relief of seeing her was
so great that he could not resist staring directly at her for several
seconds. On the following day he very nearly succeeded in speaking to
her. When he came into the canteen she was sitting at a table well out
from the wall, and was quite alone. It was early, and the place was not
very full. The queue edged forward till Winston was almost at the
counter, then was held up for two minutes because someone in front was
complaining that he had not received his tablet of saccharine. But the
girl was still alone when Winston secured his tray and began to make for
her table. He walked casually toward her, his eyes searching for a place
at some table beyond her. She was perhaps three meters away from him.
Another two seconds would do it. Then a voice behind him called,
``Smith!'' He pretended not to hear. ``Smith!'' repeated the voice, more
loudly. It was no use. He turned round. A blond-headed, silly-faced
young man named Wilsher, whom he barely knew, was inviting him with a
smile to a vacant place at his table. It was not safe to refuse. After
having been recognized, he could not go and sit at a table with an
unattended girl. It was too noticeable. He sat down with a friendly
smile. The silly blond face beamed into his. Winston had a hallucination
of himself smashing a pickax right into the middle of it. The
girl\textquotesingle s table filled up a few minutes later.

\switchcolumn

第二天,她重新出现,手臂上已没有绷带,但手腕处却贴了膏药。看到她,他非常高兴,忍不住凝视了她好几秒。接下来的一天,他差点就和她说上了话。他走进食堂,她正坐在一张远离墙壁的桌子旁,只有她一人。时间很早,人不是很多。领餐的队伍缓缓移动,温斯顿快要挪到餐台前的时候,排在他前面的一个人突然抱怨没有领到糖精,耽搁了两分钟。好在那女孩仍然独自一人坐在那里。温斯顿领到饭菜,向她走去,一面假装漫不经心,一面打量她周围的桌子,寻找空位。他离她只有三米远了,再过两秒,他就能来到她身边。但就在这时,身后,突然有人喊他的名字``史密斯'',他假装没听见,那人又喊了一句``史密斯'',声音更大了。没用。他转身一看,原来是个金色头发、模样蠢笨的年轻人。他叫威舍尔,温斯顿对他并不熟悉。他面带微笑看着温斯顿,邀请他坐在他旁边的空位上。拒绝是不安全的。在被人认出后,他不能单独和那个女孩坐在一起,否则就太引人注目了。因此,他带着友善的笑容坐下来。那个愚蠢的金发男孩也对他笑了笑。温斯顿恨不得用十字镐将他一劈为二。几分钟后,女孩所在的桌子旁也坐满了人。

\switchcolumn*

But she must have seen him coming toward her, and perhaps she would take
the hint. Next day he took care to arrive early. Sure enough, she was at
a table in about the same place, and again alone. The person immediately
ahead of him in the queue was a small, swiftly moving, beetlelike man
with a flat face and tiny, suspicious eyes. As Winston turned away from
the counter with his tray, he saw that the little man was making
straight for the girl\textquotesingle s table. His hopes sank again.
There was a vacant place at a table further away, but something in the
little man\textquotesingle s appearance suggested that he would be
sufficiently attentive to his own comfort to choose the emptiest table.
With ice at his heart Winston followed. It was no use unless he could
get the girl alone. At this moment there was a tremendous crash. The
little man was sprawling on all fours, his tray had gone flying, two
streams of soup and coffee were flowing across the floor. He started to
his feet with a malignant glance at Winston, whom he evidently suspected
of having tripped him up. But it was all right. Five seconds later, with
a thundering heart, Winston was sitting at the girl\textquotesingle s
table.

\switchcolumn

但她肯定看到他向她走去,也许她能明白这个暗示。第二天,他早早来到食堂。果然,她就坐在几乎相同的位置,又是独自一人。这次排在温斯顿前面的是个身材矮小,动作迅速,长得像甲虫一样的男人。男人的脸很扁,细小的眼睛里充满怀疑。离开餐台时,温斯顿看到这个矮个子男人正径直向那女孩走去。他的希望再次落空。稍远些的地方还有空位,但从那男人的神情看,为了让自己舒服,他一定会选择人最少的桌子。温斯顿的心凉了下来。没用,除非他能和那女孩独处。而这时,就听``嚓''一声,矮个子男人四脚朝天摔倒了,托盘飞了出去,汤和咖啡流了一地。他爬起来,恶狠狠地瞪了温斯顿一眼,他怀疑温斯顿故意将他绊倒。但这无关紧要。五秒钟后,温斯顿心跳剧烈地坐到了女孩旁边。

\switchcolumn*

He did not look at her. He unpacked his tray and promptly began eating.
It was all-important to speak at once, before anyone else came, but now
a terrible fear had taken possession of him. A week had gone by since
she had first approached. She would have changed her mind, she must have
changed her mind! It was impossible that this affair should end
successfully; such things did not happen in real life. He might have
flinched altogether from speaking if at this moment he had not seen
Ampleforth, the hairy-eared poet, wandering limply round the room with a
tray, looking for a place to sit down. In his vague way Ampleforth was
attached to Winston, and would certainly sit down at his table if he
caught sight of him. There was perhaps a minute in which to act. Both
Winston and the girl were eating steadily. The stuff they were eating
was a thin stew, actually a soup, of haricot beans. In a low murmur
Winston began speaking. Neither of them looked up; steadily they spooned
the watery stuff into their mouths, and between spoonfuls exchanged the
few necessary words in low expressionless voices.

\switchcolumn

他没看她,他将托盘放好,吃了起来。他要趁其他人到来之前赶快说几句话,这是最重要的,但他偏偏被巨大的恐惧占据。从她初次接近他算起已经一个星期了。她改变主意了,她一定改变主意了!这件事不可能成功,不可能发生在实际生活中。若不是看到安普福斯——就是那个耳朵上长着很多毛的诗人——正端着餐盘走来走去地寻找位置,他很可能会退缩,什么都不说。安普福斯对他隐约有些好感,若他发现他,肯定会坐到他桌旁。也许只有一分钟时间了,要马上开始行动。温斯顿和女孩慢吞吞地吃着,他们吃的炖菜其实就是菜豆汤,稀糊糊的。温斯顿低声说话,俩人都没抬头,不紧不慢地用勺子往嘴里送水拉拉的东西,吃的间隙,他们面无表情的轻声交谈。

\switchcolumn*

``What time do you leave work?''

\switchcolumn

``什么时候下班?''

\switchcolumn*

``Eighteen thirty.''

\switchcolumn

``18点半。''

\switchcolumn*

``Where can we meet?''

\switchcolumn

``我们在哪儿见面?''

\switchcolumn*

``Victory Square, near the monument.''

\switchcolumn

``胜利广场,纪念碑旁边。''

\switchcolumn*

``It\textquotesingle s full of telescreens.''

\switchcolumn

``那儿到处是电屏。''

\switchcolumn*

``It doesn\textquotesingle t matter if there\textquotesingle s a crowd.''

\switchcolumn

``人多就没事。''

\switchcolumn*

``Any signal?''

\switchcolumn

``有暗号吗?''

\switchcolumn*

``No. Don\textquotesingle t come up to me until you see me among a lot of
people. And don\textquotesingle t look at me. Just keep somewhere near
me.''

\switchcolumn

``没有。别靠近我,除非你看到我在很多人中间。也别看我,在我附近就行了。''

\switchcolumn*

``What time?''

\switchcolumn

``什么时间?''

\switchcolumn*

``Nineteen hours.''

\switchcolumn

``19点。''

\switchcolumn*

``All right.''

\switchcolumn

``好。''

\switchcolumn*

Ampleforth failed to see Winston and sat down at another table. The girl
finished her lunch quickly and made off, while Winston stayed to smoke a
cigarette. They did not speak again, and, so far as it was possible for
two people sitting on opposite sides of the same table, they did not
look at one another.

\switchcolumn

安普福斯没看到温斯顿,坐到了另一张桌子旁。他们没有再说话,只要两个人面对面地坐在同一张桌子前,就不能彼此相视。女孩吃完就走了,温斯顿待了一会儿,抽了支烟。

\switchcolumn*

Winston was in Victory Square before the appointed time. He wandered
round the base of the enormous fluted column, at the top of which Big
Brother\textquotesingle s statue gazed southward toward the skies where
he had vanquished the Eurasian airplanes (the Eastasian airplanes, it
had been, a few years ago) in the Battle of Airstrip One. In the street
in front of it there was a statue of a man on horseback which was
supposed to represent Oliver Cromwell. At five minutes past the hour the
girl had still not appeared. Again the terrible fear seized upon
Winston. She was not coming, she had changed her mind! He walked slowly
up to the north side of the square and got a sort of pale-colored
pleasure from identifying St. Martin\textquotesingle s church, whose
bells, when it had bells, had chimed ``You owe me three farthings.'' Then
he saw the girl standing at the base of the monument, reading or
pretending to read a poster which ran spirally up the column. It was not
safe to go near her until some more people had accumulated. There were
telescreens all round the pediment. But at this moment there was a din
of shouting and a zoom of heavy vehicles from somewhere to the left.
Suddenly everyone seemed to be running across the square. The girl
nipped nimbly round the lions at the base of the monument and joined in
the rush. Winston followed. As he ran, he gathered from some shouted
remarks that a convoy of Eurasian prisoners was passing.

\switchcolumn

温斯顿在约定时间之前赶到了胜利广场。他在一个有凹槽的巨型圆柱的基座下走来走去,圆柱顶端的老大哥雕像凝视着南方的天空,他曾在那里,在一号空降场之战中,歼灭欧亚国的飞机(就在几年前,还是东亚国飞机)。这之前的街道上,还有个骑马者的雕像,应该是奥利弗•克伦威尔。约定的时间已经过去五分钟,女孩还没有出现,恐惧抓住了温斯顿。她没来,她改变主意了!他慢慢地踱步到广场北边,认出了圣马丁教堂,一种淡淡的喜悦涌上心头。当它还有钟时,它敲出了``你欠我三个法寻''。之后,他看到了那个女孩,她就站在纪念碑的底座前看,或者说她正假装在看圆柱上的宣传画。她身边的人不多,就这样走过去不大安全,到处都装着电屏。但就在这时,左边的某个地方传来喧哗及重型汽车驶过的声音。突然间,所有人都跑过了广场。女孩轻盈地跳过位于纪念碑底座处的狮子雕像,钻进了人群里。温斯顿跟了过去,在他奔跑时,他从人们的喊叫声中得知,原来装着欧亚国俘虏的车队正在驶过。

\switchcolumn*

Already a dense mass of people was blocking the south side of the
square. Winston, at normal times the kind of person who gravitates to
the outer edge of any kind of scrimmage, shoved, butted, squirmed his
way forward into the heart of the crowd. Soon he was within
arm\textquotesingle s length of the girl, but the way was blocked by an
enormous prole and an almost equally enormous woman, presumably his
wife, who seemed to form an impenetrable wall of flesh. Winston wriggled
himself sideways, and with a violent lunge managed to drive his shoulder
between them. For a moment it felt as though his entrails were being
ground to pulp between the two muscular hips, then he had broken
through, sweating a little. He was next to the girl. They were shoulder
to shoulder, both staring fixedly in front of them.

\switchcolumn

密密麻麻的人群将广场的南边堵住。通常,在这种混乱的场合,温斯顿总会被挤到外面,但这次他却推推搡搡地向人群中挤去。很快,他和那女孩便只有一臂之遥,可偏偏一个大块头和一个女人挡在了他们中间,这女人大概是大块头的妻子,和他一样身材壮硕,他们构成一幢无法逾越的肉墙。温斯顿扭过身,用力一挤,设法将肩膀插在二人中间,有那么一会儿,他被两个肌肉结实的屁股夹住了,感觉自己的五脏六腑都被挤成了肉酱。最后他挤了出来,出了点汗。他来到女孩身边,俩人肩并肩地走着,目光动也不动地注视着前方。

\switchcolumn*

A long line of trucks, with wooden-faced guards armed with submachine
guns standing upright in each corner, was passing slowly down the
street. In the trucks little yellow men in shabby greenish uniforms were
squatting, jammed close together. Their sad Mongolian faces gazed out
over the sides of the trucks, utterly incurious. Occasionally when a
truck jolted there was a clank-clank of metal: all the prisoners were
wearing leg irons. Truckload after truckload of the sad faces passed.
Winston knew they were there, but he saw them only intermittently. The
girl\textquotesingle s shoulder, and her arm right down to the elbow,
were pressed against his. Her cheek was almost near enough for him to
feel its warmth. She had immediately taken charge of the situation, just
as she had done in the canteen. She began speaking in the same
expressionless voice as before, with lips barely moving, a mere murmur
easily drowned by the din of voices and the rumbling of the trucks.

\switchcolumn

一长队卡车缓缓地驶过街道,车上的每个角落都有面无表情、手执机枪、站得笔直的警卫。许多穿着草绿色旧军服的黄种人蹲在车厢里,紧紧地拥在一起。他们用悲伤的蒙古脸从卡车两侧向外张望,一幅漠不关心的样子。所有俘虏都戴着脚镣,不时卡车颠簸,就会听到金属撞击的叮当声。一辆又一辆的卡车载着神情凄恻的俘虏开过,温斯顿知道他们就在卡车里,但他只间或看上一眼。女孩的肩膀,还有她右肘以上的手臂都紧贴着他,她的脸颊和他如此接近,以至于他几乎可以感觉到她的体温。就像在食堂里,她掌握了主动权,用和上次一样听不出情绪的声音讲话,嘴唇几乎不动,如此低低细语很容易就被嘈杂的人声和隆隆的卡车声掩盖。

\switchcolumn*

``Can you hear me?''

\switchcolumn

``听得到我说话吗?''

\switchcolumn*

``Yes.''

\switchcolumn

``听得到。''

\switchcolumn*

``Can you get Sunday afternoon off?''

\switchcolumn

``星期天下午能调休吗?''

\switchcolumn*

``Yes.''

\switchcolumn

``能!''

\switchcolumn*

``Then listen carefully. You\textquotesingle ll have to remember this. Go
to Paddington Station---''

\switchcolumn

``那你听好,记住。到帕丁顿车站——''

\switchcolumn*

With a sort of military precision that astonished him, she outlined the
route that he was to follow. A half-hour railway journey; turn left
outside the station; two kilometers along the road; a gate with the top
bar missing; a path across a field; a grass-grown lane; a track between
bushes; a dead tree with moss on it. It was as though she had a map
inside her head. ``Can you remember all that?'' she murmured finally.

\switchcolumn

她将路线告诉给他,像制定军事计划那样清晰明了,让他大为惊讶。先坐半个小时火车,出车站向左拐,沿路走两公里,然后穿过没有横梁的大门,再穿过田野,经过一条长满荒草的小径和一条灌木丛里的小路,到了那儿会看到一棵长满青苔的死树。她的脑袋里好像有一张地图,最后,她低声地问:``都记住了吗?''

\switchcolumn*

``Yes.''

\switchcolumn

``是的。''

\switchcolumn*

``You turn left, then right, then left again. And the
gate\textquotesingle s got no top bar.''

\switchcolumn

``先向左,再向右,最后再向左。大门上面没有横梁。''

\switchcolumn*

``Yes. What time?''

\switchcolumn

``好的,什么时间?''

\switchcolumn*

``About fifteen. You may have to wait. I\textquotesingle ll get there by
another way. Are you sure you remember everything?''

\switchcolumn

``大约15点。你可能要等一会儿。我会从另外一条路赶到那里。都记下了?''

\switchcolumn*

``Yes.''

\switchcolumn

``是的。''

\switchcolumn*

``Then get away from me as quick as you can.''

\switchcolumn

``那,尽快离开我吧。''

\switchcolumn*

She need not have told him that. But for the moment they could not
extricate themselves from the crowd. The trucks were still filing past,
the people still insatiably gaping. At the start there had been a few
boos and hisses, but it came only from the Party members among the
crowd, and had soon stopped. The prevailing emotion was simply
curiosity. Foreigners, whether from Eurasia or from Eastasia, were a
kind of strange animal. One literally never saw them except in the guise
of prisoners, and even as prisoners one never got more than a momentary
glimpse of them. Nor did one know what became of them, apart from the
few who were hanged as war criminals; the others simply vanished,
presumably into forced-labor camps. The round Mongol faces had given way
to faces of a more European type, dirty, bearded, and exhausted. From
over scrubby cheekbones eyes looked into Winston\textquotesingle s,
sometimes with strange intensity, and flashed away again. The convoy was
drawing to an end. In the last truck he could see an aged man, his face
a mass of grizzled hair, standing upright with wrists crossed in front
of him, as though he were used to having them bound together. It was
almost time for Winston and the girl to part. But at the last moment,
while the crowd still hemmed them in, her hand felt for his and gave it
a fleeting squeeze.

\switchcolumn

她没必要和他说这个,但他们一时半会儿无法从人群中脱身。卡车仍在经过,人们仍在不知足地观看。人群中传来零星的嘘声,但那只是党员发出的,很快就停止了。对观看的人来说,好奇的情绪占了大部分。外国人,不管来自欧亚国还是东亚国,都是陌生的动物。除了俘虏,人们很少能看到,而就算是俘虏,也只能匆匆一瞥,况且,人们不知道这些俘虏的下场会怎样。他们中的一小部分会被当做战犯,被吊死,其他的就消失了,可能被送到劳动营里当苦力。在圆圆的蒙古脸之后,是类似欧洲人的肮脏憔悴、长满胡须的脸。他们的眼睛从长满胡楂儿的颧骨上方看着温斯顿,有时目光专注,但又很快闪过。车队过完了。在最后一辆卡车上,温斯顿看到一个上了年纪的人,他笔直地站着,双手交叉放在身前,好像习惯了它们被绑在一起,他斑白的头发披散着,挡住了脸。就要和女孩分手了。但在最后一刻,人群仍紧紧地包围着他们,她摸到了他的手,握了一会儿。

\switchcolumn*

It could not have been ten seconds, and yet it seemed a long time that
their hands were clasped together. He had time to learn every detail of
her hand. He explored the long fingers, the shapely nails, the
work-hardened palm with its row of calluses, the smooth flesh under the
wrist. Merely from feeling it he would have known it by sight. In the
same instant it occurred to him that he did not know what color the
girl\textquotesingle s eyes were. They were probably brown, but people
with dark hair sometimes had blue eyes. To turn his head and look at her
would have been inconceivable folly. With hands locked together,
invisible among the press of bodies, they stared steadily in front of
them, and instead of the eyes of the girl, the eyes of the aged prisoner
gazed mournfully at Winston out of nests of hair.

\switchcolumn

虽然不可能超过十秒,却好像握了很久。他有足够的时间去熟悉她手上的每个细节。他摸索着她长长的手指,椭圆形的指甲,长着茧子的掌心以及光滑的手腕。摸着它,好像眼睛也看到了。他想起来,他还不知道她眼睛的颜色,可能是棕色,不过黑头发的人有时也长着蓝色的眼睛。回头看她是愚蠢的。他们平静地望着前方,十指相扣,在拥挤的人群中不会被发现。代替那女孩注视他的,是那上了年纪的俘虏,他的眼睛在乱蓬蓬的头发后悲伤地看着他。

\switchcolumn*


\section{II}\label{ii-1}

\switchcolumn

\section*{二}\label{ux5341}

\switchcolumn*

Winston picked his way up the lane through dappled light and shade,
stepping out into pools of gold wherever the boughs parted. Under the
trees to the left of them the ground was misty with bluebells. The air
seemed to kiss one\textquotesingle s skin. It was the second of May.
From somewhere deeper in the heart of the wood came the droning of ring
doves.

\switchcolumn

温斯顿谨慎地穿过树影斑驳的小路,阳光从树杈间洒下,在地上形成金黄色的洼。他左边的树林下开满了迷蒙的蓝铃花。微风轻吻着人的皮肤,这是五月的第二天,树林深处,传来斑鸠的咕咕声。

\switchcolumn*

He was a bit early. There had been no difficulties about the journey,
and the girl was so evidently experienced that he was less frightened
than he would normally have been. Presumably she could be trusted to
find a safe place. In general you could not assume that you were much
safer in the country than in London. There were no telescreens, of
course, but there was always the danger of concealed microphones by
which your voice might be picked up and recognized; besides, it was not
easy to make a journey by yourself without attracting attention. For
distances of less than a hundred kilometers it was not necessary to get
your passport endorsed, but sometimes there were patrols hanging about
the railway stations, who examined the papers of any Party member they
found there and asked awkward questions. However, no patrols had
appeared, and on the walk from the station he had made sure by cautious
backward glances that he was not being followed. The train was full of
proles, in holiday mood because of the summery weather. The
wooden-seated carriage in which he traveled was filled to overflowing by
a single enormous family, ranging from a toothless great-grandmother to
a month-old baby, going out to spend an afternoon with ``in-laws'' in the
country, and, as they freely explained to Winston, to get hold of a
little black-market butter.

\switchcolumn

他来得有些早了。路上很顺利,那女孩显然经验丰富,这让他不用像平时那样担惊受怕。至于寻找安全的地方,她也许值得信赖。人们不能想当然地以为乡下一定比伦敦安全。乡下虽没有电屏,仍十分危险,不知道什么时候,你说的话就会被隐藏起来的窃听器记录辨认。不仅如此,独自出门的人还很难不被注意。一百公里内尚不需要在通行证上签注,但有时火车站旁的巡逻队会检查每一个过路党员的证件,还会问一些令人难堪的问题。然而那天巡逻队并没有出现。离开车站后,温斯顿小心翼翼地回头张望,确定没有人跟踪。天气温暖,火车里坐满了无产者,每个人都兴高采烈。他所搭的硬座车厢里,从掉光牙的老太太到刚满月的婴儿,挤满了一大家子。他们坦率地告诉他,他们要到乡下走亲戚,顺便弄些黑市黄油。

\switchcolumn*

The lane widened, and in a minute he came to the foot-path she had told
him of, a mere cattle track which plunged between the bushes. He had no
watch, but it could not be fifteen yet. The bluebells were so thick
underfoot that it was impossible not to tread on them. He knelt down and
began picking some, partly to pass the time away, but also from a vague
idea that he would like to have a bunch of flowers to offer to the girl
when they met. He had got together a big bunch and was smelling their
faint sickly scent when a sound at his back froze him, the unmistakable
crackle of a foot on twigs. He went on picking bluebells. It was the
best thing to do. It might be the girl, or he might have been followed
after all. To look round was to show guilt. He picked another and
another. A hand fell lightly on his shoulder.

\switchcolumn

路逐渐开阔起来,很快他就来到女孩所说的小路上,这条被牛群踩出来的小路就藏在灌木丛里。他没有手表,但他知道还不到15点。他的脚下到处是蓝铃花,想不踩到都不可能。他蹲下来摘了一些,一方面为了打发时间,一方面他还有个模模糊糊的想法,想在和女孩见面时送给她。他摘了一大束,闻了闻那并不美妙的花香。突然,背后传来脚步声,什么人踩在了树枝上,这声响吓得他浑身僵硬,只好继续摘花。这是最好的做法。也许是那女孩,也许是跟踪他的人,回头看就意味着做贼心虚。他一朵接一朵地摘着,一只手轻轻地落在了他的肩上。

\switchcolumn*

He looked up. It was the girl. She shook her head, evidently as a
warning that he must keep silent, then parted the bushes and quickly led
the way along the narrow track into the wood. Obviously she had been
that way before, for she dodged the boggy bits as though by habit.
Winston followed, still clasping his bunch of flowers. His first feeling
was relief, but as he watched the strong slender body moving in front of
him, with the scarlet sash that was just tight enough to bring out the
curve of her hips, the sense of his own inferiority was heavy upon him.
Even now it seemed quite likely that when she turned round and looked at
him she would draw back after all. The sweetness of the air and the
greenness of the leaves daunted him. Already, on the walk from the
station, the May sunshine had made him feel dirty and etiolated, a
creature of indoors, with the sooty dust of London in the pores of his
skin. It occurred to him that till now she had probably never seen him
in broad daylight in the open. They came to the fallen tree that she had
spoken of. The girl hopped over and forced apart the bushes, in which
there did not seem to be an opening. When Winston followed her, he found
that they were in a natural clearing, a tiny grassy knoll surrounded by
tall saplings that shut it in completely. The girl stopped and turned.

\switchcolumn

他抬起头,原来是那女孩。她摇摇头,提醒他别出声。之后,她拨开树丛,领着他沿小径向树林深处走去。她一定来过这里,她娴熟地避开那些泥坑,就好像习惯了一样。跟在后面的温斯顿仍握着那束花,起初他很放松,但当他看到她健壮苗条的身材,看到她红色腰带勾勒出的曼妙的臀部曲线,他自惭形秽。这感觉非常沉重。即使是现在,若她转身看他,她仍有可能完全退缩。甜美的风和绿油油的树叶都让他气馁。从火车站出来,五月的阳光让他觉得自己肮脏不堪,浑身苍白。他是生活在室内的人,身上的每个毛孔都塞满了伦敦的煤尘。他想,截至现在她可能从未在阳光之下见过他。他们来到她说的那棵枯树旁,女孩跃过树干,拨开灌木丛,看不出那儿有什么入口。温斯顿跟着她走进去,发现一片自然形成的空场,高高的小树就矗立在长满芳草的土墩旁,它们密密麻麻地将空场遮了起来。

\switchcolumn*

``Here we are,'' she said.

\switchcolumn

女孩停住脚步,转过身说:``我们到了。''

\switchcolumn*

He was facing her at several paces\textquotesingle{} distance. As yet he
did not dare move nearer to her.

\switchcolumn

他正对着她,和她只有几步之遥,却不敢靠近。

\switchcolumn*

``I didn\textquotesingle t want to say anything in the lane,'' she went
on, ``in case there\textquotesingle s a mike hidden there. I
don\textquotesingle t suppose there is, but there could be.
There\textquotesingle s always the chance of one of those swine
recognizing your voice. We\textquotesingle re all right here.''

\switchcolumn

``路上,我不想说话,''她说,``万一哪个地方藏着话筒。我觉得不会,可的确有这个可能。那些猪难免有哪个能认出你的声音。我们在这里就没事了。''

\switchcolumn*

He still had not the courage to approach her. ``We\textquotesingle re all
right here?'' he repeated stupidly.

\switchcolumn

他仍然没有胆量靠近她,只傻乎乎地重复着:``这里就没事了?''

\switchcolumn*

``Yes. Look at the trees.'' They were small ashes, which at some time had
been cut down and had sprouted up again into a forest of poles, none of
them thicker than one\textquotesingle s wrist. ``There\textquotesingle s
nothing big enough to hide a mike in. Besides, I\textquotesingle ve been
here before.''

\switchcolumn

``对,看这些树。''那是一些还未长大的白蜡树,它们曾被人砍掉,可它们又重新生长起来,长成一片树林。它们的枝干很细,都没有手腕粗。``这些树不够大,藏不起话筒。再说,我曾经来过这里。''

\switchcolumn*

They were only making conversation. He had managed to move closer to her
now. She stood before him very upright, with a smile on her face that
looked faintly ironical, as though she were wondering why he was so slow
to act. The bluebells had cascaded on to the ground. They seemed to have
fallen of their own accord. He took her hand.

\switchcolumn

他们漫无目的地闲聊着。他靠近她,她直直地站在他眼前,脸上带着一丝嘲讽式的微笑,似乎不明白为什么他的行动如此迟缓。他手里的蓝铃花散落了一地,就好像是它们自己掉下来的。他抓住她的手。

\switchcolumn*

``Would you believe,'' he said, ``that till this moment I
didn\textquotesingle t know what color your eyes were?'' They were brown,
he noted, a rather light shade of brown, with dark lashes. ``Now that
you\textquotesingle ve seen what I\textquotesingle m really like, can
you still bear to look at me?''

\switchcolumn

``你相信吗?''他说,``到现在我还不知道你的眼睛是什么颜色。''棕色的,他发现,它们是淡淡的棕色,附着浓黑的睫毛。``现在,你看清了我真实的样子,你受得了一直看着我吗?''

\switchcolumn*

``Yes, easily.''

\switchcolumn

``能,这没什么难的。''

\switchcolumn*

``I\textquotesingle m thirty-nine years old. I\textquotesingle ve got a
wife that I can\textquotesingle t get rid of. I\textquotesingle ve got
varicose veins. I\textquotesingle ve got five false teeth.''

\switchcolumn

``我三十九岁了,有妻子,我不能摆脱她。我有静脉曲张,还有五颗假牙。''

\switchcolumn*

``I couldn\textquotesingle t care less,'' said the girl.

\switchcolumn

``我不在乎。''女孩说。

\switchcolumn*

The next moment, it was hard to say by whose act, she was in his arms.
At the beginning he had no feeling except sheer incredulity. The
youthful body was strained against his own, the mass of dark hair was
against his face, and yes! actually she had turned her face up and he
was kissing the wide red mouth. She had clasped her arms about his neck,
she was calling him darling, precious one, loved one. He had pulled her
down on to the ground, she was utterly unresisting, he could do what he
liked with her. But the truth was that he had no physical sensation
except that of mere contact. All he felt was incredulity and pride. He
was glad that this was happening, but he had no physical desire. It was
too soon, her youth and prettiness had frightened him, he was too much
used to living without women---he did not know the reason. The girl
picked herself up and pulled a bluebell out of her hair. She sat against
him, putting her arm round his waist.

\switchcolumn

接着,说不清是谁主动,她倒在了他的怀里。一开始除了不敢相信,他什么感觉也没有。她年轻的身体紧紧地依偎着他,她乌黑的头发就贴在他的脸上,太美妙了!她扬起了脸,他吻了那微张的红唇。她搂住了他的脖子,轻轻地叫他亲爱的、宝贝、爱人。他把她拉到地上,她没有抗拒,他可以对她做任何他想做的事。不过,事实上对温斯顿来说,他并没有感受到肉体的刺激,除了单纯的触碰,更多的是骄傲和惊讶。对这件事他很高兴,但他却没有肉体的欲望,所有的一切都发生得太快了。她年轻、美丽,让他害怕,他已经太久没和女人生活在一起。不知什么原因,女孩站了起来,摘下头发上的蓝铃花。她靠着他坐着,伸手环过他的腰。

\switchcolumn*

``Never mind, dear. There\textquotesingle s no hurry.
We\textquotesingle ve got the whole afternoon. Isn\textquotesingle t
this a splendid hide-out? I found it when I got lost once on a community
hike. If anyone was coming you could hear them a hundred meters away.''

\switchcolumn

``没关系,亲爱的,别急。我们有整整一个下午的时间。这是个很棒的藏身之所,是不是?我是在一次集体远足中发现的,当时我迷了路。如果有人过来,隔着一百米就能听见。''

\switchcolumn*

``What is your name?'' said Winston.

\switchcolumn

``你叫什么?''温斯顿问。

\switchcolumn*

``Julia. I know yours. It\textquotesingle s Winston---Winston Smith.''

\switchcolumn

``朱莉亚。我知道你叫什么。温斯顿——温斯顿•史密斯。''

\switchcolumn*

``How did you find that out?''

\switchcolumn

``你怎么知道的?''

\switchcolumn*

``I expect I\textquotesingle m better at finding things out than you are,
dear. Tell me, what did you think of me before that day I gave you the
note?''

\switchcolumn

``我想,我比你更擅长调查事情。亲爱的,告诉我,在我把纸条交给你前,你觉得我怎么样?''

\switchcolumn*

He did not feel any temptation to tell lies to her. It was even a sort
of love offering to start off by telling the worst.

\switchcolumn

他一点儿都不想对她撒谎,一开始就把最糟糕的告诉她,也是一种爱的表现。

\switchcolumn*

``I hated the sight of you,'' he said. ``I wanted to rape you and then
murder you afterwards. Two weeks ago I thought seriously of smashing
your head in with a cobblestone. If you really want to know, I imagined
that you had something to do with the Thought Police.''

\switchcolumn

``看到你就觉得讨厌,''他说,``我曾想把你先奸后杀。就在两个星期前,我还想用石头砸烂你的脑袋。如果你真的想知道,我以为你和思想警察有什么联系。''

\switchcolumn*

The girl laughed delightedly, evidently taking this as a tribute to the
excellence of her disguise.

\switchcolumn

女孩开心地笑了,显然,她把这当成了对她伪装技巧的肯定。

\switchcolumn*

``Not the Thought Police! You didn\textquotesingle t honestly think
that?''

\switchcolumn

``思想警察!你真是这么想的?''

\switchcolumn*

``Well, perhaps not exactly that. But from your general
appearance---merely because you\textquotesingle re young and fresh and
healthy, you understand---I thought that probably---''

\switchcolumn

``嗯,不完全是。但是从你的外表看,因为你年轻、有活力、又健康,我想,也许——''

\switchcolumn*

``You thought I was a good Party member. Pure in word and deed. Banners,
processions, slogans, games, community hikes---all that stuff. And you
thought that if I had a quarter of a chance I\textquotesingle d denounce
you as a thought-criminal and get you killed off?''

\switchcolumn

``你觉得我是个好党员。语言和行为都很纯洁,旗帜、游行、标语、比赛、集体远足——总是这些事。你以为,只要有机会,我就会揭发你,说你是思想犯,把你杀了?''

\switchcolumn*

``Yes, something of that kind. A great many young girls are like that,
you know.''

\switchcolumn

``对,差不多就是这个。很多年轻女孩都是这样,你知道的。''

\switchcolumn*

``It\textquotesingle s this bloody thing that does it,'' she said, ripping
off the scarlet sash of the Junior Anti-Sex League and flinging it onto
a bough. Then, as though touching her waist had reminded her of
something, she felt in the pocket of her overalls and produced a small
slab of chocolate. She broke it in half and gave one of the pieces to
Winston. Even before he had taken it he knew by the smell that it was
very unusual chocolate. It was dark and shiny, and was wrapped in silver
paper. Chocolate normally was dull-brown crumbly stuff that tasted, as
nearly as one could describe it, like the smoke of a rubbish fire. But
at some time or another he had tasted chocolate like the piece she had
given him. The first whiff of its scent had stirred up some memory which
he could not pin down, but which was powerful and troubling.

\switchcolumn

``都怪这东西,''她一边说着,一边将青少年反性同盟的红色腰带扯了下来,扔到了树枝上。她的手碰到了自己的腰,这似乎让她想起什么。她从外衣口袋里拿出一小块巧克力,一分为二,将其中一块递给温斯顿。他还没有吃,就从香味上知道这巧克力很不常见。它黑得发亮,包在银纸里。而通常的巧克力都是深棕色,就像人们描绘的那样,那味道宛若烧垃圾时冒出的烟。不过,他吃过她给的这种巧克力。第一次闻到它的香味,他就隐约想起某种让人不安的、感觉强烈的记忆。

\switchcolumn*

``Where did you get this stuff?'' he said.

\switchcolumn

``你从哪儿弄到它的?''他问。

\switchcolumn*

``Black market,'' she said indifferently. ``Actually I am that sort of
girl, to look at. I\textquotesingle m good at games. I was a troop
leader in the Spies. I do voluntary work three evenings a week for the
Junior Anti-Sex League. Hours and hours I\textquotesingle ve spent
pasting their bloody rot all over London. I always carry one end of a
banner in the processions. I always look cheerful and I never shirk
anything. Always yell with the crowd, that\textquotesingle s what I say.
It\textquotesingle s the only way to be safe.''

\switchcolumn

``黑市,''她淡淡地说,``我是那种女孩:我擅长比赛,当过侦察队的中队长,每星期都有三个晚上为青少年反性同盟做义工,还在伦敦城里张贴他们胡说八道的宣传品,每次游行我都会举起横幅。我看上去总是很快乐,做什么事都不会退缩,永远和大家一起呼喊口号。这就是我要说的,这是保护自己的唯一途径。''

\switchcolumn*

The first fragment of chocolate had melted on Winston\textquotesingle s
tongue. The taste was delightful. But there was still that memory moving
round the edges of his consciousness, something strongly felt but not
reducible to definite shape, like an object seen out of the corner of
one\textquotesingle s eye. He pushed it away from him, aware only that
it was the memory of some action which he would have liked to undo but
could not.

\switchcolumn

在温斯顿的舌头上,一小片巧克力已经溶化,味道很棒。但是它唤起的记忆仍徘徊在他意识的边缘。他能强烈地感觉到它,但他又无法确定它的样子,这感觉就类似用眼角余光看到的东西。他将它搁置一边,只知道这是件让他无限后悔又无力挽回的事。

\switchcolumn*

``You are very young,'' he said. ``You are ten or fifteen years younger
than I am. What could you see to attract you in a man like me?''

\switchcolumn

``你很年轻,''他说,``你比我小了十多岁。是什么让你看上了我这样的人?''

\switchcolumn*

``It was something in your face. I thought I\textquotesingle d take a
chance. I\textquotesingle m good at spotting people who
don\textquotesingle t belong. As soon as I saw you I knew you were
against \emph{them}.''

\switchcolumn

``你的脸上有吸引我的东西。我想我要冒下险,我很擅长发现哪些人不属于他们。我一看到你,就知道你抗拒他们。''

\switchcolumn*

\emph{Them}, it appeared, meant the Party, and above all the Inner
Party, about whom she talked with an open jeering hatred which made
Winston feel uneasy, although he knew that they were safe here if they
could be safe anywhere. A thing that astonished him about her was the
coarseness of her language. Party members were supposed not to swear,
and Winston himself very seldom did swear, aloud, at any rate. Julia,
however, seemed unable to mention the Party, and especially the Inner
Party, without using the kind of words that you saw chalked up in
dripping alleyways. He did not dislike it. It was merely one symptom of
her revolt against the Party and all its ways, and somehow it seemed
natural and healthy, like the sneeze of a horse that smells bad hay.
They had left the clearing and were wandering again through the
checkered shade, with their arms round each other\textquotesingle s
waists whenever it was wide enough to walk two abreast. He noticed how
much softer her waist seemed to feel now that the sash was gone. They
did not speak above a whisper. Outside the clearing, Julia said, it was
better to go quietly. Presently they had reached the edge of the little
wood. She stopped him.

\switchcolumn

他们,她说的似乎是党,特别是内党。她嘲弄他们,并不掩饰对他们的憎恨,尽管温斯
顿知道他们待的地方比任何地方都要安全,但她还是让他不安。他非常惊讶地发现她竟
然满口脏话,按说党员是不能说脏话的,温斯顿自己也很少说,即使说,也不会那么大
声。但朱莉亚一提到党,特别是内党,就一定会用街头巷尾中那种用粉笔写出来的话。
关于这点,他不是不喜欢,这只是她对党以及党的一些做法非常反感的表现,就像马闻
到坏饲料打了喷嚏,又自然又健康。他们离开空场,在斑驳的树影下徜徉,只要小径的
宽度足够两个人并肩而行,他们就会搂住对方的腰。他发现,摘掉了腰带,她的腰软多
了。他们轻声低语,朱利亚说出空场后最好保持安静。很快他们就走到了小树林的边上,
她让他停下。

\switchcolumn*

``Don\textquotesingle t go out into the open. There might be someone
watching. We\textquotesingle re all right if we keep behind the boughs.''

\switchcolumn

``别出去。可能有人在看着呢。我们到树后去就好。''

\switchcolumn*

They were standing in the shade of hazel bushes. The sunlight, filtering
through innumerable leaves, was still hot on their faces. Winston looked
out into the field beyond, and underwent a curious, slow shock of
recognition. He knew it by sight. An old, close-bitten pasture, with a
foot-path wandering across it and a molehill here and there. In the
ragged hedge on the opposite side the boughs of the elm trees swayed
just perceptibly in the breeze, and their leaves stirred faintly in
dense masses like women\textquotesingle s hair. Surely somewhere near
by, but out of sight, there must be a stream with green pools where dace
were swimming.

\switchcolumn

他们站在榛子树的树荫下。阳光透过树叶照到他们脸上仍是热的。温斯顿向田野望去,一种奇怪的感觉涌了上来,他非常震惊,他认识这个地方。只看一眼,他就知道这儿曾经是个被动物咬得乱七八糟的牧场,一条蜿蜒的小路从中间穿过,到处都是鼹鼠的洞,牧场对面是高高低低的灌木丛,丛中的榆树依稀可见,随风轻舞,它们繁茂颤动的枝叶犹如女人的长发。附近一定有一条小溪,尽管看不见,但一定有那么一个地方,池水碧绿,鲦鱼自在游动。

\switchcolumn*

``Isn\textquotesingle t there a stream somewhere near here?'' he
whispered.

\switchcolumn

``附近有小溪吗?''他小声问。

\switchcolumn*

``That\textquotesingle s right, there is a stream. It\textquotesingle s
at the edge of the next field, actually. There are fish in it, great big
ones. You can watch them lying in the pools under the willow trees,
waving their tails.''

\switchcolumn

``有,是有一条小溪。就在那块地的边上。溪里还有鱼,很大的鱼,它们就在柳树下的水潭里摇尾巴。''

\switchcolumn*

``It\textquotesingle s the Golden Country---almost,'' he murmured.

\switchcolumn

``黄金乡,差不多就是黄金乡了。''他喃喃地说。

\switchcolumn*

``The Golden Country?''

\switchcolumn

``黄金乡?''

\switchcolumn*

``It\textquotesingle s nothing, really. A landscape I\textquotesingle ve
seen sometimes in a dream.''

\switchcolumn

``没什么,真的。那是我在梦里看到的风景。''

\switchcolumn*

``Look!'' whispered Julia.

\switchcolumn

``看!''朱莉亚轻声道。

\switchcolumn*

A thrush had alighted on a bough not five meters away, almost at the
level of their faces. Perhaps it had not seen them. It was in the sun,
they in the shade. It spread out its wings, fitted them carefully into
place again, ducked its head for a moment, as though making a sort of
obeisance to the sun, and then began to pour forth a torrent of song. In
the afternoon hush the volume of sound was startling. Winston and Julia
clung together, fascinated. The music went on and on, minute after
minute, with astonishing variations, never once repeating itself, almost
as though the bird were deliberately showing off its virtuosity.
Sometimes it stopped for a few seconds, spread out and resettled its
wings, then swelled its speckled breast and again burst into song.
Winston watched it with a sort of vague reverence. For whom, for what,
was that bird singing? No mate, no rival was watching it. What made it
sit at the edge of the lonely wood and pour its music into nothingness?
He wondered whether after all there was a microphone hidden somewhere
near. He and Julia had only spoken in low whispers, and it would not
pick up what they had said, but it would pick up the thrush. Perhaps at
the other end of the instrument some small, beetlelike man was listening
intently---listening to \emph{that}. But by degrees the flood of music
drove all speculations out of his mind. It was as though it were a kind
of liquid stuff that poured all over him and got mixed up with the
sunlight that filtered through the leaves. He stopped thinking and
merely felt. The girl\textquotesingle s waist in the bend of his arm was
soft and warm. He pulled her round so that they were breast to breast;
her body seemed to melt into his. Wherever his hands moved it was all as
yielding as water. Their mouths clung together; it was quite different
from the hard kisses they had exchanged earlier. When they moved their
faces apart again both of them sighed deeply. The bird took fright and
fled with a clatter of wings.

\switchcolumn

一只画眉停在了五米开外的树枝上,那树枝刚好和他们的脸差不多高。但它似乎并没有注意到他们。它在阳光里,他们在树荫下。它张了张翅膀,又小心地将翅膀收起,仿佛和太阳行礼一般猛地低下了头。之后它开始歌唱,美妙的声音倾泻而出,在这寂静的下午,嘹亮得惊人。温斯顿和朱莉亚紧紧地靠在一起听得入了迷。时间一分一分地过去,它仍在鸣唱,声音婉转多变,没有一次重复,就好像特意展示自己精湛的歌唱技巧。有时,它也会停下几秒舒展下羽翼,但接着它又会挺一挺那带着斑点的胸脯继续放声歌唱。温斯顿怀着崇敬地心情看着它,它为谁而唱?又为什么要唱?没有配偶,也没有竞争对手,是什么让它在这孤寂的树林里停下来对着一片空旷引吭高歌?温斯顿不知道附近是否藏有窃听话筒。他和朱莉亚的说话声很小,话筒收不到,但却收得到画眉的鸣叫。或许在话筒的另一端某个形如甲虫的小个子正专心致志地听着——听着这些。不过,画眉的歌声将他从沉思中拉了出来。那声音宛若液体,和枝叶间倾洒下的阳光融在一起倒在他身上。他停止思考,感受着一切。在他的怀抱里,女孩的腰是那样柔软而温暖。他将她的身体转过来,让她面对着他。她的身体好像与他融为一体,无论他把手放在哪里,她都如水一般驯服。他们吻在一起,和之前那僵硬的吻大为不同。将脸挪开时,两个人都深深地叹了口气。那只鸟被惊到了,扑了下翅膀,飞走了。

\switchcolumn*

Winston put his lips against her ear. ``\emph{Now},'' he whispered.

\switchcolumn

温斯顿将嘴唇贴在她的耳畔,轻声道:``现在。''

\switchcolumn*

``Not here,'' she whispered back. ``Come back to the hide-out.
It\textquotesingle s safer.''

\switchcolumn

``这里不行。''她悄声说,``回刚才那隐蔽的地方去,那里安全些。''

\switchcolumn*

Quickly, with an occasional crackle of twigs, they threaded their way
back to the clearing. When they were once inside the ring of saplings
she turned and faced him. They were both breathing fast, but the smile
had reappeared round the corners of her mouth. She stood looking at him
for an instant, then felt at the zipper of her overalls. And, yes! It
was almost as in his dream. Almost as swiftly as he had imagined it, she
had torn her clothes off, and when she flung them aside it was with that
same magnificent gesture by which a whole civilization seemed to be
annihilated. Her body gleamed white in the sun. But for a moment he did
not look at her body; his eyes were anchored by the freckled face with
its faint, bold smile. He knelt down before her and took her hands in
his.

\switchcolumn

很快,他们便回到了那块空地,路上踩折了一些树枝。当他们走进那片被小树环绕的空场,她转过身来面对他。两个人的呼吸都急促起来,她的嘴角又浮现出笑容。她站在那里,端详了他一会儿,然后伸手拉开制服上的拉链。啊,就是这样!几乎和梦中的一样,就像他想象的那样,她迅速地脱掉衣服,丢到一旁,姿态曼妙,似乎要将整个文化都摧毁殆尽。她的身体在阳光下泛着白色的光芒,但他并没有急着看她的身体,他注视着那张长着雀斑的放肆大笑的脸,被它深深吸引。他在她的面前跪下,抓住了她的手。

\switchcolumn*

``Have you done this before?''

\switchcolumn

``你之前做过吗?''

\switchcolumn*

``Of course. Hundreds of times---well, scores of times, anyway.''

\switchcolumn

``当然,好几百次了——好吧,几十次总有了。''

\switchcolumn*

``With Party members?''

\switchcolumn

``和党员一起吗?''

\switchcolumn*

``Yes, always with Party members.''

\switchcolumn

``对,总是和党员一起。''

\switchcolumn*

``With members of the Inner Party?''

\switchcolumn

``和内党党员?''

\switchcolumn*

``Not with those swine, no. But there\textquotesingle s plenty that
\emph{would} if they got half a chance. They\textquotesingle re not so
holy as they make out.''

\switchcolumn

``没和那些猪一起过,从来没有。不过,如果有机会,他们中不少人会愿意的。他们并不愿意像他们表现的那样正经。''

\switchcolumn*

His heart leapt. Scores of times she had done it; he wished it had been
hundreds---thousands. Anything that hinted at corruption always filled
him with a wild hope. Who knew? Perhaps the Party was rotten under the
surface, its cult of strenuousness and self-denial simply a sham
concealing iniquity. If he could have infected the whole lot of them
with leprosy or syphilis, how gladly he would have done so! Anything to
rot, to weaken, to undermine! He pulled her down so that they were
kneeling face to face.

\switchcolumn

他的心剧烈地跳动起来。她已经做过几十次了,他希望有几百次,几千次。任何有堕落意味的事情都会让他充满希望。谁知道呢,也许党已经败絮其中,也许提倡奋斗自律只为了掩盖罪恶。他十分乐意让他们统统染上麻风病、梅毒,如果他有能力办到的话。他把她拉下来,面对面地跪坐着。

\switchcolumn*

``Listen. The more men you\textquotesingle ve had, the more I love you.
Do you understand that?''

\switchcolumn

``听着,你有过的男人越多,我就越爱你。你明白吗?''

\switchcolumn*

``Yes, perfectly.''

\switchcolumn

``明白,完全明白。''

\switchcolumn*

``I hate purity, I hate goodness. I don\textquotesingle t want any virtue
to exist anywhere. I want everyone to be corrupt to the bones.''

\switchcolumn

``我恨纯洁,我恨善良!我不希望这世上有任何美德,我希望每个人都堕落到骨子里。''

\switchcolumn*

``Well then, I ought to suit you, dear. I\textquotesingle m corrupt to
the bones.''

\switchcolumn

``那么,我很适合你。我已经堕落到骨子里。''

\switchcolumn*

``You like doing this? I don\textquotesingle t mean simply me; I mean the
thing in itself?''

\switchcolumn

``你喜欢这样做吗?我不是说我,我是说这件事本身。''

\switchcolumn*

``I adore it.''

\switchcolumn

``我爱这件事。''

\switchcolumn*

That was above all what he wanted to hear. Not merely the love of one
person, but the animal instinct, the simple undifferentiated desire:
that was the force that would tear the Party to pieces. He pressed her
down upon the grass, among the fallen bluebells. This time there was no
difficulty. Presently the rising and falling of their breasts slowed to
normal speed, and in a sort of pleasant helplessness they fell apart.
The sun seemed to have grown hotter. They were both sleepy. He reached
out for the discarded overalls and pulled them partly over her. Almost
immediately they fell asleep and slept for about half an hour.

\switchcolumn

这正是他最希望听到的。不是爱某个人,而是爱这动物性的本能,单纯又人皆有之的欲望蕴涵着将党摧毁的力量。他将她压倒在草地上,在散落的蓝铃花中间。这次没有什么困难。他们的胸脯起伏,慢慢地恢复了正常的呼吸,带着又愉快又无助的感觉彼此分开。阳光似乎更暖了,他们睡意蒙眬。他伸手将丢在一旁的制服拉了过来盖在她身上。两个人很快睡着了,一直睡了近半个钟头。

\switchcolumn*

Winston woke first. He sat up and watched the freckled face, still
peacefully asleep, pillowed on the palm of her hand. Except for her
mouth, you could not call her beautiful. There was a line or two round
the eyes, if you looked closely. The short dark hair was extraordinarily
thick and soft. It occurred to him that he still did not know her
surname or where she lived.

\switchcolumn

温斯顿先醒了过来,他坐起身凝视着她那张长着雀斑的脸,她的头枕着手臂,平静地睡着。除了嘴唇,她说不上多漂亮,离近看,她的眼角还有一两道皱纹,她的黑色短发浓密柔软。他忽然想起来,他还不知道她姓什么,住在哪儿。

\switchcolumn*

The young, strong body, now helpless in sleep, awoke in him a pitying,
protecting feeling. But the mindless tenderness that he had felt under
the hazel tree, while the thrush was singing, had not quite come back.
He pulled the overalls aside and studied her smooth white flank. In the
old days, he thought, a man looked at a girl\textquotesingle s body and
saw that it was desirable, and that was the end of the story. But you
could not have pure love or pure lust nowadays. No emotion was pure,
because everything was mixed up with fear and hatred. Their embrace had
been a battle, the climax a victory. It was a blow struck against the
Party. It was a political act.

\switchcolumn

她年轻而健壮的身体正熟睡着,那无依无靠的样子唤起了他的怜爱和保护欲。但这不同于他在榛树下听画眉鸣唱时萌生的那种不假思索的柔情。他拉开她的制服,注视着她光滑白皙的肉体。他想,过去,男人看女人的肉体产生欲望就是故事的全部。但现在,单纯的爱和单纯的欲望都不复存在,不再有纯粹的感情,所有的一切都掺入了仇恨和恐惧。他们的拥抱是一场战斗,高潮是一次胜利。这是对党的打击,这是政治行为。

\switchcolumn*


\section{III}\label{iii-1}

\switchcolumn

\section*{三}\label{ux5341ux4e00}

\switchcolumn*

``We can come here once again,'' said Julia. ``It\textquotesingle s
generally safe to use any hide-out twice. But not for another month or
two, of course.''

\switchcolumn

``我们还能在这儿再来一次。''朱莉亚说,``随便哪个用来藏身的地方只用两次还是安全的。当然,一两个月内是不能用了。''

\switchcolumn*

As soon as she woke up her demeanor had changed. She became alert and
businesslike, put her clothes on, knotted the scarlet sash about her
waist, and began arranging the details of the journey home. It seemed
natural to leave this to her. She obviously had a practical cunning
which Winston lacked, and she seemed also to have an exhaustive
knowledge of the countryside round London, stored away from innumerable
community hikes. The route she gave him was quite different from the one
by which he had come, and brought him out at a different railway
station. ``Never go home the same way as you went out,'' she said, as
though enunciating an important general principle. She would leave
first, and Winston was to wait half an hour before following her.

\switchcolumn

她一醒来就变了样,动作干净利落。她穿上衣服,系好红饰带,开始安排回去的路线。这件事似乎理所当然地要交由她做。她显然拥有温斯顿所欠缺的处理实际问题的能力,再说她看上去对伦敦周边的乡村十分了解,无数次的集体远足让她积累了经验。她安排的回程路线和他来时的大不相同,她要他从另一个火车站出去。她说:``回去的路永远不要和来时的一样。''就好像在阐述某个重要的原理。她先离开,半个小时后,温斯顿再离开。

\switchcolumn*

She had named a place where they could meet after work, four evenings
hence. It was a street in one of the poorer quarters, where there was an
open market which was generally crowded and noisy. She would be hanging
about among the stalls, pretending to be in search of shoelaces or
sewing thread. If she judged that the coast was clear she would blow her
nose when he approached; otherwise he was to walk past her without
recognition. But with luck, in the middle of the crowd, it would be safe
to talk for a quarter of an hour and arrange another meeting.

\switchcolumn

她告诉他一个地方,四天后他们可以在下班后到那里相会。贫民区的某条街道上有个露天市场,那里总是人来人往,喧闹拥挤。她会在货摊之间转悠,假装找鞋带或缝衣线。如果她确定万无一失就在他走近时擤鼻子,否则他就装作不认识径直走过。运气好的话,他们可以在人群中平安无事地说上一刻钟的话,安排下次约会。

\switchcolumn*

``And now I must go,'' she said as soon as he had mastered his
instructions. ``I\textquotesingle m due back at nineteen-thirty.
I\textquotesingle ve got to put in two hours for the Junior Anti-Sex
League, handing out leaflets, or something. Isn\textquotesingle t it
bloody? Give me a brush-down, would you. Have I got any twigs in my
hair? Are you sure? Then good-by, my love, good-by!''

\switchcolumn

``我现在必须走了,''他一记住她的话,她就立即说道,``我要在19点30分回去,我得为青少年反性同盟花上两个小时,发发传单,或者做些其他什么事。这是不是很讨厌?能帮我梳下头吗?头发里有树枝吗?你确定没有?那么再见吧,亲爱的,再见!''

\switchcolumn*

She flung herself into his arms, kissed him almost violently, and a
moment later pushed her way through the saplings and disappeared into
the wood with very little noise. Even now he had not found out her
surname or her address. However, it made no difference, for it was
inconceivable that they could ever meet indoors or exchange any kind of
written communication.

\switchcolumn

她扑到他怀里,狠狠地吻他。过了一会儿,她从那些小树中拨开一条路,没有一点声音地消失在树林里。而他依然不知道她姓什么,住在哪儿。但这无关紧要,因为他们既不可能在室内相会,也不可能用文字交流。

\switchcolumn*

As it happened they never went back to the clearing in the wood. During
the month of May there was only one further occasion on which they
actually succeeded in making love. That was in another hiding place
known to Julia, the belfry of a ruined church in an almost-deserted
stretch of country where an atomic bomb had fallen thirty years earlier.
It was a good hiding place when once you got there, but the getting
there was very dangerous. For the rest they could meet only in the
streets, in a different place every evening and never for more than half
an hour at a time. In the street it was usually possible to talk, after
a fashion. As they drifted down the crowded pavements, not quite abreast
and never looking at one another, they carried on a curious,
intermittent conversation which flicked on and off like the beams of a
lighthouse, suddenly nipped into silence by the approach of a Party
uniform or the proximity of a telescreen, then taken up again minutes
later in the middle of a sentence, then abruptly cut short as they
parted at the agreed spot, then continued almost without introduction on
the following day. Julia appeared to be quite used to this kind of
conversation, which she called ``talking by installments.'' She was also
surprisingly adept at speaking without moving her lips. Just once in
almost a month of nightly meetings they managed to exchange a kiss. They
were passing in silence down a side street (Julia would never speak when
they were away from the main streets) when there was a deafening roar,
the earth heaved and the air darkened, and Winston found himself lying
on his side, bruised and terrified. A rocket bomb must have dropped
quite near at hand. Suddenly he became aware of Julia\textquotesingle s
face a few centimeters from his own, deathly white, as white as chalk.
Even her lips were white. She was dead! He clasped her against him, and
found that he was kissing a live warm face. But there was some powdery
stuff that got in the way of his lips. Both of their faces were thickly
coated with plaster.

\switchcolumn

实际上,后来他们再也没到树林中的那块空地去。自此之后,整个五月他们只有一次真正做爱的机会,那是在朱莉亚告诉他的另一个隐蔽之所,一个废教堂的钟楼上。三十年前一颗原子弹曾轰炸过那里。它是个很好的藏身处,只要你走得到,通往那儿的路非常危险。在其他时间,他们只能在街上相会,每晚都在不同的地方,每次都不超过半小时。通常在街上,他们还能勉强说上些话。他们在拥挤的街道上毫无目的地走着,算不上并肩而行,也从不两两相望。他们用一种奇特的方式断断续续地交谈着,就好像时亮时灭的灯塔。每每遇到身穿党员制服的人或是接近电屏,他们就突然噤声,几分钟后再接着之前中断的地方讲下去。到了约定好的分手地点,谈话也会突兀中断,到第二天晚上再直接接上。这样的说话方式朱莉亚似乎习以为常,她管这叫``分期谈话''。她讲话时不动嘴唇,令人惊讶。而在近一个月的晚间约会中,俩人只接了一次吻。当时他们正默默无语地走在一条小巷子里(朱莉亚从不在主要街道以外的地方说话),突然传来一声巨响,大地颤抖,漫天黑烟。温斯顿侧着身子倒在地上,又疼又怕。一定是火箭弹落在附近。突然他发现朱莉亚的脸离他只有几厘米,她的脸色如死人一般苍白,就像涂了白粉,嘴唇也同样惨白。她死了!他紧紧地搂住她,发现自己亲吻的却是活人才有的温暖的脸庞,一些粉末样的东西跑到他嘴里。原来,两个人的脸上都落了一层厚厚的灰泥。

\switchcolumn*

There were evenings when they reached their rendezvous and then had to
walk past one another without a sign, because a patrol had just come
round the corner or a helicopter was hovering overhead. Even if it had
been less dangerous, it would still have been difficult to find time to
meet. Winston\textquotesingle s working week was sixty hours,
Julia\textquotesingle s was even longer, and their free days varied
according to the pressure of work and did not often coincide. Julia, in
any case, seldom had an evening completely free. She spent an
astonishing amount of time in attending lectures and demonstrations,
distributing literature for the Junior Anti-Sex League, preparing
banners for Hate Week, making collections for the savings campaign, and
suchlike activities. It paid, she said; it was camouflage. If you kept
the small rules you could break the big ones. She even induced Winston
to mortgage yet another of his evenings by enrolling himself for the
part-time munition work which was done voluntarily by zealous Party
members. So, one evening every week, Winston spent four hours of
paralyzing boredom, screwing together small bits of metal which were
probably parts of bomb fuses, in a draughty ill-lit workshop where the
knocking of hammers mingled drearily with the music of the telescreens.

\switchcolumn

有几个晚上,他们来到约会的地方又不得不走开,连招呼都不能打,因为刚好有巡逻队从街角走来,或者正好有直升机在头顶转悠。即使不这么危险,他们也很难找到约会的时间。温斯顿一星期工作六十个小时,朱莉亚的工作时间更长,他们的休息日要根据工作的繁忙程度而定,经常凑不到一起。反正对朱莉亚来说,很少有哪个晚上是完全空闲的,她将大量时间花费在听演讲、参加游行、散发青少年反性同盟传单、准备仇恨周旗帜、为节约运动募捐之类的事情上。她说这是伪装,如果你能在小事上循规蹈矩,你就能在大事上打破规矩。她甚至说服温斯顿拿出一个晚上的时间兼职军用品生产,很多表现积极的党员都义务参加了。因此,每个星期都有一个晚上,温斯顿要花四个小时待在昏暗通风的工作间里,在电屏的音乐和锤子的敲打声中做着令人烦闷的工作,他要用螺丝将金属零件拧在一起,那大概是炸弹的引爆装置。

\switchcolumn*

When they met in the church tower the gaps in their fragmentary
conversation were filled up. It was a blazing afternoon. The air in the
little square chamber above the bells was hot and stagnant, and smelt
overpoweringly of pigeon dung. They sat talking for hours on the dusty,
twig-littered floor, one or other of them getting up from time to time
to cast a glance through the arrow slits and make sure that no one was
coming.

\switchcolumn

在教堂钟楼约会时,他们将之前断断续续的谈话所造成的空隙填满。那是个炎热的下午,钟楼的小方房间里空气窒闷,充斥着鸽屎味。他们坐在尘土淤积、树枝遍布的地板上聊了好几个小时的天,每隔一段时间都要从窗户缝处向外看看,以确定没有人接近。

\switchcolumn*

Julia was twenty-six years old. She lived in a hostel with thirty other
girls (``Always in the stink of women! How I hate women!'' she said
parenthetically), and she worked, as he had guessed, on the
novel-writing machines in the Fiction Department. She enjoyed her work,
which consisted chiefly in running and servicing a powerful but tricky
electric motor. She was ``not clever,'' but was fond of using her hands
and felt at home with machinery. She could describe the whole process of
composing a novel, from the general directive issued by the Planning
Committee down to the final touching-up by the Rewrite Squad. But she
was not interested in the finished product. She ``didn\textquotesingle t
much care for reading,'' she said. Books were just a commodity that had
to be produced, like jam or bootlaces.

\switchcolumn

朱莉亚二十六岁,和三十多个女孩挤在一间宿舍里(``一直生活在女人的臭味里!我是
多么恨女人!''她补充道),而正像他猜测的那样,她在小说司里负责写作机。她喜欢
她的工作,她要发动并维护一台功率超大又极其复杂的马达。她并不``聪明'',但她乐
于动手,和机器打交道让她感觉自在。从计划委员会提出总观点到最后改写队的修饰润
色,她能描绘出小说制造的整个过程,但她对最终的成品没有兴趣。她说她不大喜欢读
书,书和果酱、鞋带一样,无非是一种不得不生产的商品。

\switchcolumn*

She had no memories of anything before the early Sixties, and the only
person she had ever known who talked frequently of the days before the
Revolution was a grandfather who had disappeared when she was eight. At
school she had been captain of the hockey team and had won the
gymnastics trophy two years running. She had been a troop leader in the
Spies and a branch secretary in the Youth League before joining the
Junior Anti-Sex League. She had always borne an excellent character. She
had even (an infallible mark of good reputation) been picked out to work
in Pornosec, the sub-section of the Fiction Department which turned out
cheap pornography for distribution among the proles. It was nicknamed
Muck House by the people who worked in it, she remarked. There she had
remained for a year, helping to produce booklets in sealed packets with
titles like \emph{Spanking Stories} or \emph{One Night in a
Girls\textquotesingle{} School}, to be bought furtively by proletarian
youths who were under the impression that they were buying something
illegal.

\switchcolumn

六十年代初以前的事她已经想不起来了,在她认识的人里唯一经常谈到革命前情况的人
就是她的爷爷,而他在她八岁时就消失了。上学时她当过曲棍球队的指挥,曾连续两年
拿下体操比赛的奖杯。在参加青少年反性同盟前,她还做过侦察队的队长,青年团的支
部书记。她的表现一贯优秀,她甚至入选小说司下的色情科(这是声誉良好的标志),
那可是为群众生产低级的色情文学的地方。据她说,在那儿工作过的人管它叫``垃圾
间''。她在那儿干了一年,协助生产诸如《刺激故事集》或《女校一夜》这样装在密封
套里的书。年轻的群众偷偷地买回去,给人留下买违禁品的印象。

\switchcolumn*

``What are these books like?'' said Winston curiously.

\switchcolumn

``书里写了什么?''温斯顿好奇地问。

\switchcolumn*

``Oh, ghastly rubbish. They\textquotesingle re boring, really. They only
have six plots, but they swap them round a bit. Of course I was only on
the kaleidoscopes. I was never in the Rewrite Squad. I\textquotesingle m
not literary, dear---not even enough for that.''

\switchcolumn

``哦,简直就是垃圾。真的很无聊。它们总共只有六种情节,来回来去转着圈地用。当然我只负责搅拌机。我从来没进过改写队,我不擅长写东西,亲爱的——我做不了这个。''

\switchcolumn*

He learned with astonishment that all the workers in Pornosec, except
the head of the department, were girls. The theory was that men, whose
sex instincts were less controllable than those of women, were in
greater danger of being corrupted by the filth they handled.

\switchcolumn

他很惊讶,他这才了解原来在色情科,除了领导,其他的工作人员都是女的。有种理论说,和女人相比,男人的性本能不容易控制,男人更有可能被自己制造的色情作品侵蚀。

\switchcolumn*

``They don\textquotesingle t even like having married women there,'' she
added. ``Girls are always supposed to be so pure. Here\textquotesingle s
one who isn\textquotesingle t, anyway.''

\switchcolumn

``他们连结婚的女人都不愿意要,''她说,``人们总觉得女孩是纯洁的,但无论如何,我不是。''

\switchcolumn*

She had had her first love affair when she was sixteen, with a Party
member of sixty who later committed suicide to avoid arrest. ``And a good
job too,'' said Julia. ``Otherwise they\textquotesingle d have had my name
out of him when he confessed.'' Since then there had been various others.
Life as she saw it was quite simple. You wanted a good time; ``they,''
meaning the Party, wanted to stop you having it; you broke the rules as
best you could. She seemed to think it just as natural that ``they''
should want to rob you of your pleasures as that you should want to
avoid being caught. She hated the Party, and said so in the crudest
words, but she made no general criticism of it. Except where it touched
upon her own life she had no interest in Party doctrine. He noticed that
she never used Newspeak words, except the ones that had passed into
everyday use. She had never heard of the Brotherhood, and refused to
believe in its existence. Any kind of organized revolt against the
Party, which was bound to be a failure, struck her as stupid. The clever
thing was to break the rules and stay alive all the same. He wondered
vaguely how many others like her there might be in the younger
generation---people who had grown up in the world of the Revolution,
knowing nothing else, accepting the Party as something unalterable, like
the sky, not rebelling against its authority but simply evading it, as a
rabbit dodges a dog.

\switchcolumn

她第一次做爱是在十六岁,对方是个六十岁的党员。为了不被逮捕,他自杀了。``他做得很好,''朱莉亚说,``否则,他一招供,他们就会知道我的名字。''自那之后,她又做过很多次。在她看来生活很简单。你想快乐,``他们'',也就是党,不让你快乐,那你就要尽己所能地打破他们的规矩。她似乎觉得``他们''剥夺你的快乐就像你要避免被抓一样,都是自然而然的事。她用最粗俗的语言说她恨党,但她却没有批评党。除非和她的生活相关,她对党的理论没半点兴趣。他注意到,刨去已经成为日常用语的几个单词,她从来不讲新话。她没听说过兄弟会,也拒绝相信它的存在。在她看来但凡和党作对的组织都愚蠢之至,因为它们注定会失败。聪明的做法是既打破规矩,又保住性命。温斯顿不知道多少年轻人会像她这样,这些年轻人都是在革命后长大的,什么情况都不了解,他们眼中的党就像天空一样,是本来就有的,不会改变的,他们不会反抗它的权威,但他们却会像兔子躲避猎狗那样躲开它。

\switchcolumn*

They did not discuss the possibility of getting married. It was too
remote to be worth thinking about. No imaginable committee would ever
sanction such a marriage even if Katharine, Winston\textquotesingle s
wife, could somehow have been got rid of. It was hopeless even as a
daydream.

\switchcolumn

他们没有讨论有没有可能结婚,那太遥远了,不值得人去想。就算杀掉他的妻子凯瑟琳,也没有哪个委员会批准这样的婚姻。连做白日梦的希望都不存在。

\switchcolumn*

``What was she like, your wife?'' said Julia.

\switchcolumn

``你妻子是怎样的人?''朱莉亚问。

\switchcolumn*

``She was---do you know the Newspeak word \emph{goodthinkful}? Meaning
naturally orthodox, incapable of thinking a bad thought?''

\switchcolumn

``她是——你知道新话里有个词叫`思想好'吗?那是指天生的正经人,完全没有坏思想。''

\switchcolumn*

``No, I didn\textquotesingle t know the word, but I know the kind of
person, right enough.''

\switchcolumn

``我不知道这词,但我知道这种人,知道得够多了。''

\switchcolumn*

He began telling her the story of his married life, but curiously enough
she appeared to know the essential parts of it already. She described to
him, almost as though she had seen or felt it, the stiffening of
Katharine\textquotesingle s body as soon as he touched her, the way in
which she still seemed to be pushing him from her with all her strength,
even when her arms were clasped tightly round him. With Julia he felt no
difficulty in talking about such things; Katharine, in any case, had
long ceased to be a painful memory and become merely a distasteful one.

\switchcolumn

他开始向她讲述他的婚后生活,很奇怪,她好像早就清楚这种生活的大致状况。她向他描述他如何一碰到凯瑟琳的身体她就变得僵硬,描述她如何紧紧抱住他却仍像使劲推开他一样,就好像她曾亲眼见过,亲身经历过。和朱莉亚说这些事情很容易。无论如何那些与凯瑟琳相关的回忆已不再痛苦,它们已经变得令人讨厌。

\switchcolumn*

``I could have stood it if it hadn\textquotesingle t been for one thing,''
he said. He told her about the frigid little ceremony that Katharine had
forced him to go through on the same night every week. ``She hated it,
but nothing would make her stop doing it. She used to call it---but
you\textquotesingle ll never guess.''

\switchcolumn

``如果不是因为一件事,我还是可以忍受下去的。''温斯顿说。他告诉她,每个星期凯瑟琳都会在同一个晚上强迫他进行那没有感情的仪式。``她讨厌这事,但又没有什么能让她不做这事。她管这叫——你想都想不到。''

\switchcolumn*

``Our duty to the Party,'' said Julia promptly.

\switchcolumn

``我们对党的义务。''朱莉亚立即接道。

\switchcolumn*

``How did you know that?''

\switchcolumn

``你怎么知道?''

\switchcolumn*

``I\textquotesingle ve been at school too, dear. Sex talks once a month
for the over-sixteens. And in the Youth Movement. They rub it into you
for years. I dare say it works in a lot of cases. But of course you can
never tell; people are such hypocrites.''

\switchcolumn

``亲爱的,我也上过学的。每个月学校都会对十六岁以上的女孩做一次性教育讲座,青年团也是。他们灌输你好几年,我敢说那对很多人都起了作用。当然你也说不准,人总是虚伪的。''

\switchcolumn*

She began to enlarge upon the subject. With Julia, everything came back
to her own sexuality. As soon as this was touched upon in any way she
was capable of great acuteness. Unlike Winston, she had grasped the
inner meaning of the Party\textquotesingle s sexual puritanism. It was
not merely that the sex instinct created a world of its own which was
outside the Party\textquotesingle s control and which therefore had to
be destroyed if possible. What was more important was that sexual
privation induced hysteria, which was desirable because it could be
transformed into war fever and leader worship. The way she put it was:

\switchcolumn

她针对这个话题发起了感慨。就朱莉亚而言,性的欲望就是所有事情的出发点。无论以什么方式触及到这个问题,她都表现得极为敏锐。和温斯顿不同,她清楚党宣扬禁欲主义的深层原因。这不单因为性的本能会创造出专属于自己的,不受党操控的世界——所以必须尽可能地将它摧毁,更因为性压抑会造成歇斯底里,而这正是党希望的,因为它能够转化成对战争的狂热和对领袖的崇拜。她这样说:

\switchcolumn*

``When you make love you\textquotesingle re using up energy; and
afterwards you feel happy and don\textquotesingle t give a damn for
anything. They can\textquotesingle t bear you to feel like that. They
want you to be bursting with energy all the time. All this marching up
and down and cheering and waving flags is simply sex gone sour. If
you\textquotesingle re happy inside yourself, why should you get excited
about Big Brother and the Three-Year Plans and the Two Minutes Hate and
all the rest of their bloody rot?''

\switchcolumn

``你会在做爱的时候花光力气,之后你感到快乐,什么事都不想抱怨。而这样的感觉是他们不能容忍的。他们要你每时每刻都精力充沛。像游行、欢呼、挥舞旗子之类的事都是变了味的发泄性欲的途径。要是你内心愉悦,你又怎么会为老大哥,为三年计划,为两分钟仇恨会这些乱七八糟的东西激动?''

\switchcolumn*

That was very true, he thought. There was a direct, intimate connection
between chastity and political orthodoxy. For how could the fear, the
hatred, and the lunatic credulity which the Party needed in its members
be kept at the right pitch except by bottling down some powerful
instinct and using it as a driving force? The sex impulse was dangerous
to the Party, and the Party had turned it to account. They had played a
similar trick with the instinct of parenthood. The family could not
actually be abolished, and, indeed, people were encouraged to be fond of
their children in almost the old-fashioned way. The children, on the
other hand, were systematically turned against their parents and taught
to spy on them and report their deviations. The family had become in
effect an extension of the Thought Police. It was a device by means of
which everyone could be surrounded night and day by informers who knew
him intimately.

\switchcolumn

他想她说的是真的,禁欲和政治正统有着直接而紧密的联系。除了压抑强烈的本能,还有什么办法能让党员们像党要求的那样将恐惧、仇恨、盲从保持在一个恰当的水平上吗?对党而言,性冲动是危险的,需要利用的。他们用类似的手段对付为人父母的本能。他们不可能摧毁家庭,于是他们一方面鼓励人们用老办法爱自己的孩子。另一方面,又有系统地教孩子如何与父母作对,他们让孩子监视父母的言行,揭发父母的偏差,让家庭成为思想警察的延伸。依照这个策略,每个人都清楚无论白天还是黑夜自己都在告密者的包围下,且这告密者还是十分接近的人。

\switchcolumn*

Abruptly his mind went back to Katharine. Katharine would unquestionably
have denounced him to the Thought Police if she had not happened to be
too stupid to detect the unorthodoxy of his opinions. But what really
recalled her to him at this moment was the stifling heat of the
afternoon, which had brought the sweat out on his forehead. He began
telling Julia of something that had happened, or rather had failed to
happen, on another sweltering summer afternoon, eleven years ago.

\switchcolumn

突然,他又想起凯瑟琳。凯瑟琳太蠢了,她没有意识到他的观念不合正统,否则她一定
会向思想警察揭发他。不过,让他想起凯瑟琳的却是那炎热的下午,天气太热了,他的
头上冒出了汗。他开始向朱莉亚讲述一件事,或者说是没有发生的事,那还是在十一年
前,一个同样炎热的夏日午后。

\switchcolumn*

It was three or four months after they were married. They had lost their
way on a community hike somewhere in Kent. They had only lagged behind
the others for a couple of minutes, but they took a wrong turning, and
presently found themselves pulled up short by the edge of an old chalk
quarry. It was a sheer drop of ten or twenty meters, with boulders at
the bottom. There was nobody of whom they could ask the way. As soon as
she realized that they were lost Katharine became very uneasy. To be
away from the noisy mob of hikers even for a moment gave her a feeling
of wrongdoing. She wanted to hurry back by the way they had come and
start searching in the other direction. But at this moment Winston
noticed some tufts of loose-strife growing in the cracks of the cliff
beneath them. One tuft was of two colors, magenta and brick red,
apparently growing on the same root. He had never seen anything of the
kind before, and he called to Katharine to come and look at it.

\switchcolumn

他们婚后三四个月的时候,在去肯特郡的集体远足中迷了路,只落后了其他人几分钟,
就转错了弯。他们来到白垩矿场的边上,前面突然没了路,矿边距矿底有十几、二十几
米深,下面还尽是大石块。附近找不到可以问路的人,而凯瑟琳一发现迷路就变得十分
不安,哪怕只离开那群吵吵嚷嚷的人一会儿,她也觉得自己做错了事。她想快点顺原路
返回,看看别的方向有没有认识路的。但就在这时温斯顿注意到脚下的石缝中有几簇野
花,其中一簇有砖红和紫红两种颜色,还都长在同一条根上。温斯顿从来没有见过这样
的花,就喊凯瑟琳过来看。

\switchcolumn*

``Look, Katharine! Look at those flowers. That clump down near the
bottom. Do you see they\textquotesingle re two different colors?''

\switchcolumn

``看,凯瑟琳!看那些花,就是矿底旁边的那簇。你看到了吗?它们有两种不同的颜色。''

\switchcolumn*

She had already turned to go, but she did rather fretfully come back for
a moment. She even leaned out over the cliff face to see where he was
pointing. He was standing a little behind her, and he put his hand on
her waist to steady her. At this moment it suddenly occurred to him how
completely alone they were. There was not a human creature anywhere, not
a leaf stirring, not even a bird awake. In a place like this the danger
that there would be a hidden microphone was very small, and even if
there was a microphone it would only pick up sounds. It was the hottest,
sleepiest hour of the afternoon. The sun blazed down upon them, the
sweat tickled his face. And the thought struck him....

\switchcolumn

她已经转身走了,但她还是有些烦躁地回过头看了一眼。她从悬崖上探出身体,朝他指的地方张望。他就站在她后面一点儿,手扶着她的腰。这时他突然意识到他们有多么孤单,树叶是静止的,小鸟也好像睡着了,周围没有一个人影。这种地方不大可能藏着话筒,就算有,也只能记录下声音。这正是午后最炎热、最让人昏昏欲睡的时刻,炽热的阳光射在他们脸上,汗珠顺着他的脸颊流了下来。突然,他蹦出了一个念头\ldots\ldots{}

\switchcolumn*

``Why didn\textquotesingle t you give her a good shove?'' said Julia. ``I
would have.''

\switchcolumn

``为什么不好好推她一把?''朱莉亚说,``若是我就会。''

\switchcolumn*

``Yes, dear, you would have. I would have, if I\textquotesingle d been
the same person then as I am now. Or perhaps I
would---I\textquotesingle m not certain.''

\switchcolumn

``是的,亲爱的,你会的。若是现在的我,我也会,或者说可能会——我不能确定。''

\switchcolumn*

``Are you sorry you didn\textquotesingle t?''

\switchcolumn

``你后悔没推吗?''

\switchcolumn*

``Yes. On the whole I\textquotesingle m sorry I didn\textquotesingle t.''

\switchcolumn

``对,我后悔没推。''

\switchcolumn*

They were sitting side by side on the dusty floor. He pulled her closer
against him. Her head rested on his shoulder, the pleasant smell of her
hair conquering the pigeon dung. She was very young, he thought, she
still expected something from life, she did not understand that to push
an inconvenient person over a cliff solves nothing.

\switchcolumn

他们肩靠着肩,坐在堆满灰尘的地板上。他把她拉近,她则将头枕在他的肩上,从她头发中散发的香气盖过了鸽子屎的臭味。在他看来,她还年轻,仍对生活充满期望,她不能理解,将一个麻烦的人从悬崖上推下去不能解决任何问题。

\switchcolumn*

``Actually it would have made no difference,'' he said.

\switchcolumn

``事实上,没有什么不同。''他说。

\switchcolumn*

``Then why are you sorry you didn\textquotesingle t do it?''

\switchcolumn

``那你为什么又会后悔呢?''

\switchcolumn*

``Only because I prefer a positive to a negative. In this game that
we\textquotesingle re playing, we can\textquotesingle t win. Some kinds
of failure are better than other kinds, that\textquotesingle s all.''

\switchcolumn

``那仅仅因为相比消极,我更喜欢积极的东西。我们不可能在这场比赛中获胜,某些形式的失败会比其他形式的要好些。就是这样。''

\switchcolumn*

He felt her shoulders give a wriggle of dissent. She always contradicted
him when he said anything of this kind. She would not accept it as a law
of nature that the individual is always defeated. In a way she realized
that she herself was doomed, that sooner or later the Thought Police
would catch her and kill her, but with another part of her mind she
believed that it was somehow possible to construct a secret world in
which you could live as you chose. All you needed was luck and cunning
and boldness. She did not understand that there was no such thing as
happiness, that the only victory lay in the far future, long after you
were dead, that from the moment of declaring war on the Party it was
better to think of yourself as a corpse.

\switchcolumn

他感到她的肩膀动了一下,她不大同意他的观点。每每说起这种话,她总是不同意。她不能接受``个人终究会失败''是自然规律。从某种角度说,她能想到自己已劫数难逃,思想警察迟早会抓住她、杀死她。但在心底的某个部分,她仍相信她能建立起一个秘密的、由她自己来执掌的世界。而她所需要的无非是运气、勇气和机智。她还不懂世上并不存在什么幸福,唯一的胜利在遥远的未来,在你死了很久之后。从向党宣战的那一刻开始,你就最好把自己当成一具尸体。

\switchcolumn*

``We are the dead,'' he said.

\switchcolumn

``我们是死人。''他说。

\switchcolumn*

``We\textquotesingle re not dead yet,'' said Julia prosaically.

\switchcolumn

``我们还没死。''朱莉亚如实说。

\switchcolumn*

``Not physically. Six months, a year---five years, conceivably. I am
afraid of death. You are young, so presumably you\textquotesingle re
more afraid of it than I am. Obviously we shall put it off as long as we
can. But it makes very little difference. So long as human beings stay
human, death and life are the same thing.''

\switchcolumn

``不是指肉体。半年,一年——五年,可以想象。我怕死,你年轻,可能比我更怕。当然我们会尽可能地推迟我们的死。但这没有丝毫不同,只要人仍拥有人性,死和生就是一样的。''

\switchcolumn*

``Oh, rubbish! Which would you sooner sleep with, me or a skeleton?
Don\textquotesingle t you enjoy being alive? Don\textquotesingle t you
like feeling: This is me, this is my hand, this is my leg,
I\textquotesingle m real, I\textquotesingle m solid, I\textquotesingle m
alive! Don\textquotesingle t you like \emph{this}?''

\switchcolumn

``哦,废话!一会儿你想和谁睡觉?我还是骷髅?你不喜欢活着吗?你不喜欢这种感觉吗?这是我,这是我的手,我的腿,我是真实的,我是实实在在的,我是活着的!你难道不喜欢吗?''

\switchcolumn*

She twisted herself round and pressed her bosom against him. He could
feel her breasts, ripe yet firm, through her overalls. Her body seemed
to be pouring some of its youth and vigor into his.

\switchcolumn

她转过身,胸压着他。隔着制服,他能感觉到她那丰满坚挺的乳房。她似乎用身体将青春和活力注入到他的躯体中。

\switchcolumn*

``Yes, I like that,'' he said.

\switchcolumn

``没错,我喜欢。''他说。

\switchcolumn*

``Then stop talking about dying. And now listen, dear,
we\textquotesingle ve got to fix up about the next time we meet. We may
as well go back to the place in the wood. We\textquotesingle ve given it
a good long rest. But you must get there by a different way this time.
I\textquotesingle ve got it all planned out. You take the train---but
look, I\textquotesingle ll draw it out for you.''

\switchcolumn

``那就不要再说和死有关的东西了。现在听着,亲爱的,我们要商量下下次约会的时间。我们可以回树林那里,我们已经很久没有去过那儿了。但这次你必须从另外一条路走。我已经打算好了。你坐火车——你看,我给你画出来。''

\switchcolumn*

And in her practical way she scraped together a small square of dust,
and with a twig from a pigeon\textquotesingle s nest began drawing a map
on the floor.

\switchcolumn

她以她特有的务实作风扫出来一小堆土,然后又用从鸽子窝里拿出的树枝在地上画出一张地图。

\switchcolumn*


\section{IV}\label{iv-1}

\switchcolumn

\section*{四}\label{ux5341ux4e8c}

\switchcolumn*

Winston looked round the shabby little room above Mr.
Charrington\textquotesingle s shop. Beside the window the enormous bed
was made up, with ragged blankets and a coverless bolster. The
old-fashioned clock with the twelve-hour face was ticking away on the
mantelpiece. In the corner, on the gateleg table, the glass paperweight
which he had bought on his last visit gleamed softly out of the
half-darkness.

\switchcolumn

温斯顿环视了一下位于查林顿先生店铺上的那间简陋小屋。窗旁的大床已整理好,放着粗糙的毛毯和不带枕巾的枕头。壁炉上十二小时的老式座钟滴滴答答地走着。温斯顿上次来时买下的玻璃镇纸就摆在角落处的折叠桌上,在半明半暗的屋子里发着柔和的光。

\switchcolumn*

In the fender was a battered tin oilstove, a saucepan and two cups,
provided by Mr. Charrington. Winston lit the burner and set a pan of
water to boil. He had brought an envelope full of Victory Coffee and
some saccharine tablets. The clock\textquotesingle s hands said
seven-twenty; it was nineteen-twenty really. She was coming at
nineteen-thirty.

\switchcolumn

壁炉的围栏中有一只破旧的铁油炉,一口锅和两个杯子,这都是查林顿先生提供的。温斯顿将炉子点着,煮起了水。他带来的信封里装满了胜利牌咖啡和糖精片。钟走到7点20分,确切地说是19点20分。而她会在19点30分到。

\switchcolumn*

Folly, folly, his heart kept saying: conscious, gratuitous, suicidal
folly! Of all the crimes that a Party member could commit, this one was
the least possible to conceal. Actually the idea had first floated into
his head in the form of a vision of the glass paperweight mirrored by
the surface of the gateleg table. As he had foreseen, Mr. Charrington
had made no difficulty about letting the room. He was obviously glad of
the few dollars that it would bring him. Nor did he seem shocked or
become offensively knowing when it was made clear that Winston wanted
the room for the purpose of a love affair. Instead he looked into the
middle distance and spoke in generalities, with so delicate an air as to
give the impression that he had become partly invisible. Privacy, he
said, was a very valuable thing. Everyone wanted a place where they
could be alone occasionally. And when they had such a place, it was only
common courtesy in anyone else who knew of it to keep his knowledge to
himself. He even, seeming almost to fade out of existence as he did so,
added that there were two entries to the house, one of them through the
backyard, which gave on an alley.

\switchcolumn

愚蠢,愚蠢。他在心里不停地说:这是自找的,毫无道理的,自杀一般的蠢事。在党员可能犯的所有罪行中,这是最难隐藏的。事实上,他看到玻璃镇纸倒映在折叠桌上的影子时,第一次萌生了这个想法。正如他所料,查林顿先生痛快地将房间租给了他,他很高兴这给自己带来了几块钱的收入。当他得知温斯顿为了和情人约会而租房时,竟一点都不吃惊,也没有表现出令人讨厌的心照不宣,相反,他看着远处泛泛而谈,样子微妙,就好像他的一部分已经隐遁。他说,独处非常重要,人人都想有个地方能时不时单独待上一会儿。若他们找到了地方,别人知道了也别说什么,这是最基本的礼貌。他告诉温斯顿房子有两个入口,一条穿过后院,一条连接着小巷,而他说这话时就像真的消失了一样。

\switchcolumn*

Under the window somebody was singing. Winston peeped out, secure in the
protection of the muslin curtain. The June sun was still high in the
sky, and in the sun-filled court below a monstrous woman, solid as a
Norman pillar, with brawny red forearms and a sacking apron strapped
about her middle, was stumping to and fro between a washtub and a
clothesline, pegging out a series of square white things which Winston
recognized as babies\textquotesingle{} diapers. Whenever her mouth was
not corked with clothes pegs she was singing in a powerful contralto:

\switchcolumn

窗户下,什么人正在唱歌。温斯顿躲在平纹布的窗帘后偷看着外面。六月的太阳高高地挂在天上,阳光洒满整个院子。一个像诺曼大圆柱般高壮的女人在洗衣盆和晾衣绳间走来走去。她强壮的手臂红彤彤的,她的腰上系着粗麻布的围裙,她正往绳子上夹着一些方形的白布,温斯顿认出那是婴儿的尿布。只要嘴里没咬着夹子,她就用强有力的女低音唱:

\switchcolumn*

\begin{quotation}
  \noindent`` It was only an \textquotesingle opeless fancy,\\
  It passed like an Ipril dye,\\
  But a look an\textquotesingle{} a word an\textquotesingle{} the
  dreams they stirred\\
  They \textquotesingle ave stolen my \textquotesingle eart awye!''
\end{quotation}

\switchcolumn

\begin{quotation}
\noindent 不过是毫无希望的幻想,\\
消失得如此之快,像四月的日子。\\
但一个眼神,一句话,一个被他们唤起的梦!\\
都可将我心偷走!
\end{quotation}

\switchcolumn*

The tune had been haunting London for weeks past. It was one of
countless similar songs published for the benefit of the proles by a
sub-section of the Music Department. The words of these songs were
composed without any human intervention whatever on an instrument known
as a versificator. But the woman sang so tunefully as to turn the
dreadful rubbish into an almost pleasant sound. He could hear the woman
singing and the scrape of her shoes on the flagstones, and the cries of
the children in the street, and somewhere in the far distance a faint
roar of traffic, and yet the room seemed curiously silent, thanks to the
absence of a telescreen.

\switchcolumn

过去的几个星期这首歌在伦敦颇为盛行。它是音乐司下属的某个部门为群众出版的诸多歌曲中的一首。它的歌词由作词机制作,不需要任何人力。但那女人的歌声是如此优美,这堆可怕的垃圾竟也变得动听起来。他听到女人的歌声,听到鞋子在石板路上的摩擦声,听到大街上孩子们的叫喊声,听到远处什么地方行人的往来声,可屋子里仍安静得出奇,谢天谢地没有电屏。

\switchcolumn*

Folly, folly, folly! he thought again. It was inconceivable that they
could frequent this place for more than a few weeks without being
caught. But the temptation of having a hiding place that was truly their
own, indoors and near at hand, had been too much for both of them. For
some time after their visit to the church belfry it had been impossible
to arrange meetings. Working hours had been drastically increased in
anticipation of Hate Week. It was more than a month distant, but the
enormous, complex preparations that it entailed were throwing extra work
onto everybody. Finally both of them managed to secure a free afternoon
on the same day. They had agreed to go back to the clearing in the wood.
On the evening beforehand they met briefly in the street. As usual
Winston hardly looked at Julia as they drifted toward one another in the
crowd, but from the short glance he gave her it seemed to him that she
was paler than usual.

\switchcolumn

愚蠢,愚蠢,愚蠢!他再次想起。不能想象他们能如此频繁地约会几个星期都不被发现。但对他们来说,找到只属于他们自己的、隐秘的、在屋子里的且距离很近的地方的诱惑太大了。钟楼约会后有相当长时间他们都无法再安排见面。为了迎接仇恨周,他们的工作时间大大延长。尽管距离仇恨周还有一个多月,但庞大复杂的准备工作迫使每个人都不得不加班。他们好不容易才得以在同一个下午休息,原本计划再到那块林中空地去。而前一天晚上,他们在街上匆匆见了一面。像往常一样,在人群中相遇时,温斯顿不会去看朱莉亚的脸,但他迅速地瞥了她一眼,她的脸色好像比平时苍白得多。

\switchcolumn*

``It\textquotesingle s all off,'' she murmured as soon as she judged it
safe to speak. ``Tomorrow, I mean.''

\switchcolumn

``全完了。''一觉得安全,她就轻声说,``我是说明天。''

\switchcolumn*

``What?''

\switchcolumn

``怎么?''

\switchcolumn*

``Tomorrow afternoon. I can\textquotesingle t come.''

\switchcolumn

``明天下午我不能来。''

\switchcolumn*

``Why not?''

\switchcolumn

``为什么?''

\switchcolumn*

``Oh, the usual reason. It\textquotesingle s started early this time.''

\switchcolumn

``哦,还是那个原因,这次来得比较早。''

\switchcolumn*

For a moment he was violently angry. During the month that he had known
her the nature of his desire for her had changed. At the beginning there
had been little true sensuality in it. Their first love-making had been
simply an act of the will. But after the second time it was different.
The smell of her hair, the taste of her mouth, the feeling of her skin
seemed to have got inside him, or into the air all round him. She had
become a physical necessity, something that he not only wanted but felt
that he had a right to. When she said that she could not come, he had
the feeling that she was cheating him. But just at this moment the crowd
pressed them together and their hands accidentally met. She gave the
tips of his fingers a quick squeeze that seemed to invite not desire but
affection. It struck him that when one lived with a woman this
particular disappointment must be a normal, recurring event; and a deep
tenderness, such as he had not felt for her before, suddenly took hold
of him. He wished that they were a married couple of ten
years\textquotesingle{} standing. He wished that he were walking through
the streets with her just as they were doing now, but openly and without
fear, talking of trivialities and buying odds and ends for the
household. He wished above all that they had some place where they could
be alone together without feeling the obligation to make love every time
they met. It was not actually at that moment, but at some time on the
following day, that the idea of renting Mr.
Charrington\textquotesingle s room had occurred to him. When he
suggested it to Julia she had agreed with unexpected readiness. Both of
them knew that it was lunacy. It was as though they were intentionally
stepping nearer to their graves. As he sat waiting on the edge of the
bed he thought again of the cellars of the Ministry of Love. It was
curious how that predestined horror moved in and out of
one\textquotesingle s consciousness. There it lay, fixed in future time,
preceding death as surely as 99 precedes 100. One could not avoid it,
but one could perhaps postpone it; and yet instead, every now and again,
by a conscious, willful act, one chose to shorten the interval before it
happened.

\switchcolumn

他登时就发了脾气。认识她一个月了,对她的欲望性质已经发生了改变。开始,几乎没有什么真正的情欲,他们的第一次做爱只是简单的意识性的活动。不过第二次之后就不同了。她头发的气味,嘴唇的味道,皮肤的感觉似乎都融入了他的身体,或者说融入了环绕他的空气。她已然是生理的必需,他不单想得到她,还觉得有权得到她。当她说她不能来时,他有种被欺骗的感觉。而就在这时,人群将他们挤到一起,他们的手无意中触碰到对方。她飞快地捏了下他的指尖,它激起的似乎不是情欲,而是爱意。这让他意识到和女人一起生活,这样的失望一定是正常的,会反复出现的。突然,对她,他感到从未有过的深情厚谊。他真希望他们是结了十年婚的夫妻,他真希望两个人能大大方方地走在街上,不用担惊受怕,聊些日常琐事,买些日用杂货。而他最希望的是能有一个地方让他们两人不受打扰地待在一起,不用觉得见面就要做爱。这之后的第二天,他萌生了向查林顿先生租房子的念头。他将这个想法告诉朱莉亚,没想到她立即同意了。他们都知道这很疯狂,两个人似乎都故意向坟墓靠近。他坐在床边等她,想起仁爱部的地下室。这很奇怪,命中注定的恐怖在人的意识里钻进钻出。它就待在未来的某个时刻,好比99一定在100之前一样,它注定发生在死亡降临之前。没有人能避开它,但也有可能将它推迟。只是偶尔在神志清醒的情况下,人会任性地缩短这段时间。

\switchcolumn*

At this moment there was a quick step on the stairs. Julia burst into
the room. She was carrying a tool bag of coarse brown canvas, such as he
had sometimes seen her carrying to and fro at the Ministry. He started
forward to take her in his arms, but she disengaged herself rather
hurriedly, partly because she was still holding the tool bag.

\switchcolumn

这时,楼梯上响起急促的脚步声。朱莉亚突然出现在屋子里,带着一个棕色的帆布工具包,就是他经常看到的她上下班时带着的那个。他走过去将她揽在怀里,她却急急忙忙地挣脱开,多少因为她还提着东西。

\switchcolumn*

``Half a second,'' she said. ``Just let me show you what
I\textquotesingle ve brought. Did you bring some of that filthy Victory
Coffee? I thought you would. You can chuck it away again, because we
shan\textquotesingle t be needing it. Look here.''

\switchcolumn

``就等半秒。''她说,``看看我给你带了些什么。你是不是带了垃圾胜利咖啡了?我想你带了,你可以把它扔掉了,我们不需要它,看这儿。''

\switchcolumn*

She fell on her knees, threw open the bag, and tumbled out some spanners
and a screwdriver that filled the top part of it. Underneath was a
number of neat paper packets. The first packet that she passed to
Winston had a strange and yet vaguely familiar feeling. It was filled
with some kind of heavy, sandlike stuff which yielded wherever you
touched it.

\switchcolumn

她跪下来,打开包,将上面的扳手、螺丝刀一一拿开,露出包底下几个干干净净的纸包,她递给温斯顿的第一个纸包摸上去既熟悉又有点奇怪,装满了沉甸甸的沙子一般的东西,摸到哪儿哪就塌下去。

\switchcolumn*

``It isn\textquotesingle t sugar?'' he said.

\switchcolumn

``是糖吗?''他问。

\switchcolumn*

``Real sugar. Not saccharine, sugar. And here\textquotesingle s a loaf of
bread---proper white bread, not our bloody stuff---and a little pot of
jam. And here\textquotesingle s a tin of milk---but look! This is the
one I\textquotesingle m really proud of. I had to wrap a bit of sacking
round it, because---''

\switchcolumn

``真正的糖,不是糖精,是糖。这儿还有条面包,正经的白面包,不是咱们吃的那种劣质货——还有一小罐果酱。这是一听牛奶——但是,看!这才是让我得意的东西,我要把它包起来,因为——''

\switchcolumn*

But she did not need to tell him why she had wrapped it up. The smell
was already filling the room, a rich hot smell which seemed like an
emanation from his early childhood, but which one did occasionally meet
with even now, blowing down a passageway before a door slammed, or
diffusing itself mysteriously in a crowded street, sniffed for an
instant and then lost again.

\switchcolumn

她不需要告诉他包起它的原因。整间屋子都弥漫着浓烈的香味,那味道似乎来自温斯顿的幼年。不过,就算在今天,偶尔仍能闻到。有时它在房门关上前飘出来,穿过走廊,有时在熙熙攘攘的大街上时隐时现。

\switchcolumn*

``It\textquotesingle s coffee,'' he murmured, ``real coffee.''

\switchcolumn

``是咖啡,''他悄声道,``真正的咖啡。''

\switchcolumn*

``It\textquotesingle s Inner Party coffee. There\textquotesingle s a
whole kilo here,'' she said.

\switchcolumn

``这是内党的咖啡,这儿有整整一公斤呢。''她说。

\switchcolumn*

``How did you manage to get hold of all these things?''

\switchcolumn

``你怎么弄到这些东西的?''

\switchcolumn*

``It\textquotesingle s all Inner Party stuff. There\textquotesingle s
nothing those swine don\textquotesingle t have, nothing. But of course
waiters and servants and people pinch things, and---look, I got a little
packet of tea as well.''

\switchcolumn

``这些都是内党的东西,这些猪没有什么是弄不到的。不过,侍者、服务员还有其他一些人可以偷拿一些,看这个,我还弄到了一小包茶。''

\switchcolumn*

Winston had squatted down beside her. He tore open a corner of the
packet.

\switchcolumn

温斯顿在她身边蹲下,将纸包撕开一角。

\switchcolumn*

``It\textquotesingle s real tea. Not blackberry leaves.''

\switchcolumn

``这是真正的茶叶,不是黑莓的叶子。''

\switchcolumn*

``There\textquotesingle s been a lot of tea about lately.
They\textquotesingle ve captured India, or something,'' she said vaguely.
``But listen, dear. I want you to turn your back on me for three minutes.
Go and sit on the other side of the bed. Don\textquotesingle t go too
near the window. And don\textquotesingle t turn round till I tell you.''

\switchcolumn

``最近茶叶挺多的,他们占领了印度,还是哪儿。''她说得含含混混,``听着,亲爱的,转个身,背对着我,三分钟就好。去床那边坐吧,别太靠近窗口。我不喊你,就别转过来。''

\switchcolumn*

Winston gazed abstractedly through the muslin curtain. Down in the yard
the red-armed woman was still marching to and fro between the washtub
and the line. She took two more pegs out of her mouth and sang with deep
feeling:

\switchcolumn

温斯顿心不在焉地隔着布窗帘往外看。院子里,那个手臂通红的女人仍在洗衣盆和晾衣绳之间走来走去。她从嘴里取出两个夹子,充满感情地唱着:

\switchcolumn*

\begin{quotation}
 \noindent ``They sye that time \textquotesingle eals all things,\\
  They sye you can always forget;\\
  But the smiles an the tears across the years\\
  They twist my \textquotesingle eartstrings yet!''
\end{quotation}

\switchcolumn

\begin{quotation}
\noindent 他们说时间可以医治一切,\\
他们说你终究会忘记;\\
但这些年的笑与泪,\\
仍牵动着我的心弦。
\end{quotation}

\switchcolumn*

She knew the whole driveling song by heart, it seemed. Her voice floated
upward with the sweet summer air, very tuneful, charged with a sort of
happy melancholy. One had the feeling that she would have been perfectly
content if the June evening had been endless and the supply of clothes
inexhaustible, to remain there for a thousand years, pegging out diapers
and singing rubbish. It struck him as a curious fact that he had never
heard a member of the Party singing alone and spontaneously. It would
even have seemed slightly unorthodox, a dangerous eccentricity, like
talking to oneself. Perhaps it was only when people were somewhere near
the starvation level that they had anything to sing about.

\switchcolumn


她将这充斥着废话的歌词记在心底,她的歌声伴随着甜美的夏日空气飘扬直上,非常悦耳,还带着愉悦的忧伤。所有的一切都让人觉得假使六月的傍晚一直持续下去,假使要洗的衣服没完没了,那她就会心满意足地一面晒尿布,一面唱情歌,足足待上一千年。他突然想起来,他还从未见过哪个党员发自肺腑地独自歌唱。这有些奇怪,这样做就像自言自语,既怪异又危险。也许,人只有在濒临饿死的情况下才想放声歌唱吧。

\switchcolumn*

``You can turn round now,'' said Julia.

\switchcolumn

``现在,可以转过来了。''朱莉亚说。

\switchcolumn*

He turned round, and for a second almost failed to recognize her. What
he had actually expected was to see her naked. But she was not naked.
The transformation that had happened was much more surprising than that.
She had painted her face.

\switchcolumn

他转过身,有那么几秒他几乎认不出她。他原以为他会看到她的裸体,但他没有。她的变化比赤身裸体还要让他惊讶。她化了妆。

\switchcolumn*

She must have slipped into some shop in the proletarian quarters and
bought herself a complete set of make-up materials. Her lips were deeply
reddened, her cheeks rouged, her nose powdered; there was even a touch
of something under the eyes to make them brighter. It was not very
skillfully done, but Winston\textquotesingle s standards in such matters
were not high. He had never before seen or imagined a woman of the Party
with cosmetics on her face. The improvement in her appearance was
startling. With just a few dabs of color in the right places she had
become not only very much prettier, but, above all, far more feminine.
Her short hair and boyish overalls merely added to the effect. As he
took her in his arms a wave of synthetic violets flooded his nostrils.
He remembered the half-darkness of a basement kitchen and a
woman\textquotesingle s cavernous mouth. It was the very same scent that
she had used; but at the moment it did not seem to matter.

\switchcolumn

她一定是在群众聚居区的什么店子里买了一整套化妆品。她的嘴唇涂得鲜红,脸颊扑了腮红,鼻子也打上了粉,而她在眼皮下涂的东西则将她的双眼衬得更加明亮。她的化妆技术说不上多纯熟,不过这方面温斯顿也没有太高要求。在此之前,他还从没看到过哪个女性党员往脸上抹化妆品。她的脸神采奕奕,令人吃惊。不过是在恰当的地方轻轻地拍上一点粉就让她漂亮了那么多,不仅如此,她的女人味也更浓了,她短短的头发和男子气的制服又强化了这点。他把她抱在怀里,闻到了一股人造紫罗兰的香味。他记起来,在地下室里那间昏暗的厨房,想起那女人黑洞洞的嘴。那女人用的香水和她的一样,但现在这似乎已无关紧要。

\switchcolumn*

``Scent, too!'' he said.

\switchcolumn

``还用了香水!''他说。

\switchcolumn*

``Yes, dear, scent, too. And do you know what I\textquotesingle m going
to do next? I\textquotesingle m going to get hold of a real
woman\textquotesingle s frock from somewhere and wear it instead of
these bloody trousers. I\textquotesingle ll wear silk stockings and
high-heeled shoes! In this room I\textquotesingle m going to be a woman,
not a Party comrade.''

\switchcolumn

``没错,亲爱的,还用了香水。你知道接下来我要做什么吗?我要去弄一件真正的女式连衣裙,我要把它穿上,不再穿这讨厌的裤子了。我要穿丝袜,穿高跟鞋!在这个房间里,我要当女人,不当党员同志。''

\switchcolumn*

They flung their clothes off and climbed into the huge mahogany bed. It
was the first time that he had stripped himself naked in her presence.
Until now he had been too much ashamed of his pale and meager body, with
the varicose veins standing out on his calves and the discolored patch
over his ankle. There were no sheets, but the blanket they lay on was
threadbare and smooth, and the size and springiness of the bed
astonished both of them. ``It\textquotesingle s sure to be full of bugs,
but who cares?'' said Julia. One never saw a double bed nowadays except
in the homes of the proles. Winston had occasionally slept in one in his
boy-hood; Julia had never been in one before, so far as she could
remember.

\switchcolumn

他们脱掉衣服,爬到那巨大的红木床上。这是他第一次在她面前赤裸身体。之前他为自己那苍白消瘦的身体,那小腿上的静脉曲张以及脚踝处的疤痕感到羞愧。虽然没有床单,但垫在身下的旧毯子已经磨光了毛,十分光滑。床又大又有弹性,出乎他们的意料。``这里面肯定长满臭虫,可谁在乎呢?''朱莉亚说。除非在群众家里,现在已看不到双人床。小时候,温斯顿偶尔会在双人床上睡觉,朱莉亚则记不起自己睡过。

\switchcolumn*

Presently they fell asleep for a little while. When Winston woke up the
hands of the clock had crept round to nearly nine. He did not stir,
because Julia was sleeping with her head in the crook of his arm. Most
of her make-up had transferred itself to his own face or the bolster,
but a light stain of rouge still brought out the beauty of her
cheekbone. A yellow ray from the sinking sun fell across the foot of the
bed and lighted up the fireplace, where the water in the pan was boiling
fast. Down in the yard the woman had stopped singing, but the faint
shouts of children floated in from the street. He wondered vaguely
whether in the abolished past it had been a normal experience to lie in
bed like this, in the cool of a summer evening, a man and a woman with
no clothes on, making love when they chose, talking of what they chose,
not feeling any compulsion to get up, simply lying there and listening
to peaceful sounds outside. Surely there could never have been a time
when that seemed ordinary. Julia woke up, rubbed her eyes, and raised
herself on her elbow to look at the oilstove.

\switchcolumn

他们睡了一会儿,温斯顿醒来时,时针已接近9点。他没动,因为朱莉亚正枕在他的手臂上熟睡。她脸上的化妆品大部分都蹭到了温斯顿的脸和枕头上,但颧骨上那抹浅浅的胭脂仍能凸显她的美丽。落日的余辉映到了床腿上,照亮了壁炉。锅里的水已经开了,院子里的女人也不再歌唱,但大街上孩子们的吵闹声仍隐隐可闻。他意识蒙眬,在夏日的夜晚,男人和女人不着衣衫地躺在这样的床上,想做爱就做爱,想说什么就说什么,不会觉得必须要起来,就那么静静地躺着聆听窗外的声音。不知道在被消除掉的过去,这样的事情算不算平常?朱莉亚醒了,她揉揉眼睛,用手肘撑起身子,望向煤油炉。

\switchcolumn*

``Half that water\textquotesingle s boiled away,'' she said.
``I\textquotesingle ll get up and make some coffee in another moment.
We\textquotesingle ve got an hour. What time do they cut the lights off
at your flats?''

\switchcolumn

``水烧干一半了,''她说,``我这就起来煮咖啡。我们还有一个小时的时间,你的公寓几点熄灯?''

\switchcolumn*

``Twenty-three thirty.''

\switchcolumn

``23点30分。''

\switchcolumn*

``It\textquotesingle s twenty-three at the hostel. But you have to get in
earlier than that, because---Hi! Get out, you filthy brute!''

\switchcolumn

``宿舍里是23点,但你得早些回去。因为——嗨!滚开,你这脏东西!''

\switchcolumn*

She suddenly twisted herself over in the bed, seized a shoe from the
floor, and sent it hurtling into the corner with a boyish jerk of her
arm, exactly as he had seen her fling the dictionary at Goldstein, that
morning during the Two Minutes Hate.

\switchcolumn

她突然转身从地板上抓起一只鞋,像男孩子那样抬起胳膊朝房间的一角砸去,在上午的
两分钟仇恨会上,他曾看到她向高德斯坦因扔字典,姿势一模一样。

\switchcolumn*

``What was it?'' he said in surprise.

\switchcolumn

``那是什么?''他惊讶地问。

\switchcolumn*

``A rat. I saw him stick his beastly nose out of the wainscoting.
There\textquotesingle s a hole down there. I gave him a good fright,
anyway.''

\switchcolumn

``有只老鼠,我看到它从护墙板下面伸出鼻子。那儿有个洞。总之,我把它吓坏了。''

\switchcolumn*

``Rats!'' murmured Winston. ``In this room!''

\switchcolumn

``老鼠!''温斯顿嘀咕,``就在这屋里!''

\switchcolumn*

``They\textquotesingle re all over the place,'' said Julia indifferently
as she lay down again. ``We\textquotesingle ve even got them in the
kitchen at the hostel. Some parts of London are swarming with them. Did
you know they attack children? Yes, they do. In some of these streets a
woman daren\textquotesingle t leave a baby alone for two minutes.
It\textquotesingle s the great huge brown ones that do it. And the nasty
thing is that the brutes always---''

\switchcolumn

``它们哪儿都是。''朱莉亚躺下来,漠然地说,``我们宿舍的厨房里也有。伦敦的一些地方到处都是老鼠。你知道吗?它们还会攻击小孩。真的,它们真的那样。在那些地方的大街上,当妈妈的不敢让孩子独自待着,两分钟都不行。就是那种褐色的、体型很大的老鼠。还有恶心的事儿呢,这些令人作呕的东西总是——''

\switchcolumn*

``\emph{Don\textquotesingle t go on!}'' said Winston, with his eyes
tightly shut.

\switchcolumn

``别说了!''温斯顿闭上眼睛。

\switchcolumn*

``Dearest! You\textquotesingle ve gone quite pale. What\textquotesingle s
the matter? Do they make you feel sick?''

\switchcolumn

``亲爱的!你怎么这么苍白。出什么事了?它们让你不舒服吗?''

\switchcolumn*

``Of all horrors in the world---a rat!''

\switchcolumn

``老鼠是世界上最可怕的东西!''

\switchcolumn*

She pressed herself against him and wound her limbs round him, as though
to reassure him with the warmth of her body. He did not reopen his eyes
immediately. For several moments he had had the feeling of being back in
a nightmare which had recurred from time to time throughout his life. It
was always very much the same. He was standing in front of a wall of
darkness, and on the other side of it there was something unendurable,
something too dreadful to be faced. In the dream his deepest feeling was
always one of self-deception, because he did in fact know what was
behind the wall of darkness. With a deadly effort, like wrenching a
piece out of his own brain, he could even have dragged the thing into
the open. He always woke up without discovering what it was, but somehow
it was connected with what Julia had been saying when he cut her short.

\switchcolumn

她贴紧他,四肢环住他,似乎要用她的体温来安抚他。他没有立即张开双眼。几分钟过去了,他觉得自己又回到了那挥之不去的梦魇中。梦中的情景总是一样的。他站在一堵黑色的墙前,墙的那端是令人无法忍受的东西,它是如此可怕以至于人不能面对。在梦里,他一直有种感觉,他总是在欺骗自己。他明明知道那黑色的墙后是什么。只要他拼尽全力,他完全可以将这东西拖出来,就好像从脑子里强行取出什么一样。但每次他都在弄清它之前醒来,某种程度上,这东西和他刚刚打断的朱莉亚说的话有关。

\switchcolumn*

``I\textquotesingle m sorry,'' he said. ``It\textquotesingle s nothing. I
don\textquotesingle t like rats, that\textquotesingle s all.''

\switchcolumn

``抱歉,''他说,``没什么,我不喜欢老鼠,就这样。''

\switchcolumn*

``Don\textquotesingle t worry, dear, we\textquotesingle re not going to
have the filthy brutes in here. I\textquotesingle ll stuff the hole with
a bit of sacking before we go. And next time we come here
I\textquotesingle ll bring some plaster and bung it up properly.''

\switchcolumn

``别担心,亲爱的,我不会再让那恶心东西待在这儿。走之前,我会用布把洞堵住。下次来时,我再带些石灰,把它严严实实地塞起来。''

\switchcolumn*

Already the black instant of panic was half-forgotten. Feeling slightly
ashamed of himself, he sat up against the bedhead. Julia got out of bed,
pulled on her overalls, and made the coffee. The smell that rose from
the saucepan was so powerful and exciting that they shut the window lest
anybody outside should notice it and become inquisitive. What was even
better than the taste of the coffee was the silky texture given to it by
the sugar, a thing Winston had almost forgotten after years of
saccharine. With one hand in her pocket and a piece of bread and jam in
the other, Julia wandered about the room, glancing indifferently at the
bookcase, pointing out the best way of repairing the gateleg table,
plumping herself down in the ragged armchair to see if it was
comfortable, and examining the absurd twelve-hour clock with a sort of
tolerant amusement. She brought the glass paperweight over to the bed to
have a look at it in a better light. He took it out of her hand,
fascinated as always by the soft, rainwatery appearance of the glass.

\switchcolumn

恐慌的感觉已褪去了一半。他多少有些不好意思,便靠着床头坐起身。朱莉亚走下床,穿好衣服,做起了咖啡。一股浓郁的香味从锅里飘出来,刺激着人的感觉。他们关上窗户,不想引起外面人的注意,生怕他们寻根究底。加了糖的咖啡味道更好了,像丝绸般绵滑。吃了多年糖精的温斯顿几乎忘掉这种味道。朱莉亚一只手揣着口袋,一只手拿着抹了果酱的面包,在屋子里踱步而行,她瞥瞥书架,就折叠桌的修理方法发表看法。她用力坐了坐那把破扶手椅,看椅子是不是舒服,她又饶有兴致地查看了下那十二小时的座钟。她将玻璃镇纸拿到床上,以便在亮一点的地方看清楚它。但温斯顿却从她手里拿走了它,他被它那柔和如雨水的色泽深深吸引。

\switchcolumn*

``What is it, do you think?'' said Julia.

\switchcolumn

``你觉得它是什么?''朱莉亚问。

\switchcolumn*

``I don\textquotesingle t think it\textquotesingle s anything---I mean, I
don\textquotesingle t think it was ever put to any use.
That\textquotesingle s what I like about it. It\textquotesingle s a
little chunk of history that they\textquotesingle ve forgotten to alter.
It\textquotesingle s a message from a hundred years ago, if one knew how
to read it.''

\switchcolumn

``我想它什么都不是——我的意思是,我觉得它从未派上过用场,这正是我喜欢它的原因。要是有人能读懂它,它就是他们忘记篡改的一段历史,是从一百年前传来的信息。''

\switchcolumn*

``And that picture over there---'' she nodded at the engraving on the
opposite wall---``would that be a hundred years old?''

\switchcolumn

``还有那边的画——''她冲着对面墙上的画点了下头。``它也有一百年的历史吗?''

\switchcolumn*

``More. Two hundred, I dare say. One can\textquotesingle t tell.
It\textquotesingle s impossible to discover the age of anything
nowadays.''

\switchcolumn

``比那更早,大概有二百年了,我不敢确定,也没人说得清,今天,随便什么东西你都不可能知道它到底有多少年历史。''

\switchcolumn*

She went over to look at it. ``Here\textquotesingle s where that brute
stuck his nose out,'' she said, kicking the wainscoting immediately below
the picture. ``What is this place? I\textquotesingle ve seen it before
somewhere.''

\switchcolumn

她走过去看它。``就是这儿,老鼠从这儿伸出鼻子。''说着,她朝画下方的护墙板踹了一脚。``这是什么地方?我好像在什么地方看到过它。''

\switchcolumn*

``It\textquotesingle s a church, or at least it used to be. St.
Clement\textquotesingle s Dane its name was.'' The fragment of rhyme that
Mr. Charrington had taught him came back into his head, and he added
half-nostalgically: ``\emph{Oranges and lemons, say the bells of St.
Clement\textquotesingle s!}''

\switchcolumn

``是个教堂,至少以前是,叫圣克莱门特丹麦人。''他的脑海中再次浮现出查林顿先生教的那几句歌谣,他带着几分怀念之情唱起来:``橘子和柠檬,'圣克莱门特教堂的大钟说。
''

\switchcolumn*

To his astonishment she capped the line:

\switchcolumn

让他吃惊的是,她竟然接着唱了下去。

\switchcolumn*

\begin{quotation}
  ``You owe me three farthings, say the bells of St.
  Martin\textquotesingle s,\\
  When will you pay me? say the bells of Old Bailey---
\end{quotation}

\switchcolumn

\begin{quotation}
 \noindent ``你欠我三个法寻。圣马丁教堂的大钟说。\\
  什么时候还给我,老贝利的大钟——''
\end{quotation}

\switchcolumn*

``I can\textquotesingle t remember how it goes on after that. But anyway
I remember it ends up, \emph{Here comes a candle to light you to bed,
here comes a chopper to chop off your head!}''

\switchcolumn

``我想不起接下来怎么唱,但我好歹记得最后一句:蜡烛照着你睡觉,斧头把你头砍掉。''

\switchcolumn*

It was like the two halves of a countersign. But there must be another
line after \emph{the bells of Old Bailey}. Perhaps it could be dug out
of Mr. Charrington\textquotesingle s memory, if he were suitably
prompted.

\switchcolumn

就好像一个口令的两个部分。在``老贝利的大钟''后一定还有一段,也许只要给查林顿先生适当的提示,他就能将它从记忆中挖出来。

\switchcolumn*

``Who taught you that?'' he said.

\switchcolumn

``谁教你的?''他问。

\switchcolumn*

``My grandfather. He used to say it to me when I was a little girl. He
was vaporized when I was eight---at any rate, he disappeared. I wonder
what a lemon was,'' she added inconsequently. ``I\textquotesingle ve seen
oranges. They\textquotesingle re a kind of round yellow fruit with a
thick skin.''

\switchcolumn

``我的爷爷,当我还是个小女孩时,他经常对我唱它。我八岁的时候,他被蒸发了,不管怎样,他不见了。我想知道柠檬是什么。''她随口说道,
``我见过橘子,那是一种皮很厚的黄颜色的水果。''

\switchcolumn*

``I can remember lemons,'' said Winston. ``They were quite common in the
Fifties. They were so sour that it set your teeth on edge even to smell
them.''

\switchcolumn

``我还记得柠檬,''温斯顿说,``它在五十年代很常见。它很酸,闻一下都能把牙齿酸倒。''

\switchcolumn*

``I bet that picture\textquotesingle s got bugs behind it,'' said Julia.
``I\textquotesingle ll take it down and give it a good clean some day. I
suppose it\textquotesingle s almost time we were leaving. I must start
washing this paint off. What a bore! I\textquotesingle ll get the
lipstick off your face afterwards.''

\switchcolumn

``我打赌那画后面一定藏着臭虫。''朱莉亚说,``哪天有时间我要把它摘下来,好好打扫一番。咱们差不多该走了,我得把妆卸了。真烦人!等会儿我再把你脸上的唇膏擦掉。''

\switchcolumn*

Winston did not get up for a few minutes more. The room was darkening.
He turned over toward the light and lay gazing into the glass
paperweight. The inexhaustibly interesting thing was not the fragment of
coral but the interior of the glass itself. There was such a depth of
it, and yet it was almost as transparent as air. It was as though the
surface of the glass had been the arch of the sky, enclosing a tiny
world with its atmosphere complete. He had the feeling that he could get
inside it, and that in fact he was inside it, along with the mahogany
bed and the gateleg table and the clock and the steel engraving and the
paperweight itself. The paperweight was the room he was in, and the
coral was Julia\textquotesingle s life and his own, fixed in a sort of
eternity at the heart of the crystal.

\switchcolumn

温斯顿又在床上躺了几分钟,屋子慢慢地变暗。他转身对着光凝视那块玻璃镇纸,让人爱不释手的不是那块珊瑚,而是玻璃的内部。它虽然厚,却像空气般透明,弧形的外表如同天空的穹顶,将一个小世界连同空气都包入其中。他觉得他能进到它里面,事实上他已经在它之中了,和红木制的大床、折叠桌、座钟、钢板版画以及镇纸本身都待在它之中。镇纸就是他所在的屋子,珊瑚就是他和朱莉亚的生命,他们被固定在水晶中心,他们即永恒。

\switchcolumn*


\section{V}\label{v-1}

\switchcolumn

\section*{五}\label{ux5341ux4e09}

\switchcolumn*

Syme had vanished. A morning came, and he was missing from work; a few
thoughtless people commented on his absence. On the next day nobody
mentioned him. On the third day Winston went into the vestibule of the
Records Department to look at the notice board. One of the notices
carried a printed list of the members of the Chess Committee, of whom
Syme had been one. It looked almost exactly as it had looked
before---nothing had been crossed out---but it was one name shorter. It
was enough. Syme had ceased to exist; he had never existed.

\switchcolumn

赛姆消失了。一天早上,他没来上班。几个没脑子的人提到了他的旷工。到了第二天就再没人提起他。第三天,温斯顿到记录司的前厅看公告牌,其中一张公告上列着象棋委员会的委员名单,赛姆曾是其中之一。名单看起来和之前没什么不同——没有哪个人被划掉——但有一个名字消失了。这就够了,赛姆不存在了:他从未存在过。

\switchcolumn*

The weather was baking hot. In the labyrinthine Ministry the windowless,
air-conditioned rooms kept their normal temperature, but outside the
pavements scorched one\textquotesingle s feet and the stench of the
Tubes at the rush hours was a horror. The preparations for Hate Week
were in full swing, and the staffs of all the Ministries were working
overtime. Processions, meetings, military parades, lectures, waxwork
displays, film shows, telescreen programs all had to be organized;
stands had to be erected, effigies built, slogans coined, songs written,
rumors circulated, photographs faked. Julia\textquotesingle s unit in
the Fiction Department had been taken off the production of novels and
was rushing out a series of atrocity pamphlets. Winston, in addition to
his regular work, spent long periods every day in going through back
files of the \emph{Times} and altering and embellishing news items which
were to be quoted in speeches. Late at night, when crowds of rowdy
proles roamed the streets, the town had a curiously febrile air. The
rocket bombs crashed oftener than ever, and sometimes in the far
distance there were enormous explosions which no one could explain and
about which there were wild rumors.

\switchcolumn

天热得出奇,迷宫一般的部里没有一扇窗户,装着空调的房间尚保持着常温,但外面的人行道却热得烫脚。高峰时间的地铁里又挤又臭。仇恨周的筹备工作正进行得热烈非常。部里的所有工作人员都加班加点。游行、集会、阅兵、演讲、蜡像、展览、电影、电屏,都需要安排。看台要搭建,雕像要制造,标语要撰写,歌曲要创作,谣言要传播,照片要伪造。在小说司里,朱莉亚所在的部门已经中断了小说的制造,人们急匆匆赶制一批关于敌人暴行的小册子。至于温斯顿,除了日常工作,每天还要花很长时间检查已经过期的《泰晤士报》,对演讲中要用到的新闻进行修改、润饰。夜深了,许许多多的群众在街上闲逛,吵吵闹闹,整个城市都笼罩上怪异的狂热气氛。火箭弹的袭击更加频繁,不时便会从远处传来巨大的爆炸声。没有人说得清为什么,谣言纷纷兴起。

\switchcolumn*

The new tune which was to be the theme song of Hate Week (the ``Hate
Song,'' it was called) had already been composed and was being endlessly
plugged on the telescreens. It had a savage, barking rhythm which could
not exactly be called music, but resembled the beating of a drum. Roared
out by hundreds of voices to the tramp of marching feet, it was
terrifying. The proles had taken a fancy to it, and in the midnight
streets it competed with the still-popular ``It Was Only a Hopeless
Fancy.'' The Parsons children played it at all hours of the night and
day, unbearably, on a comb and a piece of toilet paper.
Winston\textquotesingle s evenings were fuller than ever. Squads of
volunteers, organized by Parsons, were preparing the street for Hate
Week, stitching banners, painting posters, erecting flagstaffs on the
roofs, and perilously slinging wires across the street for the reception
of streamers. Parsons boasted that Victory Mansions alone would display
four hundred meters of bunting. He was in his native element and as
happy as a lark. The heat and the manual work had even given him a
pretext for reverting to shorts and an open shirt in the evenings. He
was everywhere at once, pushing, pulling, sawing, hammering,
improvising, jollying everyone along with comradely exhortations and
giving out from every fold of his body what seemed an inexhaustible
supply of acrid-smelling sweat.

\switchcolumn

作为仇恨歌主题曲的新旋律(即《仇恨之歌》)已谱写完成,电屏里没完没了地播放着。确切地说,它算不上音乐,它曲调粗鲁,犹如野兽的嚎叫,和打鼓有些相像。几百个人和着行军步伐大声歌唱着,场面慑人。群众喜欢它,在午夜的街头它和仍然流行的《这不过是无望的单恋》交相呼应。帕森斯的孩子用梳子和卫生纸整日整夜地吹,让人难以忍受。对温斯顿来说,和以前相比,晚上的时间更紧张了。由帕森斯组织的志愿者在大街上为仇恨周作准备。他们缝制条幅,张贴宣传画,在屋顶竖旗杆,还冒险在街道上吊起铁丝悬挂欢迎彩带。帕森斯吹嘘单是胜利大厦,亮出的旗帜就有四百多米长,他天性尽露,像百灵鸟一样兴奋。炎热的天气和体力工作让他有了在晚上穿短裤、开领衫的借口。他可以突然出现在任何地方,推拉锯锤,即兴地做点儿什么,用说教的语气激励他人。同时,他的身体还没完没了地散发着难闻的汗味。

\switchcolumn*

A new poster had suddenly appeared all over London. It had no caption,
and represented simply the monstrous figure of a Eurasian soldier, three
or four meters high, striding forward with expressionless Mongolian face
and enormous boots, a submachine gun pointed from his hip. From whatever
angle you looked at the poster, the muzzle of the gun, magnified by the
foreshortening, seemed to be pointed straight at you. The thing had been
plastered on every blank space on every wall, even outnumbering the
portraits of Big Brother. The proles, normally apathetic about the war,
were being lashed into one of their periodical frenzies of patriotism.
As though to harmonize with the general mood, the rocket bombs had been
killing larger numbers of people than usual. One fell on a crowded film
theater in Stepney, burying several hundred victims among the ruins. The
whole population of the neighborhood turned out for a long, trailing
funeral which went on for hours and was in effect an indignation
meeting. Another bomb fell on a piece of waste ground which was used as
a playground, and several dozen children were blown to pieces. There
were further angry demonstrations, Goldstein was burned in effigy,
hundreds of copies of the poster of the Eurasian soldier were torn down
and added to the flames, and a number of shops were looted in the
turmoil; then a rumor flew round that spies were directing the rocket
bombs by means of wireless waves, and an old couple who were suspected
of being of foreign extraction had their house set on fire and perished
of suffocation.

\switchcolumn

突然之间,伦敦所有地方都贴上了新的宣传画。画上没有文字说明,只有一个三四米高的体型巨大的欧亚国士兵,他长着一张蒙古人的脸,面无表情地大步前进,脚上蹬着大号军靴,腰间挎着轻机枪。无论从什么角度看,依照透视原理放大的枪口都正对着你。每堵墙的空白处都贴上了这幅画,它的数量甚至多过老大哥的画像。通常,群众不关心战争,但此时他们却被激起周期性的狂热的爱国之情。似乎在呼应这普遍的情绪,死于火箭弹的人更多了。一枚炸弹掉在了斯坦普尼的一家拥挤的电影院里,将数百人埋在废墟之下。周边的居民纷纷出来参加葬礼,排起长长的送葬队伍。葬礼一直持续了几个小时,是充斥着愤怒的集会。还有一次,炸弹掉在了一块游戏用的空地上,将好几十个小孩炸成碎片。人们再次举行了愤怒的示威,焚烧起高德斯坦因的雕像,数百张画着欧亚国士兵的宣传画被撕下来扔到火里,混乱之中,一些商店遭到了抢劫。之后又有传言说,有间谍通过无线电操纵火箭弹的投放,于是一对老夫妇的房子被烧了,只因为他们被怀疑有外国血统,而他们本人也在大火中窒息而死。

\switchcolumn*

In the room over Mr. Charrington\textquotesingle s shop, when they could
get there, Julia and Winston lay side by side on a stripped bed under
the open window, naked for the sake of coolness. The rat had never come
back, but the bugs had multiplied hideously in the heat. It did not seem
to matter. Dirty or clean, the room was paradise. As soon as they
arrived they would sprinkle everything with pepper bought on the blacket
market, tear off their clothes and make love with sweating bodies, then
fall asleep and wake to find that the bugs had rallied and were massing
for the counter-attack.

\switchcolumn

只要可以去,在查林顿先生商店上的那个房间里,在敞开的窗下,朱莉亚和温斯顿并排躺在没有铺床单的大床上,为了凉快些,他们浑身赤裸。老鼠没有再来,因为天气炎热,臭虫的数量急剧增长。但这似乎无关紧要,脏也好,干净也罢,这屋子就是天堂。他们一到这里就将从黑市上买来的胡椒粉撒得到处都是,他们脱掉衣服,大汗淋漓地做爱,然后沉沉睡去。醒来之后,发现臭虫们已重整旗鼓,正聚集一起准备反攻。

\switchcolumn*

Four, five, six---seven times they met during the month of June. Winston
had dropped his habit of drinking gin at all hours. He seemed to have
lost the need for it. He had grown fatter, his varicose ulcer had
subsided, leaving only a brown stain on the skin above his ankle, his
fits of coughing in the early morning had stopped. The process of life
had ceased to be intolerable, he had no longer any impulse to make faces
at the telescreen or shout curses at the top of his voice. Now that they
had a secure hiding place, almost a home, it did not even seem a
hardship that they could only meet infrequently and for a couple of
hours at a time. What mattered was that the room over the junk shop
should exist. To know that it was there, inviolate, was almost the same
as being in it. The room was a world, a pocket of the past where extinct
animals could walk. Mr. Charrington, thought Winston, was another
extinct animal. He usually stopped to talk with Mr. Charrington for a
few minutes on his way upstairs. The old man seemed seldom or never to
go out of doors, and on the other hand to have almost no customers. He
led a ghostlike existence between the tiny, dark shop and an even tinier
back kitchen where he prepared his meals and which contained, among
other things, an unbelievably ancient gramophone with an enormous horn.
He seemed glad of the opportunity to talk. Wandering about among his
worthless stock, with his long nose and thick spectacles and his bowed
shoulders in the velvet jacket, he had always vaguely the air of being a
collector rather than a tradesman. With a sort of faded enthusiasm he
would finger this scrap of rubbish or that---a china bottle-stopper, the
painted lid of a broken snuffbox, a pinchbeck locket containing a strand
of some long-dead baby\textquotesingle s hair---never asking that
Winston should buy it, merely that he should admire it. To talk to him
was like listening to the tinkling of a worn-out musical box. He had
dragged out from the corners of his memory some more fragments of
forgotten rhymes. There was one about four and twenty blackbirds, and
another about a cow with a crumpled horn, and another about the death of
poor Cock Robin. ``It just occurred to me you might be interested,'' he
would say with a deprecating little laugh whenever he produced a new
fragment. But he could never recall more than a few lines of any one
rhyme.

\switchcolumn

整个六月,他们一共约会了四次、五次、六次——七次。温斯顿已经戒掉了不时就喝杜松子酒的习惯。他看上去已经不需如此。他胖了,静脉曲张引起的溃疡也消失了,只在脚踝上留下一块棕色的瘢痕。早上起来,他不再咳嗽,日常生活不再令他难以忍受,他也不再有冲电屏做鬼脸、骂脏话的冲动。现在他们有了固定而隐蔽的约会地点,就像一个家。即使不常见面,且每次约会也只有一两个小时的时间,他们仍不觉得辛苦。重要的是旧货铺上的屋子居然还在。知道它在哪里,无人打扰,自己也好像身处其中。这屋子就是一个世界,是过去时光的缩小版,已经绝种的动物在其中漫步。在温斯顿看来,查林顿先生就是一个``绝种''。有时,在上楼之前,他会停下来和查林顿先生聊上几分钟。这个老人看起来很少出门,或者说他从不出门,他的客人很少,他在阴暗窄小的商店和比商店更小的厨房之间过着幽灵般的生活。他在厨房里做饭,除了厨房里应有的东西,他还有一台老得让人不敢相信的带着大喇叭的唱机。他在那堆毫无价值的货品中走来走去,很高兴有机会和别人说话,他的鼻子又尖又长,架着一副厚镜片的眼镜,他的身上穿着天鹅绒的夹克,肩膀压得很低。与其说他是一个旧货商,不如说他更像一位收藏家。有时,他会带着几分深情抚摸某件破烂——瓷制的瓶塞、破鼻烟壶的彩盖、装着夭折了的婴孩头发的黄铜盒子——他从不要求温斯顿买下它们,只是说他应懂得欣赏。和他谈话,就像听老式八音盒的音乐。他从记忆深处挖出一些早已被人遗忘的歌谣的片段。比如二十四只黑画眉,弯角母牛,知更鸟之死等。``我刚好想起来,您可能会感兴趣。''每当他回忆起一个片段,他就会带着几分自嘲式的笑容说。但无论什么歌谣,他都只记得只言片语。

\switchcolumn*

Both of them knew---in a way, it was never out of their minds---that
what was now happening could not last long. There were times when the
fact of impending death seemed as palpable as the bed they lay on, and
they would cling together with a sort of despairing sensuality, like a
damned soul grasping at his last morsel of pleasure when the clock is
within five minutes of striking. But there were also times when they had
the illusion not only of safety but of permanence. So long as they were
actually in this room, they both felt, no harm could come to them.
Getting there was difficult and dangerous, but the room itself was
sanctuary. It was as when Winston had gazed into the heart of the
paperweight, with the feeling that it would be possible to get inside
that glassy world, and that once inside it time could be arrested. Often
they gave themselves up to daydreams of escape. Their luck would hold
indefinitely, and they would carry on their intrigue, just like this,
for the remainder of their natural lives. Or Katharine would die, and by
subtle maneuverings Winston and Julia would succeed in getting married.
Or they would commit suicide together. Or they would disappear, alter
themselves out of recognition, learn to speak with proletarian accents,
get jobs in a factory, and live out their lives undetected in a back
street. It was all nonsense, as they both knew. In reality there was no
escape. Even the one plan that was practicable, suicide, they had no
intention of carrying out. To hang on from day to day and from week to
week, spinning out a present that had no future, seemed an unconquerable
instinct, just as one\textquotesingle s lungs will always draw the next
breath so long as there is air available.

\switchcolumn

他们都清楚——或者说,他们从未忘记——这种状况不可能一直持续下去。有时,死神步步紧逼的感觉就像他们躺在床上一样真实。他们紧紧地贴在一起,肉欲中饱含绝望,好比一个堕入地狱的灵魂抓住最后五分钟时间去体会最后一点快感。但有时,他们也会萌生一种幻觉,他们不仅安全,且他们可以长久地这样下去。他们都觉得,只要他们真的待在这个房间里,就不会受到什么伤害。去这个房间困难重重又危险无比,但房间本身却是安全的。温斯顿注视着镇纸的中心,觉得自己有可能进入玻璃中的世界,一旦他真的进去了,时间就会停止。他们经常放纵自己沉浸在逃避现实的白日梦中,以为他们的好运会永远持续下去,以为在他们的余生里,他们可以一直像这样约会下去。或者凯瑟琳死了,温斯顿和朱莉亚通过某种狡猾的方式结婚。或者他们一起消失,改头换面,让人认不出来。他们可以模仿群众的说话腔调,到工厂做工并在某条街道的后面不为人知地过完一辈子。但他们清楚,这些都毫无意义,现实无处可逃。唯一可行的计划就是自杀,而他们无意如此。坚持一天是一天,坚持一个星期算一个星期。在看不到未来的情况下拖延时间似乎是无法压抑的本能,就好像只要有空气,人就会呼吸一样。

\switchcolumn*

Sometimes, too, they talked of engaging in active rebellion against the
Party, but with no notion of how to take the first step. Even if the
fabulous Brotherhood was a reality, there still remained the difficulty
of finding one\textquotesingle s way into it. He told her of the strange
intimacy that existed, or seemed to exist, between himself and
O\textquotesingle Brien, and of the impulse he sometimes felt simply to
walk into O\textquotesingle Brien\textquotesingle s presence, announce
that he was the enemy of the Party, and demand his help. Curiously
enough this did not strike her as an impossibly rash thing to do. She
was used to judging people by their faces, and it seemed natural to her
that Winston should believe O\textquotesingle Brien to be trustworthy on
the strength of a single flash of the eyes. Moreover she took it for
granted that everyone, or nearly everyone, secretly hated the Party and
would break the rules if he thought it safe to do so. But she refused to
believe that widespread, organized opposition existed or could exist.
The tales about Goldstein and his underground army, she said, were
simply a lot of rubbish which the Party had invented for its own
purposes and which you had to pretend to believe in. Times beyond
number, at Party rallies and spontaneous demonstrations, she had shouted
at the top of her voice for the execution of people whose names she had
never heard and in whose supposed crimes she had not the faintest
belief. When public trials were happening she had taken her place in the
detachments from the Youth League who surrounded the courts from morning
to night, chanting at intervals ``Death to the traitors!'' During the Two
Minutes Hate she always excelled all others in shouting insults at
Goldstein. Yet she had only the dimmest idea of who Goldstein was and
what doctrines he was supposed to represent. She had grown up since the
Revolution and was too young to remember the ideological battles of the
Fifties and Sixties. Such a thing as an independent political movement
was outside her imagination; and in any case the Party was invincible.
It would always exist, and it would always be the same. You could only
rebel against it by secret disobedience or, at most, by isolated acts of
violence such as killing somebody or blowing something up.

\switchcolumn

有时,他们也会讨论如何积极行动与党作对,但他们不知道要怎样走第一步。即使传说中的兄弟会真实存在,要找到并加入它们也十分困难。他告诉她,他对奥布兰有一种或者说似乎有一种微妙的亲近感,这让他偶尔会有到他面前宣称自己是党的敌人的冲动,他希望得到他的帮助。非常奇怪,她并不认为这过于冒失。她很擅长以貌取人,在她看来,温斯顿仅仅因为一个一闪而过的眼神就认为奥布兰值得信赖是自然而然的。除此之外,她还理所当然地以为每个人、几乎每个人都痛恨着党,只要能保障安全,谁都想打破规矩。不过她不认为有组织且广泛的反抗活动是存在的,它们没有可能存在。她说,高德斯坦因及其秘密军队都只是党出于某种目的编造出来的胡言乱语,你只能假装相信。说不清有多少次,在党员集会和示威活动中,她尽己所能地喊叫着要将那些她从来没听说过的人处以死刑,而她并不相信他们真的犯下了安在他们身上的罪名。公审举行时,她参加了青年团的队伍,他们将法庭团团围住,从早到晚,不时高喊``杀死卖国贼''。在两分钟仇恨会上,她总是大声咒骂高德斯坦因,比其他人还要激动。至于高德斯坦因究竟是什么人,持哪种主张,她并不了解。她在革命后长大,她太年轻了以至于一点都不记得发生在五六十年代的意识形态的斗争。诸如独立的政治运动之类的事已超出了她的想象范围。无论如何,党是战无不胜的,它会永远存在,永远保持一个样子。你能做的仅仅是秘密地反对它,至多通过孤立的暴力,比如杀死某个人或者炸掉某个东西来反抗它。

\switchcolumn*

In some ways she was far more acute than Winston, and far less
susceptible to Party propaganda. Once when he happened in some
connection to mention the war against Eurasia, she startled him by
saying casually that in her opinion the war was not happening. The
rocket bombs which fell daily on London were probably fired by the
Government of Oceania itself, ``just to keep people frightened.'' This was
an idea that had literally never occurred to him. She also stirred a
sort of envy in him by telling him that during the Two Minutes Hate her
great difficulty was to avoid bursting out laughing. But she only
questioned the teachings of the Party when they in some way touched upon
her own life. Often she was ready to accept the official mythology,
simply because the difference between truth and falsehood did not seem
important to her. She believed, for instance, having learnt it at
school, that the Party had invented airplanes. (In his own schooldays,
Winston remembered, in the late Fifties, it was only the helicopter that
the Party claimed to have invented; a dozen years later, when Julia was
at school, it was already claiming the airplane; one generation more,
and it would be claiming the steam engine.) And when he told her that
airplanes had been in existence before he was born, and long before the
Revolution, the fact struck her as totally uninteresting. After all,
what did it matter who had invented airplanes? It was rather more of a
shock to him when he discovered from some chance remark that she did not
remember that Oceania, four years ago, had been at war with Eastasia and
at peace with Eurasia. It was true that she regarded the whole war as a
sham; but apparently she had not even noticed that the name of the enemy
had changed. ``I thought we\textquotesingle d always been at war with
Eurasia,'' she said vaguely. It frightened him a little. The invention of
airplanes dated from long before her birth, but the switch-over in the
war had happened only four years ago, well after she was grown up. He
argued with her about it for perhaps a quarter of an hour. In the end he
succeeded in forcing her memory back until she did dimly recall that at
one time Eastasia and not Eurasia had been the enemy. But the issue
still struck her as unimportant. ``Who cares?'' she said impatiently.
``It\textquotesingle s always one bloody war after another, and one knows
the news is all lies anyway.''

\switchcolumn

从某种角度说,她比温斯顿更敏锐,不轻易相信党的宣传。一次,他提起和欧亚国的战争,没想到她随口就说,依她来看,根本就没有什么战争,这让他非常吃惊。她说,伦敦每天都有火箭弹落下,而那些火箭弹很有可能是大洋国政府自己发射的,``只是为了让人民一直与恐惧为伴''。他从未想到过这一点。她说,在两分钟仇恨会上,对她来说最困难的便是忍住不笑,这多少激起了他的嫉妒。然而,她只有在党的教条触及到她的生活时,才对它们提出质疑。其他时候,她经常轻易相信官方的虚假宣传,这仅仅因为就她而言它们是真是假并不重要。例如,她相信正像她在学校里学到的那样飞机是党发明的。(温斯顿记得自己上学时是50年代后期,党只说自己是直升机的发明者。而十多年后,朱莉亚上学时,就变成了飞机。再过一代,党会说蒸气机也是党发明的)
当他告诉她,在他出生之前,在革命爆发之前很久就已经有飞机存在时,她没有表现出一点兴趣。毕竟,究竟是谁发明飞机有什么关系呢?令他震惊的是,他偶然发现,她不记得就在四年前,大洋国还在和东亚国打仗,和欧亚国和平相处。没错,她觉得整个战争都是假的,但显然她并没有注意到敌人的名字已经发生了变化。``我想我们一直在和欧亚国打仗。''她含混地说。这让他有些惊讶。飞机的发明是她出生前很久的事,但战争对象的改变却只有四年,发生在她长大成人后。为此,他和她争论了一刻钟,最终他让她恢复记忆,她模模糊糊地回想起来确有一段时间敌人是东亚国不是欧亚国。但她觉得这并不重要。``谁在乎?''她不耐烦地说,``一场战争接着一场战争,总是这样,每个人都知道新闻是骗人的。''

\switchcolumn*

Sometimes he talked to her of the Records Department and the impudent
forgeries that he committed there. Such things did not appear to horrify
her. She did not feel the abyss opening beneath her feet at the thought
of lies becoming truths. He told her the story of Jones, Aaronson, and
Rutherford and the momentous slip of paper which he had once held
between his fingers. It did not make much impression on her. At first,
indeed, she failed to grasp the point of the story.

\switchcolumn

有几次他和她讲起记录司,说起那卑鄙的伪造工作。但这吓不倒她。即使想到谎言变成真理,她也感觉不到深渊就在她脚下打开。他告诉她琼斯、阿朗森和鲁瑟夫的事,还有那张在他手指中待过的意义重大的纸条。而这些都没给她留下什么印象。事实上,一开始,她还抓不住这些事的要点。

\switchcolumn*

``Were they friends of yours?'' she said.

\switchcolumn

``他们是你的朋友吗?''她问。

\switchcolumn*

``No, I never knew them. They were Inner Party members. Besides, they
were far older men than I was. They belonged to the old days, before the
Revolution. I barely knew them by sight.''

\switchcolumn

``我从来都不认识他们。他们是内党党员,年纪也比我大得多。他们属于革命之前的时代。我只能认出他们的长相。''

\switchcolumn*

``Then what was there to worry about? People are being killed off all the
time, aren\textquotesingle t they?''

\switchcolumn

``那么,还有什么可担心的?随时都有人被杀,不是吗?''

\switchcolumn*

He tried to make her understand. ``This was an exceptional case. It
wasn\textquotesingle t just a question of somebody being killed. Do you
realize that the past, starting from yesterday, has been actually
abolished? If it survives anywhere, it\textquotesingle s in a few solid
objects with no words attached to them, like that lump of glass there.
Already we know almost literally nothing about the Revolution and the
years before the Revolution. Every record has been destroyed or
falsified, every book has been rewritten, every picture has been
repainted, every statue and street and building has been renamed, every
date has been altered. And that process is continuing day by day and
minute by minute. History has stopped. Nothing exists except an endless
present in which the Party is always right. I \emph{know}, of course,
that the past is falsified, but it would never be possible for me to
prove it, even when I did the falsification myself. After the thing is
done, no evidence ever remains. The only evidence is inside my own mind,
and I don\textquotesingle t know with any certainty that any other human
being shares my memories. Just in that one instance, in my whole life, I
did possess actual concrete evidence \emph{after} the event---years
after it.''

\switchcolumn

他试图让她明白。``这事非同一般,这不是什么人被杀的问题。你有没有意识到,从昨天往前推,过去实际上被消失了?就算有些东西幸存下来,也只存在于几个具体的物件上,还没有文字说明,就像那块玻璃。关于革命和革命之前的事,我们已经什么都不知道了。每条记录都遭到销毁、篡改,每本书都经过重写,每幅画都被人重画,每座雕像、每条街道、每个大楼都已改名换姓,连日期都被一一修改。且这种事日复一日,每分每秒都在发生。历史停止了。除了党是永远正确、永无终结的,其他任何东西都不复存在。我当然知道过去遭到了篡改。但我永远都不可能证明这点,即使是我亲手篡改的。因为事情完成后,不会留下丁点证据。唯一的证据还在我的意识里,但我不知道其他人是否有和我一样的记忆。我整个生命中,只有那么一次,我居然在事情发生了多年之后掌握了切实的证据。''

\switchcolumn*

``And what good was that?''

\switchcolumn

``那有什么用吗?''

\switchcolumn*

``It was no good, because I threw it away a few minutes later. But if the
same thing happened today, I should keep it.''

\switchcolumn

``没什么用,因为几分钟后我就把它扔掉了。如果今天发生了同样的事,我应该将它留下。''

\switchcolumn*

``Well, I wouldn\textquotesingle t!'' said Julia. ``I\textquotesingle m
quite ready to take risks, but only for something worth while, not for
bits of old newspaper. What could you have done with it even if you had
kept it?''

\switchcolumn

``好吧,我不会这样!''朱莉亚说,``我做好了冒险的准备,但只是为那些值得的事,而不是为了几张旧报纸。就算你把它留下来,你又能怎样呢?''

\switchcolumn*

``Not much, perhaps. But it was evidence. It might have planted a few
doubts here and there, supposing that I\textquotesingle d dared to show
it to anybody. I don\textquotesingle t imagine that we can alter
anything in our own lifetime. But one can imagine little knots of
resistance springing up here and there---small groups of people banding
themselves together, and gradually growing, and even leaving a few
records behind, so that the next generation can carry on where we leave
off.''

\switchcolumn

``或许不能做什么,但它毕竟是证据。假如我敢把它拿给别人看,就可能在这里或那里播下怀疑之种。我不认为我们这辈子能改变什么。但可以想象,某个地方出现一小簇反抗力量,一小批人自发地汇集到一起,他们的数量渐渐增多,甚至留下一些记录,以便让下一代继续我们中断了的工作。''

\switchcolumn*

``I\textquotesingle m not interested in the next generation, dear.
I\textquotesingle m interested in \emph{us}.''

\switchcolumn

``我对下一代不感兴趣,亲爱的。我只对我们感兴趣。''

\switchcolumn*

``You\textquotesingle re only a rebel from the waist downwards,'' he told
her.

\switchcolumn

``你只有腰部以下是反叛的。''他对她说。

\switchcolumn*

She thought this brilliantly witty and flung her arms round him in
delight.

\switchcolumn

她觉得这句话说得很巧,便高兴地伸出手搂住他。

\switchcolumn*

In the ramifications of Party doctrine she had not the faintest
interest. Whenever he began to talk of the principles of Ingsoc,
doublethink, the mutability of the past and the denial of objective
reality, and to use Newspeak words, she became bored and confused and
said that she never paid any attention to that kind of thing. One knew
that it was all rubbish, so why let oneself be worried by it? She knew
when to cheer and when to boo, and that was all one needed. If he
persisted in talking of such subjects, she had a disconcerting habit of
falling asleep. She was one of those people who can go to sleep at any
hour and in any position. Talking to her, he realized how easy it was to
present an appearance of orthodoxy while having no grasp whatever of
what orthodoxy meant. In a way, the world-view of the Party imposed
itself most successfully on people incapable of understanding it. They
could be made to accept the most flagrant violations of reality, because
they never fully grasped the enormity of what was demanded of them, and
were not sufficiently interested in public events to notice what was
happening. By lack of understanding they remained sane. They simply
swallowed everything, and what they swallowed did them no harm, because
it left no residue behind, just as a grain of corn will pass undigested
through the body of a bird.

\switchcolumn

她对党的理论的衍生物毫无兴趣,每当他开始谈论英社的原则、双重思想、过去的易变性和对客观现实的否认,每当他开始使用新话单词,她就感到厌倦和困惑,说她从未注意过这些事情。大家都知道这些是胡说八道,为什么还要为它们担忧呢?她知道什么时候欢呼,什么时候发出嘘声,这就够了。若他坚持谈论这些,她就睡着了,这个习惯真让他无奈。她就是这种人,随时随地都能睡着。他发现和她讲话,不知正统为何又假装正统非常容易。从某种角度说,在灌输世界观上,对那些不能理解它含义的人,党做得最为成功。这些人能够接受公然违背现实的东西,因为他们从来意识不到针对自己的要求是蛮横无理的。他们对公众大事漠不关心,注意不到有事情发生。由于缺乏理解力,他们仍保持着头脑的清醒,对任何东西他们都能照单全收,由于什么都不会剩下,照单全收的东西也不能对他们产生危害,
好比一颗谷物未经消化地通过了小鸟的身体。

\switchcolumn*


\section{VI}\label{vi-1}

\switchcolumn

\section*{六}\label{ux5341ux56db}

\switchcolumn*

It had happened at last. The expected message had come. All his life, it
seemed to him, he had been waiting for this to happen.

\switchcolumn

终于,事情发生了。他期待的消息来了,对他来说,他等这事已经等了一辈子。

\switchcolumn*

He was walking down the long corridor at the Ministry, and he was almost
at the spot where Julia had slipped the note into his hand when he
became aware that someone larger than himself was walking just behind
him. The person, whoever it was, gave a small cough, evidently as a
prelude to speaking. Winston stopped abruptly and turned. It was
O\textquotesingle Brien.

\switchcolumn

当时他正走在部里长长的走廊上,在靠近朱莉亚给他纸条的那个地方,他发现身后跟着一个比他高大的人。那个人,不管是谁,轻轻地咳嗽了一下,显然是准备和他交谈。温斯顿猛地停了下来,转身一看。原来是奥布兰。

\switchcolumn*

At last they were face to face, and it seemed that his only impulse was
to run away. His heart bounded violently. He would have been incapable
of speaking. O\textquotesingle Brien, however, had continued forward in
the same movement, laying a friendly hand for a moment on
Winston\textquotesingle s arm, so that the two of them were walking side
by side. He began speaking with the peculiar grave courtesy that
differentiated him from the majority of Inner Party members.

\switchcolumn

他们终于面对面了,而他似乎只有想要逃跑的冲动。他心跳剧烈,说不出话,奥布兰却仍以同样的速度走着,他友好地伸出手在温斯顿的手臂上搭了一会儿,这样他们就能并肩而行。和大多数内党党员不同,他开始用他那独特的彬彬有礼的方式说话。

\switchcolumn*

``I had been hoping for an opportunity of talking to you,'' he said. ``I
was reading one of your Newspeak articles in the \emph{Times} the other
day. You take a scholarly interest in Newspeak, I believe?''

\switchcolumn

``我一直希望有机会找您谈谈,''他说,``前几天我在《泰晤士报》上读到您关于新话的文章,我猜您对新话很有学术兴趣,是吧?''

\switchcolumn*

Winston had recovered part of his self-possession. ``Hardly scholarly,''
he said. ``I\textquotesingle m only an amateur. It\textquotesingle s not
my subject. I have never had anything to do with the actual construction
of the language.''

\switchcolumn

温斯顿恢复了泰然自若的神情。``说不上学术兴趣,''他说,``仅仅是业余爱好,这不是我的专业。我也从来没有做过任何关于语言创作的实际工作。''

\switchcolumn*

``But you write it very elegantly,'' said O\textquotesingle Brien. ``That
is not only my own opinion. I was talking recently to a friend of yours
who is certainly an expert. His name has slipped my memory for the
moment.''

\switchcolumn

``但您的文章写得很精彩,''奥布兰说,``这不单是我个人的意见。最近刚刚和您的一位朋友聊过,他肯定是这方面的专家。一时半会儿想不起他叫什么。''

\switchcolumn*

Again Winston\textquotesingle s heart stirred painfully. It was
inconceivable that this was anything other than a reference to Syme. But
Syme was not only dead, he was abolished, an \emph{unperson}. Any
identifiable reference to him would have been mortally dangerous.
O\textquotesingle Brien\textquotesingle s remark must obviously have
been intended as a signal, a code word. By sharing a small act of
thoughtcrime he had turned the two of them into accomplices. They had
continued to stroll slowly down the corridor, but now
O\textquotesingle Brien halted. With the curious, disarming friendliness
that he always managed to put into the gesture, he resettled his
spectacles on his nose. Then he went on:

\switchcolumn

温斯顿的心再一次痛苦地抽搐起来。他在说赛姆,无法想象不是这样。然而赛姆不只死了,还被消失了,成了一个``非人''。任何有认同他之嫌的东西都有可能带来致命危险。显然,奥布兰打算发出一个信号,一个暗号。一起犯下微小的思想罪好让两个人变成同谋。他们继续在走廊中闲逛,奥布兰突然停下脚步,推了推架在鼻梁上的眼镜,很奇怪,这个姿势让人产生一种亲近感。他继续道:

\switchcolumn*

``What I had really intended to say was that in your article I noticed
you had used two words which have become obsolete. But they have only
become so very recently. Have you seen the tenth edition of the Newspeak
dictionary?''

\switchcolumn

``我真正想说的是,在您的文章中,我注意到您使用了两个已废弃不用的词。不过,它们是最近才被弃用的。您有没有看第十版的新话词典?''

\switchcolumn*

``No,'' said Winston. ``I didn\textquotesingle t think it had been issued
yet. We are still using the ninth in the Records Department.''

\switchcolumn

``没,''温斯顿答,``我想它还没有发行。我们在记录司使用的仍然是第九版。''

\switchcolumn*

``The tenth edition is not due to appear for some months, I believe. But
a few advance copies have been circulated. I have one myself. It might
interest you to look at it, perhaps?''

\switchcolumn

``第十版用不了几个月就会问世。不过,少部分先行版已经开始流通了。我自己就有一本。也许您有兴趣看看?''

\switchcolumn*

``Very much so,'' said Winston, immediately seeing where this tended.

\switchcolumn

``非常有兴趣。''温斯顿说,他立即领会了其中的意图。

\switchcolumn*

``Some of the new developments are most ingenious. The reduction in the
number of verbs---that is the point that will appeal to you, I think.
Let me see, shall I send a messenger to you with the dictionary? But I
am afraid I invariably forget anything of that kind. Perhaps you could
pick it up at my flat at some time that suited you? Wait. Let me give
you my address.''

\switchcolumn

``部分新发展很有独创性。动词的数目被削减。我想这点很吸引您。让我考虑一下,要不要派人将词典给您送去?可这样的事情我总是想不起来。也许,您能抽空到我住的地方来拿,您看合适吗?等等,我把地址告诉您。''

\switchcolumn*

They were standing in front of a telescreen. Somewhat absent-mindedly
O\textquotesingle Brien felt two of his pockets and then produced a
small leather-covered notebook and a gold ink pencil. Immediately
beneath the telescreen, in such a position that anyone who was watching
at the other end of the instrument could read what he was writing, he
scribbled an address, tore out the page, and handed it to Winston.

\switchcolumn

他们刚好站在电屏跟前。奥布兰多少有些心不在焉,他摸了摸他的两个口袋,掏出一个皮面小本和一支金色的墨水笔。考虑到他的位置,电屏那端的人可以清楚看到他写的是什么,他将地址写好,撕下来,交给温斯顿。

\switchcolumn*

``I am usually at home in the evenings,'' he said. ``If not, my servant
will give you the dictionary.''

\switchcolumn

``通常,我晚上都待在家里。''他说,``如果没有,我的服务人员会将词典带给您。''

\switchcolumn*

He was gone, leaving Winston holding the scrap of paper, which this time
there was no need to conceal. Nevertheless he carefully memorized what
was written on it, and some hours later dropped it into the memory hole
along with a mass of other papers.

\switchcolumn

他走了,留下温斯顿一个人站在那里,手里拿着那张纸。这次,他不需要将它藏起来,但他还是小心地记住上面的内容,几个小时后,将它和其他一大堆纸一起扔到记忆洞里。

\switchcolumn*

They had been talking to one another for a couple of minutes at the
most. There was only one meaning that the episode could possibly have.
It had been contrived as a way of letting Winston know
O\textquotesingle Brien\textquotesingle s address. This was necessary,
because except by direct inquiry it was never possible to discover where
anyone lived. There were no directories of any kind. ``If you ever want
to see me, this is where I can be found,'' was what
O\textquotesingle Brien had been saying to him. Perhaps there would even
be a message concealed somewhere in the dictionary. But at any rate, one
thing was certain. The conspiracy that he had dreamed of did exist, and
he had reached the outer edges of it.

\switchcolumn

两个人最多交谈了几分钟。整件事可能只有一个含义,即通过这种设计让温斯顿知道奥布兰的地址。这很有必要,因为除了直接询问,要知道某人住在哪里是不可能的,没有诸如通讯录之类的东西。``如果你想见我,可以到这里找我。''也许词典的某处藏着某个信息。不管怎样,有一件事可以确定。他所期待的阴谋是存在的,他已经碰触到它的边缘。

\switchcolumn*

He knew that sooner or later he would obey
O\textquotesingle Brien\textquotesingle s summons. Perhaps tomorrow,
perhaps after a long delay---he was not certain. What was happening was
only the working-out of a process that had started years ago. The first
step had been a secret, involuntary thought; the second had been the
opening of the diary. He had moved from thoughts to words, and now from
words to actions. The last step was something that would happen in the
Ministry of Love. He had accepted it. The end was contained in the
beginning. But it was frightening; or, more exactly, it was like a
foretaste of death, like being a little less alive. Even while he was
speaking to O\textquotesingle Brien, when the meaning of the words had
sunk in, a chilly shuddering feeling had taken possession of his body.
He had the sensation of stepping into the dampness of a grave, and it
was not much better because he had always known that the grave was there
and waiting for him.

\switchcolumn

他知道,他早晚都要服从奥布兰的召唤。可能是明天,可能是很久之后——他不能确定。刚刚发生的事无非是多年之前就开始进行的事情的一个表现。第一步是秘密的、偶然萌生的念头。第二步是记日记,将思想变成文字。而现在又要将文字变成行动。至于最后一步则会发生在仁爱部中,他已然接受了这个结局,它就包含在开始之中。但这让人恐惧;或者确切地说,它就像预先品尝了死亡的味道,又少活了几天。即使在和奥布兰讲话的时候,当他完全领会了话中的含义,他仍感到一股寒意,不由得浑身颤抖,就好像踏入潮湿阴冷的坟墓。就算他明白坟墓一直等在那里,他也没有感觉好些。

\switchcolumn*


\section{VII}\label{vii-1}

\switchcolumn

\section*{七}\label{ux5341ux4e94}

\switchcolumn*

Winston had woken up with his eyes full of tears. Julia rolled sleepily
against him, murmuring something that might have been
``What\textquotesingle s the matter?''

\switchcolumn

温斯顿醒来时,眼里噙满了泪。朱莉亚睡意蒙眬地靠近他,喃喃地说着,好像在问:``怎么了?''

\switchcolumn*

``I dreamt---'' he began, and stopped short. It was too complex to be put
into words. There was the dream itself, and there was a memory connected
with it that had swum into his mind in the few seconds after waking.

\switchcolumn

``我梦见——''他欲言又止。这个梦太复杂了,无法用语言描述。除了梦本身,还有与梦有关的记忆。它们在他醒来后的几秒钟里浮现在他的脑海中。

\switchcolumn*

He lay back with his eyes shut, still sodden in the atmosphere of the
dream. It was a vast, luminous dream in which his whole life seemed to
stretch out before him like a landscape on a summer evening after rain.
It had all occurred inside the glass paperweight, but the surface of the
glass was the dome of the sky, and inside the dome everything was
flooded with clear soft light in which one could see into interminable
distances. The dream had also been comprehended by---indeed, in some
sense it had consisted in---a gesture of the arm made by his mother, and
made again thirty years later by the Jewish woman he had seen on the
news film, trying to shelter the small boy from the bullets, before the
helicopters blew them both to pieces.

\switchcolumn

他重新躺下,缓缓地闭上双眼,沉浸在梦的氛围里。这是一个壮阔、明亮的梦。他的人生就像夏夜的雨后风景呈现在眼前。所有的一切都发生在玻璃镇纸中。玻璃的表面宛若苍穹,在它之下每样东西都覆着温柔清澈的光,一眼望去无边无涯。这场梦可以用他母亲的手臂姿势概括,事实上它正是由他母亲手臂的某个动作构成的。三十年后,他在一部新上映的电影中看到一个犹太女人为了保护自己的孩子不被子弹射到做出了这个动作,而这之后她们仍被直升机炸得粉碎。

\switchcolumn*

``Do you know,'' he said, ``that until this moment I believed I had
murdered my mother?''

\switchcolumn

``你知道吗?''他说,``直到现在,我都觉得我的母亲是被我害死的。''

\switchcolumn*

``Why did you murder her?'' said Julia, almost asleep.

\switchcolumn

``你为什么要杀自己的母亲?''朱莉亚问,她差不多还是睡着的。

\switchcolumn*

``I didn\textquotesingle t murder her. Not physically.''

\switchcolumn

``我没杀她,这不是肉体意义上的。''

\switchcolumn*

In the dream he had remembered his last glimpse of his mother, and
within a few moments of waking the cluster of small events surrounding
it had all come back. It was a memory that he must have deliberately
pushed out of his consciousness over many years. He was not certain of
the date, but he could not have been less than ten years old, possibly
twelve, when it had happened.

\switchcolumn

在梦里,他回忆起他看母亲的最后一眼,几分钟后,他醒了,和这情境相关的细微小事一簇簇地涌了上来。正是这个记忆,多少年来,他一直有意识地将它从意识中抹去。他无法确定具体的日期,但当时他至少有十岁,可能是十二岁。

\switchcolumn*

His father had disappeared some time earlier; how much earlier, he could
not remember. He remembered better the rackety, uneasy circumstances of
the time: the periodical panics about air raids and the sheltering in
Tube stations, the piles of rubble everywhere, the unintelligible
proclamations posted at street corners, the gangs of youths in shirts
all the same color, the enormous queues outside the bakeries, the
intermittent machine-gun fire in the distance---above all, the fact that
there was never enough to eat. He remembered long afternoons spent with
other boys in scrounging round dustbins and rubbish heaps, picking out
the ribs of cabbage leaves, potato peelings, sometimes even scraps of
stale breadcrust from which they carefully scraped away the cinders; and
also in waiting for the passing of trucks which traveled over a certain
route and were known to carry cattle feed, and which, when they jolted
over the bad patches in the road, sometimes spilt a few fragments of
oilcake.

\switchcolumn

他的父亲很早就失踪了,到底是什么时候,他记不清了。但他记得当时的环境艰难而痛苦:周期性的空袭让人惊恐不已,人们到地铁站里寻求庇护,到处都是残砖烂瓦,街角处贴着他无法理解的公告。少年们成群结队,穿着同样颜色的衣服,面包房前摆起了长龙,不时会有机枪声从远处传来。最重要的是,人们永远都吃不饱。他还记得一到下午,他就会花很长时间和其他男孩子一起在垃圾桶和废品堆里寻找卷心菜帮和马铃薯皮,有时还能翻出陈面包块,他们会非常小心地将上面的炉灰擦掉;他们清楚卡车的行驶路线,知道上面装着喂牛的饲料。他们等卡车开来,在经过那些坑洼不平的路段时,偶尔会从车上掉下几块油糕。

\switchcolumn*

When his father disappeared, his mother did not show any surprise or any
violent grief, but a sudden change came over her. She seemed to have
become completely spiritless. It was evident even to Winston that she
was waiting for something that she knew must happen. She did everything
that was needed---cooked, washed, mended, made the bed, swept the floor,
dusted the mantelpiece---always very slowly and with a curious lack of
superfluous motion, like an artist\textquotesingle s lay-figure moving
of its own accord. Her large shapely body seemed to relapse naturally
into stillness. For hours at a time she would sit almost immobile on the
bed, nursing his young sister, a tiny, ailing, very silent child of two
or three, with a face made simian by thinness. Very occasionally she
would take Winston in her arms and press him against her for a long time
without saying anything. He was aware, in spite of his youthfulness and
selfishness, that this was somehow connected with the never-mentioned
thing that was about to happen.

\switchcolumn

父亲失踪时,母亲并没有表现出强烈的惊讶或悲痛,不过,她就像变了一个人,看上去宛若行尸走肉。就连温斯顿也发现她在等一件她知道注定要发生的事。她烧饭、洗衣、缝纫、铺床、扫地、清理壁炉台上的灰尘,每件需要她做的事她都做了,但却做得很慢,也没有多余的动作,好比一架自动行走的艺术家的人体模型。她高挑匀称的身体似乎能自行静止,在床边一坐就是好几个小时,几乎一动不动地照顾他的妹妹,他的妹妹只有二三岁,弱小、多病、非常安静,脸瘦得像只猴子。偶尔,她会将温斯顿紧紧地搂在怀里,很久很久不发一言。尽管他很小,很自私,但他依然觉察到这和即将发生却从未被提起的事有关。

\switchcolumn*

He remembered the room where they lived, a dark, close-smelling room
that seemed half filled by a bed with a white counterpane. There was a
gas ring in the fender, and a shelf where food was kept, and on the
landing outside there was a brown earthenware sink, common to several
rooms. He remembered his mother\textquotesingle s statuesque body
bending over the gas ring to stir at something in a saucepan. Above all
he remembered his continuous hunger, and the fierce sordid battles at
mealtimes. He would ask his mother naggingly, over and over again, why
there was not more food, he would shout and storm at her (he even
remembered the tones of his voice, which was beginning to break
prematurely and sometimes boomed in a peculiar way), or he would attempt
a sniveling note of pathos in his efforts to get more than his share.
His mother was quite ready to give him more than his share. She took it
for granted that he, ``the boy,'' should have the biggest portion; but
however much she gave him he invariably demanded more. At every meal she
would beseech him not to be selfish and to remember that his little
sister was sick and also needed food, but it was no use. He would cry
out with rage when she stopped ladling, he would try to wrench the
saucepan and spoon out of her hands, he would grab bits from his
sister\textquotesingle s plate. He knew that he was starving the other
two, but he could not help it; he even felt that he had a right to do
it. The clamorous hunger in his belly seemed to justify him. Between
meals, if his mother did not stand guard, he was constantly pilfering at
the wretched store of food on the shelf.

\switchcolumn

他还记得他们住的那间屋子,它阴暗、封闭,一张铺着白床单的床就差不多占了一半的空间。围炉中放着煤气灶和食品架,屋外的台阶上有个公用的棕色的陶瓷池子。他记得母亲在煤气灶前弯着雕塑般的身子搅动着锅里的东西。让他印象最深的是,他总觉得很饿,吃起饭来就像打仗一样。他死死地缠着母亲,一遍又一遍地问为什么没有更多的食物,他还经常对她大吼大叫(他甚至还记得自己当时的声音,由于过早进入变声期,有时竟洪亮得出奇)。为了能多分到些吃的,他尝试着发出可怜巴巴地啜泣,而母亲也愿意多分给他,认为``男孩''理应得到最多的一份。但不管分给他多少,他仍坚持更多。因此每到吃饭的时候,母亲总会央求他不要自私,不要忘了妹妹有病,需要食物,但这没有用。她一停止盛东西,他就愤怒地大声喊叫,还试图把她手里的锅和勺夺过来,或者从妹妹的盘子里抢过来一些。他知道他让她们挨了饿,但他无可奈何,他甚至觉得他有权这样做。他饿得咕咕叫的肚子就好像在为他的行为进行辩护。在两顿饭的间隔,若母亲不注意,他还经常会从架子上偷东西吃,尽管那里的食物少得可怜。

\switchcolumn*

One day a chocolate ration was issued. There had been no such issue for
weeks or months past. He remembered quite clearly that precious little
morsel of chocolate. It was a two-ounce slab (they still talked about
ounces in those days) between the three of them. It was obvious that it
ought to be divided into three equal parts. Suddenly, as though he were
listening to somebody else, Winston heard himself demanding in a loud
booming voice that he should be given the whole piece. His mother told
him not to be greedy. There was a long, nagging argument that went round
and round, with shouts, whines, tears, remonstrances, bargainings. His
tiny sister, clinging to her mother with both hands, exactly like a baby
monkey, sat looking over her shoulder at him with large, mournful eyes.
In the end his mother broke off three-quarters of the chocolate and gave
it to Winston, giving the other quarter to his sister. The little girl
took hold of it and looked at it dully, perhaps not knowing what it was.
Winston stood watching her for a moment. Then with a sudden swift spring
he had snatched the piece of chocolate out of his
sister\textquotesingle s hand and was fleeing for the door.

\switchcolumn

一天,定量供应的巧克力发了下来。过去的几个星期、几个月都没有发。他记得非常清楚,那是一小片巧克力,很珍贵。两盎司(那时仍使用盎司)一片,三个人分。按理,应该平均地分成三份。但突然,就好像受到了什么人的指示,温斯顿听到自己像打雷那样大喊要求将整块巧克力都分给自己。母亲要他别贪心。他们争执了很久,叫喊、哀诉、哭泣、抗议、乞求。他瘦小的妹妹用双手抱住母亲,像极了小猴子。她坐着,从母亲的肩膀处望过来,眼睛大而忧伤。最后,母亲将巧克力掰开,四分之三分给温斯顿,剩下的四分之一分给他的妹妹。小女孩拿着它发呆,也许不知道它是什么。温斯顿看了她一会儿,之后,他突然迅速地跳了起来,一把抢走妹妹手中的巧克力,冲向门外。

\switchcolumn*

``Winston, Winston!'' his mother called after him. ``Come back! Give your
sister back her chocolate!''

\switchcolumn

``温斯顿,温斯顿!''母亲在后面叫他,``快回来!把巧克力还给妹妹!''

\switchcolumn*

He stopped, but he did not come back. His mother\textquotesingle s
anxious eyes were fixed on his face. Even now she was thinking about the
thing, he did not know what it was, that was on the point of happening.
His sister, conscious of having been robbed of something, had set up a
feeble wail. His mother drew her arm round the child and pressed its
face against her breast. Something in the gesture told him that his
sister was dying. He turned and fled down the stairs, with the chocolate
growing sticky in his hand.

\switchcolumn

他停下来,却没回去。母亲看着他的脸,目光里充满忧虑。即使现在他仍惦记着那件事,但就算它马上发生,他也不知道最后发生的究竟是什么。他的妹妹意识到东西被抢走,无力地哭了几下。母亲紧紧地抱住她,把她的脸贴在自己的胸口。这个姿势里有什么东西让温斯顿觉得妹妹就要死了。他转身跑下楼梯,手里的巧克力变得黏糊糊的。

\switchcolumn*

He never saw his mother again. After he had devoured the chocolate he
felt somewhat ashamed of himself and hung about in the streets for
several hours, until hunger drove him home. When he came back his mother
had disappeared. This was already becoming normal at that time. Nothing
was gone from the room except his mother and his sister. They had not
taken any clothes, not even his mother\textquotesingle s overcoat. To
this day he did not know with any certainty that his mother was dead. It
was perfectly possible that she had merely been sent to a forced-labor
camp. As for his sister, she might have been removed, like Winston
himself, to one of the colonies for homeless children (Reclamation
Centers, they were called) which had grown up as a result of the civil
war; or she might have been sent to the labor camp along with his
mother, or simply left somewhere or other to die.

\switchcolumn

他再也没看到他的母亲。在吞掉巧克力后,他有些惭愧。他在街上游荡了几个小时,直到饥饿驱使他回了家。当他回到家时,母亲已经失踪了。在当时这已经是正常现象。房间里什么都没丢,除了母亲和妹妹。他们没将衣服带走,没带走母亲的外套。今天他仍不敢确定母亲已经死了,很有可能她只是被送进劳改营。至于妹妹,也许像他一样,被送到收留无家可归的儿童的地方(他们称它为感化院),那里因内战而壮大。还有可能她和母亲一起被送进了劳改营,或者被丢到哪里,或者就那么死了。

\switchcolumn*

The dream was still vivid in his mind, especially the enveloping,
protecting gesture of the arm in which its whole meaning seemed to be
contained. His mind went back to another dream of two months ago.
Exactly as his mother had sat on the dingy white-quilted bed, with the
child clinging to her, so she had sat in the sunken ship, far underneath
him and drowning deeper every minute, but still looking up at him
through the darkening water.

\switchcolumn

这个梦在他的脑海里鲜活生动,尤其是用手臂围住什么的保护性的姿势,它似乎将整个梦的含义都容纳其中。他的思绪又回到两个月前的另一场梦上。那次母亲坐在一艘沉船中,看上去和坐在肮脏的铺着白色床单的床边一样,怀里的孩子紧紧贴着她的胸口,她在他下面很远的地方,每一分钟都在下沉,但她仍然透过越来越暗的海水望向他。

\switchcolumn*

He told Julia the story of his mother\textquotesingle s disappearance.
Without opening her eyes she rolled over and settled herself into a more
comfortable position.

\switchcolumn

他将母亲失踪的事告诉朱莉亚。她闭着眼翻了个身,好让自己睡得更舒服。

\switchcolumn*

``I expect you were a beastly little swine in those days,'' she said
indistinctly. ``All children are swine.''

\switchcolumn

``我猜那时的你就是头残忍的猪,''她嘟嘟囔囔地说,``所有的小孩都是猪。''

\switchcolumn*

``Yes. But the real point of the story---''

\switchcolumn

``没错,但这并不是这件事的真正含义——''

\switchcolumn*

From her breathing it was evident that she was going off to sleep again.
He would have liked to continue talking about his mother. He did not
suppose, from what he could remember of her, that she had been an
unusual woman, still less an intelligent one; and yet she had possessed
a kind of nobility, a kind of purity, simply because the standards that
she obeyed were private ones. Her feelings were her own, and could not
be altered from outside. It would not have occurred to her that an
action which is ineffectual thereby becomes meaningless. If you loved
someone, you loved him, and when you had nothing else to give, you still
gave him love. When the last of the chocolate was gone, his mother had
clasped the child in her arms. It was no use, it changed nothing, it did
not produce more chocolate, it did not avert the child\textquotesingle s
death or her own; but it seemed natural to her to do it. The refugee
woman in the boat had also covered the little boy with her arm, which
was no more use against the bullets than a sheet of paper. The terrible
thing that the Party had done was to persuade you that mere impulses,
mere feelings, were of no account, while at the same time robbing you of
all power over the material world. When once you were in the grip of the
Party, what you felt or did not feel, what you did or refrained from
doing, made literally no difference. Whatever happened you vanished, and
neither you nor your actions were ever heard of again. You were lifted
clean out of the stream of history. And yet to the people of only two
generations ago, this would not have seemed all-important, because they
were not attempting to alter history. They were governed by private
loyalties which they did not question. What mattered were individual
relationships, and a completely helpless gesture, an embrace, a tear, a
word spoken to a dying man, could have value in itself. The proles, it
suddenly occurred to him, had remained in this condition. They were not
loyal to a party or a country or an idea, they were loyal to one
another. For the first time in his life he did not despise the proles or
think of them merely as an inert force which would one day spring to
life and regenerate the world. The proles had stayed human. They had not
become hardened inside. They had held onto the primitive emotions which
he himself had to relearn by conscious effort. And in thinking this he
remembered, without apparent relevance, how a few weeks ago he had seen
a severed hand lying on the pavement and had kicked it into the gutter
as though it had been a cabbage stalk.

\switchcolumn

从她的呼吸来看,她又睡着了。他想继续讲他的母亲。回想记忆中的她,他不能说她是个不平凡的女人,她没别人聪明,但她却拥有高贵、纯洁的品质。她遵循着自己的行为准则,她有属于自己的情感,不会因为外界的什么东西发生改变。她从未想过无用之举无意义。若你爱某人,就去爱他,当你什么都不能给他时,你还可以给他爱。最后一块巧克力被抢走了,母亲把孩子抱在怀里。这没用,什么都改变不了,她不能再多变出一块巧克力,也不能让自己和孩子逃脱死亡。但这对她而言,似乎自然而然。轮船上那个逃难的女人同样用手臂护住她的孩子,在抵御子弹上,她并不比一张纸有用多少。恐怖的是党的所作所为让你相信,仅凭冲动,仅凭感情什么都做不了。而同时它又将你身上所有能左右物质世界的力量剥除。一旦你落入党的掌控,不管你有没有感觉,不管你做一件事还是阻止一件事,都没有什么区别。你终究会消失,你和你做的事都不再被人听到。在历史的大潮中,你将被清除得一干二净。但对两代以前的人来说,这好像不怎么重要。因为他们没打算篡改历史。他们有自己的行为准则,他们遵循这准则,毫不怀疑。个人与个人的关系非常重要,那些无用的动作,一个拥抱,一滴眼泪,对濒死之人说的一句话,都有其价值所在。他突然想到,群众仍是如此,他们不为某个党效忠,也不为某个国家、某个思想效忠,他们忠诚于彼此。他有生以来第一次没有轻视群众,他第一次将他们当做早晚会被唤醒、会令世界重生的潜在力量。群众仍保留着人性,他们的内心尚未麻木。他们仍保留着最初的情感,而他自己却要通过努力重新学会这种情感。想到这里,他记起一件毫无关联的事,几个星期前,他在路边看到一只断手,他将他踢到沟里,就像踢一棵白菜帮。

\switchcolumn*

``The proles are human beings,'' he said aloud. ``We are not human.''

\switchcolumn

``群众是人,''他大声说,``我们不是人。''

\switchcolumn*

``Why not?'' said Julia, who had woken up again.

\switchcolumn

``为什么不是?''朱莉亚醒了,说。

\switchcolumn*

He thought for a little while. ``Has it ever occurred to you,'' he said,
``that the best thing for us to do would be simply to walk out of here
before it\textquotesingle s too late, and never see each other again?''

\switchcolumn

他沉吟片刻。``你有没有想过,''他说,``在事情变糟之前,我们从这儿出去,再也不见面。''

\switchcolumn*

``Yes, dear, it has occurred to me, several times. But
I\textquotesingle m not going to do it, all the same.''

\switchcolumn

``想过,亲爱的,想过很多次。但我仍然不想这么做。''

\switchcolumn*

``We\textquotesingle ve been lucky,'' he said, ``but it
can\textquotesingle t last much longer. You\textquotesingle re young.
You look normal and innocent. If you keep clear of people like me, you
might stay alive for another fifty years.''

\switchcolumn

``我们很幸运,''他说,``但它不能持续多久了。你年轻,你看上去又自然又纯洁。如果你能避开我这种人,你可以再活上五十年。''

\switchcolumn*

``No. I\textquotesingle ve thought it all out. What you do,
I\textquotesingle m going to do. And don\textquotesingle t be too
downhearted. I\textquotesingle m rather good at staying alive.''

\switchcolumn

``不,这我都想过。你做什么,我就跟着你做什么。不要太灰心了,我很擅长生存。''

\switchcolumn*

``We may be together for another six months---a
year---there\textquotesingle s no knowing. At the end
we\textquotesingle re certain to be apart. Do you realize how utterly
alone we shall be? When once they get hold of us there will be nothing,
literally nothing, that either of us can do for the other. If I confess,
they\textquotesingle ll shoot you, and if I refuse to confess
they\textquotesingle ll shoot you just the same. Nothing that I can do
or say, or stop myself from saying, will put off your death for as much
as five minutes. Neither of us will even know whether the other is alive
or dead. We shall be utterly without power of any kind. The one thing
that matters is that we shouldn\textquotesingle t betray one another,
although even that can\textquotesingle t make the slightest difference.''

\switchcolumn

``我们能在一起六个月——一年——没人知道,最终我们一定会分开。你有没有意识到我们会处在怎样绝对的孤独中?一旦他们抓到我们,我们都没法为对方做任何事。如果我承认了,他们就会毙掉你。我拒绝承认,他们依然会毙掉你。无论我做什么,说什么,或者什么都不说,都不能把你的死推迟五分钟。我们谁也不会知道对方是死是活,我们会完全无能为力。有一件事非常重要,那就是我们不应出卖对方,尽管这样结果也不会有什么不同。''

\switchcolumn*

``If you mean confessing,'' she said, ``we shall do that, right enough.
Everybody always confesses. You can\textquotesingle t help it. They
torture you.''

\switchcolumn

``假如你指的是坦白,''她说,``我们还是会坦白的。没错,每个人都是这样。你挺不住。他们会拷打你。''

\switchcolumn*

``I don\textquotesingle t mean confessing. Confession is not betrayal.
What you say or do doesn\textquotesingle t matter; only feelings matter.
If they could make me stop loving you---that would be the real
betrayal.''

\switchcolumn

``我不是指坦白,坦白不是出卖。不管你说什么做什么都没关系:除了感情。如果他们让我停止爱你,那才真的是出卖。''

\switchcolumn*

She thought it over. ``They can\textquotesingle t do that,'' she said
finally. ``It\textquotesingle s the one thing they can\textquotesingle t
do. They can make you say anything---\emph{anything}---but they
can\textquotesingle t make you believe it. They can\textquotesingle t
get inside you.''

\switchcolumn

她想了想。``他们做不到,''最后她说,``这是唯一一件他们做不到的事。他们能让你说出任何事情——任何事情——但他们不能让你相信它。他们进不去你心里。''

\switchcolumn*

``No,'' he said a little more hopefully, ``no; that\textquotesingle s quite
true. They can\textquotesingle t get inside you. If you can \emph{feel}
that staying human is worth while, even when it can\textquotesingle t
have any result whatever, you\textquotesingle ve beaten them.''

\switchcolumn

``对,''他多了几分希望,``没错,这是事实。他们不能进入到你的心里。如果你觉得保留人性是值得的,就算它不会产生任何结果,你都战胜了他们。''

\switchcolumn*

He thought of the telescreen with its never-sleeping ear. They could spy
upon you night and day, but if you kept your head you could still outwit
them. With all their cleverness they had never mastered the secret of
finding out what another human being was thinking. Perhaps that was less
true when you were actually in their hands. One did not know what
happened inside the Ministry of Love, but it was possible to guess:
tortures, drags, delicate instruments that registered your nervous
reactions, gradual wearing-down by sleeplessness and solitude and
persistent questioning. Facts, at any rate, could not be kept hidden.
They could be tracked down by inquiry, they could be squeezed out of you
by torture. But if the object was not to stay alive but to stay human,
what difference did it ultimately make? They could not alter your
feelings; for that matter you could not alter them yourself, even if you
wanted to. They could lay bare in the utmost detail everything that you
had done or said or thought; but the inner heart, whose workings were
mysterious even to yourself, remained impregnable.

\switchcolumn

他想起电屏,想起它永无休止地监听。他们可以夜以继日地监视你,但若保持清醒,你仍然能瞒骗过他们。他们都非常聪明,可他们仍然无法掌握挖出人们秘密思想的办法。也许当你真的落到他们手里后,情况就不同了。没人知道仁爱部里发生了什么,但人们猜得出来:拷打、下药、记录你神经反应的精密仪器、不让你睡觉,一点点地削弱你。把你关在单独囚室,没完没了地进行拷问。事实上,不管怎样,没有什么能隐藏起来。他们将通过问讯追查到底。但如果你的目的不是活着而是保留人性,那会有什么不同呢?他们不能改变你的情感,就算你自己也想改变。他们能让你所做所说所想的每个细节都暴露出来,除了你的内心。即使对你自己而言,你的内心仍然是神秘的、坚不可攻的。

\switchcolumn*


\section{VIII}\label{viii-1}

\switchcolumn

\section*{八}\label{ux5341ux516d}

\switchcolumn*

They had done it, they had done it at last!

\switchcolumn

他们来了,他们终于来了!

\switchcolumn*

The room they were standing in was long-shaped and softly lit. The
telescreen was dimmed to a low murmur; the richness of the dark-blue
carpet gave one the impression of treading on velvet. At the far end of
the room O\textquotesingle Brien was sitting at a table under a
green-shaded lamp, with a mass of papers on either side of him. He had
not bothered to look up when the servant showed Julia and Winston in.

\switchcolumn

他们站在一间长方形的光线柔和的屋子里。电屏的声音被调得很低,听不清楚。地上铺着华美的深蓝色地毯,很厚,人踩上去就像踩到了天鹅绒上。在屋子的另一端,奥布兰正坐在桌子旁,桌子上的台灯罩着绿色的灯罩,桌子两边放着一大摞文件。朱莉亚和温斯顿被带了进来,奥布兰没有抬头。

\switchcolumn*

Winston\textquotesingle s heart was thumping so hard that he doubted
whether he would be able to speak. They had done it, they had done it at
last, was all he could think. It had been a rash act to come here at
all, and sheer folly to arrive together; though it was true that they
had come by different routes and only met on
O\textquotesingle Brien\textquotesingle s doorstep. But merely to walk
into such a place needed an effort of the nerve. It was only on very
rare occasions that one saw inside the dwelling places of the Inner
Party, or even penetrated into the quarter of the town where they lived.
The whole atmosphere of the huge block of flats, the richness and
spaciousness of everything, the unfamiliar smells of good food and good
tobacco, the silent and incredibly rapid lifts sliding up and down, the
white-jacketed servants hurrying to and fro---everything was
intimidating. Although he had a good pretext for coming here, he was
haunted at every step by the fear that a black-uniformed guard would
suddenly appear from round the corner, demand his papers, and order him
to get out. O\textquotesingle Brien\textquotesingle s servant, however,
had admitted the two of them without demur. He was a small, dark-haired
man in a white jacket, with a diamond-shaped, completely expressionless
face which might have been that of a Chinese. The passage down which he
led them was softly carpeted, with cream-papered walls and white
wainscoting, all exquisitely clean. That too was intimidating. Winston
could not remember ever to have seen a passageway whose walls were not
grimy from the contact of human bodies.

\switchcolumn

温斯顿的心剧烈地跳着,他怀疑自己是不是还能讲话。他只想到一件事:``他们来了,他们终于来了。''到这儿来简直太冒失了,更何况还是两个人一起,这更加愚蠢。尽管他们走的路线不同,且直到奥布兰家门口才碰面。但仅仅是到这种地方来就需要鼓足勇气。只有在极其偶然的情况下,你才有机会看到内党党员的住处到底是什么样子,或者说你才有机会穿过他们的居住区。公寓很大,非常气派,每样东西都富丽堂皇,食物和烟草散发着陌生的香味,电梯安安静静地上上下下,快得令人难以置信,穿着白色上衣的服务人员正忙忙碌碌——这一切都让人心生畏惧。虽然到这里的借口很巧妙,但他每走一步都会担心黑色制服的警卫会突然出现,管他要证件,将他赶走。但奥布兰的服务人员却毫不迟疑地允许他们进去。那是个小个子,一头黑发,穿着白色的上衣,长着一张菱形的脸,什么表情都没有,也许是中国人。他带着他们穿过走廊,地毯很软,墙壁上糊着奶油色的墙纸,护墙板刷成了白色,所有这些都干干净净,同样令人望而生畏。而温斯顿想不起来有哪堵墙没有被人蹭脏过。

\switchcolumn*

O\textquotesingle Brien had a slip of paper between his fingers and
seemed to be studying it intently. His heavy face, bent down so that one
could see the line of the nose, looked both formidable and intelligent.
For perhaps twenty seconds he sat without stirring. Then he pulled the
speakwrite toward him and rapped out a message in the hybrid jargon of
the Ministries:

\switchcolumn

奥布兰捏着一张纸条,好像正在专心研究。他的脸很大,头低着,看得清他鼻子的轮廓,他的样子充满智慧,令人敬畏。大约有二十秒,他坐在那里一动不动。之后,他将语音记录器拉出来,用各部的混合行话说道:

\switchcolumn*

``Items one comma five comma seven approved fullwise stop suggestion
contained item six doubleplus ridiculous verging crimethink cancel stop
unproceed constructionwise antegetting plusful estimates machinery
overheads stop end message.''

\switchcolumn

``项目一逗号五逗号七批准句号六项包含的建议加倍荒唐接近思想罪取消句号未处理建设性付款加上满足预算机械装置一般费用句号结束消息。''

\switchcolumn*

He rose deliberately from his chair and came toward them across the
soundless carpet. A little of the official atmosphere seemed to have
fallen away from him with the Newspeak words, but his expression was
grimmer than usual, as though he were not pleased at being disturbed.
The terror that Winston already felt was suddenly shot through by a
streak of ordinary embarrassment. It seemed to him quite possible that
he had simply made a stupid mistake. For what evidence had he in reality
that O\textquotesingle Brien was any kind of political conspirator?
Nothing but a flash of the eyes and a single equivocal remark; beyond
that, only his own secret imaginings, founded on a dream. He could not
even fall back on the pretense that he had come to borrow the
dictionary, because in that case Julia\textquotesingle s presence was
impossible to explain. As O\textquotesingle Brien passed the telescreen
a thought seemed to strike him. He stopped, turned aside, and pressed a
switch on the wall. There was a sharp snap. The voice had stopped.

\switchcolumn

他特意从椅子上站起来,安静地穿过地毯走到他们面前。讲完那串新话,他的官威似乎放下了一些,但他的神情却比平时更加阴沉,好像不高兴被人打扰。温斯顿本来就很恐惧,此时这恐惧里又增添了几分尴尬,很有可能,这是个彻头彻尾的、愚蠢的错误。他有什么证据能证明奥布兰真的是那种政治密谋者呢?除了一闪即过的目光,一句模棱两可的话,他什么都没有。此外,他拥有的只是一些秘密的念头,而这又完全建立在幻想之上。他已无路可退,他甚至不能拿借词典当借口,因为这无法解释为什么朱莉亚也在场。奥布兰走到电屏前,好像突然想起什么,停下来,转身按下墙上的一个按钮。啪的一声,电屏里的声音停止了。

\switchcolumn*

Julia uttered a tiny sound, a sort of squeak of surprise. Even in the
midst of his panic, Winston was too much taken aback to be able to hold
his tongue.

\switchcolumn

朱莉亚非常惊讶,轻轻地尖叫了一声。温斯顿虽然害怕,但他过于震惊以至于控制不住说出了口:

\switchcolumn*

``You can turn it off!'' he said.

\switchcolumn

``您可以关掉它!''

\switchcolumn*

``Yes,'' said O\textquotesingle Brien, ``we can turn it off. We have that
privilege.''

\switchcolumn

``没错,''奥布兰说,``我们能关掉它。我们有特权。''

\switchcolumn*

He was opposite them now. His solid form towered over the pair of them,
and the expression on his face was still indecipherable. He was waiting,
somewhat sternly, for Winston to speak, but about what? Even now it was
quite conceivable that he was simply a busy man wondering irritably why
he had been interrupted. Nobody spoke. After the stopping of the
telescreen the room seemed deadly silent. The seconds marched past,
enormous. With difficulty Winston continued to keep his eyes fixed on
O\textquotesingle Brien\textquotesingle s. Then suddenly the grim face
broke down into what might have been the beginnings of a smile. With his
characteristic gesture O\textquotesingle Brien resettled his spectacles
on his nose.

\switchcolumn

此时,他正对着他们。他身材魁梧,比他们高出很多,他的表情仍让人捉摸不透。他有些严厉地等着温斯顿说话。但他要说什么呢?即使是现在,你仍可以想象他很忙,他正生气地想他们为什么要打扰他。没有人说话,自从电屏被关掉,房间里就死一般的寂静。时间一点点地过去,气氛压抑。温斯顿艰难地凝视着奥布兰的眼睛。突然,奥布兰那严峻的面孔上浮现出一丝笑容。他扶了扶眼镜,这是他特有的习惯性动作。

\switchcolumn*

``Shall I say it, or will you?'' he said.

\switchcolumn

``我说,还是你说?''他问。

\switchcolumn*

``I will say it,'' said Winston promptly. ``That thing is really turned
off?''

\switchcolumn

``还是我说吧,''温斯顿迅速地接道,``那个东西真的关上了?''

\switchcolumn*

``Yes, everything is turned off. We are alone.''

\switchcolumn

``是的,都关上了。这儿只有我们自己。''

\switchcolumn*

``We have come here because---''

\switchcolumn

``我们来是因为——''

\switchcolumn*

He paused, realizing for the first time the vagueness of his own
motives. Since he did not in fact know what kind of help he expected
from O\textquotesingle Brien, it was not easy to say why he had come
here. He went on, conscious that what he was saying must sound both
feeble and pretentious:

\switchcolumn

他卡住了,他第一次意识到他的目的是如此晦暗不明。因为他没打算从奥布兰这里得到什么实际的帮助,所以要说清自己为什么来并不容易。他继续说下去,虽然他发现他的话苍白无力又空洞非常:

\switchcolumn*

``We believe that there is some kind of conspiracy, some kind of secret
organization working against the Party, and that you are involved in it.
We want to join it and work for it. We are enemies of the Party. We
disbelieve in the principles of Ingsoc. We are thought-criminals. We are
also adulterers. I tell you this because we want to put ourselves at
your mercy. If you want us to incriminate ourselves in any other way, we
are ready.''

\switchcolumn

``我相信一定存在某种密谋,一些秘密组织正在进行反党活动,而且你也参加了。我们也想参加,也想为它工作。我们是党的敌人,我们不相信英社原则。我们是思想犯、是通奸者。我之所以告诉你这些是因为我们想把自己完全交给你来安排,如果你希望我们用其他的方式证明自己,我们也心甘情愿。''

\switchcolumn*

He stopped and glanced over his shoulder, with the feeling that the door
had opened. Sure enough, the little yellow-faced servant had come in
without knocking. Winston saw that he was carrying a tray with a
decanter and glasses.

\switchcolumn

他觉得身后的门开了,便停下来回头张望。果然,那个脸色发黄的小个子服务人员没有敲门就进来了。温斯顿看到他端着盘子,盘子上有一个酒瓶和几个玻璃杯。

\switchcolumn*

``Martin is one of us,'' said O\textquotesingle Brien impassively. ``Bring
the drinks over here, Martin. Put them on the round table. Have we
enough chairs? Then we may as well sit down and talk in comfort. Bring a
chair for yourself, Martin. This is business. You can stop being a
servant for the next ten minutes.''

\switchcolumn

``马丁是自己人,''奥布兰说着,神情淡漠,``把酒拿到这边来,马丁。把它们放在圆桌上。我们的椅子够了吗?我们最好坐下来,那样谈话会舒服些。把椅子搬过来,马丁。这是公事,接下来的十分钟你不用做仆人了。''

\switchcolumn*

The little man sat down, quite at his ease, and yet still with a
servantlike air, the air of a valet enjoying a privilege. Winston
regarded him out of the corner of his eye. It struck him that the
man\textquotesingle s whole life was playing a part, and that he felt it
to be dangerous to drop his assumed personality even for a moment.
O\textquotesingle Brien took the decanter by the neck and filled up the
glasses with a dark-red liquid. It aroused in Winston dim memories of
something seen long ago on a wall or a hoarding---a vast bottle composed
of electric lights which seemed to move up and down and pour its
contents into a glass. Seen from the top the stuff looked almost black,
but in the decanter it gleamed like a ruby. It had a sour-sweet smell.
He saw Julia pick up her glass and sniff at it with frank curiosity.

\switchcolumn

小个子很自然地坐了下来,仍带着几分仆从的姿态,一个被特别对待的仆从的姿态。温斯顿用眼角余光打量着他,突然萌生一个想法,这个男人一生都在扮演着一个角色,即便只是暂时放下伪装,仍会感觉危险。奥布兰拿起酒瓶,往玻璃杯里倒入深红色的液体。让温斯顿模模糊糊地想起来,很久以前在墙上或在广告牌上看到的由电灯泡组成的大酒瓶上下移动,将瓶子里的东西倒进杯子里。从上看下去,那酒近乎黑色,但在酒瓶里,它却闪烁着如红宝石一般的光芒。它闻起来酸酸甜甜的。他看到朱莉亚拿起杯子,好奇地闻了闻。

\switchcolumn*

``It is called wine,'' said O\textquotesingle Brien with a faint smile.
``You will have read about it in books, no doubt. Not much of it gets to
the Outer Party, I am afraid.'' His face grew solemn again, and he raised
his glass: ``I think it is fitting that we should begin by drinking a
health. To our Leader: To Emmanuel Goldstein.''

\switchcolumn

``这是葡萄酒,''奥布兰微微一笑,``不用怀疑,你们在书上看到过的。不过,恐怕外党的人有的不多。''他又换上严肃的表情,拿起酒杯,``让我们为健康干杯,让我们为我们的领袖高德斯坦因干杯。''

\switchcolumn*

Winston took up his glass with a certain eagerness. Wine was a thing he
had read and dreamed about. Like the glass paperweight or Mr.
Charrington\textquotesingle s half-remembered rhymes, it belonged to the
vanished, romantic past, the olden time as he liked to call it in his
secret thoughts. For some reason he had always thought of wine as having
an intensely sweet taste, like that of blackberry jam, and an immediate
intoxicating effect. Actually, when he came to swallow it, the stuff was
distinctly disappointing. The truth was that after years of gin drinking
he could barely taste it. He set down the empty glass.

\switchcolumn

温斯顿热切地举起了他的那杯酒,他曾在书上读到过葡萄酒,做梦都想尝一尝。和那块玻璃镇纸以及被查林顿先生忘了大半的歌谣一样,它们都属于已经消失的、充满浪漫情怀的过去。他偷偷地将过去称为``老时光''。出于某种原因,他一直认为葡萄酒是甜的,就像黑莓酱,还能让人立即醉倒。事实上,他一饮而尽后却非常失望。原来这些年他喝的都是杜松子酒,已经品不出葡萄酒的美味了。他将空了的酒杯放了下来。

\switchcolumn*

``Then there is such a person as Goldstein?'' he said.

\switchcolumn

``当真有高德斯坦因这人?''他问。

\switchcolumn*

``Yes, there is such a person, and he is alive. Where, I do not know.''

\switchcolumn

``是的,真有这么一个人,而且他还活着。只是我们不知道他在哪儿。''

\switchcolumn*

``And the conspiracy---the organization? It is real? It is not simply an
invention of the Thought Police?''

\switchcolumn

``那么密谋——那个组织也是真的吗?不会是思想警察杜撰的吧。''

\switchcolumn*

``No, it is real. The Brotherhood, we call it. You will never learn much
more about the Brotherhood than that it exists and that you belong to
it. I will come back to that presently.'' He looked at his wristwatch.
``It is unwise even for members of the Inner Party to turn off the
telescreen for more than half an hour. You ought not to have come here
together, and you will have to leave separately. You, comrade---'' he
bowed his head to Julia---``will leave first. We have about twenty
minutes at our disposal. You will understand that I must start by asking
you certain questions. In general terms, what are you prepared to do?''

\switchcolumn

``是的,它们是真的。我们叫它兄弟会。你们至多知道它真实存在,你们是它的一员,其他的就别想知道了。我一会儿再回来。''他看了看手表,``即使对内党党员而言,关电屏的时间超过三十分钟也是不明智的。你们不能再一起过来,并且你们要分头离开。你,同志——''他对朱莉亚点了下头,``你先离开。我们还有大约二十分钟的时间。你们要理解,我必须先问你们一些问题。总的来说,你们打算做什么?''

\switchcolumn*

``Anything that we are capable of,'' said Winston.

\switchcolumn

``任何事,只要是我们能做的。''温斯顿说。

\switchcolumn*

O\textquotesingle Brien had turned himself a little in his chair so that
he was facing Winston. He almost ignored Julia, seeming to take it for
granted that Winston could speak for her. For a moment the lids flitted
down over his eyes. He began asking his questions in a low,
expressionless voice, as though this were a routine, a sort of
catechism, most of whose answers were known to him already.

\switchcolumn

奥布兰坐在椅子上,微微转了下身,以便让自己和温斯顿面对着面。他几乎将朱莉亚忽略了,也许他想当然地觉得温斯顿可以代表她说话。他闭了会儿眼睛,然后开始用低沉的、没有感情的语气询问,就像在例行公事,大部分问题的答案他都知道了。

\switchcolumn*

``You are prepared to give your lives?''

\switchcolumn

``你们准备好交出你们的生命了么?''

\switchcolumn*

``Yes.''

\switchcolumn

``是的。''

\switchcolumn*

``You are prepared to commit murder?''

\switchcolumn

``你们准备好去杀人了吗?''

\switchcolumn*

``Yes.''

\switchcolumn

``是的。''

\switchcolumn*

``To commit acts of sabotage which may cause the death of hundreds of
innocent people?''

\switchcolumn

``你们准备好进行破坏活动了吗?这可能会导致上百个无辜者送命。''

\switchcolumn*

``Yes.''

\switchcolumn

``是的。''

\switchcolumn*

``To betray your country to foreign powers?''

\switchcolumn

``你们准备好将祖国出卖给别的国家吗?''

\switchcolumn*

``Yes.''

\switchcolumn

``是的。''

\switchcolumn*

``You are prepared to cheat, to forge, to blackmail, to corrupt the minds
of children, to distribute habit-forming drugs, to encourage
prostitution, to disseminate venereal diseases---to do anything which is
likely to cause demoralization and weaken the power of the Party?''

\switchcolumn

``你们准备好欺骗、造假、恐吓了吗?准备好侵蚀孩子们的心灵、散发令人上瘾的药物、
鼓励卖淫、散播性病了吗?——所有可以破坏道德风气、削弱党的力量的事你们都愿意去做吗?''

\switchcolumn*

``Yes.''

\switchcolumn

``是的。''

\switchcolumn*

``If, for example, it would somehow serve our interests to throw
sulphuric acid in a child\textquotesingle s face---are you prepared to
do that?''

\switchcolumn

``那么,举个例子,假如朝一个孩子的脸上泼硫酸会对我们的利益有帮助——你们也愿意去做吗?''

\switchcolumn*

``Yes.''

\switchcolumn

``是的。''

\switchcolumn*

``You are prepared to lose your identity and live out the rest of your
life as a waiter or a dock worker?''

\switchcolumn

``你们准备好隐姓埋名,余生去做服务生或码头工人吗?''

\switchcolumn*

``Yes.''

\switchcolumn

``是的。''

\switchcolumn*

``You are prepared to commit suicide, if and when we order you to do so?''

\switchcolumn

``你们准备好自杀了吗?如果我要你们这样做。''

\switchcolumn*

``Yes.''

\switchcolumn

``是的。''

\switchcolumn*

``You are prepared, the two of you, to separate and never see one another
again?''

\switchcolumn

``你们准备好,你们两个,从此分手再不相见吗?''

\switchcolumn*

``No!'' broke in Julia.

\switchcolumn

``不!''朱莉亚突然插进来。

\switchcolumn*

It appeared to Winston that a long time passed before he answered. For a
moment he seemed even to have been deprived of the power of speech. His
tongue worked soundlessly, forming the opening syllables first of one
word, then of the other, over and over again. Until he had said it, he
did not know which word he was going to say. ``No,'' he said finally.

\switchcolumn

温斯顿觉得自己过了很久才回答这个问题,有那么一会儿,他好像被剥夺了说话的能力,
他的舌头在动,声音却发不出来,他摆出了发第一个单词第一个音的嘴型,又犹豫着摆
另一个单词另一个音的,这样反复数次,自己也不知道要说哪个单词,最终,他
说:``不。''

\switchcolumn*

``You did well to tell me,'' said O\textquotesingle Brien. ``It is
necessary for us to know everything.''

\switchcolumn

``你们能告诉我这些很好,''奥布兰说,``我们需要了解每件事。''

\switchcolumn*

He turned himself toward Julia and added in a voice with somewhat more
expression in it:

\switchcolumn

他转过身和朱莉亚说话,声音里多了几分感情。

\switchcolumn*

``Do you understand that even if he survives, it may be as a different
person? We may be obliged to give him a new identity. His face, his
movements, the shape of his hands, the color of his hair---even his
voice would be different. And you yourself might have become a different
person. Our surgeons can alter people beyond recognition. Sometimes it
is necessary. Sometimes we even amputate a limb.''

\switchcolumn

``你知道吗?就算他侥幸活了下来,他也可能是完全不同的人了。我们可能会给他一个全新的身份。他的脸,他的举止,他手部的形状,他头发的颜色——甚至是他的声音都会变得不同。而且,你也可能变成一个不同的人。我们的外科医生能够改变人的样子,让人认不出来。有些时候这是必须的,还有些时候我们甚至要切掉一个人的肢体。''

\switchcolumn*

Winston could not help snatching another sidelong glance at
Martin\textquotesingle s Mongolian face. There were no scars that he
could see. Julia had turned a shade paler, so that her freckles were
showing, but she faced O\textquotesingle Brien boldly. She murmured
something that seemed to be assent.

\switchcolumn

温斯顿不禁瞥了一眼马丁那张蒙古人的脸,看不出那上面有什么疤痕。朱莉亚的脸色变得苍白,雀斑显了出来,但她依然充满勇气地面对着奥布兰,她喃喃地说着什么,好像在表示同意。

\switchcolumn*

``Good. Then that is settled.''

\switchcolumn

``好。没问题了。''

\switchcolumn*

There was a silver box of cigarettes on the table. With a rather
absent-minded air O\textquotesingle Brien pushed them toward the others,
took one himself, then stood up and began to pace slowly to and fro, as
though he could think better standing. They were very good cigarettes,
very thick and well-packed, with an unfamiliar silkiness in the paper.
O\textquotesingle Brien looked at his wristwatch again.

\switchcolumn

桌子上有一盒银色包装的香烟,奥布兰心不在焉地将烟盒推给他们,他自己也拿了一根。之后,他站起身,踱来踱去,似乎站着思考更加方便。香烟很棒,很粗,卷得也很紧,还包着罕见的丝绸一般的纸。奥布兰又看了看手表。

\switchcolumn*

``You had better go back to your pantry, Martin,'' he said. ``I shall
switch on in a quarter of an hour. Take a good look at these
comrades\textquotesingle{} faces before you go. You will be seeing them
again. I may not.''

\switchcolumn

``最好回你的餐具室去,马丁,''他说,``我会在一刻钟内打开电屏。走之前,你要好好看看这些同志的脸。你还会再见到他们的,我可能就不会了。''

\switchcolumn*

Exactly as they had done at the front door, the little
man\textquotesingle s dark eyes flickered over their faces. There was
not a trace of friendliness in his manner. He was memorizing their
appearance, but he felt no interest in them, or appeared to feel none.
It occurred to Winston that a synthetic face was perhaps incapable of
changing its expression. Without speaking or giving any kind of
salutation, Martin went out, closing the door silently behind him.
O\textquotesingle Brien was strolling up and down, one hand in the
pocket of his black overalls, the other holding his cigarette.

\switchcolumn

和在门口时一样,这个黑眼睛的小个子打量着他们的脸,举止中没有一点友好的表现。他记住了他们的样子,对他们没有兴趣,至少表面上没有。温斯顿想,也许人工制造的脸无法改变的表情。马丁什么话都没说,什么问候都没留,就那么走了出去,出去之前,他轻轻地关上了门。奥布兰在屋子里走来走去,一只手插在黑色制服的口袋里,一只手夹着香烟。

\switchcolumn*

``You understand,'' he said, ``that you will be fighting in the dark. You
will always be in the dark. You will receive orders and you will obey
them, without knowing why. Later I shall send you a book from which you
will learn the true nature of the society we live in, and the strategy
by which we shall destroy it. When you have read the book, you will be
full members of the Brotherhood. But between the general aims that we
are fighting for, and the immediate tasks of the moment, you will never
know anything. I tell you that the Brotherhood exists, but I cannot tell
you whether it numbers a hundred members, or ten million. From your
personal knowledge you will never be able to say that it numbers even as
many as a dozen. You will have three or four contacts, who will be
renewed from time to time as they disappear. As this was your first
contact, it will be preserved. When you receive orders, they will come
from me. If we find it necessary to communicate with you, it will be
through Martin. When you are finally caught, you will confess. That is
unavoidable. But you will have very little to confess, other than your
own actions. You will not be able to betray more than a handful of
unimportant people. Probably you will not even betray me. By that time I
may be dead, or I shall have become a different person, with a different
face.''

\switchcolumn

``你们明白,''他说,``你们将在黑暗中战斗。你们将永远身处黑暗。你们会收到命令、会服从命令,但你们不知道为什么。过些时候,我会给你们一本书,你们可以从中得知我们所生活的社会的真相,你们还能学到将之摧毁的策略。读完这本书,你们就是兄弟会的正式成员,但除了我们所奋斗的总目标和眼下的具体任务,你们不会知道任何东西。我只能告诉你们兄弟会真的存在,至于它到底有多少成员,是一百个还是一千万个,我都不能对你们说。从你们个人的经验来看,你们永远不会认识十个以上的会员。你们会有三四个联系人,每隔一段时间更新一次,原有的人就消失不见了。而这是你们的首次联系,会保持下去。你们接到的命令都将由我发出。若有需要,我们会通过马丁来联络你们。最后被抓到,你们会招供,这无可避免。但你们没什么能招的,只有你们自己做的那些事是例外。你们不可能出卖重要人物,也许你们连我也出卖不了。那时我可能已经死了,或者换上了不同的面孔。''

\switchcolumn*

He continued to move to and fro over the soft carpet. In spite of the
bulkiness of his body there was a remarkable grace in his movements. It
came out even in the gesture with which he thrust a hand into his
pocket, or manipulated a cigarette. More even than of strength, he gave
an impression of confidence and of an understanding tinged by irony.
However much in earnest he might be, he had nothing of the
single-mindedness that belongs to a fanatic. When he spoke of murder,
suicide, venereal disease, amputated limbs, and altered faces, it was
with a faint air of persiflage. ``This is unavoidable,'' his voice seemed
to say; ``this is what we have got to do, unflinchingly. But this is not
what we shall be doing when life is worth living again.'' A wave of
admiration, almost of worship, flowed out from Winston toward
O\textquotesingle Brien. For the moment he had forgotten the shadowy
figure of Goldstein. When you looked at
O\textquotesingle Brien\textquotesingle s powerful shoulders and his
blunt-featured face, so ugly and yet so civilized, it was impossible to
believe that he could be defeated. There was no stratagem that he was
not equal to, no danger that he could not foresee. Even Julia seemed to
be impressed. She had let her cigarette go out and was listening
intently. O\textquotesingle Brien went on:

\switchcolumn

他继续在柔软的地毯上走来走去,虽然他身材魁梧,可他的举止却非常优雅。即使是将
手放进口袋或拨弄香烟这样的小动作,也能反映出来。相比强硬有力,他给人留下的更
深的印象是自信、体贴,且这体贴中有那么一丝嘲讽的意蕴。尽管他可能是认真的,但
他身上并没有狂热分子专有的执拗。当他说谋杀、自杀、性病、断肢、换脸时,隐约带
着嘲弄的神情。``这无可避免,''听他的语气,好像在说,``这正是我们要做的,没有
妥协的余地。但如果生命值得再来一次,我们就不会做它了。''对奥布兰,温斯顿有一
种崇敬,甚至是崇拜的感情。一时间,他忘记了高德斯坦因那蒙眬的形象。看看奥布兰
那强壮有力的肩膀,那坚毅的面孔,如此丑陋又如此文雅,你不可能觉得他可以被击败
的。没有什么战术他不能胜任,没有什么危险他不能预见。就连朱莉亚也像感动了,她
专注地听着,香烟熄了都不知道。奥布兰继续说:

\switchcolumn*

``You will have heard rumors of the existence of the Brotherhood. No
doubt you have formed your own picture of it. You have imagined,
probably, a huge underworld of conspirators, meeting secretly in
cellars, scribbling messages on walls, recognizing one another by code
words or by special movements of the hand. Nothing of the kind exists.
The members of the Brotherhood have no way of recognizing one another,
and it is impossible for any one member to be aware of the identity of
more than a very few others. Goldstein himself, if he fell into the
hands of the Thought Police, could not give them a complete list of
members, or any information that would lead them to a complete list. No
such list exists. The Brotherhood cannot be wiped out because it is not
an organization in the ordinary sense. Nothing holds it together except
an idea which is indestructible. You will never have anything to sustain
you except the idea. You will get no comradeship and no encouragement.
When finally you are caught, you will get no help. We never help our
members. At most, when it is absolutely necessary that someone should be
silenced, we are occasionally able to smuggle a razor blade into a
prisoner\textquotesingle s cell. You will have to get used to living
without results and without hope. You will work for a while, you will be
caught, you will confess, and then you will die. Those are the only
results that you will ever see. There is no possibility that any
perceptible change will happen within our own lifetime. We are the dead.
Our only true life is in the future. We shall take part in it as
handfuls of dust and splinters of bone. But how far away that future may
be, there is no knowing. It might be a thousand years. At present
nothing is possible except to extend the area of sanity little by
little. We cannot act collectively. We can only spread our knowledge
outwards from individual to individual, generation after generation. In
the face of the Thought Police, there is no other way.''

\switchcolumn

``你们会听到关于兄弟会的传闻,不要怀疑,对它,你们已经形成了自己的想法。你们可能把它想成一个巨大的密谋者的地下组织,在地下室召开秘密会议,在墙上张贴信息,用暗号或特殊的手势确认彼此的身份。这样的事情是不存在的。兄弟会的成员不能相互辨认,任何一个成员都只能接触到极少数的其他成员。就连高德斯坦因自己,若是落到了思想警察的手上,也不能向他们提供所有成员的名单或任何和这名单有关的信息。没有这种名单,兄弟会之所以不会被除掉就是因为它不是一般意义上的组织,是坚不可摧的信念将它凝聚起来,除此之外别无其他。同样的,你们只能仰仗信念,别的什么都依靠不了。没有同志间的友情作支撑,也得不到鼓舞激励。最后,你们被抓住,也不会有人来救你们。我们从来不会救助我们的成员。最多,在必须将某人灭口的情况下,我们有时会将刀片偷运进牢房。你们不得不适应这种没有结果,没有希望的境遇。你们会工作一段时间,会被逮捕,会招供,会死去。这是你们能看到的仅有的结果。我们活着的时候不会遇见任何明显的变化。我们是死人。我们唯一真正的生活在未来。我们将以几捧尘土,几副枯骨的姿态进入未来。没有哪个人知道未来还有多远,它也许是一千年。目前,我们能做的只能是一点点地扩大头脑清醒的人的范围。由于无法进行集体行动,我们能做的只能是传播我们的想法,从一个人到另一个人,从这一代到下一代。面对思想警察,你别无他法。''

\switchcolumn*

He halted and looked for the third time at his wristwatch.

\switchcolumn

他停下来,第三次看他的表。

\switchcolumn*

``It is almost time for you to leave, comrade,'' he said to Julia. ``Wait.
The decanter is still half full.''

\switchcolumn

``已经到了你回去的时间了。''他对朱莉亚说,``等等,还有半瓶酒。''

\switchcolumn*

He filled the glasses and raised his own glass by the stem.

\switchcolumn

他将大家的杯子倒满并举起自己的那杯。

\switchcolumn*

``What shall it be this time?'' he said, still with the same faint
suggestion of irony. ``To the confusion of the Thought Police? To the
death of Big Brother? To humanity? To the future?''

\switchcolumn

``这次是为什么而干杯呢?''他说,语调中仍有一丝嘲讽,``为了让思想警察慌乱?为了让老大哥死?为了人性?为了未来?''

\switchcolumn*

``To the past,'' said Winston.

\switchcolumn

``为过去。''温斯顿说。

\switchcolumn*

``The past is more important,'' agreed O\textquotesingle Brien gravely.
They emptied their glasses, and a moment later Julia stood up to go.
O\textquotesingle Brien took a small box from the top of a cabinet and
handed her a flat white tablet which he told her to place on her tongue.
It was important, he said, not to go out smelling of wine: the lift
attendants were very observant. As soon as the door had shut behind her
he appeared to forget her existence. He took another pace or two up and
down, then stopped.

\switchcolumn

``过去更加重要。''奥布兰庄重地表示同意。他们喝光了杯子中的酒,又待了一会儿,朱莉亚起身要走。奥布兰从柜子顶上取下一个小盒子,并从盒子里拿出一个白色的药片让她放在舌头上。这很重要,他说,出去后不要让别人闻到酒味,电梯员非常敏感。而她一关上门,他就像忘记她的存在一样,又在屋子里走了两步,然后停下来。

\switchcolumn*

``There are details to be settled,'' he said. ``I assume that you have a
hiding place of some kind?''

\switchcolumn

``还有些细节要解决,''他说,``我猜你应该有什么藏身的地方。''

\switchcolumn*

Winston explained about the room over Mr. Charrington\textquotesingle s
shop.

\switchcolumn

温斯顿向他描述了查林顿先生商店上的那间屋子。

\switchcolumn*

``That will do for the moment. Later we will arrange something else for
you. It is important to change one\textquotesingle s hiding place
frequently. Meanwhile I shall send you a copy of \emph{the book}---''
even O\textquotesingle Brien, Winston noticed, seemed to pronounce the
words as though they were in italics---``Goldstein\textquotesingle s
book, you understand, as soon as possible. It may be some days before I
can get hold of one. There are not many in existence, as you can
imagine. The Thought Police hunts them down and destroys them almost as
fast as we can produce them. It makes very little difference. The book
is indestructible. If the last copy were gone, we could reproduce it
almost word for word. Do you carry a brief case to work with you?'' he
added.

\switchcolumn

``现在还能用。过后我们会另外给你安排个地方。经常更换藏身之所是重要的。同时,我还要将那本书带给你——''温斯顿注意到奥布兰在提到``高德斯坦因的书''时加重了语气。``你清楚的,我会尽快给你,可能过几天才能拿到。书的数量很少,你能想象。思想警察发现它们、销毁它们的速度就像它们出版的速度一样快。但这不要紧。它是不会被摧毁的。哪怕最后一本也被带走,我们依然能一个字一个字地再印出来。你上班时带公文包吗?''他补充道。

\switchcolumn*

``As a rule, yes.''

\switchcolumn

``会带的。''

\switchcolumn*

``What is it like?''

\switchcolumn

``什么样子的?''

\switchcolumn*

``Black, very shabby. With two straps.''

\switchcolumn

``黑色的,非常旧,有两条带子。''

\switchcolumn*

``Black, two straps, very shabby---good. One day in the fairly near
future---I cannot give a date---one of the messages among your
morning\textquotesingle s work will contain a misprinted word, and you
will have to ask for a repeat. On the following day you will go to work
without your brief case. At some time during the day, in the street, a
man will touch you on the arm and say, `I think you have
dropped your brief case.' The one he gives you will
contain a copy of Goldstein\textquotesingle s book. You will return it
within fourteen days.''

\switchcolumn

``黑色的,两条带子,非常旧——好的。最近几天——我给不了具体的日期——早上,你工作的时候会收到一个通知,里面有个字印错了,你务必要求重发。第二天上班时就不要带公文包了。路上会有人拍你的手臂,对你说`我想你把公文包弄丢了'。他给你的公文包里就有高德斯坦因的书。你要在十四天内还回来。''

\switchcolumn*

They were silent for a moment.

\switchcolumn

他们沉默了一会。

\switchcolumn*

``There are a couple of minutes before you need go,'' said
O\textquotesingle Brien. ``We shall meet again---if we do meet again---''

\switchcolumn

``再过几分钟你就得走了,''奥布兰说,``我们会再见面的——如果还有见面的机会——''

\switchcolumn*

Winston looked up at him. ``In the place where there is no darkness?'' he
said hesitantly.

\switchcolumn

温斯顿看着他,有些犹豫地说:``在没有黑暗的地方?''

\switchcolumn*

O\textquotesingle Brien nodded without appearance of surprise. ``In the
place where there is no darkness,'' he said, as though he had recognized
the allusion. ``And in the meantime, is there anything that you wish to
say before you leave? Any message? Any question?''

\switchcolumn

奥布兰点了点头,一点都不吃惊。``在没有黑暗的地方。''他说,就好像他明白它暗示的是什么。``在离开前,还有什么想说的吗?有口信吗?有问题吗?''

\switchcolumn*

Winston thought. There did not seem to be any further question that he
wanted to ask; still less did he feel any impulse to utter high-sounding
generalities. Instead of anything directly connected with
O\textquotesingle Brien or the Brotherhood, there came into his mind a
sort of composite picture of the dark bedroom where his mother had spent
her last days, and the little room over Mr.
Charrington\textquotesingle s shop, and the glass paperweight, and the
steel engraving in its rosewood frame. Almost at random he said:

\switchcolumn

温斯顿想了想,似乎没有想问的问题,也不觉得有必要讲一些高调的话。出现在他脑海中的并不是奥布兰或兄弟会,而是一幅意象复杂的画面,他母亲最后呆的那间黑暗的屋子,查林顿先生商店上的小房间、玻璃镇纸、带着玫瑰木框架的钢板画\ldots\ldots 他差不多是脱口而出:

\switchcolumn*

``Did you ever happen to hear an old rhyme that begins \emph{Oranges and
lemons, say the bells of St. Clement\textquotesingle s}?''

\switchcolumn

``你听没听到过这样一首老歌,开头是橘子和柠檬。圣克莱门特教堂的大钟说。''

\switchcolumn*

Again O\textquotesingle Brien nodded. With a sort of grave courtesy he
completed the stanza:

\switchcolumn

奥布兰点了点头,优雅而郑重地唱完了这段:

\switchcolumn*

\begin{quotation}
  \noindent ``Oranges and lemons, say the bells of St. Clement\textquotesingle s,\\
  You owe me three farthings, say the bells of St. Martin\textquotesingle s,\\
  When will you pay me? say the bells of Old Bailey,\\
  When I grow rich, say the bells of Shoreditch.''
\end{quotation}

\switchcolumn

\begin{quotation}
\noindent 橘子和柠檬。圣克莱门特教堂的大钟说。\\
你欠我3个法寻。圣马丁教堂的大钟说。\\
你什么时候还?老百利教堂的大钟说。\\
等我发了财,肖尔迪奇的大钟说。
\end{quotation}

\switchcolumn*

``You knew the last line!'' said Winston.

\switchcolumn

``你知道最后一句!''温斯顿说。

\switchcolumn*

``Yes, I knew the last line. And now, I am afraid, it is time for you to
go. But wait. You had better let me give you one of these tablets.''

\switchcolumn

``没错,我知道最后一句。我恐怕你现在必须回去了。但等等,最好也给你一片药。''

\switchcolumn*

As Winston stood up O\textquotesingle Brien held out a hand. His
powerful grip crushed the bones of Winston\textquotesingle s palm. At
the door Winston looked back, but O\textquotesingle Brien seemed already
to be in process of putting him out of mind. He was waiting with his
hand on the switch that controlled the telescreen. Beyond him Winston
could see the writing table with its green-shaded lamp and the
speakwrite and the wire baskets deep-laden with papers. The incident was
closed. Within thirty seconds, it occurred to him,
O\textquotesingle Brien would be back at his interrupted and important
work on behalf of the Party.

\switchcolumn

当温斯顿起身的时候,奥布兰伸出了手。他用力握了握他的手,几乎将温斯顿的手掌都给捏碎了。走到门口时,温斯顿回头看了一眼,奥布兰似乎正要将他忘掉。他把手放在电屏的开关上,正等他离开。而在他的身后,温斯顿看到桌子上那罩着绿灯罩的台灯、语音记录器以及装满文件的铁篮子。他想,三十秒钟内,奥布兰就会重新开始刚刚中断的党的重要工作。

\switchcolumn*


\section{IX}\label{ix}

\switchcolumn

\section*{九}\label{ux5341ux4e03}

\switchcolumn*

Winston was gelatinous with fatigue. Gelatinous was the right word. It
had come into his head spontaneously. His body seemed to have not only
the weakness of a jelly, but its translucency. He felt that if he held
up his hand he would be able to see the light through it. All the blood
and lymph had been drained out of him by an enormous debauch of work,
leaving only a frail structure of nerves, bones, and skin. All
sensations seemed to be magnified. His overalls fretted his shoulders,
the pavement tickled his feet, even the opening and closing of a hand
was an effort that made his joints creak.

\switchcolumn

温斯顿累得像一摊糨糊。糨糊,是个非常贴切的词,它自然而然地出现在他的脑海中。他的身体不仅像糨糊般瘫软,还像糨糊般透明,他觉得若他将手举起,甚至能看到光从手中透出来。高强度的工作几乎将他的血液和淋巴液都榨干了,只剩下由神经、骨骼、皮肤构成的脆弱的架子。所有的感觉都好像变得敏感起来,制服摩擦着肩膀,人行道让脚底发痒,就连手掌的张合都额外费力,关节也发出咔嚓的声响。

\switchcolumn*

He had worked more than ninety hours in five days. So had everyone else
in the Ministry. Now it was all over, and he had literally nothing to
do, no Party work of any description, until tomorrow morning. He could
spend six hours in the hiding place and another nine in his own bed.
Slowly, in mild afternoon sunshine, he walked up a dingy street in the
direction of Mr. Charrington\textquotesingle s shop, keeping one eye
open for the patrols, but irrationally convinced that this afternoon
there was no danger of anyone interfering with him. The heavy brief case
that he was carrying bumped against his knees at each step, sending a
tingling sensation up and down the skin of his leg. Inside it was
\emph{the book}, which he had now had in his possession for six days and
had not yet opened, nor even looked at.

\switchcolumn

五天里,他已经工作了九十多个小时,部里的其他人也是如此。现在,工作都结束了,到明天上午之前,他无事可做,党什么工作都没给他安排。他可以在那秘密的藏身之处待上六个小时,再回自家的床上躺上9个小时。他漫步在午后的阳光下,沿着一条脏兮兮的巷子前往查林顿先生的商店,一路上他一直提防巡逻队。但同时,他又很不现实地认为在这样一个下午不会有被人打扰的危险。他的公文包很重,每走一步都会撞到他的膝盖,让他整条腿都觉得麻麻的。那本书就放在公文包里,已经6天了,他还没有将它打开,甚至没有看它一眼。

\switchcolumn*

On the sixth day of Hate Week, after the processions, the speeches, the
shouting, the singing, the banners, the posters, the films, the
waxworks, the rolling of drums and squealing of trumpets, the tramp of
marching feet, the grinding of the caterpillars of tanks, the roar of
massed planes, the booming of guns---after six days of this, when the
great orgasm was quivering to its climax and the general hatred of
Eurasia had boiled up into such delirium that if the crowd could have
got their hands on the two thousand Eurasian war criminals who were to
be publicly hanged on the last day of the proceedings, they would
unquestionably have torn them to pieces---at just this moment it had
been announced that Oceania was not after all at war with Eurasia.
Oceania was at war with Eastasia. Eurasia was an ally.

\switchcolumn

仇恨周已进行了六天,这六天里每天都充斥着游行、演讲、呐喊、歌唱、旗帜、宣传画、电影、蜡像,每天都有军鼓的轰响、小号的尖啸、正步前进的隆隆声,以及坦克的碾磨声、飞机的轰鸣声、枪炮的鸣响声。人们极度兴奋,颤抖着达到了高潮,对欧亚国恨得发狂,如果仇恨周最后一天公开绞死的两千名欧亚国战俘落到他们手上,毫无疑问会被撕成碎片。但就在这个时候,大洋国突然宣布,大洋国并没有和欧亚国打仗,大洋国在和东亚国交火,欧亚国是大洋国的盟友。

\switchcolumn*

There was, of course, no admission that any change had taken place.
Merely it became known, with extreme suddenness and everywhere at once,
that Eastasia and not Eurasia was the enemy. Winston was taking part in
a demonstration in one of the central London squares at the moment when
it happened. It was night, and the white faces and the scarlet banners
were luridly floodlit. The square was packed with several thousand
people, including a block of about a thousand schoolchildren in the
uniform of the Spies. On a scarlet-draped platform an orator of the
Inner Party, a small lean man with disproportionately long arms and a
large, bald skull over which a few lank locks straggled, was haranguing
the crowd. A little Rumpelstiltskin figure, contorted with hatred, he
gripped the neck of the microphone with one hand while the other,
enormous at the end of a bony arm, clawed the air menacingly above his
head. His voice, made metallic by the amplifiers, boomed forth an
endless catalogue of atrocities, massacres, deportations, lootings,
rapings, torture of prisoners, bombing of civilians, lying propaganda,
unjust aggressions, broken treaties. It was almost impossible to listen
to him without being first convinced and then maddened. At every few
moments the fury of the crowd boiled over and the voice of the speaker
was drowned by a wild beastlike roaring that rose uncontrollably from
thousands of throats. The most savage yells of all came from the
schoolchildren. The speech had been proceeding for perhaps twenty
minutes when a messenger hurried onto the platform and a scrap of paper
was slipped into the speaker\textquotesingle s hand. He unrolled and
read it without pausing in his speech. Nothing altered in his voice or
manner, or in the content of what he was saying, but suddenly the names
were different. Without words said, a wave of understanding rippled
through the crowd. Oceania was at war with Eastasia! The next moment
there was a tremendous commotion. The banners and posters with which the
square was decorated were all wrong! Quite half of them had the wrong
faces on them. It was sabotage! The agents of Goldstein had been at
work! There was a riotous interlude while posters were ripped from the
walls, banners torn to shreds and trampled underfoot. The Spies
performed prodigies of activity in clambering over the rooftops and
cutting the streamers that fluttered from the chimneys. But within two
or three minutes it was all over. The orator, still gripping the neck of
the microphone, his shoulders hunched forward, his free hand clawing at
the air, had gone straight on with his speech. One minute more, and the
feral roars of rage were again bursting from the crowd. The Hate
continued exactly as before, except that the target had been changed.

\switchcolumn

当然,没有哪个人承认有变化发生。突然之间,无论在什么地方,所有人都知道敌人是东亚国不是欧亚国。这一切发生时温斯顿正在伦敦的中心广场参加示威游行。当时正值夜晚,白生生的人脸和绯红的旗帜都映着斑斓的灯。广场上挤了好几千人,包括一千多名身着少年侦察队制服的学生。在用红布装饰的舞台上一名内党党员正对着群众高谈阔论,那是一个身材瘦小的男人,胳膊长得不合比例,硕大的脑袋上只有几缕头发,活似神话中的侏儒怪。愤怒让他身体扭曲,他的一只手抓着话筒,另一只手——他的手臂很细,手掌却十分宽大——疯狂地在头顶上挥舞。他滔滔不绝地控诉着敌人的暴行,比如屠杀、流放、抢劫、强奸、虐待俘虏、轰炸平民,还有充斥着谎言的宣传、非正义的进攻和对条约的背叛。他的声音经过扩音器沾染上金属的味道,几乎没有人不相信他的话,也没有人不为他的话感到愤怒。每隔几分钟就众怒腾腾,他的声音即被淹没在数千人如野兽般不受控制的咆哮里,而最为粗野的咆哮来自学童。讲话大约进行到二十分钟,一名通讯员匆匆走上讲台,将一张纸条塞进他手里。他打开纸条,没有停止讲话,无论是声音还是讲话的样子都没有发生改变。但,突然名称变了。无须解释,就像一波海浪扫过人群,人们心领神会。大洋国是在和东亚国打仗!接着便是一阵可怕的混乱。广场上,那些旗帜、宣传画统统搞错了!它们中至少有一半画错了人物的脸。这是破坏!是高德斯坦因的人干的!人群中出现骚乱,人们撕下墙上的宣传画,将旗帜扯得粉碎,踩在脚下。少年侦察队表现非凡,他们爬上屋顶,剪断了挂在烟囱上的横幅。不过,在两三分钟内,这些都结束了。演讲者肩膀前耸,一手抓着话筒,一手在头上挥舞,演讲仍在继续。一分钟后,人群中又会爆发愤怒的咆哮。除了仇恨对象的改变,仇恨周将一如既往地进行。

\switchcolumn*

The thing that impressed Winston in looking back was that the speaker
had switched from one line to the other actually in mid-sentence, not
only without a pause, but without even breaking the syntax. But at the
moment he had other things to preoccupy him. It was during the moment of
disorder while the posters were being torn down that a man whose face he
did not see had tapped him on the shoulder and said, ``Excuse me, I think
you\textquotesingle ve dropped your brief case.'' He took the brief case
abstractedly, without speaking. He knew that it would be days before he
had an opportunity to look inside it. The instant that the demonstration
was over he went straight to the Ministry of Truth, though the time was
now nearly twenty-three hours. The entire staff of the Ministry had done
likewise. The orders already issuing from the telescreens, recalling
them to their posts, were hardly necessary.

\switchcolumn

回想起来,最让温斯顿印象深刻的是演讲者竟然在讲到一半时变换了演讲的对象,可他不仅没有片刻停顿,还没有打乱句子的结构。但在当时,他正全神贯注地想着其他的事。就在人们撕毁宣传画的时候,一个男人拍了拍他的肩膀,说:``对不起,我想你弄丢了你的公文包。''他没看清那人的长相,什么都没说,心不在焉地接过了公文包。他知道要过上几天他才有机会看看里面的东西。示威一结束他就返回了真理部,尽管时间已接近23点。部里的人都一样,电屏里传出命令,要大家回岗,不过,这根本就没有必要。

\switchcolumn*

Oceania was at war with Eastasia: Oceania had always been at war with
Eastasia. A large part of the political literature of five years was now
completely obsolete. Reports and records of all kinds, newspapers,
books, pamphlets, films, sound tracks, photographs---all had to be
rectified at lightning speed. Although no directive was ever issued, it
was known that the chiefs of the Department intended that within one
week no reference to the war with Eurasia, or the alliance with
Eastasia, should remain in existence anywhere. The work was
overwhelming, all the more so because the processes that it involved
could not be called by their true names. Everyone in the Records
Department worked eighteen hours in the twenty-four, with two three-hour
snatches of sleep. Mattresses were brought up from the cellars and
pitched all over the corridors; meals consisted of sandwiches and
Victory Coffee wheeled round on trolleys by attendants from the canteen.
Each time that Winston broke off for one of his spells of sleep he tried
to leave his desk clear of work, and each time that he crawled back,
sticky-eyed and aching, it was to find that another shower of paper
cylinders had covered the desk like a snowdrift, half burying the
speakwrite and overflowing onto the floor, so that the first job was
always to stack them into a neat-enough pile to give him room to work.
What was worst of all was that the work was by no means purely
mechanical. Often it was enough merely to substitute one name for
another, but any detailed report of events demanded care and
imagination. Even the geographical knowledge that one needed in
transferring the war from one part of the world to another was
considerable.

\switchcolumn

大洋国正在和东亚国打仗:大洋国一直都在和东亚国打仗。过去五年的大部分政治文件都作废了。各种报告、档案、报纸、书籍、小册子、电影、录音、照片——所有的一切都要闪电般地改好。虽然没有明确的命令,但大家都心知肚明。记录司的领导计划在一个星期内消除掉所有提到和欧亚国打仗、同东亚国结盟的东西。工作多得要将人淹没,再加上此事不能明说,工作就愈发艰巨。记录司的人每天都要工作十八个小时,睡眠被分成两次,每次三小时。从地下室搬出的床垫铺满了整个走廊。食物被放在手推车上由食堂的工作人员推过来,包括夹肉面包和胜利牌咖啡。每次睡觉前,温斯顿都尽可能将桌子上的工作做完,但当他睡眼惺忪、腰酸背痛地回来时,桌上的纸卷就又堆得像雪山一般了,不仅将大半个语音记录器埋了起来,还多得掉到了地上。因此,首先要做的就是将它们整理好,以腾出工作的空间。最糟糕的是,这些工作都并非全是机械性的,尽管大多时候只需要更换下名字,但一些详细叙述事件的文件就需要人特别仔细并发挥想象力。即使是将一场战争从世界的一个地区挪到另一个地区,你就需要相当多的地理知识。

\switchcolumn*

By the third day his eyes ached unbearably and his spectacles needed
wiping every few minutes. It was like struggling with some crushing
physical task, something which one had the right to refuse and which one
was nevertheless neurotically anxious to accomplish. In so far as he had
time to remember it, he was not troubled by the fact that every word he
murmured into the speakwrite, every stroke of his ink pencil, was a
deliberate lie. He was as anxious as anyone else in the Department that
the forgery should be perfect. On the morning of the sixth day the
dribble of cylinders slowed down. For as much as half an hour nothing
came out of the tube; then one more cylinder, then nothing. Everywhere
at about the same time the work was easing off. A deep and as it were
secret sigh went through the Department. A mighty deed, which could
never be mentioned, had been achieved. It was now impossible for any
human being to prove by documentary evidence that the war with Eurasia
had ever happened. At twelve hundred it was unexpectedly announced that
all workers in the Ministry were free till tomorrow morning. Winston,
still carrying the brief case containing \emph{the book}, which had
remained between his feet while he worked and under his body while he
slept, went home, shaved himself, and almost fell asleep in his bath,
although the water was barely more than tepid.

\switchcolumn

到第三天,他的眼睛已疼得难以忍受,每隔几分钟就要擦擦眼镜。这就像在努力完成一件折磨人身体的力气活,你有权拒绝它,却又神经质地想尽快做完。他对语音记录器说下的每句话,他用墨水笔写下的每个字,都是深思熟虑的谎言,但据他回忆,他并没有为此感到不安。他像司里的其他人一样,都希望将谎言说得完美。到第六天早上,纸卷的数量少了,有半个小时管道没有送出任何东西。之后,送来一个纸卷,再之后就没有了。几乎在同一时间,那里的工作都完成了。司里的人深深地、悄悄地叹了口气。这件不能提起的伟大的工作终于搞定了。现在,任何人都拿不出能够证明和欧亚国交过战的文件。12点,所有工作人员都出乎意料地收到了休息到明天早上的通知。温斯顿一直将装着那本书的公文包带在身边。工作时,他将它夹在两脚之间,睡觉时又将它压在身子底下。回家了,他刮了胡子,洗了澡,虽然水不暖,他还是差点在浴缸里睡着。

\switchcolumn*

With a sort of voluptuous creaking in his joints he climbed the stair
above Mr. Charrington\textquotesingle s shop. He was tired, but not
sleepy any longer. He opened the window, lit the dirty little oilstove,
and put on a pan of water for coffee. Julia would arrive presently;
meanwhile there was \emph{the book}. He sat down in the sluttish
armchair and undid the straps of the brief case.

\switchcolumn

在查林顿先生的店铺里上楼梯时,他很享受关节吱吱作响的感觉。他很疲倦,但不想睡觉,他将窗户打开,点着了那脏兮兮的煤油炉,在上面放了一壶水,准备煮咖啡。朱莉亚一会儿就到,那本书就放在这里。他在那把肮脏的扶手椅上坐下来,松开了公文包的带子。

\switchcolumn*

A heavy black volume, amateurishly bound, with no name or title on the
cover. The print also looked slightly irregular. The pages were worn at
the edges, and fell apart easily, as though the book had passed through
many hands. The inscription on the title page ran:

\switchcolumn

这是一本黑色的、厚厚的书,装订很差,封面上没有作者的名字,也没写书名。印刷的字体微微有些不同,书的页边磨损得厉害,一不小心就会散开,看起来这本书已经被很多人转手。书的扉页上印着:

\switchcolumn*

\begin{center}
THE THEORY AND PRACTICE\\
OF OLIGARCHICAL COLLECTIVISM\\
by\\
\textsc{Emmanuel Goldstein}
\end{center}

\switchcolumn

\begin{center}
  《寡头集体主义的理论与实践》\\
  埃曼纽尔•高德斯坦因 著
\end{center}

\switchcolumn*

[Winston began reading.]

\switchcolumn

温斯顿读了起来:

\switchcolumn*

\begin{center}
\emph{Chapter 1.}\\
IGNORANCE IS STRENGTH.
\end{center}

\switchcolumn

\begin{center}
第一章 无知即力量
\end{center}

\switchcolumn*

\begin{quotation}
Throughout recorded time, and probably since the end of the Neolithic
Age, there have been three kinds of people in the world, the High, the
Middle, and the Low. They have been subdivided in many ways, they have
borne countless different names, and their relative numbers, as well as
their attitude toward one another, have varied from age to age; but the
essential structure of society has never altered. Even after enormous
upheavals and seemingly irrevocable changes, the same pattern has always
reasserted itself, just as a gyroscope will always return to
equilibrium, however far it is pushed one way or the other.\par
The aims of these three groups are entirely irreconcilable....
\end{quotation}

\switchcolumn

\begin{quotation}
有史以来,大约从新石器时代结束开始,世界上就有三种人:上等人、中等人、下等人。按照不同的方式继续划分,他们有过很多名字。他们的相对人数以及对彼此的态度都因时而异,然而社会的基本结构却没有发生改变。即使在经历了重大剧变和看起来不可挽回的变化后,依然能恢复其原有的格局,就好像无论向哪个方向推进,陀螺仪都能恢复平衡。\par
这三个阶层的目标完全无法协调\ldots\ldots{}
\end{quotation}

\switchcolumn*

Winston stopped reading, chiefly in order to appreciate the fact that he
\emph{was} reading, in comfort and safety. He was alone: no telescreen,
no ear at the keyhole, no nervous impulse to glance over his shoulder or
cover the page with his hand. The sweet summer air played against his
cheek. From somewhere far away there floated the faint shouts of
children; in the room itself there was no sound except the insect voice
of the clock. He settled deeper into the armchair and put his feet up on
the fender. It was bliss, it was eternity. Suddenly, as one sometimes
does with a book of which one knows that one will ultimately read and
reread every word, he opened it at a different place and found himself
at the third chapter. He went on reading:

\switchcolumn

温斯顿停了下来,仔细回味这件事,他正在读书,舒适而安全。他独自一人,没有电屏,也没有隔墙之耳,他无需紧张兮兮地向后张望,也无需慌张地将书合上,夏季的清风扑面而来,远处隐约传来孩子们的叫喊,屋子里只有座钟滴答作响。他深深地坐在扶手椅中,把脚放在壁炉的挡板上。这真是一种享受,永远不变。突然,就好比有时会知道最终要将书里的每个字都一读再读,他随便翻到某一页,刚好是第三章。他接着读了下去。

\switchcolumn*

\begin{center}
Chapter 3.\par
WAR IS PEACE.
\end{center}

\switchcolumn

\begin{center}
第三章 战争即和平
\end{center}

\switchcolumn*

\begin{quotation}
The splitting-up of the world into three great superstates was an event
which could be and indeed was foreseen before the middle of the
twentieth century. With the absorption of Europe by Russia and of the
British Empire by the United States, two of the three existing powers,
Eurasia and Oceania, were already effectively in being. The third,
Eastasia, only emerged as a distinct unit after another decade of
confused fighting. The frontiers between the three superstates are in
some places arbitrary, and in others they fluctuate according to the
fortunes of war, but in general they follow geographical lines. Eurasia
comprises the whole of the northern part of the European and Asiatic
land-mass, from Portugal to the Bering Strait. Oceania comprises the
Americas, the Atlantic islands including the British Isles, Australasia,
and the southern portion of Africa. Eastasia, smaller than the others
and with a less definite western frontier, comprises China and the
countries to the south of it, the Japanese islands and a large but
fluctuating portion of Manchuria, Mongolia, and Tibet.
\end{quotation}

\switchcolumn

\begin{quotation}
20世纪中期之前,便可以预见世界会分成三个超级大国。由于俄国吞并了欧洲,美国吞
并了大英帝国,现有的三个大国中,有两个在当时就已经是切实的存在,即欧亚国和大
洋国。第三个是东亚国,其在经过十年的混战后出现。三国的边界有些乃任意划定,有
些则随战争的胜负情况变化,但总体而言,它们按照地理的界线进行划分。欧亚国包括
整个欧亚大陆的北部,从葡萄牙到白令海峡。大洋国包括美洲、大西洋诸岛、不列颠各
岛、澳大利亚和非洲南部。东亚国相对较小,西部边界未明,包括中国和中国以南各个
国家、日本各岛及满洲、蒙古、西藏等广阔却边界不明的地区。
\end{quotation}

\switchcolumn*

\begin{quotation}
In one combination or another, these three superstates are permanently
at war, and have been so for the past twenty-five years. War, however,
is no longer the desperate, annihilating struggle that it was in the
early decades of the twentieth century. It is a warfare of limited aims
between combatants who are unable to destroy one another, have no
material cause for fighting, and are not divided by any genuine
ideological difference. This is not to say that either the conduct of
war, or the prevailing attitude toward it, has become less bloodthirsty
or more chivalrous. On the contrary, war hysteria is continuous and
universal in all countries, and such acts as raping, looting, the
slaughter of children, the reduction of whole populations to slavery,
and reprisals against prisoners which extend even to boiling and burying
alive, are looked upon as normal, and, when they are committed by
one\textquotesingle s own side and not by the enemy, meritorious. But in
a physical sense war involves very small numbers of people, mostly
highly trained specialists, and causes comparatively few casualties. The
fighting, when there is any, takes place on the vague frontiers whose
whereabouts the average man can only guess at, or round the Floating
Fortresses which guard strategic spots on the sea lanes. In the centers
of civilization war means no more than a continuous shortage of
consumption goods, and the occasional crash of a rocket bomb which may
cause a few scores of deaths. War has in fact changed its character.
More exactly, the reasons for which war is waged have changed in their
order of importance. Motives which were already present to some small
extent in the great wars of the early twentieth century have now become
dominant and are consciously recognized and acted upon.
\end{quotation}

\switchcolumn

\begin{quotation}
这三个超级大国总是联合一个攻打另一个,总是处于交战状态。过去的二十五年里,一
直如此。但战争已经不再像20世纪初期那样不顾一切,你死我活。交战双方的冲突目标
是有限的,谁都没有能力将对方摧毁,同时既没有重要的开战原因,也没有意识形态上
的分歧。只是这并不意味着战争的方式、态度已不那么血腥,也并不意味着它多了几分
正义。相反,非理性的好战情绪在各国国内都是长期的、普遍的存在,诸如强奸、抢劫、
屠杀儿童、奴役人民、报复,甚至烹煮、活埋战俘等,都被当做理所当然,不仅如此,
如果这些事情是自己而不是敌人所为,还会被认为值得称颂。其实战争只将极少数人卷
入其中,他们大多是受过高级训练的专家,因此战争所造成的伤亡相对较少。战争通常
发生在界线模糊的边境地区,人们只能揣测它的情况,或者发生在扼守海道的水上堡垒
附近。在城市中心,战争的意义仅仅是消费品长期短缺和偶尔落下的火箭弹造成数十人
死亡。实际上战争的特点已经改变。具体说来,战争爆发原因在重要性次序上有了变化。
在20世纪初的战争中居于次要地位的战争动机如今已跃升为主导地位,被有意识地认可
并执行。
\end{quotation}

\switchcolumn*

\begin{quotation}
To understand the nature of the present war---for in spite of the
regrouping which occurs every few years, it is always the same war---one
must realize in the first place that it is impossible for it to be
decisive. None of the three superstates could be definitely conquered
even by the other two in combination. They are too evenly matched, and
their natural defenses are too formidable. Eurasia is protected by its
vast land spaces, Oceania by the width of the Atlantic and the Pacific,
Eastasia by the fecundity and industriousness of its inhabitants.
Secondly, there is no longer, in a material sense, anything to fight
about. With the establishment of self-contained economies, in which
production and consumption are geared to one another, the scramble for
markets which was a main cause of previous wars has come to an end,
while the competition for raw materials is no longer a matter of life
and death. In any case, each of the three superstates is so vast that it
can obtain almost all the materials that it needs within its own
boundaries. In so far as the war has a direct economic purpose, it is a
war for labor power. Between the frontiers of the superstates, and not
permanently in the possession of any of them, there lies a rough
quadrilateral with its corners at Tangier, Brazzaville, Darwin, and Hong
Kong, containing within it about a fifth of the population of the earth.
It is for the possession of these thickly populated regions, and of the
northern ice cap, that the three powers are constantly struggling. In
practice no one power ever controls the whole of the disputed area.
Portions of it are constantly changing hands, and it is the chance of
seizing this or that fragment by a sudden stroke of treachery that
dictates the endless changes of alignment.
\end{quotation}

\switchcolumn

\begin{quotation}
为了理解当下战争的性质——尽管战争中的敌友关系每隔几年就要发生变化,但战争本
身依然是那场战争。首先人们必须明白,它不可能有一个确定的结局。三个超级大国的
任何一个都不可能被彻底征服,甚至不可能被另外两国的联盟彻底征服,它们过于势均
力敌,且彼此间的天然防线过于强大。欧亚国被辽阔的陆地所保护,大洋国依仗大西洋
和太平洋,东亚国有勤劳而善于生产的民众。其次,已不存在让战争爆发的物质诱因。
由于建立了自给自足的经济体制,消费和生产彼此配合。争夺市场已不再是战争的主要
原因,争夺原材料也不再重要。无论如何,这三个大国的面积是如此辽阔,以至于它们
都可以在本国国内找到差不多所有他们所需的原材料。如果说战争有什么直接的经济目
的的话,那就是对劳动力的争夺。在三个大国中间有一块类似长方形的区域,以丹吉尔、
布拉柴维尔、达尔文港和香港为四角,住着全世界约五分之一的人口,没有任何一个国
家能将该区域长期据为己有。为了争夺这块人口稠密的地区以及覆盖着冰雪的北部地区,
三个大国争斗不休。而其实从未有哪个大国有能力控制所有争议地区。一些地区不断易
手,致使结盟关系不断变化,因为各国只有通过突然的背信弃义才有机会夺取某块地方。
\end{quotation}

\switchcolumn*

\begin{quotation}
All of the disputed territories contain valuable minerals, and some of
them yield important vegetable products such as rubber which in colder
climates it is necessary to synthesize by comparatively expensive
methods. But above all they contain a bottomless reserve of cheap labor.
Whichever power controls equatorial Africa, or the countries of the
Middle East, or Southern India, or the Indonesian Archipelago, disposes
also of the bodies of scores or hundreds of millions of ill-paid and
hard-working coolies. The inhabitants of these areas, reduced more or
less openly to the status of slaves, pass continually from conqueror to
conqueror, and are expended like so much coal or oil in the race to turn
out more armaments, to capture more territory, to control more labor
power, to turn out more armaments, to capture more territory, and so on
indefinitely. It should be noted that the fighting never really moves
beyond the edges of the disputed areas. The frontiers of Eurasia flow
back and forth between the basin of the Congo and the northern shore of
the Mediterranean; the islands of the Indian Ocean and the Pacific are
constantly being captured and recaptured by Oceania or by Eastasia; in
Mongolia the dividing line between Eurasia and Eastasia is never stable;
round the Pole all three powers lay claim to enormous territories which
in fact are largely uninhabited and unexplored; but the balance of power
always remains roughly even, and the territory which forms the heartland
of each superstate always remains inviolate. Moreover, the labor of the
exploited peoples round the Equator is not really necessary to the
world\textquotesingle s economy. They add nothing to the wealth of the
world, since whatever they produce is used for purposes of war, and the
object of waging a war is always to be in a better position in which to
wage another war. By their labor the slave populations allow the tempo
of continuous warfare to be speeded up. But if they did not exist, the
structure of world society, and the process by which it maintains
itself, would not be essentially different.
\end{quotation}

\switchcolumn

\begin{quotation}
所有争议地区都蕴藏着珍贵的矿藏,其中一些还出产重要的植物产品,如橡胶。在较为
寒冷的地区只能通过费用高昂的人工手段合成出来。但最为重要的是取之不尽的廉价的
劳动力储备。不管是谁,只要有能力控制赤道非洲,或者中东地区、南印度、印度尼西
亚群岛,就相当于拥有了亿万个价格低廉、工作勤力的苦力。这些地区的居民已或多或
少沦为奴隶,不断地从一个征服者手里转到另一个征服者手里,他们就像煤和石油等消
费品,被用来生产更多的军备、占据更多的土地、控制更多的劳力,再生产更多的军备、
占据更多的土地、控制更多的劳力,周而复始,永不停止。不过应该注意的是,战争从
来没有超过争议地区的边界。欧亚国的边界在刚果河盆地和地中海北岸游移,印度洋和
太平洋的岛屿时而被大洋国占领,时而被东亚国占领。在蒙古,欧亚国和东亚国的边界
线一直没有固定下来。而在北极地区,尽管三国都宣称自己才是其广袤土地的拥有者,
可实际上这里既无人烟,也没有经过勘测。总之,三国始终保持着力量上的平衡,其中
心地带都没有遭遇过侵犯。此外,赤道地区被剥削的民众的劳动力对世界经济而言也并
非必不可缺。他们不能为世界增添什么财富。不管他们生产了什么,都要被用于战争,
而战争又总以``在发动另一场战争时居于更有利的地位''作为目的。凭借着这些被奴役
人口的劳动力这场绵延不绝的战争的进行速度加快了。但,倘若他们不存在,全球社会
的结构以及维持这种结构的形式也不会有本质的不同。
\end{quotation}

\switchcolumn*

\begin{quotation}
The primary aim of modern warfare (in accordance with the principles of
\emph{doublethink}, this aim is simultaneously recognized and not
recognized by the directing brains of the Inner Party) is to use up the
products of the machine without raising the general standard of living.
Ever since the end of the nineteenth century, the problem of what to do
with the surplus of consumption goods has been latent in industrial
society. At present, when few human beings even have enough to eat, this
problem is obviously not urgent, and it might not have become so, even
if no artificial processes of destruction had been at work. The world of
today is a bare, hungry, dilapidated place compared with the world that
existed before 1914, and still more so if compared with the imaginary
future to which the people of that period looked forward. In the early
twentieth century, the vision of a future society unbelievably rich,
leisured, orderly and efficient---a glittering antiseptic world of glass
and steel and snow-white concrete---was part of the consciousness of
nearly every literate person. Science and technology were developing at
a prodigious speed, and it seemed natural to assume that they would go
on developing. This failed to happen, partly because of the
impoverishment caused by a long series of wars and revolutions, partly
because scientific and technical progress depended on the empirical
habit of thought, which could not survive in a strictly regimented
society. As a whole the world is more primitive today than it was fifty
years ago. Certain backward areas have advanced, and various devices,
always in some way connected with warfare and police espionage, have
been developed, but experiment and invention have largely stopped, and
the ravages of the atomic war of the Nineteen-fifties have never been
fully repaired. Nevertheless the dangers inherent in the machine are
still there. From the moment when the machine first made its appearance
it was clear to all thinking people that the need for human drudgery,
and therefore to a great extent for human inequality, had disappeared.
If the machine were used deliberately for that end, hunger, overwork,
dirt, illiteracy, and disease could be eliminated within a few
generations. And in fact, without being used for any such purpose, but
by a sort of automatic process---by producing wealth which it was
sometimes impossible not to distribute---the machine did raise the
living standards of the average human being very greatly over a period
of about fifty years at the end of the nineteenth and the beginning of
the twentieth centuries.
\end{quotation}

\switchcolumn

\begin{quotation}
现代战争的主要目标(依照双重思想的原则,对这一目标,内党的智囊既承认又不承认)
是消耗掉机器生产的产品而不提高总体的生活水平。从19世纪末开始,如何处理剩余消
费品的问题就潜藏在工业社会内。现在,几乎没有人吃得饱饭,这个问题显然并不紧迫,
就算不人为地销毁剩余消费品,该问题可能也不会迫切。今天的世界和1914年以前相比,
贫乏、饥饿、破败,如果和那个时代的人所畅想的未来相比,则更是如此。20世纪早期,
人们想象中的未来富足、安逸、有序、高效——那是一个由玻璃和洁白如雪的混凝土构
建起的闪亮而清洁的世界——它是几乎每个有文化的人意识的一部分。当时,科技正迅
猛发展,且看起来会一直这样发展下去。但这并未发生,一部分因为长时间的战争和革
命造成了贫困,一部分因为科技的进步需要经验主义的思维习惯,这种习惯又无法在管
制严格的社会中存在。总而言之,今天的世界比五十年前更加原始。一些落后的地区虽
然有所发展,且一些设备——在某种程度上总和战争、警察的侦察活动有关——也有所
进步,但大部分实验和发明都停止了。50年代核战造成的破坏从未恢复完全。机器蕴涵
的危险仍然存在。从机器出现的那天起,所有有思考能力的人都以为,人类再不用做那
些苦差事了,人与人之间的不平等也将在很大程度上消弭。如果当初有意识地站在这个
立场上使用机器,那么饥饿、过劳、肮脏、文盲、疾病就可以在几代人之内消失。实际
上,机器虽然没有以此为目的为人使用,但这种状况仍自然而然地发生了——由于有时
机器生产的财富不得不被分配掉,在19世纪末到20世纪初的五十多年里,机器在客观上
确实提高了人们总体的生活水平。
\end{quotation}

\switchcolumn*

\begin{quotation}
But it was also clear that an all-round increase in wealth threatened
the destruction---indeed, in some sense was the destruction---of a
hierarchical society. In a world in which everyone worked short hours,
had enough to eat, lived in a house with a bathroom and a refrigerator,
and possessed a motorcar or even an airplane, the most obvious and
perhaps the most important form of inequality would already have
disappeared. If it once became general, wealth would confer no
distinction. It was possible, no doubt, to imagine a society in which
wealth, in the sense of personal possessions and luxuries, should be
evenly distributed, while \emph{power} remained in the hands of a small
privileged caste. But in practice such a society could not long remain
stable. For if leisure and security were enjoyed by all alike, the great
mass of human beings who are normally stupefied by poverty would become
literate and would learn to think for themselves; and when once they had
done this, they would sooner or later realize that the privileged
minority had no function, and they would sweep it away. In the long run,
a hierarchical society was only possible on a basis of poverty and
ignorance. To return to the agricultural past, as some thinkers about
the beginning of the twentieth century dreamed of doing, was not a
practicable solution. It conflicted with the tendency toward
mechanization which had become quasi-instinctive throughout almost the
whole world, and moreover, any country which remained industrially
backward was helpless in a military sense and was bound to be dominated,
directly or indirectly, by its more advanced rivals.\par
Nor was it a satisfactory solution to keep the masses in poverty by
restricting the output of goods. This happened to a great extent during
the final phase of capitalism, roughly between 1920 and 1940. The
economy of many countries was allowed to stagnate, land went out of
cultivation, capital equipment was not added to, great blocks of the
population were prevented from working and kept half alive by State
charity. But this, too, entailed military weakness, and since the
privations it inflicted were obviously unnecessary, it made opposition
inevitable. The problem was how to keep the wheels of industry turning
without increasing the real wealth of the world. Goods must be produced,
but they must not be distributed. And in practice the only way of
achieving this was by continuous warfare.
\end{quotation}

\switchcolumn

\begin{quotation}
但同样明确的是,从某种意义上说,财富的全面增长的确会带来毁灭性的危险——对等
级社会的毁灭。如果世界上每个人都只工作很短的时间就能填饱肚子,住上有浴室和冰
箱的房子,拥有私人汽车甚至飞机,那么最显而易见也最重要的不平等形式可能就不复
存在。若财富普及开来,就不会有财富的差别。可能,毫无疑问,可以设想这样一个社
会,在私人财产和奢侈品方面,财富平均分配,而权力仍把持在少数享有特权的人手里。
只是,在现实中这种社会不可能保持长期的稳定。因为如果每个人都能拥有安逸且有保
障的生活,那么原先因贫困而蒙昧的大多数人就可以成为有文化且能够独立思考的人。
而一旦他们做到这点,则早晚会意识到少数特权阶层毫无用途,进而会将其清除。就长
远来看,等级社会只能建立在贫困和无知的基础上。20世纪初部分思想家梦想回到过去
的农业社会,这不能解决实际的问题。它和机械化的大趋势相冲,而后者几乎成了这个
世界的发展本能。更何况,任何一个国家若在工业上落后了,其军事便无所依靠,它将
或直接或间接地受制于比它先进的竞争对手。
\end{quotation}

\switchcolumn*

\begin{quotation}
The essential act of war is destruction, not necessarily of human lives,
but of the products of human labor. War is a way of shattering to
pieces, or pouring into the stratosphere, or sinking in the depths of
the sea, materials which might otherwise be used to make the masses too
comfortable, and hence, in the long run, too intelligent. Even when
weapons of war are not actually destroyed, their manufacture is still a
convenient way of expending labor power without producing anything that
can be consumed. A Floating Fortress, for example, has locked up in it
the labor that would build several hundred cargo ships. Ultimately it is
scrapped as obsolete, never having brought any material benefit to
anybody, and with further enormous labors another Floating Fortress is
built. In principle the war effort is always so planned as to eat up any
surplus that might exist after meeting the bare needs of the population.
In practice the needs of the population are always underestimated, with
the result that there is a chronic shortage of half the necessities of
life; but this is looked on as an advantage. It is deliberate policy to
keep even the favored groups somewhere near the brink of hardship,
because a general state of scarcity increases the importance of small
privileges and thus magnifies the distinction between one group and
another. By the standards of the early twentieth century, even a member
of the Inner Party lives an austere, laborious kind of life.
Nevertheless, the few luxuries that he does enjoy---his large
well-appointed flat, the better texture of his clothes, the better
quality of his food and drink and tobacco, his two or three servants,
his private motorcar or helicopter---set him in a different world from a
member of the Outer Party, and the members of the Outer Party have a
similar advantage in comparison with the submerged masses whom we call
``the proles.'' The social atmosphere is that of a besieged city, where
the possession of a lump of horseflesh makes the difference between
wealth and poverty. And at the same time the consciousness of being at
war, and therefore in danger, makes the handing-over of all power to a
small caste seem the natural, unavoidable condition of survival.
\end{quotation}

\switchcolumn

\begin{quotation}
用限制商品生产的办法让民众一直处在贫困之中,并不是令人满意的解决之道。资本主
义末期,即大约1920年到1940年间,曾大范围地使用这一办法。很多国家坐视经济停滞,
土地荒芜,并拒绝增加资本设备。以至于相当多的人找不到工作,只能靠政府救济艰难
度日。而这也造成了军事上的疲软无力。由于限制生产所造成的贫困毫无必要,则注定
遭到人们的反对。问题就在这里:人们要如何既确保经济的轮子持续转动,又不让这个
世界上的财富有所增加?必须生产商品,但商品不一定要分配出去。为此,切实可行的
办法便是不停地进行战争。
\end{quotation}

\switchcolumn*

\begin{quotation}
War, it will be seen, not only accomplishes the necessary destruction,
but accomplishes it in a psychologically acceptable way. In principle it
would be quite simple to waste the surplus labor of the world by
building temples and pyramids, by digging holes and filling them up
again, or even by producing vast quantities of goods and then setting
fire to them. But this would provide only the economic and not the
emotional basis for a hierarchical society. What is concerned here is
not the morale of the masses, whose attitude is unimportant so long as
they are kept steadily at work, but the morale of the Party itself. Even
the humblest Party member is expected to be competent, industrious, and
even intelligent within narrow limits, but it is also necessary that he
should be a credulous and ignorant fanatic whose prevailing moods are
fear, hatred, adulation, and orgiastic triumph. In other words it is
necessary that he should have the mentality appropriate to a state of
war. It does not matter whether the war is actually happening, and,
since no decisive victory is possible, it does not matter whether the
war is going well or badly. All that is needed is that a state of war
should exist. The splitting of the intelligence which the Party requires
of its members, and which is more easily achieved in an atmosphere of
war, is now almost universal, but the higher up the ranks one goes, the
more marked it becomes. It is precisely in the Inner Party that war
hysteria and hatred of the enemy are strongest. In his capacity as an
administrator, it is often necessary for a member of the Inner Party to
know that this or that item of war news is untruthful, and he may often
be aware that the entire war is spurious and is either not happening or
is being waged for purposes quite other than the declared ones; but such
knowledge is easily neutralized by the technique of \emph{doublethink}.
Meanwhile no Inner Party member wavers for an instant in his mystical
belief that the war is real, and that it is bound to end victoriously,
with Oceania the undisputed master of the entire world.
\end{quotation}

\switchcolumn

\begin{quotation}
战争最基本的行为就是毁灭,不一定要毁灭人的生命,还要毁灭人的劳动产品。物质很
可能会让人的生活过于安逸,从长远而言亦会让人过于聪明,而战争就是将物质撕碎,
或升腾成烟,或倾入深海的一种途径。即便在战争中,武器没有被销毁,其生产仍不失
为既消耗劳动力又不生产任何消费品的便捷方式。举个例子,建造一座水上堡垒所耗费
的劳动力可以制造出数百艘货轮。而它最终会因废弃被拆成废料,不会给任何人带来任
何物质利益。但这之后,还会用大量的劳动力建造另一个水上堡垒。原则上,战争总是
努力计划将满足人们最低需求后可能产生的剩余物质消耗干净,现实中人们的需求又总
被低估,导致有一半的生活必需品长期供不应求。但这却被认为是有益的,是经过审慎
思考的政策,甚至一些利益团体也徘徊在困苦边缘,因为普遍的匮乏可以凸显小特权的
重要性并扩大不同阶级的差别。用20世纪初的标准来衡量,即使是内党党员,其生活也
称得上简朴,其工作也算得上繁重。然而,他却有为数不多的奢侈享受——设施完备又
宽敞的住房、质地更好的衣服、质量更高的烟酒饮食、两三个仆人、私人汽车和直升
机——这让他的生活和外党党员截然不同。与之类似,和被称作``群众''的大多数底层
民众相比,外党党员又处在更有利的境地。社会就像一座封闭的城市,一块马肉就能体
现出贫富之别,有即富,无则贫。同时,由于意识到正身处战争,人人自危,要想生存
下去,就理所当然地要将所有权力都交给少部分人。
\end{quotation}

\switchcolumn*

\begin{quotation}
All members of the Inner Party believe in this coming conquest as an
article of faith. It is to be achieved either by gradually acquiring
more and more territory and so building up an overwhelming preponderance
of power, or by the discovery of some new and unanswerable weapon. The
search for new weapons continues unceasingly, and is one of the very few
remaining activities in which the inventive or speculative type of mind
can find any outlet. In Oceania at the present day, Science, in the old
sense, has almost ceased to exist. In Newspeak there is no word for
``Science.'' The empirical method of thought, on which all the scientific
achievements of the past were founded, is opposed to the most
fundamental principles of Ingsoc. And even technological progress only
happens when its products can in some way be used for the diminution of
human liberty. In all the useful arts the world is either standing still
or going backwards. The fields are cultivated with horse plows while
books are written by machinery. But in matters of vital
importance---meaning, in effect, war and police espionage---the
empirical approach is still encouraged, or at least tolerated. The two
aims of the Party are to conquer the whole surface of the earth and to
extinguish once and for all the possibility of independent thought.
There are therefore two great problems which the Party is concerned to
solve. One is how to discover, against his will, what another human
being is thinking, and the other is how to kill several hundred million
people in a few seconds without giving warning beforehand. In so far as
scientific research still continues, this is its subject matter. The
scientist of today is either a mixture of psychologist and inquisitor,
studying with extraordinary minuteness the meaning of facial
expressions, gestures, and tones of voice, and testing the
truth-producing effects of drugs, shock therapy, hypnosis, and physical
torture; or he is a chemist, physicist, or biologist concerned only with
such branches of his special subject as are relevant to the taking of
life. In the vast laboratories of the Ministry of Peace, and in the
experimental stations hidden in the Brazilian forests, or in the
Australian desert, or on lost islands of the Antarctic, the teams of
experts are indefatigably at work. Some are concerned simply with
planning the logistics of future wars; others devise larger and larger
rocket bombs, more and more powerful explosives, and more and more
impenetrable armor-plating; others search for new and deadlier gases, or
for soluble poisons capable of being produced in such quantities as to
destroy the vegetation of whole continents, or for breeds of disease
germs immunized against all possible antibodies; others strive to
produce a vehicle that shall bore its way under the soil like a
submarine under the water, or an airplane as independent of its base as
a sailing ship; others explore even remoter possibilities such as
focusing the sun\textquotesingle s rays through lenses suspended
thousands of kilometers away in space, or producing artificial
earthquakes and tidal waves by tapping the heat at the
earth\textquotesingle s center.
\end{quotation}

\switchcolumn

\begin{quotation}
可以看出,战争摧毁了必须摧毁的东西,人们在心理上也能接受其所使用的方式。原则
上,建庙宇、修金字塔,挖出一个坑再将它填上,或者将生产出的大量物品烧成灰烬,
都可以消耗剩余劳力。但它只能为等级社会提供经济基础,提供不了感情基础。这里需
要关心的不是群众的感情,只要群众不停止工作,他们的态度就无关紧要。真正需要关
心的是党员的情绪。即便是地位最卑微的党员也被要求有能力、勤劳、在特定范围内头
脑聪明,同时又必须具备轻信、盲从、狂热的特点,恐惧、仇恨、崇拜、狂喜应占据他
情绪的主要部分。用另一种方式说,他的精神应该和战争两相呼应。战争是否真的发生
了并不重要,考虑到不可能有决定性的胜利,战争进行得如何也没什么重要。重要的是
在战争的氛围下,党员更容易做到党所要求的智力的分裂——这在现在已非常普遍,党
员的地位越高,表现得就越明显。论起对战争的狂热和对敌人的仇恨,内党的态度最为
强烈。作为管理者,一个内党党员必须知道这条或那条和战争有关的消息并不真实,他
可能经常发现整场战争都是捏造的,要么现在没有发生,要么其目的完全不是宣称的那
样。但是,在双重思想的作用下,他所了解的很容易被消除。没有一个内党党员不坚信
战争真实存在,他们那神秘的信念从不曾动摇,他们相信大洋国一定能取得最终胜利,
成为这个世界无可争议的主宰者。
\end{quotation}

\switchcolumn*

\begin{quotation}
But none of these projects ever comes anywhere near realization, and
none of the three superstates ever gains a significant lead on the
others. What is more remarkable is that all three powers already
possess, in the atomic bomb, a weapon far more powerful than any that
their present researches are likely to discover. Although the Party,
according to its habit, claims the invention for itself, atomic bombs
first appeared as early as the Nineteen-forties, and were first used on
a large scale about ten years later. At that time some hundreds of bombs
were dropped on industrial centers, chiefly in European Russia, Western
Europe, and North America. The effect was to convince the ruling groups
of all countries that a few more atomic bombs would mean the end of
organized society, and hence of their own power. Thereafter, although no
formal agreement was ever made or hinted at, no more bombs were dropped.
All three powers merely continue to produce atomic bombs and store them
up against the decisive opportunity which they all believe will come
sooner or later. And meanwhile the art of war has remained almost
stationary for thirty or forty years. Helicopters are more used than
they were formerly, bombing planes have been largely superseded by
self-propelled projectiles, and the fragile movable battleship has given
way to the almost unsinkable Floating Fortress; but otherwise there has
been little development. The tank, the submarine, the torpedo, the
machine gun, even the rifle and the hand grenade are still in use. And
in spite of the endless slaughters reported in the press and on the
telescreens, the desperate battles of earlier wars, in which hundreds of
thousands or even millions of men were often killed in a few weeks, have
never been repeated.
\end{quotation}

\switchcolumn

\begin{quotation}
所有内党党员都坚信他们即将征服世界。这种征服要么通过不断地占领土地、树立压倒
性的力量来实现,要么通过研发所向披靡的新式武器来实现。研发新式武器的工作持续
不断地进行着,这是有创造力又擅长思考的人所能找到的为数不多的能发挥才智的活动
之一。在今天的大洋国,传统学科已不复存在,新话中没有``科学''一词。过去的科学
成就建立在以经验主义为基础的思维方法上,和英社的根本原则相违背。技术的进步只
能发生在其产品将以某种方式限制人类自由的情况下。所有实用技术不是裹足不前,就
是出现倒退。耕田要靠马拉犁,书籍要用机器写。但在重要问题上——即战争和警察的
侦察活动——仍鼓励,或者至少允许使用经验的方法。党有两个目标:征服世界和彻底
消灭独立思考。因此,党需要解决两个大问题。一个是如何违背一个人的意愿,知道他
到底在想什么。另一个是,如何在几秒钟之内不加警告地杀死数亿人。科学活动之所以
仍在进行,就是为了这两个题目。因此今天只存在两种科学家,一种集心理学家和审讯
者于一身,对人的表情、姿态、声调所包含的意义作细致入微的研究,他们考察药物、
休克疗法、催眠、严刑拷打在促使人吐露实情上所起到的作用;另一种是化学家、物理
学家和生物学家,他们只关心自己专业中和杀人有关的东西。他们不知疲倦地工作在和
平部那巨大的实验室里,工作在巴西密林的深处、澳大利亚的沙漠,以及人迹罕至的南
极小岛上隐藏的实验站里。他们有的专注于制定未来战争的后勤计划;有的则执迷于设
计越来越大的火箭弹、越来越强的爆炸物和越来越厚的装甲板;有的在寻找更加致命的
毒气,可以将地球上所有植物都杀死的、能够进行大规模生产的可溶性毒药,以及不惧
任何抗生素的病菌;有的正致力于建造类似潜艇在水里行驶那样在地下穿梭的汽车和像
帆船那样无需基地的飞机;还有的在研究遥不可及的东西,比如在数千公里外的太空中
安置透镜,将太阳的光芒聚集起来,在地球中间开个孔人为地制造地震和海啸。
\end{quotation}

\switchcolumn*

\begin{quotation}
None of the three superstates ever attempts any maneuver which involves
the risk of serious defeat. When any large operation is undertaken, it
is usually a surprise attack against an ally. The strategy that all
three powers are following, or pretend to themselves that they are
following, is the same. The plan is, by a combination of fighting,
bargaining, and well-timed strokes of treachery, to acquire a ring of
bases completely encircling one or other of the rival states, and then
to sign a pact of friendship with that rival and remain on peaceful
terms for so many years as to lull suspicion to sleep. During this time
rockets loaded with atomic bombs can be assembled at all the strategic
spots; finally they will all be fired simultaneously, with effects so
devastating as to make retaliation impossible. It will then be time to
sign a pact of friendship with the remaining world power, in preparation
for another attack. This scheme, it is hardly necessary to say, is a
mere daydream, impossible of realization. Moreover, no fighting ever
occurs except in the disputed areas round the Equator and the Pole; no
invasion of enemy territory is ever undertaken. This explains the fact
that in some places the frontiers between the superstates are arbitrary.
Eurasia, for example, could easily conquer the British Isles, which are
geographically part of Europe, or on the other hand it would be possible
for Oceania to push its frontiers to the Rhine or even to the Vistula.
But this would violate the principle, followed on all sides though never
formulated, of cultural integrity. If Oceania were to conquer the areas
that used once to be known as France and Germany, it would be necessary
either to exterminate the inhabitants, a task of great physical
difficulty, or to assimilate a population of about a hundred million
people, who, so far as technical development goes, are roughly on the
Oceanic level. The problem is the same for all three superstates. It is
absolutely necessary to their structure that there should be no contact
with foreigners except, to a limited extent, with war prisoners and
colored slaves. Even the official ally of the moment is always regarded
with the darkest suspicion. War prisoners apart, the average citizen of
Oceania never sets eyes on a citizen of either Eurasia or Eastasia, and
he is forbidden the knowledge of foreign languages. If he were allowed
contact with foreigners he would discover that they are creatures
similar to himself and that most of what he has been told about them is
lies. The sealed world in which he lives would be broken, and the fear,
hatred, and self-righteousness on which his morale depends might
evaporate. It is therefore realized on all sides that however often
Persia, or Egypt, or Java, or Ceylon may change hands, the main
frontiers must never be crossed by anything except bombs.
\end{quotation}

\switchcolumn

\begin{quotation}
这些计划没有一个能够完成,三个超级大国也没有哪个有能力领先另外两个。三个大国
都已拥有原子弹,它的威力比它们正在研发的任何武器都大得多。党总是习惯说自己是
原子弹的发明者,而原子弹早在20世纪40年代就出现了,在大约十年后开始大规模使用。
当时,数以百计的原子弹落在了工业中心,主要集中在俄国的欧洲部分、西欧和北美。
它让所有国家的统治集团都意识到再扔几颗原子弹,有组织的社会就会完蛋,他们的权
力就会随之终结。因此从此以后,虽然没有签署或暗示有什么正式协定,再也没有原子
弹掉下来。三个大国仅仅是制造、储备原子弹,他们相信迟早有一天,他们要一决胜负。
同时,在长达三四十年之久的时间里,各国在战术上几乎都没有取得进展。当然,相比
从前,直升飞机使用得更加频繁,轰炸机在很大程度上已经被自动推进的炮弹取代,脆
弱的军舰也让位给不易沉没的水上堡垒。但在其他方面就进步寥寥。坦克、潜艇、鱼雷、
机枪,甚至步枪和手榴弹仍在使用。尽管报纸和电屏都没完没了地播着和杀戮有关的报
道,但早期战争中那种在几个星期里就杀死数十万,甚至数百万人的情况再未发生。
\end{quotation}

\switchcolumn*

\begin{quotation}
Under this lies a fact never mentioned aloud, but tacitly understood and
acted upon: namely, that the conditions of life in all three superstates
are very much the same. In Oceania the prevailing philosophy is called
Ingsoc, in Eurasia it is called Neo-Bolshevism, and in Eastasia it is
called by a Chinese name usually translated as Death-worship, but
perhaps better rendered as Obliteration of the Self. The citizen of
Oceania is not allowed to know anything of the tenets of the other two
philosophies, but he is taught to execrate them as barbarous outrages
upon morality and common sense. Actually the three philosophies are
barely distinguishable, and the social systems which they support are
not distinguishable at all. Everywhere there is the same pyramidal
structure, the same worship of a semi-divine leader, the same economy
existing by and for continuous warfare. It follows that the three
superstates not only cannot conquer one another, but would gain no
advantage by doing so. On the contrary, so long as they remain in
conflict they prop one another up, like three sheaves of corn. And, as
usual, the ruling groups of all three powers are simultaneously aware
and unaware of what they are doing. Their lives are dedicated to world
conquest, but they also know that it is necessary that the war should
continue everlastingly and without victory. Meanwhile the fact that
there is no danger of conquest makes possible the denial of reality
which is the special feature of Ingsoc and its rival systems of thought.
Here it is necessary to repeat what has been said earlier, that by
becoming continuous war has fundamentally changed its character.
\end{quotation}

\switchcolumn

\begin{quotation}
在战略部署上,三个超级大国都不敢冒失败的风险,其所有大规模军事行动通常都以对
盟国的突然袭击为开始。三个大国的战略都一样,即通过战斗、谈判和时机掌握得刚刚
好的背叛,夺取一批基地,将敌国包围起来。然后和该国签订友好条约,保持多年的和
平关系,让对方放下戒心,疏忽大意。再趁机将装有核弹头的火箭集中在所有战略要点,
让它们在同一时间发射,给对方造成毁灭性打击,令对方失去反击能力。之后,再和剩
下的国家签订友好条约,为下一次突袭作准备。这种计划简直不值一说,它无异于白日
梦,没有丝毫实现的可能。不仅如此,在赤道及北极等争议地区外,还没有发生过战争,
也没有哪个国家入侵过其他国家。这就解释了为什么超级大国之间的一些地区其界线是
随意划定的。比如欧亚国可以轻而易举地拿下不列颠诸岛,后者从地理上看属于欧洲。
而大洋国也可以将边界拓展到莱茵河,甚至维斯图拉河。但这样一来就违背了各国都遵
守的不成文的文化完整原则,如果大洋国要占领曾被称作法兰西和德意志的地方,它要
么将当地居民全部消灭,要么将大约一亿人同化成在技术方面和大洋国居民水平相当的
人。这实施起来极其困难。对此,三个大国面临的问题都一样。除了和战俘、黑人奴隶
进行有限的接触,他们的国家结构决定他们绝对不能和外国人有任何联系。即便是对正
式的盟友,他们也总带着深深的怀疑。刨去战俘,普通的大洋国公民从来没有见过一个
欧亚国或东亚国的公民,就连学习外语都遭到禁止。如果允许他和外国人接触,他会发
现他们和自己一样,会知道他被告知的大部分和外国人有关的事情都是谎言。如此,他
所生活的封闭世界就会被打破,他的信念,譬如恐惧、仇恨以及自以为是的正义都会蒸
发消失。因此,三国都清楚,不管波斯、埃及、爪哇、锡兰怎样几经易手,除了炮弹,
都不能越过其主要边界。
\end{quotation}

\switchcolumn*

\begin{quotation}
In past ages, a war, almost by definition, was something that sooner or
later came to an end, usually in unmistakable victory or defeat. In the
past, also, war was one of the main instruments by which human societies
were kept in touch with physical reality. All rulers in all ages have
tried to impose a false view of the world upon their followers, but they
could not afford to encourage any illusion that tended to impair
military efficiency. So long as defeat meant the loss of independence,
or some other result generally held to be undesirable, the precautions
against defeat had to be serious. Physical facts could not be ignored.
In philosophy, or religion, or ethics, or politics, two and two might
make five, but when one was designing a gun or an airplane they had to
make four. Inefficient nations were always conquered sooner or later,
and the struggle for efficiency was inimical to illusions. Moreover, to
be efficient it was necessary to be able to learn from the past, which
meant having a fairly accurate idea of what had happened in the past.
Newspapers and history books were, of course, always colored and biased,
but falsification of the kind that is practiced today would have been
impossible. War was a sure safeguard of sanity, and so far as the ruling
classes were concerned it was probably the most important of all
safeguards. While wars could be won or lost, no ruling class could be
completely irresponsible.
\end{quotation}

\switchcolumn

\begin{quotation}
其中,有个事实虽未被大声提起,却会被三国默认并视为行为准则:三个大国的生活水
平都差不多。大洋国盛行的哲学是英社原则,在欧亚国它叫新布尔什维主义,到了东亚
国它有个中文名字,通常翻译成``崇死'',但也许译成``自我消除''更好。大洋国的公
民禁止了解其他两国的哲学原则,却被教导要憎恨它们,将它们当作对道德、常识的野
蛮践踏。其实这三种哲学难分你我,其所拥护的社会制度也是如此。哪里都是金字塔式
的建筑,哪里都有对宛若半神的领袖的崇拜,哪里都在用战争维系经济并让经济为战争
服务。这就是为什么这三个超级大国都不能将对方征服的原因。此外,就算征服了也没
什么利益可得。而相反,只要三个国家一直在打仗,就会像三束捆在一起的秸秆一般彼
此支撑。通常来说,三国的统治集团对他们的所作所为既了解又不了解。他们为征服世
界而活,但同时他们也明白战争必须永远进行下去,不能有胜利。而征服世界可能会使
他们拒绝面对现实,但却不会对什么东西造成威胁。这正是英社及和它对立的另外两种
思想体系的特点。在此,有必要重复一下之前说过的话,持续不断的战争已经从根本上
改变了战争本身的性质。
\end{quotation}

\switchcolumn*

\begin{quotation}
But when war becomes literally continuous, it also ceases to be
dangerous. When war is continuous there is no such thing as military
necessity. Technical progress can cease and the most palpable facts can
be denied or disregarded. As we have seen, researches that could be
called scientific are still carried out for the purposes of war, but
they are essentially a kind of daydreaming, and their failure to show
results is not important. Efficiency, even military efficiency, is no
longer needed. Nothing is efficient in Oceania except the Thought
Police. Since each of the three superstates is unconquerable, each is in
effect a separate universe within which almost any perversion of thought
can be safely practiced. Reality only exerts its pressure through the
needs of everyday life---the need to eat and drink, to get shelter and
clothing, to avoid swallowing poison or stepping out of top-story
windows, and the like. Between life and death, and between physical
pleasure and physical pain, there is still a distinction, but that is
all. Cut off from contact with the outer world, and with the past, the
citizen of Oceania is like a man in interstellar space, who has no way
of knowing which direction is up and which is down. The rulers of such a
state are absolute, as the Pharaohs or the Caesars could not be. They
are obliged to prevent their followers from starving to death in numbers
large enough to be inconvenient, and they are obliged to remain at the
same low level of military technique as their rivals; but once that
minimum is achieved, they can twist reality into whatever shape they
choose.
\end{quotation}

\switchcolumn

\begin{quotation}
在过去,以战争的定义来看,战争早晚会有结束的一天,要么胜利,要么失败,结果一
清二楚。曾几何时,战争也是人类社会和客观世界保持联系的主要手段之一。每个时代
的统治者都试图将虚假的世界观强加给他的追随者,但他们却不会鼓励那种有可能损害
其军事效能的幻想。战败就意味着丧失独立,或者其他糟糕的事,因此必须认真地提前
采取措施以确保胜利。客观现实不能被忽视。在哲学、宗教、伦理、政治等领域,二加
二可能等于五。但在设计枪支或飞机时,二加二必须等于四。效益低下的民族迟早会被
征服,幻想又无益于效益的提高。更何况,提高效益有必要向过去学习,这就意味着人
们要对发生过的事有准确而公正的判断。当然,之前的报纸和历史书总不免带有主观色
彩和偏见,但却不可能像今天这样伪造事实。战争确能让人保持清醒,对统治集团来说,
它很可能是最重要的保持清醒的方式。战争也许会输,也许会赢,没有哪个统治集团能
完全地置身事外。
\end{quotation}

\switchcolumn*

\begin{quotation}
The war, therefore, if we judge it by the standards of previous wars, is
merely an imposture. It is like the battles between certain ruminant
animals whose horns are set at such an angle that they are incapable of
hurting one another. But though it is unreal it is not meaningless. It
eats up the surplus of consumable goods, and it helps to preserve the
special mental atmosphere that a hierarchical society needs. War, it
will be seen, is now a purely internal affair. In the past, the ruling
groups of all countries, although they might recognize their common
interest and therefore limit the destructiveness of war, did fight
against one another, and the victor always plundered the vanquished. In
our own day they are not fighting against one another at all. The war is
waged by each ruling group against its own subjects, and the object of
the war is not to make or prevent conquests of territory, but to keep
the structure of society intact. The very word ``war,'' therefore, has
become misleading. It would probably be accurate to say that by becoming
continuous war has ceased to exist. The peculiar pressure that it
exerted on human beings between the Neolithic Age and the early
twentieth century has disappeared and been replaced by something quite
different. The effect would be much the same if the three superstates,
instead of fighting one another, should agree to live in perpetual
peace, each inviolate within its own boundaries. For in that case each
would still be a self-contained universe, freed forever from the
sobering influence of external danger. A peace that was truly permanent
would be the same as a permanent war. This---although the vast majority
of Party members understand it only in a shallower sense---is the inner
meaning of the Party slogan: \textsc{WAR IS PEACE}.
\end{quotation}

\switchcolumn

\begin{quotation}
但是,当战争确实变得没完没了时,它就不再危险,也不存在什么军事的需要。技术的
进步可以停止,再明确的事实都可以忽视、否认。正像我们看到的那样,尽管被称作
``科学的''研究一直在为战争服务,但它们从本质上看无非是白日梦,没有成果也不要
紧。效益,即便是军事效益,都不被需要。在大洋国,除了思想警察,没有哪件事能产
生效益。三个超级大国都不会被征服,每个国家都是一个独立的小世界,几乎任何颠倒
是非的观念都可以安然地于其中大行其道,现实的压力只表现在日常生活中的需求
上——对吃喝的需求,对住房的需求,对服装的需求,以及对避免服毒、避免从高处坠
落的需求。在生与死之间,在肉体的痛苦与快乐上,仍存在差别,但也仅仅是这样罢了。
大洋国的公民就像生活在星际之间,他们与外界隔绝,与过去隔绝,不知自己身处何处。
在这样的国家,统治者即独裁者,法老或恺撒都未曾如此。为避免对自己不利,他们不
能让自己的人民饿死太多,也必须让自己的军事技术水平和对手的一样低。一旦这些条
件得到最低限度的满足,他们就可以将现实扭曲成任何他们想要的样子。
\end{quotation}

\switchcolumn*

Winston stopped reading for a moment. Somewhere in remote distance a
rocket bomb thundered. The blissful feeling of being alone with the
forbidden book, in a room with no telescreen, had not worn off. Solitude
and safety were physical sensations, mixed up somehow with the tiredness
of his body, the softness of the chair, the touch of the faint breeze
from the window that played upon his cheek. The book fascinated him, or
more exactly it reassured him. In a sense it told him nothing that was
new, but that was part of the attraction. It said what he would have
said, if it had been possible for him to set his scattered thoughts in
order. It was the product of a mind similar to his own, but enormously
more powerful, more systematic, less fear-ridden. The best books, he
perceived, are those that tell you what you know already. He had just
turned back to Chapter 1 when he heard Julia\textquotesingle s footstep
on the stair and started out of his chair to meet her. She dumped her
brown tool bag on the floor and flung herself into his arms. It was more
than a week since they had seen one another.

\switchcolumn

\begin{quotation}
因此,若我们用战争的传统定义来衡量现在的战争,后者就是假的,它就好像两头反刍
动物间的争斗,其犄角的角度决定了它们不会让彼此受伤。但是,战争虽不真实,却仍
有意义。它耗尽了剩余消费品,有助于保持等级社会所需的特殊的精神氛围。战争,就
现在来看,完全成了内政。在过去,尽管各国的统治集团可能因为意识到他们的共同利
益而对战争的破坏性加以限制,但他们依然会攻击对方,胜利者还总会对失败者进行掠
夺。而到我们生活的时代,国家之间不再相互争斗,统治集团对自己的国民发动战争,
战争的目的既不是征服他国也不是保卫本国,而是确保社会结构的完整。因此,战争一
词便会让人产生误解。确切地说,也许是这样,当战争变得没完没了,战争本身就不存
在了。从新石器时代到20世纪早期人类所感受到的那种战争的压力已经消失,取而代之
以完全不同的东西。倘若这三个大国不再交火,永远和平相处,且都不会侵犯对方边界,
那结果也没有什么不同。因为在这种情况下,每个国家仍然是自给自足的小世界,外来
的危险永远无法对它产生重大影响。这就是党的标语``战争即和平''的深层含义,绝大
多数党员对它的了解都非常肤浅。
\end{quotation}

\switchcolumn*

``I\textquotesingle ve got \emph{the book},'' he said as they disentangled
themselves.

\switchcolumn

温斯顿放下书,停了一会儿。远处的某个地方火箭弹爆炸了,发出雷一般的响声。独自一人待在没有电屏的房间里读禁书的快感还没有消退。他用身体体会着孤独与安全,这种感觉又和肉体的疲惫、座椅的柔软及从窗外吹进的微风轻抚人脸时的感觉混杂在一起。书深深地吸引了他,更准确地说,它让他心安。从某种意义上看,它吸引他的就在于它没有讲什么新东西,它说出了他想说的,如果他能将头脑里的断章残片整理好的话。它的作者有着和他类似的想法,但却比他更加有力、更加系统,也更加无畏。在他看来,最好的书就是将你已经知道的东西告知给你。他刚刚将书翻回第一章,就听到朱莉亚上楼梯的声音。他从椅子里站起来接她。她将棕色的工具包往地上一扔,扑进他的怀里,他们已经有一个多星期没见面了。

\switchcolumn*

``Oh, you\textquotesingle ve got it? Good,'' she said without much
interest, and almost immediately knelt down beside the oilstove to make
the coffee.

\switchcolumn

``我拿到那本书了。''他松开她,说。

\switchcolumn*

They did not return to the subject until they had been in bed for half
an hour. The evening was just cool enough to make it worth while to pull
up the counterpane. From below came the familiar sound of singing and
the scrape of boots on the flagstones. The brawny red-armed woman whom
Winston had seen there on his first visit was almost a fixture in the
yard. There seemed to be no hour of daylight when she was not marching
to and fro between the washtub and the line, alternately gagging herself
with clothes pegs and breaking forth into lusty song. Julia had settled
down on her side and seemed to be already on the point of falling
asleep. He reached out for the book, which was lying on the floor, and
sat up against the bedhead.

\switchcolumn

``噢,你拿到了?很好,''她并没有多大兴趣,一边说一边迅速地蹲了下去,在煤油炉旁煮起了咖啡。

\switchcolumn*

``We must read it,'' he said. ``You too. All members of the Brotherhood
have to read it.''

\switchcolumn

他们在床上躺了半个小时,才重新说起这个话题。夜里很凉,他们拉起床罩盖在身上。楼下传来熟悉的歌声和靴子行走在石板路上的声音。温斯顿第一次来这里时看到的那个红胳膊的健壮女人似乎已成为这院中风景的一部分。白天,没有一个小时她不是在洗衣盆和晾衣绳中走来走去,她要么叼着晾衣夹,要么唱着欢快的歌。朱莉亚侧着身子躺着,看上去快要睡着了。他伸手拿起放在地板上的书,靠着床头坐起了身。

\switchcolumn*

``You read it,'' she said with her eyes shut. ``Read it aloud.
That\textquotesingle s the best way. Then you can explain it to me as
you go.''

\switchcolumn

``我们必须读下它,''他说,``你也是。所有兄弟会的会员都要读。''

\switchcolumn*

The clock\textquotesingle s hands said six, meaning eighteen. They had
three or four hours ahead of them. He propped the book against his knees
and began reading:
\begin{center}
\emph{Chapter 1.}

\switchcolumn

``你读吧,''她闭着眼睛说,``大点声读。这样最好了,你可以边读边解释给我听。''

\switchcolumn*

IGNORANCE IS STRENGTH.
\end{center}

\switchcolumn

时针指到6点,也就是18点,他们还有三四个小时的时间。他把书放在膝盖上,读了起来。

\switchcolumn*

\begin{quotation}
Throughout recorded time, and probably since the end of the Neolithic
Age, there have been three kinds of people in the world, the High, the
Middle, and the Low. They have been subdivided in many ways, they have
borne countless different names, and their relative numbers, as well as
their attitude toward one another, have varied from age to age; but the
essential structure of society has never altered. Even after enormous
upheavals and seemingly irrevocable changes, the same pattern has always
reasserted itself, just as a gyroscope will always return to
equilibrium, however far it is pushed one way or the other.
\end{quotation}

\switchcolumn

第一章 无知即力量

\switchcolumn*

``Julia, are you awake?'' said Winston.

\switchcolumn

有史以来,大约从新石器时代结束开始,世界上就有三种人:上等人、中等人、下等人。按照不同的方式继续划分,他们有过很多名字。他们的相对人数以及对彼此的态度都因时而异,然而社会的基本结构却没有发生改变。即使在经历了重大剧变和看起来不可挽回的变化后,依然能恢复其原有的格局,就好像无论向哪个方向推进,陀螺仪都能恢复平衡。

\switchcolumn*

``Yes, my love, I\textquotesingle m listening. Go on.
It\textquotesingle s marvelous.''

\switchcolumn

``朱莉亚,你还醒着吗?''温斯顿问。

\switchcolumn*

He continued reading:

\switchcolumn

``是的,亲爱的,我听着呢,继续,写得真好。''

\switchcolumn*

\begin{quotation}
The aims of these three groups are entirely irreconcilable. The aim of
the High is to remain where they are. The aim of the Middle is to change
places with the High. The aim of the Low, when they have an aim---for it
is an abiding characteristic of the Low that they are too much crushed
by drudgery to be more than intermittently conscious of anything outside
their daily lives---is to abolish all distinctions and create a society
in which all men shall be equal. Thus throughout history a struggle
which is the same in its main outlines recurs over and over again. For
long periods the High seem to be securely in power, but sooner or later
there always comes a moment when they lose either their belief in
themselves, or their capacity to govern efficiently, or both. They are
then overthrown by the Middle, who enlist the Low on their side by
pretending to them that they are fighting for liberty and justice. As
soon as they have reached their objective, the Middle thrust the Low
back into their old position of servitude, and themselves become the
High. Presently a new Middle group splits off from one of the other
groups, or from both of them, and the struggle begins over again. Of the
three groups, only the Low are never even temporarily successful in
achieving their aims. It would be an exaggeration to say that throughout
history there has been no progress of a material kind. Even today, in a
period of decline, the average human being is physically better off than
he was a few centuries ago. But no advance in wealth, no softening of
manners, no reform or revolution has ever brought human equality a
millimeter nearer. From the point of view of the Low, no historic change
has ever meant much more than a change in the name of their masters.

\switchcolumn

他继续读了下去:

\switchcolumn*

By the late nineteenth century the recurrence of this pattern had become
obvious to many observers. There then arose schools of thinkers who
interpreted history as a cyclical process and claimed to show that
inequality was the unalterable law of human life. This doctrine, of
course, had always had its adherents, but in the manner in which it was
now put forward there was a significant change. In the past the need for
a hierarchical form of society had been the doctrine specifically of the
High. It had been preached by kings and aristocrats and by the priests,
lawyers, and the like who were parasitical upon them, and it had
generally been softened by promises of compensation in an imaginary
world beyond the grave. The Middle, so long as it was struggling for
power, had always made use of such terms as freedom, justice, and
fraternity. Now, however, the concept of human brotherhood began to be
assailed by people who were not yet in positions of command, but merely
hoped to be so before long. In the past the Middle had made revolutions
under the banner of equality, and then had established a fresh tyranny
as soon as the old one was overthrown. The new Middle groups in effect
proclaimed their tyranny beforehand. Socialism, a theory which appeared
in the early nineteenth century and was the last link in a chain of
thought stretching back to the slave rebellions of antiquity, was still
deeply infected by the Utopianism of past ages. But in each variant of
Socialism that appeared from about 1900 onwards the aim of establishing
liberty and equality was more and more openly abandoned. The new
movements which appeared in the middle years of the century, Ingsoc in
Oceania, Neo-Bolshevism in Eurasia, Death-worship, as it is commonly
called, in Eastasia, had the conscious aim of perpetuating unfreedom and
inequality. These new movements, of course, grew out of the old ones and
tended to keep their names and pay lip-service to their ideology. But
the purpose of all of them was to arrest progress and freeze history at
a chosen moment. The familiar pendulum swing was to happen once more,
and then stop. As usual, the High were to be turned out by the Middle,
who would then become the High; but this time, by conscious strategy,
the High would be able to maintain their position permanently.

\switchcolumn

这三个阶层的目标完全无法协调。上等人的目标是保住自己的地位。中等人的目标是和上等人调换地位。至于下等人,他们被欺压得太厉害,生活太艰苦,以至于偶尔才能想起日常生活以外的事,这已成为他们的一大特点,若他们真有目标,那无外乎是消除一切差别,建立一个人人平等的社会。因此,纵观历史,类似的斗争总是一而再地上演,很长时间,上等人都看似坚固地把持着权力,但总有一天,他们要么不再相信自己,要么不再相信自己还能够进行统治,又或者二者皆有。之后,他们会被中等人推翻,中等人装出为自由和正义而战的样子,将下等人争取到自己的一边。但只要中等人达成目的,他们就会把下等人重新推回被奴役的地位,而他们自己则摇身变成上等人。不久,新的中等阶层又会从某一等人或某两等人中分化出来,这场斗争又再次开始。三等人中,只有下等人从未成功实现,哪怕是暂时实现过自己的目标。如果说历史上人们从来没有取得过实质性的进步,不免有些夸张。就算在今天,在这个衰败的时代,人们平均的物质生活水平也比几个世纪前要好。但无论是财富的增长,还是行为的文明程度,抑或是改革与革命,都没能让人类向平等前进一星半点。站在下等人的视角上,改朝换代无非是主宰者的名字发生变化而已。

\switchcolumn*

The new doctrines arose partly because of the accumulation of historical
knowledge, and the growth of the historical sense, which had hardly
existed before the nineteenth century. The cyclical movement of history
was now intelligible, or appeared to be so; and if it was intelligible,
then it was alterable. But the principal, underlying cause was that, as
early as the beginning of the twentieth century, human equality had
become technically possible. It was still true that men were not equal
in their native talents and that functions had to be specialized in ways
that favored some individuals against others; but there was no longer
any real need for class distinctions or for large differences of wealth.
In earlier ages, class distinctions had been not only inevitable but
desirable. Inequality was the price of civilization. With the
development of machine production, however, the case was altered. Even
if it was still necessary for human beings to do different kinds of
work, it was no longer necessary for them to live at different social or
economic levels. Therefore, from the point of view of the new groups who
were on the point of seizing power, human equality was no longer an
ideal to be striven after, but a danger to be averted. In more primitive
ages, when a just and peaceful society was in fact not possible, it had
been fairly easy to believe in it. The idea of an earthly paradise in
which men should live together in a state of brotherhood, without laws
and without brute labor, had haunted the human imagination for thousands
of years. And this vision had had a certain hold even on the groups who
actually profited by each historic change. The heirs of the French,
English, and American revolutions had partly believed in their own
phrases about the rights of man, freedom of speech, equality before the
law, and the like, and had even allowed their conduct to be influenced
by them to some extent. But by the fourth decade of the twentieth
century all the main currents of political thought were authoritarian.
The earthly paradise had been discredited at exactly the moment when it
became realizable. Every new political theory, by whatever name it
called itself, led back to hierarchy and regimentation. And in the
general hardening of outlook that set in round about 1930, practices
which had been long abandoned, in some cases for hundreds of
years---imprisonment without trial, the use of war prisoners as slaves,
public executions, torture to extract confessions, the use of hostages
and the deportation of whole populations---not only became common again,
but were tolerated and even defended by people who considered themselves
enlightened and progressive.

\switchcolumn

19世纪晚期,很多观察家都注意到这种明显的反复情况。一些思想学派也由此产生,他们将历史诠释成一个循环发展的过程,并声称不平等是人类生活的永恒法则。这一理论一直都不乏拥护者,不过现在它的提出方式有了重大改变。在过去,社会需要分成三六九等的说法特别为上等人所强调,国王、特权者、教士、律师,以及其他寄生者们都在宣扬这一学说,通常,他们向人们承诺,人们可以在死后的世界里得到补偿,并以此来缓和等级间的矛盾。而中等人只要还在为权力进行斗争,就一定会鼓吹自由、正义、博爱,但是现在,人类皆兄弟的观点却遭到了暂未掌权、不久之后可能掌权的人的攻击。过去,中等人打着平等的旗号进行革命,在推翻专制后,建立新的专制。现在,新的中等人会事先声明他们就是要建立专制。19世纪初期出现了社会主义理论,该理论是可以追溯到古奴隶反叛时代的思想链条中的最后一环,深受历代乌托邦思想的影响。不过,在1900年后出现的若干社会主义理论都越来越公开地摒弃关于建立自由与平等的社会的目标。本世纪中期出现了新的社会运动,在大洋国它是英社,在欧亚国它是新布尔什维主义,在东亚国则是崇死,它们都有明确的目标让不自由、不平等永远持续下去。它们都由旧运动发展而来,都趋向于保持原有的名字并对旧有的意识形态作口头宣传,都把阻挠进步、让历史在某一时刻冻结起来当作目的。众所周知的钟摆现象会再次发生,随后停止。像之前一样,上等人被中等人推翻,中等人成为上等人。只是这次由于采取了明智的策略,上等人将永远保持自己的地位。之所以出现新的学说,一部分因为历史知识的积累和历史意识的产生,这在19世纪之前几乎没有。今天,对历史的循环运动性,人们已有所了解,或者表面上是这样。而既然了解它,就可以改变它。但最主要最根本的原因却是早在20世纪初,人类的平等在技术上便成为可能。虽然人的天赋有高低,职责各不同,有些人就是要比其他人都强是事实,但已经没有让阶级存在差别、制造财富悬殊的必要。在早些时候,阶级的差别不仅不可避免,还是理所应当的。不平等是文明的代价。而随着机器大生产的发展,情况发生了变化。即便人们仍要从事不同的工作,他们也没有必要拥有不同的社会地位,没有必要生活在不同的经济水平下。因此,在那些即将夺取权力的人看来,他们用不着再为人类的平等进行奋斗,而是要避开危险。在更加久远的时代,当建立公正和平的社会行不通时,人们反倒更相信这种社会的存在。几千年来,人们一直梦想拥有一个人人亲如兄弟、没有法律也没有牲畜般艰苦劳动的人间天堂,一些人即使在每一次历史变革中都得到切实的好处,也仍抱有这种梦想。法、英、美等国大革命的继承者在一定程度上对自己所说的人权、言论自由、法律面前人人平等信以为真,甚至其行为业已在某种程度上受到影响。但是到了20世纪40年代,所有主流政治思想就都是独裁主义的了。人间天堂在即将实现的时候遭到了人们的怀疑。所有新的政治理论,不管自称什么,都发生后退,主张等级化和组织化。大约在1930年的时候,在形势普遍严峻的情况下,一些长期弃用甚至几百年都没用过的做法,比如未经审讯的关押、把战俘当奴隶用、公开处决、严刑逼供、扣押人质、放逐大批人口等,不仅又成为寻常的做法,还得到了那些自认开明、进步的人的容忍、辩护。

\switchcolumn*

It was only after a decade of national wars, civil wars, revolutions and
counterrevolutions in all parts of the world that Ingsoc and its rivals
emerged as fully worked-out political theories. But they had been
foreshadowed by the various systems, generally called totalitarian,
which had appeared earlier in the century, and the main outlines of the
world which would emerge from the prevailing chaos had long been
obvious. What kind of people would control this world had been equally
obvious. The new aristocracy was made up for the most part of
bureaucrats, scientists, technicians, trade-union organizers, publicity
experts, sociologists, teachers, journalists, and professional
politicians. These people, whose origins lay in the salaried middle
class and the upper grades of the working class, had been shaped and
brought together by the barren world of monopoly industry and
centralized government. As compared with their opposite numbers in past
ages, they were less avaricious, less tempted by luxury, hungrier for
pure power, and, above all, more conscious of what they were doing and
more intent on crushing opposition. This last difference was cardinal.
By comparison with that existing today, all the tyrannies of the past
were half-hearted and inefficient. The ruling groups were always
infected to some extent by liberal ideas, and were content to leave
loose ends everywhere, to regard only the overt act, and to be
uninterested in what their subjects were thinking. Even the Catholic
Church of the Middle Ages was tolerant by modern standards. Part of the
reason for this was that in the past no government had the power to keep
its citizens under constant surveillance. The invention of print,
however, made it easier to manipulate public opinion, and the film and
the radio carried the process further. With the development of
television, and the technical advance which made it possible to receive
and transmit simultaneously on the same instrument, private life came to
an end. Every citizen, or at least every citizen important enough to be
worth watching, could be kept for twenty-four hours a day under the eyes
of the police and in the sound of official propaganda, with all other
channels of communication closed. The possibility of enforcing not only
complete obedience to the will of the State, but complete uniformity of
opinion on all subjects, now existed for the first time.

\switchcolumn

英社及与它并立的政治理论得到贯彻执行还是在世界经历了长达十年的国际战争、内战、革命与反革命之后。不过,此前已有一些体制预见了它们的出现,这些体制通常被称作集权主义,出现在本世纪早些时候。世界的主要轮廓将在动乱之中浮现出来,是显而易见的,哪种人将掌控这个世界同样显而易见。新的特权阶层大部分由官僚、科学家、技师、工会组织者、宣传专家、社会学家、教师、记者、职业政客组成。他们出身于领工资的中产阶级和上层工人阶级。他们由垄断工业和中央集权政府下的贫瘠世界塑造、聚集。相较于过去,他们没那么贪婪,也没那么奢侈,但他们却更渴望拥有绝对的权力。更重要的是,他们对自己的所作所为有着更清醒的认识,在镇压反对者上更加坚决。后者是极其重要的,和今天相比,曾经的僭主政治既不彻底,又缺乏效率。其统治集团总不免受到自由主义思想的影响,心甘情愿地让该思想在所有地方都留下影迹,他们只关注那些明目张胆的行为,对国民的想法则不感兴趣。在今天看来,就连中世纪的教会都算得上宽容。这一部分因为,在过去没有哪个政府有能力将自己的公民置于不断地监控之下。但随着印刷术的发明,操纵公众意见变得更加容易,之后电影和无线电的发明又加剧了这一情形。电视业的发展以及用同一设备进行信息手法的技术的进步,终结了人们的私生活。所有公民,至少那些重要的、需要被关注的公民,可能一天24小时全部处在警察的监视之下和官方的宣传之中。其他的信息渠道通通被切断。至此,第一次,不仅所有公民都要完全屈服于国家意志,对任何问题,他们的看法还都将被统一起来。

\switchcolumn*

After the revolutionary period of the Fifties and Sixties, society
regrouped itself, as always, into High, Middle, and Low. But the new
High group, unlike all its forerunners, did not act upon instinct but
knew what was needed to safeguard its position. It had long been
realized that the only secure basis for oligarchy is collectivism.
Wealth and privilege are most easily defended when they are possessed
jointly. The so-called ``abolition of private property'' which took place
in the middle years of the century meant, in effect, the concentration
of property in far fewer hands than before; but with this difference,
that the new owners were a group instead of a mass of individuals.
Individually, no member of the Party owns anything, except petty
personal belongings. Collectively, the Party owns everything in Oceania,
because it controls everything and disposes of the products as it thinks
fit. In the years following the Revolution it was able to step into this
commanding position almost unopposed, because the whole process was
represented as an act of collectivization. It had always been assumed
that if the capitalist class were expropriated, Socialism must follow;
and unquestionably the capitalists had been expropriated. Factories,
mines, land, houses, transport---everything had been taken away from
them; and since these things were no longer private property, it
followed that they must be public property. Ingsoc, which grew out of
the earlier Socialist movement and inherited its phraseology, has in
fact carried out the main item in the Socialist program, with the
result, foreseen and intended beforehand, that economic inequality has
been made permanent.

\switchcolumn

在五六十年代的革命之后,像往常一样,社会重新划分成上等人、中等人、下等人。但新的上等人和之前的并不相同,他们并非凭直觉行事,他们知道要确保地位的稳固需要做些什么。他们一早就意识到,对寡头政治而言最牢靠的基础便是集体主义。财富和特权若为集体所有,保护它们就最为容易。本世纪中期出现的所谓的``消灭私有制''运动,实际上意味着将财产集中到更少的人手中。不同的在于,新的财产所有者不是若干个人,而是一个集团。就个人而言,除了少得可怜的个人物品,党员什么财产都没有。但对集体来说,在大洋国,所有的一切都属于党,因为所有的一切都在它控制之下,它会按照它所认为合适的方式处理产品。革命后的几年里,党之所以能够未经任何反抗就占据统治地位,就是因为这一过程是以集体化的形式表现出来的。人们总会这样想,如果剥夺了资产阶级的所有权,社会主义必紧随而至。资产阶级的财产的确被剥夺了,工厂、矿山、土地、房屋、运输工具——他们拥有的一切都被夺走了。这些东西不再是私有财产,它们必然成为公有财产。英社诞生于早期社会主义运动,继承了后者的用词,也在事实上完成了社会主义纲领的主要部分,得到了将经济不平等永久化的结果。而这既是可以预见的,也是它有意为之。

\switchcolumn*

But the problems of perpetuating a hierarchical society go deeper than
this. There are only four ways in which a ruling group can fall from
power. Either it is conquered from without, or it governs so
inefficiently that the masses are stirred to revolt, or it allows a
strong and discontented Middle Group to come into being, or it loses its
own self-confidence and willingness to govern. These causes do not
operate singly, and as a rule all four of them are present in some
degree. A ruling class which could guard against all of them would
remain in power permanently. Ultimately the determining factor is the
mental attitude of the ruling class itself.

\switchcolumn

但是等级社会的永久化却是更加深刻的问题。只有在四种情况下统治集团才会丧失权力:被外部势力征服,管理效率低下造成大众起义,一个强大的、心怀不满的中等阶层出现,它自己失去了统治信心和统治的意愿。四者都并非独自发生作用,某种程度上说,它们总是同时存在已成为一种规律。统治集团若能防止它们出现,就能长久地统治下去。最终起决定性作用的还是统治集团本身的精神状态。

\switchcolumn*

After the middle of the present century, the first danger had in reality
disappeared. Each of the three powers which now divide the world is in
fact unconquerable, and could only become conquerable through slow
demographic changes which a government with wide powers can easily
avert. The second danger, also, is only a theoretical one. The masses
never revolt of their own accord, and they never revolt merely because
they are oppressed. Indeed, so long as they are not permitted to have
standards of comparison, they never even become aware that they are
oppressed. The recurrent economic crises of past times were totally
unnecessary and are not now permitted to happen, but other and equally
large dislocations can and do happen without having political results,
because there is no way in which discontent can become articulate. As
for the problem of overproduction, which has been latent in our society
since the development of machine technique, it is solved by the device
of continuous warfare (\emph{see} Chapter 3), which is also useful in
keying up public morale to the necessary pitch. From the point of view
of our present rulers, therefore, the only genuine dangers are the
splitting-off of a new group of able, underemployed, power-hungry
people, and the growth of liberalism and skepticism in their own ranks.
The problem, that is to say, is educational. It is a problem of
continuously molding the consciousness both of the directing group and
of the larger executive group that lies immediately below it. The
consciousness of the masses needs only to be influenced in a negative
way.

\switchcolumn

本世纪中期以后,第一种危险已在现实生活中绝迹,三个超级大国瓜分了世界,除非通过人口的缓慢变化,其每一个都不可征服,但只要其政府拥有广泛的权力,这点就很容易避免。第二种危险仅仅存在于理论中,人们从来不会自发地起来造反,从来不会只因为受压迫就揭竿而起。确实,只要没有比较标准,他们就意识不到他们正遭受压迫。之前频频发生的经济危机完全没有必要,而且也不许在现在发生。至于其他大范围的混乱就算确有可能出现,也不会产生政治后果;因为没有能让不满的声音清晰地表达出来的途径。说起生产过剩的问题,自机械技术发展以来就一直潜伏在我们的社会中,可以依靠不断地进行的战争来解决(参见第三章),战争有利于将公众士气激励到必要的高度。按照我们当前统治者的观点,只有一个真正的危险,那就是从他们自己的阶层分化出一个有能力、有权欲、又没能充分发挥才干的新的集团,这会让他们之中出现自由主义和怀疑主义。也就是说,问题在于教育。要不断地加强领导集团以及其下关系紧密又人数众多的执行集团的觉悟,但对大众的觉悟就只需通过消极的方式加以影响。

\switchcolumn*

Given this background, one could infer, if one did not know it already,
the general structure of Oceanic society. At the apex of the pyramid
comes Big Brother. Big Brother is infallible and all-powerful. Every
success, every achievement, every victory, every scientific discovery,
all knowledge, all wisdom, all happiness, all virtue, are held to issue
directly from his leadership and inspiration. Nobody has ever seen Big
Brother. He is a face on the hoardings, a voice on the telescreen. We
may be reasonably sure that he will never die, and there is already
considerable uncertainty as to when he was born. Big Brother is the
guise in which the Party chooses to exhibit itself to the world. His
function is to act as a focusing point for love, fear, and reverence,
emotions which are more easily felt toward an individual than toward an
organization. Below Big Brother comes the Inner Party, its numbers
limited to six millions, or something less than two per cent of the
population of Oceania. Below the Inner Party comes the Outer Party,
which, if the Inner Party is described as the brain of the State, may be
justly likened to the hands. Below that come the dumb masses whom we
habitually refer to as ``the proles,'' numbering perhaps eighty-five per
cent of the population. In the terms of our earlier classification, the
proles are the Low, for the slave populations of the equatorial lands,
who pass constantly from conqueror to conqueror, are not a permanent or
necessary part of the structure.

\switchcolumn

即便一个人之前不清楚大洋国的主要结构,在了解了这一背景后,也可以将其推断出来。金字塔的顶端是老大哥。老大哥绝对正确,无所不能。所有成就,所有成功,所有胜利,所有科学发明,所有知识,所有智慧,所有幸福,所有美德都直接来自于他的领导和激励。没人见过老大哥。他是宣传栏上的一张脸,是电屏上的一个声音。我们确信他永远不会死,而他出生的时间则没人能确定。老大哥是党挑选出来向世界展示自我的一个形象,他的作用是充当热爱、恐惧、崇敬等情感的焦点。老大哥之下是内党,其人数限制在六百万之内,在大洋国人口中不到2\%。内党之下便是外党,如果把内党形容成国家的大脑,外党就像国家的手。外党下面是愚笨的大众,我们习惯上将他们称作``群众'',他们大概占了总人口的85\%。按照之前的社会分类方法划分,群众就是下等人,由于赤道一带被奴役的人总是从一个征服者手上转到另一个征服者手上,所以不能算作社会的固定组成或必要部分。

\switchcolumn*

In principle, membership in these three groups is not hereditary. The
child of Inner Party parents is in theory not born into the Inner Party.
Admission to either branch of the Party is by examination, taken at the
age of sixteen. Nor is there any racial discrimination, or any marked
domination of one province by another. Jews, Negroes, South Americans of
pure Indian blood are to be found in the highest ranks of the Party, and
the administrators of any area are always drawn from the inhabitants of
that area. In no part of Oceania do the inhabitants have the feeling
that they are a colonial population ruled from a distant capital.
Oceania has no capital, and its titular head is a person whose
whereabouts nobody knows. Except that English is its chief lingua franca
and Newspeak its official language, it is not centralized in any way.
Its rulers are not held together by blood ties but by adherence to a
common doctrine. It is true that our society is stratified, and very
rigidly stratified, on what at first sight appear to be hereditary
lines. There is far less to-and-fro movement between the different
groups than happened under capitalism or even in the pre-industrial
ages. Between the two branches of the Party there is a certain amount of
interchange, but only so much as will ensure that weaklings are excluded
from the Inner Party and that ambitious members of the Outer Party are
made harmless by allowing them to rise. Proletarians, in practice, are
not allowed to graduate into the Party. The most gifted among them, who
might possibly become nuclei of discontent, are simply marked down by
the Thought Police and eliminated. But this state of affairs is not
necessarily permanent, nor is it a matter of principle. The Party is not
a class in the old sense of the word. It does not aim at transmitting
power to its own children, as such; and if there were no other way of
keeping the ablest people at the top, it would be perfectly prepared to
recruit an entire new generation from the ranks of the proletariat. In
the crucial years, the fact that the Party was not a hereditary body did
a great deal to neutralize opposition. The older kind of Socialist, who
had been trained to fight against something called ``class privilege,''
assumed that what is not hereditary cannot be permanent. He did not see
that the continuity of an oligarchy need not be physical, nor did he
pause to reflect that hereditary aristocracies have always been
short-lived, whereas adoptive organizations such as the Catholic Church
have sometimes lasted for hundreds or thousands of years. The essence of
oligarchical rule is not father-to-son inheritance, but the persistence
of a certain world-view and a certain way of life, imposed by the dead
upon the living. A ruling group is a ruling group so long as it can
nominate its successors. The Party is not concerned with perpetuating
its blood but with perpetuating itself. \emph{Who} wields power is not
important, provided that the hierarchical structure remains always the
same.

\switchcolumn

原则上说,这三等人的身份并不是世袭的。父母都是内党党员的孩子在理论上并非一出生就是内党党员。不管是加入内党还是外党,都需要在16岁时参加考试,不存在任何种族歧视,也不存在某个地区压制另一个地区的情况。人们可以在党的最高层中找到犹太人、黑人、纯印第安人血统的南美人,且每个地区的行政官员总是从当地居民中选拔出来。在大洋国没有哪个地方的居民觉得自己被殖民,也没有哪个地方的居民觉得自己被远方的首都所管理。大洋国没有首都,其元首只徒有其名,没人知道他在哪里。除了英语是通用语,新话是官方语外,在语言方面不存在任何形式的统一。这些统治者不是靠血缘联系在一起,而是靠共同的信仰。没错,我们的社会是分等级的,而且是严格的分级,一眼看过去好像是按世袭划分。不同阶层间的流动比资本主义或前工业时代的都要小得多。党的两个分支间存在一定数量的人员流动,但这只是为了剔除掉内党中的低能分子,并给外党里野心家以向上爬的机会,免得他们造成危害。无产者实际上是不允许入党的,他们中最有天分的很有可能成为不满情绪的核心,会引起思想警察的注意并遭到清除。但这种情况并非一成不变,也不是主要的原则问题。党在定义上已不是一个阶级,它不再以``将权力传给子女''作为目标,若没有其他办法让最能干的人留在党的最高层,它很乐意从无产者中征募一代新人。在关键时期,在消除反对力量上,党的这种非世袭性起到了很大作用。老一代的社会主义者一直接受反对阶级特权的训练,都以为非世袭的东西不可能永久存在。他们没看出来寡头政治的延续不一定需要什么实际的依托,也没有想到世袭贵族一向短命,倒是像天主教那样的组织方式有时竟可维持几百或几千年。寡头政治的关键不是父子相传,而是让一种由死者加诸于生者的特定的世界观、生活方式延续下去。只要能指派后继者,统治集团就永远是统治集团。党关心的不是血统的不朽,而是党本身的不朽。只要等级结构永恒不变,谁掌权并不重要。

\switchcolumn*

All the beliefs, habits, tastes, emotions, mental attitudes that
characterize our time are really designed to sustain the mystique of the
Party and prevent the true nature of present-day society from being
perceived. Physical rebellion, or any preliminary move toward rebellion,
is at present not possible. From the proletarians nothing is to be
feared. Left to themselves, they will continue from generation to
generation and from century to century, working, breeding, and dying,
not only without any impulse to rebel, but without the power of grasping
that the world could be other than it is. They could only become
dangerous if the advance of industrial technique made it necessary to
educate them more highly; but, since military and commercial rivalry are
no longer important, the level of popular education is actually
declining. What opinions the masses hold, or do not hold, is looked on
as a matter of indifference. They can be granted intellectual liberty
because they have no intellect. In a Party member, on the other hand,
not even the smallest deviation of opinion on the most unimportant
subject can be tolerated.

\switchcolumn

我们时代的所有信念、习惯、兴趣、情感、思想状况都是设计好的,都是为了保持党的神秘性,避免什么人看穿当前社会的真相。造反,或者任何和造反有关的准备工作,在目前都不可能发生。不用担心无产阶级,随他们去,他们会一代又一代、一个世纪又一个世纪地劳作、繁衍、死亡,不仅不会有揭竿而起的冲动,也无法理解世界还可以有其他的样子。只有当工业技术发展到需要为他们提供更高级的教育,他们才会成为危险分子。然而,由于军事竞争和商业竞争都已无足轻重,公众的教育水平实际上下降了。大众有见解,大众没见解,都没什么不同。他们之所以可以拥有思考的自由是因为他们根本没有思考能力。而对党员来说,即便在最不重要的事情上的最轻微的思想出轨,都不被容忍。

\switchcolumn*

A Party member lives from birth to death under the eye of the Thought
Police. Even when he is alone he can never be sure that he is alone.
Wherever he may be, asleep or awake, working or resting, in his bath or
in bed, he can be inspected without warning and without knowing that he
is being inspected. Nothing that he does is indifferent. His
friendships, his relaxations, his behavior toward his wife and children,
the expression of his face when he is alone, the words he mutters in
sleep, even the characteristic movements of his body, are all jealously
scrutinized. Not only any actual misdemeanor, but any eccentricity,
however small, any change of habits, any nervous mannerism that could
possibly be the symptom of an inner struggle, is certain to be detected.
He has no freedom of choice in any direction whatever. On the other
hand, his actions are not regulated by law or by any clearly formulated
code of behavior. In Oceania there is no law. Thoughts and actions
which, when detected, mean certain death are not formally forbidden, and
the endless purges, arrests, tortures, imprisonments, and vaporizations
are not inflicted as punishment for crimes which have actually been
committed, but are merely the wiping-out of persons who might perhaps
commit a crime at some time in the future. A Party member is required to
have not only the right opinions, but the right instincts. Many of the
beliefs and attitudes demanded of him are never plainly stated, and
could not be stated without laying bare the contradictions inherent in
Ingsoc. If he is a person naturally orthodox (in Newspeak, a
\emph{goodthinker}), he will in all circumstances know, without taking
thought, what is the true belief or the desirable emotion. But in any
case an elaborate mental training, undergone in childhood and grouping
itself round the Newspeak words \emph{crimestop}, \emph{blackwhite}, and
\emph{doublethink}, makes him unwilling and unable to think too deeply
on any subject whatever.

\switchcolumn

党员从生到死,都生活在思想警察的监视之中。就算独自一人,也永远不敢确定自己是否真是独自一人。无论在哪里,是睡着还是醒着,是工作还是休息,是泡在浴缸里还是躺在床上,他都有可能被人监视。而这监视不会有通知,也不会被察觉。他的所作所为都至关重要。他的友谊、他的娱乐、他对妻儿的态度、他独处时的表情、他沉睡时的梦呓、他特有的肢体动作都将受到谨慎而详细的检查。不要说确实有不检的行为,任何细小的乖张之举,任何习惯上的变化,任何神经质的怪癖——任何有可能反映内心争斗的征兆,都一定会被发现,对任何事情他都没有选择的自由。另一方面,他的行为并不受什么法律或什么明文规定的约束。大洋国没有法律。一些思想、行为,尽管没有被正式禁止,一旦被查到就意味着死,无休无止的清洗、逮捕、拷打、监禁、蒸发都并非是对所犯之罪进行的惩罚,而仅仅为了将那些有可能在未来犯罪的人清除。对一个党员来说,光有正确的思想还不够,他还要有正确的本能。至于他需要具备什么样的信念、态度则从未被清楚说明,因为要想在不暴露英社内在矛盾的情况下说明这些根本不可能。若他天生正统(用新话说就是``思想好的人''),那他无须思考就知道什么是正确的信念,什么是应有的情感,且在任何情况下都是如此。无论如何,当他还是孩子时就接受了复杂的充斥着新话词语的思想训练,如停止犯罪、黑白、双重思想。这让他不愿意也没能力对问题有深刻的思考。

\switchcolumn*

A Party member is expected to have no private emotions and no respites
from enthusiasm. He is supposed to live in a continuous frenzy of hatred
of foreign enemies and internal traitors, triumph over victories, and
self-abasement before the power and wisdom of the Party. The discontents
produced by his bare, unsatisfying life are deliberately turned outwards
and dissipated by such devices as the Two Minutes Hate, and the
speculations which might possibly induce a skeptical or rebellious
attitude are killed in advance by his early acquired inner discipline.
The first and simplest stage in the discipline, which can be taught even
to young children, is called, in Newspeak, \emph{crimestop}.
\emph{Crimestop} means the faculty of stopping short, as though by
instinct, at the threshold of any dangerous thought. It includes the
power of not grasping analogies, of failing to perceive logical errors,
of misunderstanding the simplest arguments if they are inimical to
Ingsoc, and of being bored or repelled by any train of thought which is
capable of leading in a heretical direction. \emph{Crimestop}, in short,
means protective stupidity. But stupidity is not enough. On the
contrary, orthodoxy in the full sense demands a control over
one\textquotesingle s own mental processes as complete as that of a
contortionist over his body. Oceanic society rests ultimately on the
belief that Big Brother is omnipotent and that the Party is infallible.
But since in reality Big Brother is not omnipotent and the Party is not
infallible, there is need for an unwearying, moment-to-moment
flexibility in the treatment of facts. The key word here is
\emph{blackwhite}. Like so many Newspeak words, this word has two
mutually contradictory meanings. Applied to an opponent, it means the
habit of impudently claiming that black is white, in contradiction of
the plain facts. Applied to a Party member, it means a loyal willingness
to say that black is white when Party discipline demands this. But it
means also the ability to \emph{believe} that black is white, and more,
to \emph{know} that black is white, and to forget that one has ever
believed the contrary. This demands a continuous alteration of the past,
made possible by the system of thought which really embraces all the
rest, and which is known in Newspeak as \emph{doublethink}.

\switchcolumn

作为一个党员不应该有任何私人情感,其对党的热情不应有丝毫松懈。他应该生活在对内外敌人的疯狂的仇恨中,生活在对胜利的欢欣鼓舞中,并拜倒在党的英明和力量之下。他对贫乏又不尽如人意的生活的不满,被小心引导,宣泄出来,在两分钟仇恨会上消散得无影无踪。而那些有可能促发怀疑或反抗情绪的思想,则会被他早年接受的内心训练扼杀。用新话来说,这种训练的最初也是最简单的阶段便是``停止犯罪'',它被教授给幼小的孩子。所谓停止犯罪,即是指在危险思想即将萌生的时候,如本能一般,迅速地停止思考。它包括以下一些内容:无法进行类比、看不到逻辑的谬误、不能理解最简单的抨击英社的理论,以及对任何可能发展成异端的思想感到厌倦。概括地说,停止犯罪就意味着把愚蠢当成保护措施。但只有愚蠢还不够,相反,正统要求人像柔术师控制自己的身体那样控制自己的思路。大洋国社会的终极信仰是:老大哥无所不能,党永远正确。但因为在现实生活中,老大哥并非无所不能,党也并非永远正确,对待事实,人们就需要时刻保持灵活性,且不能有丝毫懈怠。对此,有一个关键词``黑白'',像很多新话词语一样,它包含两个相互矛盾的意思。用在敌人身上,就意味着肆无忌惮地、罔顾事实地说黑为白。用在党员身上,就意味着根据党的纪律要求,出于忠诚说黑是白,但它还意味着相信黑即是白的能力,还包括知道黑即是白并忘记自己曾经相信过相反的东西的能力。如此,无休无止地篡改过去就成了一种需要,而篡改过去只有通过一种的确能轻而易举包容一切的思想体系才能做到。用新话来说,便是双重思想。

\switchcolumn*

The alteration of the past is necessary for two reasons, one of which is
subsidiary and, so to speak, precautionary. The subsidiary reason is
that the Party member, like the proletarian, tolerates present-day
conditions partly because he has no standards of comparison. He must be
cut off from the past, just as he must be cut off from foreign
countries, because it is necessary for him to believe that he is better
off than his ancestors and that the average level of material comfort is
constantly rising. But by far the more important reason for the
readjustment of the past is the need to safeguard the infallibility of
the Party. It is not merely that speeches, statistics, and records of
every kind must be constantly brought up to date in order to show that
the predictions of the Party were in all cases right. It is also that no
change in doctrine or in political alignment can ever be admitted. For
to change one\textquotesingle s mind, or even one\textquotesingle s
policy, is a confession of weakness. If, for example, Eurasia or
Eastasia (whichever it may be) is the enemy today, then that country
must always have been the enemy. And if the facts say otherwise, then
the facts must be altered. Thus history is continuously rewritten. This
day-to-day falsification of the past, carried out by the Ministry of
Truth, is as necessary to the stability of the regime as the work of
repression and espionage carried out by the Ministry of Love.

\switchcolumn

篡改过去之所以必要有两个原因,一个是次要的,也可以说是预防性的,那就是党员之所以能像群众一样忍受当下的生活条件,部分由于他没有比较的标准。为了让他相信他比他的祖先生活得更好,为了让他相信平均的物质水平有在提高,就必须让他同过去断绝开来,就像将他和外国断绝开一样。而另一个原因则重要得多,即确保党永远正确。为了让党的预言在任何情况下都准确无误,要不断地修改过去的讲话、统计资料、各种记录,同时不能承认党的教义或大洋国的政治结盟情况发生过什么变化,因为承认这些就相当于承认自己有错。比如,今天的敌人是欧亚国或东亚国(不管是哪一国),那它就必须永远是敌人。如果事实与之矛盾,就必须篡改事实,历史因此不断被重写。由真理部负责的篡改工作每天都在进行,一如仁爱部要从事侦察和镇压的工作,这是维护政权稳定的必需。

\switchcolumn*

The mutability of the past is the central tenet of Ingsoc. Past events,
it is argued, have no objective existence, but survive only in written
records and in human memories. The past is whatever the records and the
memories agree upon. And since the Party is in full control of all
records, and in equally full control of the minds of its members, it
follows that the past is whatever the Party chooses to make it. It also
follows that though the past is alterable, it never has been altered in
any specific instance. For when it has been recreated in whatever shape
is needed at the moment, then this new version \emph{is} the past, and
no different past can ever have existed. This holds good even when, as
often happens, the same event has to be altered out of recognition
several times in the course of a year. At all times the Party is in
possession of absolute truth, and clearly the absolute can never have
been different from what it is now. It will be seen that the control of
the past depends above all on the training of memory. To make sure that
all written records agree with the orthodoxy of the moment is merely a
mechanical act. But it is also necessary to \emph{remember} that events
happened in the desired manner. And if it is necessary to rearrange
one\textquotesingle s memories or to tamper with written records, then
it is necessary to \emph{forget} that one has done so. The trick of
doing this can be learned like any other mental technique. It \emph{is}
learned by the majority of Party members, and certainly by all who are
intelligent as well as orthodox. In Oldspeak it is called, quite
frankly, ``reality control.'' In Newspeak it is called \emph{doublethink},
though \emph{doublethink} comprises much else as well.

\switchcolumn

过去变化无常是英社的中心原则。英社认为过去并不是客观的存在,它只存在于文字记录和人类的记忆里。只要记录和记忆一致,不管是什么,都是过去。党既有能力全面掌控所有记录,也有能力全面掌控党员的思想,党想让过去是什么样那它就是什么样。不过,与此同时,虽然过去可以被篡改,但就具体事件而言,过去从未被篡改,任何事件都是如此。因为,无论当时出于什么目的将过去改头换面,改后的新样子即是过去,不能存在与这个过去不同的过去。这种情况经常发生,当同样一件事在一年之内被篡改好几次且面目全非时,依然如此。党无时无刻不掌握着绝对真理,显然,既然是绝对真理就不可能和现在的情况有什么出入。由此可见,控制过去首先要仰仗于对记忆的训练,而确保所有文字记录都和当下的正统思想相吻合不过是一种机械式的行为,不仅如此,还需要记住事情是按照人的意愿发生的。如果重新安排记忆或篡改文字记录是必需的,那么忘记自己曾做过这样的事也是必需的。人们可以像学会其他思考方法一样学会这种思考方法,大部分党员都学会了,更不要说那些又聪明又正统的人。在老话中,它被直白地称作``现实控制'',在新话里,它被称为``双重思想'',不过``双重思想''还包括其他一些东西。

\switchcolumn*

\emph{Doublethink} means the power of holding two contradictory beliefs
in one\textquotesingle s mind simultaneously, and accepting both of
them. The Party intellectual knows in which direction his memories must
be altered; he therefore knows that he is playing tricks with reality;
but by the exercise of \emph{doublethink} he also satisfies himself that
reality is not violated. The process has to be conscious, or it would
not be carried out with sufficient precision, but it also has to be
unconscious, or it would bring with it a feeling of falsity and hence of
guilt. \emph{Doublethink} lies at the very heart of Ingsoc, since the
essential act of the Party is to use conscious deception while retaining
the firmness of purpose that goes with complete honesty. To tell
deliberate lies while genuinely believing in them, to forget any fact
that has become inconvenient, and then, when it becomes necessary again,
to draw it back from oblivion for just so long as it is needed, to deny
the existence of objective reality and all the while to take account of
the reality which one denies---all this is indispensably necessary. Even
in using the word \emph{doublethink} it is necessary to exercise
\emph{doublethink}. For by using the word one admits that one is
tampering with reality; by a fresh act of \emph{doublethink} one erases
this knowledge; and so on indefinitely, with the lie always one leap
ahead of the truth. Ultimately it is by means of \emph{doublethink} that
the Party has been able---and may, for all we know, continue to be able
for thousands of years---to arrest the course of history.

\switchcolumn

双重思想意味着一个人的思想中同时存在两种相互矛盾的信念,且两种信念还都为这个人所接受。党的知识分子知道自己的记忆应该往哪个方向转变,因此他清楚自己在戏弄现实。但是通过实行双重思想,他会让自己相信现实没有受到损害。这一步必须是有意为之的,否则就不够精确,但它又必须是无意为之的,否则就会让人觉得虚假,并由此产生罪恶感。双重思想是英社的核心思想,因为保证目标坚定不移需要绝对的诚实,但在保证目标坚定不移的同时进行有意识地欺骗又是党的本质性行动。一方面有意说谎,一方面又对谎言信以为真,忘掉那些令人为难的事实,然后再在需要的时候,将它们从记忆深处拉出来;否认客观现实的存在,同时又考虑被否认的现实——所有这些都必不可少;甚至在使用双重思想一词时也必须用到双重思想。因为谁使用这个词谁就相当于承认篡改现实,而再用一次双重思想,就能将其所知的篡改行为抹去。如此循环,永不停止。最后,凭借着``双重思想'',党可以——也许正像我们所知道的那样,继续左右历史数千年——阻止历史的发展。

\switchcolumn*

All past oligarchies have fallen from power either because they ossified
or because they grew soft. Either they became stupid and arrogant,
failed to adjust themselves to changing circumstances, and were
overthrown, or they became liberal and cowardly, made concessions when
they should have used force, and once again were overthrown. They fell,
that is to say, either through consciousness or through unconsciousness.
It is the achievement of the Party to have produced a system of thought
in which both conditions can exist simultaneously. And upon no other
intellectual basis could the dominion of the Party be made permanent. If
one is to rule, and to continue ruling, one must be able to dislocate
the sense of reality. For the secret of rulership is to combine a belief
in one\textquotesingle s own infallibility with the power to learn from
past mistakes.

\switchcolumn

历史上所有的寡头体制都倒台了,这要么是因为其自身的僵化,要么是因为懦弱,它们不是因为愚蠢自大,不能适应环境的变化而被推翻,就是因为变得开明怯懦,在该使用武力的时候选择妥协而被颠覆。它们的失败或者是有意识的,或者是无意识的。而党的成功恰恰在于它制造出一种能让两种情况同时并存的思想体系。除此之外,没有任何其他的思想基础能让党的统治永恒不变。如果哪个人要进行统治,且希望自己的统治持续下去,那他就必须具备让人的现实感发生错乱的能力,因为统治的秘诀就是:将认为自己永远正确的信念和从过去错误中吸取到的教训结合起来。

\switchcolumn*

It need hardly be said that the subtlest practitioners of
\emph{doublethink} are those who invented \emph{doublethink} and know
that it is a vast system of mental cheating. In our society, those who
have the best knowledge of what is happening are also those who are
furthest from seeing the world as it is. In general, the greater the
understanding, the greater the delusion: the more intelligent, the less
sane. One clear illustration of this is the fact that war hysteria
increases in intensity as one rises in the social scale. Those whose
attitude toward the war is most nearly rational are the subject peoples
of the disputed territories. To these people the war is simply a
continuous calamity which sweeps to and fro over their bodies like a
tidal wave. Which side is winning is a matter of complete indifference
to them. They are aware that a change of overlordship means simply that
they will be doing the same work as before for new masters who treat
them in the same manner as the old ones. The slightly more favored
workers whom we call ``the proles'' are only intermittently conscious of
the war. When it is necessary they can be prodded into frenzies of fear
and hatred, but when left to themselves they are capable of forgetting
for long periods that the war is happening. It is in the ranks of the
Party, and above all of the Inner Party, that the true war enthusiasm is
found. World-conquest is believed in most firmly by those who know it to
be impossible. This peculiar linking-together of opposites---knowledge
with ignorance, cynicism with fanaticism---is one of the chief
distinguishing marks of Oceanic society. The official ideology abounds
with contradictions even where there is no practical reason for them.
Thus, the Party rejects and vilifies every principle for which the
Socialist movement originally stood, and it chooses to do this in the
name of Socialism. It preaches a contempt for the working class
unexampled for centuries past, and it dresses its members in a uniform
which was at one time peculiar to manual workers and was adopted for
that reason. It systematically undermines the solidarity of the family,
and it calls its leader by a name which is a direct appeal to the
sentiment of family loyalty. Even the names of the four Ministries by
which we are governed exhibit a sort of impudence in their deliberate
reversal of the facts. The Ministry of Peace concerns itself with war,
the Ministry of Truth with lies, the Ministry of Love with torture, and
the Ministry of Plenty with starvation. These contradictions are not
accidental, nor do they result from ordinary hypocrisy: they are
deliberate exercises in \emph{doublethink}. For it is only by
reconciling contradictions that power can be retained indefinitely. In
no other way could the ancient cycle be broken. If human equality is to
be forever averted---if the High, as we have called them, are to keep
their places permanently---then the prevailing mental condition must be
controlled insanity.

\switchcolumn

不用说,双重思想最巧妙的实施者就是发明双重思想并深知它是一个强大的思想欺骗系统的人。在我们的社会中,对世界了解最多的人对世界最不了解。总而言之,理解得最透彻的,也是误解最深的,越是聪明就越是愚蠢。举个典型例子,越是社会地位高的人,对战争越歇斯底里。而对战争持理性态度的往往是那些身处争议地区的被统治的人。在他们看来,战争无非是一场持续性灾难,如海浪一般反复冲刷他们的身体。对他们来说,谁取得胜利都没有分别。就算统治者发生变化,他们也仍然要做和从前一样的工作,新统治者对待他们的方式也和旧统治者的别无二致。而被我们称作``群众''的工人地位稍高一些,他们只偶尔意识到战争的存在。必要的时候可以刺激他们,让他们陷入强烈的恐惧和仇恨,但如果不理他们,那他们很长时间都不会想起战争正在发生。真正的战争狂热存在于党的内部,尤其是内党,坚信世界可以被征服的正是那些知道这不可能的人。这种奇特的对立统一关系——知与无知,冷漠自私与狂热盲从——就是大洋国社会区别于其他社会的显著标志。官方的意识形态充满矛盾,就算没有什么客观需要也是如此。因此,早期社会主义运动的原则没有一个不遭到党的抵制和中伤,而且党还是打着社会主义的旗号这样做的。过去几个世纪都没有这样的例子,党号召人们看低工人阶级,又因为这一原则,党要求党员穿上曾经只有工人才穿的制服。党有条不紊地削弱家庭的凝聚力,但它又用能唤起家庭忠诚感的名字称呼党的领导;甚至统治我们的四个部的名字,在歪曲事实上也已达到厚颜无耻的地步。和平部负责战争,真理部负责说谎,仁爱部负责用刑,富部负责制造饥饿。这样的矛盾并非偶然,也不是由通常意义上的虚伪所致,它是故意运用双重思想的结果。因为只有协调好矛盾,才能确保权力千秋万代,要打破古老的循环也只有如此。若想人类的平等永不实现,若上等人——我们所说的——要永远居于上等地位,那么就必须将社会的主流心理控制在一个疯狂的状态中。

\switchcolumn*

But there is one question which until this moment we have almost
ignored. It is: \emph{why} should human equality be averted? Supposing
that the mechanics of the process have been rightly described, what is
the motive for this huge, accurately planned effort to freeze history at
a particular moment of time?

\switchcolumn

不过,至此,我们差点忽略一个问题:为什么要避免人类平等?如果对以上这些方法的描述是正确的,那么如此声势浩大又深思熟虑地努力冻结某一时期的历史,又是出于什么样的动机呢?

\switchcolumn*

Here we reach the central secret. As we have seen, the mystique of the
Party, and above all of the Inner Party, depends upon
\emph{doublethink}. But deeper than this lies the original motive, the
never-questioned instinct that first led to the seizure of power and
brought \emph{doublethink}, the Thought Police, continuous warfare, and
all the other necessary paraphernalia into existence afterwards. This
motive really consists ...
\end{quotation}

\switchcolumn

此时我们已经触到最关键的秘密。正如我们所看到的,党是神秘的,尤其是内党,它的神秘性必要通过双重思想来实现。然而还有比这更深刻更原始的动机,即从未被质疑过的人的本能。是它导致了夺权的行动,是它带来了双重思想、思想警察、无止境的战争以及其他一些随之而来的东西,这个动机实际上包括\ldots\ldots{}

\switchcolumn*

Winston became aware of silence, as one becomes aware of a new sound. It
seemed to him that Julia had been very still for some time past. She was
lying on her side, naked from the waist upwards, with her cheek pillowed
on her hand and one dark lock tumbling across her eyes. Her breast rose
and fell slowly and regularly.

\switchcolumn

温斯顿发现周围非常安静,就好像发现了一种新的声音。他觉得朱莉亚已经很长时间没动窝了。她侧身躺着,腰部以上都赤裸着,她的脸枕在手上,一缕黑发垂在她的眼睛上面,她的胸脯缓慢而规律性地起伏着。

\switchcolumn*

``Julia.''

\switchcolumn

``朱莉亚。''

\switchcolumn*

No answer.

\switchcolumn

没有回答。

\switchcolumn*

``Julia, are you awake?''

\switchcolumn

``朱莉亚,你醒着吗?''

\switchcolumn*

No answer. She was asleep. He shut the book, put it carefully on the
floor, lay down, and pulled the coverlet over both of them.

\switchcolumn

没有回答,她睡着了。他合上书,小心地放在地上,然后躺下身,拉起床罩,将两个人都盖了起来。

\switchcolumn*

He had still, he reflected, not learned the ultimate secret. He
understood \emph{how}; he did not understand \emph{why}. Chapter 1, like
Chapter 3, had not actually told him anything that he did not know; it
had merely systematized the knowledge that he possessed already. But
after reading it he knew better than before that he was not mad. Being
in a minority, even a minority of one, did not make you mad. There was
truth and there was untruth, and if you clung to the truth even against
the whole world, you were not mad. A yellow beam from the sinking sun
slanted in through the window and fell across the pillow. He shut his
eyes. The sun on his face and the girl\textquotesingle s smooth body
touching his own gave him a strong, sleepy, confident feeling. He was
safe, everything was all right. He fell asleep murmuring ``Sanity is not
statistical,'' with the feeling that this remark contained in it a
profound wisdom.

\switchcolumn

他仍然不知道那个最大的秘密是什么,他想。他清楚怎样做,却不知道这样做的原因。第一章和第三章一样,都没有告诉他他不知道的东西,都只是把他了解的系统化。但读过之后,他比之前更加确定他没有疯。作为少数派,哪怕是只有一个人的少数派,也不能断定你是疯的。世上既有真理又有非真理,若你紧握真理,就算全世界都反对你,你也没有发疯。夕阳将黄色的光芒斜斜地照进窗户,照在枕头上。他闭上眼,洒在他脸颊上的阳光和紧贴着他的女孩的光滑的身体,让他产生一种强烈的、混杂着睡意的自信。他很安全,每件事都还好。他嗫嚅着``理智不是统计学上的'',睡着了,觉得这句话包含着深奥的智慧。

\switchcolumn*


\section{X}\label{x}

\switchcolumn

\section*{十}\label{ux5341ux516b}

\switchcolumn*

When he woke it was with the sensation of having slept for a long time,
but a glance at the old-fashioned clock told him that it was only
twenty-thirty. He lay dozing for a little while; then the usual
deep-lunged singing struck up from the yard below:

\switchcolumn

温斯顿醒了过来,觉得自己睡了很长时间,他扫了眼老式座钟,发现只有20点30分。他又打了一会儿盹,窗下的院子里再度传来那熟悉、低沉的歌:

\switchcolumn*

\begin{quotation}
  ``It was only an \textquotesingle opeless fancy,\\
  It passed like an Ipril dye,\\
  But a look an\textquotesingle{} a word an\textquotesingle{} the
  dreams they stirred\\
  They \textquotesingle ave stolen my \textquotesingle eart awye!''
\end{quotation}

\switchcolumn

\begin{quotation}
  \noindent
不过是毫无希望的幻想,\\
消失得如此之快,像四月的日子。\\
但一个眼神,一句话,一个被他们唤起的梦!\\
都可将我心偷走!
\end{quotation}

\switchcolumn*

The driveling song seemed to have kept its popularity. You still heard
it all over the place. It had outlived the ``Hate Song.'' Julia woke at
the sound, stretched herself luxuriously, and got out of bed.

\switchcolumn

这首烂歌似乎一直很流行,比《仇恨之歌》流行的时间还长,你在哪儿都能听到它。朱莉亚被歌声吵醒,她舒服地伸了个懒腰,起身下床。

\switchcolumn*

``I\textquotesingle m hungry,'' she said. ``Let\textquotesingle s make some
more coffee. Damn! The stove\textquotesingle s gone out and the
water\textquotesingle s cold.'' She picked the stove up and shook it.
``There\textquotesingle s no oil in it.''

\switchcolumn

``我饿了,''她说,``我们再煮点儿咖啡吧。见鬼!炉子熄了,水也凉了。''她提起炉子摇了摇,``没油了。''

\switchcolumn*

``We can get some from old Charrington, I expect.''

\switchcolumn

``可以管老查林顿要些,我猜。''

\switchcolumn*

``The funny thing is I made sure it was full. I\textquotesingle m going
to put my clothes on,'' she added. ``It seems to have got colder.''

\switchcolumn

``真奇怪,我敢肯定炉子原来是满的。我得把衣服穿上,''她说,``好像变冷了。''

\switchcolumn*

Winston also got up and dressed himself. The indefatigable voice sang
on:

\switchcolumn

温斯顿也起来了,穿好了衣服。那声音不知疲倦地唱道:

\switchcolumn*

\begin{quotation}
  ``They sye that time \textquotesingle eals all things,\\
  They sye you can always forget;\\
  But the smiles an\textquotesingle{} the tears across the years\\
  They twist my \textquotesingle eartstrings yet!''
\end{quotation}

\switchcolumn

\begin{quotation}
  \noindent
他们说时间可以医治一切,\\
他们说你终究会忘记;\\
但这些年的笑与泪,\\
仍牵动着我的心弦。
\end{quotation}

\switchcolumn*

As he fastened the belt of his overalls he strolled across to the
window. The sun must have gone down behind the houses; it was not
shining into the yard any longer. The flagstones were wet as though they
had just been washed, and he had the feeling that the sky had been
washed too, so fresh and pale was the blue between the chimney pots.
Tirelessly the woman marched to and fro, corking and uncorking herself,
singing and falling silent, and pegging out more diapers, and more and
yet more. He wondered whether she took in washing for a living, or was
merely the slave of twenty or thirty grandchildren. Julia had come
across to his side; together they gazed down with a sort of fascination
at the sturdy figure below. As he looked at the woman in her
characteristic attitude, her thick arms reaching up for the line, her
powerful marelike buttocks protruded, it struck him for the first time
that she was beautiful. It had never before occurred to him that the
body of a woman of fifty, blown up to monstrous dimensions by
childbearing, then hardened, roughened by work till it was coarse in the
grain like an overripe turnip, could be beautiful. But it was so, and
after all, he thought, why not? The solid, contourless body, like a
block of granite, and the rasping red skin, bore the same relation to
the body of a girl as the rose-hip to the rose. Why should the fruit be
held inferior to the flower?

\switchcolumn

他一面扎着工作服上的腰带,一面向窗户走去。太阳一定是落到房后去了,院子里不再有阳光。地上的石板湿乎乎的,就好像刚被冲洗过,他觉得天空也好像被洗过,从屋顶的烟囱间望去,能看到清新的暗蓝色天空。那个女人走来走去,没有一点疲惫的样子,她一会儿将衣夹衔在嘴里,一会儿又取出来,她一会儿大声歌唱,一会儿又默不作声,她没完没了地挂着尿布。温斯顿不知道她究竟是以洗衣为生,还是只单纯地为二三十个孙儿卖苦力。朱莉亚靠了过来,来到他身边,他们站在一起出神地看着下面那结实的身影。那个女人有一些独特的举止,他看着她,看见她将粗壮的手臂伸向晾衣绳,她的屁股壮得好像母马,他第一次意识到她很美。她的身体因生儿育女像充足了气一般巨大,又因为辛苦劳作变得粗糙强壮,宛若熟透的大萝卜。在此之前他从来都没想过,一个五十岁妇女的身体还会是美丽的,但她又的确很美。为什么不可以呢?他想,这壮实的、缺乏线条感的身体就像一块大理石,它和那红色的、粗糙的皮肤一起与少女的身体放在一处,就像玫瑰果与玫瑰花。为什么果实要比花朵逊色呢?

\switchcolumn*

``She\textquotesingle s beautiful,'' he murmured.

\switchcolumn

``她真美!''他喃喃地说。

\switchcolumn*

``She\textquotesingle s a meter across the hips, easily,'' said Julia.

\switchcolumn

``她的屁股至少有一米宽。''朱莉亚说。

\switchcolumn*

``That is her style of beauty,'' said Winston.

\switchcolumn

``那正是她的美丽所在。''温斯顿说。

\switchcolumn*

He held Julia\textquotesingle s supple waist easily encircled by his
arm. From the hip to the knee her flank was against his. Out of their
bodies no child would ever come. That was the one thing they could never
do. Only by word of mouth, from mind to mind, could they pass on the
secret. The woman down there had no mind, she had only strong arms, a
warm heart, and a fertile belly. He wondered how many children she had
given birth to. It might easily be fifteen. She had had her momentary
flowering, a year, perhaps, of wildrose beauty, and then she had
suddenly swollen like a fertilized fruit and grown hard and red and
coarse, and then her life had been laundering, scrubbing, darning,
cooking, sweeping, polishing, mending, scrubbing, laundering, first for
children, then for grandchildren, over thirty unbroken years. At the end
of it she was still singing. The mystical reverence that he felt for her
was somehow mixed up with the aspect of the pale, cloudless sky,
stretching away behind the chimney pots into interminable distances. It
was curious to think that the sky was the same for everybody, in Eurasia
or Eastasia as well as here. And the people under the sky were also very
much the same---everywhere, all over the world, hundreds or thousands of
millions of people just like this, people ignorant of one
another\textquotesingle s existence, held apart by walls of hatred and
lies, and yet almost exactly the same---people who had never learned to
think but who were storing up in their hearts and bellies and muscles
the power that would one day overturn the world. If there was hope, it
lay in the proles! Without having read to the end of \emph{the book}, he
knew that that must be Goldstein\textquotesingle s final message. The
future belonged to the proles. And could he be sure that when their time
came, the world they constructed would not be just as alien to him,
Winston Smith, as the world of the Party? Yes, because at the least it
would be a world of sanity. Where there is equality there can be sanity.
Sooner or later it would happen: strength would change into
consciousness. The proles were immortal; you could not doubt it when you
looked at that valiant figure in the yard. In the end their awakening
would come. And until that happened, though it might be a thousand
years, they would stay alive against all the odds, like birds, passing
on from body to body the vitality which the Party did not share and
could not kill.

\switchcolumn

他用一只手就可以轻松地将朱莉亚的纤腰揽起来,她的身体从臀到膝紧紧地贴着他。但他们两个人却不能生育孩子,且这件事他们永远都不能做。他们只能靠语言来交流思想,交换秘密。而楼下的那个女人则没有思想,她有的是强健的臂膀、温暖的心灵和多产的肚皮。他想知道她到底生了多少个孩子,可能最少有十五个。她曾拥有短暂的、花一般的年华,也许有那么一年,她像野玫瑰一样诱人,但之后就仿佛突然受精的果子,她变得强壮、红润、粗放。再之后,她就生活在洗衣、擦地、缝补、做饭、扫地、擦抹、修理上。先是为子女,然后为孙儿,三十年中从未间断,可到了最后,她依然在歌唱。温斯顿崇敬她,这感觉很神秘地和屋顶烟囱后那片清新无云的天空混杂在一起。很奇怪,对任何人来说,天空都是一样的,不管在欧亚国,还是在东亚国,抑或是这里。天空下的人也没什么不同——世界的每个角落都差不多,几亿或者几十亿的人都不知道彼此的存在。但是,虽然仇恨与谎言筑起的高墙将他们彼此隔绝,他们仍然大同小异——他们不知道该如何思考,但他们的心灵、他们的身体中、他们的肌肉里,却聚积着力量,且总有一天他们要用这种力量颠覆世界。如果有希望,它们就在群众身上!他无须将那本书读完,他知道那是高德斯坦因留下的最后讯息。未来属于群众。他能确定,在群众的时代,群众建立的世界不会和党建立的世界一样吗?他能确定他不会融不进那世界吗?能,他能确定,因为那至少会是一个神志健全的世界。哪里有平等,哪里就能拥有健全的神志。力量改变意识迟早会发生。群众不朽,只要看看院子里那无所畏惧的身影,你就不会怀疑。终有一天他们会觉醒,尽管这可能要用上一千多年,尽管生存下去需要克服各种不利条件,他们就像飞鸟,将生命的活力从一个躯体传递到另一个躯体,而党既没有这种活力,又无法将它扼杀。

\switchcolumn*

``Do you remember,'' he said, ``the thrush that sang to us, that first day,
at the edge of the wood?''

\switchcolumn

``你记得吗?''他问,``约会的第一天,在树林边上对我们唱歌的画眉?''

\switchcolumn*

``He wasn\textquotesingle t singing to us,'' said Julia. ``He was singing
to please himself. Not even that. He was just singing.''

\switchcolumn

``它没对我们唱歌,''朱莉亚说,``它唱歌是为了自己高兴,甚至不能这么说。它只是在唱歌罢了。''

\switchcolumn*

The birds sang, the proles sang, the Party did not sing. All round the
world, in London and New York, in Africa and Brazil and in the
mysterious, forbidden lands beyond the frontiers, in the streets of
Paris and Berlin, in the villages of the endless Russian plain, in the
bazaars of China and Japan---everywhere stood the same solid
unconquerable figure, made monstrous by work and childbearing, toiling
from birth to death and still singing. Out of those mighty loins a race
of conscious beings must one day come. You were the dead; theirs was the
future. But you could share in that future if you kept alive the mind as
they kept alive the body, and passed on the secret doctrine that two
plus two make four.

\switchcolumn

鸟唱歌,群众唱歌,党不唱歌。综观世界,在伦敦,在纽约,在非洲,在巴西,在边境地区的神秘禁地,在巴黎、柏林的大街,在广袤的俄罗斯平原的村庄,在中国和日本的市集——到处都伫立着强健又不可战胜的身躯,这些身躯因工作和生儿育女变得庞大,从出生到死都辛劳不休,都唱个不停。总有一天,他们强壮有力的腹部将生出神志清醒的民族。你是死的,他们是未来。但是,如果你能像他们保持身体的生命力那样保持头脑的生命力,让诸如二加二等于四这样的秘密学说传递下去,那你也能分享到未来。

\switchcolumn*

``We are the dead,'' he said.

\switchcolumn

``我们是死人。''他说。

\switchcolumn*

``We are the dead,'' echoed Julia dutifully.

\switchcolumn

``我们是死人。''朱莉亚顺从地应和道。

\switchcolumn*

``You are the dead,'' said an iron voice behind them.

\switchcolumn

``你们是死人。''从他们背后,传来一个冷酷的声音。

\switchcolumn*

They sprang apart. Winston\textquotesingle s entrails seemed to have
turned into ice. He could see the white all round the irises of
Julia\textquotesingle s eyes. Her face had turned a milky yellow. The
smear of rouge that was still on each cheekbone stood out sharply,
almost as though unconnected with the skin beneath.

\switchcolumn

他们跳着分开了。温斯顿觉得自己的内脏都结成了冰。他能看到朱莉亚的眼珠四周已泛起了白色,她的脸变得蜡黄,衬得脸上的腮红更加显眼,就好像浮出了皮肤表面。

\switchcolumn*

``You are the dead,'' repeated the iron voice.

\switchcolumn

``你们是死人。''又是那个冷酷的声音。

\switchcolumn*

``It was behind the picture,'' breathed Julia.

\switchcolumn

``在画后面。''朱莉亚悄悄地说。

\switchcolumn*

``It was behind the picture,'' said the voice. ``Remain exactly where you
are. Make no movement until you are ordered.''

\switchcolumn

``在画后面。''那声音说,``站那儿别动,别动,直到命令你们动。''

\switchcolumn*

It was starting, it was starting at last! They could do nothing except
stand gazing into one another\textquotesingle s eyes. To run for life,
to get out of the house before it was too late---no such thought
occurred to them. Unthinkable to disobey the iron voice from the wall.
There was a snap as though a catch had been turned back, and a crash of
breaking glass. The picture had fallen to the floor, uncovering the
telescreen behind it.

\switchcolumn

开始了,终于开始了!他们什么都不能做,只能凝视着对方的眼睛。快逃命吧,快逃出房子,不然就太晚了——他们从未萌生这样的念头,他们不敢违背墙里发出的冷酷声音,想都不敢想。只听``啪''的一声,好像有什么东西翻了过来,紧接着,又听到玻璃被打碎的声音,那幅画落到了地板上,露出了后面的电屏。

\switchcolumn*

``Now they can see us,'' said Julia.

\switchcolumn

``现在他们能看见我们了。''朱莉亚说。

\switchcolumn*

``Now we can see you,'' said the voice. ``Stand out in the middle of the
room. Stand back to back. Clasp your hands behind your heads. Do not
touch one another.''

\switchcolumn

``现在我们可以看见你们。''那声音说,``站到屋子中间,背靠背,手放头上,不许接触对方。''

\switchcolumn*

They were not touching, but it seemed to him that he could feel
Julia\textquotesingle s body shaking. Or perhaps it was merely the
shaking of his own. He could just stop his teeth from chattering, but
his knees were beyond his control. There was a sound of trampling boots
below, inside the house and outside. The yard seemed to be full of men.
Something was being dragged across the stones. The
woman\textquotesingle s singing had stopped abruptly. There was a long,
rolling clang, as though the washtub had been flung across the yard, and
then a confusion of angry shouts which ended in a yell of pain.

\switchcolumn

他们没有互相接触,但他好像可以感觉到朱莉亚的颤抖,也许是他自己在颤抖。他控制住牙齿,以免它们上下打战,但他却控制不住自己的膝盖。楼下传来皮靴声,屋里屋外都听得见。院子里也似乎站满了人。有东西拖过石板地,女人的歌声戛然而止。又有什么东西滚了过去,发出长长的声音,好像是洗衣盆翻过了院子,接着是混乱而愤怒的叫喊,最后是一阵痛苦的尖叫。

\switchcolumn*

``The house is surrounded,'' said Winston.

\switchcolumn

``房子被包围了。''温斯顿说。

\switchcolumn*

``The house is surrounded,'' said the voice.

\switchcolumn

``房子被包围了。''那声音说。

\switchcolumn*

He heard Julia snap her teeth together. ``I suppose we may as well say
good-by,'' she said.

\switchcolumn

他听见朱莉亚咬紧牙齿。``我猜,我们还是说再见吧。''她说。

\switchcolumn*

``You may as well say good-by,'' said the voice. And then another quite
different voice, a thin, cultivated voice which Winston had the
impression of having heard before, struck in: ``And by the way, while we
are on the subject, \emph{Here comes a candle to light you to bed, here
comes a chopper to chop off your head!}''

\switchcolumn

``你们还是说再见吧。''那声音说。之后,又传来一个截然不同的、细声细气的、文雅
的声音,这声音温斯顿曾经听过。``顺便说一句,在我们谈论这个话题的同时,这儿有
一根蜡烛照你上床,这儿还有一把斧头砍你脑袋!''

\switchcolumn*

Something crashed on to the bed behind Winston\textquotesingle s back.
The head of a ladder had been thrust through the window and had burst in
the frame. Someone was climbing through the window. There was a stampede
of boots up the stairs. The room was full of solid men in black
uniforms, with iron-shod boots on their feet and truncheons in their
hands.

\switchcolumn

有东西摔到了温斯顿身后的那张床上,一张梯子伸进窗户,压坏了窗框,有人顺着梯子爬进窗户。楼梯上再度响起皮靴的声音,满屋子都是穿着黑色制服的彪形大汉,他们登着钉有铁掌的皮靴,手里抓着警棍。

\switchcolumn*

Winston was not trembling any longer. Even his eyes he barely moved. One
thing alone mattered: to keep still, to keep still and not give them an
excuse to hit you! A man with a smooth prizefighter\textquotesingle s
jowl in which the mouth was only a slit paused opposite him, balancing
his truncheon meditatively between thumb and forefinger. Winston met his
eyes. The feeling of nakedness, with one\textquotesingle s hands behind
one\textquotesingle s head and one\textquotesingle s face and body all
exposed, was almost unbearable. The man protruded the tip of a white
tongue, licked the place where his lips should have been, and then
passed on. There was another crash. Someone had picked up the glass
paperweight from the table and smashed it to pieces on the hearthstone.

\switchcolumn

温斯顿不再发抖了,眼珠也一动不动。只有一件事是重要的:待着别动,待着别动,不要给他们殴打你的理由!一个男人站在他前面,那人的下巴像拳击手一样平坦,嘴巴细成一条缝,他用食指和大拇指夹着警棍,好像在思考着什么。他看着温斯顿,让温斯顿有一种赤身裸体的感觉,由于手放在脑后,温斯顿的脸和身体完全暴露在他的面前,这简直让人难以忍受。那人伸出白色的舌尖,舔了舔嘴唇,走开了。这时又有东西被打破,有人拿起桌子上的玻璃镇纸,将它摔在壁炉的石头上,摔碎了。

\switchcolumn*

The fragment of coral, a tiny crinkle of pink like a sugar rosebud from
a cake, rolled across the mat. How small, thought Winston, how small it
always was! There was a gasp and a thump behind him, and he received a
violent kick on the ankle which nearly flung him off his balance. One of
the men had smashed his fist into Julia\textquotesingle s solar plexus,
doubling her up like a pocket ruler. She was thrashing about on the
floor, fighting for breath. Winston dared not turn his head even by a
millimeter, but sometimes her livid, gasping face came within the angle
of his vision. Even in his terror it was as though he could feel the
pain in his own body, the deadly pain which nevertheless was less urgent
than the struggle to get back her breath. He knew what it was like: the
terrible, agonizing pain which was there all the while but could not be
suffered yet, because before all else it was necessary to be able to
breathe. Then two of the men hoisted her up by knees and shoulders and
carried her out of the room like a sack. Winston had a glimpse of her
face, upside down, yellow and contorted, with the eyes shut, and still
with a smear of rouge on either cheek; and that was the last he saw of
her.

\switchcolumn

珊瑚的碎片从毯子上滚了过去,它就像蛋糕上糖玫瑰花的花蕾,是一小片粉红色的皱巴巴的东西,它是如此地渺小,温斯顿想,它一直都是这样渺小。温斯顿的身后传来吸气的声音,接着砰的一声,他的脚踝被人狠狠地踹了一脚,几乎让他失去平衡。其中一个男人一拳打到朱莉亚的腹部,打得她像折尺那样弯下了身子,在地板上扭动,喘不上气。温斯顿连微微转下脑袋的胆子都没有,但偶尔他还是能从眼角看到她面色惨白、呼吸困难的样子。即便身处恐惧,他的身体也仿佛能体会到她的疼痛,但就算最致命的疼痛也比不上让她恢复呼吸这般紧要。他清楚这种感觉,这可怕的、令人提心吊胆的疼痛一直在那儿却又没法克服,因为呼吸才是最为重要的事。两个男人将朱莉亚抬了起来,他们拉着她的肩膀和膝盖,像抬麻袋似的将她抬出了屋。温斯顿瞥到她的脸,她的脸向上仰起,脸色发黄,她的五官扭曲着,双目紧闭,她的面颊上仍能看出腮红的痕迹。这就是他看到她的最后一眼。

\switchcolumn*

He stood dead still. No one had hit him yet. Thoughts which came of
their own accord but seemed totally uninteresting began to flit through
his mind. He wondered whether they had got Mr. Charrington. He wondered
what they had done to the woman in the yard. He noticed that he badly
wanted to urinate, and felt a faint surprise, because he had done so
only two or three hours ago. He noticed that the clock on the
mantelpiece said nine, meaning twenty-one. But the light seemed too
strong. Would not the light be fading at twenty-one hours on an August
evening? He wondered whether after all he and Julia had mistaken the
time---had slept the clock round and thought it was twenty-thirty when
really it was nought eight-thirty on the following morning. But he did
not pursue the thought further. It was not interesting.

\switchcolumn

他像死人般站在那里,还没有遭到殴打,他的脑海中自动浮现出好几种想法,但都毫无意义。他想知道他们是否抓住了查林顿先生,想知道他们会怎样处理院子里的那个女人。他有点吃惊,自己竟很想撒尿,而就在两三个小时前,他才刚刚尿过。他注意到壁炉上的座钟指到了9点,即21点。可光线好像太强了些,难道8月的晚上,到了21点,天还不会黑吗?他怀疑他和朱莉亚把时间搞错了——他们睡了十二个小时,他们以为是20点30分,实际却是第二天早上8点30分。他没有再想下去,这没意义。

\switchcolumn*

There was another, lighter step in the passage. Mr. Charrington came
into the room. The demeanor of the black-uniformed men suddenly became
more subdued. Something had also changed in Mr.
Charrington\textquotesingle s appearance. His eye fell on the fragments
of the glass paperweight.

\switchcolumn

一阵较轻的脚步声从走廊里传来,查林顿先生走进屋子。穿黑制服的人突然变得恭顺起来。查林顿先生的样子发生了一些改变。他看了看玻璃镇纸的碎片。

\switchcolumn*

``Pick up those pieces,'' he said sharply.

\switchcolumn

``把碎片捡起来。''他厉声说。

\switchcolumn*

A man stooped to obey. The cockney accent had disappeared; Winston
suddenly realized whose voice it was that he had heard a few moments ago
on the telescreen. Mr. Charrington was still wearing his old velvet
jacket, but his hair, which had been almost white, had turned black.
Also he was not wearing his spectacles. He gave Winston a single sharp
glance, as though verifying his identity, and then paid no more
attention to him. He was still recognizable, but he was not the same
person any longer. His body had straightened, and seemed to have grown
bigger. His face had undergone only tiny changes that had nevertheless
worked a complete transformation. The black eyebrows were less bushy,
the wrinkles were gone, the whole lines of the face seemed to have
altered; even the nose seemed shorter. It was the alert, cold face of a
man of about five-and-thirty. It occurred to Winston that for the first
time in his life he was looking, with knowledge, at a member of the
Thought Police.

\switchcolumn

一个男人按照他的命令弯下腰。查林顿先生的方言腔消失了,温斯顿突然意识到几分钟前他听到的电屏上的声音来自于谁。查林顿先生仍旧穿着那件旧天鹅绒夹克,但他近乎全白的头发现在却变成了黑色。他不再戴着眼镜,他目光凌厉地看了温斯顿一眼,似乎在确认他的身份,然后便不再注意他。温斯顿仍然认得他的样子,但他已是另一个人。他的身体伸直了,看起来变大了,脸的变化很微小,却令他完全改头换面,他的眉毛变少了,皱纹没了,整个脸的轮廓都不一样了,甚至鼻子也好像变短了。这是一张大约三十五岁的男人的脸,机警,冷酷。它让温斯顿意识到,这是他有生以来第一次在已知的情况下,看到一个思想警察。

\switchcolumn*

\clearpage
\part{THREE}\label{three}

\switchcolumn

\part*{第三部分}

\switchcolumn*

\section{I}

\switchcolumn

\section*{一}\label{ux5341ux4e5d}

\switchcolumn*

He did not know where he was. Presumably he was in the Ministry of Love;
but there was no way of making certain.

\switchcolumn

他不知道自己身在哪里,也许在仁爱部,但他没法确定。

\switchcolumn*

He was in a high-ceilinged windowless cell with walls of glittering
white porcelain. Concealed lamps flooded it with cold light, and there
was a low, steady humming sound which he supposed had something to do
with the air supply. A bench, or shelf, just wide enough to sit on ran
round the wall, broken only by the door and, at the end opposite the
door, a lavatory pan with no wooden seat. There were four telescreens,
one in each wall.

\switchcolumn

他待在有着高高天花板却没有窗户的牢房里,牢房的墙壁贴满了白色的瓷砖,隐蔽式电灯发出的冷光充满了整个房间。屋子里有一种微小的、没完没了的嗡嗡声,他猜可能和空气供给装置有关。沿墙安有板凳,或者说是木架,其宽度只够人将将坐下,板凳很长,只有门那里没有。而门的对面有一个没有坐圈的马桶。牢房有四个电屏,分别安装在四面墙上。

\switchcolumn*

There was a dull aching in his belly. It had been there ever since they
had bundled him into the closed van and driven him away. But he was also
hungry, with a gnawing, unwholesome kind of hunger. It might be
twenty-four hours since he had eaten, it might be thirty-six. He still
did not know, probably never would know, whether it had been morning or
evening when they arrested him. Since he was arrested he had not been
fed.

\switchcolumn

他的腹部隐隐作痛。自从他们将他扔进一辆封闭的货车带走以后,它就一直在疼。但同
时他也很饿,是那种痛苦的、不健康的饥饿。他大概有24小时没吃东西了,也可能是36
小时。他还是搞不清,他们究竟是在早上抓住的他,还是晚上,或许他永远无法弄清了。
被捕后他就没吃过东西。

\switchcolumn*

He sat as still as he could on the narrow bench, with his hands crossed
on his knee. He had already learned to sit still. If you made unexpected
movements they yelled at you from the telescreen. But the craving for
food was growing upon him. What he longed for above all was a piece of
bread. He had an idea that there were a few breadcrumbs in the pocket of
his overalls. It was even possible---he thought this because from time
to time something seemed to tickle his leg---that there might be a
sizable bit of crust there. In the end the temptation to find out
overcame his fear; he slipped a hand into his pocket.

\switchcolumn

他尽可能地静静地坐在长凳上,双手交叉放在膝上,他已学会一动不动地坐着。若他乱动,他们就会从电屏里向他大吼。但对食物的渴望愈发强烈,他很想吃上一片面包。他依稀记得在他制服的口袋里还有一点儿面包屑,可能还是很大的一块。有个东西不时就会碰到他的腿,让他觉得,口袋里也许还装着一块相当大的面包。最后,想一探究竟的心情战胜了恐惧,他悄悄地将手伸向口袋。

\switchcolumn*

``Smith!'' yelled a voice from the telescreen. ``6079 Smith W! Hands out of
pockets in the cells!''

\switchcolumn

``史密斯!''电屏里的声音喊道,``6079号史密斯!在牢房里把手放在口袋外!''

\switchcolumn*

He sat still again, his hands crossed on his knee. Before being brought
here he had been taken to another place which must have been an ordinary
prison or a temporary lock-up used by the patrols. He did not know how
long he had been there; some hours, at any rate; with no clocks and no
daylight it was hard to gauge the time. It was a noisy, evil-smelling
place. They had put him into a cell similar to the one he was now in,
but filthily dirty and at all times crowded by ten or fifteen people.
The majority of them were common criminals, but there were a few
political prisoners among them. He had sat silent against the wall,
jostled by dirty bodies, too preoccupied by fear and the pain in his
belly to take much interest in his surroundings, but still noticing the
astonishing difference in demeanor between the Party prisoners and the
others. The Party prisoners were always silent and terrified, but the
ordinary criminals seemed to care nothing for anybody. They yelled
insults at the guards, fought back fiercely when their belongings were
impounded, wrote obscene words on the floor, ate smuggled food which
they produced from mysterious hiding places in their clothes, and even
shouted down the telescreen when it tried to restore order. On the other
hand, some of them seemed to be on good terms with the guards, called
them by nicknames, and tried to wheedle cigarettes through the spy-hole
in the door. The guards, too, treated the common criminals with a
certain forbearance, even when they had to handle them roughly. There
was much talk about the forced-labor camps to which most of the
prisoners expected to be sent. It was ``all right'' in the camps, he
gathered, so long as you had good contacts and knew the ropes. There
were bribery, favoritism, and racketeering of every kind, there were
homosexuality and prostitution, there was even illicit alcohol distilled
from potatoes. The positions of trust were given only to the common
criminals, especially the gangsters and the murderers, who formed a sort
of aristocracy. All the dirty jobs were done by the politicals.

\switchcolumn

他又一动不动地坐着,双手交叉放在膝上。在被带到这里之前,他曾被送去另一个地方,那里要么是普通监狱,要么是巡逻队的临时拘留所。他不知道在那里待了多长时间,怎么也有几个小时。那里没有钟没有阳光,要确定时间十分困难。那是个吵闹的,气味恶心的地方。他们将他关进和现在这牢房差不多的牢房里,但那里脏得要命,总是关着十或十五个人。那些人大部分都是普通罪犯,其中只有少数几个政治犯。他沉默地靠墙而坐,被脏兮兮的人夹着。尽管他的心被恐惧和腹部的疼痛占据,以至于他并不关心周围的环境。但他仍然吃惊地发现党员囚犯和其他囚犯在举止上区别明显。党员囚犯总是一声不响,战战兢兢,普通囚犯好像对什么都无所谓。他们吼叫着辱骂看守,在财物被没收时拼命反抗。他们在地板上写下下流的单词,把食物藏在衣服里偷运进牢房吃掉,甚至在电屏试图维持秩序时,依然大声喧哗。另一方面,他们中的一些又似乎和看守关系很好,他们叫看守的绰号,从门上的监视孔往外塞香烟。相比之下,看守对普通囚犯也更加宽容,即使他们不得不粗暴地管理他们。由于大部分囚犯都要被送往劳改营,牢房里有很多这方面的讨论。按照温斯顿的推断,只要你知道规矩,搞好关系,劳改营也还不错。行贿、走后门、敲诈、同性恋、卖淫以及用土豆酿制非法的酒精饮品,那里都有。而在劳改营,只有普通罪犯能够得到信任,尤其是帮派分子和杀人犯,他们是监狱里的特权阶层。所有脏活儿都由政治犯包揽。

\switchcolumn*

There was a constant come-and-go of prisoners of every description: drug
peddlers, thieves, bandits, black marketeers, drunks, prostitutes. Some
of the drunks were so violent that the other prisoners had to combine to
suppress them. An enormous wreck of a woman, aged about sixty, with
great tumbling breasts and thick coils of white hair which had come down
in her struggles, was carried in, kicking and shouting, by four guards
who had hold of her one at each corner. They wrenched off the boots with
which she had been trying to kick them, and dumped her down across
Winston\textquotesingle s lap, almost breaking his thighbones. The woman
hoisted herself upright and followed them out with a yell of ``F------
bastards!'' Then, noticing that she was sitting on something uneven, she
slid off Winston\textquotesingle s knees onto the bench.

\switchcolumn

各种各样的犯人在监狱里来来往往:毒贩、小偷、强盗、黑市商贩、酒鬼、妓女。有些酒鬼是如此凶猛,其他犯人要联合起来才能将其制服。一个身材庞大的六十岁左右的女人被四个看守抓着四肢抬了进来。她硕大的乳房在胸前晃荡,她盘起的浓密的白发因为挣扎而散落下来,她一边乱踢乱踹,一边大声喊叫。他们脱下她的靴子,将她扔到温斯顿的身上,几乎将后者的大腿骨压坏。女人坐起身,冲着看守的背影叫骂:``操,杂种!''之后,她注意到自己坐得不平,便从温斯顿的膝盖上滑下来,坐到长凳上。

\switchcolumn*

``Beg pardon, dearie,'' she said. ``I wouldn\textquotesingle t
\textquotesingle a sat on you, only the buggers put me there. They dono
\textquotesingle ow to treat a lady, do they?'' She paused, patted her
breast, and belched. ``Pardon,'' she said, ``I ain\textquotesingle t
meself, quite.''

\switchcolumn

``很抱歉,亲爱的,''她说。``若不是这些下贱东西推我,我是不会坐在你身上的。他们不懂要如何对待女士。''她停下来,拍了拍胸口,打了个嗝。``抱歉,我不大舒服。''

\switchcolumn*

She leant forward and vomited copiously on the floor.

\switchcolumn

她的身子向前倾去,她吐了,在地板上吐了好大一摊。

\switchcolumn*

``Thass better,'' she said, leaning back with closed eyes. ``Never keep it
down, thass what I say. Get it up while it\textquotesingle s fresh on
your stomach, like.''

\switchcolumn

``好多了,''她边说边向后靠去,闭上了眼睛。``我的意思是,别忍着,趁你的胃还没开始消化,吐出来。''

\switchcolumn*

She revived, turned to have another look at Winston, and seemed
immediately to take a fancy to him. She put a vast arm round his
shoulder and drew him toward her, breathing beer and vomit into his
face.

\switchcolumn

她恢复过来,转过头看了看温斯顿,似乎立刻喜欢上了他,她用粗壮的手臂搂住温斯顿的肩膀,将他拉向自己,一股啤酒味和呕吐的气味扑到了温斯顿的脸上。

\switchcolumn*

``Wass your name, dearie?'' she said.

\switchcolumn

``你叫什么? 亲爱的。''她问。

\switchcolumn*

``Smith,'' said Winston.

\switchcolumn

``史密斯。''温斯顿说。

\switchcolumn*

``Smith?'' said the woman. ``Thass funny. My name\textquotesingle s Smith
too. Why,'' she added sentimentally, ``I might be your mother!''

\switchcolumn

``史密斯?''女人说,``这真有趣,我也叫史密斯。为什么会这样?''她的声音里多了几分感情,``也许,我是你的母亲!''

\switchcolumn*

She might, thought Winston, be his mother. She was about the right age
and physique, and it was probable that people changed somewhat after
twenty years in a forced-labor camp.

\switchcolumn

他想,她真有可能是他的母亲。她的年龄和体型都与他的母亲相符,人在劳动营里呆上二十年,外表很可能会发生变化。

\switchcolumn*

No one else had spoken to him. To a surprising extent the ordinary
criminals ignored the Party prisoners. ``The pol\emph{its},'' they called
them, with a sort of uninterested contempt. The Party prisoners seemed
terrified of speaking to anybody, and above all of speaking to one
another. Only once, when two Party members, both women, were pressed
close together on the bench, he overheard amid the din of voices a few
hurriedly whispered words; and in particular a reference to something
called ``room one-oh-one,'' which he did not understand.

\switchcolumn

再没有囚犯和他讲话。令人奇怪的是,普通囚犯对党员囚犯视若无睹。前者称呼后者为``党奴'',多少带着轻蔑。党员囚犯好像很怕和别人说话,特别害怕和人交流。只有一次,两个坐在长凳上的女党员被挤到了一起,他在一片嘈杂声中听到她们匆匆的交谈。她们的声音很轻,她们特别提到了``101号房'',温斯顿不明白这是什么意思。

\switchcolumn*

It might be two or three hours ago that they had brought him here. The
dull pain in his belly never went away, but sometimes it grew better and
sometimes worse, and his thoughts expanded or contracted accordingly.
When it grew worse he thought only of the pain itself, and of his desire
for food. When it grew better, panic took hold of him. There were
moments when he foresaw the things that would happen to him with such
actuality that his heart galloped and his breath stopped. He felt the
smash of truncheons on his elbows and iron-shod boots on his shins; he
saw himself groveling on the floor, screaming for mercy through broken
teeth. He hardly thought of Julia. He could not fix his mind on her. He
loved her and would not betray her; but that was only a fact, known as
he knew the rules of arithmetic. He felt no love for her, and he hardly
even wondered what was happening to her. He thought oftener of
O\textquotesingle Brien, with a flickering hope. O\textquotesingle Brien
must know that he had been arrested. The Brotherhood, he had said, never
tried to save its members. But there was the razor blade; they would
send the razor blade if they could. There would be perhaps five seconds
before the guards could rush into the cell. The blade would bite into
him with a sort of burning coldness, and even the fingers that held it
would be cut to the bone. Everything came back to his sick body, which
shrank trembling from the smallest pain. He was not certain that he
would use the razor blade even if he got the chance. It was more natural
to exist from moment to moment, accepting another ten
minutes\textquotesingle{} life even with the certainty that there was
torture at the end of it.

\switchcolumn

大约两三个钟头前,他们将他带到这里。腹部的隐痛从未消退,时轻时重,他的思绪也随着这痛楚时而轻松,时而紧张。疼痛严重时,他只想着疼痛本身,只为饥饿难过。疼痛减轻时,恐惧便充斥他的内心。每当他想到将要发生在自己身上的事,就仿佛身临其境,心跳厉害,呼吸困难。他感觉警棍打到他的手肘,钉着铁掌的皮靴踢到他的小腿。他看到自己在地板上爬行,牙齿被打掉,尖叫着求饶。他几乎没想到朱莉亚。他不能将思绪集中在她身上。他爱她,不会背叛她,但这仅仅是一个事实,一件和他所知的数学规律一样的事实。他感觉不到对她的爱,也没想过她究竟会怎样。但他倒常常抱着一线希望想起奥布兰。奥布兰可能已经知道他被捕了。他说过兄弟会从来不会营救它的成员。但是他们有刮胡刀片,如果他们能将刀片送进来的话。在看守冲进牢房前,只要五秒就够了。刀片将带着灼热的冰冷感切入他的身体,拿着它的手指也会被割出骨头。所有感觉都重新回到他这病恹恹的身体上,哪怕是最轻微的疼痛也会让他蜷缩着身体抖个不停。即便有机会,他也不能确定他真的会用那个刀片。比这更理所当然的是活一天算一天,多活十分钟也好,哪怕最终将遭到毒打。

\switchcolumn*

Sometimes he tried to calculate the number of porcelain bricks in the
walls of the cell. It should have been easy, but he always lost count at
some point or another. More often he wondered where he was, and what
time of day it was. At one moment he felt certain that it was broad
daylight outside, and at the next equally certain that it was pitch
darkness. In this place, he knew instinctively, the lights would never
be turned out. It was the place with no darkness: he saw now why
O\textquotesingle Brien had seemed to recognize the allusion. In the
Ministry of Love there were no windows. His cell might be at the heart
of the building or against its outer wall; it might be ten floors below
ground, or thirty above it. He moved himself mentally from place to
place, and tried to determine by the feeling of his body whether he was
perched high in the air or buried deep underground.

\switchcolumn

有时他试着去数墙上的瓷砖。这应该不难,但他总是忘记自己数了多少。而他更常思考的是他究竟身在何处,此时是什么时间。有次,他可以肯定外面是白天,但很快,他又觉得外面一定漆黑一片。在这里,直觉让他知道灯光永不熄灭。这个地方没有黑暗,现在他才明白为什么奥布兰看起来对这一隐喻心领神会。仁爱部没有窗户。他的牢房可能在大楼的中心,也可能对着外墙,可能在地下十层,也可能在地上三十层。他任思绪从一个地方转移到另一个地方,并尝试着根据身体的感觉确定自己是身处高空还是身处地下。

\switchcolumn*

There was a sound of marching boots outside. The steel door opened with
a clang. A young officer, a trim black-uniformed figure who seemed to
glitter all over with polished leather and whose pale, straight-featured
face was like a wax mask, stepped smartly through the doorway. He
motioned to the guards outside to bring in the prisoner they were
leading. The poet Ampleforth shambled into the cell. The door clanged
shut again.

\switchcolumn

牢房外响起一阵皮靴走动的声音。铁制的大门打开了。一个年轻的官员敏捷地走了进来,他身着黑色的制服,整个人像擦亮的皮革一般光彩照人,他那苍白又线条分明的脸孔宛若蜡制的面具。他示意门外的看守将犯人带进来。诗人安普福斯跌跌撞撞地走进牢房。之后,门又哐啷一声关上了。

\switchcolumn*

Ampleforth made one or two uncertain movements from side to side, as
though having some idea that there was another door to go out of, and
then began to wander up and down the cell. He had not yet noticed
Winston\textquotesingle s presence. His troubled eyes were gazing at the
wall about a meter above the level of Winston\textquotesingle s head. He
was shoeless; large, dirty toes were sticking out of the holes in his
socks. He was also several days away from a shave. A scrubby beard
covered his face to the cheekbones, giving him an air of ruffianism that
went oddly with his large weak frame and nervous movements.

\switchcolumn

安普福斯有些迟疑地挪动了一下,似乎觉得还有道门要进,然后他开始在牢房里踱起步来。他还没发现温斯顿的存在。他目光忧郁地盯着温斯顿头上一米处的墙。他没穿鞋,肮脏的脚趾从袜子的破洞处露出来。由于好几天都没刮胡子,胡楂儿布满他的脸颊,一直延伸到颧骨,赋予他一种凶狠的面貌。这面貌搭配上他那高大瘦弱的身体以及紧张兮兮的举止,让他看起来非常奇怪。

\switchcolumn*

Winston roused himself a little from his lethargy. He must speak to
Ampleforth, and risk the yell from the telescreen. It was even
conceivable that Ampleforth was the bearer of the razor blade.

\switchcolumn

温斯顿很累,但还是稍稍振作下精神。他必须和安普福斯说上几句,哪怕冒着被电屏呵斥的危险,可以想象安普福斯就是送刀片的人。

\switchcolumn*

``Ampleforth,'' he said.

\switchcolumn

``安普福斯。''他说。

\switchcolumn*

There was no yell from the telescreen. Ampleforth paused, mildly
startled. His eyes focused themselves slowly on Winston.

\switchcolumn

电屏没有发出呵斥声。安普福斯停下脚步,有点吃惊。他慢慢地将目光聚焦到温斯顿身
上。

\switchcolumn*

``Ah, Smith!'' he said. ``You, too!''

\switchcolumn

``啊,史密斯!''他说,``你也在这儿!''

\switchcolumn*

``What are you in for?''

\switchcolumn

``你怎么到这儿来了?''

\switchcolumn*

``To tell you the truth---'' He sat down awkwardly on the bench opposite
Winston. ``There is only one offense, is there not?'' he said.

\switchcolumn

``实话和你说——''他笨拙地在温斯顿对面的长凳上坐下来。``这儿只有一种罪名,不是吗?''

\switchcolumn*

``And have you committed it?''

\switchcolumn

``你犯了这罪吗?''

\switchcolumn*

``Apparently I have.''

\switchcolumn

``显然,我犯了。''

\switchcolumn*

He put a hand to his forehead and pressed his temples for a moment, as
though trying to remember something.

\switchcolumn

他将一只手放在额头,压了会儿太阳穴,好像在努力尝试记起什么。

\switchcolumn*

``These things happen,'' he began vaguely. ``I have been able to recall one
instance---a possible instance. It was an indiscretion, undoubtedly. We were
producing a definitive edition of the poems of Kipling. I allowed the word
`God' to remain at the end of a line. I could not help it!'' he added almost
indignantly, raising his face to look at Winston. ``It was impossible to
change the line. The rhyme was `rod.' Do you realize that there are only
twelve rhymes to `rod' in the entire language? For days I had racked my
brains. There \emph{was} no other rhyme.''

\switchcolumn

``这事的确会发生,''他含糊地说。``我想起一个例子——一个可能发生的例子。不用怀疑,那就是粗心大意。我们正在出版吉卜林诗集的最终版本。我保留了其中一句诗的最后一个单词`上帝'(god),我也没办法!''他抬起脸看着温斯顿,愤怒地补充道。``这行诗不可能改。它的韵脚是`棍子'(rod),你知道所有词汇里只有十二个词符合这个韵脚。我绞尽脑汁想了好几天,想不出其他的词。''

\switchcolumn*

The expression on his face changed. The annoyance passed out of it and
for a moment he looked almost pleased. A sort of intellectual warmth,
the joy of the pedant who has found out some useless fact, shone through
the dirt and scrubby hair.

\switchcolumn

他的表情发生了变化,恼怒的神情消失了,有那么一会儿他几乎是愉快的。他短而脏的毛发上闪烁着智慧的光芒,闪烁着书呆子发现没用的事实后的喜悦。

\switchcolumn*

``Has it ever occurred to you,'' he said, ``that the whole history of
English poetry has been determined by the fact that the English language
lacks rhymes?''

\switchcolumn

``你想过吗,''他说,``整个英国诗歌史都是由英语韵脚的稀少决定的?''

\switchcolumn*

No, that particular thought had never occurred to Winston. Nor, in the
circumstances, did it strike him as very important or interesting.

\switchcolumn

不,温斯顿从没想过这个问题。在这种情况下,他不觉得这个问题有多重要多有趣。

\switchcolumn*

``Do you know what time of day it is?'' he said.

\switchcolumn

``你知道现在是什么时间吗?''他问。

\switchcolumn*

Ampleforth looked startled again. ``I had hardly thought about it. They
arrested me---it could be two days ago---perhaps three.'' His eyes
flitted round the walls, as though he half expected to find a window
somewhere. ``There is no difference between night and day in this place.
I do not see how one can calculate the time.''

\switchcolumn

安普福斯又吃了一惊。``这我倒没怎么想过,他们大约在两天前,也可能是三天前逮捕了我。''他的眼睛在墙上扫来扫去,似乎想在上面找到窗户。``这里的白天与黑夜没什么区别。我不认为有人能算出时间。''

\switchcolumn*

They talked desultorily for some minutes, then, without apparent reason,
a yell from the telescreen bade them be silent. Winston sat quietly, his
hands crossed. Ampleforth, too large to sit in comfort on the narrow
bench, fidgeted from side to side, clasping his lank hands first round
one knee, then round the other. The telescreen barked at him to keep
still. Time passed. Twenty minutes, an hour---it was difficult to judge.
Once more there was a sound of boots outside. Winston\textquotesingle s
entrails contracted. Soon, very soon, perhaps in five minutes, perhaps
now, the tramp of boots would mean that his own turn had come.

\switchcolumn

他们东拉西扯地聊了几分钟,接着,电屏毫无预兆地发出呵斥,禁止他们讲话。温斯顿安静地坐着,双手交叉。安普福斯身材高大,在狭窄的长凳上坐得很不舒服,他焦躁地挪动着,瘦骨嶙峋的双手一会儿握在这个膝盖上,一会儿又握在另一个膝盖上。电屏冲着他大吼,让他待着别动。时间就这样过去了。二十分钟,一个小时——很难判断过了多久,外面又响起皮靴的声音,温斯顿的心缩了起来。快了,就快了,也许五分钟,也许就是现在。皮靴的踩踏声意味着就要轮到他了。

\switchcolumn*

The door opened. The cold-faced young officer stepped into the cell.
With a brief movement of the hand he indicated Ampleforth.

\switchcolumn

门开了,那个表情冷酷的年轻官员走进牢房。他指着安普福斯做了个简洁的手势。

\switchcolumn*

``Room 101,'' he said.

\switchcolumn

``101号房。''他说。

\switchcolumn*

Ampleforth marched clumsily out between the guards, his face vaguely
perturbed, but uncomprehending.

\switchcolumn

安普福斯被看守架着,踉踉跄跄地走了出去,脸上隐约有不安的神情,让人捉摸不透。

\switchcolumn*

What seemed like a long time passed. The pain in
Winston\textquotesingle s belly had revived. His mind sagged round and
round on the same track, like a ball falling again and again into the
same series of slots. He had only six thoughts. The pain in his belly; a
piece of bread; the blood and the screaming; O\textquotesingle Brien;
Julia; the razor blade. There was another spasm in his entrails; the
heavy boots were approaching. As the door opened, the wave of air that
it created brought in a powerful smell of cold sweat. Parsons walked
into the cell. He was wearing khaki shorts and a sports shirt.

\switchcolumn

似乎过了很久。温斯顿的腹部又疼了起来。他精神萎靡,他的思绪在同一条思路上来来回回,就好像一个球一而再地落在同一个凹槽里。他能想到的只有6件事:肚子疼、一片面包、鲜血和尖叫、奥布兰、朱莉亚、刀片。他的内脏又抽搐起来,沉重的皮靴声越来越近。门打开时,一股浓烈的汗臭钻了进来。帕森斯走进牢房,他穿着卡其布短裤和运动衫。

\switchcolumn*

This time Winston was startled into self-forgetfulness.

\switchcolumn

这次,温斯顿惊讶得忘记了自己的存在。

\switchcolumn*

``\emph{You} here!'' he said.

\switchcolumn

``你怎么在这里!''他说。

\switchcolumn*

Parsons gave Winston a glance in which there was neither interest nor
surprise, but only misery. He began walking jerkily up and down,
evidently unable to keep still. Each time he straightened his pudgy
knees it was apparent that they were trembling. His eyes had a
wide-open, staring look, as though he could not prevent himself from
gazing at something in the middle distance.

\switchcolumn

帕森斯漠然地看了温斯顿一眼,一点不惊讶,眼神里只有痛苦。他不安地走来走去,停不下来。每当他伸直短粗的腿,膝盖那里就会哆嗦。他的眼睛张得大大的,凝视着什么,就好像不能不注视不远的前方。

\switchcolumn*

``What are you in for?'' said Winston.

\switchcolumn

``你因为什么进来的?''温斯顿说。

\switchcolumn*

``Thoughtcrime!'' said Parsons, almost blubbering. The tone of his voice
implied at once a complete admission of his guilt and a sort of
incredulous horror that such a word could be applied to himself. He
paused opposite Winston and began eagerly appealing to him: ``You
don\textquotesingle t think they\textquotesingle ll shoot me, do you,
old chap? They don\textquotesingle t shoot you if you
haven\textquotesingle t actually done anything---only thoughts, which
you can\textquotesingle t help? I know they give you a fair hearing. Oh,
I trust them for that! They\textquotesingle ll know my record,
won\textquotesingle t they? \emph{You} know what kind of a chap I was.
Not a bad chap in my way. Not brainy, of course, but keen. I tried to do
my best for the Party, didn\textquotesingle t I? I\textquotesingle ll
get off with five years, don\textquotesingle t you think? Or even ten
years? A chap like me could make himself pretty useful in a labor camp.
They wouldn\textquotesingle t shoot me for going off the rails just
once?''

\switchcolumn

``思想罪!''帕森斯说,几乎是在抽泣。他说话的腔调表明他已经完全承认了自己的罪行,但他又很震惊,不敢相信这个词居然会用在自己身上。他面冲温斯顿,停下脚步,急切地对他说:``你不认为他们会枪毙我吧,是吗,老兄?他们不会枪毙你的,只要你实际上什么事都没做,除了想一想,这你可控制不了,不是吗?我知道他们会给你一个公正的申诉机会。噢,我相信他们会这么做!他们了解我的表现,不是吗?你知道我的为人,我人不坏。没脑子是真,但我很热情。我努力为党做到最好,难道不是吗?我会被关上五年,你觉得呢?还是说十年?我这种人在劳改营很有用。他们不会因为我的一次过失就枪毙我吧?''

\switchcolumn*

``Are you guilty?'' said Winston.

\switchcolumn

``你有罪吗?''温斯顿问。

\switchcolumn*

``Of course I\textquotesingle m guilty!'' cried Parsons with a servile
glance at the telescreen. ``You don\textquotesingle t think the Party
would arrest an innocent man, do you?'' His froglike face grew calmer,
and even took on a slightly sanctimonious expression. ``Thoughtcrime is a
dreadful thing, old man,'' he said sententiously. ``It\textquotesingle s
insidious. It can get hold of you without your even knowing it. Do you
know how it got hold of me? In my sleep! Yes, that\textquotesingle s a
fact. There I was, working away, trying to do my bit---never knew I had
any bad stuff in my mind at all. And then I started talking in my sleep.
Do you know what they heard me saying?''

\switchcolumn

``当然,我有罪!''帕森斯边哭边卑微地看了看电屏,``你不会以为党会逮捕无辜的人吧,对不对?''他青蛙一般的脸平静了一些,甚至还有几分神圣。``思想罪是可怕的,老兄,''他简洁地说,``它很阴险。它会在你不知道的情况下控制你。你知道它是怎样控制我的吗?在我睡觉的时候!没错,事实就是这样。我是这种人,工作努力,做什么事都尽自己的本分——从来都不知道自己的思想里有什么坏东西。后来我开始说梦话。你知道他们听到我说了什么吗?''

\switchcolumn*

He sank his voice, like someone who is obliged for medical reasons to
utter an obscenity.

\switchcolumn

他压低声音,就像某些人出于医学方面的考虑而被迫说脏话一样。

\switchcolumn*

```Down with Big Brother!' Yes, I said that! Said it over and over again, it
seems. Between you and me, old man, I\textquotesingle m glad they got me
before it went any further. Do you know what I\textquotesingle m going to
say to them when I go up before the tribunal? `Thank you,' I\textquotesingle
m going to say, `thank you for saving me before it was too
late.'{}''

\switchcolumn

``\,`打倒老大哥!'是的,我是这么说的,似乎说了一遍又一遍。老兄,这话只有咱们两人知道,我很庆幸,他们在我得寸进尺之前抓住了我。你知道到了法庭上我会对他们说些什么吗?我要说`谢谢你们,谢谢你们及时拯救了我。'\,''

\switchcolumn*

``Who denounced you?'' said Winston.

\switchcolumn

``是谁揭发的你?''温斯顿问。

\switchcolumn*

``It was my little daughter,'' said Parsons with a sort of doleful pride.
``She listened at the keyhole. Heard what I was saying, and nipped off to
the patrols the very next day. Pretty smart for a nipper of seven, eh? I
don\textquotesingle t bear her any grudge for it. In fact
I\textquotesingle m proud of her. It shows I brought her up in the right
spirit, anyway.''

\switchcolumn

``我的小女儿,''帕森斯带着一种又伤心又自豪的神情说。``她从锁孔里偷听到的。她听到我说了什么,第二天就向巡逻队告发了我。对一个七岁的孩子来说,她可真聪明,是不是?我一点儿也不怨她,事实上,我为她骄傲,不管怎样,这说明我用正确的思想教育了她。''

\switchcolumn*

He made a few more jerky movements up and down, several times casting a
longing glance at the lavatory pan. Then he suddenly ripped down his
shorts.

\switchcolumn

他又愚蠢地走来走去,他渴望地看了马桶好几眼,然后,他突然拉下裤子。

\switchcolumn*

``Excuse me, old man,'' he said. ``I can\textquotesingle t help it.
It\textquotesingle s the waiting.''

\switchcolumn

``不好意思,老兄,''他说,``我憋不住了,憋了好久了。''

\switchcolumn*

He plumped his large posteriors onto the lavatory pan. Winston covered
his face with his hands.

\switchcolumn

他的大屁股猛地一下坐到了马桶上。温斯顿用手遮住了脸。

\switchcolumn*

``Smith!'' yelled the voice from the telescreen. ``6079 Smith W! Uncover
your face. No faces covered in the cells.''

\switchcolumn

``史密斯!''电屏里传来吼叫声,``6079号史密斯!把脸露出来,牢房里不许遮脸。''

\switchcolumn*

Winston uncovered his face. Parsons used the lavatory, loudly and
abundantly. It then turned out that the plug was defective, and the cell
stank abominably for hours afterwards.

\switchcolumn

温斯顿把手挪开。帕森斯大声地、痛痛快快地用了马桶。然后才发现抽水装置不能用。牢房里令人恶心的臭气一连几个小时都挥之不去。

\switchcolumn*

Parsons was removed. More prisoners came and went mysteriously. One, a
woman, was consigned to ``Room 101,'' and, Winston noticed, seemed to
shrivel and turn a different color when she heard the words. A time came
when, if it had been morning when he was brought here, it would be
afternoon; or if it had been afternoon, then it would be midnight. There
were six prisoners in the cell, men and women. All sat very still.
Opposite Winston there sat a man with a chinless, toothy face exactly
like that of some large, harmless rodent. His fat, mottled cheeks were
so pouched at the bottom that it was difficult not to believe that he
had little stores of food tucked away there. His pale-gray eyes flitted
timorously from face to face, and turned quickly away again when he
caught anyone\textquotesingle s eye.

\switchcolumn

帕森斯被带走了,犯人们来来往往,非常神秘。一次,一个女犯人要被送往101号房。温斯顿注意到,她一听到这个词脸色就变了,人也缩了起来。当时——如果他是上午进来的,这件事就发生在下午;如果他是下午进来的,那就发生在半夜——牢房里有六个人,有男有女,所有人都一动不动地坐着。在温斯顿对面的,是一个没有下巴、牙齿外露的男人,长得很像一只巨大的无害的啮齿动物。他肥胖的双颊长满斑点,像袋子那样垂下来,那样子让人觉得他有在里面藏了吃的东西。他浅灰色的眼睛怯生生地扫视着别人,每当和人目光相交,他就会立刻将视线挪开。

\switchcolumn*

The door opened, and another prisoner was brought in whose appearance
sent a momentary chill through Winston. He was a commonplace,
mean-looking man who might have been an engineer or technician of some
kind. But what was startling was the emaciation of his face. It was like
a skull. Because of its thinness the mouth and eyes looked
disproportionately large, and the eyes seemed filled with a murderous,
unappeasable hatred of somebody or something.

\switchcolumn

门开了,又有犯人被带进来,他的样子让温斯顿心下一寒。他的样子很普通,有些猥琐,他可能是个工程师,也可能是某种技师。但他的脸却令人吃惊地消瘦,就像一个骷髅。由于消瘦,他的眼睛和嘴大得不成比例,且他的眼睛里还充满杀气,一种对某人或某物无法遏止的憎恨。

\switchcolumn*

The man sat down on the bench at a little distance from Winston. Winston
did not look at him again, but the tormented, skull-like face was as
vivid in his mind as though it had been straight in front of his eyes.
Suddenly he realized what was the matter. The man was dying of
starvation. The same thought seemed to occur almost simultaneously to
everyone in the cell. There was a very faint stirring all the way round
the bench. The eyes of the chinless man kept flitting toward the
skull-faced man, then turning guiltily away, then being dragged back by
an irresistible attraction. Presently he began to fidget on his seat. At
last he stood up, waddled clumsily across the cell, dug down into the
pocket of his overalls, and, with an abashed air, held out a grimy piece
of bread to the skull-faced man.

\switchcolumn

那人在温斯顿的附近坐了下来。温斯顿没再去看他,但在他的脑海中,那宛若骷髅的痛苦的脸却异常生动,似乎就摆在他眼前。他突然意识到这是怎么一回事。这个男人就要饿死了。牢房里的人几乎在同一时刻想到了这件事。长凳上出现一阵轻微的骚动。那个没有下巴的人一直在打量这个骷髅般的男人,他有些愧疚地移开目光,可之后他的目光又会被无法抗拒的吸引力拉回来。他开始坐不住了,终于他站起身,蹒跚地穿过牢房,他将一只手伸进制服的口袋,有些不好意思地拿出一片脏兮兮的面包,把它递给骷髅脸的人。

\switchcolumn*

There was a furious, deafening roar from the telescreen. The chinless
man jumped in his tracks. The skull-faced man had quickly thrust his
hands behind his back, as though demonstrating to all the world that he
refused the gift.

\switchcolumn

一个愤怒的、震耳欲聋的声音从电屏中传来,把没有下巴的人吓了一跳。骷髅脸的人立即将手背到身后,似乎向全世界证明他拒绝了这份礼物。

\switchcolumn*

``Bumstead!'' roared the voice. ``2713 Bumstead J! Let fall that piece of
bread.''

\switchcolumn

``巴姆斯蒂德!''电屏吼道,``2713号巴姆斯蒂德!把面包扔地上!''

\switchcolumn*

The chinless man dropped the piece of bread on the floor.

\switchcolumn

没下巴的人将那片面包放到了地上。

\switchcolumn*

``Remain standing where you are,'' said the voice. ``Face the door. Make no
movement.''

\switchcolumn

``站那儿别动,''那声音说,``面朝门,不许动!''

\switchcolumn*

The chinless man obeyed. His large pouchy cheeks were quivering
uncontrollably. The door clanged open. As the young officer entered and
stepped aside, there emerged from behind him a short stumpy guard with
enormous arms and shoulders. He took his stand opposite the chinless
man, and then, at a signal from the officer, let free a frightful blow,
with all the weight of his body behind it, full in the chinless
man\textquotesingle s mouth. The force of it seemed almost to knock him
clear of the floor. His body was flung across the cell and fetched up
against the base of the lavatory seat. For a moment he lay as though
stunned, with dark blood oozing from his mouth and nose. A very faint
whimpering or squeaking, which seemed unconscious, came out of him. Then
he rolled over and raised himself unsteadily on hands and knees. Amid a
stream of blood and saliva, the two halves of a dental plate fell out of
his mouth.

\switchcolumn

没下巴的人照做了。他又大又鼓的双颊不受控制地颤抖起来。咣当一声,门开了。年轻的官员走进来站到一边,从他身后闪过一个矮胖的、有着粗壮臂膀的看守。他站在没下巴的人的对面,然后在官员的示意下,用尽全身力气,狠狠地在没有下巴的人的嘴上打了一拳,力量大到让后者飞了起来。没下巴的人倒在了牢房的另一边,马桶的底座截住了他。有那么一阵,他躺在那里昏了过去,嘴巴和鼻子都流出深色的血。他不自觉地发出微弱地啜泣声,或者说是呻吟声。之后,他翻转过来,摇摇晃晃地用手和膝盖撑起身体,混着血和口水,吐出一副被打成两半的假牙。

\switchcolumn*

The prisoners sat very still, their hands crossed on their knees. The
chinless man climbed back into his place. Down one side of his face the
flesh was darkening. His mouth had swollen into a shapeless
cherry-colored mass with a black hole in the middle of it. From time to
time a little blood dripped onto the breast of his overalls. His gray
eyes still flitted from face to face, more guiltily than ever, as though
he were trying to discover how much the others despised him for his
humiliation.

\switchcolumn

犯人们仍一动不动地坐着,双手交叉放在膝上。没有下巴的人爬回原来的位置,他一边
脸的下半部分开始发青。嘴巴也肿得失去了形状,变成一团樱桃色的中间有黑洞的东西。
不时有血滴到他胸前的制服上。他灰色的眼睛仍在打量别人的脸,目光比之前更加惊惶,
就好像要弄明白因为这丢脸的行为其他人到底有多么瞧不起他。

\switchcolumn*

The door opened. With a small gesture the officer indicated the
skull-faced man.

\switchcolumn

门开了。那个官员向骷髅脸的人做了个不大的手势。

\switchcolumn*

``Room 101,'' he said.

\switchcolumn

``101号房。''他说。

\switchcolumn*

There was a gasp and a flurry at Winston\textquotesingle s side. The man
had actually flung himself on his knees on the floor, with his hands
clasped together.

\switchcolumn

温斯顿的身边有人吸了口气,一阵不安。那个男人猛地跪倒在地板上,双手紧紧地握在一起。

\switchcolumn*

``Comrade! Officer!'' he cried. ``You don\textquotesingle t have to take me
to that place! Haven\textquotesingle t I told you everything already?
What else is it you want to know? There\textquotesingle s nothing I
wouldn\textquotesingle t confess, nothing! Just tell me what it is and
I\textquotesingle ll confess it straight off. Write it down and
I\textquotesingle ll sign it---anything! Not room 101!''

\switchcolumn

``同志!长官!''他哭喊着,``别送我去那儿了!我已经将一切都告诉你们了,不是吗?你们还想知道什么?我都招了,什么都招了!只要告诉我你们想知道什么,我就把一切都告诉你们。把它们写下来,我会签字的——做什么都行!除了101号房!''

\switchcolumn*

``Room 101,'' said the officer.

\switchcolumn

``101号房。''那个官员说。

\switchcolumn*

The man\textquotesingle s face, already very pale, turned a color
Winston would not have believed possible. It was definitely,
unmistakably, a shade of green.

\switchcolumn

此时,男人原本十分苍白的脸变成了温斯顿不敢相信的绿色,这点毫无疑问。

\switchcolumn*

``Do anything to me!'' he yelled. ``You\textquotesingle ve been starving me
for weeks. Finish it off and let me die. Shoot me. Hang me. Sentence me
to twenty-five years. Is there somebody else you want me to give away?
Just say who it is and I\textquotesingle ll tell you anything you want.
I don\textquotesingle t care who it is or what you do to them.
I\textquotesingle ve got a wife and three children. The biggest of them
isn\textquotesingle t six years old. You can take the whole lot of them
and cut their throats in front of my eyes, and I\textquotesingle ll
stand by and watch it. But not room 101!''

\switchcolumn

``对我做什么都行!''他大叫,``你们饿了我几个星期了。就饿死我吧。枪毙我、吊死我、关我个二十五年。还需要我出卖谁吗?只要说出他是谁,我就把你们想知道的统统说出来。我不关心他是谁,也随便你们怎样对付他。我有妻子和三个孩子,最大的还不到六岁。你们可以把他们都带来,当着我的面割断他们的喉咙,我就站在这儿看。但是,千万别带我去101号房间!''

\switchcolumn*

``Room 101,'' said the officer.

\switchcolumn

``101号房!''官员说。

\switchcolumn*

The man looked frantically round at the other prisoners, as though with
some idea that he could put another victim in his own place. His eyes
settled on the smashed face of the chinless man. He flung out a lean
arm.

\switchcolumn

那人疯子一般地扫视了其他的犯人,就好像要让什么人作他的替死鬼。他将目光停留在没有下巴的人那张被打成重伤的脸上。突然,他举起了瘦巴巴的手。

\switchcolumn*

``That\textquotesingle s the one you ought to be taking, not me!'' he
shouted. ``You didn\textquotesingle t hear what he was saying after they
bashed his face. Give me a chance and I\textquotesingle ll tell you
every word of it. \emph{He\textquotesingle s} the one
that\textquotesingle s against the Party, not me.'' The guards stepped
forward. The man\textquotesingle s voice rose to a shriek. ``You
didn\textquotesingle t hear him!'' he repeated. ``Something went wrong
with the telescreen. \emph{He\textquotesingle s} the one you want. Take
him, not me!''

\switchcolumn

``应该带这个人走,不是我!''他大喊,``你们没听见他被打后说了什么。给我个机会吧,他说的每个字我都会告诉你们。反党的是他,不是我。''见看守向前迈了一步,那人尖叫起来。``你们没听见他说的话!''他重复道,``电屏出毛病了。他才是你们想要的人,带他走,不是我!''

\switchcolumn*

The two sturdy guards had stooped to take him by the arms. But just at
this moment he flung himself across the floor of the cell and grabbed
one of the iron legs that supported the bench. He had set up a wordless
howling, like an animal. The guards took hold of him to wrench him
loose, but he clung on with astonishing strength. For perhaps twenty
seconds they were hauling at him. The prisoners sat quiet, their hands
crossed on their knees, looking straight in front of them. The howling
stopped; the man had no breath left for anything except hanging on. Then
there was a different kind of cry. A kick from a guard\textquotesingle s
boot had broken the fingers of one of his hands. They dragged him to his
feet.

\switchcolumn

两个身材健硕的看守弯下腰,抓住他的手臂。但就在这时,他扑倒在牢房的地板上,抓住靠墙而设的长凳的铁腿,像野兽那样号叫起来。看守揪扯着他的身子,想把他拽开。但他死死地抓着,力气大得惊人。大约有二十秒,他们一直在拉他。其他犯人都安静地坐着,双手交叉放于膝上,直视前方。号叫停止了,男人憋住呼吸,除了抓住板凳腿,他已没有力气做别的事。接着,他发出和之前截然不同的哭声。有个警卫用皮靴踢断了他的手指,他们将他拖了起来。

\switchcolumn*

``Room 101,'' said the officer.

\switchcolumn

``101号房。''官员说。

\switchcolumn*

The man was led out, walking unsteadily, with head sunken, nursing his
crushed hand, all the fight gone out of him.

\switchcolumn

男人被带走了,他低垂着头,步履蹒跚,护着那只被踢伤的手,不再做任何抵抗。

\switchcolumn*

A long time passed. If it had been midnight when, the skull-faced man
was taken away, it was morning; if morning, it was afternoon. Winston
was alone, and had been alone for hours. The pain of sitting on the
narrow bench was such that often he got up and walked about, unreproved
by the telescreen. The piece of bread still lay where the chinless man
had dropped it. At the beginning it needed a hard effort not to look at
it, but presently hunger gave way to thirst. His mouth was sticky and
evil-tasting. The humming sound and the unvarying white light induced a
sort of faintness, an empty feeling inside his head. He would get up
because the ache in his bones was no longer bearable, and then would sit
down again almost at once because he was too dizzy to make sure of
staying on his feet. Whenever his physical sensations were a little
under control the terror returned. Sometimes with a fading hope he
thought of O\textquotesingle Brien and the razor blade. It was thinkable
that the razor blade might arrive concealed in his food, if he were ever
fed. More dimly he thought of Julia. Somewhere or other she was
suffering, perhaps far worse than he. She might be screaming with pain
at this moment. He thought: ``If I could save Julia by doubling my own
pain, would I do it? Yes, I would.'' But that was merely an intellectual
decision, taken because he knew that he ought to take it. He did not
feel it. In this place you could not feel anything, except pain and the
foreknowledge of pain. Besides, was it possible, when you were actually
suffering it, to wish for any reason whatever that your own pain should
increase? But that question was not answerable yet.

\switchcolumn

很长时间过去了。若这形如骷髅的男子是在午夜时分被带走,那么现在便是早上。若他在早上被带走,现在就是下午。温斯顿已经有好几个小时都独自一人待在那里。狭窄的板凳硌得他很疼,他不时就要站起来走上一会儿,电屏倒没有因为这个呵斥他。那片面包仍然留在没下巴的人丢下它的地方。起初,他很难不去看它,但没过多久,饥饿就被干渴取代。他的嘴巴黏糊糊的,散发着臭味。嗡嗡的声响和单调的白色灯光让他感到几分眩晕,脑袋也变得空洞。由于无法忍受刻骨的疼痛,他想站起身,然而紧接着他又不得不坐下,他的头太晕了,以至于他无法立住脚跟。每当他稍稍控制住身体的感觉,恐惧之感就会卷土重来。有时,他有些侥幸地想着奥布兰的刀片。如果提供他吃的东西,可以想象它就藏在他的食物里。他昏昏沉沉地想起朱莉亚,也许她正在某个地方承受比他还要剧烈的痛苦。此时此刻,她很有可能疼得高声尖叫。他想:``若把我的痛苦增加一倍就可以解救朱莉亚,我会愿意吗?会的,我会。''但这仅仅是理智上的决定,因为他知道他应该这么做。而感觉上,他不想如此。在这里,除了疼痛,你别无他感。此外,当你实实在在地承受疼痛,不管原因如何,你有可能真心希望疼痛再增加一些吗?他还无法回答这个问题。

\switchcolumn*

The boots were approaching again. The door opened.
O\textquotesingle Brien came in.

\switchcolumn

皮靴的声音又一次由远及近。门开了,奥布兰走了进来。

\switchcolumn*

Winston started to his feet. The shock of the sight had driven all
caution out of him. For the first time in many years he forgot the
presence of the telescreen.

\switchcolumn

温斯顿跳了起来,震惊得将所有戒备都抛至脑后。这是许多年来,他第一次忘记电屏的存在。

\switchcolumn*

``They\textquotesingle ve got you too!'' he cried.

\switchcolumn

``他们也抓到你了!''他叫道。

\switchcolumn*

``They got me a long time ago,'' said O\textquotesingle Brien with a mild,
almost regretful irony. He stepped aside. From behind him there emerged
a broad-chested guard with a long black truncheon in his hand.

\switchcolumn

``他们很久以前就抓到我了。''奥布兰平静又多少有些遗憾地讽刺道。他向旁边一闪,从他身后冒出一个胸部硕大、拎着黑色长棍的看守。

\switchcolumn*

``You knew this, Winston,'' said O\textquotesingle Brien.
``Don\textquotesingle t deceive yourself. You did know it---you have
always known it.''

\switchcolumn

``你明白的,温斯顿,''奥布兰说。``不要再欺骗自己了。你明白的,你一直都明白。''

\switchcolumn*

Yes, he saw now, he had always known it. But there was no time to think
of that. All he had eyes for was the truncheon in the
guard\textquotesingle s hand. It might fall anywhere: on the crown, on
the tip of the ear, on the upper arm, on the elbow---

\switchcolumn

没错,现在他懂了,他一直都懂。但他没有时间考虑这些,他的眼里只有看守的警棍。它可能落在任何地方:他的头顶、他的耳朵尖、他胳膊的上端、他的肘关节——

\switchcolumn*

The elbow! He had slumped to his knees, almost paralyzed, clasping the
stricken elbow with his other hand. Everything had exploded into yellow
light. Inconceivable, inconceivable that one blow could cause such pain!
The light cleared and he could see the other two looking down at him.
The guard was laughing at his contortions. One question at any rate was
answered. Never, for any reason on earth, could you wish for an increase
of pain. Of pain you could wish only one thing: that it should stop.
Nothing in the world was so bad as physical pain. In the face of pain
there are no heroes, no heroes, he thought over and over as he writhed
on the floor, clutching uselessly at his disabled left arm.

\switchcolumn

是手肘!他猛地跪了下去,几乎瘫痪,用另一只手抱住了挨打的手肘。周围的一切都炸成了黄色的光。不可思议,简直不可思议,不过打了一下就疼成这样!视线逐渐清晰,他能看见还有两个人正俯视着他。看守嘲笑着他扭曲的身体。至少有一个问题有了答案。不管有怎样的理由,你都不会希望疼痛加剧。对疼痛,你只有一个念头:让它停止。世界上没有比肉体的疼痛更糟糕的事了。疼痛面前无英雄,无英雄。他一边反复想着,一边徒劳地抱着被打伤的左臂,在地板上翻滚。

\switchcolumn*


\section{II}\label{ii-2}

\switchcolumn

\section*{二}\label{ux4e8cux5341}

\switchcolumn*

He was lying on something that felt like a camp bed, except that it was
higher off the ground and that he was fixed down in some way so that he
could not move. Light that seemed stronger than usual was falling on his
face. O\textquotesingle Brien was standing at his side, looking down at
him intently. At the other side of him stood a man in a white coat,
holding a hypodermic syringe.

\switchcolumn

他躺在类似行军床又比行军床高一些的东西上,身体被绑住,动弹不得。光照着他的脸,比平时的要强烈一些。奥布兰站在他的一边,正低着头专注地看着他。他的另一边则站着一个穿着白大褂、拿着注射器的男人。

\switchcolumn*

Even after his eyes were open he took in his surroundings only
gradually. He had the impression of swimming up into this room from some
quite different world, a sort of underwater world far beneath it. How
long he had been down there he did not know. Since the moment when they
arrested him he had not seen darkness or daylight. Besides, his memories
were not continuous. There had been times when consciousness, even the
sort of consciousness that one has in sleep, had stopped dead and
started again after a blank interval. But whether the intervals were of
days or weeks or only seconds, there was no way of knowing.

\switchcolumn

他张开眼睛,逐渐看清周围的环境。他觉得自己好像是从一个完全不同的、深邃的海底世界里游进了这个房间。他不知道自己在这儿待了多久。从他们抓住他算起,他就没见过白天和黑夜。他的记忆断断续续,他的意识——有时死一般地停住,即便是在睡眠中也是如此,要经过一段空白时期才重新恢复。但这究竟需要几天、几星期,还是几秒,他就无从得知了。

\switchcolumn*

With that first blow on the elbow the nightmare had started. Later he
was to realize that all that then happened was merely a preliminary, a
routine interrogation to which nearly all prisoners were subjected.
There was a long range of crimes---espionage, sabotage, and the
like---to which everyone had to confess as a matter of course. The
confession was a formality, though the torture was real. How many times
he had been beaten, how long the beatings had continued, he could not
remember. Always there were five or six men in black uniforms at him
simultaneously. Sometimes it was fists, sometimes it was truncheons,
sometimes it was steel rods, sometimes it was boots. There were times
when he rolled about the floor, as shameless as an animal, writhing his
body this way and that in an endless, hopeless effort to dodge the
kicks, and simply inviting more and yet more kicks, in his ribs, in his
belly, on his elbows, on his shins, in his groin, in his testicles, on
the bone at the base of his spine. There were times when it went on and
on until the cruel, wicked, unforgivable thing seemed to him not that
the guards continued to beat him but that he could not force himself
into losing consciousness. There were times when his nerve so forsook
him that he began shouting for mercy even before the beating began, when
the mere sight of a fist drawn back for a blow was enough to make him
pour forth a confession of real and imaginary crimes. There were other
times when he started out with the resolve of confessing nothing, when
every word had to be forced out of him between gasps of pain, and there
were times when he feebly tried to compromise, when he said to himself:
``I will confess, but not yet. I must hold out till the pain becomes
unbearable. Three more kicks, two more kicks, and then I will tell them
what they want.'' Sometimes he was beaten till he could hardly stand,
then flung like a sack of potatoes onto the stone floor of a cell, left
to recuperate for a few hours, and then taken out and beaten again.
There were also longer periods of recovery. He remembered them dimly,
because they were spent chiefly in sleep or stupor. He remembered a cell
with a plank bed, a sort of shelf sticking out from the wall, and a tin
washbasin, and meals of hot soup and bread and sometimes coffee. He
remembered a surly barber arriving to scrape his chin and crop his hair,
and businesslike, unsympathetic men in white coats feeling his pulse,
tapping his reflexes, turning up his eyelids, running harsh fingers over
him in search of broken bones, and shooting needles into his arm to make
him sleep.

\switchcolumn

自手肘遭到第一下重击后,他的噩梦就开始了。后来,他才明白当时发生的所有事情都仅仅是个前奏,是几乎所有囚犯都会经历的例行公事般的审讯。每个人都理所当然地要承认诸如间谍、破坏等范围甚广的罪行。招供无非是个形式,拷打却是实实在在的。他已经记不清被打了多少次、每次有多久。总会有五六个身穿黑色制服的男人同时殴打他,有时用拳头,有时用警棍,有时用钢条,有时用皮靴。有那么几次,他被打得满地翻滚,活像寡廉鲜耻的牲畜,他扭动着身体,试图躲开拳打脚踢,但没有用,相反还会招致更多的踢打。他们踢在他的肋骨上、肚子上、关节上、腰上、腿上、腹沟上、睾丸上、尾椎上。很多时候,他觉得最残酷、最邪恶、最不可原谅的,不是看守们没完没了的殴打,而是他居然无法令自己丧失意识。有时,他紧张到在遭到殴打前就大喊大叫地请求饶恕,单单是拳头后扬准备出击的动作就足够让他供出一堆或真或假的罪行。有时,他决定什么都不招,只在疼得吸气的时候被迫说出只言片语。还有时,他想软弱地妥协,对自己说:``我会招供,但不是现在。我必须坚持到疼得无法忍受的时候。再踢上三脚,再踢上两脚,然后我再说他们想听的话。''有几次,他被打得站立不稳,像一袋土豆那样被扔到牢房的石头地板上,休息了几个小时,就又被带出去拷打。也有几次时间间隔得挺长,但他记忆模糊,因为他不是在睡梦中就是在昏迷中。他记得某间牢房里有一张木板床,一个安在墙上的架子和一个洗脸盆,他记得饭里有热汤和面包,偶尔还有咖啡。他记得有个粗声粗气的理发师给他刮了胡子,剪了头发,一些态度冷漠的、没有同情心的白衣人测试了他的脉搏、检查了他的反应、翻看了他的眼皮,用粗糙的手指在他身上摸来摸去查看他是否有骨折,还在他的胳膊上打了帮助他入睡的针。

\switchcolumn*

The beatings grew less frequent, and became mainly a threat, a horror to
which he could be sent back at any moment when his answers were
unsatisfactory. His questioners now were not ruffians in black uniforms
but Party intellectuals, little rotund men with quick movements and
flashing spectacles, who worked on him in relays over periods which
lasted---he thought, he could not be sure---ten or twelve hours at a
stretch. These other questioners saw to it that he was in constant
slight pain, but it was not chiefly pain that they relied on. They
slapped his face, wrung his ears, pulled his hair, made him stand on one
leg, refused him leave to urinate, shone glaring lights in his face
until his eyes ran with water; but the aim of this was simply to
humiliate him and destroy his power of arguing and reasoning. Their real
weapon was the merciless questioning that went on and on hour after
hour, tripping him up, laying traps for him, twisting everything that he
said, convicting him at every step of lies and self-contradiction, until
he began weeping as much from shame as from nervous fatigue. Sometimes
he would weep half a dozen times in a single session. Most of the time
they screamed abuse at him and threatened at every hesitation to deliver
him over to the guards again; but sometimes they would suddenly change
their tune, call him comrade, appeal to him in the name of Ingsoc and
Big Brother, and ask him sorrowfully whether even now he had not enough
loyalty to the Party left to make him wish to undo the evil he had done.
When his nerves were in rags after hours of questioning, even this
appeal could reduce him to sniveling tears. In the end the nagging
voices broke him down more completely than the boots and fists of the
guards. He became simply a mouth that uttered, a hand that signed
whatever was demanded of him. His sole concern was to find out what they
wanted him to confess, and then confess it quickly, before the bullying
started anew. He confessed to the assassination of eminent Party
members, the distribution of seditious pamphlets, embezzlement of public
funds, sale of military secrets, sabotage of every kind. He confessed
that he had been a spy in the pay of the Eastasian government as far
back as 1968. He confessed that he was a religious believer, an admirer
of capitalism, and a sexual pervert. He confessed that he had murdered
his wife, although he knew, and his questioners must have known, that
his wife was still alive. He confessed that for years he had been in
personal touch with Goldstein and had been a member of an underground
organization which had included almost every human being he had ever
known. It was easier to confess everything and implicate everybody.
Besides, in a sense it was all true. It was true that he had been the
enemy of the Party, and in the eyes of the Party there was no
distinction between the thought and the deed.

\switchcolumn

拷打没有那么频繁了,它变成了一种威胁,一种只要他不能给出满意答案便随时会遭到殴打的恐惧。现在,审问他的已不再是身穿黑色制服的恶棍,而是党的知识分子,一群行动敏捷、戴着眼镜的矮胖子。他们隔一段时间换一次班——他想,他不能确定——有一班竟持续了十或十二个小时。这群拷问者不时便让他吃一些小苦头,但他们的目的并不是制造疼痛。他们抽打他的脸,拧他的耳朵,揪扯他的头发,让他用一条腿站着,不让他小便。他们用刺眼的光照射他的脸,直到他流出眼泪。但这不过是让他感到屈辱,摧毁他争辩、讲理的能力。他们真正的武器是一个小时接一个小时、毫不停歇地向他提问,让他说错话,让他堕入陷阱,歪曲他讲的每件事,证明他所讲的都是自相矛盾的谎言,直到他因为羞愧和精神疲惫失声痛哭。有时,一次审问他就要哭上六次。大多时候,他们都扯着喉咙辱骂他,只要他稍有迟疑,就威胁将他重新交给看守。但有时,他们也会突然改变腔调,称呼他``同志'',以英社和老大哥的名义祈求他,悲悲戚戚地问他是不是对党足够忠诚,想不想刷清自己的罪恶。在经过几个小时的审问后,他的精神已濒临崩溃,即便是这样的祈求也能让他哭哭啼啼。最终,这些啰啰嗦嗦地问话比看守的拳脚更能将他彻底击垮。他成了简简单单一张嘴,要他说什么他就说什么,他成了简简单单一只手,让他签什么他就签什么。他只想知道他们究竟想让他供认什么,然后他就可以赶在恐吓到来之前,快速地供认出来。他承认他暗杀了党的杰出成员,他承认他散发了煽动性的小册子,他承认侵吞公款、出卖军事机密以及进行各式各样的破坏活动,还承认早在1968年,他就做了东亚国的间谍。他坦陈自己信仰宗教、杀害妻子——尽管他和审问他的人都明白,他的妻子还活着。他说他和高德斯坦因的私人往来以持续了很多年,他说自己是地下组织的成员,并说该组织吸纳了几乎所有他认识的人。坦白所有事情,连累所有的人,这些都相对容易。再说,从某种角度看,这的确是事实。他是党的敌人,在党眼中,思想上的敌人和行为上的没有差别。

\switchcolumn*

There were also memories of another kind. They stood out in his mind
disconnectedly, like pictures with blackness all round them.

\switchcolumn

他还有另外一种记忆,这些记忆断断续续地出现在他的脑海中,就像一幅幅被黑暗包围的图片。

\switchcolumn*

He was in a cell which might have been either dark or light, because he
could see nothing except a pair of eyes. Near at hand some kind of
instrument was ticking slowly and regularly. The eyes grew larger and
more luminous. Suddenly he floated out of his seat, dived into the eyes,
and was swallowed up.

\switchcolumn

他被关在一间牢房里,不知那里是开着灯,还是关着灯,因为除了一双眼睛,他什么都看不见。在他的手边,有一个滴答作响的、慢却准确的仪器。那双眼睛越来越大,越来越亮。突然,他从座位上飞了起来,坠到了眼睛里,被眼睛吞没。

\switchcolumn*

He was strapped into a chair surrounded by dials, under dazzling lights.
A man in a white coat was reading the dials. There was a tramp of heavy
boots outside. The door clanged open. The waxen-faced officer marched
in, followed by two guards.

\switchcolumn

他被绑在一张椅子上,椅子的周围布满了刻度盘。灯亮得刺眼,一个白衣男子看着刻度盘。门外响起了沉重的脚步声。门开了,那个长着蜡像脸的官员走了进来,身后还跟着两个看守。

\switchcolumn*

``Room 101,'' said the officer.

\switchcolumn

``101号房。''官员说。

\switchcolumn*

The man in the white coat did not turn round. He did not look at Winston
either; he was looking only at the dials.

\switchcolumn

白衣服的人既没有转身,也没有看温斯顿,只一心盯着刻度盘。

\switchcolumn*

He was rolling down a mighty corridor, a kilometer wide, full of
glorious, golden light, roaring with laughter and shouting out
confessions at the top of his voice. He was confessing everything, even
the things he had succeeded in holding back under the torture. He was
relating the entire history of his life to an audience who knew it
already. With him were the guards, the other questioners, the men in
white coats, O\textquotesingle Brien, Julia, Mr. Charrington, all
rolling down the corridor together and shouting with laughter. Some
dreadful thing which had lain embedded in the future had somehow been
skipped over and had not happened. Everything was all right, there was
no more pain, the last detail of his life was laid bare, understood,
forgiven.

\switchcolumn

温斯顿被推进一条足有一公里宽的、巨大的走廊,走廊里充满了灿烂的金色光芒,他笑着,喊着,用最大的声音招了供。他什么都招,甚至把严刑之下都坚持不说的话都说出来了。他将他的整个人生都告诉给这些早就知晓一切的听众。看守、其他审讯人员、穿白衣服的人、朱莉亚、查林顿先生,都和他在一起,都在走廊里大声地喊叫着、笑着。一些潜藏于未来的可怕之事被跳了过去,未能发生。每件事都很顺利,痛苦不再,他生命中的所有细枝末节都赤裸裸地呈现出来,得到了理解和宽恕。

\switchcolumn*

He was starting up from the plank bed in the half-certainty that he had
heard O\textquotesingle Brien\textquotesingle s voice. All through his
interrogation, although he had never seen him, he had had the feeling
that O\textquotesingle Brien was at his elbow, just out of sight. It was
O\textquotesingle Brien who was directing everything. It was he who set
the guards onto Winston and who prevented them from killing him. It was
he who decided when Winston should scream with pain, when he should have
a respite, when he should be fed, when he should sleep, when the drugs
should be pumped into his arm. It was he who asked the questions and
suggested the answers. He was the tormentor, he was the protector, he
was the inquisitor, he was the friend. And once---Winston could not
remember whether it was in drugged sleep, or in normal sleep, or even in
a moment of wakefulness---a voice murmured in his ear:
``Don\textquotesingle t worry, Winston; you are in my keeping. For seven
years I have watched over you. Now the turning point has come. I shall
save you, I shall make you perfect.'' He was not sure whether it was
O\textquotesingle Brien\textquotesingle s voice; but it was the same
voice that had said to him, ``We shall meet in the place where there is
no darkness,'' in that other dream, seven years ago.

\switchcolumn

他从木板床起来,恍惚听到奥布兰的声音。虽然在整个审讯过程中,他都没有看到奥布兰,但他有种感觉,觉得奥布兰就在他的身边,只是看不见罢了。奥布兰正是指挥这些事情的人。他命令看守打他,又不让他们将他打死。是他,决定温斯顿什么时候应该疼得尖叫,什么时候需要恢复,是他,决定什么时候让他吃饭,什么时候让他睡觉,什么时候把药物注射到他的身体里。是他,提出问题并暗示答案。他是折磨者、是保护者、是审讯者,他也是朋友。一次——温斯顿想不起来是在药物的作用下睡去的,还是自然睡着的,也有可能根本就没有睡着——有人在他耳旁低语:``别怕,温斯顿;你正在我的照看下。我观察了你整整七年。现在到了转折点了。我要拯救你,我要让你变得完美。''他不能确定这是不是奥布兰的声音,但这和七年之前的那个梦里,对他说``我们将在没有黑暗的地方相见''的声音是一样的。

\switchcolumn*

He did not remember any ending to his interrogation. There was a period
of blackness and then the cell, or room, in which he now was had
gradually materialized round him. He was almost flat on his back, and
unable to move. His body was held down at every essential point. Even
the back of his head was gripped in some manner. O\textquotesingle Brien
was looking down at him gravely and rather sadly. His face, seen from
below, looked coarse and worn, with pouches under the eyes and tired
lines from nose to chin. He was older than Winston had thought him; he
was perhaps forty-eight or fifty. Under his hand there was a dial with a
lever on top and figures running round the face.

\switchcolumn

他一点都不记得审讯是怎样结束的。在经历了一段时间的黑暗后,他来到现在所在的牢房,或者说``房间'',他慢慢地看清了周围的景物。他一直仰面平躺着,不能移动。他身体的每个重要部位都被绑住了,甚至后脑勺都被什么东西紧紧地固定住。奥布兰严肃又忧伤地俯视着他。从下面看他的脸,会发现他皮肤粗糙,神情憔悴,他的眼睛下有眼袋,因为疲惫从鼻子到下巴都长着皱纹。他比温斯顿想的要老,大概有四十八或五十岁。在他的手下,有一个带着控制杆的刻度盘,盘面上有一圈数字。

\switchcolumn*

``I told you,'' said O\textquotesingle Brien, ``that if we met again it
would be here.''

\switchcolumn

``我和你说过,''奥布兰说,``假如我们会再次相见,就是在这里。''

\switchcolumn*

``Yes,'' said Winston.

\switchcolumn

``是的。''温斯顿说。

\switchcolumn*

Without any warning except a slight movement of
O\textquotesingle Brien\textquotesingle s hand, a wave of pain flooded
his body. It was a frightening pain, because he could not see what was
happening, and he had the feeling that some mortal injury was being done
to him. He did not know whether the thing was really happening, or
whether the effect was electrically produced; but his body was being
wrenched out of shape, the joints were being slowly torn apart. Although
the pain had brought the sweat out on his forehead, the worst of all was
the fear that his backbone was about to snap. He set his teeth and
breathed hard through his nose, trying to keep silent as long as
possible.

\switchcolumn

奥布兰的手没有任何预兆地轻轻地动了一下,疼痛充斥了温斯顿的身体。这是一种令人感到恐惧的疼痛,因为他看不见到底发生了什么,只觉得自己受到了致命的伤害。他不知道伤害是真的造成了,还是电流造成的痛苦。他的身体扭曲得变了形,关节也被缓慢地扯开。疼痛让他的额头渗出汗珠,最糟糕的是,他担心自己的脊椎骨会断掉。他咬紧牙齿,艰难地用鼻子呼吸,尽可能地保持沉默。

\switchcolumn*

``You are afraid,'' said O\textquotesingle Brien, watching his face, ``that
in another moment something is going to break. Your especial fear is
that it will be your backbone. You have a vivid mental picture of the
vertebrae snapping apart and the spinal fluid dripping out of them. That
is what you are thinking, is it not, Winston?''

\switchcolumn

``你怕了,''奥布兰说,他看着他的脸,``再过一会儿就会有什么东西断掉了。你特别害怕它发生在你的脊梁骨上。你可以很生动地想象你的脊椎爆裂,脊髓流淌。你是这样想的,不是吗?温斯顿?''

\switchcolumn*

Winston did not answer. O\textquotesingle Brien drew back the lever on
the dial. The wave of pain receded almost as quickly as it had come.

\switchcolumn

温斯顿没有回答。奥布兰把控制盘上的杆子拉了回来。疼痛很快便消失了,几乎和它来时一样快。

\switchcolumn*

``That was forty,'' said O\textquotesingle Brien. ``You can see that the
numbers on this dial run up to a hundred. Will you please remember,
throughout our conversation, that I have it in my power to inflict pain
on you at any moment and to whatever degree I choose. If you tell me any
lies, or attempt to prevaricate in any way, or even fall below your
usual level of intelligence, you will cry out with pain, instantly. Do
you understand that?''

\switchcolumn

``只有40,''奥布兰说。``你能看见控制盘的数字最高是100。希望你在整个谈话过程中都记得这点,任何时候,我都有能力折磨你,想让你多疼就让你多疼。如果你对我说谎,或者试图以任何方式搪塞我,又或者你的表现低于你平时的智力水平,你都会马上疼得叫出声来,你明白我的意思吗?''

\switchcolumn*

``Yes,'' said Winston.

\switchcolumn

``明白。''温斯顿说。

\switchcolumn*

O\textquotesingle Brien\textquotesingle s manner became less severe. He
resettled his spectacles thoughtfully, and took a pace or two up and
down. When he spoke his voice was gentle and patient. He had the air of
a doctor, a teacher, even a priest, anxious to explain and persuade
rather than to punish.

\switchcolumn

奥布兰的态度不再那么严厉。他一边思考着什么,一边正了正眼镜,来回走了一两步。再开口时,他的声音变得温和而有耐心。他表现出一种医生、教师甚至是牧师的气质,只想解释说服,不想惩罚。

\switchcolumn*

``I am taking trouble with you, Winston,'' he said, ``because you are worth
trouble. You know perfectly well what is the matter with you. You have
known it for years, though you have fought against the knowledge. You
are mentally deranged. You suffer from a defective memory. You are
unable to remember real events, and you persuade yourself that you
remember other events which never happened. Fortunately it is curable.
You have never cured yourself of it, because you did not choose to.
There was a small effort of the will that you were not ready to make.
Even now, I am well aware, you are clinging to your disease under the
impression that it is a virtue. Now we will take an example. At this
moment, which power is Oceania at war with?''

\switchcolumn

``我真为你发愁,温斯顿,''他说,``因为你值得。你非常清楚你的问题在哪儿。很多年前你就知道了,但你不肯承认。你精神错乱,记忆方面有缺陷。你记不住真正发生的事,却说服自己记住没发生的事。幸运的是,这是可以治疗的,可你从来没想过将自己治好,因为你不愿意。只要在意志上稍微做些努力就可以,但你偏偏不准备这么做。即使是现在,我也清楚,你仍然坚持着这个毛病,还把它当成一种美德。现在,让我们举个例子吧。此刻,大洋国正在和哪个国家打仗?''

\switchcolumn*

``When I was arrested, Oceania was at war with Eastasia.''

\switchcolumn

``我被抓的时候,大洋国在和东亚国交战。''

\switchcolumn*

``With Eastasia. Good. And Oceania has always been at war with Eastasia,
has it not?''

\switchcolumn

``东亚国,好的。大洋国一直在和东亚国打仗,是不是?''

\switchcolumn*

Winston drew in his breath. He opened his mouth to speak and then did
not speak. He could not take his eyes away from the dial.

\switchcolumn

温斯顿吸了一口气,他张开嘴巴想说点什么,但他什么都没说。他无法将目光从刻度盘上挪开。

\switchcolumn*

``The truth, please, Winston. \emph{Your} truth. Tell me what you think
you remember.''

\switchcolumn

``请说实话,温斯顿。你要说实话,告诉我,你觉得你记得什么?''

\switchcolumn*

``I remember that until only a week before I was arrested, we were not at
war with Eastasia at all. We were in alliance with them. The war was
against Eurasia. That had lasted for four years. Before that---''

\switchcolumn

``我记得就在我被抓的一个星期前,我们还没有和东亚国交战。我们和他们结盟,在和欧亚国打仗。战争进行了四年。在这之前——''

\switchcolumn*

O\textquotesingle Brien stopped him with a movement of the hand.

\switchcolumn

奥布兰挥挥手,让他停下。

\switchcolumn*

``Another example,'' he said. ``Some years ago you had a very serious
delusion indeed. You believed that three men, three one-time Party
members named Jones, Aaronson, and Rutherford---men who were executed
for treachery and sabotage after making the fullest possible
confession---were not guilty of the crimes they were charged with. You
believed that you had seen unmistakable documentary evidence proving
that their confessions were false. There was a certain photograph about
which you had a hallucination. You believed that you had actually held
it in your hands. It was a photograph something like this.''

\switchcolumn

``再说个例子,''他说,``几年前,你的确产生过一个非常严重的幻觉。你认为有三个人,三个曾经的党员:琼斯、阿朗森和鲁瑟夫在彻底地坦白完罪行后以叛国和破坏罪被处死——你觉得他们并没有犯那些指控给他们的罪。你认为你看到了确凿无疑的证据,证明他们供认的东西纯属捏造。你产生了一种幻觉,认为真的存在那么一张照片,还认为自己真的亲手摸到过它。就是这样的照片。''

\switchcolumn*

An oblong slip of newspaper had appeared between
O\textquotesingle Brien\textquotesingle s fingers. For perhaps five
seconds it was within the angle of Winston\textquotesingle s vision. It
was a photograph, and there was no question of its identity. It was
\emph{the} photograph. It was another copy of the photograph of Jones,
Aaronson, and Rutherford at the Party function in New York, which he had
chanced upon eleven years ago and promptly destroyed. For only an
instant it was before his eyes, then it was out of sight again. But he
had seen it, unquestionably he had seen it! He made a desperate,
agonizing effort to wrench the top half of his body free. It was
impossible to move so much as a centimeter in any direction. For the
moment he had even forgotten the dial. All he wanted was to hold the
photograph in his fingers again, or at least to see it.

\switchcolumn

奥布兰用手指夹起一张剪报。在温斯顿的视线中,它大约出现了五秒。那是一张照片,不用怀疑,就是那张照片,另一个版本的琼斯、阿朗森和鲁瑟夫出席纽约党大会的照片。他在十一年前碰巧看到过,还立即将它销毁。它在他眼前只出现了那么一瞬,紧接着就消失了。但他毕竟看到了它,没错,他看到了它!他努力忍受痛苦,不顾一切地扭动着,想让上半身挣脱开来。而不管往哪个方向动,他都不可能挪动一厘米。此刻,他几乎忘记了刻度盘。他只想再一次地抓住那张照片,至少再看上一眼。

\switchcolumn*

``It exists!'' he cried.

\switchcolumn

``它存在!''他喊。

\switchcolumn*

``No,'' said O\textquotesingle Brien.

\switchcolumn

``不!''奥布兰说。

\switchcolumn*

He stepped across the room. There was a memory hole in the opposite
wall. O\textquotesingle Brien lifted the grating. Unseen, the frail slip
of paper was whirling away on the current of warm air; it was vanishing
in a flash of flame. O\textquotesingle Brien turned away from the wall.

\switchcolumn

他穿过屋子,记忆洞就在对面的墙上。奥布兰将洞盖打开,不等温斯顿看到,那张薄纸就被一股温暖的气流卷走,被火焰烧尽。奥布兰从墙的那边转过身来。

\switchcolumn*

``Ashes,'' he said. ``Not even identifiable ashes. Dust. It does not exist.
It never existed.''

\switchcolumn

``变成灰了,''他说,``甚至不是那种能够辨认出来的灰。它是尘埃,它不存在,它从来就没存在过。''

\switchcolumn*

``But it did exist! It does exist! It exists in memory. I remember it.
You remember it.''

\switchcolumn

``但它存在过!它存在的!它存在在记忆里。我记得它,你记得它。''

\switchcolumn*

``I do not remember it,'' said O\textquotesingle Brien.

\switchcolumn

``我不记得它。''奥布兰说。

\switchcolumn*

Winston\textquotesingle s heart sank. That was doublethink. He had a
feeling of deadly helplessness. If he could have been certain that
O\textquotesingle Brien was lying, it would not have seemed to matter.
But it was perfectly possible that O\textquotesingle Brien had really
forgotten the photograph. And if so, then already he would have
forgotten his denial of remembering it, and forgotten the act of
forgetting. How could one be sure that it was simply trickery? Perhaps
that lunatic dislocation in the mind could really happen: that was the
thought that defeated him.

\switchcolumn

温斯顿的心沉了下去。那是双重思想。他感到死一般的无助。如果他能确定奥布兰在说谎,那它也许就不重要了。但很有可能,奥布兰真的把这张照片忘得一干二净。若是这样,那他也已经忘掉他曾否认他记得这张照片,进而忘记``忘记''这一行为。如此,你要怎样确定它只是个骗人的把戏呢?也许,他的精神真的出现了疯狂的错乱。他被这种想法击溃了。

\switchcolumn*

O\textquotesingle Brien was looking down at him speculatively. More than
ever he had the air of a teacher taking pains with a wayward but
promising child.

\switchcolumn

奥布兰低下头看他,沉思着。和之前相比,他更像教师了,好像正辛苦地教授一个身在歧途又仍有希望的孩子。

\switchcolumn*

``There is a Party slogan dealing with the control of the past,'' he said.
``Repeat it, if you please.''

\switchcolumn

``党有一句关于用控制的方法对待过去的口号,''他说,``如果可以,请重复一遍。''

\switchcolumn*

``\textquotesingle Who controls the past controls the future; who
controls the present controls the past,\textquotesingle'' repeated
Winston obediently.

\switchcolumn

``谁控制了过去,谁就控制了未来;谁控制了现在,谁就控制了过去,''温斯顿顺从地
重复道。

\switchcolumn*

``\textquotesingle Who controls the present controls the
past,\textquotesingle'' said O\textquotesingle Brien, nodding his head
with slow approval. ``Is it your opinion, Winston, that the past has real
existence?''

\switchcolumn

``谁控制了现在,谁就控制了过去,''奥布兰说,他点了点头表示赞许。``这是你的观点吗?温斯顿,过去真的存在吗?''

\switchcolumn*

Again the feeling of helplessness descended upon Winston. His eyes
flitted toward the dial. He not only did not know whether ``yes'' or ``no''
was the answer that would save him from pain; he did not even know which
answer he believed to be the true one.

\switchcolumn

突然,温斯顿又一次有了无助的感觉。他快速地看了眼刻度盘,他不仅不知道要怎样回答才能使自己避免遭受痛苦,是``是''还是``不是'',他甚至不知道自己会相信哪个答案。

\switchcolumn*

O\textquotesingle Brien smiled faintly. ``You are no metaphysician,
Winston,'' he said. ``Until this moment you had never considered what is
meant by existence. I will put it more precisely. Does the past exist
concretely, in space? Is there somewhere or other a place, a world of
solid objects, where the past is still happening?''

\switchcolumn

奥布兰微微一笑:``你不是玄学家,温斯顿,''他说。``直到现在你都没想过`存在'意味着什么。让我说得更准确些。过去是有形地存在于空间中吗?在某个或是什么别的由物质所组成的世界里,过去仍在进行着?''

\switchcolumn*

``No.''

\switchcolumn

``不。''

\switchcolumn*

``Then where does the past exist, if at all?''

\switchcolumn

``那么过去究竟存在于哪里呢?''

\switchcolumn*

``In records. It is written down.''

\switchcolumn

``档案里。它被写了下来。''

\switchcolumn*

``In records. And---?''

\switchcolumn

``档案里,还有呢?''

\switchcolumn*

``In the mind. In human memories.''

\switchcolumn

``在意识里。在人的记忆里。''

\switchcolumn*

``In memory. Very well, then. We, the Party, control all records, and we
control all memories. Then we control the past, do we not?''

\switchcolumn

``在记忆里。非常好。那么,我们,党,掌控着所有的档案,且控制着所有记忆。那么我们就控制了过去,对不对?''

\switchcolumn*

``But how can you stop people remembering things?'' cried Winston, again
momentarily forgetting the dial. ``It is involuntary. It is outside
oneself. How can you control memory? You have not controlled mine!''

\switchcolumn

``但是,你们怎么能让人不去记住那些事情呢?''温斯顿喊了起来,又一次忘记了刻度盘。``它不由自主,它不受控制。你们怎么能控制人的记忆呢?你就没能控制我的记忆!''

\switchcolumn*

O\textquotesingle Brien\textquotesingle s manner grew stern again. He
laid his hand on the dial.

\switchcolumn

奥布兰的样子又严厉起来。他将手放在刻度盘上。

\switchcolumn*

``On the contrary,'' he said, ``\emph{you} have not controlled it. That is
what has brought you here. You are here because you have failed in
humility, in self-discipline. You would not make the act of submission
which is the price of sanity. You preferred to be a lunatic, a minority
of one. Only the disciplined mind can see reality, Winston. You believe
that reality is something objective, external, existing in its own
right. You also believe that the nature of reality is self-evident. When
you delude yourself into thinking that you see something, you assume
that everyone else sees the same thing as you. But I tell you, Winston,
that reality is not external. Reality exists in the human mind, and
nowhere else. Not in the individual mind, which can make mistakes, and
in any case soon perishes; only in the mind of the Party, which is
collective and immortal. Whatever the Party holds to be truth \emph{is}
truth. It is impossible to see reality except by looking through the
eyes of the Party. That is the fact that you have got to relearn,
Winston. It needs an act of self-destruction, an effort of the will. You
must humble yourself before you can become sane.''

\switchcolumn

``刚好相反,''他说。``是你没能控制住记忆,这就是为什么要把你带到这儿来。你到这儿来是因为你不谦卑,不自律,你的行为没能服从于理智。你更愿意当一个疯子、一个少数派。只有受过训练的头脑才能看得清何为真实,温斯顿。你相信现实就是客观的、外在的、按它自己的方式存在的东西。你还相信现实的性质不言而喻。当你被这种想法迷惑时,你就会以为你看到什么,别人也和你一样看到什么。但是,我告诉你,温斯顿。现实不是外在的。现实存在于人的意识里,除此它不存在于其他任何地方。然而,它不是存在于某个个体的意识里,因为个体会犯错误,且无论如何都会很快消失。现实只存在于党的意识里,党的意识又是集体的,不朽的。党主张的真理,不管是什么,都是真理。如果不用党的眼睛来看,你就不可能看到现实。你必须重新学习,温斯顿,这就是事实。它需要你摧毁自我,这是一种意志上的努力。你必须先让自己卑微起来,然后才能成为理智的人。''

\switchcolumn*

He paused for a few moments, as though to allow what he had been saying
to sink in.

\switchcolumn

他稍停片刻,以便对方能充分理解他所说的话。

\switchcolumn*

``Do you remember,'' he went on, ``writing in your diary,
\textquotesingle Freedom is the freedom to say that two plus two make
four\textquotesingle?''

\switchcolumn

``你还记得吗?''他继续说道,``你在日记中写`自由就是可以说二加二等于四的自由'?''

\switchcolumn*

``Yes,'' said Winston.

\switchcolumn

``记得。''温斯顿说。

\switchcolumn*

O\textquotesingle Brien held up his left hand, its back toward Winston,
with the thumb hidden and the four fingers extended.

\switchcolumn

奥布兰举起左手,用手背对着温斯顿,将大拇指藏起,其余四指伸出。

\switchcolumn*

``How many fingers am I holding up, Winston?''

\switchcolumn

``我举起了几根手指,温斯顿?''

\switchcolumn*

``Four.''

\switchcolumn

``四根。''

\switchcolumn*

``And if the Party says that it is not four but five---then how many?''

\switchcolumn

``如果党说不是四而是五呢?——那么又有几根?''

\switchcolumn*

``Four.''

\switchcolumn

``四根。''

\switchcolumn*

The word ended in a gasp of pain. The needle of the dial had shot up to
fifty-five. The sweat had sprung out all over Winston\textquotesingle s
body. The air tore into his lungs and issued again in deep groans which
even by clenching his teeth he could not stop. O\textquotesingle Brien
watched him, the four fingers still extended. He drew back the lever.
This time the pain was only slightly eased.

\switchcolumn

话音未落,他就疼得喘息起来。刻度盘上的指针一下子指到了55。温斯顿大汗淋漓。吸入肺里的空气在呼出来时变成低沉地呻吟,就算他咬紧牙齿也无法令呻吟停止。奥布兰看着他,仍旧伸着四根手指。他拉回控制杆,疼痛只减轻了一点点。

\switchcolumn*

``How many fingers, Winston?''

\switchcolumn

``几根手指,温斯顿?''

\switchcolumn*

``Four.''

\switchcolumn

``四根。''

\switchcolumn*

The needle went up to sixty.

\switchcolumn

指针指到60。

\switchcolumn*

``How many fingers, Winston?''

\switchcolumn

``几根手指,温斯顿?''

\switchcolumn*

``Four! Four! What else can I say? Four!''

\switchcolumn

``四根!四根!还能说什么?四根!''

\switchcolumn*

The needle must have risen again, but he did not look at it. The heavy,
stern face and the four fingers filled his vision. The fingers stood up
before his eyes like pillars, enormous, blurry, and seeming to vibrate,
but unmistakably four.

\switchcolumn

指针肯定升上去了,但他没往那里看。他只看到沉重而严厉的面孔和四根手指,这四根手指像四根又大又模糊的柱子一样竖立在他眼前,它们似乎在颤抖,但毫无疑问,就是四根。

\switchcolumn*

``How many fingers, Winston?''

\switchcolumn

``几根手指,温斯顿?''

\switchcolumn*

``Four! Stop it, stop it! How can you go on? Four! Four!''

\switchcolumn

``四根!停下,停下!你怎么可以继续下去?四根!四根!''

\switchcolumn*

``How many fingers, Winston?''

\switchcolumn

``几根手指,温斯顿?''

\switchcolumn*

``Five! Five! Five!''

\switchcolumn

``五根!五根!五根!''

\switchcolumn*

``No, Winston, that is no use. You are lying. You still think there are
four. How many fingers, please?''

\switchcolumn

``不,温斯顿,这没用。你在说谎,你仍然觉得是四根。请问,几根手指?''

\switchcolumn*

``Four! Five! Four! Anything you like. Only stop it, stop the pain!''

\switchcolumn

``四根!四根!四根!随你的便。只要让它停下,别让我疼!''

\switchcolumn*

Abruptly he was sitting up with
O\textquotesingle Brien\textquotesingle s arm round his shoulders. He
had perhaps lost consciousness for a few seconds. The bonds that had
held his body down were loosened. He felt very cold, he was shaking
uncontrollably, his teeth were chattering, the tears were rolling down
his cheeks. For a moment he clung to O\textquotesingle Brien like a
baby, curiously comforted by the heavy arm round his shoulders. He had
the feeling that O\textquotesingle Brien was his protector, that the
pain was something that came from outside, from some other source, and
that it was O\textquotesingle Brien who would save him from it.

\switchcolumn

靠着奥布兰环在他肩膀上的手臂,他猛地坐了起来。有那么几秒,他似乎失去了意识。绑住他身体的带子松掉了。他觉得很冷,不禁浑身颤抖,牙齿也发出咔嗒咔嗒的声音,他的脸上布满泪水。有那么一会儿,他像小孩子一样抱住奥布兰,而抱着他厚实的肩膀让他有种奇怪的舒适感。他觉得奥布兰是他的保护者,疼痛是外来的,来自于别人,只有奥布兰才能将他从疼痛中拯救出来。

\switchcolumn*

``You are a slow learner, Winston,'' said O\textquotesingle Brien gently.

\switchcolumn

``你学得很慢,温斯顿。''奥布兰温和地说。

\switchcolumn*

``How can I help it?'' he blubbered. ``How can I help seeing what is in
front of my eyes? Two and two are four.''

\switchcolumn

``我能怎么做呢?''他哭嚎着,``我要怎样才能看不到眼前的东西?二加二就是等于四。''

\switchcolumn*

``Sometimes, Winston. Sometimes they are five. Sometimes they are three.
Sometimes they are all of them at once. You must try harder. It is not
easy to become sane.''

\switchcolumn

``有时是这样,温斯顿。有时,它等于五,有时它等于三,还有时三、四、五都对,你必须再努力一些。变理智可不容易。''

\switchcolumn*

He laid Winston down on the bed. The grip on his limbs tightened again,
but the pain had ebbed away and the trembling had stopped, leaving him
merely weak and cold. O\textquotesingle Brien motioned with his head to
the man in the white coat, who had stood immobile throughout the
proceedings. The man in the white coat bent down and looked closely into
Winston\textquotesingle s eyes, felt his pulse, laid an ear against his
chest, tapped here and there; then he nodded to O\textquotesingle Brien.

\switchcolumn

他把温斯顿放到床上,温斯顿的四肢又被带子绑紧,但疼痛已经消退,颤抖也已停止,只剩下虚弱和寒冷的感觉。奥布兰用头向一个穿白色衣服的人示意,整个过程那人都一动不动地站着。白衣服的人弯下身,仔细看了看温斯顿的眼睛,又感觉了下他的脉搏,听了听他的心跳。他敲敲这儿,弄弄那儿,然后冲奥布兰点了下头。

\switchcolumn*

``Again,'' said O\textquotesingle Brien.

\switchcolumn

``继续。''奥布兰说。

\switchcolumn*

The pain flowed into Winston\textquotesingle s body. The needle must be
at seventy, seventy-five. He had shut his eyes this time. He knew that
the fingers were still there, and still four. All that mattered was
somehow to stay alive until the spasm was over. He had ceased to notice
whether he was crying out or not. The pain lessened again. He opened his
eyes. O\textquotesingle Brien had drawn back the lever.

\switchcolumn

疼痛占据了温斯顿的身体,指针一定指到了70、75。这次他闭上了眼睛,他知道手指仍在那里,仍是四根。最重要的就是活下来,直到疼痛结束。他不再关心自己是不是哭了出来。疼痛又减退了。他张开眼睛。奥布兰拉回了控制杆。

\switchcolumn*

``How many fingers, Winston?''

\switchcolumn

``几根手指,温斯顿?''

\switchcolumn*

``Four. I suppose there are four. I would see five if I could. I am
trying to see five.''

\switchcolumn

``四根。我猜是四根,如果我能,我希望看到五根,我试图看到五根。''

\switchcolumn*

``Which do you wish: to persuade me that you see five, or really to see
them?''

\switchcolumn

``你希望什么?是想说服我你看到了五根,还是真的看到五根?''

\switchcolumn*

``Really to see them.''

\switchcolumn

``真的看到五根。''

\switchcolumn*

``Again,'' said O\textquotesingle Brien.

\switchcolumn

``继续。''奥布兰说。

\switchcolumn*

Perhaps the needle was at eighty---ninety. Winston could only
intermittently remember why the pain was happening. Behind his
screwed-up eyelids a forest of fingers seemed to be moving in a sort of
dance, weaving in and out, disappearing behind one another and
reappearing again. He was trying to count them, he could not remember
why. He knew only that it was impossible to count them, and that this
was somehow due to the mysterious identity between five and four. The
pain died down again. When he opened his eyes it was to find that he was
still seeing the same thing. Innumerable fingers, like moving trees,
were still streaming past in either direction, crossing and recrossing.
He shut his eyes again.

\switchcolumn

指针可能指到了80---90。温斯顿一直都记得为什么会产生疼痛。在他紧闭的眼皮后,手指森林像跳舞一般地挪动着,它们伸进伸出,它们一会儿叠在一起,一会儿又彼此分开,一会儿被遮住消失,一会儿又重新出现。他尝试着数一数,记不清为什么,他知道仅靠数是数不清的,这是由四与五之间的神秘特质决定的。疼痛再次减轻。他睁开眼睛,看到的仍然是相同的景象。数不清的手指就像移动的树木,仍朝着某个方向交叠、分开。他又闭上了眼睛。

\switchcolumn*

``How many fingers am I holding up, Winston?''

\switchcolumn

``我举起了几根手指,温斯顿?''

\switchcolumn*

``I don\textquotesingle t know. I don\textquotesingle t know. You will
kill me if you do that again. Four, five, six---in all honesty I
don\textquotesingle t know.''

\switchcolumn

``我不知道,我不知道,如果你再这样做,就杀了我吧。四、五、六——说实话,我不知道。''

\switchcolumn*

``Better,'' said O\textquotesingle Brien.

\switchcolumn

``好点了。''奥布兰说。

\switchcolumn*

A needle slid into Winston\textquotesingle s arm. Almost in the same
instant a blissful, healing warmth spread all through his body. The pain
was already half-forgotten. He opened his eyes and looked up gratefully
at O\textquotesingle Brien. At sight of the heavy, lined face, so ugly
and so intelligent, his heart seemed to turn over. If he could have
moved he would have stretched out a hand and laid it on
O\textquotesingle Brien\textquotesingle s arm. He had never loved him so
deeply as at this moment, and not merely because he had stopped the
pain. The old feeling, that at bottom it did not matter whether
O\textquotesingle Brien was a friend or an enemy, had come back.
O\textquotesingle Brien was a person who could be talked to. Perhaps one
did not want to be loved so much as to be understood.
O\textquotesingle Brien had tortured him to the edge of lunacy, and in a
little while, it was certain, he would send him to his death. It made no
difference. In some sense that went deeper than friendship, they were
intimates; somewhere or other, although the actual words might never be
spoken, there was a place where they could meet and talk.
O\textquotesingle Brien was looking down at him with an expression which
suggested that the same thought might be in his own mind. When he spoke
it was in an easy, conversational tone.

\switchcolumn

一根针刺进温斯顿的手臂。差不多同时,一种幸福的、温暖的治愈感在他身上弥漫开来。疼痛几乎忘记了一半。他张开眼睛,感激地看了看奥布兰。看到他深沉又线条分明的面孔,它如此丑陋又如此聪明,他心潮涌动,若他能动一动身体,他就伸出手,搭在奥布兰的胳膊上。他从来没有像现在这样深爱他,这不只因为他让疼痛停止。这感觉曾出现过现在又回来了,说到底,奥布兰是朋友还是敌人无关紧要。他是那种可以与之交谈的人。也许,相比被人所爱,一个人更希望被人了解。奥布兰将他折磨得几欲崩溃,而且有那么一瞬间,可以确定,他几乎将他置之死地。这没有什么不同。从某种角度说,他们的关系比友谊更深,他们是知己。或者这里,或者那里,虽然没有说出来,可总有一个地方能让他们见面聊聊。奥布兰俯视着他,他的表情说明在他心里可能也有同样的想法。他用一种轻松的聊天式的腔调说:

\switchcolumn*

``Do you know where you are, Winston?'' he said.

\switchcolumn

``知道你在哪儿么,温斯顿?''他问。

\switchcolumn*

``I don\textquotesingle t know. I can guess. In the Ministry of Love.''

\switchcolumn

``不知道,但我猜得到,在仁爱部。''

\switchcolumn*

``Do you know how long you have been here?''

\switchcolumn

``你知道自己在这儿待了多久了吗?''

\switchcolumn*

``I don\textquotesingle t know. Days, weeks, months---I think it is
months.''

\switchcolumn

``不知道,几天、几星期、几个月——我想是一个月。''

\switchcolumn*

``And why do you imagine that we bring people to this place?''

\switchcolumn

``你想一下,为什么我们要把人带到这个地方?''

\switchcolumn*

``To make them confess.''

\switchcolumn

``让他们招供。''

\switchcolumn*

``No, that is not the reason. Try again.''

\switchcolumn

``不,不是这个原因,再想想。''

\switchcolumn*

``To punish them.''

\switchcolumn

``惩罚他们。''

\switchcolumn*

``No!'' exclaimed O\textquotesingle Brien. His voice had changed
extraordinarily, and his face had suddenly become both stern and
animated. ``No! Not merely to extract your confession, nor to punish you.
Shall I tell you why we have brought you here? To cure you! To make you
sane! Will you understand, Winston, that no one whom we bring to this
place ever leaves our hands uncured? We are not interested in those
stupid crimes that you have committed. The Party is not interested in
the overt act; the thought is all we care about. We do not merely
destroy our enemies; we change them. Do you understand what I mean by
that?''

\switchcolumn

``不!''奥布兰叫了起来。他的声音不同平常,他的脸色也突然变得严肃、激动。``不对!不单要榨出供词,也不单要惩罚你们,要我告诉你,我们为什么把你带到这儿来吗?为了医治你!让你清醒!你能理解吗,温斯顿,被我们带到这儿的,没有一个不被治好就离开的。我们对你犯下的那些愚蠢的罪行毫无兴趣。党对表面的行为没兴趣,我们关心的是思想。我们不只要打败我们的敌人,我们还要改变他们。你能理解我的意思吗?''

\switchcolumn*

He was bending over Winston. His face looked enormous because of its
nearness, and hideously ugly because it was seen from below. Moreover it
was filled with a sort of exaltation, a lunatic intensity. Again
Winston\textquotesingle s heart shrank. If it had been possible he would
have cowered deeper into the bed. He felt certain that
O\textquotesingle Brien was about to twist the dial out of sheer
wantonness. At this moment, however, O\textquotesingle Brien turned
away. He took a pace or two up and down. Then he continued less
vehemently:

\switchcolumn

他弯下腰,看了看温斯顿。由于距离很近,他的脸看起来很大,又因为从下往上看,这张脸丑得让人厌恶。不仅如此,它还呈现出一种兴奋的、疯狂的神情。温斯顿的心再度沉了下去。如果可能,他会钻到床底下去。他觉得奥布兰一定会没有节制地扭动控制杆。可就在这个时候,奥布兰转身了,他来来回回走了几步,继续说了起来,没有刚才那般激动:

\switchcolumn*

``The first thing for you to understand is that in this place there are
no martyrdoms. You have read of the religious persecutions of the past.
In the Middle Ages there was the Inquisition. It was a failure. It set
out to eradicate heresy, and ended by perpetuating it. For every heretic
it burned at the stake, thousands of others rose up. Why was that?
Because the Inquisition killed its enemies in the open, and killed them
while they were still unrepentant; in fact, it killed them because they
were unrepentant. Men were dying because they would not abandon their
true beliefs. Naturally all the glory belonged to the victim and all the
shame to the Inquisitor who burned him. Later, in the twentieth century,
there were the totalitarians, as they were called. There were the German
Nazis and the Russian Communists. The Russians persecuted heresy more
cruelly than the Inquisition had done. And they imagined that they had
learned from the mistakes of the past; they knew, at any rate, that one
must not make martyrs. Before they exposed their victims to public
trial, they deliberately set themselves to destroy their dignity. They
wore them down by torture and solitude until they were despicable,
cringing wretches, confessing whatever was put into their mouths,
covering themselves with abuse, accusing and sheltering behind one
another, whimpering for mercy. And yet after only a few years the same
thing had happened over again. The dead men had become martyrs and their
degradation was forgotten. Once again, why was it? In the first place,
because the confessions that they had made were obviously extorted and
untrue. We do not make mistakes of that kind. All the confessions that
are uttered here are true. We make them true. And, above all, we do not
allow the dead to rise up against us. You must stop imagining that
posterity will vindicate you, Winston. Posterity will never hear of you.
You will be lifted clean out from the stream of history. We shall turn
you into gas and pour you into the stratosphere. Nothing will remain of
you: not a name in a register, not a memory in a living brain. You will
be annihilated in the past as well as in the future. You will never have
existed.''

\switchcolumn

``首先,你要知道,这里没有烈士。你应该读到过去曾有过宗教迫害。中世纪有宗教法庭,它失败了。它的出发点是清除异端,它的结果却是巩固异端。它每烧死一个异端,就会有几千个异端涌现出来。为什么会这样?因为宗教法庭公开杀死敌人,他们的敌人至死都没有悔改。事实上,他们之所以要杀死他们就是因为他们不肯悔改,因为这些人不肯放弃他们真正的信仰。如此一来,所有的荣耀都自然而然地属于殉难者,所有的耻辱都自然而然地归于烧死这些人的宗教法庭。后来,到了20世纪,极权主义者出现了,他们被这样称呼。他们是德国纳粹,是俄国共产党。就迫害异端而言,俄国人比宗教法庭还要残酷。他们以为他们已经从过去的错误中取得了教训;他们明白,不管怎么说,一定不能制造烈士。他们在公开的审判上揭露他们的牺牲者,在这之前,故意摧毁他们的人格。他们通过拷打和单独禁闭打垮他们,直到他们成为卑劣的、畏畏缩缩的坏蛋,让他们承认什么,他们就承认什么。他们一边辱骂自己、攻击自己,一面又用辱骂、攻击别人的方式来保护自己,为寻求宽恕而哭泣。然而,过不了几年,同样的事情又发生了。死去的人成为烈士,他们堕落的一面被人遗忘。再说一次,为什么会这样?首先,他们的供词是被逼出来的,并不真实。我们不能再让这种错误重演。在这里所有的供词都绝对真实。我们想办法让它们真实。重要的是,我们不会让死者站起来反对我们。你千万别以为你的后代会为你申冤。温斯顿,后人永远不会知道你。你会在历史长河中消失得干干净净。我们要让你变成气体,将你倾入天空。你什么都留不下,登记簿上没有你的名字,活着的人的大脑里也没有关于你的记忆。过去也好,将来也罢,你被消灭了,你从来就没存在过。''

\switchcolumn*

Then why bother to torture me? thought Winston, with a momentary
bitterness. O\textquotesingle Brien checked his step as though Winston
had uttered the thought aloud. His large ugly face came nearer, with the
eyes a little narrowed.

\switchcolumn

那又为什么要折磨我呢?温斯顿这样想着,刹那间心生怨恨。奥布兰停下脚步,就好像温斯顿大声说出了这个想法。他将大而丑陋的脸靠近他,将眼睛眯成一条缝。

\switchcolumn*

``You are thinking,'' he said, ``that since we intend to destroy you
utterly, so that nothing that you say or do can make the smallest
difference---in that case, why do we go to the trouble of interrogating
you first? That is what you were thinking, was it not?''

\switchcolumn

``你在想事情,''他说,``我们要将你完完全全地消灭掉,你说的和你做的都不会产生什么结果——既然如此,为什么我们还要那么麻烦地审问你呢?你是这样想的吗?''

\switchcolumn*

``Yes,'' said Winston.

\switchcolumn

``是的。''温斯顿说。

\switchcolumn*

O\textquotesingle Brien smiled slightly. ``You are a flaw in the pattern,
Winston. You are a stain that must be wiped out. Did I not tell you just now
that we are different from the persecutors of the past? We are not content
with negative obedience, nor even with the most abject submission. When
finally you surrender to us, it must be of your own free will. We do not
destroy the heretic because he resists us; so long as he resists us we never
destroy him. We convert him, we capture his inner mind, we reshape him. We
burn all evil and all illusion out of him; we bring him over to our side,
not in appearance, but genuinely, heart and soul. We make him one of
ourselves before we kill him. It is intolerable to us that an erroneous
thought should exist anywhere in the world, however secret and powerless it
may be. Even in the instant of death we cannot permit any deviation. In the
old days the heretic walked to the stake still a heretic, proclaiming his
heresy, exulting in it. Even the victim of the Russian purges could carry
rebellion locked up in his skull as he walked down the passage waiting for
the bullet. But we make the brain perfect before we blow it out. The command
of the old despotisms was `Thou shalt not.' The command of the totalitarians
was `Thou shalt.' Our command is `\emph{Thou
  art}.' No one whom we bring to this place ever stands out
against us. Everyone is washed clean. Even those three miserable traitors in
whose innocence you once believed---Jones, Aaronson, and Rutherford---in the
end we broke them down. I took part in their interrogation myself. I saw
them gradually worn down, whimpering, groveling, weeping---and in the end it
was not with pain or fear, only with penitence. By the time we had finished
with them they were only the shells of men. There was nothing left in them
except sorrow for what they had done, and love of Big Brother. It was
touching to see how they loved him. They begged to be shot quickly, so that
they could die while their minds were still clean.''

\switchcolumn

奥布兰轻轻地笑了,``你是图片上的一点瑕疵,温斯顿,你是必须被清理掉的污点。我刚才不是告诉过你,我们和以往的迫害者不同吗?消极的服从不能让我们满意,甚至最卑微的屈从也不能让人满意。最终,你的屈服必须出自你的自由意志。我们消灭异端,不因为他反抗我们。只要他反抗,我们就不会将他消灭。我们要让他发生转变,征服他的思想并将他重新塑造。我们要烧掉他所有的邪恶和幻想。我们要将他拉到我们这边,不单是外表,精神、心灵、灵魂都要站在我们这边。我们要在杀死他之前将他变成我们的人。我们不能容忍世界上有错误的思想,不管它在哪里,也不管它有多么隐秘,多么微弱。一个人哪怕立即死掉,也不允许他有什么越轨的想法。过去,异端在走向火刑柱时仍是异端,仍在宣扬他的异端思想,并为此欣喜。即便是俄国大清洗中的受害者在步入刑场等候枪毙的时候,他封闭着的大脑里仍然存在着反抗的念头。但我们却要在爆掉这脑袋前将它变得完美。之前的独裁者要求`你们不能做什么',极权主义者要求`你们要做什么',我们则要求`你们要是什么'。被我们带到这儿来的人没一个能站出来反对我们。每个人都被净化干净,就连你以为的那三个可怜的叛国者——琼斯、阿朗森和鲁瑟夫——到了最后也被我们击垮。我本人也参加了对他们的审讯。我亲眼看到他们慢慢地垮了下来,他们呜咽着,匍匐着,哭泣着——最终他们有的不是疼痛和恐惧,而是悔恨。审讯结束了,他们仅有一副躯壳,除了对所做之事的懊悔和对老大哥的热爱,什么都没剩下。看着他们这样热爱他,真的很感动。他们希望尽快被枪毙,以便在思想纯洁的时候死去。''

\switchcolumn*

His voice had grown almost dreamy. The exaltation, the lunatic
enthusiasm, was still in his face. He is not pretending, thought
Winston; he is not a hypocrite; he believes every word he says. What
most oppressed him was the consciousness of his own intellectual
inferiority. He watched the heavy yet graceful form strolling to and
fro, in and out of the range of his vision. O\textquotesingle Brien was
a being in all ways larger than himself. There was no idea that he had
ever had, or could have, that O\textquotesingle Brien had not long ago
known, examined, and rejected. His mind \emph{contained}
Winston\textquotesingle s mind. But in that case how could it be true
that O\textquotesingle Brien was mad? It must be he, Winston, who was
mad. O\textquotesingle Brien halted and looked down at him. His voice
had grown stern again.

\switchcolumn

他的声音犹如梦呓,脸上仍有那种疯狂、激动的神色。温斯顿认为,他不是装模作样,他不是一个虚伪的人,他相信自己所说的每个字。最让温斯顿压抑的是,他为自己的智商感到自卑。他看着这个厚重文雅的身体走过来走过去,这身体时而出现在他的视野中,时而消失在他的视野外。从各个方面说,奥布兰都比他强大。他曾经萌生的,或可能萌生的念头,没有一个不在奥布兰的预料之中,没有一个没被奥布兰研究过、驳斥过。他的头脑涵盖了温斯顿的头脑。然而既然如此,奥布兰疯了又怎么会是真实的呢?疯的人一定是温斯顿。奥布兰停下来,低头看着他,声音又严厉起来。

\switchcolumn*

``Do not imagine that you will save yourself, Winston, however completely
you surrender to us. No one who has once gone astray is ever spared. And
even if we chose to let you live out the natural term of your life,
still you would never escape from us. What happens to you here is
forever. Understand that in advance. We shall crush you down to the
point from which there is no coming back. Things will happen to you from
which you could not recover, if you lived a thousand years. Never again
will you be capable of ordinary human feeling. Everything will be dead
inside you. Never again will you be capable of love, or friendship, or
joy of living, or laughter, or curiosity, or courage, or integrity. You
will be hollow. We shall squeeze you empty, and then we shall fill you
with ourselves.''

\switchcolumn

``别以为可以拯救自己,温斯顿,不管你如何彻头彻尾地屈服于我们。走上歧途的人没一个能幸免。即使我们选择让你活下去,活到底,你也永远别想从我们手里逃脱。在你身上发生的事会永远持续下去。你必须先明白这点。我们会将你击溃,让你无法回到原先的时间点。哪怕你活上一千年,也没法恢复原样,你不会再有一般人的情感,你内心的一切都将死去。你不能再拥有爱情、友情,或者生活的乐趣、欢笑、好奇、勇气和追求正直的心。你将成为空心人,我们会把你挤空,然后我们再用我们自己来填充你。''

\switchcolumn*

He paused and signed to the man in the white coat. Winston was aware of
some heavy piece of apparatus being pushed into place behind his head.
O\textquotesingle Brien had sat down beside the bed, so that his face
was almost on a level with Winston\textquotesingle s.

\switchcolumn

他停下来,向白衣服的人示意。温斯顿明显感觉到一个很重的器械被放到自己的脑袋底下。奥布兰在床边坐下来,他的脸几乎和温斯顿的处在同一水平线上。

\switchcolumn*

``Three thousand,'' he said, speaking over Winston\textquotesingle s head
to the man in the white coat.

\switchcolumn

``3000,''他对温斯顿头后的那个白衣人说。

\switchcolumn*

Two soft pads, which felt slightly moist, clamped themselves against
Winston\textquotesingle s temples. He quailed. There was pain coming, a
new kind of pain. O\textquotesingle Brien laid a hand reassuringly,
almost kindly, on his.

\switchcolumn

两块柔软的、稍微有些湿润的垫子夹住了温斯顿的太阳穴。他害怕了,一种完全不同的痛感向他袭来。奥布兰伸出一只手安慰他,几乎是温和地将手放在他手里。

\switchcolumn*

``This time it will not hurt,'' he said. ``Keep your eyes fixed on mine.''

\switchcolumn

``这次不会有伤害,''他说,``看着我。''

\switchcolumn*

At this moment there was a devastating explosion, or what seemed like an
explosion, though it was not certain whether there was any noise. There
was undoubtedly a blinding flash of light. Winston was not hurt, only
prostrated. Although he had already been lying on his back when the
thing happened, he had a curious feeling that he had been knocked into
that position. A terrific, painless blow had flattened him out. Also
something had happened inside his head. As his eyes regained their focus
he remembered who he was, and where he was, and recognized the face that
was gazing into his own; but somewhere or other there was a large patch
of emptiness, as though a piece had been taken out of his brain.

\switchcolumn

就在这时,突然发生了一场毁灭性的爆炸,或者说看起来像爆炸。虽然无法确定是否真的发出什么声响,但却一定有一道耀眼的闪光。温斯顿没有受伤,只是筋疲力尽了。爆炸发生时他已经平躺在那里,但他有种奇怪的感觉,觉得自己被什么东西撞到了这个位置。一种猛烈又不会使人疼痛的冲击将他翻倒。他的大脑也出了状况。视力恢复后,他能记起自己是谁,身在哪里,还认出了正盯着他看的那张脸。但说不清是哪儿却有一块很大的空白,就好像他的脑子被挖掉了一块。

\switchcolumn*

``It will not last,'' said O\textquotesingle Brien. ``Look me in the eyes.
What country is Oceania at war with?''

\switchcolumn

``不会持续很久,''奥布兰说,``看着我的眼睛。大洋国和哪个国家打仗?''

\switchcolumn*

Winston thought. He knew what was meant by Oceania, and that he himself
was a citizen of Oceania. He also remembered Eurasia and Eastasia; but
who was at war with whom he did not know. In fact he had not been aware
that there was any war.

\switchcolumn

温斯顿想了想。他知道大洋国是什么意思,知道自己是大洋国的公民,他仍然记得欧亚国和东亚国。但是他不知道大洋国在和谁打仗。实际上,他不知道有什么战争。

\switchcolumn*

``I don\textquotesingle t remember.''

\switchcolumn

``我记不起来了。''

\switchcolumn*

``Oceania is at war with Eastasia. Do you remember that now?''

\switchcolumn

``大洋国在和东亚国打仗,现在你记起来了吗?''

\switchcolumn*

``Yes.''

\switchcolumn

``记起来了。''

\switchcolumn*

``Oceania has always been at war with Eastasia. Since the beginning of
your life, since the beginning of the Party, since the beginning of
history, the war has continued without a break, always the same war. Do
you remember that?''

\switchcolumn

``大洋国一直在和东亚国打仗,从你生下来开始,从党成立开始,从有历史开始,战争就一直在进行,没有间断过,一直都是同一场战争。你记起来了吗?''

\switchcolumn*

``Yes.''

\switchcolumn

``是的。''

\switchcolumn*

``Eleven years ago you created a legend about three men who had been
condemned to death for treachery. You pretended that you had seen a
piece of paper which proved them innocent. No such piece of paper ever
existed. You invented it, and later you grew to believe in it. You
remember now the very moment at which you first invented it. Do you
remember that?''

\switchcolumn

``十一年前你编造了一个和三个因叛国罪被处死的人有关的传奇故事。你假装你看到了一张证明他们无罪的报纸。这样的报纸从来就没存在过。它是你凭空捏造的,后来,你还对它信以为真。现在你还记得你第一次编造它时的情景吧?你还记得吗?''

\switchcolumn*

``Yes.''

\switchcolumn

``记得。''

\switchcolumn*

``Just now I held up the fingers of my hand to you. You saw five fingers.
Do you remember that?''

\switchcolumn

``就在刚刚我冲你举起了手,你看到了五根手指,你记得吗?''

\switchcolumn*

``Yes.''

\switchcolumn

``记得。''

\switchcolumn*

O\textquotesingle Brien held up the fingers of his left hand, with the
thumb concealed.

\switchcolumn

奥布兰举起了他的左手,将大拇指缩了起来。

\switchcolumn*

``There are five fingers there. Do you see five fingers?''

\switchcolumn

``这里有五根手指,你能看到五根手指吗?''

\switchcolumn*

``Yes.''

\switchcolumn

``能。''

\switchcolumn*

And he did see them, for a fleeting instant, before the scenery of his
mind changed. He saw five fingers, and there was no deformity. Then
everything was normal again, and the old fear, the hatred, and the
bewilderment came crowding back again. But there had been a moment---he
did not know how long, thirty seconds, perhaps---of luminous certainty,
when each new suggestion of O\textquotesingle Brien\textquotesingle s
had filled up a patch of emptiness and become absolute truth, and when
two and two could have been three as easily as five, if that were what
was needed. It had faded out before O\textquotesingle Brien had dropped
his hand; but though he could not recapture it, he could remember it, as
one remembers a vivid experience at some remote period of
one\textquotesingle s life when one was in effect a different person.

\switchcolumn

刹那间,在大脑中的景象发生变化前,他真的看到它们了。他看见五根手指,每根都很完整。之后,每件事都恢复了正常,先前的那些恐惧、怨恨、迷惘又重新涌了回来。有那么一会儿——他不知道有多久,大概是三十秒——他非常清醒,奥布兰给出的每个新的暗示都成为绝对的真理,填补了那块空白。如果有需要,二加二等于三就像二加二等于五一样容易。然而,在奥布兰将手放下前,这情形就消失了。他无法将它复原,但他记得住,好比一个人能真切地回忆起多年前的某段经历,而当时他实际上是完全不同的人。

\switchcolumn*

``You see now,'' said O\textquotesingle Brien, ``that it is at any rate
possible.''

\switchcolumn

``现在,你看,''奥布兰说,``不管怎样,这是可能的。''

\switchcolumn*

``Yes,'' said Winston.

\switchcolumn

``没错,''温斯顿说。

\switchcolumn*

O\textquotesingle Brien stood up with a satisfied air. Over to his left
Winston saw the man in the white coat break an ampoule and draw back the
plunger of a syringe. O\textquotesingle Brien turned to Winston with a
smile. In almost the old manner he resettled his spectacles on his nose.

\switchcolumn

奥布兰站起身,整个人心满意足。温斯顿看到他左边那个穿白衣服的人打破了一支安瓿,将注射器的活塞往回抽。奥布兰笑着转向温斯顿,差不多和从前一样,正了正鼻梁上的眼镜。

\switchcolumn*

``Do you remember writing in your diary,'' he said, ``that it did not
matter whether I was a friend or an enemy, since I was at least a person
who understood you and could be talked to? You were right. I enjoy
talking to you. Your mind appeals to me. It resembles my own mind except
that you happen to be insane. Before we bring the session to an end you
can ask me a few questions, if you choose.''

\switchcolumn

``你还记得你在日记里写了什么吗?''他说,``我是敌是友无关紧要,因为我至少是个能理解你,能和你交谈的人。你说得没错。我喜欢和你讲话。你的思想吸引了我。它和我的很像,只不过你发了疯。在谈话结束前,要是你愿意,你可以问我几个问题。''

\switchcolumn*

``Any question I like?''

\switchcolumn

``我想问什么都可以?''

\switchcolumn*

``Anything.'' He saw that Winston\textquotesingle s eyes were upon the
dial. ``It is switched off. What is your first question?''

\switchcolumn

``任何问题都可以。''他发现温斯顿正在看着刻度盘。``它已经关上了。第一个问题是?''

\switchcolumn*

``What have you done with Julia?'' said Winston.

\switchcolumn

``你们对朱莉亚做了什么?''温斯顿说。

\switchcolumn*

O\textquotesingle Brien smiled again. ``She betrayed you, Winston.
Immediately---unreservedly. I have seldom seen anyone come over to us so
promptly. You would hardly recognize her if you saw her. All her
rebelliousness, her deceit, her folly, her dirty-mindedness---everything
has been burned out of her. It was a perfect conversion, a textbook
case.''

\switchcolumn

奥布兰再次露出微笑。``她出卖了你,温斯顿。非常迅速——非常彻底。我很少看到哪个人这么快就改变了立场。如果你看见她,你很难能认出她来。她所有的反抗意识、欺骗手段、她的愚蠢、她的肮脏——全都消失得无影无踪,就像教科书那样完美。''

\switchcolumn*

``You tortured her.''

\switchcolumn

``你们拷问她了?''

\switchcolumn*

O\textquotesingle Brien left this unanswered. ``Next question,'' he said.

\switchcolumn

奥布兰没有回答。``下一个问题,''他说。

\switchcolumn*

``Does Big Brother exist?''

\switchcolumn

``老大哥存在吗?''

\switchcolumn*

``Of course he exists. The Party exists. Big Brother is the embodiment of
the Party.''

\switchcolumn

``当然存在,党存在,老大哥就存在,他是党的化身。''

\switchcolumn*

``Does he exist in the same way as I exist?''

\switchcolumn

``他的存在方式和我的是一样的吗?''

\switchcolumn*

``You do not exist,'' said O\textquotesingle Brien.

\switchcolumn

``你不存在。''奥布兰说。

\switchcolumn*

Once again the sense of helplessness assailed him. He knew, or he could
imagine, the arguments which proved his own nonexistence; but they were
nonsense, they were only a play on words. Did not the statement, ``You do
not exist,'' contain a logical absurdity? But what use was it to say so?
His mind shriveled as he thought of the unanswerable, mad arguments with
which O\textquotesingle Brien would demolish him.

\switchcolumn

又是一阵无助的感觉。他知道,或者说他想象得出来,证明他不存在的理由是什么。但它们都是胡说八道,都不过是文字游戏。诸如``你不存在''这样的话难道不包含逻辑上的漏洞吗?但这样说又有什么用呢?只要想到奥布兰会用无可辩驳的疯狂观点来驳斥他,他就感到无可奈何。

\switchcolumn*

``I think I exist,'' he said wearily. ``I am conscious of my own identity.
I was born, and I shall die. I have arms and legs. I occupy a particular
point in space. No other solid object can occupy the same point
simultaneously. In that sense, does Big Brother exist?''

\switchcolumn

``我想我是存在的,''他疲惫地说,``我能意识到我自己,我出生了,我会死掉。我有胳膊有腿。我占据了一块特定的空间,同时没有什么实际的固体能占据我所在的空间。这么说的话,老大哥存在吗?''

\switchcolumn*

``It is of no importance. He exists?''

\switchcolumn

``这不是什么重要的事,他存在。''

\switchcolumn*

``Will Big Brother ever die?''

\switchcolumn

``老大哥会死吗?''

\switchcolumn*

``Of course not. How could he die? Next question.''

\switchcolumn

``当然不会死,他怎么能死呢?下一个问题。''

\switchcolumn*

``Does the Brotherhood exist?''

\switchcolumn

``兄弟会存在吗?''

\switchcolumn*

``That, Winston, you will never know. If we choose to set you free when
we have finished with you, and if you live to be ninety years old, still
you will never learn whether the answer to that question is Yes or No.
As long as you live, it will be an unsolved riddle in your mind.''

\switchcolumn

``这个,温斯顿,你永远别想知道。我们结束掉对你的工作后就会把你放出去,即使你活到九十岁,你也永远不会知道这个问题的答案是`是'还是`否'。只要你活着,它就是你心里的一道迷。''

\switchcolumn*

Winston lay silent. His breast rose and fell a little faster. He still
had not asked the question that had come into his mind the first. He had
got to ask it, and yet it was as though his tongue would not utter it.
There was a trace of amusement in
O\textquotesingle Brien\textquotesingle s face. Even his spectacles
seemed to wear an ironical gleam. He knows, thought Winston suddenly, he
knows what I am going to ask! At the thought the words burst out of him:

\switchcolumn

温斯顿安安静静地躺着,胸部的起伏比之前快了一些。他还没有问他最想问的问题,他必须问出来,然而他却完全没办法讲出来。奥布兰的脸上出现了一丝欣喜的迹象。他的眼镜片看起来也闪烁着嘲讽的光。他知道,温斯顿突然意识到,他知道我想问他什么!想到这里,他几乎脱口而出:

\switchcolumn*

``What is in Room 101?''

\switchcolumn

``101号房里有什么?''

\switchcolumn*

The expression on O\textquotesingle Brien\textquotesingle s face did not
change. He answered drily:

\switchcolumn

奥布兰还是那副表情,他干巴巴地说:

\switchcolumn*

``You know what is in Room 101, Winston. Everyone knows what is in Room
101.''

\switchcolumn

``你知道101号房里有什么,温斯顿,每个人都知道101号房里有什么。''

\switchcolumn*

He raised a finger to the man in the white coat. Evidently the session
was at an end. A needle jerked into Winston\textquotesingle s arm. He
sank almost instantly into deep sleep.

\switchcolumn

他向白衣服的人举起一根手指。显然,谈话结束了。一根针扎进温斯顿的手臂,他几乎马上便沉睡过去。

\switchcolumn*


\section{III}\label{iii-2}

\switchcolumn

\section*{三}\label{ux4e8cux5341ux4e00}

\switchcolumn*

``There are three stages in your reintegration,'' said
O\textquotesingle Brien. ``There is learning, there is understanding, and
there is acceptance. It is time for you to enter upon the second stage.''

\switchcolumn

``你的改造有三个阶段,''奥布兰说,``分别是学习、理解和接受。现在你应该进入第二个阶段了。''

\switchcolumn*

As always, Winston was lying flat on his back. But of late his bonds
were looser. They still held him to the bed, but he could move his knees
a little and could turn his head from side to side and raise his arms
from the elbow. The dial, also, had grown to be less of a terror. He
could evade its pangs if he was quick-witted enough; it was chiefly when
he showed stupidity that O\textquotesingle Brien pulled the lever.
Sometimes they got through a whole session without use of the dial. He
could not remember how many sessions there had been. The whole process
seemed to stretch out over a long, indefinite time---weeks,
possibly---and the intervals between the sessions might sometimes have
been days, sometimes only an hour or two.

\switchcolumn

像之前一样,温斯顿平躺在床上。最近,带子绑得松了一些,虽然他们依旧将他固定在床上,但他的膝盖可以稍稍活动,脑袋可以转向两边,前臂也可以抬起来了。控制盘没那么吓人了。只要脑筋动得够快,他就能逃避惩罚。通常,只有在他表现愚蠢的情况下,奥布兰才会推动控制杆。有几次的谈话都没有用到过控制盘。他不记得他们的谈话进行了多少回,这个过程似乎拖得很长,没法确定有多久——可能有几个星期,且有时两次谈话间会隔上好几天,也有时只隔上一两个小时。

\switchcolumn*

``As you lie there,'' said O\textquotesingle Brien, ``you have often
wondered---you have even asked me---why the Ministry of Love should
expend so much time and trouble on you. And when you were free you were
puzzled by what was essentially the same question. You could grasp the
mechanics of the society you lived in, but not its underlying motives.
Do you remember writing in your diary, \textquotesingle I understand
\emph{how}; I do not understand \emph{why}\textquotesingle? It was when
you thought about `why' that you
doubted your own sanity. You have read \emph{the book},
Goldstein\textquotesingle s book, or parts of it, at least. Did it tell
you anything that you did not know already?''

\switchcolumn

``当你躺在那里时,''奥布兰说,``你经常想知道,你甚至问过我——为什么仁爱部会在你身上花如此多的时间,费如此大的力。在你还自由的时候,基本上你也为同样的问题困扰。你抓住了你所生活的社会的运转规律,但你还不理解它的根本动机。你还记得你在日记上写的`我明白怎么做,我不明白为什么'?每当你思考`为什么',你就会对自己的理智表示怀疑。你已经看过了高德斯坦因的书,或者,至少看了一部分。它有没有告诉你任何你从前不知道的东西?''

\switchcolumn*

``You have read it?'' said Winston.

\switchcolumn

``你看过那书吗?''温斯顿问。

\switchcolumn*

``I wrote it. That is to say, I collaborated in writing it. No book is
produced individually, as you know.''

\switchcolumn

``那是我写的,也就是说,我参与了它的写作。你知道,没有哪本书是独自一人写出来的。''

\switchcolumn*

``Is it true, what it says?''

\switchcolumn

``书里写的是真的吗?''

\switchcolumn*

``As description, yes. The program it sets forth is nonsense. The secret
accumulation of knowledge---a gradual spread of
enlightenment---ultimately a proletarian rebellion---the overthrow of
the Party. You foresaw yourself that that was what it would say. It is
all nonsense. The proletarians will never revolt, not in a thousand
years or a million. They cannot. I do not have to tell you the reason;
you know it already. If you have ever cherished any dreams of violent
insurrection, you must abandon them. There is no way in which the Party
can be overthrown. The rule of the Party is forever. Make that the
starting point of your thoughts.''

\switchcolumn

``作为一种描述,它是真实的。但它阐述的纲领都毫无意义。悄悄地积累知识——渐渐扩大启蒙的范围——最后促发群众的革命——将党推翻。看这书之前你就知道它会这么说。这都是胡说的。群众永远不会造反,再过一千年、一百万年都不会。他们不能这么做,我也不用告诉你原因,你已经知道了。如果你曾经对暴力革命抱有什么希望,你必须将它丢掉。没有任何方法能把党推翻,党的统治将永远持续下去。要把这个作为你思想的出发点。''

\switchcolumn*

He came closer to the bed. ``Forever!'' he repeated. ``And now let us get
back to the question of `how' and `why.' You understand well enough
\emph{how} the Party maintains itself in power. Now tell me \emph{why} we
cling to power. What is our motive? Why should we want power? Go on,
speak,'' he added as Winston remained silent.

\switchcolumn

他向床走近了一些。``永远!''他重复道,``现在让我们回到这个问题`怎样做'和`为什么'。你已经足够了解党是怎样维系它的权力的。现在告诉我,我们为什么要牢牢地抓住权力?我们的动机是什么?为什么我们那么渴望权力?继续,说说吧。''见温斯顿没说话,他加了一句。

\switchcolumn*

Nevertheless Winston did not speak for another moment or two. A feeling
of weariness had overwhelmed him. The faint, mad gleam of enthusiasm had
come back into O\textquotesingle Brien\textquotesingle s face. He knew
in advance what O\textquotesingle Brien would say: that the Party did
not seek power for its own ends, but only for the good of the majority.
That it sought power because men in the mass were frail, cowardly
creatures who could not endure liberty or face the truth, and must be
ruled over and systematically deceived by others who were stronger than
themselves. That the choice for mankind lay between freedom and
happiness, and that, for the great bulk of mankind, happiness was
better. That the Party was the eternal guardian of the weak, a dedicated
sect doing evil that good might come, sacrificing its own happiness to
that of others. The terrible thing, thought Winston, the terrible thing
was that when O\textquotesingle Brien said this he would believe it. You
could see it in his face. O\textquotesingle Brien knew everything. A
thousand times better than Winston, he knew what the world was really
like, in what degradation the mass of human beings lived and by what
lies and barbarities the Party kept them there. He had understood it
all, weighed it all, and it made no difference: all was justified by the
ultimate purpose. What can you do, thought Winston, against the lunatic
who is more intelligent than yourself, who gives your arguments a fair
hearing and then simply persists in his lunacy?

\switchcolumn

谁想温斯顿又沉默了一两分钟。疲惫的感觉将他淹没。奥布兰的脸上再一次隐约地现出狂热的激情。他已经预知到奥布兰会说些什么。党谋求权力并非为了自己,而是为了大多数人。它追求权力,因为大部分人都软弱怯懦,无力承受自由也不敢面对现实。所以必须要由更加强有力的人来统治他们,有计划地欺骗他们。人类要在自由和幸福之间做出选择,而大部分人都觉得选择幸福要好些。党永远都是弱者的监护者,一伙为了让好事光临而致力于行恶作恶的人,宁可为别人的幸福牺牲自我。这很可怕,温斯顿想,可怕的地方在于在说这些话时,奥布兰打心眼里相信它们是真的。你可以从他的脸上看到这点。奥布兰无所不知,比温斯顿强了一千倍,他知道世界的真实样貌,他知道大多数人的生活都潦倒不堪,他知道党通过谎言和残暴的手段让他们身处那种境况。他了解每件事,权衡每件事,但这不重要,为了终极目的的实现,这些都是合理的。温斯顿想,面对这样一个比你聪明的疯子,这样一个温和地聆听你的观点却坚持着自己疯狂的疯子,你还能做些什么呢?

\switchcolumn*

``You are ruling over us for our own good,'' he said feebly. ``You believe
that human beings are not fit to govern themselves, and therefore---''

\switchcolumn

``你们统治我们是为我们好。''他声音微弱。``你们认为人类不适合自己管理自己,所
以——''

\switchcolumn*

He started and almost cried out. A pang of pain had shot through his
body. O\textquotesingle Brien had pushed the lever of the dial up to
thirty-five.

\switchcolumn

他几乎刚一开口就叫了起来,一阵剧痛穿过他的身体。奥布兰将控制盘上的杆子推到35。

\switchcolumn*

``That was stupid, Winston, stupid!'' he said. ``You should know better
than to say a thing like that.''

\switchcolumn

``真蠢,温斯顿,真蠢!''他说,``你应该知道,你要说得更好才行。''

\switchcolumn*

He pulled the lever back and continued:

\switchcolumn

他将控制杆拉回来,继续道:

\switchcolumn*

``Now I will tell you the answer to my question. It is this. The Party
seeks power entirely for its own sake. We are not interested in the good
of others; we are interested solely in power. Not wealth or luxury or
long life or happiness; only power, pure power. What pure power means
you will understand presently. We are different from all the oligarchies
of the past in that we know what we are doing. All the others, even
those who resembled ourselves, were cowards and hypocrites. The German
Nazis and the Russian Communists came very close to us in their methods,
but they never had the courage to recognize their own motives. They
pretended, perhaps they even believed, that they had seized power
unwillingly and for a limited time, and that just round the corner there
lay a paradise where human beings would be free and equal. We are not
like that. We know that no one ever seizes power with the intention of
relinquishing it. Power is not a means; it is an end. One does not
establish a dictatorship in order to safeguard a revolution; one makes
the revolution in order to establish the dictatorship. The object of
persecution is persecution. The object of torture is torture. The object
of power is power. Now do you begin to understand me?''

\switchcolumn

``现在让我告诉你这个问题的答案。答案是,党追求权力是为了它自己。我们对其他人的利益不感兴趣,我们只关心权力。不是财富,不是奢侈的生活,不是长生不老,不是幸福,而仅仅是权力,纯粹的权力。你很快就会知道什么叫`纯粹的权力'。我们和以往所有的寡头统治者都不同,因为我们知道我们在做什么。而所有其他人,即使是那些和我们很像的人,也都是懦夫和虚伪的人。德国纳粹和俄国共产党有着和我们类似的管理方法,但他们从来都不敢承认他们的动机。他们假装,也许他们真的相信,他们并非自愿夺取权力,而是时间有限,不久就会发生转变,会出现一个人人都自由平等的天堂。我们可不一样。我们明白,没有人出于放弃权力的目的夺取权力。权力不是手段,权力是目的。建立独裁政权不是为了捍卫革命,迫害的目的就是迫害,折磨的目的就是折磨,权力的目的就是权力。现在,你开始明白我的话了吧?''

\switchcolumn*

Winston was struck, as he had been struck before, by the tiredness of
O\textquotesingle Brien\textquotesingle s face. It was strong and fleshy
and brutal, it was full of intelligence and a sort of controlled passion
before which he felt himself helpless; but it was tired. There were
pouches under the eyes, the skin sagged from the cheekbones.
O\textquotesingle Brien leaned over him, deliberately bringing the worn
face nearer.

\switchcolumn

温斯顿被震撼了,正如之前他看到奥布兰那疲惫的面孔后大感震撼一样。那是一张坚强、丰满、残酷的脸,又是一张充满智慧的、克制感情的脸。这张脸让他感到无助,这张脸又如此倦意弥漫。眼下有眼袋,双颊的皮肤松松垮垮。奥布兰侧了侧身,特意让自己那张憔悴的脸离他近些。

\switchcolumn*

``You are thinking,'' he said, ``that my face is old and tired. You are
thinking that I talk of power, and yet I am not even able to prevent the
decay of my own body. Can you not understand, Winston, that the
individual is only a cell? The weariness of the cell is the vigor of the
organism. Do you die when you cut your fingernails?''

\switchcolumn

``你在想,''他说,``你在想我的脸怎么这么老,这么疲倦。你在想,我可以畅谈权力,却不可以阻止自己身体的衰败。你还不明白吗?温斯顿,个体仅仅是一个细胞。一个细胞的衰败是保证机体生命力的前提。你会在剪指甲的时候死掉吗?''

\switchcolumn*

He turned away from the bed and began strolling up and down again, one
hand in his pocket.

\switchcolumn

他离开床,又来来回回地走了起来,将一只手揣在口袋里。

\switchcolumn*

``We are the priests of power,'' he said. ``God is power. But at present
power is only a word so far as you are concerned. It is time for you to
gather some idea of what power means. The first thing you must realize
is that power is collective. The individual only has power in so far as
he ceases to be an individual. You know the Party slogan:
`Freedom is Slavery.' Has it ever
occurred to you that it is reversible? Slavery is freedom.
Alone---free---the human being is always defeated. It must be so,
because every human being is doomed to die, which is the greatest of all
failures. But if he can make complete, utter submission, if he can
escape from his identity, if he can merge himself in the Party so that
he \emph{is} the Party, then he is all-powerful and immortal. The second
thing for you to realize is that power is power over human beings. Over
the body---but, above all, over the mind. Power over matter---external
reality, as you would call it---is not important. Already our control
over matter is absolute.''

\switchcolumn

``我们是权力祭司,''他说,``上帝是权力。但目前对你而言权力只是一个单词。是时候让你领会一些权力的含义了。首先,你必须理解,权力属于集体。个人只有在不是`个人'的情况下才能拥有权力。你知道党的口号是`自由即奴役'。你曾经想过吗,这句话可以颠倒过来?奴役即自由。单独的、自由的人总会被击败。一定是这样,因为每个人都会死,这种失败是所有失败中最大的失败。但如果他能够完全地、绝对地服从,如果他能够从个体的身份中逃脱出来,如果他能够和党融为一体,以至于他就是党,那他就能无所不能,永生不朽。你要明白的第二件事是,权力是指对人的权力,凌驾于身体的,特别是凌驾于思想的权力。权力凌驾于物质——正如你所说的,外在的现实——这不重要。我们已经完全控制了物质。

\switchcolumn*

For a moment Winston ignored the dial. He made a violent effort to raise
himself into a sitting position, and merely succeeded in wrenching his
body painfully.

\switchcolumn

温斯顿有一会儿没有注意控制盘了,他突然想抬起身子坐起来,但他只能痛苦地扭动身体。

\switchcolumn*

``But how can you control matter?'' he burst out. ``You
don\textquotesingle t even control the climate or the law of gravity.
And there are disease, pain, death---''

\switchcolumn

``但是你们怎么能控制物质呢?''他立刻说道,``你们甚至不能控制气候和重力定律,还有疾病、疼痛、死亡——''

\switchcolumn*

O\textquotesingle Brien silenced him by a movement of the hand. ``We
control matter because we control the mind. Reality is inside the skull.
You will learn by degrees, Winston. There is nothing that we could not
do. Invisibility, levitation---anything. I could float off this floor
like a soap bubble if I wished to. I do not wish to, because the Party
does not wish it. You must get rid of those nineteenth-century ideas
about the laws of nature. We make the laws of nature.''

\switchcolumn

奥布兰晃了晃手,打住了他的话。``我们能够控制物质是因为我们控制了思想。现实存在于人的头脑里。你会一点一点地学到的,温斯顿。没有我们做不到的。隐身、腾空——任何事。只要我想,我就能像肥皂泡那样从地板上飘起来,我不想,是因为党不希望我这么做。你要把19世纪那些关于自然规律的想法都抛开,我们才是制定自然规律的人。''

\switchcolumn*

``But you do not! You are not even masters of this planet. What about
Eurasia and Eastasia? You have not conquered them yet.''

\switchcolumn

``但你们没有!你们深知不是这个星球的主人!欧亚国和东亚国呢?你们还没有征服它们。''

\switchcolumn*

``Unimportant. We shall conquer them when it suits us. And if we did not,
what difference would it make? We can shut them out of existence.
Oceania is the world.''

\switchcolumn

``这没什么重要的。我们会征服它们的,时机合适的话。就算我们不这么做,又有什么不同吗?我们能否认它们的存在。大洋国就是世界。''

\switchcolumn*

``But the world itself is only a speck of dust. And man is
tiny---helpless! How long has he been in existence? For millions of
years the earth was uninhabited.''

\switchcolumn

``但世界无非是一颗灰尘。人类渺小而无助!人类才存在多久?几百万年前地球上还没有人呢。''

\switchcolumn*

``Nonsense. The earth is as old as we are, no older. How could it be
older? Nothing exists except through human consciousness.''

\switchcolumn

``胡扯!地球的年纪和我们的一样,它并不比我们老。它怎么可能比我们要老呢?除非经过人类的意识,否则没有什么能存在。''

\switchcolumn*

``But the rocks are full of the bones of extinct animals---mammoths and
mastodons and enormous reptiles which lived here long before man was
ever heard of.''

\switchcolumn

``但岩石里满是已经灭绝的动物的骨头——我曾经听过,猛犸象、乳齿象以及巨大的爬行动物在人类出现前很久就存在了。''

\switchcolumn*

``Have you ever seen those bones, Winston? Of course not.
Nineteenth-century biologists invented them. Before man there was
nothing. After man, if he could come to an end, there would be nothing.
Outside man there is nothing.''

\switchcolumn

``那你见过这些骨头吗,温斯顿?当然没有。它们是19世纪的生物学家捏造出来的。人类出现前什么都没有。在人类消失后——如果人类灭绝了——也什么都不存在了。人类之外别无他物。''

\switchcolumn*

``But the whole universe is outside us. Look at the stars! Some of them
are a million light-years away. They are out of our reach forever.''

\switchcolumn

``可整个宇宙都在我们之外。看看星星!它们中的一些在一百万光年以外的地方。它们在我们永远都到达不了的地方。''

\switchcolumn*

``What are the stars?'' said O\textquotesingle Brien indifferently. ``They
are bits of fire a few kilometers away. We could reach them if we wanted
to. Or we could blot them out. The earth is the center of the universe.
The sun and the stars go round it.''

\switchcolumn

``星星是什么?''奥布兰漠然地说,``它们无非是几公里外的亮光。如果我们想,我们
就能到那儿去。或者,我们可以把它们清理掉。地球是宇宙的中心。太阳和星星都绕着
地球转。''

\switchcolumn*

Winston made another convulsive movement. This time he did not say
anything. O\textquotesingle Brien continued as though answering a spoken
objection:

\switchcolumn

温斯顿抽动了一下。这次他没说任何话。奥布兰继续说了下去,好像在回答一个反对意见。

\switchcolumn*

``For certain purposes, of course, that is not true. When we navigate the
ocean, or when we predict an eclipse, we often find it convenient to
assume that the earth goes round the sun and that the stars are millions
upon millions of kilometers away. But what of it? Do you suppose it is
beyond us to produce a dual system of astronomy? The stars can be near
or distant, according as we need them. Do you suppose our mathematicians
are unequal to that? Have you forgotten doublethink?''

\switchcolumn

``出于某种目的,当然,那不是真的。当我们在大海上航行的时候,或者在我们预测日食和月食时,我们经常发现,假设地球绕着太阳转,星星在亿万公里之外会非常方便。可这又算得了什么?你以为创造两套天文体系超出我们的能力了吗?星星可以近,也可以远,按照我们的需要来。你以为我们的数学家做不到吗?你把双重思想忘掉了吗?''

\switchcolumn*

Winston shrank back upon the bed. Whatever he said, the swift answer
crushed him like a bludgeon. And yet he knew, he \emph{knew}, that he
was in the right. The belief that nothing exists outside your own
mind---surely there must be some way of demonstrating that it was false.
Had it not been exposed long ago as a fallacy? There was even a name for
it, which he had forgotten. A faint smile twitched the corners of
O\textquotesingle Brien\textquotesingle s mouth as he looked down at
him.

\switchcolumn

床上,温斯顿缩起身子。不管他说什么,迅猛而来的答案都像棍子一样将他击倒。而他仍然知道,知道自己是对的。``在你的思想以外别无他物''的观念——一定有办法被证明是错误的吗?不是很久以前就被揭露说这是个谬论吗?它还有个名字,他忘了。奥布兰低头看着温斯顿,嘴角上出现一抹微笑。

\switchcolumn*

``I told you, Winston,'' he said, ``that metaphysics is not your strong
point. The word you are trying to think of is solipsism. But you are
mistaken. This is not solipsism. Collective solipsism, if you like. But
that is a different thing; in fact, the opposite thing. All this is a
digression,'' he added in a different tone. ``The real power, the power we
have to fight for night and day, is not power over things, but over
men.'' He paused, and for a moment assumed again his air of a
schoolmaster questioning a promising pupil: ``How does one man assert his
power over another, Winston?''

\switchcolumn

``我和你说过,温斯顿,''他说,``玄学不是你的强项。你尝试记起的那个词是唯我论。但你错了。这不是唯我论,这是集体唯我论。这可不是一回事,事实上,这是相反的事。这些都是题外话了。''他用一种不同的语调说。``真正的权力,我们整日整夜为之战斗的权力,不是凌驾于物体的权力,而是凌驾于人的权力。''他停了下来,有一会儿,他又换上了那副神情,好像一个老师在向一个有前途的学生提问:``一个人如何对另一个人施以权力,温斯顿?''

\switchcolumn*

Winston thought. ``By making him suffer,'' he said.

\switchcolumn

温斯顿思考了一下,``通过让他受苦。''他说。

\switchcolumn*

``Exactly. By making him suffer. Obedience is not enough. Unless he is
suffering, how can you be sure that he is obeying your will and not his
own? Power is in inflicting pain and humiliation. Power is in tearing
human minds to pieces and putting them together again in new shapes of
your own choosing. Do you begin to see, then, what kind of world we are
creating? It is the exact opposite of the stupid hedonistic Utopias that
the old reformers imagined. A world of fear and treachery and torment, a
world of trampling and being trampled upon, a world which will grow not
less but \emph{more} merciless as it refines itself. Progress in our
world will be progress toward more pain. The old civilizations claimed
that they were founded on love or justice. Ours is founded upon hatred.
In our world there will be no emotions except fear, rage, triumph, and
self-abasement. Everything else we shall destroy---everything. Already
we are breaking down the habits of thought which have survived from
before the Revolution. We have cut the links between child and parent,
and between man and man, and between man and woman. No one dares trust a
wife or a child or a friend any longer. But in the future there will be
no wives and no friends. Children will be taken from their mothers at
birth, as one takes eggs from a hen. The sex instinct will be
eradicated. Procreation will be an annual formality like the renewal of
a ration card. We shall abolish the orgasm. Our neurologists are at work
upon it now. There will be no loyalty, except loyalty toward the Party.
There will be no love, except the love of Big Brother. There will be no
laughter, except the laugh of triumph over a defeated enemy. There will
be no art, no literature, no science. When we are omnipotent we shall
have no more need of science. There will be no distinction between
beauty and ugliness. There will be no curiosity, no enjoyment of the
process of life. All competing pleasures will be destroyed. But
always---do not forget this, Winston---always there will be the
intoxication of power, constantly increasing and constantly growing
subtler. Always, at every moment, there will be the thrill of victory,
the sensation of trampling on an enemy who is helpless. If you want a
picture of the future, imagine a boot stamping on a human
face---forever.''

\switchcolumn

``一点不错。通过让他受苦。只有服从是不够的,除非让他受苦,否则你要如何确定他是服从于你的意志,而不是他自己的?权力就是要让人处在痛苦和屈辱中。权力就是要将人类的意志撕成碎片,再按照你想要的新的样子将它们重新粘起来。你是不是已经开始明白我们要创造一个怎样的世界?和过去那些改革家所畅想的愚蠢的、享乐主义的乌托邦世界刚好相反。这是一个被恐惧和背叛折磨的世界,一个充斥着践踏与被践踏的世界,一个越来越好也越来越残酷的世界。旧时代的文明自称其建立之基是爱与正义。而我们的文明则建立在仇恨之上。在我们的世界里,除了恐惧、愤怒、耀武扬威和自我贬低外,什么情感都不存在,我们会将它们统统毁灭。革命前幸存下来的思想习惯已经被我们消灭。我们割断了联结孩子和父母的纽带,割断了人与人、男人与女人的联系。没有人再敢相信妻子、孩子、朋友,且未来也不会再有妻子和朋友。孩子一出生就要离开母亲,就好像从母鸡那里拿走鸡蛋。性本能将被彻底消灭,生殖将成为更新配给证一样一年一度的形式。我们要消除性兴奋。我们的神经学专家正在做这件事。除了对党的忠诚,没有其他的忠诚,除了对老大哥的爱,没有其他的爱,除了为击败敌人而笑,没有其他的笑。艺术、文学、科学都不会有了。当我们无所不能时,我们也就不再需要什么科学。美和丑也不再有区别。不再有好奇心,在生命的进程中也不再有喜悦。所有其他类型的欢乐都要被摧毁。但是,温斯顿,别忘了,对权力的迷醉将永远存在,且这种迷醉还会越来越强烈,越来越敏感。永远,每时每刻都会有取得胜利的狂喜以及践踏一个没有还手之力的敌人的激情。如果你把未来想成一幅画,那就想象一只皮靴正踩在一个人的脸上——永远都是如此。''

\switchcolumn*

He paused as though he expected Winston to speak. Winston had tried to
shrink back into the surface of the bed again. He could not say
anything. His heart seemed to be frozen. O\textquotesingle Brien went
on:

\switchcolumn

他停下来,好像在等温斯顿说些什么。温斯顿却又一次地想往床底下钻,他没说话,心脏就像结冰一般。奥布兰继续说:

\switchcolumn*

``And remember that it is forever. The face will always be there to be
stamped upon. The heretic, the enemy of society, will always be there,
so that he can be defeated and humiliated over again. Everything that
you have undergone since you have been in our hands---all that will
continue, and worse. The espionage, the betrayals, the arrests, the
tortures, the executions, the disappearances will never cease. It will
be a world of terror as much as a world of triumph. The more the Party
is powerful, the less it will be tolerant; the weaker the opposition,
the tighter the despotism. Goldstein and his heresies will live forever.
Every day, at every moment, they will be defeated, discredited,
ridiculed, spat upon---and yet they will always survive. This drama that
I have played out with you during seven years will be played out over
and over again, generation after generation, always in subtler forms.
Always we shall have the heretic here at our mercy, screaming with pain,
broken up, contemptible---and in the end utterly penitent, saved from
himself, crawling to our feet of his own accord. That is the world that
we are preparing, Winston. A world of victory after victory, triumph
after triumph after triumph: an endless pressing, pressing, pressing
upon the nerve of power. You are beginning, I can see, to realize what
that world will be like. But in the end you will do more than understand
it. You will accept it, welcome it, become part of it.''

\switchcolumn

``记住,永远都会这样。那张脸总会在那里让人踩踏。异端,还有社会的敌人,永远都会待在那里,你可以一次又一次地击败他们、侮辱他们。你落到我们手里后经历的每件事——都将持续下去,甚至比这更坏。间谍活动、各种叛变、逮捕、虐待、处决、失踪,所有这些都不会停止。这既是个充斥着恐惧的世界,也是个充斥着狂喜的世界。党越强大,容忍度就越低,反抗的力度越弱,专制的程度就越强。高德斯坦因和他的异端说将永远存在。每一天,每一分钟,它们都会被攻击、被质疑、被嘲弄、被唾弃,然而它们又会一直留存下来。我和你在这七年里上演的这出戏会反反复复的,一代人接着一代人地演下去,但总的形式会更加巧妙。我们总是将异端带到仁爱部来,让他们因疼痛而叫喊,让他们精神崩溃,卑鄙可耻——最后翻然悔悟,拯救自己,心甘情愿地匍匐于我们脚下。这就是我们所准备的世界,温斯顿。这是一个胜利接着一个胜利的世界,是一个狂喜接着一个狂喜的世界,是无休无止的压迫着权力神经的世界。我能看出来,你开始明白了,意识到这世界将成为什么样子。而最后你不止会理解它,你还会接受它、欢迎它,成为它的一部分。''

\switchcolumn*

Winston had recovered himself sufficiently to speak. ``You
can\textquotesingle t!'' he said weakly.

\switchcolumn

温斯顿已经恢复得可以说话了。``你们不能!''他声音微弱。

\switchcolumn*

``What do you mean by that remark, Winston?''

\switchcolumn

``你这话是什么意思,温斯顿?''

\switchcolumn*

``You could not create such a world as you have just described. It is a
dream. It is impossible.''

\switchcolumn

``你们创造不出你刚才描绘的那个世界。这只是不可能实现的梦。''

\switchcolumn*

``Why?''

\switchcolumn

``为什么?''

\switchcolumn*

``It is impossible to found a civilization on fear and hatred and
cruelty. It would never endure.''

\switchcolumn

``因为不可能把文明建立在恐惧、仇恨和残忍上。这样不能持久。''

\switchcolumn*

``Why not?''

\switchcolumn

``为什么不能?''

\switchcolumn*

``It would have no vitality. It would disintegrate. It would commit
suicide.''

\switchcolumn

``它没有生命力,它会解体,它一定会自己毁灭掉自己。''

\switchcolumn*

``Nonsense. You are under the impression that hatred is more exhausting
than love. Why should it be? And if it were, what difference would that
make? Suppose that we choose to wear ourselves out faster. Suppose that
we quicken the tempo of human life till men are senile at thirty. Still
what difference would it make? Can you not understand that the death of
the individual is not death? The Party is immortal.''

\switchcolumn

``胡扯!你依然有这样的印象,觉得仇恨比爱更耗气力。为什么它应该这样呢?即使它真的这样了,又有什么不同吗?假设我们选择让自己更加迅速地耗光力气,假设我们让人生的速度加快,让人一到30岁就衰老,这又能怎样呢?你还不明白吗?个人的死亡算不上死亡,党是永生的。''

\switchcolumn*

As usual, the voice had battered Winston into helplessness. Moreover he
was in dread that if he persisted in his disagreement
O\textquotesingle Brien would twist the dial again. And yet he could not
keep silent. Feebly, without arguments, with nothing to support him
except his inarticulate horror of what O\textquotesingle Brien had said,
he returned to the attack.

\switchcolumn

温斯顿一如既往地被奥布兰的说法打击得无可奈何,此外,他也害怕,害怕若继续坚持自己的意见,奥布兰会再次打开控制盘。但他依然无法保持沉默。除了对奥布兰所说的感到无法形容的恐惧外,他找不到支持他这么说的理由,在没有论据的情况下,他无力地进行了回击。

\switchcolumn*

``I don\textquotesingle t know---I don\textquotesingle t care. Somehow
you will fail. Something will defeat you. Life will defeat you.''

\switchcolumn

``我不知道——我不管。不管怎样你们都会失败。会有什么东西击败你们的,生活就会击败你们。''

\switchcolumn*

``We control life, Winston, at all its levels. You are imagining that
there is something called human nature which will be outraged by what we
do and will turn against us. But we create human nature. Men are
infinitely malleable. Or perhaps you have returned to your old idea that
the proletarians or the slaves will arise and overthrow us. Put it out
of your mind. They are helpless, like the animals. Humanity is the
Party. The others are outside---irrelevant.''

\switchcolumn

``我们掌控生活,温斯顿,掌控它的全部。你对那个被称作`人性'的东西心存幻想,这让你对我们的所作所为感到愤怒,并让你反对我们。但我们是人性的缔造者,人的可塑性是无限的。也许你又回到了你的老思路上去,觉得群众或者奴隶会起来推翻我们。快把这想法从你的脑子里拿出去吧。他们就像动物一样无助。党就是人性,其他的都是表面之物——不相干。''

\switchcolumn*

``I don\textquotesingle t care. In the end they will beat you. Sooner or
later they will see you for what you are, and then they will tear you to
pieces.''

\switchcolumn

``我不管,最后,他们会击败你们。他们早晚会看清你们,然后他们就会把你们撕成碎片。''

\switchcolumn*

``Do you see any evidence that that is happening? Or any reason why it
should?''

\switchcolumn

``你看到什么证据证明它正在发生吗?或者任何它会这样做的理由?''

\switchcolumn*

``No. I believe it. I \emph{know} that you will fail. There is something
in the universe---I don\textquotesingle t know, some spirit, some
principle---that you will never overcome.''

\switchcolumn

``没有,但我相信会这样。我知道你们会失败。宇宙里有什么东西——我也说不清,某种精神,某种原则——你们永远不能战胜的东西。''

\switchcolumn*

``Do you believe in God, Winston?''

\switchcolumn

``你信上帝吗,温斯顿?''

\switchcolumn*

``No.''

\switchcolumn

``不信。''

\switchcolumn*

``Then what is it, this principle that will defeat us?''

\switchcolumn

``那么,那个会击败我们的原则是什么?''

\switchcolumn*

``I don\textquotesingle t know. The spirit of Man.''

\switchcolumn

``我不知道。是人的精神。''

\switchcolumn*

``And do you consider yourself a man?''

\switchcolumn

``你确信你是人吗?''

\switchcolumn*

``Yes.''

\switchcolumn

``是的。''

\switchcolumn*

``If you are a man, Winston, you are the last man. Your kind is extinct;
we are the inheritors. Do you understand that you are \emph{alone}? You
are outside history, you are nonexistent.'' His manner changed and he
said more harshly: ``And you consider yourself morally superior to us,
with our lies and our cruelty?''

\switchcolumn

``假如你是人,温斯顿,你就是最后一个人了。你这种人已经灭绝了。我们是后来的人。你没意识到你是孤单的吗?你身处历史之外,你是不存在的。''他的态度发生了变化,说起话来更严厉了,``你觉得你在道德上比我们要优越,因为我们又撒谎又残酷。''

\switchcolumn*

``Yes, I consider myself superior.''

\switchcolumn

``是的,我觉得我有这个优越感。''

\switchcolumn*

O\textquotesingle Brien did not speak. Two other voices were speaking.
After a moment Winston recognized one of them as his own. It was a sound
track of the conversation he had had with O\textquotesingle Brien, on
the night when he had enrolled himself in the Brotherhood. He heard
himself promising to lie, to steal, to forge, to murder, to encourage
drug taking and prostitution, to disseminate venereal diseases, to throw
vitriol in a child\textquotesingle s face. O\textquotesingle Brien made
a small impatient gesture, as though to say that the demonstration was
hardly worth making. Then he turned a switch and the voices stopped.

\switchcolumn

奥布兰没说话,另外两个人说了起来。过了一会儿,温斯顿意识到其中一个人是他自己。那是他在兄弟会登记的那个晚上,和奥布兰谈话的录音。他听到自己正在承诺,他会撒谎、会偷盗、会造假、会杀人、会鼓励吸毒和卖淫、会去传播性病、会朝小孩子的脸上泼硫酸。奥布兰不耐烦地做了个小手势,就好像在说还不值得这样做。接着,他转了一下按钮,声音停止了。

\switchcolumn*

``Get up from that bed,'' he said.

\switchcolumn

``从床上起来。''他说。

\switchcolumn*

The bonds had loosened themselves. Winston lowered himself to the floor
and stood up unsteadily.

\switchcolumn

绑住温斯顿的带子自己松开了,温斯顿下了床,晃晃悠悠地站在那里。

\switchcolumn*

``You are the last man,'' said O\textquotesingle Brien. ``You are the
guardian of the human spirit. You shall see yourself as you are. Take
off your clothes.''

\switchcolumn

``你是最后一个人,''奥布兰说,``你是人类精神的守护者。你会看到你是怎样一幅尊容。把衣服脱了。''

\switchcolumn*

Winston undid the bit of string that held his overalls together. The zip
fastener had long since been wrenched out of them. He could not remember
whether at any time since his arrest he had taken off all his clothes at
one time. Beneath the overalls his body was looped with filthy yellowish
rags, just recognizable as the remnants of underclothes. As he slid them
to the ground he saw that there was a three-sided mirror at the far end
of the room. He approached it, then stopped short. An involuntary cry
had broken out of him.

\switchcolumn

温斯顿解开系工作服的带子。原本用来扣住衣服的拉链早就被扯下来了。他记不清在被捕之后有没有脱掉过衣服。工作服之下,他的身体上有些脏兮兮的淡黄色的碎布,隐约可以辨出那是内衣的碎片。就在他将衣服扔到地板上时,他看到屋子的尽头有一个三面镜。他向镜子走去,可没走多久就停下脚步,忍不住发出惊叫。

\switchcolumn*

``Go on,'' said O\textquotesingle Brien. ``Stand between the wings of the
mirror. You shall see the side view as well.''

\switchcolumn

``继续走,''奥布兰说,``站在三扇镜子的中间,这样你还能看到你侧面的样子。''

\switchcolumn*

He had stopped because he was frightened. A bowed, gray-colored,
skeletonlike thing was coming toward him. Its actual appearance was
frightening, and not merely the fact that he knew it to be himself. He
moved closer to the glass. The creature\textquotesingle s face seemed to
be protruded, because of its bent carriage. A forlorn,
jailbird\textquotesingle s face with a nobby forehead running back into
a bald scalp, a crooked nose and battered-looking cheekbones above which
the eyes were fierce and watchful. The cheeks were seamed, the mouth had
a drawn-in look. Certainly it was his own face, but it seemed to him
that it had changed more than he had changed inside. The emotions it
registered would be different from the ones he felt. He had gone
partially bald. For the first moment he had thought that he had gone
gray as well, but it was only the scalp that was gray. Except for his
hands and a circle of his face, his body was gray all over with ancient,
ingrained dirt. Here and there under the dirt there were the red scars
of wounds, and near the ankle the varicose ulcer was an inflamed mass
with flakes of skin peeling off it. But the truly frightening thing was
the emaciation of his body. The barrel of the ribs was as narrow as that
of a skeleton; the legs had shrunk so that the knees were thicker than
the thighs. He saw now what O\textquotesingle Brien had meant about
seeing the side view. The curvature of the spine was astonishing. The
thin shoulders were hunched forward so as to make a cavity of the chest,
the scraggy neck seemed to be bending double under the weight of the
skull. At a guess he would have said that it was the body of a man of
sixty, suffering from some malignant disease.

\switchcolumn

他之所以要停下来是因为他害怕了。镜子里一个佝偻着的、灰白色的,犹如骷髅一般的东西正向他走来。这情形着实吓人,不仅因为他知道他面对的正是自己,他向镜子挪近,由于弯着身子,那家伙的脸看起来更突出了,那是一张绝望的囚犯的脸,高高的额角和被刮得干干净净的脑袋连在一起,鼻子弯弯的,颧骨好像被人打过,而颧骨上面则是一双目光凌厉又充满警惕的眼睛。他的脸颊上布满皱纹,嘴巴也陷了进去。毫无疑问,这就是他的脸,但它的变化似乎比他内心的变化还要大。这张脸表现出的情感和他真正感受到的并不一样。他有些秃顶了,一开始他觉得自己的头发变成了灰白色,但后来发现那原来是头皮的颜色。除了手和脸的周围,他整个身体都是灰白色的,覆满了肮脏的污垢。而在这些污垢下面,还长满了红色的伤疤。在他的脚踝旁那静脉曲张所导致的溃疡已经感染发炎,皮屑脱落。但真正可怕的是他身体的消瘦程度。他的肋骨细细的,像骷髅一般;他的腿瘦瘦的,甚至没有膝盖粗。现在,他终于明白为什么奥布兰让他看看自己的侧面。他的脊椎弯得让人害怕,他的肩膀如此羸弱向前耸出,致使胸口前形成一个空洞,他的脖子瘦骨嶙峋,几乎被头的重量压成对折。如果让他猜,他一定会说这是一个六十岁的得了恶性疾病的老人的身体。

\switchcolumn*

``You have thought sometimes,'' said O\textquotesingle Brien, ``that my
face---the face of a member of the Inner Party---looks old and worn.
What do you think of your own face?''

\switchcolumn

``有时,你会想,''奥布兰说,``我这张内党党员的脸看起来那么苍老憔悴。那么,你对你自己的脸又有什么看法呢?''

\switchcolumn*

He seized Winston\textquotesingle s shoulder and spun him round so that
he was facing him.

\switchcolumn

他抓住温斯顿的肩膀,把他转过来,让他正对着自己。

\switchcolumn*

``Look at the condition you are in!'' he said. ``Look at this filthy grime
all over your body. Look at the dirt between your toes. Look at that
disgusting running sore on your leg. Do you know that you stink like a
goat? Probably you have ceased to notice it. Look at your emaciation. Do
you see? I can make my thumb and forefinger meet around your bicep. I
could snap your neck like a carrot. Do you know that you have lost
twenty-five kilograms since you have been in our hands? Even your hair
is coming out in handfuls. Look!'' He plucked at
Winston\textquotesingle s head and brought away a tuft of hair. ``Open
your mouth. Nine, ten, eleven teeth left. How many had you when you came
to us? And the few you have left are dropping out of your head. Look
here!''

\switchcolumn

``看看你镜子里的样子!''他说,``看看你身上的污垢。看看你肮脏的脚趾中的灰尘。看看你腿上这令人恶心的伤口。你知道你臭得像头山羊吗?也许你已经注意不到这些了。看看你瘦了吧唧的样儿,看到了吗?我用大拇指和食指就能拢住你的胳膊,我掐断你的脖子就像掐一根胡萝卜。你知道吗,自从落到我们手里,你已经掉了二十五公斤的体重?就连头发都掉了一大把。看!''他在温斯顿的头上揪了一把,揪下来一大撮头发。``张开你的嘴,还有9、10、11颗牙,你到我们这儿来的时候有几颗?剩下的那几颗正从你的嘴里脱落。看这儿!''

\switchcolumn*

He seized one of Winston\textquotesingle s remaining front teeth between
his powerful thumb and forefinger. A twinge of pain shot through
Winston\textquotesingle s jaw. O\textquotesingle Brien had wrenched the
loose tooth out by the roots. He tossed it across the cell.

\switchcolumn

他用大拇指和食指抓住温斯顿剩下的一颗门牙,用力一扳。温斯顿的上颚一阵剧痛。奥布兰把那颗牙连根扳了下来,扔到牢房的另一边。

\switchcolumn*

``You are rotting away,'' he said; ``you are falling to pieces. What are
you? A bag of filth. Now turn round and look into that mirror again. Do
you see that thing facing you? That is the last man. If you are human,
that is humanity. Now put your clothes on again.''

\switchcolumn

``你正在腐烂,''他说,``你正变成碎片。你算什么呢?你是一堆垃圾。现在转过身去,再往镜子里看看。你看到面对着你的东西吗?那就是最后一个人。如果你是人,那就是人性。现在把你的衣服穿起来吧。''

\switchcolumn*

Winston began to dress himself with slow stiff movements. Until now he
had not seemed to notice how thin and weak he was. Only one thought
stirred in his mind: that he must have been in this place longer than he
had imagined. Then suddenly as he fixed the miserable rags round himself
a feeling of pity for his ruined body overcame him. Before he knew what
he was doing he had collapsed onto a small stool that stood beside the
bed and burst into tears. He was aware of his ugliness, his
gracelessness, a bundle of bones in filthy underclothes sitting weeping
in the harsh white light; but he could not stop himself.
O\textquotesingle Brien laid a hand on his shoulder, almost kindly.

\switchcolumn

温斯顿艰难地、慢慢地穿上了衣服。直到现在,他好像都没注意到自己居然这样瘦弱。他只有一个想法:他在这里待的时间一定比他以为的要长。他将这些脏兮兮的破布裹在身上,突然可怜起自己这被摧毁的身体,并被这种感觉压倒。在意识到自己做什么之前他就崩溃了,他坐在床边的小凳子上放声大哭。他意识到自己丑陋至极,毫无廉耻,是被肮脏的衣服包裹起来的正在明晃晃的灯光下哭泣的骨头。他不能控制自己。奥布兰温和地将一只手搭在他的肩膀上。

\switchcolumn*

``It will not last forever,'' he said. ``You can escape from it whenever
you choose. Everything depends on yourself.''

\switchcolumn

``不会永远这样的。''他说,``无论什么时候,你决定好了,你就可以从这里离开。所有的一切都取决于你自己。''

\switchcolumn*

``You did it!'' sobbed Winston. ``You reduced me to this state.''

\switchcolumn

``你们做的!''温斯顿抽泣着说,``你让我落到了这个地步。''

\switchcolumn*

``No, Winston, you reduced yourself to it. This is what you accepted when
you set yourself up against the Party. It was all contained in that
first act. Nothing has happened that you did not foresee.''

\switchcolumn

``不,温斯顿,是你自己让自己落到这个地步的。在你决心和党作对时,你就接受了这样的命运。你最初的行为里就包含了这点。没有哪件事的发生是你没有预料到的。''

\switchcolumn*

He paused, and then went on:

\switchcolumn

他停了一会儿,又继续说道:

\switchcolumn*

``We have beaten you, Winston. We have broken you up. You have seen what
your body is like. Your mind is in the same state. I do not think there
can be much pride left in you. You have been kicked and flogged and
insulted, you have screamed with pain, you have rolled on the floor in
your own blood and vomit. You have whimpered for mercy, you have
betrayed everybody and everything. Can you think of a single degradation
that has not happened to you?''

\switchcolumn

``我们把你打败了,温斯顿。我们摧毁了你,你可以看到你的身体像什么样子,你的意识也是一样。我不认为你还有什么引以为豪的东西。你被踢过,被鞭打过,被羞辱过,你因为疼痛而尖叫。你在地板上,在你的血和呕吐物中翻滚,你祈求饶恕,你背叛了所有人所有事。你还能想到有哪件耻辱的事在你身上没发生过吗?''

\switchcolumn*

Winston had stopped weeping, though the tears were still oozing out of
his eyes. He looked up at O\textquotesingle Brien.

\switchcolumn

温斯顿停止了哭泣,但仍有眼泪从他的眼睛里涌出来。他看着奥布兰。

\switchcolumn*

``I have not betrayed Julia,'' he said.

\switchcolumn

``我没有背叛朱莉亚。''他说。

\switchcolumn*

O\textquotesingle Brien looked down at him thoughtfully. ``No,'' he said,
``no; that is perfectly true. You have not betrayed Julia.''

\switchcolumn

奥布兰低头看着他,沉思着。``是的,''他说,``是的,这完全是事实。你没有背叛朱莉亚。''

\switchcolumn*

The peculiar reverence for O\textquotesingle Brien, which nothing seemed
able to destroy, flooded Winston\textquotesingle s heart again. How
intelligent, he thought, how intelligent! Never did
O\textquotesingle Brien fail to understand what was said to him. Anyone
else on earth would have answered promptly that he \emph{had} betrayed
Julia. For what was there that they had not screwed out of him under the
torture? He had told them everything he knew about her, her habits, her
character, her past life; he had confessed in the most trivial detail
everything that had happened at their meetings, all that he had said to
her and she to him, their black-market meals, their adulteries, their
vague plottings against the Party---everything. And yet, in the sense in
which he intended the word, he had not betrayed her. He had not stopped
loving her; his feeling toward her had remained the same.
O\textquotesingle Brien had seen what he meant without the need for
explanation.

\switchcolumn

温斯顿的心里再次充满了对奥布兰的崇敬之情,似乎没有什么东西能将他摧毁。多么理智,他想,多么理智!奥布兰没有一次不理解他的话。除了他,地球上任何一个人都会立即回答他,他已经背叛了朱莉亚。因为在拷打之下,他们还有什么没从他嘴里榨出来的呢?他将他所知道的关于她的每件事都告诉给他们。她的习惯,她的特点,她以往的人生;他还坦白了他们约会时发生的所有事情的所有细节,他们交谈的内容,他们在黑市上吃饭,他们通奸,他们模模糊糊的反党计划——所有的一切。但是,以他说话的意图来看,他没有背叛她。他没有停止爱她,他对她的感情始终如一。无须解释,奥布兰就明白了他的意思。

\switchcolumn*

``Tell me,'' he said, ``how soon will they shoot me?''

\switchcolumn

``告诉我,''他说,``还有多久会枪毙我?''

\switchcolumn*

``It might be a long time,'' said O\textquotesingle Brien. ``You are a
difficult case. But don\textquotesingle t give up hope. Everyone is
cured sooner or later. In the end we shall shoot you.''

\switchcolumn

``也许要等上很长时间,''奥布兰说,``你的情况很麻烦,但别放弃希望。每个人都会被治好的,或早或晚。最后,我们会枪毙你。''

\switchcolumn*


\section{IV}\label{iv-2}

\switchcolumn

\section*{四}\label{ux4e8cux5341ux4e8c}

\switchcolumn*

He was much better. He was growing fatter and stronger every day, if it
was proper to speak of days.

\switchcolumn

他好多了。每天,他都在变胖、变强壮,如果``每天''这个词说起来合适的话。

\switchcolumn*

The white light and the humming sound were the same as ever, but the
cell was a little more comfortable than the others he had been in. There
were a pillow and a mattress on the plank bed, and a stool to sit on.
They had given him a bath, and they allowed him to wash himself fairly
frequently in a tin basin. They even gave him warm water to wash with.
They had given him new underclothes and a clean suit of overalls. They
had dressed his varicose ulcer with soothing ointment. They had pulled
out the remnants of his teeth and given him a new set of dentures.

\switchcolumn

白色的灯光和嗡嗡的声音和之前的一样,但牢房却比他待过的要舒服一点儿。木板床上铺了床垫,放了枕头,床的边上还有凳子可以坐一坐。他们给他洗了澡,允许他不时便在锡质的盆子里冲洗一下。他们甚至给他温水供他梳洗,还给了他一套崭新的内衣和一身干净的工作服。他们在他静脉曲张引起的伤口上涂上镇痛药膏,拔掉了他剩下的牙齿,给他换上了假牙。

\switchcolumn*

Weeks or months must have passed. It would have been possible now to
keep count of the passage of time, if he had felt any interest in doing
so, since he was being fed at what appeared to be regular intervals. He
was getting, he judged, three meals in the twenty-four hours; sometimes
he wondered dimly whether he was getting them by night or by day. The
food was surprisingly good, with meat at every third meal. Once there
was even a packet of cigarettes. He had no matches, but the
never-speaking guard who brought his food would give him a light. The
first time he tried to smoke it made him sick, but he persevered, and
spun the packet out for a long time, smoking half a cigarette after each
meal.

\switchcolumn

几个星期或者几个月过去了。只要他想,现在他可以算一算到底过去了多长时间,因为他们是定时送饭。据他推测,每二十四小时吃三顿饭,有时他不大清楚他到底是在晚上还是白天吃的饭。食物好得让人吃惊,每三顿中便有一顿是有肉的。一次,他甚至还得到了一包香烟。虽然他没有火柴,但给他送饭的那个总也不说话的看守却为他点了火。他尝试着抽了第一口烟,觉得很恶心,但他最终坚持了下来。这盒烟抽了很久,他总会在饭后抽上半根。

\switchcolumn*

They had given him a white slate with a stump of pencil tied to the
corner. At first he made no use of it. Even when he was awake he was
completely torpid. Often he would lie from one meal to the next almost
without stirring, sometimes asleep, sometimes waking into vague reveries
in which it was too much trouble to open his eyes. He had long grown
used to sleeping with a strong light on his face. It seemed to make no
difference, except that one\textquotesingle s dreams were more coherent.
He dreamed a great deal all through this time, and they were always
happy dreams. He was in the Golden Country, or he was sitting among
enormous, glorious, sunlit ruins, with his mother, with Julia, with
O\textquotesingle Brien---not doing anything, merely sitting in the sun,
talking of peaceful things. Such thoughts as he had when he was awake
were mostly about his dreams. He seemed to have lost the power of
intellectual effort, now that the stimulus of pain had been removed. He
was not bored; he had no desire for conversation or distraction. Merely
to be alone, not to be beaten or questioned, to have enough to eat, and
to be clean all over, was completely satisfying.

\switchcolumn

他们交给他一块白色的板子,板子的一角系着一根铅笔。一开始,他没管它。就算在清醒时,他也是一副无精打采的样子。他经常吃完一顿饭就躺下来,毫无活力地等着下一顿饭,有时睡着了,有时虽醒着却神情恍惚地在幻想,连眼皮都不愿意睁,他早就习惯睡觉时有强光打在脸上,这除了会让一个人的梦境更加清楚外,和关灯睡觉没什么不同。这段时间,他的梦很多,还总是让人快乐的梦。他梦见自己在黄金乡,坐在一个巨大的沐浴着阳光的废墟中,和母亲、朱莉亚及奥布兰在一起——什么也不做,就那么坐着晒太阳,讲着寻常的话。在清醒时,他所想的也大多是他的梦。疼痛的刺激消失了,他看起来已经失去了理智思考的能力,不是厌倦,他只是不想说话,不想分心。他独自一人待着,不被殴打也不被审问,吃的东西够多,哪里都够干净,这真令人心满意足。

\switchcolumn*

By degrees he came to spend less time in sleep, but he still felt no
impulse to get off the bed. All he cared for was to lie quiet and feel
the strength gathering in his body. He would finger himself here and
there, trying to make sure that it was not an illusion that his muscles
were growing rounder and his skin tauter. Finally it was established
beyond a doubt that he was growing fatter; his thighs were now
definitely thicker than his knees. After that, reluctantly at first, he
began exercising himself regularly. In a little while he could walk
three kilometers, measured by pacing the cell, and his bowed shoulders
were growing straighter. He attempted more elaborate exercises, and was
astonished and humiliated to find what things he could not do. He could
not move out of a walk, he could not hold his stool out at
arm\textquotesingle s length, he could not stand on one leg without
falling over. He squatted down on his heels, and found that with
agonizing pains in thigh and calf he could just lift himself to a
standing position. He lay flat on his belly and tried to lift his weight
by his hands. It was hopeless; he could not raise himself a centimeter.
But after a few more days---a few more mealtimes---even that feat was
accomplished. A time came when he could do it six times running. He
began to grow actually proud of his body, and to cherish an intermittent
belief that his face also was growing back to normal. Only when he
chanced to put his hand on his bald scalp did he remember the seamed,
ruined face that had looked back at him out of the mirror.

\switchcolumn

渐渐的,他花在睡觉上的时间少了,但他仍不愿意从床上起来。他想安安静静地躺着,好好感受身体里力量的蓄积。他喜欢用手摸摸这里,弄弄那里,以确定这不是幻觉,他的肌肉圆滚滚地增长着,皮肤也变得紧致。最后毫无疑问,他正在长胖。现在,他的大腿着实要比膝盖粗了。在这之后,他开始定期运动,起初有些勉强,用不了多久他就能走上三公里,这是用步测牢房得出的数据。他弯曲的肩膀正在挺直,他尝试着做复杂一些的体操,但他既惊讶又不好意思地发现,有些动作他做不到。他跑不起来,举不起板凳,只要单脚站立就会摔倒。他脚后跟着力,蹲下身去,谁想大腿和小腿都疼得让人忍耐不得,迫使他不得不再站起身来。他俯卧在地上,想用手臂撑起身体,结果连一毫米都撑不起来,无可奈何。不过,仅仅过了几天——或者说几顿饭的工夫——他就做成功了,有一回,他一下子就做了六次。他开始切切实实地为自己的身体自豪,有时他相信他的脸也变回了原来的样子。只有当他把手放在他光秃秃的脑袋上时,他才会想起那张从镜子里向他张望的布满皱纹又伤痕累累的脸。

\switchcolumn*

His mind grew more active. He sat down on the plank bed, his back
against the wall and the slate on his knees, and set to work
deliberately at the task of re-educating himself.

\switchcolumn

他的意识活跃多了。他在木板床上坐着,背靠着墙,把那块板子放到了膝盖上,然后着
手工作,认真地将重新教育自己作为任务目标。

\switchcolumn*

He had capitulated; that was agreed. In reality, as he saw now, he had
been ready to capitulate long before he had taken the decision. From the
moment when he was inside the Ministry of Love---and yes, even during
those minutes when he and Julia had stood helpless while the iron voice
from the telescreen told them what to do---he had grasped the frivolity,
the shallowness of his attempt to set himself up against the power of
the Party. He knew now that for seven years the Thought Police had
watched him like a beetle under a magnifying glass. There was no
physical act, no word spoken aloud, that they had not noticed, no train
of thought that they had not been able to infer. Even the speck of
whitish dust on the cover of his diary they had carefully replaced. They
had played sound tracks to him, shown him photographs. Some of them were
photographs of Julia and himself. Yes, even... He could not fight
against the Party any longer. Besides, the Party was in the right. It
must be so: how could the immortal, collective brain be mistaken? By
what external standard could you check its judgments? Sanity was
statistical. It was merely a question of learning to think as they
thought. Only---!

\switchcolumn

他同意,他投降了。事实上,正如他现在看到的那样,在做出投降的决定以前他很早就准备投降了。从他踏入仁爱部的那一刻开始,是的,甚至从他和朱莉亚无助地站在一起听电屏里那冷酷的声音告诉他们应该做什么的那几分钟开始,他就已经明白,妄图凭一己之力与党的权力作对是多么肤浅、多么轻率。他现在知道了,七年以来思想警察观察他就像观察放大镜下的小甲虫。他的每一个肢体动作,每一句说出声的话,都被他们关注,没有他们推测不出来的想法。就连他夹在日记本中的那粒白色的灰尘也被他们仔仔细细地放回原位。他们给他放了录音,给他看了照片,其中一些照片中他和朱莉亚待在一起。是的,就算是\ldots\ldots 他再也不能和党对抗了。此外,党是正确的。一定是这样。作为一个永生不死的集体的大脑,它又怎么会出错呢?而你又要用什么客观的标准来判断它是否正确呢?神志健全是一个统计学的概念,它仅仅意味着学会按他们的想法思考问题。只不过\ldots\ldots{}

\switchcolumn*

The pencil felt thick and awkward in his fingers. He began to write down
the thoughts that came into his head. He wrote first in large clumsy
capitals:

\switchcolumn

手中的那支铅笔又粗又不好用,他开始将他的想法写下来,他先用大写字母笨拙地写下:

\switchcolumn*

\headline{FREEDOM IS SLAVERY.}

\switchcolumn

\headline{自由即奴役}

\switchcolumn*

Then almost without a pause he wrote beneath it:

\switchcolumn

紧接着他又在下面写道:

\switchcolumn*

\headline{TWO AND TWO MAKE FIVE.}

\switchcolumn

\headline{二加二等于五}

\switchcolumn*

But then there came a sort of check. His mind, as though shying away
from something, seemed unable to concentrate. He knew that he knew what
came next, but for the moment he could not recall it. When he did recall
it, it was only by consciously reasoning out what it must be; it did not
come of its own accord. He wrote:

\switchcolumn

但在这之后,他停了片刻。他的大脑似乎有意回避一些事情,好像无法集中精力。他知道接下来要写些什么,可一时半会儿又想不起来。当他想起来时,他发现那仅仅是他有意识地推理出来的,而非自然萌发的。他这样写道:

\switchcolumn*

\headline{GOD IS POWER.}

\switchcolumn

\headline{权力即上帝}

\switchcolumn*

He accepted everything. The past was alterable. The past never had been
altered. Oceania was at war with Eastasia. Oceania had always been at
war with Eastasia. Jones, Aaronson, and Rutherford were guilty of the
crimes they were charged with. He had never seen the photograph that
disproved their guilt. It had never existed; he had invented it. He
remembered remembering contrary things, but those were false memories,
products of self-deception. How easy it all was! Only surrender, and
everything else followed. It was like swimming against a current that
swept you backwards however hard you struggled, and then suddenly
deciding to turn round and go with the current instead of opposing it.
Nothing had changed except your own attitude; the predestined thing
happened in any case. He hardly knew why he had ever rebelled.
Everything was easy, except---!

\switchcolumn

他接受了一切。过去可以被篡改,过去从未被篡改。大洋国正在和东亚国打仗,大洋国一直在和东亚国打仗。琼斯、阿朗森和鲁瑟夫犯了那些指控给他们的罪行。他从来都没有看到过能够推翻他们罪行的照片。它从来就没存在过,是他凭空捏造。他还记得他曾经记得一些和这相反的事,但那些记忆是虚假的,是自欺欺人的产物。所有这些是多么地轻而易举!只要一投降,一切问题就都解决了。这就好像逆着水流游泳,尽管你很努力地挣扎,你还是会被水流卷着一路后退。然而,突然你决定转过身来顺流而行。改变的只有你自己的态度,其他的什么都没变。不管怎样,已经注定的事总是会发生的。他几乎不明白,曾经的他为什么要反抗。所有事都很容易,除了投降!

\switchcolumn*

Anything could be true. The so-called laws of nature were nonsense. The
law of gravity was nonsense. ``If I wished,'' O\textquotesingle Brien had
said, ``I could float off this floor like a soap bubble.'' Winston worked
it out. ``If he \emph{thinks} he floats off the floor, and if I
simultaneously \emph{think} I see him do it, then the thing happens.''
Suddenly, like a lump of submerged wreckage breaking the surface of
water, the thought burst into his mind: ``It doesn\textquotesingle t
really happen. We imagine it. It is hallucination.'' He pushed the
thought under instantly. The fallacy was obvious. It presupposed that
somewhere or other, outside oneself, there was a ``real'' world where
``real'' things happened. But how could there be such a world? What
knowledge have we of anything, save through our own minds? All
happenings are in the mind. Whatever happens in all minds, truly
happens.

\switchcolumn

任何事情都可能是真的。所谓的自然规律无非是胡说八道,重力定律也是胡扯。奥布兰说过:``只要我想,我就能像肥皂泡那样从地板上飘起来。''温斯顿想明白了:``如果他认为他从地板上飘了起来,而我同时也认为我看到他这样做了,那这件事就发生了。''突然,好像残骸浮出水面那样,他突然萌生了一个想法:``它不是真的发生了,它是我们想象出来的,是幻觉。''他立即将这个想法压了下去,它明显是荒谬的。因为它预先假定了一个地方,一个游离于个人之外的上演着``真实''事件的``真实''世界。但是,这样的世界又怎么可能存在呢?我们所知道的一切事情不都是通过我们的大脑得来的吗?所有事情都发生在意识里。无论是什么,只要发生在了头脑里,就真的发生了。

\switchcolumn*

He had no difficulty in disposing of the fallacy, and he was in no
danger of succumbing to it. He realized, nevertheless, that it ought
never to have occurred to him. The mind should develop a blind spot
whenever a dangerous thought presented itself. The process should be
automatic, instinctive. \emph{Crimestop}, they called it in Newspeak.

\switchcolumn

处理掉这个谬论一点儿都不困难,对他,也不存在向谬论屈服的危险。但他仍然认为,他永远不该想起它。不管什么时候,只要危险思想出现了,人的意识就应该本能地开辟出一块盲点。用新话说,这叫``停止犯罪''。

\switchcolumn*

He set to work to exercise himself in crimestop. He presented himself
with propositions---``the Party says the earth is flat,'' ``the Party says
that ice is heavier than water''---and trained himself in not seeing or
not understanding the arguments that contradicted them. It was not easy.
It needed great powers of reasoning and improvisation. The arithmetical
problems raised, for instance, by such a statement as ``two and two make
five'' were beyond his intellectual grasp. It needed also a sort of
athleticism of mind, an ability at one moment to make the most delicate
use of logic and at the next to be unconscious of the crudest logical
errors. Stupidity was as necessary as intelligence, and as difficult to
attain.

\switchcolumn

他开始就停止犯罪进行练习。他向自己提出一个题目——``党说地球是平的'',``党说冰比水重''——然后训练自己不去看也不去了解与之矛盾的观点。这可不容易。这需要非常大的推理能力和即时反应的能力。诸如``二加二等于五''这样的话都超出了他的理解水平。它同样需要大脑做一种运动,在某一时间要用逻辑处理最微妙的事件,但接下来又要忽略掉最明显的逻辑错误。愚蠢和理智一样必需,且一样难以掌握。

\switchcolumn*

All the while, with one part of his mind, he wondered how soon they
would shoot him. ``Everything depends on yourself,''
O\textquotesingle Brien had said; but he knew that there was no
conscious act by which he could bring it nearer. It might be ten minutes
hence, or ten years. They might keep him for years in solitary
confinement; they might send him to a labor camp; they might release him
for a while, as they sometimes did. It was perfectly possible that
before he was shot the whole drama of his arrest and interrogation would
be enacted all over again. The one certain thing was that death never
came at an expected moment. The tradition---the unspoken tradition:
somehow you knew it, though you never heard it said---was that they shot
you from behind, always in the back of the head, without warning, as you
walked down a corridor from cell to cell.

\switchcolumn

与此同时,他大脑中的一个部分仍在想要过多久他们才会枪毙他。奥布兰说过:``一切都取决于你。''但他知道他不能故意让这一天提早降临。可能在十分钟后,也可能在十年之后。他们可能一整年一整年地将他单独关押,可能把他送到劳动营去,还可能先释放他一段时间,他们有时会这么做。很有可能,他们会在枪毙他之前,将逮捕他、拷问他的这场戏完完整整重演一遍。有一件事可以确定,那就是死期无法预料。传统——未曾明说的传统是,尽管你从没有听说过,但你还是知道——他们会在你身后开枪,就在你沿着走廊从一间牢房走向另一间牢房时,在没有警告的情况下射向你的后脑,总是如此。

\switchcolumn*

One day---but ``one day'' was not the right expression; just as probably
it was in the middle of the night: once---he fell into a strange,
blissful reverie. He was walking down the corridor, waiting for the
bullet. He knew that it was coming in another moment. Everything was
settled, smoothed out, reconciled. There were no more doubts, no more
arguments, no more pain, no more fear. His body was healthy and strong.
He walked easily, with a joy of movement and with a feeling of walking
in sunlight. He was not any longer in the narrow white corridors of the
Ministry of Love; he was in the enormous sunlit passage, a kilometer
wide, down which he had seemed to walk in the delirium induced by drugs.
He was in the Golden Country, following the foot track across the old
rabbit-cropped pasture. He could feel the short springy turf under his
feet and the gentle sunshine on his face. At the edge of the field were
the elm trees, faintly stirring, and somewhere beyond that was the
stream where the dace lay in the green pools under the willows.

\switchcolumn

某天——但``某天''并非确切的说法,因为也有可能是在夜里。一次,他陷入了一种奇怪的、幸福的幻觉里。他在走廊里走着,等待着那颗子弹。他知道它马上就会过来。所有事情都解决了,没有了,妥协了。不再有怀疑,不再有争论,不再有痛苦,不再有恐惧。他的身体健康而强壮,他的步子非常轻快,他快乐地移动着,感觉正行走在阳光中。他再也不是走在仁爱部的白色走廊里,而是走在一条洒满阳光的、足有一公里宽的道路上。似乎因为药物的作用,他的精神极度亢奋。他在黄金乡里,正顺着一条布满脚印的小路穿过被兔子啃过的老牧场。他能感觉到脚下的小草短而柔软,能感受到阳光正温柔地照着他的脸。草场的边上有一些榆树,正轻轻地摆动着,远处的什么地方有一条小溪,鲦鱼在柳树下那绿色的池塘里游来游去。

\switchcolumn*

Suddenly he started up with a shock of horror. The sweat broke out on
his backbone. He had heard himself cry aloud:

\switchcolumn

猛然间,恐惧将他惊醒,脊骨上满是汗水。他听到自己在大喊:

\switchcolumn*

``Julia! Julia! Julia, my love! Julia!''

\switchcolumn

``朱莉亚!朱莉亚!朱莉亚,我的爱!朱莉亚!''

\switchcolumn*

For a moment he had had an overwhelming hallucination of her presence.
She had seemed to be not merely with him, but inside him. It was as
though she had got into the texture of his skin. In that moment he had
loved her far more than he had ever done when they were together and
free. Also he knew that somewhere or other she was still alive and
needed his help.

\switchcolumn

一时间,他产生了幻觉,觉得她就在这里。她不仅出现在他身边,还进到了他的身体里,
似乎就在他皮肤的纹理中。此时此刻,他更爱她了,比他们自由自在地在一起时还要爱。
他也知道,在某个地方,她依然活着且需要他的帮助。

\switchcolumn*

He lay back on the bed and tried to compose himself. What had he done?
How many years had he added to his servitude by that moment of weakness?

\switchcolumn

他躺回床上,努力让自己镇定下来。他在做什么?一时的软弱会让这被人奴役的日子增加多少年啊?

\switchcolumn*

In another moment he would hear the tramp of boots outside. They could
not let such an outburst go unpunished. They would know now, if they had
not known before, that he was breaking the agreement he had made with
them. He obeyed the Party, but he still hated the Party. In the old days
he had hidden a heretical mind beneath an appearance of conformity. Now
he had retreated a step further: in the mind he had surrendered, but he
had hoped to keep the inner heart inviolate. He knew that he was in the
wrong, but he preferred to be in the wrong. They would understand
that---O\textquotesingle Brien would understand it. It was all confessed
in that single foolish cry.

\switchcolumn

又过了一会儿,他听到外面有皮靴的声音。他们不可能在你这样爆发后不对你做什么惩
罚。若以前他们不知道,那现在他们知道了,他破坏了和他们的协议。他虽然对党俯首
帖耳,但他仍然仇恨着党。过去的日子里,他将他的异端思想隐藏在他恭顺的外表下,
现在他又退了一步。他在思想上投降了,但他仍希望保护好自己的内心。他知道自己错
了,但他甘愿如此。他们会理解的——奥布兰会理解的,那声愚蠢的呼喊把什么都坦白
了。

\switchcolumn*

He would have to start all over again. It might take years. He ran a
hand over his face, trying to familiarize himself with the new shape.
There were deep furrows in the cheeks, the cheekbones felt sharp, the
nose flattened. Besides, since last seeing himself in the glass he had
been given a complete new set of teeth. It was not easy to preserve
inscrutability when you did not know what your face looked like. In any
case, mere control of the features was not enough. For the first time he
perceived that if you want to keep a secret you must also hide it from
yourself. You must know all the while that it is there, but until it is
needed you must never let it emerge into your consciousness in any shape
that could be given a name. From now onwards he must not only think
right; he must feel right, dream right. And all the while he must keep
his hatred locked up inside him like a ball of matter which was part of
himself and yet unconnected with the rest of him, a kind of cyst.

\switchcolumn

他不得不重新来过,这可能会花上几年的时间。他伸出手,摸了摸脸,想熟悉下自己的
新样子。他的脸颊上有深深的皱纹,他的颧骨突了出来,鼻子则塌了进去。此外,自从
在镜子中看到自己的样子后,他们给他装了副崭新的假牙。在不知道自己的脸是什么样
子的情况下,你很难让自己看起来高深莫测。总之,单是控制外表的样子还不够。他第
一次意识到,如果你要把一个秘密隐藏起来,即便对你自己,你也得保密。你必须从头
到尾都知道它在哪儿,但若非必须,你万万不能让它以任何叫得出名的样子出现在你的
意识里。从今以后,他不仅要思想正确,还要感觉正确,梦得正确。整个过程中,他必
须将他的仇恨锁在心里,就好像一个既是他身体的一部分又不和他身体的其他部分发生
联系的圆球,一个囊肿。

\switchcolumn*

One day they would decide to shoot him. You could not tell when it would
happen, but a few seconds beforehand it should be possible to guess. It
was always from behind, walking down a corridor. Ten seconds would be
enough. In that time the world inside him could turn over. And then
suddenly, without a word uttered, without a check in his step, without
the changing of a line in his face---suddenly the camouflage would be
down and bang! would go the batteries of his hatred. Hatred would fill
him like an enormous roaring flame. And almost in the same instant bang!
would go the bullet, too late, or too early. They would have blown his
brain to pieces before they could reclaim it. The heretical thought
would be unpunished, unrepented, out of their reach forever. They would
have blown a hole in their own perfection. To die hating them, that was
freedom.

\switchcolumn

他们终将决定到底在哪天将他枪毙,你不可能被告知这会在什么时候发生,但你可以提
前几秒猜到。总是从你身后,当你在走廊上的时候,枪毙你。只要十秒就够了。那时,
他的内心世界会翻转过来。然后,突然,不用说任何话,不用停下脚步,也不用改变脸
上的表情——突然,伪装撕下,紧接着,砰!他的仇恨像炮群开火那样猛然爆发,像熊
熊烈焰将他烧毁。几乎在同时,砰!子弹来了,或者太晚,或者太早。在对他的大脑进
行改造前,他们就会将他轰成碎片。异端思想不受惩罚,不经悔改,永远都在他们的控
制之外,就好比他们亲手在他们完美的身体上轰开一个洞。至死都恨着他们,这就是自
由。

\switchcolumn*

He shut his eyes. It was more difficult than accepting an intellectual
discipline. It was a question of degrading himself, mutilating himself.
He had got to plunge into the filthiest of filth. What was the most
horrible, sickening thing of all? He thought of Big Brother. The
enormous face (because of constantly seeing it on posters he always
thought of it as being a meter wide), with its heavy black mustache and
the eyes that followed you to and fro, seemed to float into his mind of
its own accord. What were his true feelings toward Big Brother?

\switchcolumn

他闭上眼睛,这比接受某个思想原则还要困难。这是个自己侮辱自己,自己残虐自己的
问题,他会堕入最肮脏的污秽中。什么事最为可怕,最令人作呕?他想到老大哥。他那
张巨脸(由于他经常在宣传画上看到它,他一直觉得这张脸有一米宽)蓄着浓密的黑色
胡须,他的眼睛总是跟着人的身影转来转去。这些都自动地在他的意识里浮现出来。他
对老大哥的真实感情又是怎样的呢?

\switchcolumn*

There was a heavy tramp of boots in the passage. The steel door swung
open with a clang. O\textquotesingle Brien walked into the cell. Behind
him were the waxen-faced officer and the black-uniformed guards.

\switchcolumn

楼道里响起沉重的皮靴声。铁门哐当一声开了。奥布兰走进牢房,身后跟着那个蜡像脸的官员和穿着黑色制服的看守。

\switchcolumn*

``Get up,'' said O\textquotesingle Brien. ``Come here.''

\switchcolumn

``起来,''奥布兰说,``过来。''

\switchcolumn*

Winston stood opposite him. O\textquotesingle Brien took
Winston\textquotesingle s shoulders between his strong hands and looked
at him closely.

\switchcolumn

温斯顿站在他面前。奥布兰用强壮的双手抓住温斯顿的肩膀,近距离地看着他。

\switchcolumn*

``You have had thoughts of deceiving me,'' he said. ``That was stupid.
Stand up straighter. Look me in the face.''

\switchcolumn

``你曾想欺骗我,''他说,``这很愚蠢,站直,看着我的脸。''

\switchcolumn*

He paused, and went on in a gentler tone:

\switchcolumn

他停了一会儿,然后用较温和的语气说:

\switchcolumn*

``You are improving. Intellectually there is very little wrong with you.
It is only emotionally that you have failed to make progress. Tell me,
Winston---and remember, no lies; you know that I am always able to
detect a lie---tell me, what are your true feelings toward Big Brother?''

\switchcolumn

``你有进步。在思想上已经没什么问题了。只是在感情上没有长进。告诉我,温斯顿——记住,不能说谎,你知道我总是能发现谎言——告诉我,你对老大哥的真实感觉。''

\switchcolumn*

``I hate him.''

\switchcolumn

``我恨他。''

\switchcolumn*

``You hate him. Good. Then the time has come for you to take the last
step. You must love Big Brother. It is not enough to obey him; you must
love him.''

\switchcolumn

``你恨他,好的。接下来你要进入最后一个阶段。你必须爱老大哥,服从他还不够,你必须爱他。''

\switchcolumn*

He released Winston with a little push toward the guards.

\switchcolumn

他松开温斯顿,将他推给看守。

\switchcolumn*

``Room 101,'' he said.

\switchcolumn

``101号房间。''他说。

\switchcolumn*


\section{V}\label{v-2}

\switchcolumn

\section*{五}\label{ux4e8cux5341ux4e09}

\switchcolumn*

At each stage of his imprisonment he had known, or seemed to know,
whereabouts he was in the windowless building. Possibly there were
slight differences in the air pressure. The cells where the guards had
beaten him were below ground level. The room where he had been
interrogated by O\textquotesingle Brien was high up near the roof. This
place was many meters underground, as deep down as it was possible to
go.

\switchcolumn

刑期的每个阶段他都清楚——或者说他好像很清楚——他究竟在这没有窗户的大楼里的哪个地方。也许因为不同地方气压略有不同。他被看守殴打时所在的牢房位于地下,他被奥布兰审问的那个房间临近楼顶。而现在这个地方应该深入地下好几米,已经深到不能再深的地步了。

\switchcolumn*

It was bigger than most of the cells he had been in. But he hardly
noticed his surroundings. All he noticed was that there were two small
tables straight in front of him, each covered with green baize. One was
only a meter or two from him; the other was further away, near the door.
He was strapped upright in a chair, so tightly that he could move
nothing, not even his head. A sort of pad gripped his head from behind,
forcing him to look straight in front of him.

\switchcolumn

这间牢房比他待过的所有牢房都要大,但他几乎没有关注四周的环境,只注意到他的面前摆着两张铺着绿呢布的小桌子。一张桌子离他只有一两米,另一张则靠着门,距离稍远。他被绑到了一张椅子上,很紧,以至于他一动都没法动,脑袋都转不了。一种软垫从后面夹住了他的脑袋,迫使他只能朝着正前方看。

\switchcolumn*

For a moment he was alone, then the door opened and
O\textquotesingle Brien came in.

\switchcolumn

他在屋子里单独待了一会儿,之后,门开了,奥布兰进来了。

\switchcolumn*

``You asked me once,'' said O\textquotesingle Brien, ``what was in Room
101. I told you that you knew the answer already. Everyone knows it. The
thing that is in Room 101 is the worst thing in the world.''

\switchcolumn

``你曾经问过我,''奥布兰说,``101号房里有什么,我告诉你,你早就知道答案。每
个人都知道,101号房里的东西是世界上最可怕的。''

\switchcolumn*

The door opened again. A guard came in, carrying something made of wire,
a box or basket of some kind. He set it down on the further table.
Because of the position in which O\textquotesingle Brien was standing,
Winston could not see what the thing was.

\switchcolumn

门又开了,一个看守走了进来,手里拿着一个用铁丝编成的盒子,或者说是篮子一类的
东西,将它放在了较远的桌子上,因为奥布兰就站在那里。温斯顿认不出那是什么东
西。

\switchcolumn*

``The worst thing in the world,'' said O\textquotesingle Brien, ``varies
from individual to individual. It may be burial alive, or death by fire,
or by drowning, or by impalement, or fifty other deaths. There are cases
where it is some quite trivial thing, not even fatal.''

\switchcolumn

``世界上最可怕的事,''奥布兰说,``每个人都有每个人的看法。可能是活埋,可能是被火烧死,可能是溺水而亡,还有可能是被钉死,或者别的五十几种死法。也有时,最可怕会是一些十分琐碎的,并不致命的事。''

\switchcolumn*

He had moved a little to one side, so that Winston had a better view of
the thing on the table. It was an oblong wire cage with a handle on top
for carrying it by. Fixed to the front of it was something that looked
like a fencing mask, with the concave side outwards. Although it was
three or four meters away from him, he could see that the cage was
divided lengthways into two compartments, and that there was some kind
of creature in each. They were rats.

\switchcolumn

他向旁边挪了一点儿,温斯顿终于可以看清桌上放的到底是什么。那是一个长方形的铁笼子,顶端有一个把手,方便人将它提起。笼子前端还装着一个类似击剑面罩的东西,有个向外的凹面。尽管这东西离他有三四米远,但他仍然可以看到它被竖着分成两个部分,每个部分都关着什么动物。原来,是老鼠。

\switchcolumn*

``In your case,'' said O\textquotesingle Brien, ``the worst thing in the
world happens to be rats.''

\switchcolumn

``对你来说,''奥布兰说,``世界上最可怕的就是老鼠。''

\switchcolumn*

A sort of premonitory tremor, a fear of he was not certain what, had
passed through Winston as soon as he caught his first glimpse of the
cage. But at this moment the meaning of the masklike attachment in front
of it suddenly sank into him. His bowels seemed to turn to water.

\switchcolumn

看到这笼子的第一眼,尽管还不确定笼子里究竟有什么,他就有了一种不好的预感,浑身震颤,恐惧非常。此时,他突然明白装在笼子前端的那个形似面罩的东西有何目的。他觉得自己好像吓得失禁了。

\switchcolumn*

``You can\textquotesingle t do that!'' he cried out in a high cracked
voice. ``You couldn\textquotesingle t, you couldn\textquotesingle t!
It\textquotesingle s impossible.''

\switchcolumn

``你不能那么做!''他尖声惊叫着。``你不能,你不能!这不可能。''

\switchcolumn*

``Do you remember,'' said O\textquotesingle Brien, ``the moment of panic
that used to occur in your dreams? There was a wall of blackness in
front of you, and a roaring sound in your ears. There was something
terrible on the other side of the wall. You knew that you knew what it
was, but you dared not drag it into the open. It was the rats that were
on the other side of the wall.''

\switchcolumn

``你还记得吗,''奥布兰说,``你梦中的恐怖时刻?你面前有一堵黑色的墙,你听到动物的低吼。墙的另一边有着某种可怕的东西。你知道自己明白那是什么,但你没有胆量把它们亮出来。墙的那边就是老鼠。''

\switchcolumn*

``O\textquotesingle Brien!'' said Winston, making an effort to control his
voice. ``You know this is not necessary. What is it that you want me to
do?''

\switchcolumn

``奥布兰!''温斯顿说,他竭力控制住自己的声音。``你知道没这个必要,你想让我做什么?''

\switchcolumn*

O\textquotesingle Brien made no direct answer. When he spoke it was in
the schoolmasterish manner that he sometimes affected. He looked
thoughtfully into the distance, as though he were addressing an audience
somewhere behind Winston\textquotesingle s back.

\switchcolumn

奥布兰没有直接回答他。他说话时,又换上了那副老师的神态,他已经好几次这样装腔作势了。他沉思地看着远方,就好像在和温斯顿身后的什么听众说话一样。

\switchcolumn*

``By itself,'' he said, ``pain is not always enough. There are occasions
when a human being will stand out against pain, even to the point of
death. But for everyone there is something unendurable---something that
cannot be contemplated. Courage and cowardice are not involved. If you
are falling from a height it is not cowardly to clutch at a rope. If you
have come up from deep water it is not cowardly to fill your lungs with
air. It is merely an instinct which cannot be disobeyed. It is the same
with the rats. For you, they are unendurable. They are a form of
pressure that you cannot withstand, even if you wished to. You will do
what is required of you.''

\switchcolumn

``就疼痛本身来说,''他说,``还远远不够。有时候人是可以和疼痛对抗的,即使疼得濒临死亡。但每个人都有些无法忍受的事——一些连想都不能想的东西。这和勇敢或怯懦无关。如果你从高处跌落,抓住绳子不算怯懦,如果你从深水中浮出头,往肺里猛灌空气不算怯懦。这仅仅是一种本能,你无法将它消灭。老鼠也是一样。对你而言,它们让人无法忍受。它们是一种你无力抵抗的压力,哪怕你希望自己能抵抗得住。需要你做什么,你就会做什么。''

\switchcolumn*

``But what is it, what is it? How can I do it if I don\textquotesingle t
know what it is?''

\switchcolumn

``但那是什么?究竟是什么?我不知道要做什么,又怎么去做呢?''

\switchcolumn*

O\textquotesingle Brien picked up the cage and brought it across to the
nearer table. He set it down carefully on the baize cloth. Winston could
hear the blood singing in his ears. He had the feeling of sitting in
utter loneliness. He was in the middle of a great empty plain, a flat
desert drenched with sunlight, across which all sounds came to him out
of immense distances. Yet the cage with the rats was not two meters away
from him. They were enormous rats. They were at the age when a
rat\textquotesingle s muzzle grows blunt and fierce and his fur brown
instead of gray.

\switchcolumn

奥布兰将笼子提了起来,带到离温斯顿较近的桌子,小心翼翼地把它放在了绿呢制的桌布上。温斯顿听到血液在耳朵里轰鸣的声音,觉得自己被完全孤立了,就像待在一个巨大荒芜的平原里,一个洒满阳光的沙漠中,那些声音从极度遥远的地方穿过平原沙漠向他袭来。事实上,装着老鼠的笼子离他不到二米。它们体形硕大,口鼻那里又平又钝,模样凶狠,且都长着棕色而不是灰色的毛。

\switchcolumn*

``The rat,'' said O\textquotesingle Brien, still addressing his invisible
audience, ``although a rodent, is carnivorous. You are aware of that. You
will have heard of the things that happen in the poor quarters of this
town. In some streets a woman dare not leave her baby alone in the
house, even for five minutes. The rats are certain to attack it. Within
quite a small time they will strip it to the bones. They also attack
sick or dying people. They show astonishing intelligence in knowing when
a human being is helpless.''

\switchcolumn

``老鼠,''奥布兰仍然是一副对着看不见的听众说话的样子,``虽然是啮齿动物,但也是吃肉的。你明白。你一定听说过发生在这儿的贫民区中的事情。在一些街道,女人不敢把婴儿独自一人留在屋子里,即使只有五分钟。因为老鼠一定会袭击他,只需一会儿,它们就会把孩子的骨头啃出来。它们还会袭击生病的和将死的人。它们智力惊人,知道人什么时候是无助的。''

\switchcolumn*

There was an outburst of squeals from the cage. It seemed to reach
Winston from far away. The rats were fighting; they were trying to get
at each other through the partition. He heard also a deep groan of
despair. That, too, seemed to come from outside himself.

\switchcolumn

铁笼子里突然迸发出吱吱声,好像从很远的地方传到温斯顿这里。老鼠们在打架,它们试图穿过分隔它们的东西到另一边去。他听到一种绝望的、低沉的呻吟声,似乎也从他的身外传来。

\switchcolumn*

O\textquotesingle Brien picked up the cage, and, as he did so, pressed
something in it. There was a sharp click. Winston made a frantic effort
to tear himself loose from the chair. It was hopeless: every part of
him, even his head, was held immovably. O\textquotesingle Brien moved
the cage nearer. It was less than a meter from Winston\textquotesingle s
face.

\switchcolumn

奥布兰提起笼子,并在提起的同时,按了里面的什么东西。温斯顿听到尖锐的啪嗒声,他疯狂地试图从椅子上挣脱开来,但毫无办法。他身体的每个部分,就连他的脑袋,都被绑得动弹不得。奥布兰又把笼子挪近了一些,它和温斯顿的脸只有不到一米的距离了。

\switchcolumn*

``I have pressed the first lever,'' said O\textquotesingle Brien. ``You
understand the construction of this cage. The mask will fit over your
head, leaving no exit. When I press this other lever, the door of the
cage will slide up. These starving brutes will shoot out of it like
bullets. Have you ever seen a rat leap through the air? They will leap
onto your face and bore straight into it. Sometimes they attack the eyes
first. Sometimes they burrow through the cheeks and devour the tongue.''

\switchcolumn

``我已经按下第一个控制杆,''奥布兰说,``你清楚这笼子的构造。面罩和你的头正合适,不会留下空隙。当我按下第二个控制杆的时候。笼子的门就会打开。这些饥饿的牲畜会像子弹那样冲出来,你有没有看到过老鼠跳到空中的样子?它们会跳到你的脸上,一直往里钻。有时,它们首先会攻击你的眼睛。有时,它们会钻进你的脸颊,吞掉你的舌头。''

\switchcolumn*

The cage was nearer; it was closing in. Winston heard a succession of
shrill cries which appeared to be occurring in the air above his head.
But he fought furiously against his panic. To think, to think, even with
a split second left---to think was the only hope. Suddenly the foul
musty odor of the brutes struck his nostrils. There was a violent
convulsion of nausea inside him, and he almost lost consciousness.
Everything had gone black. For an instant he was insane, a screaming
animal. Yet he came out of the blackness clutching an idea. There was
one and only one way to save himself. He must interpose another human
being, the \emph{body} of another human being, between himself and the
rats.

\switchcolumn

笼子越来越近,越来越近。温斯顿听到一阵持续的尖叫声从他的脑袋上方发出来,但他还在激烈地和恐惧对抗。想一想,想一想,哪怕只剩下半秒——想,是唯一的希望。突然,一股牲畜的霉味直扑他的鼻子,强烈地冲击着他,让他肚子里翻江倒海。他几乎失去了意识,眼前一片漆黑。一时间他疯了,成了一只惊叫着的动物。他抱着一个想法从黑暗中挣扎出来。有一个办法,只有一个办法能让他拯救自己。他必须将另一个人,另一个人的身体插进他和老鼠之间。

\switchcolumn*

The circle of the mask was large enough now to shut out the vision of
anything else. The wire door was a couple of hand-spans from his face.
The rats knew what was coming now. One of them was leaping up and down;
the other, an old scaly grandfather of the sewers, stood up, with his
pink hands against the bars, and fiercely snuffed the air. Winston could
see the whiskers and the yellow teeth. Again the black panic took hold
of him. He was blind, helpless, mindless.

\switchcolumn

面罩的边缘足够大,大到将他视野里的一切其他事物遮挡住。铁丝制的笼门和他的脸只有一两个巴掌那么远。老鼠们知道会遇见什么,其中一只正上蹿下跳,还有一只阴沟里的家伙老得掉了毛,它直立着,用粉红色的爪子扒着铁丝,使劲地嗅着什么。温斯顿能够看到它的胡须和黄色的牙齿。黑色的恐惧再一次抓住了他,他什么都看不见,他无能为力,他的大脑一片空白。

\switchcolumn*

``It was a common punishment in Imperial China,'' said
O\textquotesingle Brien as didactically as ever.

\switchcolumn

``在中华帝国,这是常见的惩罚。''奥布兰一如既往进行说教。

\switchcolumn*

The mask was closing on his face. The wire brushed his cheek. And
then---no, it was not relief, only hope, a tiny fragment of hope. Too
late, perhaps too late. But he had suddenly understood that in the whole
world there was just \emph{one} person to whom he could transfer his
punishment---\emph{one} body that he could thrust between himself and
the rats. And he was shouting frantically, over and over:

\switchcolumn

面罩贴近他的脸,铁丝碰到他的脸颊。接着——不,那不能解除什么,只是希望,一丝细弱的希望。太晚了,也许太晚了,但他突然明白,整个世界他只能把惩罚转移到一个人身上——只有一个人的身体能够插在他和老鼠间,他疯狂地喊了起来,一遍又一遍:

\switchcolumn*

``Do it to Julia! Do it to Julia! Not me! Julia! I don\textquotesingle t
care what you do to her. Tear her face off, strip her to the bones. Not
me! Julia! Not me!''

\switchcolumn

``咬朱莉亚!咬朱莉亚!别咬我!朱莉亚!随你们怎么对她。撕开她的脸,咬她的骨头,别咬我!朱莉亚!别咬我!''

\switchcolumn*

He was falling backwards, into enormous depths, away from the rats. He
was still strapped in the chair, but he had fallen through the floor,
through the walls of the building, through the earth, through the
oceans, through the atmosphere, into outer space, into the gulfs between
the stars---always away, away, away from the rats. He was light-years
distant, but O\textquotesingle Brien was still standing at his side.
There was still the cold touch of a wire against his cheek. But through
the darkness that enveloped him he heard another metallic click, and
knew that the cage door had clicked shut and not open.

\switchcolumn

他向后倒了下去,堕入巨大的深渊,远离老鼠。他仍然被绑在椅子上,但他已穿过地板,穿过大楼的墙壁,穿过地球,穿过海洋,穿过大气层,堕入太空,堕入银河——远远地,远远地,远远地离开了老鼠。他在若干光年之外,但奥布兰依旧站在他的身边,冰冷的铁丝依旧触碰着他的脸。然而从包裹着他的黑暗里,他听到金属的撞击声,他知道笼子的门咔嚓一声关上了,没有打开。

\switchcolumn*


\section{VI}\label{vi-2}

\switchcolumn

\section*{六}\label{ux4e8cux5341ux56db}

\switchcolumn*

The Chestnut Tree was almost empty. A ray of sunlight slanting through a
window fell yellow on dusty tabletops. It was the lonely hour of
fifteen. A tinny music trickled from the telescreens.

\switchcolumn

栗树咖啡馆几乎空无一人。一道金黄色的阳光从窗户里斜斜地射进来,落在积满灰尘的桌面上。15点的咖啡馆孤单单的。电屏里传来微弱的音乐声。

\switchcolumn*

Winston sat in his usual corner, gazing into an empty glass. Now and
again he glanced up at a vast face which eyed him from the opposite
wall. \textsc{Big Brother Is Watching You}, the caption said. Unbidden, a
waiter came and filled his glass up with Victory Gin, shaking into it a
few drops from another bottle with a quill through the cork. It was
saccharine flavored with cloves, the speciality of the café.

\switchcolumn

温斯顿坐在他常坐的那个角落里,凝视着一只空玻璃杯。他不时便抬头看看对面墙上那张巨大的脸。那上面有个标题:老大哥在看着你。侍者自动过来为他倒满胜利牌杜松子酒,又从另外一个有着软木塞的瓶子里摇出几颗豆子。那是栗树咖啡馆特有的丁香味糖精。

\switchcolumn*

Winston was listening to the telescreen. At present only music was
coming out of it, but there was a possibility that at any moment there
might be a special bulletin from the Ministry of Peace. The news from
the African front was disquieting in the extreme. On and off he had been
worrying about it all day. A Eurasian army (Oceania was at war with
Eurasia; Oceania had always been at war with Eurasia) was moving
southward at terrifying speed. The mid-day bulletin had not mentioned
any definite area, but it was probable that already the mouth of the
Congo was a battlefield. Brazzaville and Leopoldville were in danger.
One did not have to look at the map to see what it meant. It was not
merely a question of losing Central Africa; for the first time in the
whole war, the territory of Oceania itself was menaced.

\switchcolumn

温斯顿正在收听电屏,此刻里面只有音乐,但和平部的特别公告随时可能出现。非洲前线的状况令人极其不安。为此,每天他无时无刻不在担心。一支欧亚国大军(大洋国正在和欧亚国打仗,大洋国一直在和欧亚国打仗)正迅速地向南方移动。中午的公报没有提任何具体地点,但战场很有可能已经转移到刚果河口。布拉柴维尔和利奥波德维尔已经岌岌可危。这不仅仅是丢掉非洲中部的问题,这场战争让大洋国的领土第一次受到威胁。

\switchcolumn*

A violent emotion, not fear exactly but a sort of undifferentiated
excitement, flared up in him, then faded again. He stopped thinking
about the war. In these days he could never fix his mind on any one
subject for more than a few moments at a time. He picked up his glass
and drained it at a gulp. As always, it made him shudder and even retch
slightly. The stuff was horrible. The cloves and saccharine, themselves
disgusting enough in their sickly way, could not disguise the flat oily
smell; and what was worst of all was that the smell of gin, which dwelt
with him night and day, was inextricably mixed up in his mind with the
smell of those---

\switchcolumn

某种强烈的情感在他心中燃起,说恐惧并不确切,那是一种无法言说的激动,但它很快就褪去了。他不再想和战争有关的事。这段时间,对任何事他都无法集中思想超过个把分钟。他拿起杯子,一饮而尽。一如往常,杜松子酒让他浑身颤抖,甚至有点儿恶心。这味道真是可怕,丁香和糖精的味道都足够令人作呕,却还是压不住酒的油味儿。最糟糕的莫过于无论白天黑夜,他的身上都沾着浓浓的杜松子酒味儿,在他的意识里,这酒味和某种气味牢牢地纠缠在一起,那是——

\switchcolumn*

He never named them, even in his thoughts, and so far as it was possible
he never visualized them. They were something that he was half aware of,
hovering close to his face, a smell that clung to his nostrils. As the
gin rose in him he belched through purple lips. He had grown fatter
since they released him, and had regained his old color---indeed, more
than regained it. His features had thickened, the skin on nose and
cheekbones was coarsely red, even the bald scalp was too deep a pink. A
waiter, again unbidden, brought the chessboard and the current issue of
the \emph{Times}, with the page turned down at the chess problem. Then,
seeing that Winston\textquotesingle s glass was empty, he brought the
gin bottle and filled it. There was no need to give orders. They knew
his habits. The chessboard was always waiting for him, his corner table
was always reserved; even when the place was full he had it to himself,
since nobody cared to be seen sitting too close to him. He never even
bothered to count his drinks. At irregular intervals they presented him
with a dirty slip of paper which they said was the bill, but he had the
impression that they always undercharged him. It would have made no
difference if it had been the other way about. He had always plenty of
money nowadays. He even had a job, a sinecure, more highly paid than his
old job had been.

\switchcolumn

他从来不说明它们是什么,哪怕它们只存在于他的头脑里,他尽己所能地不去想它们的样子。而只模模糊糊地想到它们逼近他的脸,其气味扑鼻而来,杜松子酒在他的肚子里翻滚,他用紫色的嘴唇打了一个嗝。自从他们释放了他之后,他就长胖了,气色也得以恢复——事实上,比之前的还要好。他变得强壮了,鼻子和脸颊上的皮肤泛起粗糙的红色,就连秃顶处也变成了粉红色。侍者再一次不经招呼拿来棋盘和新出的《泰晤士报》,还将报纸翻到有象棋题目的那一页。之后,他发现温斯顿的酒杯空了,就端来盛满杜松子酒的酒瓶为他斟满。不需要提什么要求,他们知道他的习惯。棋盘总会为他备好,这个角落里的位子也会为他留着,就算客人满员也是如此,因为没有人愿意被看到和他离得很近。他从来都懒得数自己喝了多少杯。过一会儿,他们就会递给他一张肮脏的纸条说是账单,但在他的印象中,他们总是算少了账。而即使反过来,多算了他的钱,也没什么分别。如今,他的钱总是够花,他甚至还有了工作,一个挂名职务,收入要比他之前的工作高得多。

\switchcolumn*

The music from the telescreen stopped and a voice took over. Winston
raised his head to listen. No bulletin from the front, however. It was
merely a brief announcement from the Ministry of Plenty. In the
preceding quarter, it appeared, the Tenth Three-Year
Plan\textquotesingle s quota for bootlaces had been overfulfilled by
ninety-eight per cent.

\switchcolumn

电屏里的音乐停止了,响起说话的声音。温斯顿扬起头听着,但却不是前线的公报,而仅仅是富部的一条短通知。上个季度第十个三年计划鞋带产量超额完成了九十八个百分点。

\switchcolumn*

He examined the chess problem and set out the pieces. It was a tricky
ending, involving a couple of knights. ``White to play and mate in two
moves.'' Winston looked up at the portrait of Big Brother. White always
mates, he thought with a sort of cloudy mysticism. Always, without
exception, it is so arranged. In no chess problem since the beginning of
the world has black ever won. Did it not symbolize the eternal,
unvarying triumph of Good over Evil? The huge face gazed back at him,
full of calm power. White always mates.

\switchcolumn

他研究了一下棋局并摆上棋子。棋的结局很有欺骗性,要用到一对马。``白棋先走,两步将死。''温斯顿看了看老大哥的画像。白棋总是将死,他有一种蒙眬而神秘的感觉。总是这样,没有例外,都是被安排好的。自世界伊始,没有一盘棋黑棋能得胜。这难道不是一个永恒的象征吗?象征善良会战胜邪恶?那张巨大的脸正盯着他看,镇定又充满力量,白棋总是将死对方。

\switchcolumn*

The voice from the telescreen paused and added in a different and much
graver tone: ``You are warned to stand by for an important announcement
at fifteen-thirty. Fifteen-thirty! This is news of the highest
importance. Take care not to miss it. Fifteen-thirty!'' The tinkling
music struck up again.

\switchcolumn

电屏上,声音暂停了一会儿,接着一个更为严肃的声音说:``大家注意,15点30分有重要通知,请做好收听准备。15点30分有非常重要的新闻,不要错过。15点30分。''之后,叮的一声,音乐又响了起来。

\switchcolumn*

Winston\textquotesingle s heart stirred. That was the bulletin from the
front; instinct told him that it was bad news that was coming. All day,
with little spurts of excitement, the thought of a smashing defeat in
Africa had been in and out of his mind. He seemed actually to see the
Eurasian army swarming across the never-broken frontier and pouring down
into the tip of Africa like a column of ants. Why had it not been
possible to outflank them in some way? The outline of the West African
coast stood out vividly in his mind. He picked up the white knight and
moved it across the board. \emph{There} was the proper spot. Even while
he saw the black horde racing southward he saw another force,
mysteriously assembled, suddenly planted in their rear, cutting their
communications by land and sea. He felt that by willing it he was
bringing that other force into existence. But it was necessary to act
quickly. If they could get control of the whole of Africa, if they had
airfields and submarine bases at the Cape, it would cut Oceania in two.
It might mean anything: defeat, breakdown, the redivision of the world,
the destruction of the Party! He drew a deep breath. An extraordinary
medley of feelings---but it was not a medley, exactly; rather it was
successive layers of feeling, in which one could not say which layer was
undermost---struggled inside him.

\switchcolumn

温斯顿心里很乱。那会是来自前线的公报。本能告诉他那是个坏消息。整整一天他都有
些激动,大洋国在非洲大败的情景不时便出现在他的脑海里。他似乎真的看到了欧亚国
大军势如破竹地通过那从未被攻克的边界,如一大队的蚂蚁涌入非洲南端。为什么不能
从侧面包抄他们呢?西非海岸的轮廓生动地浮现在他的脑海中。他拿起白棋的马在棋盘
上移动起来,正好走到合适的位置。就算在看到乌泱泱的大军向南部挺进时,他也看到
了另一支大军神秘地集合起来,突然插入他们的后方,切断了他们的海陆联系。他觉得,
凭借臆想,他正将另一支大军带入现实。只是行动一定要快。如果让他们控制了整个非
洲,如果让他们得到了好望角的机场和潜艇基地,大洋国就会被一分为二。这意味着一
切:战败、崩溃、重新瓜分世界以及党的覆灭!他深深地吸了一口气,一种复杂而奇怪
的感觉——准确地说并不是百感交集,而是一种多层次的感觉,他说不清在他内心最深
处搅动的究竟是哪个层次的感觉。

\switchcolumn*

The spasm passed. He put the white knight back in its place, but for the
moment he could not settle down to serious study of the chess problem.
His thoughts wandered again. Almost unconsciously he traced with his
finger in the dust on the table:

\switchcolumn

一阵心潮澎湃后。他把白棋的马放回原位,但此时他已无法安定下来认真思考象棋的问题。他又漫无目的地想了起来。差不多是下意识地用手指在桌面上的灰尘中写道:

\switchcolumn*

\headline{2 + 2 = 5.}

\switchcolumn

2+2=5

\switchcolumn*

``They can\textquotesingle t get inside you,'' she had said. But they
could get inside you. ``What happens to you here is \emph{forever},''
O\textquotesingle Brien had said. That was a true word. There were
things, your own acts, from which you could not recover. Something was
killed in your breast; burnt out, cauterized out.

\switchcolumn

她曾说过:``他们进不去你心里。''但他们能。奥布兰说的:``在你身上发生的事会永远持续下去。''这是事实。你永远无法恢复一些事情,一些行为。在你的胸膛里,有什么东西被杀死,被烧光,被腐蚀。

\switchcolumn*

He had seen her; he had even spoken to her. There was no danger in it.
He knew as though instinctively that they now took almost no interest in
his doings. He could have arranged to meet her a second time if either
of them had wanted to. Actually it was by chance that they had met. It
was in the Park, on a vile, biting day in March, when the earth was like
iron and all the grass seemed dead and there was not a bud anywhere
except a few crocuses which had pushed themselves up to be dismembered
by the wind. He was hurrying along with frozen hands and watering eyes
when he saw her not ten meters away from him. It struck him at once that
she had changed in some ill-defined way. They almost passed one another
without a sign; then he turned and followed her, not very eagerly. He
knew that there was no danger, nobody would take any interest in them.
She did not speak. She walked obliquely away across the grass as though
trying to get rid of him, then seemed to resign herself to having him at
her side. Presently they were in among a clump of ragged leafless
shrubs, useless either for concealment or as protection from the wind.
They halted. It was vilely cold. The wind whistled through the twigs and
fretted the occasional, dirty-looking crocuses. He put his arm round her
waist.

\switchcolumn

他见过她,还和她说了话。这不会有什么危险。他本能地知道现在他们对他的所作所为几乎没有兴趣。若他们两人中有一人愿意,他就可以安排和她再约会一次。事实上,那次见面只是偶然。那是在公园里,在三月的一个寒冷的天气很糟糕的日子里,地冻得像铁一样,草看上去也都死掉了,除了几株被风吹得支离破碎的番红花,看不到一支花骨朵。当他发现他和她的距离不过十米时,他正双手冰冷,流着眼泪,急匆匆地行走着。他一看到她就被打击到了,她变了,又说不清哪里变了。他们一个招呼都没打,擦肩而过。接着,他转过身,并不是特别急切地跟在她身后。他知道这儿没有危险,没人会对他们产生兴趣。她没说话,斜穿过草坪,好像在试图避开他,后来又似乎听任他待在她身边。不一会儿,他们就来到了粗糙的光秃秃的灌木丛中间,那里既避不住人,又躲不开风。他们停下脚步。天气太冷了,风打着哨穿过树枝,蹂躏着脏兮兮的番红花。他伸出手环住她的腰。

\switchcolumn*

There was no telescreen, but there must be hidden microphones; besides,
they could be seen. It did not matter, nothing mattered. They could have
lain down on the ground and done \emph{that} if they had wanted to. His
flesh froze with horror at the thought of it. She made no response
whatever to the clasp of his arm; she did not even try to disengage
herself. He knew now what had changed in her. Her face was sallower, and
there was a long scar, partly hidden by the hair, across her forehead
and temple; but that was not the change. It was that her waist had grown
thicker and, in a surprising way, had stiffened. He remembered how once,
after the explosion of a rocket bomb, he had helped to drag a corpse out
of some ruins, and had been astonished not only by the incredible weight
of the thing, but by its rigidity and awkwardness to handle, which made
it seem more like stone than flesh. Her body felt like that. It occurred
to him that the texture of her skin would be quite different from what
it had once been.

\switchcolumn

这里没有电屏,但多半藏着话筒,此外,他们还很有可能被人看见。但这不重要,一点都不重要。若他们想,他们大可以躺倒在地做那事。一想到这里,他就吓得浑身僵硬。而她则对他的拥抱没有丁点儿反应,甚至没有试图挣脱。现在他终于知道她发生了什么变化。她的气色很不好,脸上还有一条长长的疤痕,疤痕被头发遮住了一部分,从前额一直延续到太阳穴。但他感觉到的变化并不是这个。她的腰变粗了,很奇怪,也变硬了。记得有一次,在火箭弹爆炸之后,他帮忙将一具尸体从废墟里拖出来,他惊讶地发现尸体不仅重得让人难以置信,还非常僵硬,极难处理,就好像一块石头而非血肉之躯。她的身体就像那个尸体。他不禁觉得她皮肤的样子也可能和从前大不相同了。

\switchcolumn*

He did not attempt to kiss her, nor did they speak. As they walked back
across the grass she looked directly at him for the first time. It was
only a momentary glance, full of contempt and dislike. He wondered
whether it was a dislike that came purely out of the past or whether it
was inspired also by his bloated face and the water that the wind kept
squeezing from his eyes. They sat down on two iron chairs, side by side
but not too close together. He saw that she was about to speak. She
moved her clumsy shoe a few centimeters and deliberately crushed a twig.
Her feet seemed to have grown broader, he noticed.

\switchcolumn

他没有尝试去吻她,他们什么话都没有讲。当他们穿过草地往回走时,她第一次直视他的脸。那只是稍纵即逝的一眼,充满了轻蔑与厌恶。他不知道这厌恶究竟是单纯地源于往事,还是源于他那浮肿的脸和被风吹得淌着泪水的眼睛。他们肩并肩地在两把铁制的椅子上坐下来,并没有靠得很紧。她想说点什么,她将笨重的鞋子挪开了几厘米,还故意踩断了一根树枝。他注意到,她的脚好像变宽了。

\switchcolumn*

``I betrayed you,'' she said baldly.

\switchcolumn

``我背叛了你。''她直截了当地说。

\switchcolumn*

``I betrayed you,'' he said.

\switchcolumn

``我背叛了你。''他说。

\switchcolumn*

She gave him another quick look of dislike.

\switchcolumn

她又厌恶地快速看了他一眼。

\switchcolumn*

``Sometimes,'' she said, ``they threaten you with something---something you
can\textquotesingle t stand up to, can\textquotesingle t even think
about. And then you say, `Don\textquotesingle t do it to
me, do it to somebody else, do it to so-and-so.' And
perhaps you might pretend, afterwards, that it was only a trick and that
you just said it to make them stop and didn\textquotesingle t really
mean it. But that isn\textquotesingle t true. At the time when it
happens you do mean it. You think there\textquotesingle s no other way
of saving yourself, and you\textquotesingle re quite ready to save
yourself that way. You \emph{want} it to happen to the other person. You
don\textquotesingle t give a damn what they suffer. All you care about
is yourself.''

\switchcolumn

``有几次,''她说,``他们用你无法忍受的想都不能想的东西威胁你。然后你就会说`别这么对我,对别人这么做吧,对某人这么做吧''。你也许可以假装这是权宜之计,它不是你的本意,你只想让他们停下来。但这不是真的。事情发生时,你就是这个意思。你认为没有别的方法能救你,你希望它发生在其他人的身上。你并不在乎那人要承受什么,你只关心你自己。''

\switchcolumn*

``All you care about is yourself,'' he echoed.

\switchcolumn

``你只关心你自己。''他附和道。

\switchcolumn*

``And after that, you don\textquotesingle t feel the same toward the
other person any longer.''

\switchcolumn

``之后,你对那个人的感觉再也不一样了。''

\switchcolumn*

``No,'' he said, ``you don\textquotesingle t feel the same.''

\switchcolumn

``不一样了,''他说,``你感觉不一样了。''

\switchcolumn*

There did not seem to be anything more to say. The wind plastered their
thin overalls against their bodies. Almost at once it became
embarrassing to sit there in silence; besides, it was too cold to keep
still. She said something about catching her Tube and stood up to go.

\switchcolumn

好像再没有别的话好说。风将他们单薄的工作服吹得贴近他们的身体。差不多同时,他们觉得默默无语地坐在那里很尴尬,更何况就这么坐着也太冷了。她说她有事要赶地铁,起身要走。

\switchcolumn*

``We must meet again,'' he said.

\switchcolumn

``我们一定会再见面的。''他说。

\switchcolumn*

``Yes,'' she said, ``we must meet again.''

\switchcolumn

``是的,''她说,``我们一定会再见面。''

\switchcolumn*

He followed irresolutely for a little distance, half a pace behind her.
They did not speak again. She did not actually try to shake him off, but
walked at just such a speed as to prevent his keeping abreast of her. He
had made up his mind that he would accompany her as far as the Tube
station, but suddenly this process of trailing along in the cold seemed
pointless and unbearable. He was overwhelmed by a desire not so much to
get away from Julia as to get back to the Chestnut Tree Café, which had
never seemed so attractive as at this moment. He had a nostalgic vision
of his corner table, with the newspaper and the chessboard and the
ever-flowing gin. Above all, it would be warm in there. The next moment,
not altogether by accident, he allowed himself to become separated from
her by a small knot of people. He made a half-hearted attempt to catch
up, then slowed down, turned and made off in the opposite direction.
When he had gone fifty meters he looked back. The street was not
crowded, but already he could not distinguish her. Any one of a dozen
hurrying figures might have been hers. Perhaps her thickened, stiffened
body was no longer recognizable from behind.

\switchcolumn

他有些犹豫地又跟了她一小段距离,在她身后,保持半步之遥。他们都没有再说话。她不是真的想把他甩掉,只是她的速度刚好可以避免和他并肩而行,他决定陪着她一直走到地铁站,但突然,他又觉得在如此寒冷的天气里一路跟下去既毫无意义又很难受。于是,他强烈地想离开朱莉亚,返回栗树咖啡馆,后者似乎还从未像现在这样吸引他。他想念角落里的那张桌子,想念那里的报纸、棋盘以及一直会被斟满的杜松子酒,最重要的是那里一定很温暖。接着,并非完全出于偶然,他任由一小群人插进他和朱莉亚中间,他心不在焉地追赶她,没多久便放慢脚步,转身向相反的方向走去。走了五十米后,他又回过头看了一眼。街上的人还不算多,但已经看不清她了。她可能就在十多个形色匆忙的人当中,也许他已经不能从后面认出她那厚实、僵硬的身体了。

\switchcolumn*

``At the time when it happens,'' she had said, ``you do mean it.'' He had
meant it. He had not merely said it, he had wished it. He had wished
that she and not he should be delivered over to the------

\switchcolumn

``事情发生时,''她说过,``你就是这个意思。''他的确如此。他不仅这么说了,他还这么希望。他真的想他们应该把她而不是他,送上去——

\switchcolumn*

Something changed in the music that trickled from the telescreen. A
cracked and jeering note, a yellow note, came into it. And
then---perhaps it was not happening, perhaps it was only a memory taking
on the semblance of sound---a voice was singing:

\switchcolumn

有什么东西让电屏中原本舒缓的音乐发生了变化,出现了一个尖利又充满嘲弄的音调,一个警报式的音调。然后——也许并没有真的发生,也许仅仅是和声音有关的记忆——有个声音唱道:

\switchcolumn*

\begin{quotation}
 \noindent ``Under the spreading chestnut tree\\
  I sold you and you sold me---''
\end{quotation}

\switchcolumn

\begin{quotation}
\noindent 在栗子树的绿茵下,\\
我出卖了你,你出卖了我。
\end{quotation}

\switchcolumn*

The tears welled up in his eyes. A passing waiter noticed that his glass
was empty and came back with the gin bottle.

\switchcolumn


泪水从他的眼睛里涌了出来。一个侍者走过来,注意到他的杯子空了,就拿来杜松子酒。

\switchcolumn*

He took up his glass and sniffed at it. The stuff grew not less but more
horrible with every mouthful he drank. But it had become the element he
swam in. It was his life, his death, and his resurrection. It was gin
that sank him into stupor every night, and gin that revived him every
morning. When he woke, seldom before eleven hundred, with gummed-up
eyelids and fiery mouth and a back that seemed to be broken, it would
have been impossible even to rise from the horizontal if it had not been
for the bottle and teacup placed beside the bed overnight. Through the
mid-day hours he sat with glazed face, the bottle handy, listening to
the telescreen. From fifteen to closing time he was a fixture in the
Chestnut Tree. No one cared what he did any longer, no whistle woke him,
no telescreen admonished him. Occasionally, perhaps twice a week, he
went to a dusty, forgotten-looking office in the Ministry of Truth and
did a little work, or what was called work. He had been appointed to a
sub-committee of a sub-committee which had sprouted from one of the
innumerable committees dealing with minor difficulties that arose in the
compilation of the Eleventh Edition of the Newspeak dictionary. They
were engaged in producing something called an Interim Report, but what
it was that they were reporting on he had never definitely found out. It
was something to do with the question of whether commas should be placed
inside brackets, or outside. There were four others on the committee,
all of them persons similar to himself. There were days when they
assembled and then promptly dispersed again, frankly admitting to one
another that there was not really anything to be done. But there were
other days when they settled down to their work almost eagerly, making a
tremendous show of entering up their minutes and drafting long memoranda
which were never finished---when the argument as to what they were
supposedly arguing about grew extraordinarily involved and abstruse,
with subtle hagglings over definitions, enormous digressions,
quarrels---threats, even, to appeal to higher authority. And then
suddenly the life would go out of them and they would sit round the
table looking at one another with extinct eyes, like ghosts fading at
cock-crow.

\switchcolumn

他举杯闻了闻,虽然这东西一口比一口难喝,但他已沉溺其中。它是他的生命,他的死亡,他的复生。是杜松子酒让他在夜晚酩酊大醉,又是杜松子酒让他在上午恢复活力。他很少在11点前醒来,醒时眼睛犹如被胶水粘住睁不开,嘴里也仿佛有火烧一般。他的后背弯得像折断了一样,若不是前一天晚上在床边放了酒和茶杯,他甚至不可能爬起身来。在中午的几个小时里,他都神情呆滞地坐在那里收听电屏,手边还会放着一瓶酒。作为栗树咖啡馆的常客,他会从15点一直坐到打烊。没人在意他在做什么,电屏也不会呵斥他。有时,大概一周两次,他会去位于真理部的一间满是灰尘、被人遗忘的办公室里做点工作,或者说做一些所谓的工作。他被指派到一个委员会的下属委员会,其中前者是处理编纂第十一版新话词典细节问题的若干委员会中的一个。他们被雇来制定一个叫中期报告的东西,但他从来都不清楚这报告写的是什么,好像和逗号是写在括号里还是括号外的问题有关。委员会的其他四名成员都是和他类似的人。有时他们因为开会聚在一起,会一开完,马上就分开,彼此都坦率地承认其实没有什么事好做。但在另一些日子里,他们也会坐下来几乎是热切地工作,尽可能地表现自己,他们登记纲要,起草长长的从来没有完成过的备忘录——每当有争论,他们就会把争论变得极其深奥复杂,对定义吹毛求疵,将话题无限扯远,吵闹着相互威胁,甚至说要向上级汇报。然而,突然,他们又都没了气力,围坐在桌旁面面相觑,好像听见公鸡打鸣便隐去的鬼魂。

\switchcolumn*

The telescreen was silent for a moment. Winston raised his head again.
The bulletin! But no, they were merely changing the music. He had the
map of Africa behind his eyelids. The movement of the armies was a
diagram: a black arrow tearing vertically southward, and a white arrow
tearing horizontally eastward, across the tail of the first. As though
for reassurance he looked up at the imperturbable face in the portrait.
Was it conceivable that the second arrow did not even exist?

\switchcolumn

电屏安静了一会儿,温斯顿再一次将头抬了起来。公报!但,不是,他们仅仅是换了下
音乐。他眼前就有一幅非洲地图。军队的行动用图表的方式表示出来,一个黑色的箭头
直指南方,一个白色的箭头横躺着指着东方,并穿过了黑色箭头的尾部。似乎是为了寻
求安慰,他抬眼看了下画像上那张沉着的脸,怎么能认为第二个箭头是不存在的呢?

\switchcolumn*

His interest flagged again. He drank another mouthful of gin, picked up
the white knight, and made a tentative move. Check. But it was evidently
not the right move, because---

\switchcolumn

他的兴趣又消失了。他喝了一口杜松子酒,拿起白棋的马,试探性地走了一步。将军。
但显然这步棋下错了,因为他毫无理由地想起了一段往事。

\switchcolumn*

Uncalled, a memory floated into his mind. He saw a candle-lit room with
a vast white-counterpaned bed, and himself, a boy of nine or ten,
sitting on the floor, shaking a dice box and laughing excitedly. His
mother was sitting opposite him and also laughing.

\switchcolumn

他想起一间点着蜡烛的屋子,屋子里有一张铺着白色床单的大床,还有他自己。他是一
个九岁或十岁的男孩,正坐在地板上摇着一个骰子盒,兴奋地哈哈大笑。他的母亲坐在
他的对面,也在笑着。

\switchcolumn*

It must have been about a month before she disappeared. It was a moment
of reconciliation, when the nagging hunger in his belly was forgotten
and his earlier affection for her had temporarily revived. He remembered
the day well, a pelting, drenching day when the water streamed down the
window pane and the light indoors was too dull to read by. The boredom
of the two children in the dark, cramped bedroom became unbearable.
Winston whined and grizzled, made futile demands for food, fretted about
the room, pulling everything out of place and kicking the wainscoting
until the neighbors banged on the wall, while the younger child wailed
intermittently. In the end his mother had said, ``Now be good, and
I\textquotesingle ll buy you a toy. A lovely
toy---you\textquotesingle ll love it''; and then she had gone out in the
rain, to a little general shop which was still sporadically open near
by, and come back with a cardboard box containing an outfit of Snakes
and Ladders. He could still remember the smell of the damp cardboard. It
was a miserable outfit. The board was cracked and the tiny wooden dice
were so ill-cut that they would hardly lie on their sides. Winston
looked at the thing sulkily and without interest. But then his mother
lit a piece of candle and they sat down on the floor to play. Soon he
was wildly excited and shouting with laughter as the tiddlywinks climbed
hopefully up the ladders and then came slithering down the snakes again,
almost back to the starting point. They played eight games, winning four
each. His tiny sister, too young to understand what the game was about,
had sat propped up against a bolster, laughing because the others were
laughing. For a whole afternoon they had all been happy together, as in
his earlier childhood.

\switchcolumn

这一定是她失踪前的一个月。那时,他们已经和好,他忘记了没完没了的饥饿感,对她也暂时恢复了小时候的爱恋。他记得很清楚,那是大雨滂沱的一天,雨水顺着窗框倾泻而下,屋子里的光线太暗了,无法看书。待在阴暗狭窄的卧室里,让两个孩子十分无聊。温斯顿一面抱怨,一面大发脾气,徒劳地要着吃的。他非常焦躁,将屋子里的所有东西都扯了出来,还大踢墙板直到邻居敲打墙壁以示抗议,与此同时比他小的那个孩子不时便号啕大哭。最后,他的母亲说:``乖乖的,我去给你买玩具。一个可爱的玩具——你会喜欢它的。''然后,她就冒着大雨出去了,附近仍有几个店子开张营业。她带回来一个硬纸箱,纸箱里装着一副蛇梯棋。他仍然记得硬纸板那潮湿的气味。棋做得很差劲,棋盘裂开了缝,木质的小骰子切坏了,以至于几乎不能躺平。温斯顿闷闷不乐地看着它,一点儿兴趣都没有。但他的母亲却点亮了一支蜡烛,之后,他们便坐在地板上玩了起来。过了一会儿,他便非常兴奋,又叫又笑,那个小棋子很有希望地爬到梯子顶,可接下来又一下子掉回了起点。他们玩了八次,各赢四次。他的妹妹太小了,还不明白他们在玩什么,她靠着长枕坐在那里,他们笑,她也跟着笑。整个下午,他们都待在一起很是快乐,一如他的幼年时代。

\switchcolumn*

He pushed the picture out of his mind. It was a false memory. He was
troubled by false memories occasionally. They did not matter so long as
one knew them for what they were. Some things had happened, others had
not happened. He turned back to the chessboard and picked up the white
knight again. Almost in the same instant it dropped onto the board with
a clatter. He had started as though a pin had run into him.

\switchcolumn

他将这幅景象从意识里赶走。这记忆是虚假的。这虚假的记忆经常让他感到烦恼。不过,只要人知道它们的虚假本质,它们就不再重要。一些事情发生过,一些没有。他重新注意起象棋,再次拿起白棋的马。而几乎同时棋子啪的一声掉在了棋盘上。他吓了一跳,就像被针扎到了一样。

\switchcolumn*

A shrill trumpet call had pierced the air. It was the bulletin! Victory!
It always meant victory when a trumpet call preceded the news. A sort of
electric thrill ran through the café. Even the waiters had started and
pricked up their ears.

\switchcolumn

一个尖利的喇叭声刺破了空气。是公报!胜利!喇叭声在新闻前出现就意味着胜利的消息。咖啡馆就像通了电一般激动,就连侍者也惊呆了,竖起了耳朵。

\switchcolumn*

The trumpet call had let loose an enormous volume of noise. Already an
excited voice was gabbling from the telescreen, but even as it started
it was almost drowned by a roar of cheering from outside. The news had
run round the streets like magic. He could hear just enough of what was
issuing from the telescreen to realize that it had all happened as he
had foreseen: a vast seaborne armada secretly assembled, a sudden blow
in the enemy\textquotesingle s rear, the white arrow tearing across the
tail of the black. Fragments of triumphant phrases pushed themselves
through the din: ``Vast strategic maneuver---perfect
co-ordination---utter rout---half a million prisoners---complete
demoralization---control of the whole of Africa---bring the war within
measurable distance of its end---victory---greatest victory in human
history---victory, victory, victory!''

\switchcolumn

喇叭声引起了巨大的噪音。电屏里传出充满激情的声音,喋喋不休地说着什么,可外面庆祝式的吼叫声却几乎将它淹没。消息像变戏法那样在街上传开。他能够从电屏上听到一切都如他所料。一支海军舰队秘密地集合起来突袭了敌军后方,白色箭头穿过了黑色箭头的尾部。透过喧嚣,他断断续续地听到一些胜利的短语:``大规模的展览调动——完美的——配合——彻底击溃——五十万俘虏——完全丧失士气——控制整个非洲——战争的最终胜利可以预测——人类历史的伟大胜利——胜利,胜利,胜利!''

\switchcolumn*

Under the table Winston\textquotesingle s feet made convulsive
movements. He had not stirred from his seat, but in his mind he was
running, swiftly running, he was with the crowds outside, cheering
himself deaf. He looked up again at the portrait of Big Brother. The
colossus that bestrode the world! The rock against which the hordes of
Asia dashed themselves in vain! He thought how ten minutes ago---yes,
only ten minutes---there had still been equivocation in his heart as he
wondered whether the news from the front would be of victory or defeat.
Ah, it was more than a Eurasian army that had perished! Much had changed
in him since that first day in the Ministry of Love, but the final,
indispensable, healing change had never happened, until this moment.

\switchcolumn

桌子底下,温斯顿的脚抽筋式地抖动着。他定定地坐在那里,但在他的意识里,他却在奔跑,飞快地奔跑,和外面的人群一起,欢呼得耳朵都要聋了。他又抬头看了看老大哥的画像。这个驾驭世界的巨人!他是将亚洲之众撞得晕头转向的巨石!他想起,就在十分钟之前——没错,仅仅十分钟之前——就在他思考前线会传来胜利还是战败的消息的时候,他还满心困惑。啊,灭亡的岂止是一支欧亚国大军!从进入仁爱部的第一天起,他已发生了很大的变化,但是直到此刻,他才发生了最终的、不能缺少的、治愈性的改变。

\switchcolumn*

The voice from the telescreen was still pouring forth its tale of
prisoners and booty and slaughter, but the shouting outside had died
down a little. The waiters were turning back to their work. One of them
approached with the gin bottle. Winston, sitting in a blissful dream,
paid no attention as his glass was filled up. He was not running or
cheering any longer. He was back in the Ministry of Love, with
everything forgiven, his soul white as snow. He was in the public dock,
confessing everything, implicating everybody. He was walking down the
white-tiled corridor, with the feeling of walking in sunlight, and an
armed guard at his back. The long-hoped-for bullet was entering his
brain.

\switchcolumn

电屏中的声音仍滔滔不绝地讲着俘虏、战利品和屠杀,外面的喊叫声已经减弱了一些。侍者们又回去工作了,他们中的一个拿来杜松子酒。温斯顿沉浸在美好的梦境中,没有注意到酒杯已被斟满。他不再奔跑欢呼,他重新回到仁爱部,所有事情都得到了原谅,他的灵魂洁白似雪。他站在公开的被告席上坦白了所有事情,牵连了所有的人。他走在铺着白色瓷砖的走廊里,就像走在阳光中,身后跟着携着枪的看守。期待已久的子弹射穿了他的脑袋。

\switchcolumn*

He gazed up at the enormous face. Forty years it had taken him to learn
what kind of smile was hidden beneath the dark mustache. O cruel,
needless misunderstanding! O stubborn, self-willed exile from the loving
breast! Two gin-scented tears trickled down the sides of his nose. But
it was all right, everything was all right, the struggle was finished.
He had won the victory over himself. He loved Big Brother.

\switchcolumn

他凝望着那张巨大的面孔。四十年了,他终于领会那隐藏在黑色胡须下的微笑意味着什么。哦,这是残酷又毫无必要的误会!哦,倔强任性的,从那充满慈爱的胸膛里自我流放的人!两颗混着杜松子酒味的眼泪顺着他的鼻梁流了下来。但这也好,一切都很好,斗争结束了。他战胜了自己,他爱老大哥。

\switchcolumn*

\begin{center}
\textbf{THE END}
\end{center}

\switchcolumn

\begin{center}
\textbf{THE END}
\end{center}

\switchcolumn*


\clearpage
\part{APPENDIX}\label{appendix}

\switchcolumn

\part*{附录}

\switchcolumn*

\section{THE PRINCIPLES OF NEWSPEAK}\label{the-principles-of-newspeak}

\switchcolumn

\section*{新话原则}\label{ux65b0ux8bddux539fux5219}

\switchcolumn*

Newspeak was the official language of Oceania and had been devised to
meet the ideological needs of Ingsoc, or English Socialism. In the year
1984 there was not as yet anyone who used Newspeak as his sole means of
communication, either in speech or writing. The leading articles in the
\emph{Times} were written in it, but this was a tour de force which
could only be carried out by a specialist. It was expected that Newspeak
would have finally superseded Oldspeak (or Standard English, as we
should call it) by about the year 2050. Meanwhile it gained ground
steadily, all Party members tending to use Newspeak words and
grammatical constructions more and more in their everyday speech. The
version in use in 1984, and embodied in the Ninth and Tenth Editions of
the Newspeak dictionary, was a provisional one, and contained many
superfluous words and archaic formations which were due to be suppressed
later. It is with the final, perfected version, as embodied in the
Eleventh Edition of the dictionary, that we are concerned here.

\switchcolumn

新话是大洋国的官方语言,也是应英社,即英国社会主义的意识形态而生的一种语言。到1984年为止,还没有人能将新话作为唯一的交流用语,演讲和写作都是如此。《泰晤士报》的社论是用新话书写的,但那是只能由专家写就的经典。预计要到2050年,新话才能最终将老话(或者用我们的话说``英语'')取代。与此同时,它正逐步扩大自己的影响力,所有党员都倾向使用新话,且越来越多地在日常生活中运用新话词汇及新话语法。1984年使用的版本以及第九版和第十版的新话词典所体现的新话原则都是临时性的,其中包括大量冗余词汇和过时的词形,所有这些都会在未来予以废止。最终也是最完美的版本,是第十一版新话词典。在这里,我们所关注的即该版的新话词典。

\switchcolumn*

The purpose of Newspeak was not only to provide a medium of expression
for the world-view and mental habits proper to the devotees of Ingsoc,
but to make all other modes of thought impossible. It was intended that
when Newspeak had been adopted once and for all and Oldspeak forgotten,
a heretical thought---that is, a thought diverging from the principles
of Ingsoc---should be literally unthinkable, at least so far as thought
is dependent on words. Its vocabulary was so constructed as to give
exact and often very subtle expression to every meaning that a Party
member could properly wish to express, while excluding all other
meanings and also the possibility of arriving at them by indirect
methods. This was done partly by the invention of new words, but chiefly
by eliminating undesirable words and by stripping such words as remained
of unorthodox meanings, and so far as possible of all secondary meanings
whatever. To give a single example. The word \emph{free} still existed
in Newspeak, but it could only be used in such statements as ``This dog
is free from lice'' or ``This field is free from weeds.'' It could not be
used in its old sense of ``politically free'' or ``intellectually free,''
since political and intellectual freedom no longer existed even as
concepts, and were therefore of necessity nameless. Quite apart from the
suppression of definitely heretical words, reduction of vocabulary was
regarded as an end in itself, and no word that could be dispensed with
was allowed to survive. Newspeak was designed not to extend but to
\emph{diminish} the range of thought, and this purpose was indirectly
assisted by cutting the choice of words down to a minimum.

\switchcolumn

新话的目的不单是为英社的拥护者提供一种表达世界观和思想习惯的媒介,更是为了让其他所有的思维模式都成为不可能。它的职责是在新话被采用,且所有人都将老话遗忘后,异端思想——也就是那些违背英社原则的思想——从字面上来说,就不可能被想到,至少在思想依赖于词汇的情况下是这样。新话词汇之所以采用这样的构建方式,是为了让党员能够准确且微妙地表达出他们的正确意图,同时将其他含义和通过间接途径获得这些含义的可能性全部排除在外。要达到这个目的,部分是靠新词汇的创造,但主要是靠废除不合适的词汇以及清除残留下来的词的非正统的意义,尽最大可能地取缔它们的另一层含义。举个简单的例子,``free''在新话中仍然存在,但它只能用在以下一些语句中。如``This
dog is free from lice.''(这条狗的身上没有跳蚤)或者``This field is free
from weeds.''(这块田里没有杂草)而不能用在``politically
free''(政治自由)或``intellectually
free''(学术自由)上。因为政治自由和学术自由即使作为概念都已不再存在,所以必须不能被冠以名称。而不止要禁止使用确实包含异端思想的词,词汇的数量也被认为是为了减少而减少。但凡可以被省略的词都不允许存在。新话的目的不是扩大词的思想范围,而是将词的思想范围缩小,通过将可供选择的词的数量削减到最少,间接地达到这一目的。

\switchcolumn*

Newspeak was founded on the English language as we now know it, though
many Newspeak sentences, even when not containing newly created words,
would be barely intelligible to an English-speaker of our own day.
Newspeak words were divided into three distinct classes, known as the A
vocabulary, the B vocabulary (also called compound words), and the C
vocabulary. It will be simpler to discuss each class separately, but the
grammatical peculiarities of the language can be dealt with in the
section devoted to the A vocabulary, since the same rules held good for
all three categories.

\switchcolumn

新话的基础就是今天为我们掌握的英语,尽管一些新话句子,即使不包括新创造的词汇,对现在正使用英语的人而言也很难理解。新话词汇包括三大类,即A类词汇、B类词汇(也称复合词)、C类词汇。分别讨论这三类词汇相对简单,至于该语言的语法特点可以归入A类词汇的讨论中,因为这三类词汇都是用相同的语法规则。

\switchcolumn*

\sectionbreak

\switchcolumn

\sectionbreak

\switchcolumn*

\emph{The A vocabulary.} The A vocabulary consisted of the words needed
for the business of everyday life---for such things as eating, drinking,
working, putting on one\textquotesingle s clothes, going up and down
stairs, riding in vehicles, gardening, cooking, and the like. It was
composed almost entirely of words that we already possess---words like
\emph{hit}, \emph{run}, \emph{dog}, \emph{tree}, \emph{sugar},
\emph{house}, \emph{field}---but in comparison with the present-day
English vocabulary, their number was extremely small, while their
meanings were far more rigidly defined. All ambiguities and shades of
meaning had been purged out of them. So far as it could be achieved, a
Newspeak word of this class was simply a staccato sound expressing
\emph{one} clearly understood concept. It would have been quite
impossible to use the A vocabulary for literary purposes or for
political or philosophical discussion. It was intended only to express
simple, purposive thoughts, usually involving concrete objects or
physical actions.

\switchcolumn

\emph{A类词汇。} A类词汇包括日常生活用词,例如吃、喝、工作、穿衣、上下楼、乘车、种花、烹饪等。几乎全部由我们现有的单词组成,例如打、跑、狗、树、糖、房子、田野等。但和今天的英语词汇相比,其数量已十分稀少,含义也被严格限定。所有模棱两可、含混不清的意义都被清除。只要可以,该类别的新话词汇都表达的是单一而明确的概念,可以被简单地看做一种不连贯的声音。使用A类词汇进行文学创作或讨论政治、哲学是完全不可能的。它只起到表达单纯而目的明确的思想的作用,通常只涉及具体的事物或身体动作。

\switchcolumn*

The grammar of Newspeak had two outstanding peculiarities. The first of
these was an almost complete interchangeability between different parts
of speech. Any word in the language (in principle this applied even to
very abstract words such as \emph{if} or \emph{when}) could be used
either as verb, noun, adjective, or adverb. Between the verb and the
noun form, when they were of the same root, there was never any
variation, this rule of itself involving the destruction of many archaic
forms. The word \emph{thought}, for example, did not exist in Newspeak.
Its place was taken by \emph{think}, which did duty for both noun and
verb. No etymological principle was followed here; in some cases it was
the original noun that was chosen for retention, in other cases the
verb. Even where a noun and verb of kindred meaning were not
etymologically connected, one or other of them was frequently
suppressed. There was, for example, no such word as \emph{cut}, its
meaning being sufficiently covered by the noun-verb \emph{knife}.
Adjectives were formed by adding the suffix \emph{-ful} to the
noun-verb, and adverbs by adding \emph{-wise}. Thus, for example,
\emph{speedful} meant ``rapid'' and \emph{speedwise} meant ``quickly.''
Certain of our present-day adjectives, such as \emph{good},
\emph{strong}, \emph{big}, \emph{black}, \emph{soft}, were retained, but
their total number was very small. There was little need for them, since
almost any adjectival meaning could be arrived at by adding \emph{-ful}
to a noun-verb. None of the now-existing adverbs was retained, except
for a very few already ending in \emph{-wise}; the \emph{-wise}
termination was invariable. The word \emph{well}, for example, was
replaced by \emph{goodwise}.

\switchcolumn

新话语法有两个特点。第一个特点是,不同的词性几乎可以互相替换。任何一个单词(原则上,甚至适用于像``if''或``when''这样非常抽象的词)都可以既作动词,又作名词、形容词或副词。若词根相同,动词与名词就不存在形式上的区别,该规则使得若干旧词形被废止。如``thought''(思想)一词,在新话中就不再存在,被``think''(思考)取而代之,后者既可以充当名词又可以充当动词。在这里人们无须遵循语源学的原则:某些情况下保留原有的名词,某些情况下又保留原有的动词。即便意思相近的动词与名词之间没有语源上的联系,它们中的一个也会被弃而不用。譬如不会有``cut''(切、割),因为``knife''(刀)包含了它的意思。在名词兼动词的词后面加``ful''就能将该词变成形容词的词形,在名词后加``wise''就能得到该词的副词形态。比如``speedful''是``rapid''(迅速的)的意思,``speedwise''就是``quickly''(迅速地)的意思。如今,我们所使用的形容词,如``good''(好)、``strong''(强壮)、``big''(大)、``black''(黑)、``soft''(柔软)都保留了下来,但数量很少。人们几乎不需要用到它们,因为差不多任何一个形容词都可以通过在名词兼动词的词后面加``ful''获得。除了极个别本来就以``wise''作词尾的词,所有副词都被取消:所有副词一律以``wise''作结尾。比如``well''(好,副词)便被``goodwise''取代。

\switchcolumn*

In addition, any word---this again applied in principle to every word in
the language---could be negatived by adding the affix \emph{un-}, or
could be strengthened by the affix \emph{plus-}, or, for still greater
emphasis, \emph{doubleplus-}. Thus, for example, \emph{uncold} meant
``warm,'' while \emph{pluscold} and \emph{doublepluscold} meant,
respectively, ``very cold'' and ``superlatively cold.'' It was also
possible, as in present-day English, to modify the meaning of almost any
word by prepositional affixes such as \emph{ante-}, \emph{post-},
\emph{up-}, \emph{down-}, etc. By such methods it was found possible to
bring about an enormous diminution of vocabulary. Given, for instance,
the word \emph{good}, there was no need for such a word as \emph{bad},
since the required meaning was equally well---indeed, better---expressed
by \emph{ungood}. All that was necessary, in any case where two words
formed a natural pair of opposites, was to decide which of them to
suppress. \emph{Dark}, for example, could be replaced by \emph{unlight},
or \emph{light} by \emph{undark}, according to preference.

\switchcolumn

此外,所有单词——原则上适用于每一个单词——都能通过加``un''的前缀表示否定,或者通过加``plus''的前缀加重语气、加``doubleplus''的前缀进一步强调词意。比如``uncold''(不冷)的意思就是``warm''(暖和),``pluscold''的意思是``very
cold''(很冷),``doublepluscold''的意思是``superlatively
cold''(极冷)。一如当代英语可以通过``anti''、``post''、``up''、``down''等前缀来更改一个词的含义。使用这一方法可以让词汇的数量大为减少。例如有了单词``good''(好),就没必要有单词``bad''(坏)了,因为``ungood''即可以表示相同的含义——这样确实更好些。任何本来就词义相对的两个词,都需要决定到底要废止其中的哪一个。比如``dark''(黑暗)可以被``unlight''(不亮)取代,``light''(明亮)也可以用``undark''(不暗)取代,就看你更喜欢用哪个了。

\switchcolumn*

The second distinguishing mark of Newspeak grammar was its regularity.
Subject to a few exceptions which are mentioned below, all inflections
followed the same rules. Thus, in all verbs the preterite and the past
participle were the same and ended in \emph{-ed}. The preterite of
\emph{steal} was \emph{stealed}, the preterite of \emph{think} was
\emph{thinked}, and so on throughout the language, all such forms as
\emph{swam}, \emph{gave}, \emph{brought}, \emph{spoke}, \emph{taken},
etc., being abolished. All plurals were made by adding \emph{-s} or
\emph{-es} as the case might be. The plurals of \emph{man}, \emph{ox},
\emph{life} were \emph{mans}, \emph{oxes}, \emph{lifes}. Comparison of
adjectives was invariably made by adding \emph{-er}, \emph{-est}
(\emph{good}, \emph{gooder}, \emph{goodest}), irregular forms and the
\emph{more}, \emph{most} formation being suppressed.

\switchcolumn

新话语法的第二个明显特点是它的规律性。除了以下提到的个别情况,所有词形都按照同一规则变化。因此,所有动词的过去式和过去分词都用``ed''作结尾。``steal''(偷)的过去式就是``stealed'',``think''(思考)的过去式是``thinked'',新话中都是如此。而所有诸如``swam''(``swim''的过去式)、``gave''(``give''的过去式)、``brought''(``bring''的过去式与过去分词)、``spoke''(``speak''的过去式)、``taken''(``take''的过去分词)等都被废止。所有复数都根据情况加``s''或``es''的后缀。``man''(人)、``ox''(公牛)、``life''(生物)的复数形式分别为``mans''、``oxes''、``lifes''。形容词比较级全部加``er''的后缀,最高级加``est''的后缀,如``good''、``gooder''、``goodest'',像``more''和``most''这样的不规则的结构变化都被取消。

\switchcolumn*

The only classes of words that were still allowed to inflect irregularly
were the pronouns, the relatives, the demonstrative adjectives, and the
auxiliary verbs. All of these followed their ancient usage, except that
\emph{whom} had been scrapped as unnecessary, and the \emph{shall},
\emph{should} tenses had been dropped, all their uses being covered by
\emph{will} and \emph{would}. There were also certain irregularities in
word-formation arising out of the need for rapid and easy speech. A word
which was difficult to utter, or was liable to be incorrectly heard, was
held to be ipso facto a bad word; occasionally therefore, for the sake
of euphony, extra letters were inserted into a word or an archaic
formation was retained. But this need made itself felt chiefly in
connection with the B vocabulary. \emph{Why} so great an importance was
attached to ease of pronunciation will be made clear later in this
essay.

\switchcolumn

只有代词、关系形容词、指示形容词和助动词仍可进行不规则变化。除了``whom''因被认为是多余的而被取消,``shall''、``should''等时态被``will''、``would''代替,所有这些词都沿用旧有的使用方法。同时,一些不规则的形态变化是为了确保讲话便捷。倘一个词很难发音或极易被人听错,那么依照事实,它将被当做一个坏单词,出于悦耳的考虑,有时需要在该词中增加几个字母或保留它旧有的词形。但这种情况主要出现在B类词汇中。而为什么发音会如此重要,接下来的段落将对此做出解释。

\switchcolumn*

\sectionbreak

\switchcolumn

\sectionbreak

\switchcolumn*

\emph{The B vocabulary.} The B vocabulary consisted of words which had
been deliberately constructed for political purposes: words, that is to
say, which not only had in every case a political implication, but were
intended to impose a desirable mental attitude upon the person using
them. Without a full understanding of the principles of Ingsoc it was
difficult to use these words correctly. In some cases they could be
translated into Oldspeak, or even into words taken from the A
vocabulary, but this usually demanded a long paraphrase and always
involved the loss of certain overtones. The B words were a sort of
verbal short-hand, often packing whole ranges of ideas into a few
syllables, and at the same time more accurate and forcible than ordinary
language.

\switchcolumn

\emph{B类词汇。} B类词汇都是为了政治目的特意创造出来的,也就是说,不只每个单词都包含政治意味,且创造这些单词的目的就是为了确保使用者能够具备令当局满意的思想态度。若对英社的了解不够多,则很难使用这些单词。有时,它们可以被翻译成老话,甚至是A类词汇,但这通常需要附上大段的解释并总是会漏掉一些言外之意。B类词汇是一种非书面的简要的表达方式,常用少数几个音节表达若干个含义,比一般的语言更精确、更有力。

\switchcolumn*

The B words were in all cases compound words.\footnote{Compound words, such
  as \emph{speakwrite}, were of course to be found in the A vocabulary, but
  these were merely convenient abbreviations and had no special ideological
  color.} They consisted of two or more words, or portions of words, welded
together in an easily pronounceable form. The resulting amalgam was always a
noun-verb, and inflected according to the ordinary rules. To take a single
example: the word \emph{goodthink}, meaning, very roughly, ``orthodoxy,''
or, if one chose to regard it as a verb, ``to think in an orthodox manner.''
This inflected as follows: noun-verb, \emph{goodthink}; past tense and past
participle, \emph{goodthinked}; present participle, \emph{goodthinking};
adjective, \emph{goodthinkful}; adverb, \emph{goodthinkwise}; verbal noun,
\emph{goodthinker}.

\switchcolumn

B类词汇都是复合词。它们由两个或两个以上的单词,或者几个单词的部分,按照非常容易发音的形式组合起来。其结果便是得出了遵循一般变化规则的,且动、名词兼用的混合词。譬如``goodthink''(好思想)大致可看作``orthodoxy''(正统)的意思,如果用作动词,即指``以正统的方式思考''。而它的形态变化是:名词兼动词``goodthink'',过去式和过去分词``goodthinked'',现在分词``goodthinking'',形容词``goodthinkful'',副词``goodthinkwise''、动名词``goodthinker''。

\switchcolumn*

The B words were not constructed on any etymological plan. The words of
which they were made up could be any parts of speech, and could be
placed in any order and mutilated in any way which made them easy to
pronounce while indicating their derivation. In the word
\emph{crimethink} (thoughtcrime), for instance, the \emph{think} came
second, whereas in \emph{thinkpol} (Thought Police) it came first, and
in the latter word \emph{police} had lost its second syllable. Because
of the greater difficulty in securing euphony, irregular formations were
commoner in the B vocabulary than in the A vocabulary. For example, the
adjectival forms of \emph{Minitrue}, \emph{Minipax}, and \emph{Miniluv}
were, respectively, \emph{Minitruthful}, \emph{Minipeaceful}, and
\emph{Minilovely}, simply because \emph{-trueful}, \emph{-paxful}, and
\emph{-loveful} were slightly awkward to pronounce. In principle,
however, all B words could inflect, and all inflected in exactly the
same way.

\switchcolumn

B类词汇并非按照词源学构建,任何词性的单词都可以成为它的构成部分,且可以采取任何排列顺序,按任何一种方式进行修改,既要便于发音,又要说明来历。比如``crimethink''(思想罪)一词,``think''放在了后面,而在``thinkpol''(思想警察)一词中,它又放在了前面,同时后面的``pol''又是``police''(警察)略去第二个音节的结果。由于要让单词读起来悦耳并不容易,B类词汇中的不规则词形要比A类词汇的多。譬如``Minitrue''(真理部)、``Minipax''(和平部)、``Miniluv''(仁爱部)的形容词分别是``Minitruthful''、``Minipeaceful''、``Minilovely'',单纯因为若改为``-trueful''、``-paxful''、``-loveful''发起音来有点难,所以才不予采用。从原则上说,所有B类词汇都可以变化,且变化的方式完全相同。

\switchcolumn*

Some of the B words had highly subtilized meanings, barely intelligible
to anyone who had not mastered the language as a whole. Consider, for
example, such a typical sentence from a \emph{Times} leading article as
\emph{Oldthinkers unbellyfeel Ingsoc}. The shortest rendering that one
could make of this in Oldspeak would be: ``Those whose ideas were formed
before the Revolution cannot have a full emotional understanding of the
principles of English Socialism.'' But this is not an adequate
translation. To begin with, in order to grasp the full meaning of the
Newspeak sentence quoted above, one would have to have a clear idea of
what is meant by \emph{Ingsoc}. And, in addition, only a person
thoroughly grounded in Ingsoc could appreciate the full force of the
word \emph{bellyfeel}, which implied a blind, enthusiastic acceptance
difficult to imagine today; or of the word \emph{oldthink}, which was
inextricably mixed up with the idea of wickedness and decadence. But the
special function of certain Newspeak words, of which \emph{oldthink} was
one, was not so much to express meanings as to destroy them. These
words, necessarily few in number, had had their meanings extended until
they contained within themselves whole batteries of words which, as they
were sufficiently covered by a single comprehensive term, could now be
scrapped and forgotten. The greatest difficulty facing the compilers of
the Newspeak dictionary was not to invent new words, but, having
invented them, to make sure what they meant: to make sure, that is to
say, what ranges of words they canceled by their existence.

\switchcolumn

一些B类词汇的含义十分微妙,任何没能完全掌握这种语言的人都难以理解。以《泰晤士报》头条文章中的典型句子为例:``Olethinkers
unbellyfeel Ingsoc''用老话翻译,最短的也是``Those whose ideas were
formed before the Revolution cannot have a full emotional understanding
of the principles of English
Socialism.''(那些思想在革命前成形的人不可能在感情上对英国社会主义的原则有充分的理解)但这并没有将句子的意思完全译出来。首先,为了彻底理解新话写就的句子的含义,一个人必须清楚``Ingsoc''(英社)的概念。此外,只有对英社有了相当程度的了解,一个人才能真的体会到``bellyfeel''一词的全部力量,它的意思是一种盲目的如今以难以想象的狂热的接受。``oldthink''也是一样,该单词已经和邪恶、堕落密不可分。但是,在新话中一些单词有特殊的作用,``oldthink''就是其中之一,这类词并不是要表达某个意思,而是要将该意思消灭。这些词的数量很少,却非常必要,它们的含义被引申直到囊括进一堆单词,由于这些单词的含义都被包含在一个综合性的术语里,它们自己便可以被遗忘、被丢弃了。对新话的编纂者而言,最大的困难不是创造新词,而是在创造它们之后明确它们的含义,即在创造出它们后,确定到底该有哪类范围的词被取消。

\switchcolumn*

As we have already seen in the case of the word \emph{free}, words which
had once borne a heretical meaning were sometimes retained for the sake
of convenience, but only with the undesirable meanings purged out of
them. Countless other words such as \emph{honor}, \emph{justice},
\emph{morality}, \emph{internationalism}, \emph{democracy},
\emph{science}, and \emph{religion} had simply ceased to exist. A few
blanket words covered them, and, in covering them, abolished them. All
words groupings themselves round the concepts of liberty and equality,
for instance, were contained in the single word \emph{crimethink}, while
all words grouping themselves round the concepts of objectivity and
rationalism were contained in the single word \emph{oldthink}. Greater
precision would have been dangerous. What was required in a Party member
was an outlook similar to that of the ancient Hebrew who knew, without
knowing much else, that all nations other than his own worshiped ``false
gods.'' He did not need to know that these gods were called Baal, Osiris,
Moloch, Ashtaroth, and the like; probably the less he knew about them
the better for his orthodoxy. He knew Jehovah and the commandments of
Jehovah; he knew, therefore, that all gods with other names or other
attributes were false gods. In somewhat the same way, the Party member
knew what constituted right conduct, and in exceedingly vague,
generalized terms he knew what kinds of departure from it were possible.
His sexual life, for example, was entirely regulated by the two Newspeak
words \emph{sexcrime} (sexual immorality) and \emph{goodsex} (chastity).
\emph{Sexcrime} covered all sexual misdeeds whatever. It covered
fornication, adultery, homosexuality, and other perversions, and, in
addition, normal intercourse practiced for its own sake. There was no
need to enumerate them separately, since they were all equally culpable,
and, in principle, all punishable by death. In the C vocabulary, which
consisted of scientific and technical words, it might be necessary to
give specialized names to certain sexual aberrations, but the ordinary
citizen had no need of them. He knew what was meant by
\emph{goodsex}---that is to say, normal intercourse between man and
wife, for the sole purpose of begetting children, and without physical
pleasure on the part of the woman; all else was \emph{sexcrime}. In
Newspeak it was seldom possible to follow a heretical thought further
than the perception that it \emph{was} heretical; beyond that point the
necessary words were nonexistent.

\switchcolumn

正如我们在``free''(自由)一词上看到的一样,有时一些曾包含异端意味的词为了方便被保留下来,但只有在清除掉它身上那些不当的含义之后,才能如此。像``honour''(荣誉)、``justice''(正义)、``morality''(道德)、``internationalism''(国际主义)、``democracy''(民主)、``science''(科学)、``religion''(宗教)等词都不存在了。它们被少数几个概括性的词的词义所覆盖,进而遭到废弃。比如所有和自由平等有关的词都被``crimethink''(思想罪)包括,所有和客观、理性有关的词都被包含进``oldthink''(旧思想)当中。若再精确一些就有危险了。党员的观点应该和古希伯莱人的类似,后者被认为不应该知道太多,只需知道其他民族的人都崇拜伪神就够了。至于这些神祇是``Beal''(古凯尔特人太阳神)、``Osiris''(埃及神话中的地狱之神)、``Moloch''(古腓尼基人的火神)、``Ashtaroth''(古腓尼基人的守护女神),他们就没必要了解了。也许,照正统看来,他们知道得越少越好。由于他知道耶和华的戒律,所以他清楚所有其他名字和其他性质的神都是伪神。党员也是一样,党员知道什么是正确的行为,也非常模糊地笼统地知道哪些行为是离经叛道。例如他的性生活被两个词约束,``sexcrime''(性犯罪)和``goodsex''(好的性)。``sexcrime''涵盖了所有性方面的不端行为,包括乱伦、通奸、同性恋及其他堕落的行为,同时也包括了正常的为了性交而性交的行为。没必要对它们加以区分,它们都算犯罪,且原则上说,都要被处以死刑。C类词汇里,即由科学和技术方面的单词所构成的词,可能需要赋予某些反常的性行为以特殊的名称,但一般的公民是不需要这些词汇的。他知道``goodsex''是什么,即发生在男人和他妻子之间的正常的性行为,该性行为的唯一目的便是生儿育女,且女方没有生理上的快感。除此之外,其他的性行为都是``sexcrime''。新话里很少会有在察觉到某种想法为异端时仍继续思考下去的可能,因为除了可以意识到其为异端,其他思考所需的词都不存在。

\switchcolumn*

No word in the B vocabulary was ideologically neutral. A great many were
euphemisms. Such words, for instance, as \emph{joycamp} (forced-labor
camp) or \emph{Minipax} (Ministry of Peace, i.e., Ministry of War) meant
almost the exact opposite of what they appeared to mean. Some words, on
the other hand, displayed a frank and contemptuous understanding of the
real nature of Oceanic society. An example was \emph{prolefeed}, meaning
the rubbishy entertainment and spurious news which the Party handed out
to the masses. Other words, again, were ambivalent, having the
connotation ``good'' when applied to the Party and ``bad'' when applied to
its enemies. But in addition there were great numbers of words which at
first sight appeared to be mere abbreviations and which derived their
ideological color not from their meaning but from their structure.

\switchcolumn

B类词汇中没有哪个词在意识形态上是中立的。很多词都属委婉语。举个例子,``joycamp''(乐趣营)实际上是强制性劳动营。``Minipax
Ministry of
Peace''(和平部)实际是战争部,词的含义与字面意思几乎恰好相反。而另一方面,一些词又直白地表现出对大洋国真实情况的蔑视。比如``prolefeed''便是指党提供给群众的低级娱乐和虚假新闻。还有一些词同时包含着相互矛盾的含义,用在党的身上就是``好'',用在敌人身上就是``坏''。此外很多词,乍看无非是缩写,但其意识形态不来自于它的含义,而恰恰来自于它的结构。

\switchcolumn*

So far as it could be contrived, everything that had or might have
political significance of any kind was fitted into the B vocabulary. The
name of every organization, or body of people, or doctrine, or country,
or institution, or public building, was invariably cut down into the
familiar shape; that is, a single easily pronounced word with the
smallest number of syllables that would preserve the original
derivation. In the Ministry of Truth, for example, the Records
Department, in which Winston Smith worked, was called \emph{Recdep}, the
Fiction Department was called \emph{Ficdep}, the Teleprograms Department
was called \emph{Teledep}, and so on. This was not done solely with the
object of saving time. Even in the early decades of the twentieth
century, telescoped words and phrases had been one of the characteristic
features of political language; and it had been noticed that the
tendency to use abbreviations of this kind was most marked in
totalitarian countries and totalitarian organizations. Examples were
such words as \emph{Nazi}, \emph{Gestapo}, \emph{Comintern},
\emph{Inprecorr}, \emph{Agitprop}. In the beginning the practice had
been adopted as it were instinctively, but in Newspeak it was used with
a conscious purpose. It was perceived that in thus abbreviating a name
one narrowed and subtly altered its meaning, by cutting out most of the
associations that would otherwise cling to it. The words \emph{Communist
International}, for instance, call up a composite picture of universal
human brotherhood, red flags, barricades, Karl Marx, and the Paris
Commune. The word Comintern, on the other hand, suggests merely a
tightly knit organization and a well-defined body of doctrine. It refers
to something almost as easily recognized, and as limited in purpose, as
a chair or a table. \emph{Comintern} is a word that can be uttered
almost without taking thought, whereas \emph{Communist International} is
a phrase over which one is obliged to linger at least momentarily. In
the same way, the associations called up by a word like \emph{Minitrue}
are fewer and more controllable than those called up by \emph{Ministry
of Truth}. This accounted not only for the habit of abbreviating
whenever possible, but also for the almost exaggerated care that was
taken to make every word easily pronounceable.

\switchcolumn

只要得到精心的安排,所有有政治意义或可能拥有政治意义的词都适合纳入B类词汇中。一切组织、团体、学说、国家、机构、公共建筑的名字都毫无例外地被缩减成为人熟知的词形:一个容易发音,音节最少且保持原有词源的单词。比如在真理部,温斯顿•史密斯的记录司(the
Record Department)被称作``Recdep'';小说司(the Fiction
Department)被称作``Ficdep'';电屏节目司(the Teleprogrammes
Department)被称作``Teledep''。这样做并非简单地为了节省时间。事实上早在20世纪初,缩略语就已然成为政治语言的特点之一。人们注意到极权国家和极权组织,明显倾向使用这种缩略语。比如:``Nazi''(纳粹)、``Gestapo''(盖世太保)、``Comintern''(共产国际)、``Agitprop''(宣传鼓动)。一开始这只是本能的做法,但在新话中,它却是有意为之了。人们发现,通过将某个名字进行缩略,可以减少其中大部分惹人联想的意思,将其含义狭隘化,使其发生微妙的变化。比如``Communist
International''(共产主义国际组织)会让人想象一幅由亲如兄弟的人类、红旗、街垒、卡尔•马克思和巴黎公社共同构成的画面。但``Comintern''就只意味着一个严密的组织和一个明确完备的主义。它所涉及的就像桌椅一样容易区分、目的有限。``Comintern''一词几乎可以脱口而出。而Communist
International多少要想想才能说出口。同样的,``Minitrue''产生的联想不仅要比``Ministry
of Truth''少,也比``Ministry of
Truth''更容易控制。这既可以解释只要有机会便使用缩略语的习惯因何而来,还可以解释为什么人们要想尽办法让词更容易发音。

\switchcolumn*

In Newspeak, euphony outweighed every consideration other than
exactitude of meaning. Regularity of grammar was always sacrificed to it
when it seemed necessary. And rightly so, since what was required, above
all for political purposes, were short clipped words of unmistakable
meaning which could be uttered rapidly and which roused the minimum of
echoes in the speaker\textquotesingle s mind. The words of the B
vocabulary even gained in force from the fact that nearly all of them
were very much alike. Almost invariably these words---\emph{goodthink},
\emph{Minipax}, \emph{prolefeed}, \emph{sexcrime}, \emph{joycamp},
\emph{Ingsoc}, \emph{bellyfeel}, \emph{thinkpol}, and countless
others---were words of two or three syllables, with the stress
distributed equally between the first syllable and the last. The use of
them encouraged a gabbling style of speech, at once staccato and
monotonous. And this was exactly what was aimed at. The intention was to
make speech, and especially speech on any subject not ideologically
neutral, as nearly as possible independent of consciousness. For the
purposes of everyday life it was no doubt necessary, or sometimes
necessary, to reflect before speaking, but a Party member called upon to
make a political or ethical judgment should be able to spray forth the
correct opinions as automatically as a machine gun spraying forth
bullets. His training fitted him to do this, the language gave him an
almost foolproof instrument, and the texture of the words, with their
harsh sound and a certain willful ugliness which was in accord with the
spirit of Ingsoc, assisted the process still further.

\switchcolumn

新话中,除了词义精确外,发音悦耳也被充分考虑。若有需要,语法规则也可为此牺牲。这是理所当然的,也是必须的,出于政治上的考虑,干脆简短、语义明确的单词能够很快说出来,带给人的心灵上的回响也最为有限。事实上,B类词汇甚至因为彼此类似,而更有力量。它们常常只有两三个音节,且重音均匀地落在第一个和最后一个音节上。如``goodthink''、``Minipax''、``prolefeed''、``sexcrime''、``joycamp''、``Ingsoc''、``bellyfeel''、``thinkpol''。使用这些词汇可以让说话的腔调变得急促而含糊,断断续续又单调乏味,这正是它的目的所在,它能让讲话——特别是面对意识形态并非中立的话题时——尽可能地脱离意识。在日常生活里,人们无疑需要,或者说有时候需要在说话前想一想。但党员在发表政治或道德上的看法时,就被要求像机关枪发射子弹一样,不假思索地迸发出正确的见解。他所接受的训练让他适应这一要求,新话这工具又让他在这方面几乎万无一失。由于这些词的结构、刺耳的读音以及某些丑陋不堪的地方都符合了英社的精神,凭借着它们,他甚至可以做得更好。

\switchcolumn*

So did the fact of having very few words to choose from. Relative to our
own, the Newspeak vocabulary was tiny, and new ways of reducing it were
constantly being devised. Newspeak, indeed, differed from almost all
other languages in that its vocabulary grew smaller instead of larger
every year. Each reduction was a gain, since the smaller the area of
choice, the smaller the temptation to take thought. Ultimately it was
hoped to make articulate speech issue from the larynx without involving
the higher brain centers at all. This aim was frankly admitted in the
Newspeak word \emph{duckspeak}, meaning ``to quack like a duck.'' Like
various other words in the B vocabulary, \emph{duckspeak} was ambivalent
in meaning. Provided that the opinions which were quacked out were
orthodox ones, it implied nothing but praise, and when the \emph{Times}
referred to one of the orators of the Party as a \emph{doubleplusgood
duckspeaker} it was paying a warm and valued compliment.

\switchcolumn

事实上,人们能够选择的单词的范围非常小。和如今人们所使用的语言相比,新话的词汇量非常少,且时常便会出现减少词汇量的新方法。新话,的确,它和其他语言的区别就在于它的词汇量是逐年减少而不是逐年增多。每减少一点,就多一点收获,因为可供选择的词的范围变小,思考对人的诱惑也跟着变小。最后,它希望无须使用较高级的大脑中枢,就可以清楚地从喉咙里发出声音。新话的``duckspeak''一词就坦率地承认了这点,它的意思是``像鸭子一样叫''。和B类词汇中的绝大多数单词一样,``duckspeak''既可以是贬义的,也可以是褒义的。若人所说的是正统观点,那么,对此它除了赞美就没有别的什么含义了。当《泰晤士报》说某个演说家是``doubleplusegood
duckspeaker''时,它即是在对该演说家表示热情而珍贵的赞赏。

\switchcolumn*

\sectionbreak

\switchcolumn

\sectionbreak

\switchcolumn*

\emph{The C vocabulary.} The C vocabulary was supplementary to the
others and consisted entirely of scientific and technical terms. These
resembled the scientific terms in use today, and were constructed from
the same roots, but the usual care was taken to define them rigidly and
strip them of undesirable meanings. They followed the same grammatical
rules as the words in the other two vocabularies. Very few of the C
words had any currency either in everyday speech or in political speech.
Any scientific worker or technician could find all the words he needed
in the list devoted to his own speciality, but he seldom had more than a
smattering of the words occurring in the other lists. Only a very few
words were common to all lists, and there was no vocabulary expressing
the function of Science as a habit of mind, or a method of thought,
irrespective of its particular branches. There was, indeed, no word for
``Science,'' any meaning that it could possibly bear being already
sufficiently covered by the word \emph{Ingsoc}.

\switchcolumn

\textbf{C类词汇。} C类词汇是对其他两类词汇的补充,完全由科技术语组成,和今天我们使用的科技术语相似,由相同的词根构成,但通常都被阉割定义,且剔除掉了不合适的含义。在语法规则方面,C类词汇和其他两类词汇相同。日常交谈或政治演说都极少会用到C类词汇。而所有科学工作者或技术人员都能在与其专业相对应的专用词汇表中找到所有他需要用到的词。至于其他单词表上的词汇,他就知之甚少了。只有极少数单词是所有词汇表所共有的,也没有哪个词汇拥有将科学作为思想习惯或思想方法进行表述的功能,对此,任何科学分支都一样。事实上根本没有``Scuebce''(科学)这个单词,因为``Ingsoc''已经将它所有可能拥有的含义都包括在内了。

\switchcolumn*

\sectionbreak

\switchcolumn

\sectionbreak

\switchcolumn*

From the foregoing account it will be seen that in Newspeak the
expression of unorthodox opinions, above a very low level, was well-nigh
impossible. It was of course possible to utter heresies of a very crude
kind, a species of blasphemy. It would have been possible, for example,
to say \emph{Big Brother is ungood}. But this statement, which to an
orthodox ear merely conveyed a self-evident absurdity, could not have
been sustained by reasoned argument, because the necessary words were
not available. Ideas inimical to Ingsoc could only be entertained in a
vague wordless form, and could only be named in very broad terms which
lumped together and condemned whole groups of heresies without defining
them in doing so. One could, in fact, only use Newspeak for unorthodox
purposes by illegitimately translating some of the words back into
Oldspeak. For example, \emph{All mans are equal} was a possible Newspeak
sentence, but only in the same sense in which \emph{All men are
redhaired} is a possible Oldspeak sentence. It did not contain a
grammatical error, but it expressed a palpable untruth, i.e., that all
men are of equal size, weight, or strength. The concept of political
equality no longer existed, and this secondary meaning had accordingly
been purged out of the word \emph{equal}. In 1984, when Oldspeak was
still the normal means of communication, the danger theoretically
existed that in using Newspeak words one might remember their original
meanings. In practice it was not difficult for any person well grounded
in \emph{doublethink} to avoid doing this, but within a couple of
generations even the possibility of such a lapse would have vanished. A
person growing up with Newspeak as his sole language would no more know
that \emph{equal} had once had the secondary meaning of ``politically
equal,'' or that \emph{free} had once meant ``intellectually free,'' than,
for instance, a person who had never heard of chess would be aware of
the secondary meanings attaching to \emph{queen} and \emph{rook}. There
would be many crimes and errors which it would be beyond his power to
commit, simply because they were nameless and therefore unimaginable.
And it was to be foreseen that with the passage of time the
distinguishing characteristics of Newspeak would become more and more
pronounced---its words growing fewer and fewer, their meanings more and
more rigid, and the chance of putting them to improper uses always
diminishing.

\switchcolumn

综上所述,可以看出,在新话中,非正统的观点要想得到高级一点的表述则完全不可能。当然,异端邪说很可能以非常粗鲁的方式讲出来。比如``Big
Brother is
ungood''(老大哥不好)。但这在正统的耳朵里听起来仅仅是不言而谕的荒谬,不可能有充分的理由证明它是对的,因为找不到论证所需的单词。不利于英社的观点只能以一种模糊的无法言说的方式存在,且只能用十分笼统的术语为其命名。这些笼统的词汇囊括了各种异端邪说,无须为它们做什么定义,就对它们进行了谴责。实际上,一个人只有将这些词非法地翻译成老话,才能实现``用新话表示非正统的思想''的目的。比如,用新话也可能说出``All
mans are equal''(人人平等)这样的句子,但它却和老话里的``All men are
redhaired''(人人都是红头发)是同类型的句子。它没有语法错误,但它表述的却是一看即知的谎言——即所有人都有着相同的身高、体重或力量。政治平等的概念已不再存在,因为它的次要含义已经从``equal''一词中剔除。在1984年,老话仍是自然而然的交流方式,从理论上说,还存在着这样的危险:人们在使用新话时仍记得它们原有的含义。但站在实践的角度,任何精于双重思想的人都很容易避开这点,而再过一两代,甚至连这方面的失误也将消失。把新话作为唯一语言成长起来的人是不会知道``equal''一词曾经有过``政治平等''的次要含义的,也不会知道``free''有过``思想自由''的次要含义,一如从来没有听说过象棋的人不会了解``后''和``车''的次要含义。很多罪行和错误都超出了人的能力范畴,它们都没有名字,也无法被想象出来。可以预见,随着时间的推移,新话的特征将愈发明显——词汇的数量越来越少,含义越来越严格,将其用于非正常的目的的可能性越来越小。

\switchcolumn*

When Oldspeak had been once and for all superseded, the last link with
the past would have been severed. History had already been rewritten,
but fragments of the literature of the past survived here and there,
imperfectly censored, and so long as one retained one\textquotesingle s
knowledge of Oldspeak it was possible to read them. In the future such
fragments, even if they chanced to survive, would be unintelligible and
untranslatable. It was impossible to translate any passage of Oldspeak
into Newspeak unless it either referred to some technical process or
some very simple everyday action, or was already orthodox
(\emph{goodthinkful} would be the Newspeak expression) in tendency. In
practice this meant that no book written before approximately 1960 could
be translated as a whole. Prerevolutionary literature could only be
subjected to ideological translation---that is, alteration in sense as
well as language. Take for example the well-known passage from the
Declaration of Independence:

\switchcolumn

当老话成为过去并完全被取代后,人们和过去世界的最后一点联系也被切断了。历史被重新书写,但和过去有关的文献的碎片并没有遭到彻底的检查,仍会在这里或那里出现,只要人们还保留着和老话有关的知识,就仍有可能阅读这些碎片。但到了未来,即使这些碎片得以幸存,人们也很难理解它、翻译它,除非它只涉及某项技术进步或非常简单的日常行为,再不就是它已经倾向正统(用新话讲就是``goodthinkful'')。事实上,1960年之前的书没有一本能完完整整地翻译成新话,革命前的文献只能做意识形态上的翻译,也就是说既改变文字又改变意义。如《独立宣言》中众所周知的一段话:

\switchcolumn*

\begin{quotation}
\emph{We hold these truths to be self-evident, that all men are created
equal, that they are endowed by their Creator with certain inalienable
rights, that among these are life, liberty and the pursuit of happiness.
That to secure these rights, Governments are instituted among men,
deriving their powers from the consent of the governed. That whenever
any form of Government becomes destructive of those ends, it is the
right of the People to alter or abolish it, and to institute new
Government ...}
\end{quotation}

\switchcolumn

我们认为以下真理是不言而喻的。人人生而平等,造物主赋予他们若干不可让渡的权利,包括生命权、自由权和追求幸福的权利。为保障这些权利,人们才在他们中间设立了政府,政府的正当权力是经被治理者的同意产生的。任何形式的政府对这些目标起破坏作用,人们就有权利更换或推翻它,以建立一个新的政府\ldots\ldots{}

\switchcolumn*

It would have been quite impossible to render this into Newspeak while
keeping to the sense of the original. The nearest one could come to
doing so would be to swallow the whole passage up in the single word
\emph{crimethink}. A full translation could only be an ideological
translation, whereby Jefferson\textquotesingle s words would be changed
into a panegyric on absolute government.

\switchcolumn

要想用新话翻译出这段话的原意是根本不可能的。最贴合原意的做法就是用一个单词``crimethink''来概括这段话。因此完整的翻译只能是意识形态上的翻译,将杰斐逊的这段话翻译成对拥有绝对权力的政府的赞颂。

\switchcolumn*

A good deal of the literature of the past was, indeed, already being
transformed in this way. Considerations of prestige made it desirable to
preserve the memory of certain historical figures, while at the same
time bringing their achievements into line with the philosophy of
Ingsoc. Various writers, such as Shakespeare, Milton, Swift, Byron,
Dickens, and some others were therefore in process of translation; when
the task had been completed, their original writings, with all else that
survived of the literature of the past, would be destroyed. These
translations were a slow and difficult business, and it was not expected
that they would be finished before the first or second decade of the
twenty-first century. There were also large quantities of merely
utilitarian literature---indispensable technical manuals and the
like---that had to be treated in the same way. It was chiefly in order
to allow time for the preliminary work of translation that the final
adoption of Newspeak had been fixed for so late a date as 2050.


\switchcolumn

没错,大量过去的文献都用这种方法改写过。考虑到自己的声望,有必要保留关于某些历史人物的记忆,并让他们的成就和英社的哲学相呼应。因此诸如莎士比亚、弥尔顿、斯威夫特、拜伦、狄更斯以及其他一些作家的作品都正在被翻译。什么时候翻译完了,什么时候他们原先的作品还有所有残留下来的过去的文献,都将被毁灭。这些翻译进展得非常慢,还很艰难,21世纪的前一二十年大概完成不了。此外,还有为数不少的实用性的文献——不可或缺的技术手册之类——也要如此处理。正是考虑到要留出时间进行``翻译此前文献''的工作,新话的最终采用时间才被推迟到2050年。


\switchcolumn*

\end{paracol}