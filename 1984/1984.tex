\clearpage
\part{ONE}\label{one}

\section{I}

It was a bright cold day in April, and the clocks were striking
thirteen. Winston Smith, his chin nuzzled into his breast in an effort
to escape the vile wind, slipped quickly through the glass doors of
Victory Mansions, though not quickly enough to prevent a swirl of gritty
dust from entering along with him.

The hallway smelt of boiled cabbage and old rag mats. At one end of it a
colored poster, too large for indoor display, had been tacked to the wall.
It depicted simply an enormous face, more than a meter wide: the face of a
man of about forty-five, with a heavy black mustache and ruggedly handsome
features. Winston made for the stairs. It was no use trying the lift. Even
at the best of times it was seldom working, and at present the electric
current was cut off during daylight hours. It was part of the economy drive
in preparation for Hate Week. The flat was seven flights up, and Winston,
who was thirty-nine and had a varicose ulcer above his right ankle, went
slowly, resting several times on the way. On each landing, opposite the lift
shaft, the poster with the enormous face gazed from the wall. It was one of
those pictures which are so contrived that the eyes follow you about when
you move. \textsc{Big Brother Is Watching You}, the caption beneath it ran.

Inside the flat a fruity voice was reading out a list of figures which
had something to do with the production of pig iron. The voice came from
an oblong metal plaque like a dulled mirror which formed part of the
surface of the right-hand wall. Winston turned a switch and the voice
sank somewhat, though the words were still distinguishable. The
instrument (the telescreen, it was called) could be dimmed, but there
was no way of shutting it off completely. He moved over to the window: a
smallish, frail figure, the meagerness of his body merely emphasized by
the blue overalls which were the uniform of the Party. His hair was very
fair, his face naturally sanguine, his skin roughened by coarse soap and
blunt razor blades and the cold of the winter that had just ended.

Outside, even through the shut window pane, the world looked cold. Down
in the street little eddies of wind were whirling dust and torn paper
into spirals, and though the sun was shining and the sky a harsh blue,
there seemed to be no color in anything except the posters that were
plastered everywhere. The black-mustachio\textquotesingle d face gazed
down from every commanding corner. There was one on the house front
immediately opposite. \textsc{Big Brother Is Watching You}, the caption said,
while the dark eyes looked deep into Winston\textquotesingle s own. Down
at street level another poster, torn at one corner, flapped fitfully in
the wind, alternately covering and uncovering the single word INGSOC. In
the far distance a helicopter skimmed down between the roofs, hovered
for an instant like a bluebottle, and darted away again with a curving
flight. It was the Police Patrol, snooping into people\textquotesingle s
windows. The patrols did not matter, however. Only the Thought Police
mattered.

Behind Winston\textquotesingle s back the voice from the telescreen was
still babbling away about pig iron and the overfulfillment of the Ninth
Three-Year Plan. The telescreen received and transmitted simultaneously.
Any sound that Winston made, above the level of a very low whisper,
would be picked up by it; moreover, so long as he remained within the
field of vision which the metal plaque commanded, he could be seen as
well as heard. There was of course no way of knowing whether you were
being watched at any given moment. How often, or on what system, the
Thought Police plugged in on any individual wire was guesswork. It was
even conceivable that they watched everybody all the time. But at any
rate they could plug in your wire whenever they wanted to. You had to
live---did live, from habit that became instinct---in the assumption
that every sound you made was overheard, and, except in darkness, every
movement scrutinized.

Winston kept his back turned to the telescreen. It was safer; though, as
he well knew, even a back can be revealing. A kilometer away the
Ministry of Truth, his place of work, towered vast and white above the
grimy landscape. This, he thought with a sort of vague distaste---this
was London, chief city of Airstrip One, itself the third most populous
of the provinces of Oceania. He tried to squeeze out some childhood
memory that should tell him whether London had always been quite like
this. Were there always these vistas of rotting nineteenth-century
houses, their sides shored up with balks of timber, their windows
patched with cardboard and their roofs with corrugated iron, their crazy
garden walls sagging in all directions? And the bombed sites where the
plaster dust swirled in the air and the willow herb straggled over the
heaps of rubble; and the places where the bombs had cleared a larger
patch and there had sprung up sordid colonies of wooden dwellings like
chicken houses? But it was no use, he could not remember: nothing
remained of his childhood except a series of bright-lit tableaux,
occurring against no background and mostly unintelligible.

The Ministry of Truth---Minitrue, in Newspeak\footnote{Newspeak was the
  official language of Oceania. For an account of its structure and
  etymology, see Appendix.}---was startlingly different from any other
object in sight. It was an enormous pyramidal structure of glittering white
concrete, soaring up, terrace after terrace, three hundred meters into the
air. From where Winston stood it was just possible to read, picked out on
its white face in elegant lettering, the three slogans of the Party:

\headline{WAR IS PEACE\\
FREEDOM IS SLAVERY\\
IGNORANCE IS STRENGTH.}

The Ministry of Truth contained, it was said, three thousand rooms above
ground level, and corresponding ramifications below. Scattered about
London there were just three other buildings of similar appearance and
size. So completely did they dwarf the surrounding architecture that
from the roof of Victory Mansions you could see all four of them
simultaneously. They were the homes of the four Ministries between which
the entire apparatus of government was divided: the Ministry of Truth,
which concerned itself with news, entertainment, education, and the fine
arts; the Ministry of Peace, which concerned itself with war; the
Ministry of Love, which maintained law and order; and the Ministry of
Plenty, which was responsible for economic affairs. Their names, in
Newspeak: Minitrue, Minipax, Miniluv, and Miniplenty.

The Ministry of Love was the really frightening one. There were no
windows in it at all. Winston had never been inside the Ministry of
Love, nor within half a kilometer of it. It was a place impossible to
enter except on official business, and then only by penetrating through
a maze of barbed-wire entanglements, steel doors, and hidden machine-gun
nests. Even the streets leading up to its outer barriers were roamed by
gorilla-faced guards in black uniforms, armed with jointed truncheons.

Winston turned round abruptly. He had set his features into the
expression of quiet optimism which it was advisable to wear when facing
the telescreen. He crossed the room into the tiny kitchen. By leaving
the Ministry at this time of day he had sacrificed his lunch in the
canteen, and he was aware that there was no food in the kitchen except a
hunk of dark-colored bread which had got to be saved for
tomorrow\textquotesingle s breakfast. He took down from the shelf a
bottle of colorless liquid with a plain white label marked \textsc{VICTORY
GIN}. It gave off a sickly, oily smell, as of Chinese rice-spirit.
Winston poured out nearly a teacupful, nerved himself for a shock, and
gulped it down like a dose of medicine.

Instantly his face turned scarlet and the water ran out of his eyes. The
stuff was like nitric acid, and moreover, in swallowing it one had the
sensation of being hit on the back of the head with a rubber club. The
next moment, however, the burning in his belly died down and the world
began to look more cheerful. He took a cigarette from a crumpled packet
marked \textsc{VICTORY CIGARETTES} and incautiously held it upright, whereupon
the tobacco fell out onto the floor. With the next he was more
successful. He went back to the living room and sat down at a small
table that stood to the left of the telescreen. From the table drawer he
took out a penholder, a bottle of ink, and a thick, quarto-sized blank
book with a red back and a marbled cover.

For some reason the telescreen in the living room was in an unusual
position. Instead of being placed, as was normal, in the end wall, where
it could command the whole room, it was in the longer wall, opposite the
window. To one side of it there was a shallow alcove in which Winston
was now sitting, and which, when the flats were built, had probably been
intended to hold bookshelves. By sitting in the alcove, and keeping well
back, Winston was able to remain outside the range of the telescreen, so
far as sight went. He could be heard, of course, but so long as he
stayed in his present position he could not be seen. It was partly the
unusual geography of the room that had suggested to him the thing that
he was now about to do.

But it had also been suggested by the book that he had just taken out of
the drawer. It was a peculiarly beautiful book. Its smooth creamy paper,
a little yellowed by age, was of a kind that had not been manufactured
for at least forty years past. He could guess, however, that the book
was much older than that. He had seen it lying in the window of a frowsy
little junk shop in a slummy quarter of the town (just what quarter he
did not now remember) and had been stricken immediately by an
overwhelming desire to possess it. Party members were supposed not to go
into ordinary shops (``dealing on the free market,'' it was called), but
the rule was not strictly kept, because there were various things such
as shoelaces and razor blades which it was impossible to get hold of in
any other way. He had given a quick glance up and down the street and
then had slipped inside and bought the book for two dollars fifty. At
the time he was not conscious of wanting it for any particular purpose.
He had carried it guiltily home in his brief case. Even with nothing
written in it, it was a compromising possession.

The thing that he was about to do was to open a diary. This was not
illegal (nothing was illegal, since there were no longer any laws), but
if detected it was reasonably certain that it would be punished by
death, or at least by twenty-five years in a forced-labor camp. Winston
fitted a nib into the penholder and sucked it to get the grease off. The
pen was an archaic instrument, seldom used even for signatures, and he
had procured one, furtively and with some difficulty, simply because of
a feeling that the beautiful creamy paper deserved to be written on with
a real nib instead of being scratched with an ink pencil. Actually he
was not used to writing by hand. Apart from very short notes, it was
usual to dictate everything into the speakwrite, which was of course
impossible for his present purpose. He dipped the pen into the ink and
then faltered for just a second. A tremor had gone through his bowels.
To mark the paper was the decisive act. In small clumsy letters he
wrote:


\begin{quotation}
  April 4th, 1984.
\end{quotation}

He sat back. A sense of complete helplessness had descended upon him. To
begin with, he did not know with any certainty that this \emph{was} 1984. It
must be round about that date, since he was fairly sure that his age was
thirty-nine, and he believed that he had been born in 1944 or 1945; but it
was never possible nowadays to pin down any date within a year or two.

For whom, it suddenly occurred to him to wonder, was he writing this
diary? For the future, for the unborn. His mind hovered for a moment
round the doubtful date on the page, and then fetched up with a bump
against the Newspeak word \emph{doublethink}. For the first time the
magnitude of what he had undertaken came home to him. How could you
communicate with the future? It was of its nature impossible. Either the
future would resemble the present, in which case it would not listen to
him, or it would be different from it, and his predicament would be
meaningless.

For some time he sat gazing stupidly at the paper. The telescreen had
changed over to strident military music. It was curious that he seemed
not merely to have lost the power of expressing himself, but even to
have forgotten what it was that he had originally intended to say. For
weeks past he had been making ready for this moment, and it had never
crossed his mind that anything would be needed except courage. The
actual writing would be easy. All he had to do was to transfer to paper
the interminable restless monologue that had been running inside his
head, literally for years. At this moment, however, even the monologue
had dried up. Moreover, his varicose ulcer had begun itching unbearably.
He dared not scratch it, because if he did so it always became inflamed.
The seconds were ticking by. He was conscious of nothing except the
blankness of the page in front of him, the itching of the skin above his
ankle, the blaring of the music, and a slight booziness caused by the
gin.

Suddenly he began writing in sheer panic, only imperfectly aware of what
he was setting down. His small but childish handwriting straggled up and
down the page, shedding first its capital letters and finally even its
full stops:

\begin{quotation}
April 4th, 1984. Last night to the flicks. All war films. One very
good one of a ship full of refugees being bombed somewhere in the
Mediterranean. Audience much amused by shots of a great huge fat man
trying to swim away with a helicopter after him. first you saw him
wallowing along in the water like a porpoise, then you saw him through
the helicopters gunsights, then he was full of holes and the sea round
him turned pink and he sank as suddenly as though the holes had let in
the water. audience shouting with laughter when he sank. then you saw a
lifeboat full of children with a helicopter hovering over it. there was
a middle-aged woman might have been a jewess sitting up in the bow with
a little boy about three years old in her arms. little boy screaming
with fright and hiding his head between her breasts as if he was trying
to burrow right into her and the woman putting her arms round him and
comforting him although she was blue with fright herself. all the time
covering him up as much as possible as if she thought her arms could
keep the bullets off him. then the helicopter planted a 20 kilo bomb in
among them terrific flash and the boat went all to matchwood. then there
was a wonderful shot of a childs arm going up up up right up into the
air a helicopter with a camera in its nose must have followed it up and
there was a lot of applause from the party seats but a woman down in the
prole part of the house suddenly started kicking up a fuss and shouting
they didnt oughter of showed it not in front of kids they didnt it aint
right not in front of kids it aint until the police turned her turned
her out i dont suppose anything happened to her nobody cares what the
proles say typical prole reaction they never---
\end{quotation}

Winston stopped writing, partly because he was suffering from cramp. He
did not know what had made him pour out this stream of rubbish. But the
curious thing was that while he was doing so a totally different memory
had clarified itself in his mind, to the point where he almost felt
equal to writing it down. It was, he now realized, because of this other
incident that he had suddenly decided to come home and begin the diary
today.

It had happened that morning at the Ministry, if anything so nebulous
could be said to happen.

It was nearly eleven hundred, and in the Records Department, where
Winston worked, they were dragging the chairs out of the cubicles and
grouping them in the center of the hall, opposite the big telescreen, in
preparation for the Two Minutes Hate. Winston was just taking his place
in one of the middle rows when two people whom he knew by sight, but had
never spoken to, came unexpectedly into the room. One of them was a girl
whom he often passed in the corridors. He did not know her name, but he
knew that she worked in the Fiction Department. Presumably---since he
had sometimes seen her with oily hands and carrying a spanner---she had
some mechanical job on one of the novel-writing machines. She was a
bold-looking girl of about twenty-seven, with thick dark hair, a
freckled face, and swift, athletic movements. A narrow scarlet sash,
emblem of the Junior Anti-Sex League, was wound several times round the
waist of her overalls, just tightly enough to bring out the shapeliness
of her hips. Winston had disliked her from the very first moment of
seeing her. He knew the reason. It was because of the atmosphere of
hockey fields and cold baths and community hikes and general
clean-mindedness which she managed to carry about with her. He disliked
nearly all women, and especially the young and pretty ones. It was
always the women, and above all the young ones, who were the most
bigoted adherents of the Party, the swallowers of slogans, the amateur
spies and nosers-out of unorthodoxy. But this particular girl gave him
the impression of being more dangerous than most. Once when they passed
in the corridor she had given him a quick sidelong glance which seemed
to pierce right into him and for a moment had filled him with black
terror. The idea had even crossed his mind that she might be an agent of
the Thought Police. That, it was true, was very unlikely. Still, he
continued to feel a peculiar uneasiness, which had fear mixed up in it
as well as hostility, whenever she was anywhere near him.

The other person was a man named O\textquotesingle Brien, a member of
the Inner Party and holder of some post so important and remote that
Winston had only a dim idea of its nature. A momentary hush passed over
the group of people round the chairs as they saw the black overalls of
an Inner Party member approaching. O\textquotesingle Brien was a large,
burly man with a thick neck and a coarse, humorous, brutal face. In
spite of his formidable appearance he had a certain charm of manner. He
had a trick of resettling his spectacles on his nose which was curiously
disarming---in some indefinable way, curiously civilized. It was a
gesture which, if anyone had still thought in such terms, might have
recalled an eighteenth-century nobleman offering his snuffbox. Winston
had seen O\textquotesingle Brien perhaps a dozen times in almost as many
years. He felt deeply drawn to him, and not solely because he was
intrigued by the contrast between
O\textquotesingle Brien\textquotesingle s urbane manner and his
prizefighter\textquotesingle s physique. Much more it was because of a
secretly held belief---or perhaps not even a belief, merely a
hope---that O\textquotesingle Brien\textquotesingle s political
orthodoxy was not perfect. Something in his face suggested it
irresistibly. And again, perhaps it was not even unorthodoxy that was
written in his face, but simply intelligence. But at any rate he had the
appearance of being a person that you could talk to, if somehow you
could cheat the telescreen and get him alone. Winston had never made the
smallest effort to verify this guess; indeed, there was no way of doing
so. At this moment O\textquotesingle Brien glanced at his wristwatch,
saw that it was nearly eleven hundred, and evidently decided to stay in
the Records Department until the Two Minutes Hate was over. He took a
chair in the same row as Winston, a couple of places away. A small,
sandy-haired woman who worked in the next cubicle to Winston was between
them. The girl with dark hair was sitting immediately behind.

The next moment a hideous, grinding screech, as of some monstrous
machine running without oil, burst from the big telescreen at the end of
the room. It was a noise that set one\textquotesingle s teeth on edge
and bristled the hair at the back of one\textquotesingle s neck. The
Hate had started.

As usual, the face of Emmanuel Goldstein, the Enemy of the People, had
flashed onto the screen. There were hisses here and there among the
audience. The little sandy-haired woman gave a squeak of mingled fear
and disgust. Goldstein was the renegade and backslider who once, long
ago (how long ago, nobody quite remembered), had been one of the leading
figures of the Party, almost on a level with Big Brother himself, and
then had engaged in counter-revolutionary activities, had been condemned
to death, and had mysteriously escaped and disappeared. The program of
the Two Minutes Hate varied from day to day, but there was none in which
Goldstein was not the principal figure. He was the primal traitor, the
earliest defiler of the Party\textquotesingle s purity. All subsequent
crimes against the Party, all treacheries, acts of sabotage, heresies,
deviations, sprang directly out of his teaching. Somewhere or other he
was still alive and hatching his conspiracies: perhaps somewhere beyond
the sea, under the protection of his foreign paymasters; perhaps
even---so it was occasionally rumored---in some hiding place in Oceania
itself.

Winston\textquotesingle s diaphragm was constricted. He could never see
the face of Goldstein without a painful mixture of emotions. It was a
lean Jewish face, with a great fuzzy aureole of white hair and a small
goatee beard---a clever face, and yet somehow inherently despicable,
with a kind of senile silliness in the long thin nose near the end of
which a pair of spectacles was perched. It resembled the face of a
sheep, and the voice, too, had a sheeplike quality. Goldstein was
delivering his usual venomous attack upon the doctrines of the
Party---an attack so exaggerated and perverse that a child should have
been able to see through it, and yet just plausible enough to fill one
with an alarmed feeling that other people, less level-headed than
oneself, might be taken in by it. He was abusing Big Brother, he was
denouncing the dictatorship of the Party, he was demanding the immediate
conclusion of peace with Eurasia, he was advocating freedom of speech,
freedom of the press, freedom of assembly, freedom of thought, he was
crying hysterically that the revolution had been betrayed---and all this
in rapid polysyllabic speech which was a sort of parody of the habitual
style of the orators of the Party, and even contained Newspeak words:
more Newspeak words, indeed, than any Party member would normally use in
real life. And all the while, lest one should be in any doubt as to the
reality which Goldstein\textquotesingle s specious claptrap covered,
behind his head on the telescreen there marched the endless columns of
the Eurasian army---row after row of solid-looking men with
expressionless Asiatic faces, who swam up to the surface of the screen
and vanished, to be replaced by others exactly similar. The dull,
rhythmic tramp of the soldiers\textquotesingle{} boots formed the
background to Goldstein\textquotesingle s bleating voice.

Before the Hate had proceeded for thirty seconds, uncontrollable
exclamations of rage were breaking out from half the people in the room.
The self-satisfied sheeplike face on the screen, and the terrifying
power of the Eurasian army behind it, were too much to be borne;
besides, the sight or even the thought of Goldstein produced fear and
anger automatically. He was an object of hatred more constant than
either Eurasia or Eastasia, since when Oceania was at war with one of
these powers it was generally at peace with the other. But what was
strange was that although Goldstein was hated and despised by everybody,
although every day, and a thousand times a day, on platforms, on the
telescreen, in newspapers, in books, his theories were refuted, smashed,
ridiculed, held up to the general gaze for the pitiful rubbish that they
were---in spite of all this, his influence never seemed to grow less.
Always there were fresh dupes waiting to be seduced by him. A day never
passed when spies and saboteurs acting under his directions were not
unmasked by the Thought Police. He was the commander of a vast shadowy
army, an underground network of conspirators dedicated to the overthrow
of the State. The Brotherhood, its name was supposed to be. There were
also whispered stories of a terrible book, a compendium of all the
heresies, of which Goldstein was the author and which circulated
clandestinely here and there. It was a book without a title. People
referred to it, if at all, simply as \emph{the book}. But one knew of
such things only through vague rumors. Neither the Brotherhood nor
\emph{the book} was a subject that any ordinary Party member would
mention if there was a way of avoiding it.

In its second minute the Hate rose to a frenzy. People were leaping up
and down in their places and shouting at the tops of their voices in an
effort to drown the maddening bleating voice that came from the screen.
The little sandy-haired woman had toned bright pink, and her mouth was
opening and shutting like that of a landed fish. Even
O\textquotesingle Brien\textquotesingle s heavy face was flushed. He was
sitting very straight in his chair, his powerful chest swelling and
quivering as though he were standing up to the assault of a wave. The
dark-haired girl behind Winston had begun crying out ``Swine! Swine!
Swine!'' and suddenly she picked up a heavy Newspeak dictionary and flung
it at the screen. It struck Goldstein\textquotesingle s nose and bounced
off; the voice continued inexorably. In a lucid moment Winston found
that he was shouting with the others and kicking his heel violently
against the rung of his chair. The horrible thing about the Two Minutes
Hate was not that one was obliged to act a part, but that it was
impossible to avoid joining in. Within thirty seconds any pretense was
always unnecessary. A hideous ecstasy of fear and vindictiveness, a
desire to kill, to torture, to smash faces in with a sledge hammer,
seemed to flow through the whole group of people like an electric
current, turning one even against one\textquotesingle s will into a
grimacing, screaming lunatic. And yet the rage that one felt was an
abstract, undirected emotion which could be switched from one object to
another like the flame of a blowlamp. Thus, at one moment
Winston\textquotesingle s hatred was not turned against Goldstein at
all, but, on the contrary, against Big Brother, the Party, and the
Thought Police; and at such moments his heart went out to the lonely,
derided heretic on the screen, sole guardian of truth and sanity in a
world of lies. And yet the very next instant he was at one with the
people about him, and all that was said of Goldstein seemed to him to be
true. At those moments his secret loathing of Big Brother changed into
adoration, and Big Brother seemed to tower up, an invincible, fearless
protector, standing like a rock against the hordes of Asia, and
Goldstein, in spite of his isolation, his helplessness, and the doubt
that hung about his very existence, seemed like some sinister enchanter,
capable by the mere power of his voice of wrecking the structure of
civilization.

It was even possible, at moments, to switch one\textquotesingle s hatred
this way or that by a voluntary act. Suddenly, by the sort of violent
effort with which one wrenches one\textquotesingle s head away from the
pillow in a nightmare, Winston succeeded in transferring his hatred from
the face on the screen to the dark-haired girl behind him. Vivid,
beautiful hallucinations flashed through his mind. He would flog her to
death with a rubber truncheon. He would tie her naked to a stake and
shoot her full of arrows like Saint Sebastian. He would ravish her and
cut her throat at the moment of climax. Better than before, moreover, he
realized \emph{why} it was that he hated her. He hated her because she
was young and pretty and sexless, because he wanted to go to bed with
her and would never do so, because round her sweet supple waist, which
seemed to ask you to encircle it with your arm, there was only the
odious scarlet sash, aggressive symbol of chastity.

The Hate rose to its climax. The voice of Goldstein had become an actual
sheep\textquotesingle s bleat, and for an instant the face changed into
that of a sheep. Then the sheep-face melted into the figure of a
Eurasian soldier who seemed to be advancing, huge and terrible, his
submachine gun roaring, and seeming to spring out of the surface of the
screen, so that some of the people in the front row actually flinched
backwards in their seats. But in the same moment, drawing a deep sigh of
relief from everybody, the hostile figure melted into the face of Big
Brother, black-haired, black mustachio\textquotesingle d, full of power
and mysterious calm, and so vast that it almost filled up the screen.
Nobody heard what Big Brother was saying. It was merely a few words of
encouragement, the sort of words that are uttered in the din of battle,
not distinguishable individually but restoring confidence by the fact of
being spoken. Then the face of Big Brother faded away again, and instead
the three slogans of the Party stood out in bold capitals:

\headline{WAR IS PEACE\\
FREEDOM IS SLAVERY\\
IGNORANCE IS STRENGTH}

But the face of Big Brother seemed to persist for several seconds on the
screen, as though the impact that it had made on
everyone\textquotesingle s eyeballs were too vivid to wear off
immediately. The little sandy-haired woman had flung herself forward
over the back of the chair in front of her. With a tremulous murmur that
sounded like ``My Savior!'' she extended her arms toward the screen. Then
she buried her face in her hands. It was apparent that she was uttering
a prayer.

At this moment the entire group of people broke into a deep, slow,
rhythmical chant of ``B-B!... B-B!... B-B!'' over and over again, very
slowly, with a long pause between the first ``B'' and the second---a
heavy, murmurous sound, somehow curiously savage, in the background of
which one seemed to hear the stamp of naked feet and the throbbing of
tom-toms. For perhaps as much as thirty seconds they kept it up. It was
a refrain that was often heard in moments of overwhelming emotion.
Partly it was a sort of hymn to the wisdom and majesty of Big Brother,
but still more it was an act of self-hypnosis, a deliberate drowning of
consciousness by means of rhythmic noise. Winston\textquotesingle s
entrails seemed to grow cold. In the Two Minutes Hate he could not help
sharing in the general delirium, but this subhuman chanting of ``B-B!...
B-B!'' always filled him with horror. Of course he chanted with the rest:
it was impossible to do otherwise. To dissemble your feelings, to
control your face, to do what everyone else was doing, was an
instinctive reaction. But there was a space of a couple of seconds
during which the expression in his eyes might conceivably have betrayed
him. And it was exactly at this moment that the significant thing
happened---if, indeed, it did happen.

Momentarily he caught O\textquotesingle Brien\textquotesingle s eye.
O\textquotesingle Brien had stood up. He had taken off his spectacles
and was in the act of resettling them on his nose with his
characteristic gesture. But there was a fraction of a second when their
eyes met, and for as long as it took to happen Winston knew---yes, he
\emph{knew}!---that O\textquotesingle Brien was thinking the same thing
as himself. An unmistakable message had passed. It was as though their
two minds had opened and the thoughts were flowing from one into the
other through their eyes. ``I am with you,'' O\textquotesingle Brien
seemed to be saying to him. ``I know precisely what you are feeling. I
know all about your contempt, your hatred, your disgust. But
don\textquotesingle t worry, I am on your side!'' And then the flash of
intelligence was gone, and O\textquotesingle Brien\textquotesingle s
face was as inscrutable as everybody else\textquotesingle s.

That was all, and he was already uncertain whether it had happened. Such
incidents never had any sequel. All that they did was to keep alive in
him the belief, or hope, that others besides himself were the enemies of
the Party. Perhaps the rumors of vast underground conspiracies were true
after all---perhaps the Brotherhood really existed! It was impossible,
in spite of the endless arrests and confessions and executions, to be
sure that the Brotherhood was not simply a myth. Some days he believed
in it, some days not. There was no evidence, only fleeting glimpses that
might mean anything or nothing: snatches of overheard conversation,
faint scribbles on lavatory walls---once, even, when two strangers met,
a small movement of the hands which had looked as though it might be a
signal of recognition. It was all guesswork: very likely he had imagined
everything. He had gone back to his cubicle without looking at
O\textquotesingle Brien again. The idea of following up their momentary
contact hardly crossed his mind. It would have been inconceivably
dangerous even if he had known how to set about doing it. For a second,
two seconds, they had exchanged an equivocal glance, and that was the
end of the story. But even that was a memorable event, in the locked
loneliness in which one had to live.

Winston roused himself and sat up straighter. He let out a belch. The
gin was rising from his stomach.

His eyes refocused on the page. He discovered that while he sat
helplessly musing he had also been writing, as though by automatic
action. And it was no longer the same cramped awkward handwriting as
before. His pen had slid voluptuously over the smooth paper, printing in
large neat capitals---

\headline{DOWN WITH BIG BROTHER\\
DOWN WITH BIG BROTHER\\
DOWN WITH BIG BROTHER\\
DOWN WITH BIG BROTHER\\
DOWN WITH BIG BROTHER
}

over and over again, filling half a page.

He could not help feeling a twinge of panic. It was absurd, since the
writing of those particular words was not more dangerous than the
initial act of opening the diary; but for a moment he was tempted to
tear out the spoiled pages and abandon the enterprise altogether.

He did not do so, however, because he knew that it was useless. Whether
he wrote DOWN WITH BIG BROTHER, or whether he refrained from writing it,
made no difference. Whether he went on with the diary, or whether he did
not go on with it, made no difference. The Thought Police would get him
just the same. He had committed---would still have committed, even if he
had never set pen to paper---the essential crime that contained all
others in itself. Thoughtcrime, they called it. Thoughtcrime was not a
thing that could be concealed forever. You might dodge successfully for
a while, even for years, but sooner or later they were bound to get you.

It was always at night---the arrests invariably happened at night. The
sudden jerk out of sleep, the rough hand shaking your shoulder, the
lights glaring in your eyes, the ring of hard faces round the bed. In
the vast majority of cases there was no trial, no report of the arrest.
People simply disappeared, always during the night. Your name was
removed from the registers, every record of everything you had ever done
was wiped out, your one-time existence was denied and then forgotten.
You were abolished, annihilated: \emph{vaporized} was the usual word.

For a moment he was seized by a kind of hysteria. He began writing in a
hurried untidy scrawl:

\begin{quotation}
theyll shoot me i dont care theyll shoot me in the back of the
neck i dont care down with big brother they always shoot you in the back
of the neck i dont care down with big brother---
\end{quotation}

He sat back in his chair, slightly ashamed of himself, and laid down the
pen. The next moment he started violently. There was a knocking at the
door.

Already! He sat as still as a mouse, in the futile hope that whoever it
was might go away after a single attempt. But no, the knocking was
repeated. The worst thing of all would be to delay. His heart was
thumping like a drum, but his face, from long habit, was probably
expressionless. He got up and moved heavily toward the door.


\section{II}\label{ii}

As he put his hand to the doorknob Winston saw that he had left the
diary open on the table. DOWN WITH BIG BROTHER was written all over it,
in letters almost big enough to be legible across the room. It was an
inconceivably stupid thing to have done. But, he realized, even in his
panic he had not wanted to smudge the creamy paper by shutting the book
while the ink was wet.

He drew in his breath and opened the door. Instantly a warm wave of
relief flowed through him. A colorless, crushed-looking woman, with
wispy hair and a lined face, was standing outside.

``Oh, comrade,'' she began in a dreary, whining sort of voice, ``I thought
I heard you come in. Do you think you could come across and have a look
at our kitchen sink? It\textquotesingle s got blocked up and---''

It was Mrs. Parsons, the wife of a neighbor on the same floor. (``Mrs.''
was a word somewhat discountenanced by the Party---you were supposed to
call everyone ``comrade''---but with some women one used it
instinctively.) She was a woman of about thirty, but looking much older.
One had the impression that there was dust in the creases of her face.
Winston followed her down the passage. These amateur repair jobs were an
almost daily irritation. Victory Mansions were old flats, built in 1930
or thereabouts, and were falling to pieces. The plaster flaked
constantly from ceilings and walls, the pipes burst in every hard frost,
the roof leaked whenever there was snow, the heating system was usually
running at half steam when it was not closed down altogether from
motives of economy. Repairs, except what you could do for yourself, had
to be sanctioned by remote committees which were liable to hold up even
the mending of a window pane for two years.

``Of course it\textquotesingle s only because Tom isn\textquotesingle t
home,'' said Mrs. Parsons vaguely.

The Parsons\textquotesingle s flat was bigger than
Winston\textquotesingle s, and dingy in a different way. Everything had
a battered, trampled-on look, as though the place had just been visited
by some large violent animal. Games impedimenta---hockey sticks, boxing
gloves, a burst football, a pair of sweaty shorts turned inside
out---lay all over the floor, and on the table there was a litter of
dirty dishes and dog-eared exercise books. On the walls were scarlet
banners of the Youth League and the Spies, and a full-sized poster of
Big Brother. There was the usual boiled-cabbage smell, common to the
whole building, but it was shot through by a sharper reek of sweat,
which---one knew this at the first sniff, though it was hard to say
how---was the sweat of some person not present at the moment. In another
room someone with a comb and a piece of toilet paper was trying to keep
tune with the military music which was still issuing from the
telescreen.

``It\textquotesingle s the children,'' said Mrs. Parsons, casting a
half-apprehensive glance at the door. ``They haven\textquotesingle t been
out today. And of course---''

She had a habit of breaking off her sentences in the middle. The kitchen
sink was full nearly to the brim with filthy greenish water which smelt
worse than ever of cabbage. Winston knelt down and examined the
angle-joint of the pipe. He hated using his hands, and he hated bending
down, which was always liable to start him coughing. Mrs. Parsons looked
on helplessly.

``Of course if Tom was home he\textquotesingle d put it right in a
moment,'' she said. ``He loves anything like that. He\textquotesingle s
ever so good with his hands, Tom is.''

Parsons was Winston\textquotesingle s fellow employee at the Ministry of
Truth. He was a fattish but active man of paralyzing stupidity, a mass
of imbecile enthusiasms---one of those completely unquestioning, devoted
drudges on whom, more even than on the Thought Police, the stability of
the Party depended. At thirty-five he had just been unwillingly evicted
from the Youth League, and before graduating into the Youth League he
had managed to stay on in the Spies for a year beyond the statutory age.
At the Ministry he was employed in some subordinate post for which
intelligence was not required, but on the other hand he was a leading
figure on the Sports Committee and all the other committees engaged in
organizing community hikes, spontaneous demonstrations, savings
campaigns, and voluntary activities generally. He would inform you with
quiet pride, between whiffs of his pipe, that he had put in an
appearance at the Community Center every evening for the past four
years. An overpowering smell of sweat, a sort of unconscious testimony
to the strenuousness of his life, followed him about wherever he went,
and even remained behind him after he had gone.

``Have you got a spanner?'' said Winston, fiddling with the nut on the
angle-joint.

``A spanner,'' said Mrs. Parsons, immediately becoming invertebrate. ``I
don\textquotesingle t know, I\textquotesingle m sure. Perhaps the
children---''

There was a trampling of boots and another blast on the comb as the
children charged into the living room. Mrs. Parsons brought the spanner.
Winston let out the water and disgustedly removed the clot of human hair
that had blocked up the pipe. He cleaned his fingers as best he could in
the cold water from the tap and went back into the other room.

``Up with your hands!'' yelled a savage voice.

A handsome, tough-looking boy of nine had popped up from behind the
table and was menacing him with a toy automatic pistol, while his small
sister, about two years younger, made the same gesture with a fragment
of wood. Both of them were dressed in the blue shorts, gray shirts, and
red neckerchiefs which were the uniform of the Spies. Winston raised his
hands above his head, but with an uneasy feeling, so vicious was the
boy\textquotesingle s demeanor, that it was not altogether a game.

``You\textquotesingle re a traitor!'' yelled the boy.
``You\textquotesingle re a thought-criminal! You\textquotesingle re a
Eurasian spy! I\textquotesingle ll shoot you, I\textquotesingle ll
vaporize you, I\textquotesingle ll send you to the salt mines!''

Suddenly they were both leaping round him, shouting ``Traitor!'' and
``Thought-criminal!'', the little girl imitating her brother in every
movement. It was somehow slightly frightening, like the gamboling of
tiger cubs which will soon grow up into man-eaters. There was a sort of
calculating ferocity in the boy\textquotesingle s eye, a quite evident
desire to hit or kick Winston and a consciousness of being very nearly
big enough to do so. It was a good job it was not a real pistol he was
holding, Winston thought.

Mrs. Parsons\textquotesingle s eyes flitted nervously from Winston to
the children, and back again. In the better light of the living room he
noticed with interest that there actually \emph{was} dust in the creases
of her face.

``They do get so noisy,'' she said. ``They\textquotesingle re disappointed
because they couldn\textquotesingle t go to see the hanging,
that\textquotesingle s what it is. I\textquotesingle m too busy to take
them, and Tom won\textquotesingle t be back from work in time.''

``Why can\textquotesingle t we go and see the hanging?'' roared the boy in
his huge voice.

``Want to see the hanging! Want to see the hanging!'' chanted the little
girl, still capering round.

Some Eurasian prisoners, guilty of war crimes, were to be hanged in the
Park that evening, Winston remembered. This happened about once a month,
and was a popular spectacle. Children always clamored to be taken to see
it. He took his leave of Mrs. Parsons and made for the door. But he had
not gone six steps down the passage when something hit the back of his
neck an agonizingly painful blow. It was as though a red-hot wire had
been jabbed into him. He spun round just in time to see Mrs. Parsons
dragging her son back into the doorway while the boy pocketed a
catapult.

``Goldstein!'' bellowed the boy as the door closed on him. But what most
struck Winston was the look of helpless fright on the
woman\textquotesingle s grayish face.

Back in the flat he stepped quickly past the telescreen and sat down at
the table again, still rubbing his neck. The music from the telescreen
had stopped. Instead, a clipped military voice was reading out, with a
sort of brutal relish, a description of the armaments of the new
Floating Fortress which had just been anchored between Iceland and the
Faroe Islands.

With those children, he thought, that wretched woman must lead a life of
terror. Another year, two years, and they would be watching her night
and day for symptoms of unorthodoxy. Nearly all children nowadays were
horrible. What was worst of all was that by means of such organizations
as the Spies they were systematically turned into ungovernable little
savages, and yet this produced in them no tendency whatever to rebel
against the discipline of the Party. On the contrary, they adored the
Party and everything connected with it. The songs, the processions, the
banners, the hiking, the drilling with dummy rifles, the yelling of
slogans, the worship of Big Brother---it was all a sort of glorious game
to them. All their ferocity was turned outwards, against the enemies of
the State, against foreigners, traitors, saboteurs, thought-criminals.
It was almost normal for people over thirty to be frightened of their
own children. And with good reason, for hardly a week passed in which
the \emph{Times} did not carry a paragraph describing how some
eavesdropping little sneak---``child hero'' was the phrase generally
used---had overheard some compromising remark and denounced his parents
to the Thought Police.

The sting of the catapult bullet had worn off. He picked up his pen
half-heartedly, wondering whether he could find something more to write
in the diary. Suddenly he began thinking of O\textquotesingle Brien
again.

Years ago---how long was it? Seven years it must be---he had dreamed
that he was walking through a pitch-dark room. And someone sitting to
one side of him had said as he passed: ``We shall meet in the place where
there is no darkness.'' It was said very quietly, almost casually---a
statement, not a command. He had walked on without pausing. What was
curious was that at the time, in the dream, the words had not made much
impression on him. It was only later and by degrees that they had seemed
to take on significance. He could not now remember whether it was before
or after having the dream that he had seen O\textquotesingle Brien for
the first time; nor could he remember when he had first identified the
voice as O\textquotesingle Brien\textquotesingle s. But at any rate the
identification existed. It was O\textquotesingle Brien who had spoken to
him out of the dark.

Winston had never been able to feel sure---even after this
morning\textquotesingle s flash of the eyes it was still impossible to
be sure---whether O\textquotesingle Brien was a friend or an enemy. Nor
did it even seem to matter greatly. There was a link of understanding
between them more important than affection or partisanship. ``We shall
meet in the place where there is no darkness,'' he had said. Winston did
not know what it meant, only that in some way or another it would come
true.

The voice from the telescreen paused. A trumpet call, clear and
beautiful, floated into the stagnant air. The voice continued raspingly:

``Attention! Your attention, please! A newsflash has this moment arrived
from the Malabar front. Our forces in South India have won a glorious
victory. I am authorized to say that the action we are now reporting may
well bring the war within measurable distance of its end. Here is the
newsflash---''

Bad news coming, thought Winston. And sure enough, following on a gory
description of the annihilation of a Eurasian army, with stupendous
figures of killed and prisoners, came the announcement that, as from
next week, the chocolate ration would be reduced from thirty grams to
twenty.

Winston belched again. The gin was wearing off, leaving a deflated
feeling. The telescreen---perhaps to celebrate the victory, perhaps to
drown the memory of the lost chocolate---crashed into ``Oceania,
\textquotesingle tis for thee.'' You were supposed to stand to attention.
However, in his present position he was invisible.

``Oceania, \textquotesingle tis for thee'' gave way to lighter music.
Winston walked over to the window, keeping his back to the telescreen.
The day was still cold and clear. Somewhere far away a rocket bomb
exploded with a dull, reverberating roar. About twenty or thirty of them
a week were falling on London at present.

Down in the street the wind flapped the torn poster to and fro, and the
word INGSOC fitfully appeared and vanished. Ingsoc. The sacred
principles of Ingsoc. Newspeak, doublethink, the mutability of the past.
He felt as though he were wandering in the forests of the sea bottom,
lost in a monstrous world where he himself was the monster. He was
alone. The past was dead, the future was unimaginable. What certainty
had he that a single human creature now living was on his side? And what
way of knowing that the dominion of the Party would not endure \emph{for
ever}? Like an answer, the three slogans on the white face of the
Ministry of Truth came back at him:

\headline{WAR IS PEACE\\
FREEDOM IS SLAVERY\\
IGNORANCE IS STRENGTH
}

He took a twenty-five-cent piece out of his pocket. There, too, in tiny
clear lettering, the same slogans were inscribed, and on the other face
of the coin the head of Big Brother. Even from the coin the eyes pursued
you. On coins, on stamps, on the covers of books, on banners, on
posters, and on the wrapping of a cigarette packet---everywhere. Always
the eyes watching you and the voice enveloping you. Asleep or awake,
working or eating, indoors or out of doors, in the bath or in bed---no
escape. Nothing was your own except the few cubic centimeters inside
your skull.

The sun had shifted round, and the myriad windows of the Ministry of
Truth, with the light no longer shining on them, looked grim as the
loopholes of a fortress. His heart quailed before the enormous pyramidal
shape. It was too strong, it could not be stormed. A thousand rocket
bombs would not batter it down. He wondered again for whom he was
writing the diary. For the future, for the past---for an age that might
be imaginary. And in front of him there lay not death but annihilation.
The diary would be reduced to ashes and himself to vapor. Only the
Thought Police would read what he had written, before they wiped it out
of existence and out of memory. How could you make appeal to the future
when not a trace of you, not even an anonymous word scribbled on a piece
of paper, could physically survive?

The telescreen struck fourteen. He must leave in ten minutes. He had to
be back at work by fourteen-thirty.

Curiously, the chiming of the hour seeming to have put new heart into
him. He was a lonely ghost uttering a truth that nobody would ever hear.
But so long as he uttered it, in some obscure way the continuity was not
broken. It was not by making yourself heard but by staying sane that you
carried on the human heritage. He went back to the table, dipped his
pen, and wrote:

\begin{quotation}
To the future or to the past, to a time when thought is free, when
men are different from one another and do not live alone---to a time
when truth exists and what is done cannot be undone:\par
From the age of uniformity, from the age of solitude, from the age
of Big Brother, from the age of doublethink---greetings!
\end{quotation}

He was already dead, he reflected. It seemed to him that it was only
now, when he had begun to be able to formulate his thoughts, that he had
taken the decisive step. The consequences of every act are included in
the act itself. He wrote:

\begin{quotation}
Thoughtcrime does not entail death: thoughtcrime IS death.
\end{quotation}

Now that he had recognized himself as a dead man it became important to
stay alive as long as possible. Two fingers of his right hand were
inkstained. It was exactly the kind of detail that might betray you.
Some nosing zealot in the Ministry (a woman, probably; someone like the
little sandy-haired woman or the dark-haired girl from the Fiction
Department) might start wondering why he had been writing during the
lunch interval, why he had used an old-fashioned pen, \emph{what} he had
been writing---and then drop a hint in the appropriate quarter. He went
to the bathroom and carefully scrubbed the ink away with the gritty
dark-brown soap which rasped your skin like sandpaper and was therefore
well adapted for this purpose.

He put the diary away in the drawer. It was quite useless to think of
hiding it, but he could at least make sure whether or not its existence
had been discovered. A hair laid across the page-ends was too obvious.
With the tip of his finger he picked up an identifiable grain of whitish
dust and deposited it on the corner of the cover, where it was bound to
be shaken off if the book was moved.


\section{III}\label{iii}

Winston was dreaming of his mother.

He must, he thought, have been ten or eleven years old when his mother
had disappeared. She was a tall, statuesque, rather silent woman with
slow movements and magnificent fair hair. His father he remembered more
vaguely as dark and thin, dressed always in neat dark clothes (Winston
remembered especially the very thin soles of his
father\textquotesingle s shoes) and wearing spectacles. The two of them
must evidently have been swallowed up in one of the first great purges
of the Fifties.

At this moment his mother was sitting in some place deep down beneath
him, with his young sister in her arms. He did not remember his sister
at all, except as a tiny, feeble baby, always silent, with large,
watchful eyes. Both of them were looking up at him. They were down in
some subterranean place---the bottom of a well, for instance, or a very
deep grave---but it was a place which, already far below him, was itself
moving downwards. They were in the saloon of a sinking ship, looking up
at him through the darkening water. There was still air in the saloon,
they could still see him and he them, but all the while they were
sinking down, down into the green waters which in another moment must
hide them from sight forever. He was out in the light and air while they
were being sucked down to death, and they were down there \emph{because}
he was up here. He knew it and they knew it, and he could see the
knowledge in their faces. There was no reproach either in their faces or
in their hearts, only the knowledge that they must die in order that he
might remain alive, and that this was part of the unavoidable order of
things.

He could not remember what had happened, but he knew in his dream that
in some way the lives of his mother and his sister had been sacrificed
to his own. It was one of those dreams which, while retaining the
characteristic dream scenery, are a continuation of
one\textquotesingle s intellectual life, and in which one becomes aware
of facts and ideas which still seem new and valuable after one is awake.
The thing that now suddenly struck Winston was that his
mother\textquotesingle s death, nearly thirty years ago, had been tragic
and sorrowful in a way that was no longer possible. Tragedy, he
perceived, belonged to the ancient time, to a time when there were still
privacy, love, and friendship, and when the members of a family stood by
one another without needing to know the reason. His
mother\textquotesingle s memory tore at his heart because she had died
loving him, when he was too young and selfish to love her in return, and
because somehow, he did not remember how, she had sacrificed herself to
a conception of loyalty that was private and unalterable. Such things,
he saw, could not happen today. Today there were fear, hatred, and pain,
but no dignity of emotion, no deep or complex sorrows. All this he
seemed to see in the large eyes of his mother and his sister, looking up
at him through the green water, hundreds of fathoms down and still
sinking.

Suddenly he was standing on short springy turf, on a summer evening when
the slanting rays of the sun gilded the ground. The landscape that he
was looking at recurred so often in his dreams that he was never fully
certain whether or not he had seen it in the real world. In his waking
thoughts he called it the Golden Country. It was an old, rabbit-bitten
pasture, with a foot track wandering across it and a molehill here and
there. In the ragged hedge on the opposite side of the field the boughs
of the elm trees were swaying very faintly in the breeze, their leaves
just stirring in dense masses like women\textquotesingle s hair.
Somewhere near at hand, though out of sight, there was a clear,
slow-moving stream where dace were swimming in the pools under the
willow trees.

The girl with dark hair was coming toward him across the field. With
what seemed a single movement she tore off her clothes and flung them
disdainfully aside. Her body was white and smooth, but it aroused no
desire in him; indeed, he barely looked at it. What overwhelmed him in
that instant was admiration for the gesture with which she had thrown
her clothes aside. With its grace and carelessness it seemed to
annihilate a whole culture, a whole system of thought, as though Big
Brother and the Party and the Thought Police could all be swept into
nothingness by a single splendid movement of the arm. That too was a
gesture belonging to the ancient time. Winston woke up with the word
``Shakespeare'' on his lips.

The telescreen was giving forth an ear-splitting whistle which continued
on the same note for thirty seconds. It was nought seven fifteen,
getting-up time for office workers. Winston wrenched his body out of
bed---naked, for a member of the Outer Party received only three
thousand clothing coupons annually, and a suit of pajamas was six
hundred---and seized a dingy singlet and a pair of shorts that were
lying across a chair. The Physical Jerks would begin in three minutes.
The next moment he was doubled up by a violent coughing fit which nearly
always attacked him soon after waking up. It emptied his lungs so
completely that he could only begin breathing again by lying on his back
and taking a series of deep gasps. His veins had swelled with the effort
of the cough, and the varicose ulcer had started itching.

``Thirty to forty group!'' yapped a piercing female voice. ``Thirty to
forty group! Take your places, please. Thirties to forties!''

Winston sprang to attention in front of the telescreen, upon which the
image of a youngish woman, scrawny but muscular, dressed in tunic and
gym shoes, had already appeared.

``Arms bending and stretching!'' she rapped out. ``Take your time by me.
\emph{One}, two, three, four! \emph{One}, two, three, four! Come on,
comrades, put a bit of life into it! \emph{One}, two, three, four!
\emph{One}, two, three, four!...''

The pain of the coughing fit had not quite driven out of
Winston\textquotesingle s mind the impression made by his dream, and the
rhythmic movements of the exercise restored it somewhat. As he
mechanically shot his arms back and forth, wearing on his face the look
of grim enjoyment which was considered proper during the Physical Jerks,
he was struggling to think his way backward into the dim period of his
early childhood. It was extraordinarily difficult. Beyond the late
Fifties everything faded. When there were no external records that you
could refer to, even the outline of your own life lost its sharpness.
You remembered huge events which had quite probably not happened, you
remembered the detail of incidents without being able to recapture their
atmosphere, and there were long blank periods to which you could assign
nothing. Everything had been different then. Even the names of
countries, and their shapes on the map, had been different. Airstrip
One, for instance, had not been so called in those days: it had been
called England or Britain, though London, he felt fairly certain, had
always been called London.

Winston could not definitely remember a time when his country had not
been at war, but it was evident that there had been a fairly long
interval of peace during his childhood, because one of his early
memories was of an air raid which appeared to take everyone by surprise.
Perhaps it was the time when the atomic bomb had fallen on Colchester.
He did not remember the raid itself, but he did remember his
father\textquotesingle s hand clutching his own as they hurried down,
down, down into some place deep in the earth, round and round a spiral
suitcase which rang under his feet and which finally wearied his legs
that he began whimpering and they had to stop and rest. His mother, in
her slow dreamy way, was following a long way behind them. She was
carrying his baby sister---or perhaps it was only a bundle of blankets
that she was carrying: he was not certain whether his sister had been
born then. Finally they had emerged into a noisy, crowded place which he
had realized to be a Tube station.

There were people sitting all over the stone-flagged floor, and other
people, packed tightly together, were sitting on metal bunks, one above
the other. Winston and his mother and father found themselves a place on
the floor, and near them an old man and an old woman were sitting side
by side on a bunk. The old man had on a decent dark suit and a black
cloth cap pushed back from very white hair; his face was scarlet and his
eyes were blue and full of tears. He reeked of gin. It seemed to breathe
out of his skin in place of sweat, and one could have fancied that the
tears welling from his eyes were pure gin. But though slightly drunk he
was also suffering under some grief that was genuine and unbearable. In
his childish way Winston grasped that some terrible thing, something
that was beyond forgiveness and could never be remedied, had just
happened. It also seemed to him that he knew what it was. Someone whom
the old man loved, a little granddaughter perhaps, had been killed.
Every few minutes the old man kept repeating:

``We didn\textquotesingle t ought to \textquotesingle ave trusted
\textquotesingle em. I said so, Ma, didn\textquotesingle t I?
That\textquotesingle s what come of trusting \textquotesingle em. I said
so all along. We didn\textquotesingle t ought to \textquotesingle ave
trusted the buggers.''

But which buggers they didn\textquotesingle t ought to have trusted
Winston could not now remember.

Since about that time, war had been literally continuous, though
strictly speaking it had not always been the same war. For several
months during his childhood there had been confused street fighting in
London itself, some of which he remembered vividly. But to trace out the
history of the whole period, to say who was fighting whom at any given
moment, would have been utterly impossible, since no written record, and
no spoken word, ever made mention of any other alignment than the
existing one. At this moment, for example, in 1984 (if it was 1984),
Oceania was at war with Eurasia and in alliance with Eastasia. In no
public or private utterance was it ever admitted that the three powers
had at any time been grouped along different lines. Actually, as Winston
well knew, it was only four years since Oceania had been at war with
Eastasia and in alliance with Eurasia. But that was merely a piece of
furtive knowledge which he happened to possess because his memory was
not satisfactorily under control. Officially the change of partners had
never happened. Oceania was at war with Eurasia: therefore Oceania had
always been at war with Eurasia. The enemy of the moment always
represented absolute evil, and it followed that any past or future
agreement with him was impossible.

The frightening thing, he reflected for the ten thousandth time as he
forced his shoulders painfully backward (with hands on hips, they were
gyrating their bodies from the waist, an exercise that was supposed to
be good for the back muscles)---the frightening thing was that it might
all be true. If the Party could thrust its hand into the past and say of
this or that event, \emph{it never happened}---that, surely, was more
terrifying than mere torture and death.

The Party said that Oceania had never been in alliance with Eurasia. He,
Winston Smith, knew that Oceania had been in alliance with Eurasia as
short a time as four years ago. But where did that knowledge exist? Only
in his own consciousness, which in any case must soon be annihilated.
And if all others accepted the lie which the Party imposed---if all
records told the same tale---then the lie passed into history and became
truth. ``Who controls the past,'' ran the Party slogan, ``controls the
future: who controls the present controls the past.'' And yet the past,
though of its nature alterable, never had been altered. Whatever was
true now was true from everlasting to everlasting. It was quite simple.
All that was needed was an unending series of victories over your own
memory. ``Reality control,'' they called it; in Newspeak, ``doublethink.''

``Stand easy!'' barked the instructress, a little more genially.

Winston sank his arms to his sides and slowly refilled his lungs with
air. His mind slid away into the labyrinthine world of doublethink. To
know and not to know, to be conscious of complete truthfulness while
telling carefully constructed lies, to hold simultaneously two opinions
which canceled out, knowing them to be contradictory and believing in
both of them, to use logic against logic, to repudiate morality while
laying claim to it, to believe that democracy was impossible and that
the Party was the guardian of democracy, to forget whatever it was
necessary to forget, then to draw it back into memory again at the
moment when it was needed, and then promptly to forget it again, and
above all, to apply the same process to the process itself---that was
the ultimate subtlety: consciously to induce unconsciousness, and then,
once again, to become unconscious of the act of hypnosis you had just
performed. Even to understand the word ``doublethink'' involved the use of
doublethink.

The instructress had called them to attention again. ``And now
let\textquotesingle s see which of us can touch our toes!'' she said
enthusiastically. ``Right over from the hips, please, comrades.
\emph{One}-two! \emph{One}-two!...''

Winston loathed this exercise, which sent shooting pains all the way
from his heels to his buttocks and often ended by bringing on another
coughing fit. The half-pleasant quality went out of his meditations. The
past, he reflected, had not merely been altered, it had been actually
destroyed. For how could you establish even the most obvious fact when
there existed no record outside your own memory? He tried to remember in
what year he had first heard mention of Big Brother. He thought it must
have been at some time in the Sixties, but it was impossible to be
certain. In the Party histories, of course, Big Brother figured as the
leader and guardian of the Revolution since its very earliest days. His
exploits had been gradually pushed backwards in time until already they
extended into the fabulous world of the Forties and the Thirties, when
the capitalists in their strange cylindrical hats still rode through the
streets of London in great gleaming motor cars or horse carriages with
glass sides. There was no knowing how much of this legend was true and
how much invented. Winston could not even remember at what date the
Party itself had come into existence. He did not believe he had ever
heard the word Ingsoc before 1960, but it was possible that in its
Oldspeak form---``English Socialism,'' that is to say---it had been
current earlier. Everything melted into mist. Sometimes, indeed, you
could put your finger on a definite lie. It was not true, for example,
as was claimed in the Party history books, that the Party had invented
airplanes. He remembered airplanes since his earliest childhood. But you
could prove nothing. There was never any evidence. Just once in his
whole life he had held in his hands unmistakable documentary proof of
the falsification of a historical fact. And on that occasion---

``Smith!'' screamed the shrewish voice from the telescreen. ``6079 Smith W!
Yes, \emph{you}! Bend lower, please! You can do better than that.
You\textquotesingle re not trying. Lower, please!
\emph{That\textquotesingle s} better, comrade. Now stand at ease, the
whole squad, and watch me.''

A sudden hot sweat had broken out all over Winston\textquotesingle s
body. His face remained completely inscrutable. Never show dismay! Never
show resentment! A single flicker of the eyes could give you away. He
stood watching while the instructress raised her arms above her head
and---one could not say gracefully, but with remarkable neatness and
efficiency---bent over and tucked the first joint of her fingers under
her toes.

``\emph{There}, comrades! \emph{That\textquotesingle s} how I want to see
you doing it. Watch me again. I\textquotesingle m thirty-nine and
I\textquotesingle ve had four children. Now look.'' She bent over again.
``You see \emph{my} knees aren\textquotesingle t bent. You can all do it
if you want to,'' she added as she straightened herself up. ``Anyone under
forty-five is perfectly capable of touching his toes. We
don\textquotesingle t all have the privilege of fighting in the front
line, but at least we can all keep fit. Remember our boys on the Malabar
front! And the sailors in the Floating Fortresses! Just think what
\emph{they} have to put up with. Now try again. That\textquotesingle s
better, comrade, that\textquotesingle s \emph{much} better,'' she added
encouragingly as Winston, with a violent lunge, succeeded in touching
his toes with knees unbent, for the first time in several years.


\section{IV}\label{iv}

With the deep, unconscious sigh which not even the nearness of the
telescreen could prevent him from uttering when his
day\textquotesingle s work started, Winston pulled the speakwrite toward
him, blew the dust from its mouthpiece, and put on his spectacles. Then
he unrolled and clipped together four small cylinders of paper which had
already flopped out of the pneumatic tube on the right-hand side of his
desk.

In the walls of the cubicle there were three orifices. To the right of
the speakwrite, a small pneumatic tube for written messages; to the
left, a larger one for newspapers; and in the side wall, within easy
reach of Winston\textquotesingle s arm, a large oblong slit protected by
a wire grating. This last was for the disposal of waste paper. Similar
slits existed in thousands or tens of thousands throughout the building,
not only in every room but at short intervals in every corridor. For
some reason they were nicknamed memory holes. When one knew that any
document was due for destruction, or even when one saw a scrap of waste
paper lying about, it was an automatic action to lift the flap of the
nearest memory hole and drop it in, whereupon it would be whirled away
on a current of warm air to the enormous furnaces which were hidden
somewhere in the recesses of the building.

Winston examined the four slips of paper which he had unrolled. Each
contained a message of only one or two lines, in the abbreviated
jargon---not actually Newspeak, but consisting largely of Newspeak
words---which was used in the Ministry for internal purposes. They ran:

\begin{quotation}
times 17.3.84 bb speech malreported africa rectify

times 19.12.83 forecasts 3 yp 4th quarter 83 misprints verify current
issue

times 14.2.84 miniplenty malquoted chocolate rectify

times 3.12.83 reporting bb dayorder doubleplusungood refs unpersons
rewrite fullwise upsub antefiling.
\end{quotation}

With a faint feeling of satisfaction Winston laid the fourth message
aside. It was an intricate and responsible job and had better be dealt
with last. The other three were routine matters, though the second one
would probably mean some tedious wading through lists of figures.

Winston dialed ``back numbers'' on the telescreen and called for the
appropriate issues of the \emph{Times}, which slid out of the pneumatic
tube after only a few minutes\textquotesingle{} delay. The messages he
had received referred to articles or news items which for one reason or
another it was thought necessary to alter, or, as the official phrase
had it, to rectify. For example, it appeared from the \emph{Times} of
the seventeenth of March that Big Brother, in his speech of the previous
day, had predicted that the South Indian front would remain quiet but
that a Eurasian offensive would shortly be launched in North Africa. As
it happened, the Eurasian Higher Command had launched its offensive in
South India and left North Africa alone. It was therefore necessary to
rewrite a paragraph of Big Brother\textquotesingle s speech in such a
way as to make him predict the thing that had actually happened. Or
again, the \emph{Times} of the nineteenth of December had published the
official forecasts of the output of various classes of consumption goods
in the fourth quarter of 1983, which was also the sixth quarter of the
Ninth Three-Year Plan. Today\textquotesingle s issue contained a
statement of the actual output, from which it appeared that the
forecasts were in every instance grossly wrong.
Winston\textquotesingle s job was to rectify the original figures by
making them agree with the later ones. As for the third message, it
referred to a very simple error which could be set right in a couple of
minutes. As short a time ago as February, the Ministry of Plenty had
issued a promise (a ``categorical pledge'' were the official words) that
there would be no reduction of the chocolate ration during 1984.
Actually, as Winston was aware, the chocolate ration was to be reduced
from thirty grams to twenty at the end of the present week. All that was
needed was to substitute for the original promise a warning that it
would probably be necessary to reduce the ration at some time in April.

As soon as Winston had dealt with each of the messages, he clipped his
speakwritten corrections to the appropriate copy of the \emph{Times} and
pushed them into the pneumatic tube. Then, with a movement which was as
nearly as possible unconscious, he crumpled up the original message and
any notes that he himself had made, and dropped them into the memory
hole to be devoured by the flames.

What happened in the unseen labyrinth to which the pneumatic tubes led,
he did not know in detail, but he did know in general terms. As soon as
all the corrections which happened to be necessary in any particular
number of the \emph{Times} had been assembled and collated, that number
would be reprinted, the original copy destroyed, and the corrected copy
placed on the files in its stead. This process of continuous alteration
was applied not only to newspapers, but to books, periodicals,
pamphlets, posters, leaflets, films, sound tracks, cartoons,
photographs---to every kind of literature or documentation which might
conceivably hold any political or ideological significance. Day by day
and almost minute by minute the past was brought up to date. In this way
every prediction made by the Party could be shown by documentary
evidence to have been correct; nor was any item of news, or any
expression of opinion, which conflicted with the needs of the moment,
ever allowed to remain on record. All history was a palimpsest, scraped
clean and reinscribed exactly as often as was necessary. In no case
would it have been possible, once the deed was done, to prove that any
falsification had taken place. The largest section of the Records
Department, far larger than the one in which Winston worked, consisted
simply of persons whose duty it was to track down and collect all copies
of books, newspapers, and other documents which had been superseded and
were due for destruction. A number of the \emph{Times} which might,
because of changes in political alignment, or mistaken prophecies
uttered by Big Brother, have been rewritten a dozen times still stood on
the files bearing its original date, and no other copy existed to
contradict it. Books, also, were recalled and rewritten again and again,
and were invariably reissued without any admission that any alteration
had been made. Even the written instructions which Winston received, and
which he invariably got rid of as soon as he had dealt with them, never
stated or implied that an act of forgery was to be committed; always the
reference was to slips, errors, misprints, or misquotations which it was
necessary to put right in the interests of accuracy.

But actually, he thought as he readjusted the Ministry of
Plenty\textquotesingle s figures, it was not even forgery. It was merely
the substitution of one piece of nonsense for another. Most of the
material that you were dealing with had no connection with anything in
the real world, not even the kind of connection that is contained in a
direct lie. Statistics were just as much a fantasy in their original
version as in their rectified version. A great deal of the time you were
expected to make them up out of your head. For example, the Ministry of
Plenty\textquotesingle s forecast had estimated the output of boots for
the quarter at a hundred and forty-five million pairs. The actual output
was given as sixty-two millions. Winston, however, in rewriting the
forecast, marked the figure down to fifty-seven millions, so as to allow
for the usual claim that the quota had been overfulfilled. In any case,
sixty-two millions was no nearer the truth than fifty-seven millions, or
than a hundred and forty-five millions. Very likely no boots had been
produced at all. Likelier still, nobody knew how many had been produced,
much less cared. All one knew was that every quarter astronomical
numbers of boots were produced on paper, while perhaps half the
population of Oceania went barefoot. And so it was with every class of
recorded fact, great or small. Everything faded away into a shadow-world
in which, finally, even the date of the year had become uncertain.

Winston glanced across the hall. In the corresponding cubicle on the
other side a small, precise-looking, dark-chinned man named Tillotson
was working steadily away, with a folded newspaper on his knee and his
mouth very close to the mouthpiece of the speakwrite. He had the air of
trying to keep what he was saying a secret between himself and the
telescreen. He looked up, and his spectacles darted a hostile flash in
Winston\textquotesingle s direction.

Winston hardly knew Tillotson, and had no idea what work he was employed
on. People in the Records Department did not readily talk about their
jobs. In the long, windowless hall, with its double row of cubicles and
its endless rustle of papers and hum of voices murmuring into
speakwrites, there were quite a dozen people whom Winston did not even
know by name, though he daily saw them hurrying to and fro in the
corridors or gesticulating in the Two Minutes Hate. He knew that in the
cubicle next to him the little woman with sandy hair toiled day in, day
out, simply at tracking down and deleting from the press the names of
people who had been vaporized and were therefore considered never to
have existed. There was a certain fitness in this, since her own husband
had been vaporized a couple of years earlier. And a few cubicles away a
mild, ineffectual, dreamy creature named Ampleforth, with very hairy
ears and a surprising talent for juggling with rhymes and meters, was
engaged in producing garbled versions---definitive texts, they were
called---of poems which had become ideologically offensive but which for
one reason or another were to be retained in the anthologies. And this
hall, with its fifty workers or thereabouts, was only one sub-section, a
single cell, as it were, in the huge complexity of the Records
Department. Beyond, above, below, were other swarms of workers engaged
in an unimaginable multitude of jobs. There were the huge printing shops
with their sub-editors, their typography experts, and their elaborately
equipped studios for the faking of photographs. There was the
teleprograms section with its engineers, its producers, and its teams of
actors specially chosen for their skill in imitating voices. There were
the armies of reference clerks whose job was simply to draw up lists of
books and periodicals which were due for recall. There were the vast
repositories where the corrected documents were stored, and the hidden
furnaces where the original copies were destroyed. And somewhere or
other, quite anonymous, there were the directing brains who coordinated
the whole effort and laid down the lines of policy which made it
necessary that this fragment of the past should be preserved, that one
falsified, and the other rubbed out of existence.

And the Records Department, after all, was itself only a single branch
of the Ministry of Truth, whose primary job was not to reconstruct the
past but to supply the citizens of Oceania with newspapers, films,
textbooks, telescreen programs, plays, novels---with every conceivable
kind of information, instruction, or entertainment, from a statue to a
slogan, from a lyric poem to a biological treatise, and from a
child\textquotesingle s spelling book to a Newspeak dictionary. And the
Ministry had not only to supply the multifarious needs of the Party, but
also to repeat the whole operation at a lower level for the benefit of
the proletariat. There was a whole chain of separate departments dealing
with proletarian literature, music, drama, and entertainment generally.
Here were produced rubbishy newspapers containing almost nothing except
sport, crime, and astrology, sensational five-cent novelettes, films
oozing with sex, and sentimental songs which were composed entirely by
mechanical means on a special kind of kaleidoscope known as a
versificator. There was even a whole sub-section---\emph{Pornosec}, it
was called in Newspeak---engaged in producing the lowest kind of
pornography, which was sent out in sealed packets and which no Party
member, other than those who worked on it, was permitted to look at.

Three messages had slid out of the pneumatic tube while Winston was
working; but they were simple matters, and he had disposed of them
before the Two Minutes Hate interrupted him. When the Hate was over he
returned to his cubicle, took the Newspeak dictionary from the shelf,
pushed the speakwrite to one side, cleaned his spectacles, and settled
down to his main job of the morning.

Winston\textquotesingle s greatest pleasure in life was in his work.
Most of it was a tedious routine, but included in it there were also
jobs so difficult and intricate that you could lose yourself in them as
in the depths of a mathematical problem---delicate pieces of forgery in
which you had nothing to guide you except your knowledge of the
principles of Ingsoc and your estimate of what the Party wanted you to
say. Winston was good at this kind of thing. On occasion he had even
been entrusted with the rectification of the \emph{Times} leading
articles, which were written entirely in Newspeak. He unrolled the
message that he had set aside earlier. It ran:

\begin{quotation}
times 3.12.83 reporting bb dayorder doubleplusungood refs unpersons
rewrite fullwise upsub antefiling.
\end{quotation}

In Oldspeak (or standard English) this might be rendered:

\begin{quotation}
The reporting of Big Brother\textquotesingle s Order for the Day in the
\emph{Times} of December 3rd 1983 is extremely unsatisfactory and makes
references to nonexistent persons. Rewrite it in full and submit your
draft to higher authority before filing.
\end{quotation}

Winston read through the offending article. Big
Brother\textquotesingle s Order for the Day, it seemed, had been chiefly
devoted to praising the work of an organization known as FFCC, which
supplied cigarettes and other comforts to the sailors in the Floating
Fortresses. A certain Comrade Withers, a prominent member of the Inner
Party, had been singled out for special mention and awarded a
decoration, the Order of Conspicuous Merit, Second Class.

Three months later FFCC had suddenly been dissolved with no reasons
given. One could assume that Withers and his associates were now in
disgrace, but there had been no report of the matter in the press or on
the telescreen. That was to be expected, since it was unusual for
political offenders to be put on trial or even publicly denounced. The
great purges involving thousands of people, with public trials of
traitors and thought-criminals who made abject confession of their
crimes and were afterwards executed, were special showpieces not
occurring oftener than once in a couple of years. More commonly, people
who had incurred the displeasure of the Party simply disappeared and
were never heard of again. One never had the smallest clue as to what
had happened to them. In some cases they might not even be dead. Perhaps
thirty people personally known to Winston, not counting his parents, had
disappeared at one time or another.

Winston stroked his nose gently with a paper clip. In the cubicle across
the way Comrade Tillotson was still crouching secretively over his
speakwrite. He raised his head for a moment: again the hostile
spectacle-flash. Winston wondered whether Comrade Tillotson was engaged
on the same job as himself. It was perfectly possible. So tricky a piece
of work would never be entrusted to a single person; on the other hand,
to turn it over to a committee would be to admit openly that an act of
fabrication was taking place. Very likely as many as a dozen people were
now working away on rival versions of what Big Brother had actually
said. And presently some master brain in the Inner Party would select
this version or that, would re-edit it and set in motion the complex
processes of cross-referencing that would be required, and then the
chosen lie would pass into the permanent records and become truth.

Winston did not know why Withers had been disgraced. Perhaps it was for
corruption or incompetence. Perhaps Big Brother was merely getting rid
of a too-popular subordinate. Perhaps Withers or someone close to him
had been suspected of heretical tendencies. Or perhaps---what was
likeliest of all---the thing had simply happened because purges and
vaporizations were a necessary part of the mechanics of government. The
only real clue lay in the words ``refs unpersons,'' which indicated that
Withers was already dead. You could not invariably assume this to be the
case when people were arrested. Sometimes they were released and allowed
to remain at liberty for as much as a year or two years before being
executed. Very occasionally some person whom you had believed dead long
since would make a ghostly reappearance at some public trial where he
would implicate hundreds of others by his testimony before vanishing,
this time forever. Withers, however, was already an \emph{unperson}. He
did not exist; he had never existed. Winston decided that it would not
be enough simply to reverse the tendency of Big
Brother\textquotesingle s speech. It was better to make it deal with
something totally unconnected with its original subject.

He might turn the speech into the usual denunciation of traitors and
thought-criminals, but that was a little too obvious, while to invent a
victory at the front, or some triumph of overproduction in the Ninth
Three-Year Plan, might complicate the records too much. What was needed
was a piece of pure fantasy. Suddenly there sprang into his mind,
ready-made as it were, the image of a certain Comrade Ogilvy, who had
recently died in battle, in heroic circumstances. There were occasions
when Big Brother devoted his Order for the Day to commemorating some
humble, rank-and-file Party member whose life and death he held up as an
example worthy to be followed. Today he should commemorate Comrade
Ogilvy. It was true that there was no such person as Comrade Ogilvy, but
a few lines of print and a couple of faked photographs would soon bring
him into existence.

Winston thought for a moment, then pulled the speakwrite toward him and
began dictating in Big Brothers familiar style: a style at once military
and pedantic, and, because of a trick of asking questions and then
promptly answering them (``What lessons do we learn from this fact,
comrades? The lessons---which is also one of the fundamental principles
of Ingsoc---that,'' etc., etc.), easy to imitate.

At the age of three Comrade Ogilvy had refused all toys except a drum, a
submachine gun, and a model helicopter. At six---a year early, by a
special relaxation of the rules---he had joined the Spies; at nine he
had been a troop leader. At eleven he had denounced his uncle to the
Thought Police after overhearing a conversation which appeared to him to
have criminal tendencies. At seventeen he had been a district organizer
of the Junior Anti-Sex League. At nineteen he had designed a hand
grenade which had been adopted by the Ministry of Peace and which, at
its first trial, had killed thirty-one Eurasian prisoners in one burst.
At twenty-three he had perished in action. Pursued by enemy jet planes
while flying over the Indian Ocean with important despatches, he had
weighted his body with his machine gun and leapt out of the helicopter
into deep water, despatches and all---an end, said Big Brother, which it
was impossible to contemplate without feelings of envy. Big Brother
added a few remarks on the purity and single-mindedness of Comrade
Ogilvy\textquotesingle s life. He was a total abstainer and a nonsmoker,
had no recreations except a daily hour in the gymnasium, and had taken a
vow of celibacy, believing marriage and the care of a family to be
incompatible with a twenty-four-hour-a-day devotion to duty. He had no
subjects of conversation except the principles of Ingsoc, and no aim in
life except the defeat of the Eurasian enemy and the hunting-down of
spies, saboteurs, thought-criminals, and traitors generally.

Winston debated with himself whether to award Comrade Ogilvy the Order
of Conspicuous Merit; in the end he decided against it because of the
unnecessary cross-referencing that it would entail.

Once again he glanced at his rival in the opposite cubicle. Something
seemed to tell him with certainty that Tillotson was busy on the same
job as himself. There was no way of knowing whose version would finally
be adopted, but he felt a profound conviction that it would be his own.
Comrade Ogilvy, unimagined an hour ago, was now a fact. It struck him as
curious that you could create dead men but not living ones. Comrade
Ogilvy, who had never existed in the present, now existed in the past,
and when once the act of forgery was forgotten, he would exist just as
authentically, and upon the same evidence, as Charlemagne or Julius
Caesar.


\section{V}\label{v}

In the low-ceilinged canteen, deep under ground, the lunch queue jerked
slowly forward. The room was already very full and deafeningly noisy.
From the grille at the counter the steam of stew came pouring forth,
with a sour metallic smell which did not quite overcome the fumes of
Victory Gin. On the far side of the room there was a small bar, a mere
hole in the wall, where gin could be bought at ten cents the large nip.

``Just the man I was looking for,'' said a voice at
Winston\textquotesingle s back.

He turned round. It was his friend Syme, who worked in the Research
Department. Perhaps ``friend'' was not exactly the right word. You did not
have friends nowadays, you had comrades; but there were some comrades
whose society was pleasanter than that of others. Syme was a
philologist, a specialist in Newspeak. Indeed, he was one of the
enormous team of experts now engaged in compiling the Eleventh Edition
of the Newspeak dictionary. He was a tiny creature, smaller than
Winston, with dark hair and large, protuberant eyes, at once mournful
and derisive, which seemed to search your face closely while he was
speaking to you.

``I wanted to ask you whether you\textquotesingle d got any razor
blades,'' he said.

``Not one!'' said Winston with a sort of guilty haste.
``I\textquotesingle ve tried all over the place. They
don\textquotesingle t exist any longer.''

Everyone kept asking you for razor blades. Actually he had two unused
ones which he was hoarding up. There had been a famine of them for
months past. At any given moment there was some necessary article which
the Party shops were unable to supply. Sometimes it was buttons,
sometimes it was darning wool, sometimes it was shoelaces; at present it
was razor blades. You could only get hold of them, if at all, by
scrounging more or less furtively on the ``free'' market.

``I\textquotesingle ve been using the same blade for six weeks,'' he added
untruthfully.

The queue gave another jerk forward. As they halted he turned and faced
Syme again. Each of them took a greasy metal tray from a pile at the
edge of the counter.

``Did you go and see the prisoners hanged yesterday?'' said Syme.

``I was working,'' said Winston indifferently. ``I shall see it on the
flicks, I suppose.''

``A very inadequate substitute,'' said Syme.

His mocking eyes roved over Winston\textquotesingle s face. ``I know
you,'' the eyes seemed to say, ``I see through you. I know very well why
you didn\textquotesingle t go to see those prisoners hanged.'' In an
intellectual way, Syme was venomously orthodox. He would talk with a
disagreeable gloating satisfaction of helicopter raids on enemy
villages, the trials and confessions of thought-criminals, the
executions in the cellars of the Ministry of Love. Talking to him was
largely a matter of getting him away from such subjects and entangling
him, if possible, in the technicalities of Newspeak, on which he was
authoritative and interesting. Winston turned his head a little aside to
avoid the scrutiny of the large dark eyes.

``It was a good hanging,'' said Syme reminiscently. ``I think it spoils it
when they tie their feet together. I like to see them kicking. And above
all, at the end, the tongue sticking right out, and blue---a quite
bright blue. That\textquotesingle s the detail that appeals to me.''

``Nex\textquotesingle, please!'' yelled the white-aproned prole with the
ladle.

Winston and Syme pushed their trays beneath the grille. Onto each was
dumped swiftly the regulation lunch---a metal pannikin of pinkish-gray
stew, a hunk of bread, a cube of cheese, a mug of milkless Victory
Coffee, and one saccharine tablet.

``There\textquotesingle s a table over there, under that telescreen,''
said Syme. ``Let\textquotesingle s pick up a gin on the way.''

The gin was served out to them in handleless china mugs. They threaded
their way across the crowded room and unpacked their trays onto the
metal-topped table, on one corner of which someone had left a pool of
stew, a filthy liquid mess that had the appearance of vomit. Winston
took up his mug of gin, paused for an instant to collect his nerve, and
gulped the oily-tasting stuff down. When he had winked the tears out of
his eyes he suddenly discovered that he was hungry. He began swallowing
spoonfuls of the stew, which, in among its general sloppiness, had cubes
of spongy pinkish stuff which was probably a preparation of meat.
Neither of them spoke again till they had emptied their pannikins. From
the table at Winston\textquotesingle s left, a little behind his back,
someone was talking rapidly and continuously, a harsh gabble almost like
the quacking of a duck, which pierced the general uproar of the room.

``How is the dictionary getting on?'' said Winston, raising his voice to
overcome the noise.

``Slowly,'' said Syme. ``I\textquotesingle m on the adjectives.
It\textquotesingle s fascinating.''

He had brightened up immediately at the mention of Newspeak. He pushed
his pannikin aside, took up his hunk of bread in one delicate hand and
his cheese in the other, and leaned across the table so as to be able to
speak without shouting.

``The Eleventh Edition is the definitive edition,'' he said.
``We\textquotesingle re getting the language into its final shape---the
shape it\textquotesingle s going to have when nobody speaks anything
else. When we\textquotesingle ve finished with it, people like you will
have to learn it all over again. You think, I dare say, that our chief
job is inventing new words. But not a bit of it! We\textquotesingle re
destroying words---scores of them, hundreds of them, every day.
We\textquotesingle re cutting the language down to the bone. The
Eleventh Edition won\textquotesingle t contain a single word that will
become obsolete before the year 2050.''

He bit hungrily into his bread and swallowed a couple of mouthfuls, then
continued speaking, with a sort of pedant\textquotesingle s passion. His
thin dark face had become animated, his eyes had lost their mocking
expression and grown almost dreamy.

``It\textquotesingle s a beautiful thing, the destruction of words. Of
course the great wastage is in the verbs and adjectives, but there are
hundreds of nouns that can be got rid of as well. It isn\textquotesingle t
only the synonyms; there are also the antonyms. After all, what
justification is there for a word which is simply the opposite of some other
word? A word contains its opposite in itself. Take `good,' for instance. If
you have a word like `good,' what need is there for a word like `bad'?
`Ungood' will do just as well---better, because it\textquotesingle s an
exact opposite, which the other is not. Or again, if you want a stronger
version of `good,' what sense is there in having a whole string of vague
useless words like `excellent' and `splendid' and all the rest of them?
`Plusgood' covers the meaning, or `doubleplusgood' if you want something
stronger still. Of course we use those forms already, but in the final
version of Newspeak there\textquotesingle ll be nothing else. In the end the
whole notion of goodness and badness will be covered by only six words---in
reality, only one word. Don\textquotesingle t you see the beauty of that,
Winston? It was B.B.\textquotesingle s idea originally, of course,'' he
added as an afterthought.

A sort of vapid eagerness flitted across Winston\textquotesingle s face
at the mention of Big Brother. Nevertheless Syme immediately detected a
certain lack of enthusiasm.

``You haven\textquotesingle t a real appreciation of Newspeak, Winston,''
he said almost sadly. ``Even when you write it you\textquotesingle re
still thinking in Oldspeak. I\textquotesingle ve read some of those
pieces that you write in the \emph{Times} occasionally.
They\textquotesingle re good enough, but they\textquotesingle re
translations. In your heart you\textquotesingle d prefer to stick to
Oldspeak, with all its vagueness and its useless shades of meaning. You
don\textquotesingle t grasp the beauty of the destruction of words. Do
you know that Newspeak is the only language in the world whose
vocabulary gets smaller every year?''

Winston did know that, of course. He smiled, sympathetically he hoped,
not trusting himself to speak. Syme bit off another fragment of the
dark-colored bread, chewed it briefly, and went on:

``Don\textquotesingle t you see that the whole aim of Newspeak is to
narrow the range of thought? In the end we shall make thoughtcrime
literally impossible, because there will be no words in which to express
it. Every concept that can ever be needed will be expressed by exactly
\emph{one} word, with its meaning rigidly defined and all its subsidiary
meanings rubbed out and forgotten. Already, in the Eleventh Edition,
we\textquotesingle re not far from that point. But the process will
still be continuing long after you and I are dead. Every year fewer and
fewer words, and the range of consciousness always a little smaller.
Even now, of course, there\textquotesingle s no reason or excuse for
committing thoughtcrime. It\textquotesingle s merely a question of
self-discipline, reality-control. But in the end there
won\textquotesingle t be any need even for that. The Revolution will be
complete when the language is perfect. Newspeak is Ingsoc and Ingsoc is
Newspeak,'' he added with a sort of mystical satisfaction. ``Has it ever
occurred to you, Winston, that by the year 2050, at the very latest, not
a single human being will be alive who could understand such a
conversation as we are having now?''

``Except---'' began Winston doubtfully, and then stopped.

It had been on the tip of his tongue to say ``Except the proles,'' but he
checked himself, not feeling fully certain that this remark was not in
some way unorthodox. Syme, however, had divined what he was about to
say.

``The proles are not human beings,'' he said carelessly. ``By
2050---earlier, probably---all real knowledge of Oldspeak will have
disappeared. The whole literature of the past will have been destroyed.
Chaucer, Shakespeare, Milton, Byron---they\textquotesingle ll exist only in
Newspeak versions, not merely changed into something different, but actually
changed into something contradictory of what they used to be. Even the
literature of the Party will change. Even the slogans will change. How could
you have a slogan like `freedom is slavery' when the concept of freedom has
been abolished? The whole climate of thought will be different. In fact
there will \emph{be} no thought, as we understand it now. Orthodoxy means
not thinking---not needing to think. Orthodoxy is unconsciousness.''

One of these days, thought Winston with sudden deep conviction, Syme
will be vaporized. He is too intelligent. He sees too clearly and speaks
too plainly. The Party does not like such people. One day he will
disappear. It is written in his face.

Winston had finished his bread and cheese. He turned a little sideways
in his chair to drink his mug of coffee. At the table on his left the
man with the strident voice was still talking remorselessly away. A
young woman who was perhaps his secretary, and who was sitting with her
back to Winston, was listening to him and seemed to be eagerly agreeing
with everything that he said. From time to time Winston caught some such
remark as ``I think you\textquotesingle re \emph{so} right, I do
\emph{so} agree with you,'' uttered in a youthful and rather silly
feminine voice. But the other voice never stopped for an instant, even
when the girl was speaking. Winston knew the man by sight, though he
knew no more about him than that he held some important post in the
Fiction Department. He was a man of about thirty, with a muscular throat
and a large, mobile mouth. His head was thrown back a little, and
because of the angle at which he was sitting, his spectacles caught the
light and presented to Winston two blank discs instead of eyes. What was
slightly horrible was that from the stream of sound that poured out of
his mouth, it was almost impossible to distinguish a single word. Just
once Winston caught a phrase---``complete and final elimination of
Goldsteinism''---jerked out very rapidly and, as it seemed, all in one
piece, like a line of type cast solid. For the rest it was just a noise,
a quack-quack-quacking. And yet, though you could not actually hear what
the man was saying, you could not be in any doubt about its general
nature. He might be denouncing Goldstein and demanding sterner measures
against thought-criminals and saboteurs, he might be fulminating against
the atrocities of the Eurasian army, he might be praising Big Brother or
the heroes on the Malabar front---it made no difference. Whatever it
was, you could be certain that every word of it was pure orthodoxy, pure
Ingsoc. As he watched the eyeless face with the jaw moving rapidly up
and down, Winston had a curious feeling that this was not a real human
being but some kind of dummy. It was not the man\textquotesingle s brain
that was speaking; it was his larynx. The stuff that was coming out of
him consisted of words, but it was not speech in the true sense: it was
a noise uttered in unconsciousness, like the quacking of a duck.

Syme had fallen silent for a moment, and with the handle of his spoon
was tracing patterns in the puddle of stew. The voice from the other
table quacked rapidly on, easily audible in spite of the surrounding
din.

``There is a word in Newspeak,'' said Syme. ``I don\textquotesingle t know
whether you know it: \emph{duckspeak}, to quack like a duck. It is one
of those interesting words that have two contradictory meanings. Applied
to an opponent, it is abuse; applied to someone you agree with, it is
praise.''

Unquestionably Syme will be vaporized, Winston thought again. He thought
it with a kind of sadness, although well knowing that Syme despised him
and slightly disliked him, and was fully capable of denouncing him as a
thought-criminal if he saw any reason for doing so. There was something
subtly wrong with Syme. There was something that he lacked: discretion,
aloofness, a sort of saving stupidity. You could not say that he was
unorthodox. He believed in the principles of Ingsoc, he venerated Big
Brother, he rejoiced over victories, he hated heretics, not merely with
sincerity but with a sort of restless zeal, an up-to-dateness of
information, which the ordinary Party member did not approach. Yet a
faint air of disreputability always clung to him. He said things that
would have been better unsaid, he had read too many books, he frequented
the Chestnut Tree Café, haunt of painters and musicians. There was no
law, not even an unwritten law, against frequenting the Chestnut Tree
Café, yet the place was somehow ill-omened. The old, discredited leaders
of the Party had been used to gather there before they were finally
purged. Goldstein himself, it was said, had sometimes been seen there,
years and decades ago. Syme\textquotesingle s fate was not difficult to
foresee. And yet it was a fact that if Syme grasped, even for three
seconds, the nature of his, Winston\textquotesingle s, secret opinions,
he would betray him instantly to the Thought Police. So would anybody
else, for that matter, but Syme more than most. Zeal was not enough.
Orthodoxy was unconsciousness.

Syme looked up. ``Here comes Parsons,'' he said.

Something in the tone of his voice seemed to add, ``that bloody fool.''
Parsons, Winston\textquotesingle s fellow tenant at Victory Mansions,
was in fact threading his way across the room---tubby, middle-sized man
with fair hair and a froglike face. At thirty-five he was already
putting on rolls of fat at neck and waistline, but his movements were
brisk and boyish. His whole appearance was that of a little boy grown
large, so much so that although he was wearing the regulation overalls,
it was almost impossible not to think of him as being dressed in the
blue shorts, gray shirt, and red neckerchief of the Spies. In
visualizing him one saw always a picture of dimpled knees and sleeves
rolled back from pudgy forearms. Parsons did, indeed, invariably revert
to shorts when a community hike or any other physical activity gave him
an excuse for doing so. He greeted them both with a cheery ``Hullo,
hullo!'' and sat down at the table, giving off an intense smell of sweat.
Beads of moisture stood out all over his pink face. His powers of
sweating were extraordinary. At the Community Center you could always
tell when he had been playing table tennis by the dampness of the bat
handle. Syme had produced a strip of paper on which there was a long
column of words, and was studying it with an ink pencil between his
fingers.

``Look at him working away in the lunch hour,'' said Parsons, nudging
Winston. ``Keenness, eh? What\textquotesingle s that
you\textquotesingle ve got there, old boy? Something a bit too brainy
for me, I expect. Smith, old boy, I\textquotesingle ll tell you why
I\textquotesingle m chasing you. It\textquotesingle s that sub you
forgot to give me.''

``Which sub is that?'' said Winston, automatically feeling for money.
About a quarter of one\textquotesingle s salary had to be ear-marked for
voluntary subscriptions, which were so numerous that it was difficult to
keep track of them.

``For Hate Week. You know---the house-by-house fund. I\textquotesingle m
treasurer for our block. We\textquotesingle re making an all-out
effort---going to put on a tremendous show. I tell you, it
won\textquotesingle t be my fault if old Victory Mansions
doesn\textquotesingle t have the biggest outfit of flags in the whole
street. Two dollars you promised me.''

Winston found and handed over two creased and filthy notes, which
Parsons entered in a small notebook, in the neat handwriting of the
illiterate.

``By the way, old boy,'' he said, ``I hear that little beggar of mine let
fly at you with his catapult yesterday. I gave him a good dressing down
for it. In fact I told him I\textquotesingle d take the catapult away if
he does it again.''

``I think he was a little upset at not going to the execution,'' said
Winston.

``Ah, well---what I mean to say, shows the right spirit,
doesn\textquotesingle t it? Mischievous little beggars they are, both of
them, but talk about keenness! All they think about is the Spies, and
the war, of course. D\textquotesingle you know what that little girl of
mine did last Saturday, when her troop was on a hike out Berkhampstead
way? She got two other girls to go with her, slipped off from the hike,
and spent the whole afternoon following a strange man. They kept on his
tail for two hours, right through the woods, and then, when they got
into Amersham, handed him over to the patrols.''

``What did they do that for?'' said Winston, somewhat taken aback. Parsons
went on triumphantly:

``My kid made sure he was some kind of enemy agent---might have been
dropped by parachute, for instance. But here\textquotesingle s the
point, old boy. What do you think put her onto him in the first place?
She spotted he was wearing a funny kind of shoes---said
she\textquotesingle d never seen anyone wearing shoes like that before.
So the chances were he was a foreigner. Pretty smart for a nipper of
seven, eh?''

``What happened to the man?'' said Winston.

``Ah, that I couldn\textquotesingle t say, of course. But I
wouldn\textquotesingle t be altogether surprised if---'' Parsons made the
motion of aiming a rifle, and clicked his tongue for the explosion.

``Good,'' said Syme abstractedly, without looking up from his strip of
paper.

``Of course we can\textquotesingle t afford to take chances,'' agreed
Winston dutifully.

``What I mean to say, there is a war on,'' said Parsons.

As though in confirmation of this, a trumpet call floated from the
telescreen just above their heads. However, it was not the proclamation
of a military victory this time, but merely an announcement from the
Ministry of Plenty.

``Comrades!'' cried an eager youthful voice. ``Attention, comrades! We have
glorious news for you. We have won the battle for production! Returns
now completed of the output of all classes of consumption goods show
that the standard of living has risen by no less than twenty per cent
over the past year. All over Oceania this morning there were
irrepressible spontaneous demonstrations when workers marched out of
factories and offices and paraded through the streets with banners
voicing their gratitude to Big Brother for the new, happy life which his
wise leadership has bestowed upon us. Here are some of the completed
figures. Foodstuffs---''

The phrase ``our new, happy life'' recurred several times. It had been a
favorite of late with the Ministry of Plenty. Parsons, his attention
caught by the trumpet call, sat listening with a sort of gaping
solemnity, a sort of edified boredom. He could not follow the figures,
but he was aware that they were in some way a cause for satisfaction. He
had lugged out a huge and filthy pipe which was already half full of
charred tobacco. With the tobacco ration at a hundred grams a week it
was seldom possible to fill a pipe up to the top. Winston was smoking a
Victory Cigarette which he held carefully horizontal. The new ration did
not start till tomorrow and he had only four cigarettes left. For the
moment he had shut his ears to the remoter noises and was listening to
the stuff that streamed out of the telescreen. It appeared that there
had even been demonstrations to thank Big Brother for raising the
chocolate ration to twenty grams a week. And only yesterday, he
reflected, it had been announced that the ration was to be reduced to
twenty grams a week. Was it possible that they could swallow that, after
only twenty-four hours? Yes, they swallowed it. Parsons swallowed it
easily, with the stupidity of an animal. The eyeless creature at the
other table swallowed it fanatically, passionately, with a furious
desire to track down, denounce, and vaporize anyone who should suggest
that last week the ration had been thirty grams. Syme, too---in some
more complex way, involving doublethink---Syme swallowed it. Was he,
then, \emph{alone} in the possession of a memory?

The fabulous statistics continued to pour out of the telescreen. As
compared with last year there was more food, more clothes, more houses,
more furniture, more cooking pots, more fuel, more ships, more
helicopters, more books, more babies---more of everything except
disease, crime, and insanity. Year by year and minute by minute,
everybody and everything was whizzing rapidly upwards. As Syme had done
earlier, Winston had taken up his spoon and was dabbling in the
pale-colored gravy that dribbled across the table, drawing a long streak
of it out into a pattern. He meditated resentfully on the physical
texture of life. Had it always been like this? Had food always tasted
like this? He looked round the canteen. A low-ceilinged, crowded room,
its walls grimy from the contact of innumerable bodies; battered metal
tables and chairs, placed so close together that you sat with elbows
touching; bent spoons, dented trays, coarse white mugs; all surfaces
greasy, grime in every crack; and a sourish, composite smell of bad gin
and bad coffee and metallic stew and dirty clothes. Always in your
stomach and in your skin there was a sort of protest, a feeling that you
had been cheated of something that you had a right to. It was true that
he had no memories of anything greatly different. In any time that he
could accurately remember, there had never been quite enough to eat, one
had never had socks or underclothes that were not full of holes,
furniture had always been battered and rickety, rooms underheated, tube
trains crowded, houses falling to pieces, bread dark-colored, tea a
rarity, coffee filthy-tasting, cigarettes insufficient---nothing cheap
and plentiful except synthetic gin. And though, of course, it grew worse
as one\textquotesingle s body aged, was it not a sign that this was
\emph{not} the natural order of things, if one\textquotesingle s heart
sickened at the discomfort and dirt and scarcity, the interminable
winters, the stickiness of one\textquotesingle s socks, the lifts that
never worked, the cold water, the gritty soap, the cigarettes that came
to pieces, the food with its strange evil tastes? Why should one feel it
to be intolerable unless one had some kind of ancestral memory that
things had once been different?

He looked round the canteen again. Nearly everyone was ugly, and would
still have been ugly even if dressed otherwise than in the uniform blue
overalls. On the far side of the room, sitting at a table alone, a
small, curiously beetlelike man was drinking a cup of coffee, his little
eyes darting suspicious glances from side to side. How easy it was,
thought Winston, if you did not look about you, to believe that the
physical type set up by the Party as an ideal---tall muscular youths and
deep-bosomed maidens, blond-haired, vital, sunburnt, carefree---existed
and even predominated. Actually, so far as he could judge, the majority
of people in Airstrip One were small, dark, and ill-favored. It was
curious how that beetlelike type proliferated in the Ministries: little
dumpy men, growing stout very early in life, with short legs, swift
scuttling movements, and fat inscrutable faces with very small eyes. It
was the type that seemed to flourish best under the dominion of the
Party.

The announcement from the Ministry of Plenty ended on another trumpet
call and gave way to tinny music. Parsons, stirred to vague enthusiasm
by the bombardment of figures, took his pipe out of his mouth.

``The Ministry of Plenty\textquotesingle s certainly done a good job this
year,'' he said with a knowing shake of his head. ``By the way, Smith old
boy, I suppose you haven\textquotesingle t got any razor blades you can
let me have?''

``Not one,'' said Winston. ``I\textquotesingle ve been using the same blade
for six weeks myself.''

``Ah, well---just thought I\textquotesingle d ask you, old boy.''

``Sorry,'' said Winston.

The quacking voice from the next table, temporarily silenced during the
Ministry\textquotesingle s announcement, had started up again, as loud
as ever. For some reason Winston suddenly found himself thinking of Mrs.
Parsons, with her wispy hair and the dust in the creases of her face.
Within two years those children would be denouncing her to the Thought
Police. Mrs. Parsons would be vaporized. Syme would be vaporized.
Winston would be vaporized. O\textquotesingle Brien would be vaporized.
Parsons, on the other hand, would never be vaporized. The eyeless
creature with the quacking voice would never be vaporized. The little
beetlelike men who scuttled so nimbly through the labyrinthine corridors
of Ministries---they, too, would never be vaporized. And the girl with
dark hair, the girl from the Fiction Department---she would never be
vaporized either. It seemed to him that he knew instinctively who would
survive and who would perish, though just what it was that made for
survival, it was not easy to say.

At this moment he was dragged out of his reverie with a violent jerk.
The girl at the next table had turned partly round and was looking at
him. It was the girl with dark hair. She was looking at him in a
sidelong way, but with curious intensity. The instant that she caught
his eye she looked away again.

The sweat started out on Winston\textquotesingle s backbone. A horrible
pang of terror went through him. It was gone almost at once, but it left
a sort of nagging uneasiness behind. Why was she watching him? Why did
she keep following him about? Unfortunately he could not remember
whether she had already been at that table when he arrived, or had come
there afterwards. But yesterday, at any rate, during the Two Minutes
Hate, she had sat immediately behind him when there was no apparent need
to do so. Quite likely her real object had been to listen to him and
make sure whether he was shouting loudly enough.

His earlier thought returned to him: probably she was not actually a
member of the Thought Police, but then it was precisely the amateur spy
who was the greatest danger of all. He did not know how long she had
been looking at him, but perhaps for as much as five minutes, and it was
possible that his features had not been perfectly under control. It was
terribly dangerous to let your thoughts wander when you were in any
public place or within range of a telescreen. The smallest thing could
give you away. A nervous tic, an unconscious look of anxiety, a habit of
muttering to yourself---anything that carried with it the suggestion of
abnormality, of having something to hide. In any case, to wear an
improper expression on your face (to look incredulous when a victory was
announced, for example) was itself a punishable offense. There was even
a word for it in Newspeak: \emph{facecrime}, it was called.

The girl had turned her back on him again. Perhaps after all she was not
really following him about; perhaps it was coincidence that she had sat
so close to him two days running. His cigarette had gone out, and he
laid it carefully on the edge of the table. He would finish smoking it
after work, if he could keep the tobacco in it. Quite likely the person
at the next table was a spy of the Thought Police, and quite likely he
would be in the cellars of the Ministry of Love within three days, but a
cigarette end must not be wasted. Syme had folded up his strip of paper
and stowed it away in his pocket. Parsons had begun talking again.

``Did I ever tell you, old boy,'' he said, chuckling round the stem of his
pipe, ``about the time when those two nippers of mine set fire to the old
market-woman\textquotesingle s skirt because they saw her wrapping up
sausages in a poster of B.B.? Sneaked up behind her and set fire to it
with a box of matches. Burned her quite badly, I believe. Little
beggars, eh? But keen as mustard! That\textquotesingle s a first-rate
training they give them in the Spies nowadays---better than in my day,
even. What d\textquotesingle you think\textquotesingle s the latest
thing they\textquotesingle ve served them out with? Ear trumpets for
listening through keyholes! My little girl brought one home the other
night---tried it out on our sitting room door, and reckoned she could
hear twice as much as with her ear to the hole. Of course
it\textquotesingle s only a toy, mind you. Still, gives
\textquotesingle em the right idea, eh?''

At this moment the telescreen let out a piercing whistle. It was the
signal to return to work. All three men sprang to their feet to join in
the struggle round the lifts, and the remaining tobacco fell out of
Winston\textquotesingle s cigarette.


\section{VI}\label{vi}

Winston was writing in his diary:

\begin{quotation}
It was three years ago. It was on a dark evening, in a narrow side
street near one of the big railway stations. She was standing near a
doorway in the wall, under a street lamp that hardly gave any light. She
had a young face, painted very thick. It was really the paint that
appealed to me, the whiteness of it, like a mask, and the bright red
lips. Party women never paint their faces. There was nobody else in the
street, and no telescreens. She said two dollars. I---
\end{quotation}

For the moment it was too difficult to go on. He shut his eyes and
pressed his fingers against them, trying to squeeze out the vision that
kept recurring. He had an almost overwhelming temptation to shout a
string of filthy words at the top of his voice. Or to bang his head
against the wall, to kick over the table and hurl the inkpot through the
window---to do any violent or noisy or painful thing that might black
out the memory that was tormenting him.

Your worst enemy, he reflected, was your own nervous system. At any
moment the tension inside you was liable to translate itself into some
visible symptom. He thought of a man whom he had passed in the street a
few weeks back: a quite ordinary-looking man, a Party member, aged
thirty-five or forty, tallish and thin, carrying a brief case. They were
a few meters apart when the left side of the man\textquotesingle s face
was suddenly contorted by a sort of spasm. It happened again just as
they were passing one another: it was only a twitch, a quiver, rapid as
the clicking of a camera shutter, but obviously habitual. He remembered
thinking at the time: that poor devil is done for. And what was
frightening was that the action was quite possibly unconscious. The most
deadly danger of all was talking in your sleep. There was no way of
guarding against that, so far as he could see.

He drew in his breath and went on writing:

\begin{quotation}
I went with her through the doorway and across a backyard into a
basement kitchen. There was a bed against the wall, and a lamp on the
table, turned down very low. She---
\end{quotation}

His teeth were set on edge. He would have liked to spit. Simultaneously
with the woman in the basement kitchen he thought of Katharine, his
wife. Winston was married---had been married, at any rate; probably he
still was married, for so far as he knew his wife was not dead. He
seemed to breathe again the warm stuffy odor of the basement kitchen, an
odor compounded of bugs and dirty clothes and villainous cheap scent,
but nevertheless alluring, because no woman of the Party ever used
scent, or could be imagined as doing so. Only the proles used scent. In
his mind the smell of it was inextricably mixed up with fornication.

When he had gone with that woman it had been his first lapse in two
years or thereabouts. Consorting with prostitutes was forbidden, of
course, but it was one of those rules that you could occasionally nerve
yourself to break. It was dangerous, but it was not a life-and-death
matter. To be caught with a prostitute might mean five years in a
forced-Plabor camp: not more, if you had committed no other offense. And
it was easy enough, provided that you could avoid being caught in the
act. The poorer quarters swarmed with women who were ready to sell
themselves. Some could even be purchased for a bottle of gin, which the
proles were not supposed to drink. Tacitly the Party was even inclined
to encourage prostitution, as an outlet for instincts which could not be
altogether suppressed. Mere debauchery did not matter very much, so long
as it was furtive and joyless, and only involved the women of a
submerged and despised class. The unforgivable crime was promiscuity
between Party members. But---though this was one of the crimes that the
accused in the great purges invariably confessed to---it was difficult
to imagine any such thing actually happening.

The aim of the Party was not merely to prevent men and women from
forming loyalties which it might not be able to control. Its real,
undeclared purpose was to remove all pleasure from the sexual act. Not
love so much as eroticism was the enemy, inside marriage as well as
outside it. All marriages between Party members had to be approved by a
committee appointed for the purpose, and---though the principle was
never clearly stated---permission was always refused if the couple
concerned gave the impression of being physically attracted to one
another. The only recognized purpose of marriage was to beget children
for the service of the Party. Sexual intercourse was to be looked on as
a slightly disgusting minor operation, like having an enema. This again
was never put into plain words, but in an indirect way it was rubbed
into every Party member from childhood onwards. There were even
organizations such as the Junior Anti-Sex League which advocated
complete celibacy for both sexes. All children were to be begotten by
artificial insemination (\emph{artsem}, it was called in Newspeak) and
brought up in public institutions. This, Winston was aware, was not
meant altogether seriously, but somehow it fitted in with the general
ideology of the Party. The Party was trying to kill the sex instinct,
or, if it could not be killed, then to distort it and dirty it. He did
not know why this was so, but it seemed natural that it should be so.
And so far as the women were concerned, the Party\textquotesingle s
efforts were largely successful.

He thought again of Katharine. It must be nine, ten---nearly eleven
years since they had parted. It was curious how seldom he thought of
her. For days at a time he was capable of forgetting that he had ever
been married. They had only been together for about fifteen months. The
Party did not permit divorce, but it rather encouraged separation in
cases where there were no children.

Katharine was a tall, fair-haired girl, very straight, with splendid
movements. She had a bold, aquiline face, a face that one might have
called noble until one discovered that there was as nearly as possible
nothing behind it. Very early in their married life he had
decided---though perhaps it was only that he knew her more intimately
than he knew most people---that she had without exception the most
stupid, vulgar, empty mind that he had ever encountered. She had not a
thought in her head that was not a slogan, and there was no imbecility,
absolutely none, that she was not capable of swallowing if the Party
handed it out to her. ``The human sound track'' he nicknamed her in his
own mind. Yet he could have endured living with her if it had not been
for just one thing---sex.

As soon as he touched her she seemed to wince and stiffen. To embrace
her was like embracing a jointed wooden image. And what was strange was
that even when she was clasping him against her he had the feeling that
she was simultaneously pushing him away with all her strength. The
rigidity of her muscles managed to convey that impression. She would lie
there with shut eyes, neither resisting nor co-operating, but
\emph{submitting}. It was extraordinarily embarrassing and, after a
while, horrible. But even then he could have borne living with her if it
had been agreed that they should remain celibate. But curiously enough
it was Katharine who refused this. They must, she said, produce a child
if they could. So the performance continued to happen, once a week quite
regularly, whenever it was not impossible. She used even to remind him
of it in the morning, as something which had to be done that evening and
which must not be forgotten. She had two names for it. One was ``making a
baby,'' and the other was ``our duty to the Party'' (yes, she had actually
used that phrase). Quite soon he grew to have a feeling of positive
dread when the appointed day came round. But luckily no child appeared,
and in the end she agreed to give up trying, and soon afterwards they
parted.

Winston sighed inaudibly. He picked up his pen again and wrote:

\begin{quotation}
She threw herself down on the bed, and at once, without any kind
of preliminary, in the most coarse, horrible way you can imagine, pulled
up her skirt. I---
\end{quotation}

He saw himself standing there in the dim lamplight, with the smell of
bugs and cheap scent in his nostrils, and in his heart a feeling of
defeat and resentment which even at that moment was mixed up with the
thought of Katharine\textquotesingle s white body, frozen forever by the
hypnotic power of the Party. Why did it always have to be like this? Why
could he not have a woman of his own instead of these filthy scuffles at
intervals of years? But a real love affair was an almost unthinkable
event. The women of the Party were all alike. Chastity was as deeply
ingrained in them as Party loyalty. By careful early conditioning, by
games and cold water, by the rubbish that was dinned into them at school
and in the Spies and the Youth League, by lectures, parades, songs,
slogans, and martial music, the natural feeling had been driven out of
them. His reason told him that there must be exceptions, but his heart
did not believe it. They were all impregnable, as the Party intended
that they should be. And what he wanted, more even than to be loved, was
to break down that wall of virtue, even if it were only once in his
whole life. The sexual act, successfully performed, was rebellion.
Desire was thoughtcrime. Even to have awakened Katharine, if he could
have achieved it, would have been like a seduction, although she was his
wife.

But the rest of the story had got to be written down. He wrote:

\begin{quotation}
I turned up the lamp. When I saw her in the light---
\end{quotation}

After the darkness, the feeble light of the paraffin lamp had seemed
very bright. For the first time he could see the woman properly. He had
taken a step toward her and then halted, full of lust and terror. He was
painfully conscious of the risk he had taken in coming here. It was
perfectly possible that the patrols would catch him on the way out; for
that matter they might be waiting outside the door at this moment. If he
went away without even doing what he had come here to do---!

It had got to be written down, it had got to be confessed. What he had
suddenly seen in the lamplight was that the woman was \emph{old}. The
paint was plastered so thick on her face that it looked as though it
might crack like a cardboard mask. There were streaks of white in her
hair; but the truly dreadful detail was that her mouth had fallen a
little open, revealing nothing except a cavernous blackness. She had no
teeth at all.

He wrote hurriedly, in scrabbling handwriting:

\begin{quotation}
When I saw her in the light she was quite an old woman, fifty
years old at least. But I went ahead and did it just the same.
\end{quotation}

He pressed his fingers against his eyelids again. He had written it down
at last, but it made no difference. The therapy had not worked. The urge
to shout filthy words at the top of his voice was as strong as ever.


\section{VII}\label{vii}

\emph{If there is hope} [wrote Winston] \emph{it lies in the proles}.

\sectionbreak

If there was hope, it \emph{must} lie in the proles, because only there,
in those swarming disregarded masses, eighty-five per cent of the
population of Oceania, could the force to destroy the Party ever be
generated. The Party could not be overthrown from within. Its enemies,
if it had any enemies, had no way of coming together or even of
identifying one another. Even if the legendary Brotherhood existed, as
just possibly it might, it was inconceivable that its members could ever
assemble in larger numbers than twos and threes. Rebellion meant a look
in the eyes, an inflection of the voice; at the most, an occasional
whispered word. But the proles, if only they could somehow become
conscious of their own strength, would have no need to conspire. They
needed only to rise up and shake themselves like a horse shaking off
flies. If they chose they could blow the Party to pieces tomorrow
morning. Surely sooner or later it must occur to them to do it. And
yet---!

He remembered how once he had been walking down a crowded street when a
tremendous shout of hundreds of voices---women\textquotesingle s
voices---had burst from a side street a little way ahead. It was a great
formidable cry of anger and despair, a deep loud ``Oh-o-o-o-oh!'' that
went humming on like the reverberation of a bell. His heart had leapt.
It\textquotesingle s started! he had thought. A riot! The proles are
breaking loose at last! When he had reached the spot it was to see a mob
of two or three hundred women crowding round the stalls of a street
market, with faces as tragic as though they had been the doomed
passengers on a sinking ship. But at this moment the general despair
broke down into a multitude of individual quarrels. It appeared that one
of the stalls had been selling tin saucepans. They were wretched, flimsy
things, but cooking pots of any kind were always difficult to get. Now
the supply had unexpectedly given out. The successful women, bumped and
jostled by the rest, were trying to make off with their saucepans while
dozens of others clamored round the stall, accusing the stallkeeper of
favoritism and of having more saucepans somewhere in reserve. There was
a fresh outburst of yells. Two bloated women, one of them with her hair
coming down, had got hold of the same saucepan and were trying to tear
it out of one another\textquotesingle s hands. For a moment they were
both tugging, and then the handle came off. Winston watched them
disgustedly. And yet, just for a moment, what almost frightening power
had sounded in that cry from only a few hundred throats! Why was it that
they could never shout like that about anything that mattered?

He wrote:

\begin{quotation}
Until they become conscious they will never rebel, and until after
they have rebelled they cannot become conscious.
\end{quotation}

That, he reflected, might almost have been a transcription from one of
the Party textbooks. The Party claimed, of course, to have liberated the
proles from bondage. Before the Revolution they had been hideously
oppressed by the capitalists, they had been starved and flogged, women
had been forced to work in the coal mines (women still did work in the
coal mines, as a matter of fact), children had been sold into the
factories at the age of six. But simultaneously, true to the principles
of doublethink, the Party taught that the proles were natural inferiors
who must be kept in subjection, like animals, by the application of a
few simple rules. In reality very little was known about the proles. It
was not necessary to know much. So long as they continued to work and
breed, their other activities were without importance. Left to
themselves, like cattle turned loose upon the plains of Argentina, they
had reverted to a style of life that appeared to be natural to them, a
sort of ancestral pattern. They were born, they grew up in the gutters,
they went to work at twelve, they passed through a brief blossoming
period of beauty and sexual desire, they married at twenty, they were
middle-aged at thirty, they died, for the most part, at sixty. Heavy
physical work, the care of home and children, petty quarrels with
neighbors, films, football, beer, and, above all, gambling filled up the
horizon of their minds. To keep them in control was not difficult. A few
agents of the Thought Police moved always among them, spreading false
rumors and marking down and eliminating the few individuals who were
judged capable of becoming dangerous; but no attempt was made to
indoctrinate them with the ideology of the Party. It was not desirable
that the proles should have strong political feelings. All that was
required of them was a primitive patriotism which could be appealed to
whenever it was necessary to make them accept longer working hours or
shorter rations. And even when they became discontented, as they
sometimes did, their discontent led nowhere, because, being without
general ideas, they could only focus it on petty specific grievances.
The larger evils invariably escaped their notice. The great majority of
proles did not even have telescreens in their homes. Even the civil
police interfered with them very little. There was a vast amount of
criminality in London, a whole world-within-a-world of thieves, bandits,
prostitutes, drug peddlers, and racketeers of every description; but
since it all happened among the proles themselves, it was of no
importance. In all questions of morals they were allowed to follow their
ancestral code. The sexual puritanism of the Party was not imposed upon
them. Promiscuity went unpunished; divorce was permitted. For that
matter, even religious worship would have been permitted if the proles
had shown any sign of needing or wanting it. They were beneath
suspicion. As the Party slogan put it: ``Proles and animals are free.''

Winston reached down and cautiously scra\-tched his varicose ulcer. It had
begun itching again. The thing you invariably came back to was the
impossibility of knowing what life before the Revolution had really been
like. He took out of the drawer a copy of a children\textquotesingle s
history textbook which he had borrowed from Mrs. Parsons, and began
copying a passage into the diary:

\begin{quotation}
  In the old days [it ran], before the glorious Revolution, London was not
  the beautiful city that we know today. It was a dark, dirty, miserable
  place where hardly anybody had enough to eat and where hundreds and
  thousands of poor people had no boots on their feet and not even a roof to
  sleep under. Children no older than you are had to work twelve hours a day
  for cruel masters, who flogged them with whips if they worked too slowly
  and fed them on nothing but stale breadcrusts and water. But in among all
  this terrible poverty there were just a few great big beautiful houses
  that were lived in by rich men who had as many as thirty servants to look
  after them. These rich men were called capitalists. They were fat, ugly
  men with wicked faces, like the one in the picture on the opposite page.
  You can see that he is dressed in a long black coat which was called a
  frock coat, and a queer, shiny hat shaped like a stovepipe, which was
  called a top hat. This was the uniform of the capitalists, and no one else
  was allowed to wear it. The capitalists owned everything in the world, and
  everyone else was their slave. They owned all the land, all the houses,
  all the factories, and all the money. If anyone disobeyed them they could
  throw him into prison, or they could take his job away and starve him to
  death. When any ordinary person spoke to a capitalist he had to cringe and
  bow to him, and take off his cap and address him as ``Sir.'' The chief of
  all the capitalists was called the King, and---
\end{quotation}

But he knew the rest of the catalogue. There would be mention of the
bishops in their lawn sleeves, the judges in their ermine robes, the
pillory, the stocks, the treadmill, the
cat-o\textquotesingle-nine-tails, the Lord Mayor\textquotesingle s
Banquet, and the practice of kissing the Pope\textquotesingle s toe.
There was also something called the \emph{jus primae noctis}, which
would probably not be mentioned in a textbook for children. It was the
law by which every capitalist had the right to sleep with any woman
working in one of his factories.

How could you tell how much of it was lies? It \emph{might} be true that
the average human being was better off now than he had been before the
Revolution. The only evidence to the contrary was the mute protest in
your own bones, the instinctive feeling that the conditions you lived in
were intolerable and that at some other time they must have been
different. It struck him that the truly characteristic thing about
modern life was not its cruelty and insecurity, but simply its bareness,
its dinginess, its listlessness. Life, if you looked about you, bore no
resemblance not only to the lies that streamed out of the telescreens,
but even to the ideals that the Party was trying to achieve. Great areas
of it, even for a Party member, were neutral and nonpolitical, a matter
of slogging through dreary jobs, fighting for a place on the Tube,
darning a worn-out sock, cadging a saccharine tablet, saving a cigarette
end. The ideal set up by the Party was something huge, terrible, and
glittering---a world of steel and concrete, of monstrous machines and
terrifying weapons---a nation of warriors and fanatics, marching forward
in perfect unity, all thinking the same thoughts and shouting the same
slogans, perpetually working, fighting, triumphing, persecuting---three
hundred million people all with the same face. The reality was decaying,
dingy cities, where underfed people shuffled to and fro in leaky shoes,
in patched-up nineteenth-century houses that smelt always of cabbage and
bad lavatories. He seemed to see a vision of London, vast and ruinous,
city of a million dust bins, and mixed up with it was a picture of Mrs.
Parsons, a woman with lined face and wispy hair, fiddling helplessly
with a blocked wastepipe.

He reached down and scratched his ankle again. Day and night the
telescreens bruised your ears with statistics proving that people today
had more food, more clothes, better houses, better recreations---that
they lived longer, worked shorter hours, were bigger, healthier,
stronger, happier, more intelligent, better educated, than the people of
fifty years ago. Not a word of it could ever be proved or disproved. The
Party claimed, for example, that today forty per cent of adult proles
were literate; before the Revolution, it was said, the number had only
been fifteen per cent. The Party claimed that the infant mortality rate
was now only a hundred and sixty per thousand, whereas before the
Revolution it had been three hundred---and so it went on. It was like a
single equation with two unknowns. It might very well be that literally
every word in the history books, even the things that one accepted
without question, was pure fantasy. For all he knew there might never
have been any such law as the \emph{jus primae noctis}, or any such
creature as a capitalist, or any such garment as a top hat.

Everything faded into mist. The past was erased, the erasure was
forgotten, the lie became truth. Just once in his life he had
possessed---\emph{after} the event: that was what counted---concrete,
unmistakable evidence of an act of falsification. He had held it between
his fingers for as long as thirty seconds. In 1973, it must have
been---at any rate, it was at about the time when he and Katharine had
parted. But the really relevant date was seven or eight years earlier.

The story really began in the middle Sixties, the period of the great
purges in which the original leaders of the Revolution were wiped out
once and for all. By 1970 none of them was left, except Big Brother
himself. All the rest had by that time been exposed as traitors and
counter-revolutionaries. Goldstein had fled and was hiding, no one knew
where, and of the others, a few had simply disappeared, while the
majority had been executed after spectacular public trials at which they
made confession of their crimes. Among the last survivors were three men
named Jones, Aaronson, and Rutherford. It must have been in 1965 that
these three had been arrested. As often happened, they had vanished for
a year or more, so that one did not know whether they were alive or
dead, and then had suddenly been brought forth to incriminate themselves
in the usual way. They had confessed to intelligence with the enemy (at
that date, too, the enemy was Eurasia), embezzlement of public funds,
the murder of various trusted Party members, intrigues against the
leadership of Big Brother which had started long before the Revolution
happened, and acts of sabotage causing the death of hundreds of
thousands of people. After confessing to these things they had been
pardoned, reinstated in the Party, and given posts which were in fact
sinecures but which sounded important. All three had written long,
abject articles in the \emph{Times}, analyzing the reasons for their
defection and promising to make amends.

Some time after their release Winston had actually seen all three of
them in the Chestnut Tree Café. He remembered the sort of terrified
fascination with which he had watched them out of the corner of his eye.
They were men far older than himself, relics of the ancient world,
almost the last great figures left over from the heroic early days of
the Party. The glamor of the underground struggle and the civil war
still faintly clung to them. He had the feeling, though already at that
time facts and dates were growing blurry, that he had known their names
years earlier than he had known that of Big Brother. But also they were
out-Plaws, enemies, untouchables, doomed with absolute certainty to
extinction within a year or two. No one who had once fallen into the
hands of the Thought Police ever escaped in the end. They were corpses
waiting to be sent back to the grave.

There was no one at any of the tables nearest to them. It was not wise
even to be seen in the neighborhood of such people. They were sitting in
silence before glasses of the gin flavored with cloves which was the
speciality of the café. Of the three, it was Rutherford whose appearance
had most impressed Winston. Rutherford had once been a famous
caricaturist, whose brutal cartoons had helped to inflame popular
opinion before and during the Revolution. Even now, at long intervals,
his cartoons were appearing in the \emph{Times}. They were simply an
imitation of his earlier manner, and curiously lifeless and
unconvincing. Always they were a rehashing of the ancient themes---slum
tenements, starving children, street battles, capitalists in top
hats---even on the barricades the capitalists still seemed to cling to
their top hats---an endless, hopeless effort to get back into the past.
He was a monstrous man, with a mane of greasy gray hair, his face
pouched and seamed, with protuberant lips. At one time he must have been
immensely strong; now his great body was sagging, sloping, bulging,
falling away in every direction. He seemed to be breaking up before
one\textquotesingle s eyes, like a mountain crumbling.

It was the lonely hour of fifteen. Winston could not now remember how he
had come to be in the café at such a time. The place was almost empty. A
tinny music was trickling from the telescreens. The three men sat in
their corner almost motionless, never speaking. Uncommanded, the waiter
brought fresh glasses of gin. There was a chessboard on the table beside
them, with the pieces set out, but no game started. And then, for
perhaps half a minute in all, something happened to the telescreens. The
tune that they were playing changed, and the tone of the music changed
too. There came into it---but it was something hard to describe. It was
a peculiar, cracked, braying, jeering note; in his mind Winston called
it a yellow note. And then a voice from the telescreen was singing:

\begin{quotation}
  \noindent ``Under the spreading chestnut tree\\
  I sold you and you sold me:\\
  There lie they, and here lie we\\
  Under the spreading chestnut tree.''
\end{quotation}

The three men never stirred. But when Winston glanced again at
Rutherford\textquotesingle s ruinous face, he saw that his eyes were
full of tears. And for the first time he noticed, with a kind of inward
shudder, and yet not knowing \emph{at what} he shuddered, that both
Aaronson and Rutherford had broken noses.

A little later all three were rearrested. It appeared that they had
engaged in fresh conspiracies from the very moment of their release. At
their second trial they confessed to all their old crimes over again,
with a whole string of new ones. They were executed, and their fate was
recorded in the Party histories, a warning to posterity. About five
years after this, in 1973, Winston was unrolling a wad of documents
which had just flopped out of the pneumatic tube onto his desk when he
came on a fragment of paper which had evidently been slipped in among
the others and then forgotten. The instant he had flattened it out he
saw its significance. It was a half-page torn out of the \emph{Times} of
about ten years earlier---the top half of the page, so that it included
the date---and it contained a photograph of the delegates at some Party
function in New York. Prominent in the middle of the group were Jones,
Aaronson, and Rutherford. There was no mistaking them; in any case their
names were in the caption at the bottom.

The point was that at both trials all three men had confessed that on
that date they had been on Eurasian soil. They had flown from a secret
airfield in Canada to a rendezvous somewhere in Siberia, and had
conferred with members of the Eurasian General Staff, to whom they had
betrayed important military secrets. The date had stuck in
Winston\textquotesingle s memory because it chanced to be Midsummer Day;
but the whole story must be on record in countless other places as well.
There was only one possible conclusion: the confessions were lies.

Of course, this was not in itself a discovery. Even at that time Winston
had not imagined that the people who were wiped out in the purges had
actually committed the crimes that they were accused of. But this was
concrete evidence; it was a fragment of the abolished past, like a
fossil bone which turns up in the wrong stratum and destroys a
geological theory. It was enough to blow the Party to atoms, if in some
way it could have been published to the world and its significance made
known.

He had gone straight on working. As soon as he saw what the photograph
was, and what it meant, he had covered it up with another sheet of
paper. Luckily, when he unrolled it, it had been upside-down from the
point of view of the telescreen.

He took his scribbling pad on his knee and pushed back his chair, so as
to get as far away from the telescreen as possible. To keep your face
expressionless was not difficult, and even your breathing could be
controlled, with an effort; but you could not control the beating of
your heart, and the telescreen was quite delicate enough to pick it up.
He let what he judged to be ten minutes go by, tormented all the while
by the fear that some accident---a sudden draught blowing across his
desk, for instance---would betray him. Then, without uncovering it
again, he dropped the photograph into the memory hole, along with some
other waste papers. Within another minute, perhaps, it would have
crumbled into ashes.

That was ten---eleven years ago. Today, probably, he would have kept
that photograph. It was curious that the fact of having held it in his
fingers seemed to him to make a difference even now, when the photograph
itself, as well as the event it recorded, was only memory. Was the
Party\textquotesingle s hold upon the past less strong, he wondered,
because a piece of evidence which existed no longer \emph{had once}
existed?

But today, supposing that it could be somehow resurrected from its
ashes, the photograph might not even be evidence. Already, at the time
when he made his discovery, Oceania was no longer at war with Eurasia,
and it must have been to the agents of Eastasia that the three dead men
had betrayed their country. Since then there had been other
changes---two, three, he could not remember how many. Very likely the
confessions had been rewritten and rewritten until the original facts
and dates no longer had the smallest significance. The past not only
changed, but changed continuously. What most afflicted him with the
sense of nightmare was that he had never clearly understood \emph{why}
the huge imposture was undertaken. The immediate advantages of
falsifying the past were obvious, but the ultimate motive was
mysterious. He took up his pen again and wrote:

\begin{quotation}
I understand HOW: I do not understand WHY.
\end{quotation}

He wondered, as he had many times wondered before, whether he himself
was a lunatic. Perhaps a lunatic was simply a minority of one. At one
time it had been a sign of madness to believe that the earth goes round
the sun; today, to believe that the past is unalterable. He might be
\emph{alone} in holding that belief, and if alone, then a lunatic. But
the thought of being a lunatic did not greatly trouble him; the horror
was that he might also be wrong.

He picked up the children\textquotesingle s history book and looked at
the portrait of Big Brother which formed its frontispiece. The hypnotic
eyes gazed into his own. It was as though some huge force were pressing
down upon you---something that penetrated inside your skull, battering
against your brain, frightening you out of your beliefs, persuading you,
almost, to deny the evidence of your senses. In the end the Party would
announce that two and two made five, and you would have to believe it.
It was inevitable that they should make that claim sooner or later: the
logic of their position demanded it. Not merely the validity of
experience, but the very existence of external reality was tacitly
denied by their philosophy. The heresy of heresies was common sense. And
what was terrifying was not that they would kill you for thinking
otherwise, but that they might be right. For, after all, how do we know
that two and two make four? Or that the force of gravity works? Or that
the past is unchangeable? If both the past and the external world exist
only in the mind, and if the mind itself is controllable---what then?

But no! His courage seemed suddenly to stiffen of its own accord. The
face of O\textquotesingle Brien, not called up by any obvious
association, had floated into his mind. He knew, with more certainty
than before, that O\textquotesingle Brien was on his side. He was
writing the diary for O\textquotesingle Brien---\emph{to}
O\textquotesingle Brien; it was like an interminable letter which no one
would ever read, but which was addressed to a particular person and took
its color from that fact.

The Party told you to reject the evidence of your eyes and ears. It was
their final, most essential command. His heart sank as he thought of the
enormous power arrayed against him, the ease with which any Party
intellectual would overthrow him in debate, the subtle arguments which
he would not be able to understand, much less answer. And yet he was in
the right! They were wrong and he was right. The obvious, the silly, and
the true had got to be defended. Truisms are true, hold on to that! The
solid world exists, its laws do not change. Stones are hard, water is
wet, objects unsupported fall toward the earth\textquotesingle s center.
With the feeling that he was speaking to O\textquotesingle Brien, and
also that he was setting forth an important axiom, he wrote:

\begin{quotation}
Freedom is the freedom to say that two plus two make four. If that
is granted, all else follows.
\end{quotation}


\section{VIII}\label{viii}

From somewhere at the bottom of a passage the smell of roasting
coffee---real coffee, not Victory Coffee---came floating out into the
street. Winston paused involuntarily. For perhaps two seconds he was
back in the half-forgotten world of his childhood. Then a door banged,
seeming to cut off the smell as abruptly as though it had been a sound.

He had walked several kilometers over pavements, and his varicose ulcer
was throbbing. This was the second time in three weeks that he had
missed an evening at the Community Center: a rash act, since you could
be certain that the number of your attendances at the Center were
carefully checked. In principle a Party member had no spare time, and
was never alone except in bed. It was assumed that when he was not
working, eating, or sleeping he would be taking part in some kind of
communal recreations; to do anything that suggested a taste for
solitude, even to go for a walk by yourself, was always slightly
dangerous. There was a word for it in Newspeak: \emph{ownlife}, it was
called, meaning individualism and eccentricity. But this evening as he
came out of the Ministry the balminess of the April air had tempted him.
The sky was a warmer blue than he had seen it that year, and suddenly
the long, noisy evening at the Center, the boring, exhausting games, the
lectures, the creaking camaraderie oiled by gin, had seemed intolerable.
On impulse he had turned away from the bus stop and wandered off into
the labyrinth of London, first south, then east, then north again,
losing himself along unknown streets and hardly bothering in which
direction he was going.

``If there is hope,'' he had written in the diary, ``it lies in the
proles.'' The words kept coming back to him, statement of a mystical
truth and a palpable absurdity. He was somewhere in the vague,
brown-colored slums to the north and east of what had once been Saint
Pancras Station. He was walking up a cobbled street of little two-story
houses with battered doorways which gave straight on the pavement and
which were somehow curiously suggestive of rat holes. There were puddles
of filthy water here and there among the cobbles. In and out of the dark
doorways, and down narrow alleyways that branched off on either side,
people swarmed in astonishing numbers---girls in full bloom, with
crudely lipsticked mouths, and youths who chased the girls, and swollen
waddling women who showed you what the girls would be like in ten years
time, and old bent creatures shuffling along on splayed feet, and ragged
barefooted children who played in the puddles and then scattered at
angry yells from their mothers. Perhaps a quarter of the windows in the
street were broken and boarded up. Most of the people paid no attention
to Winston; a few eyed him with a sort of guarded curiosity. Two
monstrous women with brick-red forearms folded across their aprons were
talking outside a doorway. Winston caught scraps of conversation as he
approached.

```Yes,' I says to \textquotesingle er, `that\textquotesingle s all very
well,' I says. `But if you\textquotesingle d of been in my place
you\textquotesingle d of done the same as what I done. It\textquotesingle s
easy to criticize,' I says, `but you ain\textquotesingle t got the same
problems as what I got.'{}''

``Ah,'' said the other, ``that\textquotesingle s jest it.
That\textquotesingle s jest where it is.''

The strident voices stopped abruptly. The women studied him in hostile
silence as he went past. But it was not hostility, exactly; merely a
kind of wariness, a momentary stiffening, as at the passing of some
unfamiliar animal. The blue overalls of the Party could not be a common
sight in a street like this. Indeed, it was unwise to be seen in such
places, unless you had definite business there. The patrols might stop
you if you happened to run into them. ``May I see your papers, comrade?
What are you doing here? What time did you leave work? Is this your
usual way home?''---and so on and so forth. Not that there was any rule
against walking home by an unusual route, but it was enough to draw
attention to you if the Thought Police heard about it.

Suddenly the whole street was in commotion. There were yells of warning
from all sides. People were shooting into the doorways like rabbits. A
young woman leapt out of a doorway a little ahead of Winston, grabbed up
a tiny child playing in a puddle, whipped her apron round it, and leapt
back again, all in one movement. At the same instant a man in a
concertina-like black suit, who had emerged from a side alley, ran
toward Winston, pointing excitedly to the sky.

``Steamer!'' he yelled. ``Look out, guv\textquotesingle nor! Bang
over\textquotesingle ead! Lay down quick!''

``Steamer'' was a nickname which, for some reason, the proles applied to
rocket bombs. Winston promptly flung himself on his face. The proles
were nearly always right when they gave you a warning of this kind. They
seemed to possess some kind of instinct which told them several seconds
in advance when a rocket was coming, although the rockets supposedly
traveled faster than sound. Winston clasped his forearms about his head.
There was a roar that seemed to make the pavement heave; a shower of
light objects pattered onto his back. When he stood up he found that he
was covered with fragments of glass from the nearest window.

He walked on. The bomb had demolished a group of houses two hundred
meters up the street. A black plume of smoke hung in the sky, and below
it a cloud of plaster dust in which a crowd was already forming round
the ruins. There was a little pile of plaster lying on the pavement
ahead of him, and in the middle of it he could see a bright red streak.
When he got up to it he saw that it was a human hand severed at the
wrist. Apart from the bloody stump, the hand was so completely whitened
as to resemble a plaster cast.

He kicked the thing into the gutter, and then, to avoid the crowd,
turned down a side street to the right. Within three or four minutes he
was out of the area which the bomb had affected, and the sordid swarming
life of the streets was going on as though nothing had happened. It was
nearly twenty hours, and the drinking shops which the proles frequented
(``pubs,'' they called them) were choked with customers. From their grimy
swing doors, endlessly opening and shutting, there came forth a smell of
urine, sawdust, and sour beer. In an angle formed by a projecting house
front three men were standing very close together, the middle one of
them holding a folded-up newspaper which the other two were studying
over his shoulders. Even before he was near enough to make out the
expression on their faces, Winston could see absorption in every line of
their bodies. It was obviously some serious piece of news that they were
reading. He was a few paces away from them when suddenly the group broke
up and two of the men were in violent altercation. For a moment they
seemed almost on the point of blows.

``Can\textquotesingle t you bleeding well listen to what I say? I tell
you no number ending in seven ain\textquotesingle t won for over
fourteen months!''

``Yes it \textquotesingle as, then!''

``No, it \textquotesingle as not! Back \textquotesingle ome I got the
\textquotesingle ole lot of \textquotesingle em for over two years wrote
down on a piece of paper. I takes \textquotesingle em down
reg\textquotesingle lar as the clock. An\textquotesingle{} I tell you,
no number ending in seven---''

``Yes, a seven \emph{\textquotesingle as} won! I could pretty near tell
you the bleeding number. Four oh seven, it ended in. It were in
February---second week in February.''

``February your grandmother! I got it all down in black and white.
An\textquotesingle{} I tell you, no number---''

``Oh, pack it in!'' said the third man.

They were talking about the Lottery. Winston looked back when he had
gone thirty meters. They were still arguing, with vivid, passionate
faces. The Lottery, with its weekly pay-out of enormous prizes, was the
one public event to which the proles paid serious attention. It was
probable that there were some millions of proles for whom the Lottery
was the principal if not the only reason for remaining alive. It was
their delight, their folly, their anodyne, their intellectual stimulant.
Where the Lottery was concerned, even people who could barely read and
write seemed capable of intricate calculations and staggering feats of
memory. There was a whole tribe of men who made a living simply by
selling systems, forecasts, and lucky amulets. Winston had nothing to do
with the running of the Lottery, which was managed by the Ministry of
Plenty, but he was aware (indeed everyone in the Party was aware) that
the prizes were largely imaginary. Only small sums were actually paid
out, the winners of the big prizes being nonexistent persons. In the
absence of any real intercommunication between one part of Oceania and
another, this was not difficult to arrange.

But if there was hope, it lay in the proles. You had to cling on to
that. When you put it in words it sounded reasonable; it was when you
looked at the human beings passing you on the pavement that it became an
act of faith. The street into which he had turned ran downhill. He had a
feeling that he had been in this neighborhood before, and that there was
a main thoroughfare not far away. From somewhere ahead there came a din
of shouting voices. The street took a sharp turn and then ended in a
flight of steps which led down into a sunken alley where a few
stallkeepers were selling tired-looking vegetables. At this moment
Winston remembered where he was. The alley led out into the main street,
and down the next turning, not five minutes away, was the junk shop
where he had bought the blank book which was now his diary. And in a
small stationer\textquotesingle s shop not far away he had bought his
penholder and his bottle of ink.

He paused for a moment at the top of the steps. On the opposite side of
the alley there was a dingy little pub whose windows appeared to be
frosted over but in reality were merely coated with dust. A very old
man, bent but active, with white mustaches that bristled forward like
those of a prawn, pushed open the swing door and went in. As Winston
stood watching it occurred to him that the old man, who must be eighty
at the least, had already been middle-aged when the Revolution happened.
He and a few others like him were the last links that now existed with
the vanished world of capitalism. In the Party itself there were not
many people left whose ideas had been formed before the Revolution. The
older generation had mostly been wiped out in the great purges of the
Fifties and Sixties, and the few who survived had long ago been
terrified into complete intellectual surrender. If there was anyone
still alive who could give you a truthful account of conditions in the
early part of the century, it could only be a prole. Suddenly the
passage from the history book that he had copied into his diary came
back into Winston\textquotesingle s mind, and a lunatic impulse took
hold of him. He would go into the pub, he would scrape acquaintance with
that old man and question him. He would say to him: ``Tell me about your
life when you were a boy. What was it like in those days? Were things
better than they are now, or were they worse?''

Hurriedly, lest he should have time to become frightened, he descended
the steps and crossed the narrow street. It was madness, of course. As
usual, there was no definite rule against talking to proles and
frequenting their pubs, but it was far too unusual an action to pass
unnoticed. If the patrols appeared he might plead an attack of
faintness, but it was not likely that they would believe him. He pushed
open the door, and a hideous cheesy smell of sour beer hit him in the
face. As he entered, the din of voices dropped to about half its volume.
Behind his back he could feel everyone eyeing his blue overalls. A game
of darts which was going on at the other end of the room interrupted
itself for perhaps as much as thirty seconds. The old man whom he had
followed was standing at the bar, having some kind of altercation with
the barman, a large, stout, hook-nosed young man with enormous forearms.
A knot of others, standing round with glasses in their hands, were
watching the scene.

``I arst you civil enough, didn\textquotesingle t I?'' said the old man,
straightening his shoulders pugnaciously. ``You telling me you
ain\textquotesingle t got a pint mug in the \textquotesingle ole
bleeding boozer?''

``And what in hell\textquotesingle s name \emph{is} a pint?'' said the
barman, leaning forward with the tips of his fingers on the counter.

``\textquotesingle Ark at \textquotesingle im! Calls
\textquotesingle isself a barman and don\textquotesingle t know what a
pint is! Why, a pint\textquotesingle s the \textquotesingle alf of a
quart, and there\textquotesingle s four quarts to the gallon.
\textquotesingle Ave to teach you the A, B, C next.''

``Never heard of \textquotesingle em,'' said the barman shortly. ``Liter
and half liter---that\textquotesingle s all we serve.
There\textquotesingle s the glasses on the shelf in front of you.''

``I likes a pint,'' persisted the old man. ``You could \textquotesingle a
drawed me off a pint easy enough. We didn\textquotesingle t
\textquotesingle ave these bleeding liters when I was a young man.''

``When you were a young man we were all living in the treetops,'' said the
barman, with a glance at the other customers.

There was a shout of laughter, and the uneasiness caused by
Winston\textquotesingle s entry seemed to disappear. The old
man\textquotesingle s white-stubbled face had flushed pink. He turned
away, muttering to himself, and bumped into Winston. Winston caught him
gently by the arm.

``May I offer you a drink?'' he said.

``You\textquotesingle re a gent,'' said the other, straightening his
shoulders again. He appeared not to have noticed
Winston\textquotesingle s blue overalls. ``Pint!'' he added aggressively
to the barman. ``Pint of wallop.''

The barman swished two half-liters of dark-brown beer into thick glasses
which he had rinsed in a bucket under the counter. Beer was the only
drink you could get in prole pubs. The proles were supposed not to drink
gin, though in practice they could get hold of it easily enough. The
game of darts was in full swing again, and the knot of men at the bar
had begun talking about lottery tickets. Winston\textquotesingle s
presence was forgotten for a moment. There was a deal table under the
window where he and the old man could talk without fear of being
overheard. It was horribly dangerous, but at any rate there was no
telescreen in the room, a point he had made sure of as soon as he came
in.

``\textquotesingle E could \textquotesingle a drawed me off a pint,''
grumbled the old man as he settled down behind his glass. ``A
\textquotesingle alf liter ain\textquotesingle t enough. It
don\textquotesingle t satisfy. And a \textquotesingle ole
liter\textquotesingle s too much. It starts my bladder running. Let
alone the price.''

``You must have seen great changes since you were a young man,'' said
Winston tentatively.

The old man\textquotesingle s pale blue eyes moved from the darts board
to the bar, and from the bar to the door of the Gents, as though it were
in the barroom that he expected the changes to have occurred.

``The beer was better,'' he said finally. ``And cheaper! When I was a young
man, mild beer---wallop, we used to call it---was fourpence a pint. That
was before the war, of course.''

``Which war was that?'' said Winston.

``It\textquotesingle s all wars,'' said the old man vaguely. He took up
his glass, and his shoulders straightened again.
``\textquotesingle Ere\textquotesingle s wishing you the very best of
\textquotesingle ealth!''

In his lean throat the sharp-pointed Adam\textquotesingle s apple made a
surprisingly rapid up-and-down movement, and the beer vanished. Winston
went to the bar and came back with two more half-liters. The old man
appeared to have forgotten his prejudice against drinking a full liter.

``You are very much older than I am,'' said Winston. ``You must have been a
grown man before I was born. You can remember what it was like in the
old days, before the Revolution. People of my age don\textquotesingle t
really know anything about those times. We can only read about them in
books, and what it says in the books may not be true. I should like your
opinion on that. The history books say that life before the Revolution
was completely different from what it is now. There was the most
terrible oppression, injustice, poverty---worse than anything we can
imagine. Here in London, the great mass of the people never had enough
to eat from birth to death. Half of them hadn\textquotesingle t even
boots on their feet. They worked twelve hours a day, they left school at
nine, they slept ten in a room. And at the same time there were a very
few people, only a few thousands---the capitalists, they were
called---who were rich and powerful. They owned everything that there
was to own. They lived in great gorgeous houses with thirty servants,
they rode about in motor cars and four-horse carriages, they drank
champagne, they wore top hats---''

The old man brightened suddenly.

``Top \textquotesingle ats!'' he said. ``Funny you should mention
\textquotesingle em. The same thing come into my \textquotesingle ead
only yesterday, I donno why. I was jest thinking, I
ain\textquotesingle t seen a top \textquotesingle at in years. Gorn
right out, they \textquotesingle ave. The last time I wore one was at my
sister-in-law\textquotesingle s funeral. And that was---well, I
couldn\textquotesingle t give you the date, but it must
\textquotesingle a been fifty year ago. Of course it was only
\textquotesingle ired for the occasion, you understand.''

``It isn\textquotesingle t very important about the top hats,'' said
Winston patiently. ``The point is, these capitalists---they and a few
lawyers and priests and so forth who lived on them---were the lords of
the earth. Everything existed for their benefit. You---the ordinary
people, the workers---were their slaves. They could do what they liked
with you. They could ship you off to Canada like cattle. They could
sleep with your daughters if they chose. They could order you to be
flogged with something called a cat-o\textquotesingle-nine-tails. You
had to take your cap off when you passed them. Every capitalist went
about with a gang of lackeys who---''

The old man brightened again.

``Lackeys!'' he said. ``Now there\textquotesingle s a word I
ain\textquotesingle t \textquotesingle eard since ever so long. Lackeys!
That reg\textquotesingle lar takes me back, that does. I recollect---oh,
donkey\textquotesingle s years ago---I used to sometimes go to
\textquotesingle Yde Park of a Sunday afternoon to \textquotesingle ear
the blokes making speeches. Salvation Army, Roman Catholics, Jews,
Indians---all sorts, there was. And there was one bloke---well, I
couldn\textquotesingle t give you \textquotesingle is name, but a real
powerful speaker, \textquotesingle e was. \textquotesingle E
didn\textquotesingle t \textquotesingle alf give it \textquotesingle em!
`Lackeys!' \textquotesingle e says.
`Lackeys of the bourgeoisie! Flunkies of the ruling
class!' Parasites---that was another of them. And
\textquotesingle yenas---\textquotesingle e def\textquotesingle nitely
called \textquotesingle em \textquotesingle yenas. Of course
\textquotesingle e was referring to the Labour Party, you understand.''

Winston had the feeling that they were talking at cross purposes.

``What I really wanted to know was this,'' he said. ``Do you feel that you
have more freedom now than you had in those days? Are you treated more
like a human being? In the old days, the rich people, the people at the
top---''

``The \textquotesingle Ouse of Lords,'' put in the old man reminiscently.

``The House of Lords, if you like. What I am asking is, were these people
able to treat you as an inferior, simply because they were rich and you
were poor? Is it a fact, for instance, that you had to call them
`Sir' and take off your cap when you
passed them?''

The old man appeared to think deeply. He drank off about a quarter of
his beer before answering.

``Yes,'' he said. ``They liked you to touch your cap to
\textquotesingle em. It showed respect, like. I didn\textquotesingle t
agree with it, myself, but I done it often enough. Had to, as you might
say.''

``And was it usual---I\textquotesingle m only quoting what
I\textquotesingle ve read in history books---was it usual for these
people and their servants to push you off the pavement into the gutter?''

``One of em pushed me once,'' said the old man. ``I recollect it as if it
was yesterday. It was Boat Race night---terrible rowdy they used to get
on Boat Race night---and I bumps into a young bloke on Shaftesbury
Avenue. Quite the gent, \textquotesingle e was---dress shirt, top
\textquotesingle at, black overcoat. \textquotesingle E was kind of
zigzagging across the pavement, and I bumps into \textquotesingle im
accidental-like. \textquotesingle E says, `Why
can\textquotesingle t you look where you\textquotesingle re
going?' \textquotesingle e says. I says,
`Ju think you\textquotesingle ve bought the bleeding
pavement?' \textquotesingle E says,
`I\textquotesingle ll twist your bloody
\textquotesingle ead off if you get fresh with me.' I
says, `You\textquotesingle re drunk.
I\textquotesingle ll give you in charge in \textquotesingle alf a
minute,' I says. \textquotesingle An if
you\textquotesingle ll believe me, \textquotesingle e puts
\textquotesingle is \textquotesingle and on my chest and gives me a
shove as pretty near sent me under the wheels of a bus. Well, I was
young in them days, and I was going to \textquotesingle ave fetched
\textquotesingle im one, only---''

A sense of helplessness took hold of Winston. The old
man\textquotesingle s memory was nothing but a rubbish heap of details.
One could question him all day without getting any real information. The
Party histories might still be true, after a fashion; they might even be
completely true. He made a last attempt.

``Perhaps I have not made myself clear,'' he said. ``What
I\textquotesingle m trying to say is this. You have been alive a very
long time; you lived half your life before the Revolution. In 1925, for
instance, you were already grown up. Would you say, from what you can
remember, that life in 1925 was better than it is now, or worse? If you
could choose, would you prefer to live then or now?''

The old man looked meditatively at the darts board. He finished up his
beer, more slowly than before. When he spoke it was with a tolerant,
philosophic air, as though the beer had mellowed him.

``I know what you expect me to say,'' he said. ``You expect me to say as
I\textquotesingle d sooner be young again. Most people\textquotesingle d
say they\textquotesingle d sooner be young, if you arst
\textquotesingle em. You got your \textquotesingle ealth and strength
when you\textquotesingle re young. When you get to my time of life you
ain\textquotesingle t never well. I suffer something wicked from my
feet, and my bladder\textquotesingle s jest terrible. Six and seven
times a night it \textquotesingle as me out of bed. On the other
\textquotesingle and there\textquotesingle s great advantages in being a
old man. You ain\textquotesingle t got the same worries. No truck with
women, and that\textquotesingle s a great thing. I ain\textquotesingle t
\textquotesingle ad a woman for near on thirty year, if
you\textquotesingle d credit it. Nor wanted to, what\textquotesingle s
more.''

Winston sat back against the window sill. It was no use going on. He was
about to buy some more beer when the old man suddenly got up and
shuffled rapidly into the stinking urinal at the side of the room. The
extra half-liter was already working on him. Winston sat for a minute or
two gazing at his empty glass, and hardly noticed when his feet carried
him out into the street again. Within twenty years at the most, he
reflected, the huge and simple question, ``Was life better before the
Revolution than it is now?'' would have ceased once and for all to be
answerable. But in effect it was unanswerable even now, since the few
scattered survivors from the ancient world were incapable of comparing
one age with another. They remembered a million useless things, a
quarrel with a workmate, a hunt for a lost bicycle pump, the expression
on a long-dead sister\textquotesingle s face, the swirls of dust on a
windy morning seventy years ago; but all the relevant facts were outside
the range of their vision. They were like the ant, which can see small
objects but not large ones. And when memory failed and written records
were falsified---when that happened, the claim of the Party to have
improved the conditions of human life had got to be accepted, because
there did not exist, and never again could exist, any standard against
which it could be tested.

At this moment his train of thought stopped abruptly. He halted and
looked up. He was in a narrow street, with a few dark little shops
interspersed among dwelling houses. Immediately above his head there
hung three discolored metal balls which looked as if they had once been
gilded. He seemed to know the place. Of course! He was standing outside
the junk shop where he had bought the diary.

A twinge of fear went through him. It had been a sufficiently rash act
to buy the book in the beginning, and he had sworn never to come near
the place again. And yet the instant that he allowed his thoughts to
wander, his feet had brought him back here of their own accord. It was
precisely against suicidal impulses of this kind that he had hoped to
guard himself by opening the diary. At the same time he noticed that
although it was nearly twenty-one hours the shop was still open. With
the feeling that he would be less conspicuous inside than hanging about
on the pavement, he stepped through the doorway. If questioned, he could
plausibly say that he was trying to buy razor blades.

The proprietor had just lighted a hanging oil lamp which gave off an
unclean but friendly smell. He was a man of perhaps sixty, frail and
bowed, with a long, benevolent nose, and mild eyes distorted by thick
spectacles. His hair was almost white, but his eyebrows were bushy and
still black. His spectacles, his gentle, fussy movements, and the fact
that he was wearing an aged jacket of black velvet, gave him a vague air
of intellectuality, as though he had been some kind of literary man, or
perhaps a musician. His voice was soft, as though faded, and his accent
less debased than that of the majority of proles.

``I recognized you on the pavement,'' he said immediately.
``You\textquotesingle re the gentleman that bought the young
lady\textquotesingle s keepsake album. That was a beautiful bit of
paper, that was. Cream laid, it used to be called.
There\textquotesingle s been no paper like that made for---oh, I dare
say fifty years.'' He peered at Winston over the top of his spectacles.
``Is there anything special I can do for you? Or did you just want to
look round?''

``I was passing,'' said Winston vaguely. ``I just looked in. I
don\textquotesingle t want anything in particular.''

``It\textquotesingle s just as well,'' said the other, ``because I
don\textquotesingle t suppose I could have satisfied you.'' He made an
apologetic gesture with his soft-palmed hand. ``You see how it is; an
empty shop, you might say. Between you and me, the antique
trade\textquotesingle s just about finished. No demand any longer, and
no stock either. Furniture, china, glass---it\textquotesingle s all been
broken up by degrees. And of course the metal stuff\textquotesingle s
mostly been melted down. I haven\textquotesingle t seen a brass
candlestick in years.''

The tiny interior of the shop was in fact uncomfortably full, but there
was almost nothing in it of the slightest value. The floor space was
very restricted, because all round the walls were stacked innumerable
dusty picture frames. In the window there were trays of nuts and bolts,
worn-out chisels, penknives with broken blades, tarnished watches that
did not even pretend to be in going order, and other miscellaneous
rubbish. Only on a small table in the corner was there a litter of odds
and ends---lacquered snuffboxes, agate brooches, and the like---which
looked as though they might include something interesting. As Winston
wandered toward the table his eye was caught by a round, smooth thing
that gleamed softly in the lamplight, and he picked it up.

It was a heavy lump of glass, curved on one side, flat on the other,
making almost a hemisphere. There was a peculiar softness, as of
rainwater, in both the color and the texture of the glass. At the heart
of it, magnified by the curved surface, there was a strange, pink,
convoluted object that recalled a rose or a sea anemone.

``What is it?'' said Winston, fascinated.

``That\textquotesingle s coral, that is,'' said the old man. ``It must have
come from the Indian Ocean. They used to kind of embed it in the glass.
That wasn\textquotesingle t made less than a hundred years ago. More, by
the look of it.''

``It\textquotesingle s a beautiful thing,'' said Winston.

``It is a beautiful thing,'' said the other appreciatively. ``But
there\textquotesingle s not many that\textquotesingle d say so
nowadays.'' He coughed. ``Now, if it so happened that you wanted to buy
it, that\textquotesingle d cost you four dollars. I can remember when a
thing like that would have fetched eight pounds, and eight pounds
was---well, I can\textquotesingle t work it out, but it was a lot of
money. But who cares about genuine antiques nowadays---even the few
that\textquotesingle s left?''

Winston immediately paid over the four dollars and slid the coveted
thing into his pocket. What appealed to him about it was not so much its
beauty as the air it seemed to possess of belonging to an age quite
different from the present one. The soft, rainwatery glass was not like
any glass that he had ever seen. The thing was doubly attractive because
of its apparent uselessness, though he could guess that it must once
have been intended as a paperweight. It was very heavy in his pocket,
but fortunately it did not make much of a bulge. It was a queer thing,
even a compromising thing, for a Party member to have in his possession.
Anything old, and for that matter anything beautiful, was always vaguely
suspect. The old man had grown noticeably more cheerful after receiving
the four dollars. Winston realized that he would have accepted three or
even two.

``There\textquotesingle s another room upstairs that you might care to
take a look at,'' he said. ``There\textquotesingle s not much in it. Just
a few pieces. We\textquotesingle ll do with a light if
we\textquotesingle re going upstairs.''

He lit another lamp and, with bowed back, led the way slowly up the
steep and worn stairs and along a tiny passage, into a room which did
not give on the street but looked out on a cobbled yard and a forest of
chimney pots. Winston noticed that the furniture was still arranged as
though the room were meant to be lived in. There was a strip of carpet
on the floor, a picture or two on the walls, and a deep, slatternly
armchair drawn up to the fireplace. An old-fashioned glass clock with a
twelve-hour face was ticking away on the mantlepiece. Under the window,
and occupying nearly a quarter of the room, was an enormous bed with the
mattress still on it.

``We lived here till my wife died,'' said the old man half apologetically.
``I\textquotesingle m selling the furniture off by little and little. Now
that\textquotesingle s a beautiful mahogany bed, or at least it would be
if you could get the bugs out of it. But I dare say
you\textquotesingle d find it a little bit cumbersome.''

He was holding the lamp high up, so as to illumine the whole room, and
in the warm dim light the place looked curiously inviting. The thought
flitted through Winston\textquotesingle s mind that it would probably be
quite easy to rent the room for a few dollars a week, if he dared to
take the risk. It was a wild, impossible notion, to be abandoned as soon
as thought of; but the room had awakened in him a sort of nostalgia, a
sort of ancestral memory. It seemed to him that he knew exactly what it
felt like to sit in a room like this, in an armchair beside an open fire
with your feet in the fender and a kettle on the hob, utterly alone,
utterly secure, with nobody watching you, no voice pursuing you, no
sound except the singing of the kettle and the friendly ticking of the
clock.

``There\textquotesingle s no telescreen!'' he could not help murmuring.

``Ah,'' said the old man, ``I never had one of those things. Too expensive.
And I never seemed to feel the need of it, somehow. Now
that\textquotesingle s a nice gateleg table in the corner there. Though
of course you\textquotesingle d have to put new hinges on it if you
wanted to use the flaps.''

There was a small bookcase in the other corner, and Winston had already
gravitated toward it. It contained nothing but rubbish. The hunting-down
and destruction of books had been done with the same thoroughness in the
prole quarters as everywhere else. It was very unlikely that there
existed anywhere in Oceania a copy of a book printed earlier than 1960.
The old man, still carrying the lamp, was standing in front of a picture
in a rosewood frame which hung on the other side of the fireplace,
opposite the bed.

``Now, if you happen to be interested in old prints at all---'' he began
delicately.

Winston came across to examine the picture. It was a steel engraving of
an oval building with rectangular windows, and a small tower in front.
There was a railing running round the building, and at the rear end
there was what appeared to be a statue. Winston gazed at it for some
moments. It seemed vaguely familiar, though he did not remember the
statue.

``The frame\textquotesingle s fixed to the wall,'' said the old man, ``but
I could unscrew it for you, I dare say.''

``I know that building,'' said Winston finally. ``It\textquotesingle s a
ruin now. It\textquotesingle s in the middle of the street outside the
Palace of Justice.''

``That\textquotesingle s right. Outside the Law Courts. It was bombed
in---oh, many years ago. It was a church at one time. St.
Clement\textquotesingle s Dane, its name was.'' He smiled apologetically,
as though conscious of saying something slightly ridiculous, and added:
``\emph{Oranges and lemons, say the bells of St.
Clement\textquotesingle s!}''

``What\textquotesingle s that?'' said Winston.

``Oh---\emph{Oranges and lemons, say the bells of St.
Clement\textquotesingle s}. That was a rhyme we had when I was a little
boy. How it goes on I don\textquotesingle t remember, but I do know it
ended up, \emph{Here comes a candle to light you to bed, Here comes a
chopper to chop off your head}. It was a kind of a dance. They held out
their arms for you to pass under, and when they came to \emph{Here comes
a chopper to chop off your head} they brought their arms down and caught
you. It was just names of churches. All the London churches were in
it---all the principal ones, that is.''

Winston wondered vaguely to what century the church belonged. It was
always difficult to determine the age of a London building. Anything
large and impressive, if it was reasonably new in appearance, was
automatically claimed as having been built since the Revolution, while
anything that was obviously of earlier date was ascribed to some dim
period called the Middle Ages. The centuries of capitalism were held to
have produced nothing of any value. One could not learn history from
architecture any more than one could learn it from books. Statues,
inscriptions, memorial stones, the names of streets---anything that
might throw light upon the past had been systematically altered.

``I never knew it had been a church,'' he said.

``There\textquotesingle s a lot of them left, really,'' said the old man,
``though they\textquotesingle ve been put to other uses. Now, how did
that rhyme go? Ah! I\textquotesingle ve got it!

\begin{quotation}
  Oranges and lemons, say the bells of St.
  Clement\textquotesingle s,\\
  You owe me three farthings, say the bells of St. Martin\textquotesingle
  s---
\end{quotation}

there, now, that\textquotesingle s as far as I can get. A farthing, that
was a small copper coin, looked something like a cent.''

``Where was St. Martin\textquotesingle s?'' said Winston.

``St. Martin\textquotesingle s? That\textquotesingle s still standing.
It\textquotesingle s in Victory Square, alongside the picture gallery. A
building with a kind of a triangular porch and pillars in front, and a
big flight of steps.''

Winston knew the place well. It was a museum used for propaganda
displays of various kinds---scale models of rocket bombs and Floating
Fortresses, waxwork tableaux illustrating enemy atrocities, and the
like.

``St. Martin\textquotesingle s in the Fields it used to be called,''
supplemented the old man, ``though I don\textquotesingle t recollect any
fields anywhere in those parts.''

Winston did not buy the picture. It would have been an even more
incongruous possession than the glass paperweight, and impossible to
carry home, unless it were taken out of its frame. But he lingered for
some minutes more, talking to the old man, whose name, he discovered,
was not Weeks---as one might have gathered from the inscription over the
shopfront---but Charrington. Mr. Charrington, it seemed, was a widower
aged sixty-three and had inhabited this shop for thirty years.
Throughout that time he had been intending to alter the name over the
window, but had never quite got to the point of doing it. All the while
that they were talking the half-remembered rhyme kept running through
Winston\textquotesingle s head: \emph{Oranges and lemons, say the bells
of St. Clement\textquotesingle s, You owe me three farthings, say the
bells of St. Martin\textquotesingle s!} It was curious, but when you
said it to yourself you had the illusion of actually hearing bells, the
bells of a lost London that still existed somewhere or other, disguised
and forgotten. From one ghostly steeple after another he seemed to hear
them pealing forth. Yet so far as he could remember he had never in real
life heard church bells ringing.

He got away from Mr. Charrington and went down the stairs alone, so as
not to let the old man see him reconnoitering the street before stepping
out of the door. He had already made up his mind that after a suitable
interval---a month, say---he would take the risk of visiting the shop
again. It was perhaps not more dangerous than shirking an evening at the
Center. The serious piece of folly had been to come back here in the
first place, after buying the diary and without knowing whether the
proprietor of the shop could be trusted. However---!

Yes, he thought again, he would come back. He would buy further scraps
of beautiful rubbish. He would buy the engraving of St.
Clement\textquotesingle s Dane, take it out of its frame, and carry it
home concealed under the jacket of his overalls. He would drag the rest
of that poem out of Mr. Charrington\textquotesingle s memory. Even the
lunatic project of renting the room upstairs flashed momentarily through
his mind again. For perhaps five seconds exaltation made him careless,
and he stepped out onto the pavement without so much as a preliminary
glance through the window. He had even started humming to an improvised
tune---

\begin{quotation}
\noindent Oranges and lemons, say the bells of St.
Clement\textquotesingle s,\\
You owe me three farthings, say the---\\
\end{quotation}

Suddenly his heart seemed to turn to ice and his bowels to water. A
figure in blue overalls was coming down the pavement, not ten meters
away. It was the girl from the Fiction Department, the girl with dark
hair. The light was failing, but there was no difficulty in recognizing
her. She looked him straight in the face, then walked quickly on as
though she had not seen him.

For a few seconds Winston was too paralyzed to move. Then he turned to
the right and walked heavily away, not noticing for the moment that he
was going in the wrong direction. At any rate, one question was settled.
There was no doubting any longer that the girl was spying on him. She
must have followed him here, because it was not credible that by pure
chance she should have happened to be walking on the same evening up the
same obscure back street, kilometers distant from any quarter where
Party members lived. It was too great a coincidence. Whether she was
really an agent of the Thought Police, or simply an amateur spy actuated
by officiousness, hardly mattered. It was enough that she was watching
him. Probably she had seen him go into the pub as well.

It was an effort to walk. The lump of glass in his pocket banged against
his thigh at each step, and he was half minded to take it out and throw
it away. The worst thing was the pain in his belly. For a couple of
minutes he had the feeling that he would die if he did not reach a
lavatory soon. But there would be no public lavatories in a quarter like
this. Then the spasm passed, leaving a dull ache behind.

The street was a blind alley. Winston halted, stood for several seconds
wondering vaguely what to do, then turned round and began to retrace his
steps. As he turned it occurred to him that the girl had only passed him
three minutes ago and that by running he could probably catch up with
her. He could keep on her track till they were in some quiet place, and
then smash her skull in with a cobblestone. The piece of glass in his
pocket would be heavy enough for the job. But he abandoned the idea
immediately, because even the thought of making any physical effort was
unbearable. He could not run, he could not strike a blow. Besides, she
was young and lusty and would defend herself. He thought also of
hurrying to the Community Center and staying there till the place
closed, so as to establish a partial alibi for the evening. But that too
was impossible. A deadly lassitude had taken hold of him. All he wanted
was to get home quickly and then sit down and be quiet.

It was after twenty-two hours when he got back to the flat. The lights
would be switched off at the main at twenty-three thirty. He went into
the kitchen and swallowed nearly a teacupful of Victory Gin. Then he
went to the table in the alcove, sat down, and took the diary out of the
drawer. But he did not open it at once. From the telescreen a brassy
female voice was squalling a patriotic song. He sat staring at the
marbled cover of the book, trying without success to shut the voice out
of his consciousness.

It was at night that they came for you, always at night. The proper
thing was to kill yourself before they got you. Undoubtedly some people
did so. Many of the disappearances were actually suicides. But it needed
desperate courage to kill yourself in a world where firearms, or any
quick and certain poison, were completely unprocurable. He thought with
a kind of astonishment of the biological uselessness of pain and fear,
the treachery of the human body which always freezes into inertia at
exactly the moment when a special effort is needed. He might have
silenced the dark-haired girl if only he had acted quickly enough; but
precisely because of the extremity of his danger he had lost the power
to act. It struck him that in moments of crisis one is never fighting
against an external enemy but always against one\textquotesingle s own
body. Even now, in spite of the gin, the dull ache in his belly made
consecutive thought impossible. And it is the same, he perceived, in all
seemingly heroic or tragic situations. On the battlefield, in the
torture chamber, on a sinking ship, the issues that you are fighting for
are always forgotten, because the body swells up until it fills the
universe, and even when you are not paralyzed by fright or screaming
with pain, life is a moment-to-moment struggle against hunger or cold or
sleeplessness, against a sour stomach or an aching tooth.

He opened the diary. It was important to write something down. The woman
on the telescreen had started a new song. Her voice seemed to stick into
his brain like jagged splinters of glass. He tried to think of
O\textquotesingle Brien, for whom, or to whom, the diary was written,
but instead he began thinking of the things that would happen to him
after the Thought Police took him away. It would not matter if they
killed you at once. To be killed was what you expected. But before death
(nobody spoke of such things, yet everybody knew of them) there was the
routine of confession that had to be gone through: the groveling on the
floor and screaming for mercy, the crack of broken bones, the smashed
teeth and bloody clots of hair. Why did you have to endure it, since the
end was always the same? Why was it not possible to cut a few days or
weeks out of your life? Nobody ever escaped detection, and nobody ever
failed to confess. When once you had succumbed to thoughtcrime it was
certain that by a given date you would be dead. Why then did that
horror, which altered nothing, have to lie embedded in future time?

He tried with a little more success than before to summon up the image
of O\textquotesingle Brien. ``We shall meet in the place where there is
no darkness,'' O\textquotesingle Brien had said to him. He knew what it
meant, or thought he knew. The place where there is no darkness was the
imagined future, which one would never see, but which, by foreknowledge,
one could mystically share in. But with the voice from the telescreen
nagging at his ears he could not follow the train of thought further. He
put a cigarette in his mouth. Half the tobacco promptly fell out on to
his tongue, a bitter dust which was difficult to spit out again. The
face of Big Brother swam into his mind, displacing that of
O\textquotesingle Brien. Just as he had done a few days earlier, he slid
a coin out of his pocket and looked at it. The face gazed up at him,
heavy, calm, protecting, but what kind of smile was hidden beneath the
dark mustache? Like a leaden knell the words came back at him:

\headline{
  WAR IS PEACE\\
  FREEDOM IS SLAVERY\\
  IGNORANCE IS STRENGTH. }


\clearpage
\part{TWO}\label{two}

\section{I}

It was the middle of the morning, and Winston had left his cubicle to go
to the lavatory.

A solitary figure was coming toward him from the other end of the long,
brightly lit corridor. It was the girl with dark hair. Four days had
gone past since the evening when he had run into her outside the junk
shop. As she came nearer he saw that her right arm was in a sling, not
noticeable at a distance because it was of the same color as her
overalls. Probably she had crushed her hand while swinging round one of
the big kaleidoscopes on which the plots of novels were ``roughed in.'' It
was a common accident in the Fiction Department.

They were perhaps four meters apart when the girl stumbled and fell
almost flat on her face. A sharp cry of pain was wrung out of her. She
must have fallen right on the injured arm. Winston stopped short. The
girl had risen to her knees. Her face had turned a milky yellow color
against which her mouth stood out redder than ever. Her eyes were fixed
on his, with an appealing expression that looked more like fear than
pain.

A curious emotion stirred in Winston\textquotesingle s heart. In front
of him was an enemy who was trying to kill him; in front of him, also,
was a human creature, in pain and perhaps with a broken bone. Already he
had instinctively started forward to help her. In the moment when he had
seen her fall on the bandaged arm, it had been as though he felt the
pain in his own body.

``You\textquotesingle re hurt?'' he said.

``It\textquotesingle s nothing. My arm. It\textquotesingle ll be all
right in a second.''

She spoke as though her heart were fluttering. She had certainly turned
very pale.

``You haven\textquotesingle t broken anything?''

``No, I\textquotesingle m all right. It hurt for a moment,
that\textquotesingle s all.''

She held out her free hand to him, and he helped her up. She had
regained some of her color, and appeared very much better.

``It\textquotesingle s nothing,'' she repeated shortly. ``I only gave my
wrist a bit of a bang. Thanks, comrade!''

And with that she walked on in the direction in which she had been
going, as briskly as though it had really been nothing. The whole
incident could not have taken as much as half a minute. Not to let
one\textquotesingle s feelings appear in one\textquotesingle s face was
a habit that had acquired the status of an instinct, and in any case
they had been standing straight in front of a telescreen when the thing
happened. Nevertheless it had been very difficult not to betray a
momentary surprise, for in the two or three seconds while he was helping
her up the girl had slipped something into his hand. There was no
question that she had done it intentionally. It was something small and
flat. As he passed through the lavatory door he transferred it to his
pocket and felt it with the tips of his fingers. It was a scrap of paper
folded into a square.

While he stood at the urinal he managed, with a little more fingering,
to get it unfolded. Obviously there must be a message of some kind
written on it. For a moment he was tempted to take it into one of the
water closets and read it at once. But that would be shocking folly, as
he well knew. There was no place where you could be more certain that
the telescreens were watched continuously.

He went back to his cubicle, sat down, threw the fragment of paper
casually among the other papers on the desk, put on his spectacles and
hitched the speakwrite toward him. ``Five minutes,'' he told himself,
``five minutes at the very least!'' His heart bumped in his breast with
frightening loudness. Fortunately the piece of work he was engaged on
was mere routine, the rectification of a long list of figures, not
needing close attention.

Whatever was written on the paper, it must have some kind of political
meaning. So far as he could see there were two possibilities. One, much
the more likely, was that the girl was an agent of the Thought Police,
just as he had feared. He did not know why the Thought Police should
choose to deliver their messages in such a fashion, but perhaps they had
their reasons. The thing that was written on the paper might be a
threat, a summons, an order to commit suicide, a trap of some
description. But there was another, wilder possibility that kept raising
its head, though he tried vainly to suppress it. This was, that the
message did not come from the Thought Police at all, but from some kind
of underground organization. Perhaps the Brotherhood existed after all!
Perhaps the girl was part of it! No doubt the idea was absurd, but it
had sprung into his mind in the very instant of feeling the scrap of
paper in his hand. It was not till a couple of minutes later that the
other, more probable explanation had occurred to him. And even now,
though his intellect told him that the message probably meant
death---still, that was not what he believed, and the unreasonable hope
persisted, and his heart banged, and it was with difficulty that he kept
his voice from trembling as he murmured his figures into the speakwrite.

He rolled up the completed bundle of work and slid it into the pneumatic
tube. Eight minutes had gone by. He readjusted his spectacles on his
nose, sighed, and drew the next batch of work toward him, with the scrap
of paper on top of it. He flattened it out. On it was written, in a
large unformed handwriting:

\begin{quotation}
I love you.
\end{quotation}


For several seconds he was too stunned even to throw the incriminating
thing into the memory hole. When he did so, although he knew very well
the danger of showing too much interest, he could not resist reading it
once again, just to make sure that the words were really there.

For the rest of the morning it was very difficult to work. What was even
worse than having to focus his mind on a series of niggling jobs was the
need to conceal his agitation from the telescreen. He felt as though a
fire were burning in his belly. Lunch in the hot, crowded, noise-filled
canteen was torment. He had hoped to be alone for a little while during
the lunch hour, but as bad luck would have it the imbecile Parsons
flopped down beside him, the tang of his sweat almost defeating the
tinny smell of stew, and kept up a stream of talk about the preparations
for Hate Week. He was particularly enthusiastic about a papier-mâché
model of Big Brother\textquotesingle s head, two meters wide, which was
being made for the occasion by his daughter\textquotesingle s troop of
Spies. The irritating thing was that in the racket of voices Winston
could hardly hear what Parsons was saying, and was constantly having to
ask for some fatuous remark to be repeated. Just once he caught a
glimpse of the girl, at a table with two other girls at the far end of
the room. She appeared not to have seen him, and he did not look in that
direction again.

The afternoon was more bearable. Immediately after lunch there arrived a
delicate, difficult piece of work which would take several hours and
necessitated putting everything else aside. It consisted in falsifying a
series of production reports of two years ago in such a way as to cast
discredit on a prominent member of the Inner Party who was now under a
cloud. This was the kind of thing that Winston was good at, and for more
than two hours he succeeded in shutting the girl out of his mind
altogether. Then the memory of her face came back, and with it a raging,
intolerable desire to be alone. Until he could be alone it was
impossible to think this new development out. Tonight was one of his
nights at the Community Center. He wolfed another tasteless meal in the
canteen, hurried off to the Center, took part in the solemn foolery of a
``discussion group,'' played two games of table tennis, swallowed several
glasses of gin, and sat for half an hour through a lecture entitled
``Ingsoc in relation to chess.'' His soul writhed with boredom, but for
once he had had no impulse to shirk his evening at the Center. At the
sight of the words \emph{I love you} the desire to stay alive had welled
up in him, and the taking of minor risks suddenly seemed stupid. It was
not till twenty-three hours, when he was home and in bed---in the
darkness, where you were safe even from the telescreen so long as you
kept silent---that he was able to think continuously.

It was a physical problem that had to be solved: how to get in touch
with the girl and arrange a meeting. He did not consider any longer the
possibility that she might be laying some kind of trap for him. He knew
that it was not so, because of her unmistakable agitation when she
handed him the note. Obviously she had been frightened out of her wits,
as well she might be. Nor did the idea of refusing her advances even
cross his mind. Only five nights ago he had contemplated smashing her
skull in with a cobblestone; but that was of no importance. He thought
of her naked, youthful body, as he had seen it in his dream. He had
imagined her a fool like all the rest of them, her head stuffed with
lies and hatred, her belly full of ice. A kind of fever seized him at
the thought that he might lose her, the white youthful body might slip
away from him! What he feared more than anything else was that she would
simply change her mind if he did not get in touch with her quickly. But
the physical difficulty of meeting was enormous. It was like trying to
make a move at chess when you were already mated. Whichever way you
turned, the telescreen faced you. Actually, all the possible ways of
communicating with her had occurred to him within five minutes of
reading the note; but now, with time to think, he went over them one by
one, as though laying out a row of instruments on a table.

Obviously the kind of encounter that had happened this morning could not
be repeated. If she had worked in the Records Department it might have
been comparatively simple, but he had only a very dim idea whereabouts
in the building the Fiction Department lay, and he had no pretext for
going there. If he had known where she lived, and at what time she left
work, he could have contrived to meet her somewhere on her way home; but
to try to follow her home was not safe, because it would mean loitering
about outside the Ministry, which was bound to be noticed. As for
sending a letter through the mails, it was out of the question. By a
routine that was not even secret, all letters were opened in transit.
Actually, few people ever wrote letters. For the messages that it was
occasionally necessary to send, there were printed postcards with long
lists of phrases, and you struck out the ones that were inapplicable. In
any case he did not know the girl\textquotesingle s name, let alone her
address. Finally he decided that the safest place was the canteen. If he
could get her at a table by herself, somewhere in the middle of the
room, not too near the telescreens, and with a sufficient buzz of
conversation all round---if these conditions endured for, say, thirty
seconds, it might be possible to exchange a few words.

For a week after this, life was like a restless dream. On the next day
she did not appear in the canteen until he was leaving it, the whistle
having already blown. Presumably she had been changed onto a later
shift. They passed each other without a glance. On the day after that
she was in the canteen at the usual time, but with three other girls and
immediately under a telescreen. Then for three dreadful days she did not
appear at all. His whole mind and body seemed to be afflicted with an
unbearable sensitivity, a sort of transparency, which made every
movement, every sound, every contact, every word that he had to speak or
listen to, an agony. Even in sleep he could not altogether escape from
her image. He did not touch the diary during those days. If there was
any relief, it was in his work, in which he could sometimes forget
himself for ten minutes at a stretch. He had absolutely no clue as to
what had happened to her. There was no inquiry he could make. She might
have been vaporized, she might have committed suicide, she might have
been transferred to the other end of Oceania---worst and likeliest of
all, she might simply have changed her mind and decided to avoid him.

The next day she reappeared. Her arm was out of the sling and she had a
band of sticking plaster round her wrist. The relief of seeing her was
so great that he could not resist staring directly at her for several
seconds. On the following day he very nearly succeeded in speaking to
her. When he came into the canteen she was sitting at a table well out
from the wall, and was quite alone. It was early, and the place was not
very full. The queue edged forward till Winston was almost at the
counter, then was held up for two minutes because someone in front was
complaining that he had not received his tablet of saccharine. But the
girl was still alone when Winston secured his tray and began to make for
her table. He walked casually toward her, his eyes searching for a place
at some table beyond her. She was perhaps three meters away from him.
Another two seconds would do it. Then a voice behind him called,
``Smith!'' He pretended not to hear. ``Smith!'' repeated the voice, more
loudly. It was no use. He turned round. A blond-headed, silly-faced
young man named Wilsher, whom he barely knew, was inviting him with a
smile to a vacant place at his table. It was not safe to refuse. After
having been recognized, he could not go and sit at a table with an
unattended girl. It was too noticeable. He sat down with a friendly
smile. The silly blond face beamed into his. Winston had a hallucination
of himself smashing a pickax right into the middle of it. The
girl\textquotesingle s table filled up a few minutes later.

But she must have seen him coming toward her, and perhaps she would take
the hint. Next day he took care to arrive early. Sure enough, she was at
a table in about the same place, and again alone. The person immediately
ahead of him in the queue was a small, swiftly moving, beetlelike man
with a flat face and tiny, suspicious eyes. As Winston turned away from
the counter with his tray, he saw that the little man was making
straight for the girl\textquotesingle s table. His hopes sank again.
There was a vacant place at a table further away, but something in the
little man\textquotesingle s appearance suggested that he would be
sufficiently attentive to his own comfort to choose the emptiest table.
With ice at his heart Winston followed. It was no use unless he could
get the girl alone. At this moment there was a tremendous crash. The
little man was sprawling on all fours, his tray had gone flying, two
streams of soup and coffee were flowing across the floor. He started to
his feet with a malignant glance at Winston, whom he evidently suspected
of having tripped him up. But it was all right. Five seconds later, with
a thundering heart, Winston was sitting at the girl\textquotesingle s
table.

He did not look at her. He unpacked his tray and promptly began eating.
It was all-important to speak at once, before anyone else came, but now
a terrible fear had taken possession of him. A week had gone by since
she had first approached. She would have changed her mind, she must have
changed her mind! It was impossible that this affair should end
successfully; such things did not happen in real life. He might have
flinched altogether from speaking if at this moment he had not seen
Ampleforth, the hairy-eared poet, wandering limply round the room with a
tray, looking for a place to sit down. In his vague way Ampleforth was
attached to Winston, and would certainly sit down at his table if he
caught sight of him. There was perhaps a minute in which to act. Both
Winston and the girl were eating steadily. The stuff they were eating
was a thin stew, actually a soup, of haricot beans. In a low murmur
Winston began speaking. Neither of them looked up; steadily they spooned
the watery stuff into their mouths, and between spoonfuls exchanged the
few necessary words in low expressionless voices.

``What time do you leave work?''

``Eighteen thirty.''

``Where can we meet?''

``Victory Square, near the monument.''

``It\textquotesingle s full of telescreens.''

``It doesn\textquotesingle t matter if there\textquotesingle s a crowd.''

``Any signal?''

``No. Don\textquotesingle t come up to me until you see me among a lot of
people. And don\textquotesingle t look at me. Just keep somewhere near
me.''

``What time?''

``Nineteen hours.''

``All right.''

Ampleforth failed to see Winston and sat down at another table. The girl
finished her lunch quickly and made off, while Winston stayed to smoke a
cigarette. They did not speak again, and, so far as it was possible for
two people sitting on opposite sides of the same table, they did not
look at one another.

Winston was in Victory Square before the appointed time. He wandered
round the base of the enormous fluted column, at the top of which Big
Brother\textquotesingle s statue gazed southward toward the skies where
he had vanquished the Eurasian airplanes (the Eastasian airplanes, it
had been, a few years ago) in the Battle of Airstrip One. In the street
in front of it there was a statue of a man on horseback which was
supposed to represent Oliver Cromwell. At five minutes past the hour the
girl had still not appeared. Again the terrible fear seized upon
Winston. She was not coming, she had changed her mind! He walked slowly
up to the north side of the square and got a sort of pale-colored
pleasure from identifying St. Martin\textquotesingle s church, whose
bells, when it had bells, had chimed ``You owe me three farthings.'' Then
he saw the girl standing at the base of the monument, reading or
pretending to read a poster which ran spirally up the column. It was not
safe to go near her until some more people had accumulated. There were
telescreens all round the pediment. But at this moment there was a din
of shouting and a zoom of heavy vehicles from somewhere to the left.
Suddenly everyone seemed to be running across the square. The girl
nipped nimbly round the lions at the base of the monument and joined in
the rush. Winston followed. As he ran, he gathered from some shouted
remarks that a convoy of Eurasian prisoners was passing.

Already a dense mass of people was blocking the south side of the
square. Winston, at normal times the kind of person who gravitates to
the outer edge of any kind of scrimmage, shoved, butted, squirmed his
way forward into the heart of the crowd. Soon he was within
arm\textquotesingle s length of the girl, but the way was blocked by an
enormous prole and an almost equally enormous woman, presumably his
wife, who seemed to form an impenetrable wall of flesh. Winston wriggled
himself sideways, and with a violent lunge managed to drive his shoulder
between them. For a moment it felt as though his entrails were being
ground to pulp between the two muscular hips, then he had broken
through, sweating a little. He was next to the girl. They were shoulder
to shoulder, both staring fixedly in front of them.

A long line of trucks, with wooden-faced guards armed with submachine
guns standing upright in each corner, was passing slowly down the
street. In the trucks little yellow men in shabby greenish uniforms were
squatting, jammed close together. Their sad Mongolian faces gazed out
over the sides of the trucks, utterly incurious. Occasionally when a
truck jolted there was a clank-clank of metal: all the prisoners were
wearing leg irons. Truckload after truckload of the sad faces passed.
Winston knew they were there, but he saw them only intermittently. The
girl\textquotesingle s shoulder, and her arm right down to the elbow,
were pressed against his. Her cheek was almost near enough for him to
feel its warmth. She had immediately taken charge of the situation, just
as she had done in the canteen. She began speaking in the same
expressionless voice as before, with lips barely moving, a mere murmur
easily drowned by the din of voices and the rumbling of the trucks.

``Can you hear me?''

``Yes.''

``Can you get Sunday afternoon off?''

``Yes.''

``Then listen carefully. You\textquotesingle ll have to remember this. Go
to Paddington Station---''

With a sort of military precision that astonished him, she outlined the
route that he was to follow. A half-hour railway journey; turn left
outside the station; two kilometers along the road; a gate with the top
bar missing; a path across a field; a grass-grown lane; a track between
bushes; a dead tree with moss on it. It was as though she had a map
inside her head. ``Can you remember all that?'' she murmured finally.

``Yes.''

``You turn left, then right, then left again. And the
gate\textquotesingle s got no top bar.''

``Yes. What time?''

``About fifteen. You may have to wait. I\textquotesingle ll get there by
another way. Are you sure you remember everything?''

``Yes.''

``Then get away from me as quick as you can.''

She need not have told him that. But for the moment they could not
extricate themselves from the crowd. The trucks were still filing past,
the people still insatiably gaping. At the start there had been a few
boos and hisses, but it came only from the Party members among the
crowd, and had soon stopped. The prevailing emotion was simply
curiosity. Foreigners, whether from Eurasia or from Eastasia, were a
kind of strange animal. One literally never saw them except in the guise
of prisoners, and even as prisoners one never got more than a momentary
glimpse of them. Nor did one know what became of them, apart from the
few who were hanged as war criminals; the others simply vanished,
presumably into forced-labor camps. The round Mongol faces had given way
to faces of a more European type, dirty, bearded, and exhausted. From
over scrubby cheekbones eyes looked into Winston\textquotesingle s,
sometimes with strange intensity, and flashed away again. The convoy was
drawing to an end. In the last truck he could see an aged man, his face
a mass of grizzled hair, standing upright with wrists crossed in front
of him, as though he were used to having them bound together. It was
almost time for Winston and the girl to part. But at the last moment,
while the crowd still hemmed them in, her hand felt for his and gave it
a fleeting squeeze.

It could not have been ten seconds, and yet it seemed a long time that
their hands were clasped together. He had time to learn every detail of
her hand. He explored the long fingers, the shapely nails, the
work-hardened palm with its row of calluses, the smooth flesh under the
wrist. Merely from feeling it he would have known it by sight. In the
same instant it occurred to him that he did not know what color the
girl\textquotesingle s eyes were. They were probably brown, but people
with dark hair sometimes had blue eyes. To turn his head and look at her
would have been inconceivable folly. With hands locked together,
invisible among the press of bodies, they stared steadily in front of
them, and instead of the eyes of the girl, the eyes of the aged prisoner
gazed mournfully at Winston out of nests of hair.


\section{II}\label{ii-1}

Winston picked his way up the lane through dappled light and shade,
stepping out into pools of gold wherever the boughs parted. Under the
trees to the left of them the ground was misty with bluebells. The air
seemed to kiss one\textquotesingle s skin. It was the second of May.
From somewhere deeper in the heart of the wood came the droning of ring
doves.

He was a bit early. There had been no difficulties about the journey,
and the girl was so evidently experienced that he was less frightened
than he would normally have been. Presumably she could be trusted to
find a safe place. In general you could not assume that you were much
safer in the country than in London. There were no telescreens, of
course, but there was always the danger of concealed microphones by
which your voice might be picked up and recognized; besides, it was not
easy to make a journey by yourself without attracting attention. For
distances of less than a hundred kilometers it was not necessary to get
your passport endorsed, but sometimes there were patrols hanging about
the railway stations, who examined the papers of any Party member they
found there and asked awkward questions. However, no patrols had
appeared, and on the walk from the station he had made sure by cautious
backward glances that he was not being followed. The train was full of
proles, in holiday mood because of the summery weather. The
wooden-seated carriage in which he traveled was filled to overflowing by
a single enormous family, ranging from a toothless great-grandmother to
a month-old baby, going out to spend an afternoon with ``in-laws'' in the
country, and, as they freely explained to Winston, to get hold of a
little black-market butter.

The lane widened, and in a minute he came to the foot-path she had told
him of, a mere cattle track which plunged between the bushes. He had no
watch, but it could not be fifteen yet. The bluebells were so thick
underfoot that it was impossible not to tread on them. He knelt down and
began picking some, partly to pass the time away, but also from a vague
idea that he would like to have a bunch of flowers to offer to the girl
when they met. He had got together a big bunch and was smelling their
faint sickly scent when a sound at his back froze him, the unmistakable
crackle of a foot on twigs. He went on picking bluebells. It was the
best thing to do. It might be the girl, or he might have been followed
after all. To look round was to show guilt. He picked another and
another. A hand fell lightly on his shoulder.

He looked up. It was the girl. She shook her head, evidently as a
warning that he must keep silent, then parted the bushes and quickly led
the way along the narrow track into the wood. Obviously she had been
that way before, for she dodged the boggy bits as though by habit.
Winston followed, still clasping his bunch of flowers. His first feeling
was relief, but as he watched the strong slender body moving in front of
him, with the scarlet sash that was just tight enough to bring out the
curve of her hips, the sense of his own inferiority was heavy upon him.
Even now it seemed quite likely that when she turned round and looked at
him she would draw back after all. The sweetness of the air and the
greenness of the leaves daunted him. Already, on the walk from the
station, the May sunshine had made him feel dirty and etiolated, a
creature of indoors, with the sooty dust of London in the pores of his
skin. It occurred to him that till now she had probably never seen him
in broad daylight in the open. They came to the fallen tree that she had
spoken of. The girl hopped over and forced apart the bushes, in which
there did not seem to be an opening. When Winston followed her, he found
that they were in a natural clearing, a tiny grassy knoll surrounded by
tall saplings that shut it in completely. The girl stopped and turned.

``Here we are,'' she said.

He was facing her at several paces\textquotesingle{} distance. As yet he
did not dare move nearer to her.

``I didn\textquotesingle t want to say anything in the lane,'' she went
on, ``in case there\textquotesingle s a mike hidden there. I
don\textquotesingle t suppose there is, but there could be.
There\textquotesingle s always the chance of one of those swine
recognizing your voice. We\textquotesingle re all right here.''

He still had not the courage to approach her. ``We\textquotesingle re all
right here?'' he repeated stupidly.

``Yes. Look at the trees.'' They were small ashes, which at some time had
been cut down and had sprouted up again into a forest of poles, none of
them thicker than one\textquotesingle s wrist. ``There\textquotesingle s
nothing big enough to hide a mike in. Besides, I\textquotesingle ve been
here before.''

They were only making conversation. He had managed to move closer to her
now. She stood before him very upright, with a smile on her face that
looked faintly ironical, as though she were wondering why he was so slow
to act. The bluebells had cascaded on to the ground. They seemed to have
fallen of their own accord. He took her hand.

``Would you believe,'' he said, ``that till this moment I
didn\textquotesingle t know what color your eyes were?'' They were brown,
he noted, a rather light shade of brown, with dark lashes. ``Now that
you\textquotesingle ve seen what I\textquotesingle m really like, can
you still bear to look at me?''

``Yes, easily.''

``I\textquotesingle m thirty-nine years old. I\textquotesingle ve got a
wife that I can\textquotesingle t get rid of. I\textquotesingle ve got
varicose veins. I\textquotesingle ve got five false teeth.''

``I couldn\textquotesingle t care less,'' said the girl.

The next moment, it was hard to say by whose act, she was in his arms.
At the beginning he had no feeling except sheer incredulity. The
youthful body was strained against his own, the mass of dark hair was
against his face, and yes! actually she had turned her face up and he
was kissing the wide red mouth. She had clasped her arms about his neck,
she was calling him darling, precious one, loved one. He had pulled her
down on to the ground, she was utterly unresisting, he could do what he
liked with her. But the truth was that he had no physical sensation
except that of mere contact. All he felt was incredulity and pride. He
was glad that this was happening, but he had no physical desire. It was
too soon, her youth and prettiness had frightened him, he was too much
used to living without women---he did not know the reason. The girl
picked herself up and pulled a bluebell out of her hair. She sat against
him, putting her arm round his waist.

``Never mind, dear. There\textquotesingle s no hurry.
We\textquotesingle ve got the whole afternoon. Isn\textquotesingle t
this a splendid hide-out? I found it when I got lost once on a community
hike. If anyone was coming you could hear them a hundred meters away.''

``What is your name?'' said Winston.

``Julia. I know yours. It\textquotesingle s Winston---Winston Smith.''

``How did you find that out?''

``I expect I\textquotesingle m better at finding things out than you are,
dear. Tell me, what did you think of me before that day I gave you the
note?''

He did not feel any temptation to tell lies to her. It was even a sort
of love offering to start off by telling the worst.

``I hated the sight of you,'' he said. ``I wanted to rape you and then
murder you afterwards. Two weeks ago I thought seriously of smashing
your head in with a cobblestone. If you really want to know, I imagined
that you had something to do with the Thought Police.''

The girl laughed delightedly, evidently taking this as a tribute to the
excellence of her disguise.

``Not the Thought Police! You didn\textquotesingle t honestly think
that?''

``Well, perhaps not exactly that. But from your general
appearance---merely because you\textquotesingle re young and fresh and
healthy, you understand---I thought that probably---''

``You thought I was a good Party member. Pure in word and deed. Banners,
processions, slogans, games, community hikes---all that stuff. And you
thought that if I had a quarter of a chance I\textquotesingle d denounce
you as a thought-criminal and get you killed off?''

``Yes, something of that kind. A great many young girls are like that,
you know.''

``It\textquotesingle s this bloody thing that does it,'' she said, ripping
off the scarlet sash of the Junior Anti-Sex League and flinging it onto
a bough. Then, as though touching her waist had reminded her of
something, she felt in the pocket of her overalls and produced a small
slab of chocolate. She broke it in half and gave one of the pieces to
Winston. Even before he had taken it he knew by the smell that it was
very unusual chocolate. It was dark and shiny, and was wrapped in silver
paper. Chocolate normally was dull-brown crumbly stuff that tasted, as
nearly as one could describe it, like the smoke of a rubbish fire. But
at some time or another he had tasted chocolate like the piece she had
given him. The first whiff of its scent had stirred up some memory which
he could not pin down, but which was powerful and troubling.

``Where did you get this stuff?'' he said.

``Black market,'' she said indifferently. ``Actually I am that sort of
girl, to look at. I\textquotesingle m good at games. I was a troop
leader in the Spies. I do voluntary work three evenings a week for the
Junior Anti-Sex League. Hours and hours I\textquotesingle ve spent
pasting their bloody rot all over London. I always carry one end of a
banner in the processions. I always look cheerful and I never shirk
anything. Always yell with the crowd, that\textquotesingle s what I say.
It\textquotesingle s the only way to be safe.''

The first fragment of chocolate had melted on Winston\textquotesingle s
tongue. The taste was delightful. But there was still that memory moving
round the edges of his consciousness, something strongly felt but not
reducible to definite shape, like an object seen out of the corner of
one\textquotesingle s eye. He pushed it away from him, aware only that
it was the memory of some action which he would have liked to undo but
could not.

``You are very young,'' he said. ``You are ten or fifteen years younger
than I am. What could you see to attract you in a man like me?''

``It was something in your face. I thought I\textquotesingle d take a
chance. I\textquotesingle m good at spotting people who
don\textquotesingle t belong. As soon as I saw you I knew you were
against \emph{them}.''

\emph{Them}, it appeared, meant the Party, and above all the Inner
Party, about whom she talked with an open jeering hatred which made
Winston feel uneasy, although he knew that they were safe here if they
could be safe anywhere. A thing that astonished him about her was the
coarseness of her language. Party members were supposed not to swear,
and Winston himself very seldom did swear, aloud, at any rate. Julia,
however, seemed unable to mention the Party, and especially the Inner
Party, without using the kind of words that you saw chalked up in
dripping alleyways. He did not dislike it. It was merely one symptom of
her revolt against the Party and all its ways, and somehow it seemed
natural and healthy, like the sneeze of a horse that smells bad hay.
They had left the clearing and were wandering again through the
checkered shade, with their arms round each other\textquotesingle s
waists whenever it was wide enough to walk two abreast. He noticed how
much softer her waist seemed to feel now that the sash was gone. They
did not speak above a whisper. Outside the clearing, Julia said, it was
better to go quietly. Presently they had reached the edge of the little
wood. She stopped him.

``Don\textquotesingle t go out into the open. There might be someone
watching. We\textquotesingle re all right if we keep behind the boughs.''

They were standing in the shade of hazel bushes. The sunlight, filtering
through innumerable leaves, was still hot on their faces. Winston looked
out into the field beyond, and underwent a curious, slow shock of
recognition. He knew it by sight. An old, close-bitten pasture, with a
foot-path wandering across it and a molehill here and there. In the
ragged hedge on the opposite side the boughs of the elm trees swayed
just perceptibly in the breeze, and their leaves stirred faintly in
dense masses like women\textquotesingle s hair. Surely somewhere near
by, but out of sight, there must be a stream with green pools where dace
were swimming.

``Isn\textquotesingle t there a stream somewhere near here?'' he
whispered.

``That\textquotesingle s right, there is a stream. It\textquotesingle s
at the edge of the next field, actually. There are fish in it, great big
ones. You can watch them lying in the pools under the willow trees,
waving their tails.''

``It\textquotesingle s the Golden Country---almost,'' he murmured.

``The Golden Country?''

``It\textquotesingle s nothing, really. A landscape I\textquotesingle ve
seen sometimes in a dream.''

``Look!'' whispered Julia.

A thrush had alighted on a bough not five meters away, almost at the
level of their faces. Perhaps it had not seen them. It was in the sun,
they in the shade. It spread out its wings, fitted them carefully into
place again, ducked its head for a moment, as though making a sort of
obeisance to the sun, and then began to pour forth a torrent of song. In
the afternoon hush the volume of sound was startling. Winston and Julia
clung together, fascinated. The music went on and on, minute after
minute, with astonishing variations, never once repeating itself, almost
as though the bird were deliberately showing off its virtuosity.
Sometimes it stopped for a few seconds, spread out and resettled its
wings, then swelled its speckled breast and again burst into song.
Winston watched it with a sort of vague reverence. For whom, for what,
was that bird singing? No mate, no rival was watching it. What made it
sit at the edge of the lonely wood and pour its music into nothingness?
He wondered whether after all there was a microphone hidden somewhere
near. He and Julia had only spoken in low whispers, and it would not
pick up what they had said, but it would pick up the thrush. Perhaps at
the other end of the instrument some small, beetlelike man was listening
intently---listening to \emph{that}. But by degrees the flood of music
drove all speculations out of his mind. It was as though it were a kind
of liquid stuff that poured all over him and got mixed up with the
sunlight that filtered through the leaves. He stopped thinking and
merely felt. The girl\textquotesingle s waist in the bend of his arm was
soft and warm. He pulled her round so that they were breast to breast;
her body seemed to melt into his. Wherever his hands moved it was all as
yielding as water. Their mouths clung together; it was quite different
from the hard kisses they had exchanged earlier. When they moved their
faces apart again both of them sighed deeply. The bird took fright and
fled with a clatter of wings.

Winston put his lips against her ear. ``\emph{Now},'' he whispered.

``Not here,'' she whispered back. ``Come back to the hide-out.
It\textquotesingle s safer.''

Quickly, with an occasional crackle of twigs, they threaded their way
back to the clearing. When they were once inside the ring of saplings
she turned and faced him. They were both breathing fast, but the smile
had reappeared round the corners of her mouth. She stood looking at him
for an instant, then felt at the zipper of her overalls. And, yes! It
was almost as in his dream. Almost as swiftly as he had imagined it, she
had torn her clothes off, and when she flung them aside it was with that
same magnificent gesture by which a whole civilization seemed to be
annihilated. Her body gleamed white in the sun. But for a moment he did
not look at her body; his eyes were anchored by the freckled face with
its faint, bold smile. He knelt down before her and took her hands in
his.

``Have you done this before?''

``Of course. Hundreds of times---well, scores of times, anyway.''

``With Party members?''

``Yes, always with Party members.''

``With members of the Inner Party?''

``Not with those swine, no. But there\textquotesingle s plenty that
\emph{would} if they got half a chance. They\textquotesingle re not so
holy as they make out.''

His heart leapt. Scores of times she had done it; he wished it had been
hundreds---thousands. Anything that hinted at corruption always filled
him with a wild hope. Who knew? Perhaps the Party was rotten under the
surface, its cult of strenuousness and self-denial simply a sham
concealing iniquity. If he could have infected the whole lot of them
with leprosy or syphilis, how gladly he would have done so! Anything to
rot, to weaken, to undermine! He pulled her down so that they were
kneeling face to face.

``Listen. The more men you\textquotesingle ve had, the more I love you.
Do you understand that?''

``Yes, perfectly.''

``I hate purity, I hate goodness. I don\textquotesingle t want any virtue
to exist anywhere. I want everyone to be corrupt to the bones.''

``Well then, I ought to suit you, dear. I\textquotesingle m corrupt to
the bones.''

``You like doing this? I don\textquotesingle t mean simply me; I mean the
thing in itself?''

``I adore it.''

That was above all what he wanted to hear. Not merely the love of one
person, but the animal instinct, the simple undifferentiated desire:
that was the force that would tear the Party to pieces. He pressed her
down upon the grass, among the fallen bluebells. This time there was no
difficulty. Presently the rising and falling of their breasts slowed to
normal speed, and in a sort of pleasant helplessness they fell apart.
The sun seemed to have grown hotter. They were both sleepy. He reached
out for the discarded overalls and pulled them partly over her. Almost
immediately they fell asleep and slept for about half an hour.

Winston woke first. He sat up and watched the freckled face, still
peacefully asleep, pillowed on the palm of her hand. Except for her
mouth, you could not call her beautiful. There was a line or two round
the eyes, if you looked closely. The short dark hair was extraordinarily
thick and soft. It occurred to him that he still did not know her
surname or where she lived.

The young, strong body, now helpless in sleep, awoke in him a pitying,
protecting feeling. But the mindless tenderness that he had felt under
the hazel tree, while the thrush was singing, had not quite come back.
He pulled the overalls aside and studied her smooth white flank. In the
old days, he thought, a man looked at a girl\textquotesingle s body and
saw that it was desirable, and that was the end of the story. But you
could not have pure love or pure lust nowadays. No emotion was pure,
because everything was mixed up with fear and hatred. Their embrace had
been a battle, the climax a victory. It was a blow struck against the
Party. It was a political act.


\section{III}\label{iii-1}

``We can come here once again,'' said Julia. ``It\textquotesingle s
generally safe to use any hide-out twice. But not for another month or
two, of course.''

As soon as she woke up her demeanor had changed. She became alert and
businesslike, put her clothes on, knotted the scarlet sash about her
waist, and began arranging the details of the journey home. It seemed
natural to leave this to her. She obviously had a practical cunning
which Winston lacked, and she seemed also to have an exhaustive
knowledge of the countryside round London, stored away from innumerable
community hikes. The route she gave him was quite different from the one
by which he had come, and brought him out at a different railway
station. ``Never go home the same way as you went out,'' she said, as
though enunciating an important general principle. She would leave
first, and Winston was to wait half an hour before following her.

She had named a place where they could meet after work, four evenings
hence. It was a street in one of the poorer quarters, where there was an
open market which was generally crowded and noisy. She would be hanging
about among the stalls, pretending to be in search of shoelaces or
sewing thread. If she judged that the coast was clear she would blow her
nose when he approached; otherwise he was to walk past her without
recognition. But with luck, in the middle of the crowd, it would be safe
to talk for a quarter of an hour and arrange another meeting.

``And now I must go,'' she said as soon as he had mastered his
instructions. ``I\textquotesingle m due back at nineteen-thirty.
I\textquotesingle ve got to put in two hours for the Junior Anti-Sex
League, handing out leaflets, or something. Isn\textquotesingle t it
bloody? Give me a brush-down, would you. Have I got any twigs in my
hair? Are you sure? Then good-by, my love, good-by!''

She flung herself into his arms, kissed him almost violently, and a
moment later pushed her way through the saplings and disappeared into
the wood with very little noise. Even now he had not found out her
surname or her address. However, it made no difference, for it was
inconceivable that they could ever meet indoors or exchange any kind of
written communication.

As it happened they never went back to the clearing in the wood. During
the month of May there was only one further occasion on which they
actually succeeded in making love. That was in another hiding place
known to Julia, the belfry of a ruined church in an almost-deserted
stretch of country where an atomic bomb had fallen thirty years earlier.
It was a good hiding place when once you got there, but the getting
there was very dangerous. For the rest they could meet only in the
streets, in a different place every evening and never for more than half
an hour at a time. In the street it was usually possible to talk, after
a fashion. As they drifted down the crowded pavements, not quite abreast
and never looking at one another, they carried on a curious,
intermittent conversation which flicked on and off like the beams of a
lighthouse, suddenly nipped into silence by the approach of a Party
uniform or the proximity of a telescreen, then taken up again minutes
later in the middle of a sentence, then abruptly cut short as they
parted at the agreed spot, then continued almost without introduction on
the following day. Julia appeared to be quite used to this kind of
conversation, which she called ``talking by installments.'' She was also
surprisingly adept at speaking without moving her lips. Just once in
almost a month of nightly meetings they managed to exchange a kiss. They
were passing in silence down a side street (Julia would never speak when
they were away from the main streets) when there was a deafening roar,
the earth heaved and the air darkened, and Winston found himself lying
on his side, bruised and terrified. A rocket bomb must have dropped
quite near at hand. Suddenly he became aware of Julia\textquotesingle s
face a few centimeters from his own, deathly white, as white as chalk.
Even her lips were white. She was dead! He clasped her against him, and
found that he was kissing a live warm face. But there was some powdery
stuff that got in the way of his lips. Both of their faces were thickly
coated with plaster.

There were evenings when they reached their rendezvous and then had to
walk past one another without a sign, because a patrol had just come
round the corner or a helicopter was hovering overhead. Even if it had
been less dangerous, it would still have been difficult to find time to
meet. Winston\textquotesingle s working week was sixty hours,
Julia\textquotesingle s was even longer, and their free days varied
according to the pressure of work and did not often coincide. Julia, in
any case, seldom had an evening completely free. She spent an
astonishing amount of time in attending lectures and demonstrations,
distributing literature for the Junior Anti-Sex League, preparing
banners for Hate Week, making collections for the savings campaign, and
suchlike activities. It paid, she said; it was camouflage. If you kept
the small rules you could break the big ones. She even induced Winston
to mortgage yet another of his evenings by enrolling himself for the
part-time munition work which was done voluntarily by zealous Party
members. So, one evening every week, Winston spent four hours of
paralyzing boredom, screwing together small bits of metal which were
probably parts of bomb fuses, in a draughty ill-lit workshop where the
knocking of hammers mingled drearily with the music of the telescreens.

When they met in the church tower the gaps in their fragmentary
conversation were filled up. It was a blazing afternoon. The air in the
little square chamber above the bells was hot and stagnant, and smelt
overpoweringly of pigeon dung. They sat talking for hours on the dusty,
twig-littered floor, one or other of them getting up from time to time
to cast a glance through the arrow slits and make sure that no one was
coming.

Julia was twenty-six years old. She lived in a hostel with thirty other
girls (``Always in the stink of women! How I hate women!'' she said
parenthetically), and she worked, as he had guessed, on the
novel-writing machines in the Fiction Department. She enjoyed her work,
which consisted chiefly in running and servicing a powerful but tricky
electric motor. She was ``not clever,'' but was fond of using her hands
and felt at home with machinery. She could describe the whole process of
composing a novel, from the general directive issued by the Planning
Committee down to the final touching-up by the Rewrite Squad. But she
was not interested in the finished product. She ``didn\textquotesingle t
much care for reading,'' she said. Books were just a commodity that had
to be produced, like jam or bootlaces.

She had no memories of anything before the early Sixties, and the only
person she had ever known who talked frequently of the days before the
Revolution was a grandfather who had disappeared when she was eight. At
school she had been captain of the hockey team and had won the
gymnastics trophy two years running. She had been a troop leader in the
Spies and a branch secretary in the Youth League before joining the
Junior Anti-Sex League. She had always borne an excellent character. She
had even (an infallible mark of good reputation) been picked out to work
in Pornosec, the sub-section of the Fiction Department which turned out
cheap pornography for distribution among the proles. It was nicknamed
Muck House by the people who worked in it, she remarked. There she had
remained for a year, helping to produce booklets in sealed packets with
titles like \emph{Spanking Stories} or \emph{One Night in a
Girls\textquotesingle{} School}, to be bought furtively by proletarian
youths who were under the impression that they were buying something
illegal.

``What are these books like?'' said Winston curiously.

``Oh, ghastly rubbish. They\textquotesingle re boring, really. They only
have six plots, but they swap them round a bit. Of course I was only on
the kaleidoscopes. I was never in the Rewrite Squad. I\textquotesingle m
not literary, dear---not even enough for that.''

He learned with astonishment that all the workers in Pornosec, except
the head of the department, were girls. The theory was that men, whose
sex instincts were less controllable than those of women, were in
greater danger of being corrupted by the filth they handled.

``They don\textquotesingle t even like having married women there,'' she
added. ``Girls are always supposed to be so pure. Here\textquotesingle s
one who isn\textquotesingle t, anyway.''

She had had her first love affair when she was sixteen, with a Party
member of sixty who later committed suicide to avoid arrest. ``And a good
job too,'' said Julia. ``Otherwise they\textquotesingle d have had my name
out of him when he confessed.'' Since then there had been various others.
Life as she saw it was quite simple. You wanted a good time; ``they,''
meaning the Party, wanted to stop you having it; you broke the rules as
best you could. She seemed to think it just as natural that ``they''
should want to rob you of your pleasures as that you should want to
avoid being caught. She hated the Party, and said so in the crudest
words, but she made no general criticism of it. Except where it touched
upon her own life she had no interest in Party doctrine. He noticed that
she never used Newspeak words, except the ones that had passed into
everyday use. She had never heard of the Brotherhood, and refused to
believe in its existence. Any kind of organized revolt against the
Party, which was bound to be a failure, struck her as stupid. The clever
thing was to break the rules and stay alive all the same. He wondered
vaguely how many others like her there might be in the younger
generation---people who had grown up in the world of the Revolution,
knowing nothing else, accepting the Party as something unalterable, like
the sky, not rebelling against its authority but simply evading it, as a
rabbit dodges a dog.

They did not discuss the possibility of getting married. It was too
remote to be worth thinking about. No imaginable committee would ever
sanction such a marriage even if Katharine, Winston\textquotesingle s
wife, could somehow have been got rid of. It was hopeless even as a
daydream.

``What was she like, your wife?'' said Julia.

``She was---do you know the Newspeak word \emph{goodthinkful}? Meaning
naturally orthodox, incapable of thinking a bad thought?''

``No, I didn\textquotesingle t know the word, but I know the kind of
person, right enough.''

He began telling her the story of his married life, but curiously enough
she appeared to know the essential parts of it already. She described to
him, almost as though she had seen or felt it, the stiffening of
Katharine\textquotesingle s body as soon as he touched her, the way in
which she still seemed to be pushing him from her with all her strength,
even when her arms were clasped tightly round him. With Julia he felt no
difficulty in talking about such things; Katharine, in any case, had
long ceased to be a painful memory and become merely a distasteful one.

``I could have stood it if it hadn\textquotesingle t been for one thing,''
he said. He told her about the frigid little ceremony that Katharine had
forced him to go through on the same night every week. ``She hated it,
but nothing would make her stop doing it. She used to call it---but
you\textquotesingle ll never guess.''

``Our duty to the Party,'' said Julia promptly.

``How did you know that?''

``I\textquotesingle ve been at school too, dear. Sex talks once a month
for the over-sixteens. And in the Youth Movement. They rub it into you
for years. I dare say it works in a lot of cases. But of course you can
never tell; people are such hypocrites.''

She began to enlarge upon the subject. With Julia, everything came back
to her own sexuality. As soon as this was touched upon in any way she
was capable of great acuteness. Unlike Winston, she had grasped the
inner meaning of the Party\textquotesingle s sexual puritanism. It was
not merely that the sex instinct created a world of its own which was
outside the Party\textquotesingle s control and which therefore had to
be destroyed if possible. What was more important was that sexual
privation induced hysteria, which was desirable because it could be
transformed into war fever and leader worship. The way she put it was:

``When you make love you\textquotesingle re using up energy; and
afterwards you feel happy and don\textquotesingle t give a damn for
anything. They can\textquotesingle t bear you to feel like that. They
want you to be bursting with energy all the time. All this marching up
and down and cheering and waving flags is simply sex gone sour. If
you\textquotesingle re happy inside yourself, why should you get excited
about Big Brother and the Three-Year Plans and the Two Minutes Hate and
all the rest of their bloody rot?''

That was very true, he thought. There was a direct, intimate connection
between chastity and political orthodoxy. For how could the fear, the
hatred, and the lunatic credulity which the Party needed in its members
be kept at the right pitch except by bottling down some powerful
instinct and using it as a driving force? The sex impulse was dangerous
to the Party, and the Party had turned it to account. They had played a
similar trick with the instinct of parenthood. The family could not
actually be abolished, and, indeed, people were encouraged to be fond of
their children in almost the old-fashioned way. The children, on the
other hand, were systematically turned against their parents and taught
to spy on them and report their deviations. The family had become in
effect an extension of the Thought Police. It was a device by means of
which everyone could be surrounded night and day by informers who knew
him intimately.

Abruptly his mind went back to Katharine. Katharine would unquestionably
have denounced him to the Thought Police if she had not happened to be
too stupid to detect the unorthodoxy of his opinions. But what really
recalled her to him at this moment was the stifling heat of the
afternoon, which had brought the sweat out on his forehead. He began
telling Julia of something that had happened, or rather had failed to
happen, on another sweltering summer afternoon, eleven years ago.

It was three or four months after they were married. They had lost their
way on a community hike somewhere in Kent. They had only lagged behind
the others for a couple of minutes, but they took a wrong turning, and
presently found themselves pulled up short by the edge of an old chalk
quarry. It was a sheer drop of ten or twenty meters, with boulders at
the bottom. There was nobody of whom they could ask the way. As soon as
she realized that they were lost Katharine became very uneasy. To be
away from the noisy mob of hikers even for a moment gave her a feeling
of wrongdoing. She wanted to hurry back by the way they had come and
start searching in the other direction. But at this moment Winston
noticed some tufts of loose-strife growing in the cracks of the cliff
beneath them. One tuft was of two colors, magenta and brick red,
apparently growing on the same root. He had never seen anything of the
kind before, and he called to Katharine to come and look at it.

``Look, Katharine! Look at those flowers. That clump down near the
bottom. Do you see they\textquotesingle re two different colors?''

She had already turned to go, but she did rather fretfully come back for
a moment. She even leaned out over the cliff face to see where he was
pointing. He was standing a little behind her, and he put his hand on
her waist to steady her. At this moment it suddenly occurred to him how
completely alone they were. There was not a human creature anywhere, not
a leaf stirring, not even a bird awake. In a place like this the danger
that there would be a hidden microphone was very small, and even if
there was a microphone it would only pick up sounds. It was the hottest,
sleepiest hour of the afternoon. The sun blazed down upon them, the
sweat tickled his face. And the thought struck him....

``Why didn\textquotesingle t you give her a good shove?'' said Julia. ``I
would have.''

``Yes, dear, you would have. I would have, if I\textquotesingle d been
the same person then as I am now. Or perhaps I
would---I\textquotesingle m not certain.''

``Are you sorry you didn\textquotesingle t?''

``Yes. On the whole I\textquotesingle m sorry I didn\textquotesingle t.''

They were sitting side by side on the dusty floor. He pulled her closer
against him. Her head rested on his shoulder, the pleasant smell of her
hair conquering the pigeon dung. She was very young, he thought, she
still expected something from life, she did not understand that to push
an inconvenient person over a cliff solves nothing.

``Actually it would have made no difference,'' he said.

``Then why are you sorry you didn\textquotesingle t do it?''

``Only because I prefer a positive to a negative. In this game that
we\textquotesingle re playing, we can\textquotesingle t win. Some kinds
of failure are better than other kinds, that\textquotesingle s all.''

He felt her shoulders give a wriggle of dissent. She always contradicted
him when he said anything of this kind. She would not accept it as a law
of nature that the individual is always defeated. In a way she realized
that she herself was doomed, that sooner or later the Thought Police
would catch her and kill her, but with another part of her mind she
believed that it was somehow possible to construct a secret world in
which you could live as you chose. All you needed was luck and cunning
and boldness. She did not understand that there was no such thing as
happiness, that the only victory lay in the far future, long after you
were dead, that from the moment of declaring war on the Party it was
better to think of yourself as a corpse.

``We are the dead,'' he said.

``We\textquotesingle re not dead yet,'' said Julia prosaically.

``Not physically. Six months, a year---five years, conceivably. I am
afraid of death. You are young, so presumably you\textquotesingle re
more afraid of it than I am. Obviously we shall put it off as long as we
can. But it makes very little difference. So long as human beings stay
human, death and life are the same thing.''

``Oh, rubbish! Which would you sooner sleep with, me or a skeleton?
Don\textquotesingle t you enjoy being alive? Don\textquotesingle t you
like feeling: This is me, this is my hand, this is my leg,
I\textquotesingle m real, I\textquotesingle m solid, I\textquotesingle m
alive! Don\textquotesingle t you like \emph{this}?''

She twisted herself round and pressed her bosom against him. He could
feel her breasts, ripe yet firm, through her overalls. Her body seemed
to be pouring some of its youth and vigor into his.

``Yes, I like that,'' he said.

``Then stop talking about dying. And now listen, dear,
we\textquotesingle ve got to fix up about the next time we meet. We may
as well go back to the place in the wood. We\textquotesingle ve given it
a good long rest. But you must get there by a different way this time.
I\textquotesingle ve got it all planned out. You take the train---but
look, I\textquotesingle ll draw it out for you.''

And in her practical way she scraped together a small square of dust,
and with a twig from a pigeon\textquotesingle s nest began drawing a map
on the floor.


\section{IV}\label{iv-1}

Winston looked round the shabby little room above Mr.
Charrington\textquotesingle s shop. Beside the window the enormous bed
was made up, with ragged blankets and a coverless bolster. The
old-fashioned clock with the twelve-hour face was ticking away on the
mantelpiece. In the corner, on the gateleg table, the glass paperweight
which he had bought on his last visit gleamed softly out of the
half-darkness.

In the fender was a battered tin oilstove, a saucepan and two cups,
provided by Mr. Charrington. Winston lit the burner and set a pan of
water to boil. He had brought an envelope full of Victory Coffee and
some saccharine tablets. The clock\textquotesingle s hands said
seven-twenty; it was nineteen-twenty really. She was coming at
nineteen-thirty.

Folly, folly, his heart kept saying: conscious, gratuitous, suicidal
folly! Of all the crimes that a Party member could commit, this one was
the least possible to conceal. Actually the idea had first floated into
his head in the form of a vision of the glass paperweight mirrored by
the surface of the gateleg table. As he had foreseen, Mr. Charrington
had made no difficulty about letting the room. He was obviously glad of
the few dollars that it would bring him. Nor did he seem shocked or
become offensively knowing when it was made clear that Winston wanted
the room for the purpose of a love affair. Instead he looked into the
middle distance and spoke in generalities, with so delicate an air as to
give the impression that he had become partly invisible. Privacy, he
said, was a very valuable thing. Everyone wanted a place where they
could be alone occasionally. And when they had such a place, it was only
common courtesy in anyone else who knew of it to keep his knowledge to
himself. He even, seeming almost to fade out of existence as he did so,
added that there were two entries to the house, one of them through the
backyard, which gave on an alley.

Under the window somebody was singing. Winston peeped out, secure in the
protection of the muslin curtain. The June sun was still high in the
sky, and in the sun-filled court below a monstrous woman, solid as a
Norman pillar, with brawny red forearms and a sacking apron strapped
about her middle, was stumping to and fro between a washtub and a
clothesline, pegging out a series of square white things which Winston
recognized as babies\textquotesingle{} diapers. Whenever her mouth was
not corked with clothes pegs she was singing in a powerful contralto:

\begin{quotation}
  \noindent`` It was only an \textquotesingle opeless fancy,\\
  It passed like an Ipril dye,\\
  But a look an\textquotesingle{} a word an\textquotesingle{} the
  dreams they stirred\\
  They \textquotesingle ave stolen my \textquotesingle eart awye!''
\end{quotation}

The tune had been haunting London for weeks past. It was one of
countless similar songs published for the benefit of the proles by a
sub-section of the Music Department. The words of these songs were
composed without any human intervention whatever on an instrument known
as a versificator. But the woman sang so tunefully as to turn the
dreadful rubbish into an almost pleasant sound. He could hear the woman
singing and the scrape of her shoes on the flagstones, and the cries of
the children in the street, and somewhere in the far distance a faint
roar of traffic, and yet the room seemed curiously silent, thanks to the
absence of a telescreen.

Folly, folly, folly! he thought again. It was inconceivable that they
could frequent this place for more than a few weeks without being
caught. But the temptation of having a hiding place that was truly their
own, indoors and near at hand, had been too much for both of them. For
some time after their visit to the church belfry it had been impossible
to arrange meetings. Working hours had been drastically increased in
anticipation of Hate Week. It was more than a month distant, but the
enormous, complex preparations that it entailed were throwing extra work
onto everybody. Finally both of them managed to secure a free afternoon
on the same day. They had agreed to go back to the clearing in the wood.
On the evening beforehand they met briefly in the street. As usual
Winston hardly looked at Julia as they drifted toward one another in the
crowd, but from the short glance he gave her it seemed to him that she
was paler than usual.

``It\textquotesingle s all off,'' she murmured as soon as she judged it
safe to speak. ``Tomorrow, I mean.''

``What?''

``Tomorrow afternoon. I can\textquotesingle t come.''

``Why not?''

``Oh, the usual reason. It\textquotesingle s started early this time.''

For a moment he was violently angry. During the month that he had known
her the nature of his desire for her had changed. At the beginning there
had been little true sensuality in it. Their first love-making had been
simply an act of the will. But after the second time it was different.
The smell of her hair, the taste of her mouth, the feeling of her skin
seemed to have got inside him, or into the air all round him. She had
become a physical necessity, something that he not only wanted but felt
that he had a right to. When she said that she could not come, he had
the feeling that she was cheating him. But just at this moment the crowd
pressed them together and their hands accidentally met. She gave the
tips of his fingers a quick squeeze that seemed to invite not desire but
affection. It struck him that when one lived with a woman this
particular disappointment must be a normal, recurring event; and a deep
tenderness, such as he had not felt for her before, suddenly took hold
of him. He wished that they were a married couple of ten
years\textquotesingle{} standing. He wished that he were walking through
the streets with her just as they were doing now, but openly and without
fear, talking of trivialities and buying odds and ends for the
household. He wished above all that they had some place where they could
be alone together without feeling the obligation to make love every time
they met. It was not actually at that moment, but at some time on the
following day, that the idea of renting Mr.
Charrington\textquotesingle s room had occurred to him. When he
suggested it to Julia she had agreed with unexpected readiness. Both of
them knew that it was lunacy. It was as though they were intentionally
stepping nearer to their graves. As he sat waiting on the edge of the
bed he thought again of the cellars of the Ministry of Love. It was
curious how that predestined horror moved in and out of
one\textquotesingle s consciousness. There it lay, fixed in future time,
preceding death as surely as 99 precedes 100. One could not avoid it,
but one could perhaps postpone it; and yet instead, every now and again,
by a conscious, willful act, one chose to shorten the interval before it
happened.

At this moment there was a quick step on the stairs. Julia burst into
the room. She was carrying a tool bag of coarse brown canvas, such as he
had sometimes seen her carrying to and fro at the Ministry. He started
forward to take her in his arms, but she disengaged herself rather
hurriedly, partly because she was still holding the tool bag.

``Half a second,'' she said. ``Just let me show you what
I\textquotesingle ve brought. Did you bring some of that filthy Victory
Coffee? I thought you would. You can chuck it away again, because we
shan\textquotesingle t be needing it. Look here.''

She fell on her knees, threw open the bag, and tumbled out some spanners
and a screwdriver that filled the top part of it. Underneath was a
number of neat paper packets. The first packet that she passed to
Winston had a strange and yet vaguely familiar feeling. It was filled
with some kind of heavy, sandlike stuff which yielded wherever you
touched it.

``It isn\textquotesingle t sugar?'' he said.

``Real sugar. Not saccharine, sugar. And here\textquotesingle s a loaf of
bread---proper white bread, not our bloody stuff---and a little pot of
jam. And here\textquotesingle s a tin of milk---but look! This is the
one I\textquotesingle m really proud of. I had to wrap a bit of sacking
round it, because---''

But she did not need to tell him why she had wrapped it up. The smell
was already filling the room, a rich hot smell which seemed like an
emanation from his early childhood, but which one did occasionally meet
with even now, blowing down a passageway before a door slammed, or
diffusing itself mysteriously in a crowded street, sniffed for an
instant and then lost again.

``It\textquotesingle s coffee,'' he murmured, ``real coffee.''

``It\textquotesingle s Inner Party coffee. There\textquotesingle s a
whole kilo here,'' she said.

``How did you manage to get hold of all these things?''

``It\textquotesingle s all Inner Party stuff. There\textquotesingle s
nothing those swine don\textquotesingle t have, nothing. But of course
waiters and servants and people pinch things, and---look, I got a little
packet of tea as well.''

Winston had squatted down beside her. He tore open a corner of the
packet.

``It\textquotesingle s real tea. Not blackberry leaves.''

``There\textquotesingle s been a lot of tea about lately.
They\textquotesingle ve captured India, or something,'' she said vaguely.
``But listen, dear. I want you to turn your back on me for three minutes.
Go and sit on the other side of the bed. Don\textquotesingle t go too
near the window. And don\textquotesingle t turn round till I tell you.''

Winston gazed abstractedly through the muslin curtain. Down in the yard
the red-armed woman was still marching to and fro between the washtub
and the line. She took two more pegs out of her mouth and sang with deep
feeling:

\begin{quotation}
 \noindent ``They sye that time \textquotesingle eals all things,\\
  They sye you can always forget;\\
  But the smiles an the tears across the years\\
  They twist my \textquotesingle eartstrings yet!''
\end{quotation}

She knew the whole driveling song by heart, it seemed. Her voice floated
upward with the sweet summer air, very tuneful, charged with a sort of
happy melancholy. One had the feeling that she would have been perfectly
content if the June evening had been endless and the supply of clothes
inexhaustible, to remain there for a thousand years, pegging out diapers
and singing rubbish. It struck him as a curious fact that he had never
heard a member of the Party singing alone and spontaneously. It would
even have seemed slightly unorthodox, a dangerous eccentricity, like
talking to oneself. Perhaps it was only when people were somewhere near
the starvation level that they had anything to sing about.

``You can turn round now,'' said Julia.

He turned round, and for a second almost failed to recognize her. What
he had actually expected was to see her naked. But she was not naked.
The transformation that had happened was much more surprising than that.
She had painted her face.

She must have slipped into some shop in the proletarian quarters and
bought herself a complete set of make-up materials. Her lips were deeply
reddened, her cheeks rouged, her nose powdered; there was even a touch
of something under the eyes to make them brighter. It was not very
skillfully done, but Winston\textquotesingle s standards in such matters
were not high. He had never before seen or imagined a woman of the Party
with cosmetics on her face. The improvement in her appearance was
startling. With just a few dabs of color in the right places she had
become not only very much prettier, but, above all, far more feminine.
Her short hair and boyish overalls merely added to the effect. As he
took her in his arms a wave of synthetic violets flooded his nostrils.
He remembered the half-darkness of a basement kitchen and a
woman\textquotesingle s cavernous mouth. It was the very same scent that
she had used; but at the moment it did not seem to matter.

``Scent, too!'' he said.

``Yes, dear, scent, too. And do you know what I\textquotesingle m going
to do next? I\textquotesingle m going to get hold of a real
woman\textquotesingle s frock from somewhere and wear it instead of
these bloody trousers. I\textquotesingle ll wear silk stockings and
high-heeled shoes! In this room I\textquotesingle m going to be a woman,
not a Party comrade.''

They flung their clothes off and climbed into the huge mahogany bed. It
was the first time that he had stripped himself naked in her presence.
Until now he had been too much ashamed of his pale and meager body, with
the varicose veins standing out on his calves and the discolored patch
over his ankle. There were no sheets, but the blanket they lay on was
threadbare and smooth, and the size and springiness of the bed
astonished both of them. ``It\textquotesingle s sure to be full of bugs,
but who cares?'' said Julia. One never saw a double bed nowadays except
in the homes of the proles. Winston had occasionally slept in one in his
boy-hood; Julia had never been in one before, so far as she could
remember.

Presently they fell asleep for a little while. When Winston woke up the
hands of the clock had crept round to nearly nine. He did not stir,
because Julia was sleeping with her head in the crook of his arm. Most
of her make-up had transferred itself to his own face or the bolster,
but a light stain of rouge still brought out the beauty of her
cheekbone. A yellow ray from the sinking sun fell across the foot of the
bed and lighted up the fireplace, where the water in the pan was boiling
fast. Down in the yard the woman had stopped singing, but the faint
shouts of children floated in from the street. He wondered vaguely
whether in the abolished past it had been a normal experience to lie in
bed like this, in the cool of a summer evening, a man and a woman with
no clothes on, making love when they chose, talking of what they chose,
not feeling any compulsion to get up, simply lying there and listening
to peaceful sounds outside. Surely there could never have been a time
when that seemed ordinary. Julia woke up, rubbed her eyes, and raised
herself on her elbow to look at the oilstove.

``Half that water\textquotesingle s boiled away,'' she said.
``I\textquotesingle ll get up and make some coffee in another moment.
We\textquotesingle ve got an hour. What time do they cut the lights off
at your flats?''

``Twenty-three thirty.''

``It\textquotesingle s twenty-three at the hostel. But you have to get in
earlier than that, because---Hi! Get out, you filthy brute!''

She suddenly twisted herself over in the bed, seized a shoe from the
floor, and sent it hurtling into the corner with a boyish jerk of her
arm, exactly as he had seen her fling the dictionary at Goldstein, that
morning during the Two Minutes Hate.

``What was it?'' he said in surprise.

``A rat. I saw him stick his beastly nose out of the wainscoting.
There\textquotesingle s a hole down there. I gave him a good fright,
anyway.''

``Rats!'' murmured Winston. ``In this room!''

``They\textquotesingle re all over the place,'' said Julia indifferently
as she lay down again. ``We\textquotesingle ve even got them in the
kitchen at the hostel. Some parts of London are swarming with them. Did
you know they attack children? Yes, they do. In some of these streets a
woman daren\textquotesingle t leave a baby alone for two minutes.
It\textquotesingle s the great huge brown ones that do it. And the nasty
thing is that the brutes always---''

``\emph{Don\textquotesingle t go on!}'' said Winston, with his eyes
tightly shut.

``Dearest! You\textquotesingle ve gone quite pale. What\textquotesingle s
the matter? Do they make you feel sick?''

``Of all horrors in the world---a rat!''

She pressed herself against him and wound her limbs round him, as though
to reassure him with the warmth of her body. He did not reopen his eyes
immediately. For several moments he had had the feeling of being back in
a nightmare which had recurred from time to time throughout his life. It
was always very much the same. He was standing in front of a wall of
darkness, and on the other side of it there was something unendurable,
something too dreadful to be faced. In the dream his deepest feeling was
always one of self-deception, because he did in fact know what was
behind the wall of darkness. With a deadly effort, like wrenching a
piece out of his own brain, he could even have dragged the thing into
the open. He always woke up without discovering what it was, but somehow
it was connected with what Julia had been saying when he cut her short.

``I\textquotesingle m sorry,'' he said. ``It\textquotesingle s nothing. I
don\textquotesingle t like rats, that\textquotesingle s all.''

``Don\textquotesingle t worry, dear, we\textquotesingle re not going to
have the filthy brutes in here. I\textquotesingle ll stuff the hole with
a bit of sacking before we go. And next time we come here
I\textquotesingle ll bring some plaster and bung it up properly.''

Already the black instant of panic was half-forgotten. Feeling slightly
ashamed of himself, he sat up against the bedhead. Julia got out of bed,
pulled on her overalls, and made the coffee. The smell that rose from
the saucepan was so powerful and exciting that they shut the window lest
anybody outside should notice it and become inquisitive. What was even
better than the taste of the coffee was the silky texture given to it by
the sugar, a thing Winston had almost forgotten after years of
saccharine. With one hand in her pocket and a piece of bread and jam in
the other, Julia wandered about the room, glancing indifferently at the
bookcase, pointing out the best way of repairing the gateleg table,
plumping herself down in the ragged armchair to see if it was
comfortable, and examining the absurd twelve-hour clock with a sort of
tolerant amusement. She brought the glass paperweight over to the bed to
have a look at it in a better light. He took it out of her hand,
fascinated as always by the soft, rainwatery appearance of the glass.

``What is it, do you think?'' said Julia.

``I don\textquotesingle t think it\textquotesingle s anything---I mean, I
don\textquotesingle t think it was ever put to any use.
That\textquotesingle s what I like about it. It\textquotesingle s a
little chunk of history that they\textquotesingle ve forgotten to alter.
It\textquotesingle s a message from a hundred years ago, if one knew how
to read it.''

``And that picture over there---'' she nodded at the engraving on the
opposite wall---``would that be a hundred years old?''

``More. Two hundred, I dare say. One can\textquotesingle t tell.
It\textquotesingle s impossible to discover the age of anything
nowadays.''

She went over to look at it. ``Here\textquotesingle s where that brute
stuck his nose out,'' she said, kicking the wainscoting immediately below
the picture. ``What is this place? I\textquotesingle ve seen it before
somewhere.''

``It\textquotesingle s a church, or at least it used to be. St.
Clement\textquotesingle s Dane its name was.'' The fragment of rhyme that
Mr. Charrington had taught him came back into his head, and he added
half-nostalgically: ``\emph{Oranges and lemons, say the bells of St.
Clement\textquotesingle s!}''

To his astonishment she capped the line:

\begin{quotation}
  ``You owe me three farthings, say the bells of St.
  Martin\textquotesingle s,\\
  When will you pay me? say the bells of Old Bailey---
\end{quotation}

``I can\textquotesingle t remember how it goes on after that. But anyway
I remember it ends up, \emph{Here comes a candle to light you to bed,
here comes a chopper to chop off your head!}''

It was like the two halves of a countersign. But there must be another
line after \emph{the bells of Old Bailey}. Perhaps it could be dug out
of Mr. Charrington\textquotesingle s memory, if he were suitably
prompted.

``Who taught you that?'' he said.

``My grandfather. He used to say it to me when I was a little girl. He
was vaporized when I was eight---at any rate, he disappeared. I wonder
what a lemon was,'' she added inconsequently. ``I\textquotesingle ve seen
oranges. They\textquotesingle re a kind of round yellow fruit with a
thick skin.''

``I can remember lemons,'' said Winston. ``They were quite common in the
Fifties. They were so sour that it set your teeth on edge even to smell
them.''

``I bet that picture\textquotesingle s got bugs behind it,'' said Julia.
``I\textquotesingle ll take it down and give it a good clean some day. I
suppose it\textquotesingle s almost time we were leaving. I must start
washing this paint off. What a bore! I\textquotesingle ll get the
lipstick off your face afterwards.''

Winston did not get up for a few minutes more. The room was darkening.
He turned over toward the light and lay gazing into the glass
paperweight. The inexhaustibly interesting thing was not the fragment of
coral but the interior of the glass itself. There was such a depth of
it, and yet it was almost as transparent as air. It was as though the
surface of the glass had been the arch of the sky, enclosing a tiny
world with its atmosphere complete. He had the feeling that he could get
inside it, and that in fact he was inside it, along with the mahogany
bed and the gateleg table and the clock and the steel engraving and the
paperweight itself. The paperweight was the room he was in, and the
coral was Julia\textquotesingle s life and his own, fixed in a sort of
eternity at the heart of the crystal.


\section{V}\label{v-1}

Syme had vanished. A morning came, and he was missing from work; a few
thoughtless people commented on his absence. On the next day nobody
mentioned him. On the third day Winston went into the vestibule of the
Records Department to look at the notice board. One of the notices
carried a printed list of the members of the Chess Committee, of whom
Syme had been one. It looked almost exactly as it had looked
before---nothing had been crossed out---but it was one name shorter. It
was enough. Syme had ceased to exist; he had never existed.

The weather was baking hot. In the labyrinthine Ministry the windowless,
air-conditioned rooms kept their normal temperature, but outside the
pavements scorched one\textquotesingle s feet and the stench of the
Tubes at the rush hours was a horror. The preparations for Hate Week
were in full swing, and the staffs of all the Ministries were working
overtime. Processions, meetings, military parades, lectures, waxwork
displays, film shows, telescreen programs all had to be organized;
stands had to be erected, effigies built, slogans coined, songs written,
rumors circulated, photographs faked. Julia\textquotesingle s unit in
the Fiction Department had been taken off the production of novels and
was rushing out a series of atrocity pamphlets. Winston, in addition to
his regular work, spent long periods every day in going through back
files of the \emph{Times} and altering and embellishing news items which
were to be quoted in speeches. Late at night, when crowds of rowdy
proles roamed the streets, the town had a curiously febrile air. The
rocket bombs crashed oftener than ever, and sometimes in the far
distance there were enormous explosions which no one could explain and
about which there were wild rumors.

The new tune which was to be the theme song of Hate Week (the ``Hate
Song,'' it was called) had already been composed and was being endlessly
plugged on the telescreens. It had a savage, barking rhythm which could
not exactly be called music, but resembled the beating of a drum. Roared
out by hundreds of voices to the tramp of marching feet, it was
terrifying. The proles had taken a fancy to it, and in the midnight
streets it competed with the still-popular ``It Was Only a Hopeless
Fancy.'' The Parsons children played it at all hours of the night and
day, unbearably, on a comb and a piece of toilet paper.
Winston\textquotesingle s evenings were fuller than ever. Squads of
volunteers, organized by Parsons, were preparing the street for Hate
Week, stitching banners, painting posters, erecting flagstaffs on the
roofs, and perilously slinging wires across the street for the reception
of streamers. Parsons boasted that Victory Mansions alone would display
four hundred meters of bunting. He was in his native element and as
happy as a lark. The heat and the manual work had even given him a
pretext for reverting to shorts and an open shirt in the evenings. He
was everywhere at once, pushing, pulling, sawing, hammering,
improvising, jollying everyone along with comradely exhortations and
giving out from every fold of his body what seemed an inexhaustible
supply of acrid-smelling sweat.

A new poster had suddenly appeared all over London. It had no caption,
and represented simply the monstrous figure of a Eurasian soldier, three
or four meters high, striding forward with expressionless Mongolian face
and enormous boots, a submachine gun pointed from his hip. From whatever
angle you looked at the poster, the muzzle of the gun, magnified by the
foreshortening, seemed to be pointed straight at you. The thing had been
plastered on every blank space on every wall, even outnumbering the
portraits of Big Brother. The proles, normally apathetic about the war,
were being lashed into one of their periodical frenzies of patriotism.
As though to harmonize with the general mood, the rocket bombs had been
killing larger numbers of people than usual. One fell on a crowded film
theater in Stepney, burying several hundred victims among the ruins. The
whole population of the neighborhood turned out for a long, trailing
funeral which went on for hours and was in effect an indignation
meeting. Another bomb fell on a piece of waste ground which was used as
a playground, and several dozen children were blown to pieces. There
were further angry demonstrations, Goldstein was burned in effigy,
hundreds of copies of the poster of the Eurasian soldier were torn down
and added to the flames, and a number of shops were looted in the
turmoil; then a rumor flew round that spies were directing the rocket
bombs by means of wireless waves, and an old couple who were suspected
of being of foreign extraction had their house set on fire and perished
of suffocation.

In the room over Mr. Charrington\textquotesingle s shop, when they could
get there, Julia and Winston lay side by side on a stripped bed under
the open window, naked for the sake of coolness. The rat had never come
back, but the bugs had multiplied hideously in the heat. It did not seem
to matter. Dirty or clean, the room was paradise. As soon as they
arrived they would sprinkle everything with pepper bought on the blacket
market, tear off their clothes and make love with sweating bodies, then
fall asleep and wake to find that the bugs had rallied and were massing
for the counter-attack.

Four, five, six---seven times they met during the month of June. Winston
had dropped his habit of drinking gin at all hours. He seemed to have
lost the need for it. He had grown fatter, his varicose ulcer had
subsided, leaving only a brown stain on the skin above his ankle, his
fits of coughing in the early morning had stopped. The process of life
had ceased to be intolerable, he had no longer any impulse to make faces
at the telescreen or shout curses at the top of his voice. Now that they
had a secure hiding place, almost a home, it did not even seem a
hardship that they could only meet infrequently and for a couple of
hours at a time. What mattered was that the room over the junk shop
should exist. To know that it was there, inviolate, was almost the same
as being in it. The room was a world, a pocket of the past where extinct
animals could walk. Mr. Charrington, thought Winston, was another
extinct animal. He usually stopped to talk with Mr. Charrington for a
few minutes on his way upstairs. The old man seemed seldom or never to
go out of doors, and on the other hand to have almost no customers. He
led a ghostlike existence between the tiny, dark shop and an even tinier
back kitchen where he prepared his meals and which contained, among
other things, an unbelievably ancient gramophone with an enormous horn.
He seemed glad of the opportunity to talk. Wandering about among his
worthless stock, with his long nose and thick spectacles and his bowed
shoulders in the velvet jacket, he had always vaguely the air of being a
collector rather than a tradesman. With a sort of faded enthusiasm he
would finger this scrap of rubbish or that---a china bottle-stopper, the
painted lid of a broken snuffbox, a pinchbeck locket containing a strand
of some long-dead baby\textquotesingle s hair---never asking that
Winston should buy it, merely that he should admire it. To talk to him
was like listening to the tinkling of a worn-out musical box. He had
dragged out from the corners of his memory some more fragments of
forgotten rhymes. There was one about four and twenty blackbirds, and
another about a cow with a crumpled horn, and another about the death of
poor Cock Robin. ``It just occurred to me you might be interested,'' he
would say with a deprecating little laugh whenever he produced a new
fragment. But he could never recall more than a few lines of any one
rhyme.

Both of them knew---in a way, it was never out of their minds---that
what was now happening could not last long. There were times when the
fact of impending death seemed as palpable as the bed they lay on, and
they would cling together with a sort of despairing sensuality, like a
damned soul grasping at his last morsel of pleasure when the clock is
within five minutes of striking. But there were also times when they had
the illusion not only of safety but of permanence. So long as they were
actually in this room, they both felt, no harm could come to them.
Getting there was difficult and dangerous, but the room itself was
sanctuary. It was as when Winston had gazed into the heart of the
paperweight, with the feeling that it would be possible to get inside
that glassy world, and that once inside it time could be arrested. Often
they gave themselves up to daydreams of escape. Their luck would hold
indefinitely, and they would carry on their intrigue, just like this,
for the remainder of their natural lives. Or Katharine would die, and by
subtle maneuverings Winston and Julia would succeed in getting married.
Or they would commit suicide together. Or they would disappear, alter
themselves out of recognition, learn to speak with proletarian accents,
get jobs in a factory, and live out their lives undetected in a back
street. It was all nonsense, as they both knew. In reality there was no
escape. Even the one plan that was practicable, suicide, they had no
intention of carrying out. To hang on from day to day and from week to
week, spinning out a present that had no future, seemed an unconquerable
instinct, just as one\textquotesingle s lungs will always draw the next
breath so long as there is air available.

Sometimes, too, they talked of engaging in active rebellion against the
Party, but with no notion of how to take the first step. Even if the
fabulous Brotherhood was a reality, there still remained the difficulty
of finding one\textquotesingle s way into it. He told her of the strange
intimacy that existed, or seemed to exist, between himself and
O\textquotesingle Brien, and of the impulse he sometimes felt simply to
walk into O\textquotesingle Brien\textquotesingle s presence, announce
that he was the enemy of the Party, and demand his help. Curiously
enough this did not strike her as an impossibly rash thing to do. She
was used to judging people by their faces, and it seemed natural to her
that Winston should believe O\textquotesingle Brien to be trustworthy on
the strength of a single flash of the eyes. Moreover she took it for
granted that everyone, or nearly everyone, secretly hated the Party and
would break the rules if he thought it safe to do so. But she refused to
believe that widespread, organized opposition existed or could exist.
The tales about Goldstein and his underground army, she said, were
simply a lot of rubbish which the Party had invented for its own
purposes and which you had to pretend to believe in. Times beyond
number, at Party rallies and spontaneous demonstrations, she had shouted
at the top of her voice for the execution of people whose names she had
never heard and in whose supposed crimes she had not the faintest
belief. When public trials were happening she had taken her place in the
detachments from the Youth League who surrounded the courts from morning
to night, chanting at intervals ``Death to the traitors!'' During the Two
Minutes Hate she always excelled all others in shouting insults at
Goldstein. Yet she had only the dimmest idea of who Goldstein was and
what doctrines he was supposed to represent. She had grown up since the
Revolution and was too young to remember the ideological battles of the
Fifties and Sixties. Such a thing as an independent political movement
was outside her imagination; and in any case the Party was invincible.
It would always exist, and it would always be the same. You could only
rebel against it by secret disobedience or, at most, by isolated acts of
violence such as killing somebody or blowing something up.

In some ways she was far more acute than Winston, and far less
susceptible to Party propaganda. Once when he happened in some
connection to mention the war against Eurasia, she startled him by
saying casually that in her opinion the war was not happening. The
rocket bombs which fell daily on London were probably fired by the
Government of Oceania itself, ``just to keep people frightened.'' This was
an idea that had literally never occurred to him. She also stirred a
sort of envy in him by telling him that during the Two Minutes Hate her
great difficulty was to avoid bursting out laughing. But she only
questioned the teachings of the Party when they in some way touched upon
her own life. Often she was ready to accept the official mythology,
simply because the difference between truth and falsehood did not seem
important to her. She believed, for instance, having learnt it at
school, that the Party had invented airplanes. (In his own schooldays,
Winston remembered, in the late Fifties, it was only the helicopter that
the Party claimed to have invented; a dozen years later, when Julia was
at school, it was already claiming the airplane; one generation more,
and it would be claiming the steam engine.) And when he told her that
airplanes had been in existence before he was born, and long before the
Revolution, the fact struck her as totally uninteresting. After all,
what did it matter who had invented airplanes? It was rather more of a
shock to him when he discovered from some chance remark that she did not
remember that Oceania, four years ago, had been at war with Eastasia and
at peace with Eurasia. It was true that she regarded the whole war as a
sham; but apparently she had not even noticed that the name of the enemy
had changed. ``I thought we\textquotesingle d always been at war with
Eurasia,'' she said vaguely. It frightened him a little. The invention of
airplanes dated from long before her birth, but the switch-over in the
war had happened only four years ago, well after she was grown up. He
argued with her about it for perhaps a quarter of an hour. In the end he
succeeded in forcing her memory back until she did dimly recall that at
one time Eastasia and not Eurasia had been the enemy. But the issue
still struck her as unimportant. ``Who cares?'' she said impatiently.
``It\textquotesingle s always one bloody war after another, and one knows
the news is all lies anyway.''

Sometimes he talked to her of the Records Department and the impudent
forgeries that he committed there. Such things did not appear to horrify
her. She did not feel the abyss opening beneath her feet at the thought
of lies becoming truths. He told her the story of Jones, Aaronson, and
Rutherford and the momentous slip of paper which he had once held
between his fingers. It did not make much impression on her. At first,
indeed, she failed to grasp the point of the story.

``Were they friends of yours?'' she said.

``No, I never knew them. They were Inner Party members. Besides, they
were far older men than I was. They belonged to the old days, before the
Revolution. I barely knew them by sight.''

``Then what was there to worry about? People are being killed off all the
time, aren\textquotesingle t they?''

He tried to make her understand. ``This was an exceptional case. It
wasn\textquotesingle t just a question of somebody being killed. Do you
realize that the past, starting from yesterday, has been actually
abolished? If it survives anywhere, it\textquotesingle s in a few solid
objects with no words attached to them, like that lump of glass there.
Already we know almost literally nothing about the Revolution and the
years before the Revolution. Every record has been destroyed or
falsified, every book has been rewritten, every picture has been
repainted, every statue and street and building has been renamed, every
date has been altered. And that process is continuing day by day and
minute by minute. History has stopped. Nothing exists except an endless
present in which the Party is always right. I \emph{know}, of course,
that the past is falsified, but it would never be possible for me to
prove it, even when I did the falsification myself. After the thing is
done, no evidence ever remains. The only evidence is inside my own mind,
and I don\textquotesingle t know with any certainty that any other human
being shares my memories. Just in that one instance, in my whole life, I
did possess actual concrete evidence \emph{after} the event---years
after it.''

``And what good was that?''

``It was no good, because I threw it away a few minutes later. But if the
same thing happened today, I should keep it.''

``Well, I wouldn\textquotesingle t!'' said Julia. ``I\textquotesingle m
quite ready to take risks, but only for something worth while, not for
bits of old newspaper. What could you have done with it even if you had
kept it?''

``Not much, perhaps. But it was evidence. It might have planted a few
doubts here and there, supposing that I\textquotesingle d dared to show
it to anybody. I don\textquotesingle t imagine that we can alter
anything in our own lifetime. But one can imagine little knots of
resistance springing up here and there---small groups of people banding
themselves together, and gradually growing, and even leaving a few
records behind, so that the next generation can carry on where we leave
off.''

``I\textquotesingle m not interested in the next generation, dear.
I\textquotesingle m interested in \emph{us}.''

``You\textquotesingle re only a rebel from the waist downwards,'' he told
her.

She thought this brilliantly witty and flung her arms round him in
delight.

In the ramifications of Party doctrine she had not the faintest
interest. Whenever he began to talk of the principles of Ingsoc,
doublethink, the mutability of the past and the denial of objective
reality, and to use Newspeak words, she became bored and confused and
said that she never paid any attention to that kind of thing. One knew
that it was all rubbish, so why let oneself be worried by it? She knew
when to cheer and when to boo, and that was all one needed. If he
persisted in talking of such subjects, she had a disconcerting habit of
falling asleep. She was one of those people who can go to sleep at any
hour and in any position. Talking to her, he realized how easy it was to
present an appearance of orthodoxy while having no grasp whatever of
what orthodoxy meant. In a way, the world-view of the Party imposed
itself most successfully on people incapable of understanding it. They
could be made to accept the most flagrant violations of reality, because
they never fully grasped the enormity of what was demanded of them, and
were not sufficiently interested in public events to notice what was
happening. By lack of understanding they remained sane. They simply
swallowed everything, and what they swallowed did them no harm, because
it left no residue behind, just as a grain of corn will pass undigested
through the body of a bird.


\section{VI}\label{vi-1}

It had happened at last. The expected message had come. All his life, it
seemed to him, he had been waiting for this to happen.

He was walking down the long corridor at the Ministry, and he was almost
at the spot where Julia had slipped the note into his hand when he
became aware that someone larger than himself was walking just behind
him. The person, whoever it was, gave a small cough, evidently as a
prelude to speaking. Winston stopped abruptly and turned. It was
O\textquotesingle Brien.

At last they were face to face, and it seemed that his only impulse was
to run away. His heart bounded violently. He would have been incapable
of speaking. O\textquotesingle Brien, however, had continued forward in
the same movement, laying a friendly hand for a moment on
Winston\textquotesingle s arm, so that the two of them were walking side
by side. He began speaking with the peculiar grave courtesy that
differentiated him from the majority of Inner Party members.

``I had been hoping for an opportunity of talking to you,'' he said. ``I
was reading one of your Newspeak articles in the \emph{Times} the other
day. You take a scholarly interest in Newspeak, I believe?''

Winston had recovered part of his self-possession. ``Hardly scholarly,''
he said. ``I\textquotesingle m only an amateur. It\textquotesingle s not
my subject. I have never had anything to do with the actual construction
of the language.''

``But you write it very elegantly,'' said O\textquotesingle Brien. ``That
is not only my own opinion. I was talking recently to a friend of yours
who is certainly an expert. His name has slipped my memory for the
moment.''

Again Winston\textquotesingle s heart stirred painfully. It was
inconceivable that this was anything other than a reference to Syme. But
Syme was not only dead, he was abolished, an \emph{unperson}. Any
identifiable reference to him would have been mortally dangerous.
O\textquotesingle Brien\textquotesingle s remark must obviously have
been intended as a signal, a code word. By sharing a small act of
thoughtcrime he had turned the two of them into accomplices. They had
continued to stroll slowly down the corridor, but now
O\textquotesingle Brien halted. With the curious, disarming friendliness
that he always managed to put into the gesture, he resettled his
spectacles on his nose. Then he went on:

``What I had really intended to say was that in your article I noticed
you had used two words which have become obsolete. But they have only
become so very recently. Have you seen the tenth edition of the Newspeak
dictionary?''

``No,'' said Winston. ``I didn\textquotesingle t think it had been issued
yet. We are still using the ninth in the Records Department.''

``The tenth edition is not due to appear for some months, I believe. But
a few advance copies have been circulated. I have one myself. It might
interest you to look at it, perhaps?''

``Very much so,'' said Winston, immediately seeing where this tended.

``Some of the new developments are most ingenious. The reduction in the
number of verbs---that is the point that will appeal to you, I think.
Let me see, shall I send a messenger to you with the dictionary? But I
am afraid I invariably forget anything of that kind. Perhaps you could
pick it up at my flat at some time that suited you? Wait. Let me give
you my address.''

They were standing in front of a telescreen. Somewhat absent-mindedly
O\textquotesingle Brien felt two of his pockets and then produced a
small leather-covered notebook and a gold ink pencil. Immediately
beneath the telescreen, in such a position that anyone who was watching
at the other end of the instrument could read what he was writing, he
scribbled an address, tore out the page, and handed it to Winston.

``I am usually at home in the evenings,'' he said. ``If not, my servant
will give you the dictionary.''

He was gone, leaving Winston holding the scrap of paper, which this time
there was no need to conceal. Nevertheless he carefully memorized what
was written on it, and some hours later dropped it into the memory hole
along with a mass of other papers.

They had been talking to one another for a couple of minutes at the
most. There was only one meaning that the episode could possibly have.
It had been contrived as a way of letting Winston know
O\textquotesingle Brien\textquotesingle s address. This was necessary,
because except by direct inquiry it was never possible to discover where
anyone lived. There were no directories of any kind. ``If you ever want
to see me, this is where I can be found,'' was what
O\textquotesingle Brien had been saying to him. Perhaps there would even
be a message concealed somewhere in the dictionary. But at any rate, one
thing was certain. The conspiracy that he had dreamed of did exist, and
he had reached the outer edges of it.

He knew that sooner or later he would obey
O\textquotesingle Brien\textquotesingle s summons. Perhaps tomorrow,
perhaps after a long delay---he was not certain. What was happening was
only the working-out of a process that had started years ago. The first
step had been a secret, involuntary thought; the second had been the
opening of the diary. He had moved from thoughts to words, and now from
words to actions. The last step was something that would happen in the
Ministry of Love. He had accepted it. The end was contained in the
beginning. But it was frightening; or, more exactly, it was like a
foretaste of death, like being a little less alive. Even while he was
speaking to O\textquotesingle Brien, when the meaning of the words had
sunk in, a chilly shuddering feeling had taken possession of his body.
He had the sensation of stepping into the dampness of a grave, and it
was not much better because he had always known that the grave was there
and waiting for him.


\section{VII}\label{vii-1}

Winston had woken up with his eyes full of tears. Julia rolled sleepily
against him, murmuring something that might have been
``What\textquotesingle s the matter?''

``I dreamt---'' he began, and stopped short. It was too complex to be put
into words. There was the dream itself, and there was a memory connected
with it that had swum into his mind in the few seconds after waking.

He lay back with his eyes shut, still sodden in the atmosphere of the
dream. It was a vast, luminous dream in which his whole life seemed to
stretch out before him like a landscape on a summer evening after rain.
It had all occurred inside the glass paperweight, but the surface of the
glass was the dome of the sky, and inside the dome everything was
flooded with clear soft light in which one could see into interminable
distances. The dream had also been comprehended by---indeed, in some
sense it had consisted in---a gesture of the arm made by his mother, and
made again thirty years later by the Jewish woman he had seen on the
news film, trying to shelter the small boy from the bullets, before the
helicopters blew them both to pieces.

``Do you know,'' he said, ``that until this moment I believed I had
murdered my mother?''

``Why did you murder her?'' said Julia, almost asleep.

``I didn\textquotesingle t murder her. Not physically.''

In the dream he had remembered his last glimpse of his mother, and
within a few moments of waking the cluster of small events surrounding
it had all come back. It was a memory that he must have deliberately
pushed out of his consciousness over many years. He was not certain of
the date, but he could not have been less than ten years old, possibly
twelve, when it had happened.

His father had disappeared some time earlier; how much earlier, he could
not remember. He remembered better the rackety, uneasy circumstances of
the time: the periodical panics about air raids and the sheltering in
Tube stations, the piles of rubble everywhere, the unintelligible
proclamations posted at street corners, the gangs of youths in shirts
all the same color, the enormous queues outside the bakeries, the
intermittent machine-gun fire in the distance---above all, the fact that
there was never enough to eat. He remembered long afternoons spent with
other boys in scrounging round dustbins and rubbish heaps, picking out
the ribs of cabbage leaves, potato peelings, sometimes even scraps of
stale breadcrust from which they carefully scraped away the cinders; and
also in waiting for the passing of trucks which traveled over a certain
route and were known to carry cattle feed, and which, when they jolted
over the bad patches in the road, sometimes spilt a few fragments of
oilcake.

When his father disappeared, his mother did not show any surprise or any
violent grief, but a sudden change came over her. She seemed to have
become completely spiritless. It was evident even to Winston that she
was waiting for something that she knew must happen. She did everything
that was needed---cooked, washed, mended, made the bed, swept the floor,
dusted the mantelpiece---always very slowly and with a curious lack of
superfluous motion, like an artist\textquotesingle s lay-figure moving
of its own accord. Her large shapely body seemed to relapse naturally
into stillness. For hours at a time she would sit almost immobile on the
bed, nursing his young sister, a tiny, ailing, very silent child of two
or three, with a face made simian by thinness. Very occasionally she
would take Winston in her arms and press him against her for a long time
without saying anything. He was aware, in spite of his youthfulness and
selfishness, that this was somehow connected with the never-mentioned
thing that was about to happen.

He remembered the room where they lived, a dark, close-smelling room
that seemed half filled by a bed with a white counterpane. There was a
gas ring in the fender, and a shelf where food was kept, and on the
landing outside there was a brown earthenware sink, common to several
rooms. He remembered his mother\textquotesingle s statuesque body
bending over the gas ring to stir at something in a saucepan. Above all
he remembered his continuous hunger, and the fierce sordid battles at
mealtimes. He would ask his mother naggingly, over and over again, why
there was not more food, he would shout and storm at her (he even
remembered the tones of his voice, which was beginning to break
prematurely and sometimes boomed in a peculiar way), or he would attempt
a sniveling note of pathos in his efforts to get more than his share.
His mother was quite ready to give him more than his share. She took it
for granted that he, ``the boy,'' should have the biggest portion; but
however much she gave him he invariably demanded more. At every meal she
would beseech him not to be selfish and to remember that his little
sister was sick and also needed food, but it was no use. He would cry
out with rage when she stopped ladling, he would try to wrench the
saucepan and spoon out of her hands, he would grab bits from his
sister\textquotesingle s plate. He knew that he was starving the other
two, but he could not help it; he even felt that he had a right to do
it. The clamorous hunger in his belly seemed to justify him. Between
meals, if his mother did not stand guard, he was constantly pilfering at
the wretched store of food on the shelf.

One day a chocolate ration was issued. There had been no such issue for
weeks or months past. He remembered quite clearly that precious little
morsel of chocolate. It was a two-ounce slab (they still talked about
ounces in those days) between the three of them. It was obvious that it
ought to be divided into three equal parts. Suddenly, as though he were
listening to somebody else, Winston heard himself demanding in a loud
booming voice that he should be given the whole piece. His mother told
him not to be greedy. There was a long, nagging argument that went round
and round, with shouts, whines, tears, remonstrances, bargainings. His
tiny sister, clinging to her mother with both hands, exactly like a baby
monkey, sat looking over her shoulder at him with large, mournful eyes.
In the end his mother broke off three-quarters of the chocolate and gave
it to Winston, giving the other quarter to his sister. The little girl
took hold of it and looked at it dully, perhaps not knowing what it was.
Winston stood watching her for a moment. Then with a sudden swift spring
he had snatched the piece of chocolate out of his
sister\textquotesingle s hand and was fleeing for the door.

``Winston, Winston!'' his mother called after him. ``Come back! Give your
sister back her chocolate!''

He stopped, but he did not come back. His mother\textquotesingle s
anxious eyes were fixed on his face. Even now she was thinking about the
thing, he did not know what it was, that was on the point of happening.
His sister, conscious of having been robbed of something, had set up a
feeble wail. His mother drew her arm round the child and pressed its
face against her breast. Something in the gesture told him that his
sister was dying. He turned and fled down the stairs, with the chocolate
growing sticky in his hand.

He never saw his mother again. After he had devoured the chocolate he
felt somewhat ashamed of himself and hung about in the streets for
several hours, until hunger drove him home. When he came back his mother
had disappeared. This was already becoming normal at that time. Nothing
was gone from the room except his mother and his sister. They had not
taken any clothes, not even his mother\textquotesingle s overcoat. To
this day he did not know with any certainty that his mother was dead. It
was perfectly possible that she had merely been sent to a forced-labor
camp. As for his sister, she might have been removed, like Winston
himself, to one of the colonies for homeless children (Reclamation
Centers, they were called) which had grown up as a result of the civil
war; or she might have been sent to the labor camp along with his
mother, or simply left somewhere or other to die.

The dream was still vivid in his mind, especially the enveloping,
protecting gesture of the arm in which its whole meaning seemed to be
contained. His mind went back to another dream of two months ago.
Exactly as his mother had sat on the dingy white-quilted bed, with the
child clinging to her, so she had sat in the sunken ship, far underneath
him and drowning deeper every minute, but still looking up at him
through the darkening water.

He told Julia the story of his mother\textquotesingle s disappearance.
Without opening her eyes she rolled over and settled herself into a more
comfortable position.

``I expect you were a beastly little swine in those days,'' she said
indistinctly. ``All children are swine.''

``Yes. But the real point of the story---''

From her breathing it was evident that she was going off to sleep again.
He would have liked to continue talking about his mother. He did not
suppose, from what he could remember of her, that she had been an
unusual woman, still less an intelligent one; and yet she had possessed
a kind of nobility, a kind of purity, simply because the standards that
she obeyed were private ones. Her feelings were her own, and could not
be altered from outside. It would not have occurred to her that an
action which is ineffectual thereby becomes meaningless. If you loved
someone, you loved him, and when you had nothing else to give, you still
gave him love. When the last of the chocolate was gone, his mother had
clasped the child in her arms. It was no use, it changed nothing, it did
not produce more chocolate, it did not avert the child\textquotesingle s
death or her own; but it seemed natural to her to do it. The refugee
woman in the boat had also covered the little boy with her arm, which
was no more use against the bullets than a sheet of paper. The terrible
thing that the Party had done was to persuade you that mere impulses,
mere feelings, were of no account, while at the same time robbing you of
all power over the material world. When once you were in the grip of the
Party, what you felt or did not feel, what you did or refrained from
doing, made literally no difference. Whatever happened you vanished, and
neither you nor your actions were ever heard of again. You were lifted
clean out of the stream of history. And yet to the people of only two
generations ago, this would not have seemed all-important, because they
were not attempting to alter history. They were governed by private
loyalties which they did not question. What mattered were individual
relationships, and a completely helpless gesture, an embrace, a tear, a
word spoken to a dying man, could have value in itself. The proles, it
suddenly occurred to him, had remained in this condition. They were not
loyal to a party or a country or an idea, they were loyal to one
another. For the first time in his life he did not despise the proles or
think of them merely as an inert force which would one day spring to
life and regenerate the world. The proles had stayed human. They had not
become hardened inside. They had held onto the primitive emotions which
he himself had to relearn by conscious effort. And in thinking this he
remembered, without apparent relevance, how a few weeks ago he had seen
a severed hand lying on the pavement and had kicked it into the gutter
as though it had been a cabbage stalk.

``The proles are human beings,'' he said aloud. ``We are not human.''

``Why not?'' said Julia, who had woken up again.

He thought for a little while. ``Has it ever occurred to you,'' he said,
``that the best thing for us to do would be simply to walk out of here
before it\textquotesingle s too late, and never see each other again?''

``Yes, dear, it has occurred to me, several times. But
I\textquotesingle m not going to do it, all the same.''

``We\textquotesingle ve been lucky,'' he said, ``but it
can\textquotesingle t last much longer. You\textquotesingle re young.
You look normal and innocent. If you keep clear of people like me, you
might stay alive for another fifty years.''

``No. I\textquotesingle ve thought it all out. What you do,
I\textquotesingle m going to do. And don\textquotesingle t be too
downhearted. I\textquotesingle m rather good at staying alive.''

``We may be together for another six months---a
year---there\textquotesingle s no knowing. At the end
we\textquotesingle re certain to be apart. Do you realize how utterly
alone we shall be? When once they get hold of us there will be nothing,
literally nothing, that either of us can do for the other. If I confess,
they\textquotesingle ll shoot you, and if I refuse to confess
they\textquotesingle ll shoot you just the same. Nothing that I can do
or say, or stop myself from saying, will put off your death for as much
as five minutes. Neither of us will even know whether the other is alive
or dead. We shall be utterly without power of any kind. The one thing
that matters is that we shouldn\textquotesingle t betray one another,
although even that can\textquotesingle t make the slightest difference.''

``If you mean confessing,'' she said, ``we shall do that, right enough.
Everybody always confesses. You can\textquotesingle t help it. They
torture you.''

``I don\textquotesingle t mean confessing. Confession is not betrayal.
What you say or do doesn\textquotesingle t matter; only feelings matter.
If they could make me stop loving you---that would be the real
betrayal.''

She thought it over. ``They can\textquotesingle t do that,'' she said
finally. ``It\textquotesingle s the one thing they can\textquotesingle t
do. They can make you say anything---\emph{anything}---but they
can\textquotesingle t make you believe it. They can\textquotesingle t
get inside you.''

``No,'' he said a little more hopefully, ``no; that\textquotesingle s quite
true. They can\textquotesingle t get inside you. If you can \emph{feel}
that staying human is worth while, even when it can\textquotesingle t
have any result whatever, you\textquotesingle ve beaten them.''

He thought of the telescreen with its never-sleeping ear. They could spy
upon you night and day, but if you kept your head you could still outwit
them. With all their cleverness they had never mastered the secret of
finding out what another human being was thinking. Perhaps that was less
true when you were actually in their hands. One did not know what
happened inside the Ministry of Love, but it was possible to guess:
tortures, drags, delicate instruments that registered your nervous
reactions, gradual wearing-down by sleeplessness and solitude and
persistent questioning. Facts, at any rate, could not be kept hidden.
They could be tracked down by inquiry, they could be squeezed out of you
by torture. But if the object was not to stay alive but to stay human,
what difference did it ultimately make? They could not alter your
feelings; for that matter you could not alter them yourself, even if you
wanted to. They could lay bare in the utmost detail everything that you
had done or said or thought; but the inner heart, whose workings were
mysterious even to yourself, remained impregnable.


\section{VIII}\label{viii-1}

They had done it, they had done it at last!

The room they were standing in was long-shaped and softly lit. The
telescreen was dimmed to a low murmur; the richness of the dark-blue
carpet gave one the impression of treading on velvet. At the far end of
the room O\textquotesingle Brien was sitting at a table under a
green-shaded lamp, with a mass of papers on either side of him. He had
not bothered to look up when the servant showed Julia and Winston in.

Winston\textquotesingle s heart was thumping so hard that he doubted
whether he would be able to speak. They had done it, they had done it at
last, was all he could think. It had been a rash act to come here at
all, and sheer folly to arrive together; though it was true that they
had come by different routes and only met on
O\textquotesingle Brien\textquotesingle s doorstep. But merely to walk
into such a place needed an effort of the nerve. It was only on very
rare occasions that one saw inside the dwelling places of the Inner
Party, or even penetrated into the quarter of the town where they lived.
The whole atmosphere of the huge block of flats, the richness and
spaciousness of everything, the unfamiliar smells of good food and good
tobacco, the silent and incredibly rapid lifts sliding up and down, the
white-jacketed servants hurrying to and fro---everything was
intimidating. Although he had a good pretext for coming here, he was
haunted at every step by the fear that a black-uniformed guard would
suddenly appear from round the corner, demand his papers, and order him
to get out. O\textquotesingle Brien\textquotesingle s servant, however,
had admitted the two of them without demur. He was a small, dark-haired
man in a white jacket, with a diamond-shaped, completely expressionless
face which might have been that of a Chinese. The passage down which he
led them was softly carpeted, with cream-papered walls and white
wainscoting, all exquisitely clean. That too was intimidating. Winston
could not remember ever to have seen a passageway whose walls were not
grimy from the contact of human bodies.

O\textquotesingle Brien had a slip of paper between his fingers and
seemed to be studying it intently. His heavy face, bent down so that one
could see the line of the nose, looked both formidable and intelligent.
For perhaps twenty seconds he sat without stirring. Then he pulled the
speakwrite toward him and rapped out a message in the hybrid jargon of
the Ministries:

``Items one comma five comma seven approved fullwise stop suggestion
contained item six doubleplus ridiculous verging crimethink cancel stop
unproceed constructionwise antegetting plusful estimates machinery
overheads stop end message.''

He rose deliberately from his chair and came toward them across the
soundless carpet. A little of the official atmosphere seemed to have
fallen away from him with the Newspeak words, but his expression was
grimmer than usual, as though he were not pleased at being disturbed.
The terror that Winston already felt was suddenly shot through by a
streak of ordinary embarrassment. It seemed to him quite possible that
he had simply made a stupid mistake. For what evidence had he in reality
that O\textquotesingle Brien was any kind of political conspirator?
Nothing but a flash of the eyes and a single equivocal remark; beyond
that, only his own secret imaginings, founded on a dream. He could not
even fall back on the pretense that he had come to borrow the
dictionary, because in that case Julia\textquotesingle s presence was
impossible to explain. As O\textquotesingle Brien passed the telescreen
a thought seemed to strike him. He stopped, turned aside, and pressed a
switch on the wall. There was a sharp snap. The voice had stopped.

Julia uttered a tiny sound, a sort of squeak of surprise. Even in the
midst of his panic, Winston was too much taken aback to be able to hold
his tongue.

``You can turn it off!'' he said.

``Yes,'' said O\textquotesingle Brien, ``we can turn it off. We have that
privilege.''

He was opposite them now. His solid form towered over the pair of them,
and the expression on his face was still indecipherable. He was waiting,
somewhat sternly, for Winston to speak, but about what? Even now it was
quite conceivable that he was simply a busy man wondering irritably why
he had been interrupted. Nobody spoke. After the stopping of the
telescreen the room seemed deadly silent. The seconds marched past,
enormous. With difficulty Winston continued to keep his eyes fixed on
O\textquotesingle Brien\textquotesingle s. Then suddenly the grim face
broke down into what might have been the beginnings of a smile. With his
characteristic gesture O\textquotesingle Brien resettled his spectacles
on his nose.

``Shall I say it, or will you?'' he said.

``I will say it,'' said Winston promptly. ``That thing is really turned
off?''

``Yes, everything is turned off. We are alone.''

``We have come here because---''

He paused, realizing for the first time the vagueness of his own
motives. Since he did not in fact know what kind of help he expected
from O\textquotesingle Brien, it was not easy to say why he had come
here. He went on, conscious that what he was saying must sound both
feeble and pretentious:

``We believe that there is some kind of conspiracy, some kind of secret
organization working against the Party, and that you are involved in it.
We want to join it and work for it. We are enemies of the Party. We
disbelieve in the principles of Ingsoc. We are thought-criminals. We are
also adulterers. I tell you this because we want to put ourselves at
your mercy. If you want us to incriminate ourselves in any other way, we
are ready.''

He stopped and glanced over his shoulder, with the feeling that the door
had opened. Sure enough, the little yellow-faced servant had come in
without knocking. Winston saw that he was carrying a tray with a
decanter and glasses.

``Martin is one of us,'' said O\textquotesingle Brien impassively. ``Bring
the drinks over here, Martin. Put them on the round table. Have we
enough chairs? Then we may as well sit down and talk in comfort. Bring a
chair for yourself, Martin. This is business. You can stop being a
servant for the next ten minutes.''

The little man sat down, quite at his ease, and yet still with a
servantlike air, the air of a valet enjoying a privilege. Winston
regarded him out of the corner of his eye. It struck him that the
man\textquotesingle s whole life was playing a part, and that he felt it
to be dangerous to drop his assumed personality even for a moment.
O\textquotesingle Brien took the decanter by the neck and filled up the
glasses with a dark-red liquid. It aroused in Winston dim memories of
something seen long ago on a wall or a hoarding---a vast bottle composed
of electric lights which seemed to move up and down and pour its
contents into a glass. Seen from the top the stuff looked almost black,
but in the decanter it gleamed like a ruby. It had a sour-sweet smell.
He saw Julia pick up her glass and sniff at it with frank curiosity.

``It is called wine,'' said O\textquotesingle Brien with a faint smile.
``You will have read about it in books, no doubt. Not much of it gets to
the Outer Party, I am afraid.'' His face grew solemn again, and he raised
his glass: ``I think it is fitting that we should begin by drinking a
health. To our Leader: To Emmanuel Goldstein.''

Winston took up his glass with a certain eagerness. Wine was a thing he
had read and dreamed about. Like the glass paperweight or Mr.
Charrington\textquotesingle s half-remembered rhymes, it belonged to the
vanished, romantic past, the olden time as he liked to call it in his
secret thoughts. For some reason he had always thought of wine as having
an intensely sweet taste, like that of blackberry jam, and an immediate
intoxicating effect. Actually, when he came to swallow it, the stuff was
distinctly disappointing. The truth was that after years of gin drinking
he could barely taste it. He set down the empty glass.

``Then there is such a person as Goldstein?'' he said.

``Yes, there is such a person, and he is alive. Where, I do not know.''

``And the conspiracy---the organization? It is real? It is not simply an
invention of the Thought Police?''

``No, it is real. The Brotherhood, we call it. You will never learn much
more about the Brotherhood than that it exists and that you belong to
it. I will come back to that presently.'' He looked at his wristwatch.
``It is unwise even for members of the Inner Party to turn off the
telescreen for more than half an hour. You ought not to have come here
together, and you will have to leave separately. You, comrade---'' he
bowed his head to Julia---``will leave first. We have about twenty
minutes at our disposal. You will understand that I must start by asking
you certain questions. In general terms, what are you prepared to do?''

``Anything that we are capable of,'' said Winston.

O\textquotesingle Brien had turned himself a little in his chair so that
he was facing Winston. He almost ignored Julia, seeming to take it for
granted that Winston could speak for her. For a moment the lids flitted
down over his eyes. He began asking his questions in a low,
expressionless voice, as though this were a routine, a sort of
catechism, most of whose answers were known to him already.

``You are prepared to give your lives?''

``Yes.''

``You are prepared to commit murder?''

``Yes.''

``To commit acts of sabotage which may cause the death of hundreds of
innocent people?''

``Yes.''

``To betray your country to foreign powers?''

``Yes.''

``You are prepared to cheat, to forge, to blackmail, to corrupt the minds
of children, to distribute habit-forming drugs, to encourage
prostitution, to disseminate venereal diseases---to do anything which is
likely to cause demoralization and weaken the power of the Party?''

``Yes.''

``If, for example, it would somehow serve our interests to throw
sulphuric acid in a child\textquotesingle s face---are you prepared to
do that?''

``Yes.''

``You are prepared to lose your identity and live out the rest of your
life as a waiter or a dock worker?''

``Yes.''

``You are prepared to commit suicide, if and when we order you to do so?''

``Yes.''

``You are prepared, the two of you, to separate and never see one another
again?''

``No!'' broke in Julia.

It appeared to Winston that a long time passed before he answered. For a
moment he seemed even to have been deprived of the power of speech. His
tongue worked soundlessly, forming the opening syllables first of one
word, then of the other, over and over again. Until he had said it, he
did not know which word he was going to say. ``No,'' he said finally.

``You did well to tell me,'' said O\textquotesingle Brien. ``It is
necessary for us to know everything.''

He turned himself toward Julia and added in a voice with somewhat more
expression in it:

``Do you understand that even if he survives, it may be as a different
person? We may be obliged to give him a new identity. His face, his
movements, the shape of his hands, the color of his hair---even his
voice would be different. And you yourself might have become a different
person. Our surgeons can alter people beyond recognition. Sometimes it
is necessary. Sometimes we even amputate a limb.''

Winston could not help snatching another sidelong glance at
Martin\textquotesingle s Mongolian face. There were no scars that he
could see. Julia had turned a shade paler, so that her freckles were
showing, but she faced O\textquotesingle Brien boldly. She murmured
something that seemed to be assent.

``Good. Then that is settled.''

There was a silver box of cigarettes on the table. With a rather
absent-minded air O\textquotesingle Brien pushed them toward the others,
took one himself, then stood up and began to pace slowly to and fro, as
though he could think better standing. They were very good cigarettes,
very thick and well-packed, with an unfamiliar silkiness in the paper.
O\textquotesingle Brien looked at his wristwatch again.

``You had better go back to your pantry, Martin,'' he said. ``I shall
switch on in a quarter of an hour. Take a good look at these
comrades\textquotesingle{} faces before you go. You will be seeing them
again. I may not.''

Exactly as they had done at the front door, the little
man\textquotesingle s dark eyes flickered over their faces. There was
not a trace of friendliness in his manner. He was memorizing their
appearance, but he felt no interest in them, or appeared to feel none.
It occurred to Winston that a synthetic face was perhaps incapable of
changing its expression. Without speaking or giving any kind of
salutation, Martin went out, closing the door silently behind him.
O\textquotesingle Brien was strolling up and down, one hand in the
pocket of his black overalls, the other holding his cigarette.

``You understand,'' he said, ``that you will be fighting in the dark. You
will always be in the dark. You will receive orders and you will obey
them, without knowing why. Later I shall send you a book from which you
will learn the true nature of the society we live in, and the strategy
by which we shall destroy it. When you have read the book, you will be
full members of the Brotherhood. But between the general aims that we
are fighting for, and the immediate tasks of the moment, you will never
know anything. I tell you that the Brotherhood exists, but I cannot tell
you whether it numbers a hundred members, or ten million. From your
personal knowledge you will never be able to say that it numbers even as
many as a dozen. You will have three or four contacts, who will be
renewed from time to time as they disappear. As this was your first
contact, it will be preserved. When you receive orders, they will come
from me. If we find it necessary to communicate with you, it will be
through Martin. When you are finally caught, you will confess. That is
unavoidable. But you will have very little to confess, other than your
own actions. You will not be able to betray more than a handful of
unimportant people. Probably you will not even betray me. By that time I
may be dead, or I shall have become a different person, with a different
face.''

He continued to move to and fro over the soft carpet. In spite of the
bulkiness of his body there was a remarkable grace in his movements. It
came out even in the gesture with which he thrust a hand into his
pocket, or manipulated a cigarette. More even than of strength, he gave
an impression of confidence and of an understanding tinged by irony.
However much in earnest he might be, he had nothing of the
single-mindedness that belongs to a fanatic. When he spoke of murder,
suicide, venereal disease, amputated limbs, and altered faces, it was
with a faint air of persiflage. ``This is unavoidable,'' his voice seemed
to say; ``this is what we have got to do, unflinchingly. But this is not
what we shall be doing when life is worth living again.'' A wave of
admiration, almost of worship, flowed out from Winston toward
O\textquotesingle Brien. For the moment he had forgotten the shadowy
figure of Goldstein. When you looked at
O\textquotesingle Brien\textquotesingle s powerful shoulders and his
blunt-featured face, so ugly and yet so civilized, it was impossible to
believe that he could be defeated. There was no stratagem that he was
not equal to, no danger that he could not foresee. Even Julia seemed to
be impressed. She had let her cigarette go out and was listening
intently. O\textquotesingle Brien went on:

``You will have heard rumors of the existence of the Brotherhood. No
doubt you have formed your own picture of it. You have imagined,
probably, a huge underworld of conspirators, meeting secretly in
cellars, scribbling messages on walls, recognizing one another by code
words or by special movements of the hand. Nothing of the kind exists.
The members of the Brotherhood have no way of recognizing one another,
and it is impossible for any one member to be aware of the identity of
more than a very few others. Goldstein himself, if he fell into the
hands of the Thought Police, could not give them a complete list of
members, or any information that would lead them to a complete list. No
such list exists. The Brotherhood cannot be wiped out because it is not
an organization in the ordinary sense. Nothing holds it together except
an idea which is indestructible. You will never have anything to sustain
you except the idea. You will get no comradeship and no encouragement.
When finally you are caught, you will get no help. We never help our
members. At most, when it is absolutely necessary that someone should be
silenced, we are occasionally able to smuggle a razor blade into a
prisoner\textquotesingle s cell. You will have to get used to living
without results and without hope. You will work for a while, you will be
caught, you will confess, and then you will die. Those are the only
results that you will ever see. There is no possibility that any
perceptible change will happen within our own lifetime. We are the dead.
Our only true life is in the future. We shall take part in it as
handfuls of dust and splinters of bone. But how far away that future may
be, there is no knowing. It might be a thousand years. At present
nothing is possible except to extend the area of sanity little by
little. We cannot act collectively. We can only spread our knowledge
outwards from individual to individual, generation after generation. In
the face of the Thought Police, there is no other way.''

He halted and looked for the third time at his wristwatch.

``It is almost time for you to leave, comrade,'' he said to Julia. ``Wait.
The decanter is still half full.''

He filled the glasses and raised his own glass by the stem.

``What shall it be this time?'' he said, still with the same faint
suggestion of irony. ``To the confusion of the Thought Police? To the
death of Big Brother? To humanity? To the future?''

``To the past,'' said Winston.

``The past is more important,'' agreed O\textquotesingle Brien gravely.
They emptied their glasses, and a moment later Julia stood up to go.
O\textquotesingle Brien took a small box from the top of a cabinet and
handed her a flat white tablet which he told her to place on her tongue.
It was important, he said, not to go out smelling of wine: the lift
attendants were very observant. As soon as the door had shut behind her
he appeared to forget her existence. He took another pace or two up and
down, then stopped.

``There are details to be settled,'' he said. ``I assume that you have a
hiding place of some kind?''

Winston explained about the room over Mr. Charrington\textquotesingle s
shop.

``That will do for the moment. Later we will arrange something else for
you. It is important to change one\textquotesingle s hiding place
frequently. Meanwhile I shall send you a copy of \emph{the book}---''
even O\textquotesingle Brien, Winston noticed, seemed to pronounce the
words as though they were in italics---``Goldstein\textquotesingle s
book, you understand, as soon as possible. It may be some days before I
can get hold of one. There are not many in existence, as you can
imagine. The Thought Police hunts them down and destroys them almost as
fast as we can produce them. It makes very little difference. The book
is indestructible. If the last copy were gone, we could reproduce it
almost word for word. Do you carry a brief case to work with you?'' he
added.

``As a rule, yes.''

``What is it like?''

``Black, very shabby. With two straps.''

``Black, two straps, very shabby---good. One day in the fairly near
future---I cannot give a date---one of the messages among your
morning\textquotesingle s work will contain a misprinted word, and you
will have to ask for a repeat. On the following day you will go to work
without your brief case. At some time during the day, in the street, a
man will touch you on the arm and say, `I think you have
dropped your brief case.' The one he gives you will
contain a copy of Goldstein\textquotesingle s book. You will return it
within fourteen days.''

They were silent for a moment.

``There are a couple of minutes before you need go,'' said
O\textquotesingle Brien. ``We shall meet again---if we do meet again---''

Winston looked up at him. ``In the place where there is no darkness?'' he
said hesitantly.

O\textquotesingle Brien nodded without appearance of surprise. ``In the
place where there is no darkness,'' he said, as though he had recognized
the allusion. ``And in the meantime, is there anything that you wish to
say before you leave? Any message? Any question?''

Winston thought. There did not seem to be any further question that he
wanted to ask; still less did he feel any impulse to utter high-sounding
generalities. Instead of anything directly connected with
O\textquotesingle Brien or the Brotherhood, there came into his mind a
sort of composite picture of the dark bedroom where his mother had spent
her last days, and the little room over Mr.
Charrington\textquotesingle s shop, and the glass paperweight, and the
steel engraving in its rosewood frame. Almost at random he said:

``Did you ever happen to hear an old rhyme that begins \emph{Oranges and
lemons, say the bells of St. Clement\textquotesingle s}?''

Again O\textquotesingle Brien nodded. With a sort of grave courtesy he
completed the stanza:

\begin{quotation}
  \noindent ``Oranges and lemons, say the bells of St. Clement\textquotesingle s,\\
  You owe me three farthings, say the bells of St. Martin\textquotesingle s,\\
  When will you pay me? say the bells of Old Bailey,\\
  When I grow rich, say the bells of Shoreditch.''
\end{quotation}

``You knew the last line!'' said Winston.

``Yes, I knew the last line. And now, I am afraid, it is time for you to
go. But wait. You had better let me give you one of these tablets.''

As Winston stood up O\textquotesingle Brien held out a hand. His
powerful grip crushed the bones of Winston\textquotesingle s palm. At
the door Winston looked back, but O\textquotesingle Brien seemed already
to be in process of putting him out of mind. He was waiting with his
hand on the switch that controlled the telescreen. Beyond him Winston
could see the writing table with its green-shaded lamp and the
speakwrite and the wire baskets deep-laden with papers. The incident was
closed. Within thirty seconds, it occurred to him,
O\textquotesingle Brien would be back at his interrupted and important
work on behalf of the Party.


\section{IX}\label{ix}

Winston was gelatinous with fatigue. Gelatinous was the right word. It
had come into his head spontaneously. His body seemed to have not only
the weakness of a jelly, but its translucency. He felt that if he held
up his hand he would be able to see the light through it. All the blood
and lymph had been drained out of him by an enormous debauch of work,
leaving only a frail structure of nerves, bones, and skin. All
sensations seemed to be magnified. His overalls fretted his shoulders,
the pavement tickled his feet, even the opening and closing of a hand
was an effort that made his joints creak.

He had worked more than ninety hours in five days. So had everyone else
in the Ministry. Now it was all over, and he had literally nothing to
do, no Party work of any description, until tomorrow morning. He could
spend six hours in the hiding place and another nine in his own bed.
Slowly, in mild afternoon sunshine, he walked up a dingy street in the
direction of Mr. Charrington\textquotesingle s shop, keeping one eye
open for the patrols, but irrationally convinced that this afternoon
there was no danger of anyone interfering with him. The heavy brief case
that he was carrying bumped against his knees at each step, sending a
tingling sensation up and down the skin of his leg. Inside it was
\emph{the book}, which he had now had in his possession for six days and
had not yet opened, nor even looked at.

On the sixth day of Hate Week, after the processions, the speeches, the
shouting, the singing, the banners, the posters, the films, the
waxworks, the rolling of drums and squealing of trumpets, the tramp of
marching feet, the grinding of the caterpillars of tanks, the roar of
massed planes, the booming of guns---after six days of this, when the
great orgasm was quivering to its climax and the general hatred of
Eurasia had boiled up into such delirium that if the crowd could have
got their hands on the two thousand Eurasian war criminals who were to
be publicly hanged on the last day of the proceedings, they would
unquestionably have torn them to pieces---at just this moment it had
been announced that Oceania was not after all at war with Eurasia.
Oceania was at war with Eastasia. Eurasia was an ally.

There was, of course, no admission that any change had taken place.
Merely it became known, with extreme suddenness and everywhere at once,
that Eastasia and not Eurasia was the enemy. Winston was taking part in
a demonstration in one of the central London squares at the moment when
it happened. It was night, and the white faces and the scarlet banners
were luridly floodlit. The square was packed with several thousand
people, including a block of about a thousand schoolchildren in the
uniform of the Spies. On a scarlet-draped platform an orator of the
Inner Party, a small lean man with disproportionately long arms and a
large, bald skull over which a few lank locks straggled, was haranguing
the crowd. A little Rumpelstiltskin figure, contorted with hatred, he
gripped the neck of the microphone with one hand while the other,
enormous at the end of a bony arm, clawed the air menacingly above his
head. His voice, made metallic by the amplifiers, boomed forth an
endless catalogue of atrocities, massacres, deportations, lootings,
rapings, torture of prisoners, bombing of civilians, lying propaganda,
unjust aggressions, broken treaties. It was almost impossible to listen
to him without being first convinced and then maddened. At every few
moments the fury of the crowd boiled over and the voice of the speaker
was drowned by a wild beastlike roaring that rose uncontrollably from
thousands of throats. The most savage yells of all came from the
schoolchildren. The speech had been proceeding for perhaps twenty
minutes when a messenger hurried onto the platform and a scrap of paper
was slipped into the speaker\textquotesingle s hand. He unrolled and
read it without pausing in his speech. Nothing altered in his voice or
manner, or in the content of what he was saying, but suddenly the names
were different. Without words said, a wave of understanding rippled
through the crowd. Oceania was at war with Eastasia! The next moment
there was a tremendous commotion. The banners and posters with which the
square was decorated were all wrong! Quite half of them had the wrong
faces on them. It was sabotage! The agents of Goldstein had been at
work! There was a riotous interlude while posters were ripped from the
walls, banners torn to shreds and trampled underfoot. The Spies
performed prodigies of activity in clambering over the rooftops and
cutting the streamers that fluttered from the chimneys. But within two
or three minutes it was all over. The orator, still gripping the neck of
the microphone, his shoulders hunched forward, his free hand clawing at
the air, had gone straight on with his speech. One minute more, and the
feral roars of rage were again bursting from the crowd. The Hate
continued exactly as before, except that the target had been changed.

The thing that impressed Winston in looking back was that the speaker
had switched from one line to the other actually in mid-sentence, not
only without a pause, but without even breaking the syntax. But at the
moment he had other things to preoccupy him. It was during the moment of
disorder while the posters were being torn down that a man whose face he
did not see had tapped him on the shoulder and said, ``Excuse me, I think
you\textquotesingle ve dropped your brief case.'' He took the brief case
abstractedly, without speaking. He knew that it would be days before he
had an opportunity to look inside it. The instant that the demonstration
was over he went straight to the Ministry of Truth, though the time was
now nearly twenty-three hours. The entire staff of the Ministry had done
likewise. The orders already issuing from the telescreens, recalling
them to their posts, were hardly necessary.

Oceania was at war with Eastasia: Oceania had always been at war with
Eastasia. A large part of the political literature of five years was now
completely obsolete. Reports and records of all kinds, newspapers,
books, pamphlets, films, sound tracks, photographs---all had to be
rectified at lightning speed. Although no directive was ever issued, it
was known that the chiefs of the Department intended that within one
week no reference to the war with Eurasia, or the alliance with
Eastasia, should remain in existence anywhere. The work was
overwhelming, all the more so because the processes that it involved
could not be called by their true names. Everyone in the Records
Department worked eighteen hours in the twenty-four, with two three-hour
snatches of sleep. Mattresses were brought up from the cellars and
pitched all over the corridors; meals consisted of sandwiches and
Victory Coffee wheeled round on trolleys by attendants from the canteen.
Each time that Winston broke off for one of his spells of sleep he tried
to leave his desk clear of work, and each time that he crawled back,
sticky-eyed and aching, it was to find that another shower of paper
cylinders had covered the desk like a snowdrift, half burying the
speakwrite and overflowing onto the floor, so that the first job was
always to stack them into a neat-enough pile to give him room to work.
What was worst of all was that the work was by no means purely
mechanical. Often it was enough merely to substitute one name for
another, but any detailed report of events demanded care and
imagination. Even the geographical knowledge that one needed in
transferring the war from one part of the world to another was
considerable.

By the third day his eyes ached unbearably and his spectacles needed
wiping every few minutes. It was like struggling with some crushing
physical task, something which one had the right to refuse and which one
was nevertheless neurotically anxious to accomplish. In so far as he had
time to remember it, he was not troubled by the fact that every word he
murmured into the speakwrite, every stroke of his ink pencil, was a
deliberate lie. He was as anxious as anyone else in the Department that
the forgery should be perfect. On the morning of the sixth day the
dribble of cylinders slowed down. For as much as half an hour nothing
came out of the tube; then one more cylinder, then nothing. Everywhere
at about the same time the work was easing off. A deep and as it were
secret sigh went through the Department. A mighty deed, which could
never be mentioned, had been achieved. It was now impossible for any
human being to prove by documentary evidence that the war with Eurasia
had ever happened. At twelve hundred it was unexpectedly announced that
all workers in the Ministry were free till tomorrow morning. Winston,
still carrying the brief case containing \emph{the book}, which had
remained between his feet while he worked and under his body while he
slept, went home, shaved himself, and almost fell asleep in his bath,
although the water was barely more than tepid.

With a sort of voluptuous creaking in his joints he climbed the stair
above Mr. Charrington\textquotesingle s shop. He was tired, but not
sleepy any longer. He opened the window, lit the dirty little oilstove,
and put on a pan of water for coffee. Julia would arrive presently;
meanwhile there was \emph{the book}. He sat down in the sluttish
armchair and undid the straps of the brief case.

A heavy black volume, amateurishly bound, with no name or title on the
cover. The print also looked slightly irregular. The pages were worn at
the edges, and fell apart easily, as though the book had passed through
many hands. The inscription on the title page ran:

\begin{center}
THE THEORY AND PRACTICE\\
OF OLIGARCHICAL COLLECTIVISM\\
by\\
\textsc{Emmanuel Goldstein}
\end{center}

[Winston began reading.]

\begin{center}
\emph{Chapter 1.}\\
IGNORANCE IS STRENGTH.
\end{center}

\begin{quotation}
Throughout recorded time, and probably since the end of the Neolithic
Age, there have been three kinds of people in the world, the High, the
Middle, and the Low. They have been subdivided in many ways, they have
borne countless different names, and their relative numbers, as well as
their attitude toward one another, have varied from age to age; but the
essential structure of society has never altered. Even after enormous
upheavals and seemingly irrevocable changes, the same pattern has always
reasserted itself, just as a gyroscope will always return to
equilibrium, however far it is pushed one way or the other.\par
The aims of these three groups are entirely irreconcilable....
\end{quotation}

Winston stopped reading, chiefly in order to appreciate the fact that he
\emph{was} reading, in comfort and safety. He was alone: no telescreen,
no ear at the keyhole, no nervous impulse to glance over his shoulder or
cover the page with his hand. The sweet summer air played against his
cheek. From somewhere far away there floated the faint shouts of
children; in the room itself there was no sound except the insect voice
of the clock. He settled deeper into the armchair and put his feet up on
the fender. It was bliss, it was eternity. Suddenly, as one sometimes
does with a book of which one knows that one will ultimately read and
reread every word, he opened it at a different place and found himself
at the third chapter. He went on reading:

\begin{center}
Chapter 3.\par
WAR IS PEACE.
\end{center}

\begin{quotation}
The splitting-up of the world into three great superstates was an event
which could be and indeed was foreseen before the middle of the
twentieth century. With the absorption of Europe by Russia and of the
British Empire by the United States, two of the three existing powers,
Eurasia and Oceania, were already effectively in being. The third,
Eastasia, only emerged as a distinct unit after another decade of
confused fighting. The frontiers between the three superstates are in
some places arbitrary, and in others they fluctuate according to the
fortunes of war, but in general they follow geographical lines. Eurasia
comprises the whole of the northern part of the European and Asiatic
land-mass, from Portugal to the Bering Strait. Oceania comprises the
Americas, the Atlantic islands including the British Isles, Australasia,
and the southern portion of Africa. Eastasia, smaller than the others
and with a less definite western frontier, comprises China and the
countries to the south of it, the Japanese islands and a large but
fluctuating portion of Manchuria, Mongolia, and Tibet.
\end{quotation}

\begin{quotation}
In one combination or another, these three superstates are permanently
at war, and have been so for the past twenty-five years. War, however,
is no longer the desperate, annihilating struggle that it was in the
early decades of the twentieth century. It is a warfare of limited aims
between combatants who are unable to destroy one another, have no
material cause for fighting, and are not divided by any genuine
ideological difference. This is not to say that either the conduct of
war, or the prevailing attitude toward it, has become less bloodthirsty
or more chivalrous. On the contrary, war hysteria is continuous and
universal in all countries, and such acts as raping, looting, the
slaughter of children, the reduction of whole populations to slavery,
and reprisals against prisoners which extend even to boiling and burying
alive, are looked upon as normal, and, when they are committed by
one\textquotesingle s own side and not by the enemy, meritorious. But in
a physical sense war involves very small numbers of people, mostly
highly trained specialists, and causes comparatively few casualties. The
fighting, when there is any, takes place on the vague frontiers whose
whereabouts the average man can only guess at, or round the Floating
Fortresses which guard strategic spots on the sea lanes. In the centers
of civilization war means no more than a continuous shortage of
consumption goods, and the occasional crash of a rocket bomb which may
cause a few scores of deaths. War has in fact changed its character.
More exactly, the reasons for which war is waged have changed in their
order of importance. Motives which were already present to some small
extent in the great wars of the early twentieth century have now become
dominant and are consciously recognized and acted upon.
\end{quotation}

\begin{quotation}
To understand the nature of the present war---for in spite of the
regrouping which occurs every few years, it is always the same war---one
must realize in the first place that it is impossible for it to be
decisive. None of the three superstates could be definitely conquered
even by the other two in combination. They are too evenly matched, and
their natural defenses are too formidable. Eurasia is protected by its
vast land spaces, Oceania by the width of the Atlantic and the Pacific,
Eastasia by the fecundity and industriousness of its inhabitants.
Secondly, there is no longer, in a material sense, anything to fight
about. With the establishment of self-contained economies, in which
production and consumption are geared to one another, the scramble for
markets which was a main cause of previous wars has come to an end,
while the competition for raw materials is no longer a matter of life
and death. In any case, each of the three superstates is so vast that it
can obtain almost all the materials that it needs within its own
boundaries. In so far as the war has a direct economic purpose, it is a
war for labor power. Between the frontiers of the superstates, and not
permanently in the possession of any of them, there lies a rough
quadrilateral with its corners at Tangier, Brazzaville, Darwin, and Hong
Kong, containing within it about a fifth of the population of the earth.
It is for the possession of these thickly populated regions, and of the
northern ice cap, that the three powers are constantly struggling. In
practice no one power ever controls the whole of the disputed area.
Portions of it are constantly changing hands, and it is the chance of
seizing this or that fragment by a sudden stroke of treachery that
dictates the endless changes of alignment.
\end{quotation}

\begin{quotation}
All of the disputed territories contain valuable minerals, and some of
them yield important vegetable products such as rubber which in colder
climates it is necessary to synthesize by comparatively expensive
methods. But above all they contain a bottomless reserve of cheap labor.
Whichever power controls equatorial Africa, or the countries of the
Middle East, or Southern India, or the Indonesian Archipelago, disposes
also of the bodies of scores or hundreds of millions of ill-paid and
hard-working coolies. The inhabitants of these areas, reduced more or
less openly to the status of slaves, pass continually from conqueror to
conqueror, and are expended like so much coal or oil in the race to turn
out more armaments, to capture more territory, to control more labor
power, to turn out more armaments, to capture more territory, and so on
indefinitely. It should be noted that the fighting never really moves
beyond the edges of the disputed areas. The frontiers of Eurasia flow
back and forth between the basin of the Congo and the northern shore of
the Mediterranean; the islands of the Indian Ocean and the Pacific are
constantly being captured and recaptured by Oceania or by Eastasia; in
Mongolia the dividing line between Eurasia and Eastasia is never stable;
round the Pole all three powers lay claim to enormous territories which
in fact are largely uninhabited and unexplored; but the balance of power
always remains roughly even, and the territory which forms the heartland
of each superstate always remains inviolate. Moreover, the labor of the
exploited peoples round the Equator is not really necessary to the
world\textquotesingle s economy. They add nothing to the wealth of the
world, since whatever they produce is used for purposes of war, and the
object of waging a war is always to be in a better position in which to
wage another war. By their labor the slave populations allow the tempo
of continuous warfare to be speeded up. But if they did not exist, the
structure of world society, and the process by which it maintains
itself, would not be essentially different.
\end{quotation}

\begin{quotation}
The primary aim of modern warfare (in accordance with the principles of
\emph{doublethink}, this aim is simultaneously recognized and not
recognized by the directing brains of the Inner Party) is to use up the
products of the machine without raising the general standard of living.
Ever since the end of the nineteenth century, the problem of what to do
with the surplus of consumption goods has been latent in industrial
society. At present, when few human beings even have enough to eat, this
problem is obviously not urgent, and it might not have become so, even
if no artificial processes of destruction had been at work. The world of
today is a bare, hungry, dilapidated place compared with the world that
existed before 1914, and still more so if compared with the imaginary
future to which the people of that period looked forward. In the early
twentieth century, the vision of a future society unbelievably rich,
leisured, orderly and efficient---a glittering antiseptic world of glass
and steel and snow-white concrete---was part of the consciousness of
nearly every literate person. Science and technology were developing at
a prodigious speed, and it seemed natural to assume that they would go
on developing. This failed to happen, partly because of the
impoverishment caused by a long series of wars and revolutions, partly
because scientific and technical progress depended on the empirical
habit of thought, which could not survive in a strictly regimented
society. As a whole the world is more primitive today than it was fifty
years ago. Certain backward areas have advanced, and various devices,
always in some way connected with warfare and police espionage, have
been developed, but experiment and invention have largely stopped, and
the ravages of the atomic war of the Nineteen-fifties have never been
fully repaired. Nevertheless the dangers inherent in the machine are
still there. From the moment when the machine first made its appearance
it was clear to all thinking people that the need for human drudgery,
and therefore to a great extent for human inequality, had disappeared.
If the machine were used deliberately for that end, hunger, overwork,
dirt, illiteracy, and disease could be eliminated within a few
generations. And in fact, without being used for any such purpose, but
by a sort of automatic process---by producing wealth which it was
sometimes impossible not to distribute---the machine did raise the
living standards of the average human being very greatly over a period
of about fifty years at the end of the nineteenth and the beginning of
the twentieth centuries.
\end{quotation}

\begin{quotation}
But it was also clear that an all-round increase in wealth threatened
the destruction---indeed, in some sense was the destruction---of a
hierarchical society. In a world in which everyone worked short hours,
had enough to eat, lived in a house with a bathroom and a refrigerator,
and possessed a motorcar or even an airplane, the most obvious and
perhaps the most important form of inequality would already have
disappeared. If it once became general, wealth would confer no
distinction. It was possible, no doubt, to imagine a society in which
wealth, in the sense of personal possessions and luxuries, should be
evenly distributed, while \emph{power} remained in the hands of a small
privileged caste. But in practice such a society could not long remain
stable. For if leisure and security were enjoyed by all alike, the great
mass of human beings who are normally stupefied by poverty would become
literate and would learn to think for themselves; and when once they had
done this, they would sooner or later realize that the privileged
minority had no function, and they would sweep it away. In the long run,
a hierarchical society was only possible on a basis of poverty and
ignorance. To return to the agricultural past, as some thinkers about
the beginning of the twentieth century dreamed of doing, was not a
practicable solution. It conflicted with the tendency toward
mechanization which had become quasi-instinctive throughout almost the
whole world, and moreover, any country which remained industrially
backward was helpless in a military sense and was bound to be dominated,
directly or indirectly, by its more advanced rivals.\par
Nor was it a satisfactory solution to keep the masses in poverty by
restricting the output of goods. This happened to a great extent during
the final phase of capitalism, roughly between 1920 and 1940. The
economy of many countries was allowed to stagnate, land went out of
cultivation, capital equipment was not added to, great blocks of the
population were prevented from working and kept half alive by State
charity. But this, too, entailed military weakness, and since the
privations it inflicted were obviously unnecessary, it made opposition
inevitable. The problem was how to keep the wheels of industry turning
without increasing the real wealth of the world. Goods must be produced,
but they must not be distributed. And in practice the only way of
achieving this was by continuous warfare.
\end{quotation}

\begin{quotation}
The essential act of war is destruction, not necessarily of human lives,
but of the products of human labor. War is a way of shattering to
pieces, or pouring into the stratosphere, or sinking in the depths of
the sea, materials which might otherwise be used to make the masses too
comfortable, and hence, in the long run, too intelligent. Even when
weapons of war are not actually destroyed, their manufacture is still a
convenient way of expending labor power without producing anything that
can be consumed. A Floating Fortress, for example, has locked up in it
the labor that would build several hundred cargo ships. Ultimately it is
scrapped as obsolete, never having brought any material benefit to
anybody, and with further enormous labors another Floating Fortress is
built. In principle the war effort is always so planned as to eat up any
surplus that might exist after meeting the bare needs of the population.
In practice the needs of the population are always underestimated, with
the result that there is a chronic shortage of half the necessities of
life; but this is looked on as an advantage. It is deliberate policy to
keep even the favored groups somewhere near the brink of hardship,
because a general state of scarcity increases the importance of small
privileges and thus magnifies the distinction between one group and
another. By the standards of the early twentieth century, even a member
of the Inner Party lives an austere, laborious kind of life.
Nevertheless, the few luxuries that he does enjoy---his large
well-appointed flat, the better texture of his clothes, the better
quality of his food and drink and tobacco, his two or three servants,
his private motorcar or helicopter---set him in a different world from a
member of the Outer Party, and the members of the Outer Party have a
similar advantage in comparison with the submerged masses whom we call
``the proles.'' The social atmosphere is that of a besieged city, where
the possession of a lump of horseflesh makes the difference between
wealth and poverty. And at the same time the consciousness of being at
war, and therefore in danger, makes the handing-over of all power to a
small caste seem the natural, unavoidable condition of survival.
\end{quotation}

\begin{quotation}
War, it will be seen, not only accomplishes the necessary destruction,
but accomplishes it in a psychologically acceptable way. In principle it
would be quite simple to waste the surplus labor of the world by
building temples and pyramids, by digging holes and filling them up
again, or even by producing vast quantities of goods and then setting
fire to them. But this would provide only the economic and not the
emotional basis for a hierarchical society. What is concerned here is
not the morale of the masses, whose attitude is unimportant so long as
they are kept steadily at work, but the morale of the Party itself. Even
the humblest Party member is expected to be competent, industrious, and
even intelligent within narrow limits, but it is also necessary that he
should be a credulous and ignorant fanatic whose prevailing moods are
fear, hatred, adulation, and orgiastic triumph. In other words it is
necessary that he should have the mentality appropriate to a state of
war. It does not matter whether the war is actually happening, and,
since no decisive victory is possible, it does not matter whether the
war is going well or badly. All that is needed is that a state of war
should exist. The splitting of the intelligence which the Party requires
of its members, and which is more easily achieved in an atmosphere of
war, is now almost universal, but the higher up the ranks one goes, the
more marked it becomes. It is precisely in the Inner Party that war
hysteria and hatred of the enemy are strongest. In his capacity as an
administrator, it is often necessary for a member of the Inner Party to
know that this or that item of war news is untruthful, and he may often
be aware that the entire war is spurious and is either not happening or
is being waged for purposes quite other than the declared ones; but such
knowledge is easily neutralized by the technique of \emph{doublethink}.
Meanwhile no Inner Party member wavers for an instant in his mystical
belief that the war is real, and that it is bound to end victoriously,
with Oceania the undisputed master of the entire world.
\end{quotation}

\begin{quotation}
All members of the Inner Party believe in this coming conquest as an
article of faith. It is to be achieved either by gradually acquiring
more and more territory and so building up an overwhelming preponderance
of power, or by the discovery of some new and unanswerable weapon. The
search for new weapons continues unceasingly, and is one of the very few
remaining activities in which the inventive or speculative type of mind
can find any outlet. In Oceania at the present day, Science, in the old
sense, has almost ceased to exist. In Newspeak there is no word for
``Science.'' The empirical method of thought, on which all the scientific
achievements of the past were founded, is opposed to the most
fundamental principles of Ingsoc. And even technological progress only
happens when its products can in some way be used for the diminution of
human liberty. In all the useful arts the world is either standing still
or going backwards. The fields are cultivated with horse plows while
books are written by machinery. But in matters of vital
importance---meaning, in effect, war and police espionage---the
empirical approach is still encouraged, or at least tolerated. The two
aims of the Party are to conquer the whole surface of the earth and to
extinguish once and for all the possibility of independent thought.
There are therefore two great problems which the Party is concerned to
solve. One is how to discover, against his will, what another human
being is thinking, and the other is how to kill several hundred million
people in a few seconds without giving warning beforehand. In so far as
scientific research still continues, this is its subject matter. The
scientist of today is either a mixture of psychologist and inquisitor,
studying with extraordinary minuteness the meaning of facial
expressions, gestures, and tones of voice, and testing the
truth-producing effects of drugs, shock therapy, hypnosis, and physical
torture; or he is a chemist, physicist, or biologist concerned only with
such branches of his special subject as are relevant to the taking of
life. In the vast laboratories of the Ministry of Peace, and in the
experimental stations hidden in the Brazilian forests, or in the
Australian desert, or on lost islands of the Antarctic, the teams of
experts are indefatigably at work. Some are concerned simply with
planning the logistics of future wars; others devise larger and larger
rocket bombs, more and more powerful explosives, and more and more
impenetrable armor-plating; others search for new and deadlier gases, or
for soluble poisons capable of being produced in such quantities as to
destroy the vegetation of whole continents, or for breeds of disease
germs immunized against all possible antibodies; others strive to
produce a vehicle that shall bore its way under the soil like a
submarine under the water, or an airplane as independent of its base as
a sailing ship; others explore even remoter possibilities such as
focusing the sun\textquotesingle s rays through lenses suspended
thousands of kilometers away in space, or producing artificial
earthquakes and tidal waves by tapping the heat at the
earth\textquotesingle s center.
\end{quotation}

\begin{quotation}
But none of these projects ever comes anywhere near realization, and
none of the three superstates ever gains a significant lead on the
others. What is more remarkable is that all three powers already
possess, in the atomic bomb, a weapon far more powerful than any that
their present researches are likely to discover. Although the Party,
according to its habit, claims the invention for itself, atomic bombs
first appeared as early as the Nineteen-forties, and were first used on
a large scale about ten years later. At that time some hundreds of bombs
were dropped on industrial centers, chiefly in European Russia, Western
Europe, and North America. The effect was to convince the ruling groups
of all countries that a few more atomic bombs would mean the end of
organized society, and hence of their own power. Thereafter, although no
formal agreement was ever made or hinted at, no more bombs were dropped.
All three powers merely continue to produce atomic bombs and store them
up against the decisive opportunity which they all believe will come
sooner or later. And meanwhile the art of war has remained almost
stationary for thirty or forty years. Helicopters are more used than
they were formerly, bombing planes have been largely superseded by
self-propelled projectiles, and the fragile movable battleship has given
way to the almost unsinkable Floating Fortress; but otherwise there has
been little development. The tank, the submarine, the torpedo, the
machine gun, even the rifle and the hand grenade are still in use. And
in spite of the endless slaughters reported in the press and on the
telescreens, the desperate battles of earlier wars, in which hundreds of
thousands or even millions of men were often killed in a few weeks, have
never been repeated.
\end{quotation}

\begin{quotation}
None of the three superstates ever attempts any maneuver which involves
the risk of serious defeat. When any large operation is undertaken, it
is usually a surprise attack against an ally. The strategy that all
three powers are following, or pretend to themselves that they are
following, is the same. The plan is, by a combination of fighting,
bargaining, and well-timed strokes of treachery, to acquire a ring of
bases completely encircling one or other of the rival states, and then
to sign a pact of friendship with that rival and remain on peaceful
terms for so many years as to lull suspicion to sleep. During this time
rockets loaded with atomic bombs can be assembled at all the strategic
spots; finally they will all be fired simultaneously, with effects so
devastating as to make retaliation impossible. It will then be time to
sign a pact of friendship with the remaining world power, in preparation
for another attack. This scheme, it is hardly necessary to say, is a
mere daydream, impossible of realization. Moreover, no fighting ever
occurs except in the disputed areas round the Equator and the Pole; no
invasion of enemy territory is ever undertaken. This explains the fact
that in some places the frontiers between the superstates are arbitrary.
Eurasia, for example, could easily conquer the British Isles, which are
geographically part of Europe, or on the other hand it would be possible
for Oceania to push its frontiers to the Rhine or even to the Vistula.
But this would violate the principle, followed on all sides though never
formulated, of cultural integrity. If Oceania were to conquer the areas
that used once to be known as France and Germany, it would be necessary
either to exterminate the inhabitants, a task of great physical
difficulty, or to assimilate a population of about a hundred million
people, who, so far as technical development goes, are roughly on the
Oceanic level. The problem is the same for all three superstates. It is
absolutely necessary to their structure that there should be no contact
with foreigners except, to a limited extent, with war prisoners and
colored slaves. Even the official ally of the moment is always regarded
with the darkest suspicion. War prisoners apart, the average citizen of
Oceania never sets eyes on a citizen of either Eurasia or Eastasia, and
he is forbidden the knowledge of foreign languages. If he were allowed
contact with foreigners he would discover that they are creatures
similar to himself and that most of what he has been told about them is
lies. The sealed world in which he lives would be broken, and the fear,
hatred, and self-righteousness on which his morale depends might
evaporate. It is therefore realized on all sides that however often
Persia, or Egypt, or Java, or Ceylon may change hands, the main
frontiers must never be crossed by anything except bombs.
\end{quotation}

\begin{quotation}
Under this lies a fact never mentioned aloud, but tacitly understood and
acted upon: namely, that the conditions of life in all three superstates
are very much the same. In Oceania the prevailing philosophy is called
Ingsoc, in Eurasia it is called Neo-Bolshevism, and in Eastasia it is
called by a Chinese name usually translated as Death-worship, but
perhaps better rendered as Obliteration of the Self. The citizen of
Oceania is not allowed to know anything of the tenets of the other two
philosophies, but he is taught to execrate them as barbarous outrages
upon morality and common sense. Actually the three philosophies are
barely distinguishable, and the social systems which they support are
not distinguishable at all. Everywhere there is the same pyramidal
structure, the same worship of a semi-divine leader, the same economy
existing by and for continuous warfare. It follows that the three
superstates not only cannot conquer one another, but would gain no
advantage by doing so. On the contrary, so long as they remain in
conflict they prop one another up, like three sheaves of corn. And, as
usual, the ruling groups of all three powers are simultaneously aware
and unaware of what they are doing. Their lives are dedicated to world
conquest, but they also know that it is necessary that the war should
continue everlastingly and without victory. Meanwhile the fact that
there is no danger of conquest makes possible the denial of reality
which is the special feature of Ingsoc and its rival systems of thought.
Here it is necessary to repeat what has been said earlier, that by
becoming continuous war has fundamentally changed its character.
\end{quotation}

\begin{quotation}
In past ages, a war, almost by definition, was something that sooner or
later came to an end, usually in unmistakable victory or defeat. In the
past, also, war was one of the main instruments by which human societies
were kept in touch with physical reality. All rulers in all ages have
tried to impose a false view of the world upon their followers, but they
could not afford to encourage any illusion that tended to impair
military efficiency. So long as defeat meant the loss of independence,
or some other result generally held to be undesirable, the precautions
against defeat had to be serious. Physical facts could not be ignored.
In philosophy, or religion, or ethics, or politics, two and two might
make five, but when one was designing a gun or an airplane they had to
make four. Inefficient nations were always conquered sooner or later,
and the struggle for efficiency was inimical to illusions. Moreover, to
be efficient it was necessary to be able to learn from the past, which
meant having a fairly accurate idea of what had happened in the past.
Newspapers and history books were, of course, always colored and biased,
but falsification of the kind that is practiced today would have been
impossible. War was a sure safeguard of sanity, and so far as the ruling
classes were concerned it was probably the most important of all
safeguards. While wars could be won or lost, no ruling class could be
completely irresponsible.
\end{quotation}

\begin{quotation}
But when war becomes literally continuous, it also ceases to be
dangerous. When war is continuous there is no such thing as military
necessity. Technical progress can cease and the most palpable facts can
be denied or disregarded. As we have seen, researches that could be
called scientific are still carried out for the purposes of war, but
they are essentially a kind of daydreaming, and their failure to show
results is not important. Efficiency, even military efficiency, is no
longer needed. Nothing is efficient in Oceania except the Thought
Police. Since each of the three superstates is unconquerable, each is in
effect a separate universe within which almost any perversion of thought
can be safely practiced. Reality only exerts its pressure through the
needs of everyday life---the need to eat and drink, to get shelter and
clothing, to avoid swallowing poison or stepping out of top-story
windows, and the like. Between life and death, and between physical
pleasure and physical pain, there is still a distinction, but that is
all. Cut off from contact with the outer world, and with the past, the
citizen of Oceania is like a man in interstellar space, who has no way
of knowing which direction is up and which is down. The rulers of such a
state are absolute, as the Pharaohs or the Caesars could not be. They
are obliged to prevent their followers from starving to death in numbers
large enough to be inconvenient, and they are obliged to remain at the
same low level of military technique as their rivals; but once that
minimum is achieved, they can twist reality into whatever shape they
choose.
\end{quotation}

\begin{quotation}
The war, therefore, if we judge it by the standards of previous wars, is
merely an imposture. It is like the battles between certain ruminant
animals whose horns are set at such an angle that they are incapable of
hurting one another. But though it is unreal it is not meaningless. It
eats up the surplus of consumable goods, and it helps to preserve the
special mental atmosphere that a hierarchical society needs. War, it
will be seen, is now a purely internal affair. In the past, the ruling
groups of all countries, although they might recognize their common
interest and therefore limit the destructiveness of war, did fight
against one another, and the victor always plundered the vanquished. In
our own day they are not fighting against one another at all. The war is
waged by each ruling group against its own subjects, and the object of
the war is not to make or prevent conquests of territory, but to keep
the structure of society intact. The very word ``war,'' therefore, has
become misleading. It would probably be accurate to say that by becoming
continuous war has ceased to exist. The peculiar pressure that it
exerted on human beings between the Neolithic Age and the early
twentieth century has disappeared and been replaced by something quite
different. The effect would be much the same if the three superstates,
instead of fighting one another, should agree to live in perpetual
peace, each inviolate within its own boundaries. For in that case each
would still be a self-contained universe, freed forever from the
sobering influence of external danger. A peace that was truly permanent
would be the same as a permanent war. This---although the vast majority
of Party members understand it only in a shallower sense---is the inner
meaning of the Party slogan: \textsc{WAR IS PEACE}.
\end{quotation}

Winston stopped reading for a moment. Somewhere in remote distance a
rocket bomb thundered. The blissful feeling of being alone with the
forbidden book, in a room with no telescreen, had not worn off. Solitude
and safety were physical sensations, mixed up somehow with the tiredness
of his body, the softness of the chair, the touch of the faint breeze
from the window that played upon his cheek. The book fascinated him, or
more exactly it reassured him. In a sense it told him nothing that was
new, but that was part of the attraction. It said what he would have
said, if it had been possible for him to set his scattered thoughts in
order. It was the product of a mind similar to his own, but enormously
more powerful, more systematic, less fear-ridden. The best books, he
perceived, are those that tell you what you know already. He had just
turned back to Chapter 1 when he heard Julia\textquotesingle s footstep
on the stair and started out of his chair to meet her. She dumped her
brown tool bag on the floor and flung herself into his arms. It was more
than a week since they had seen one another.

``I\textquotesingle ve got \emph{the book},'' he said as they disentangled
themselves.

``Oh, you\textquotesingle ve got it? Good,'' she said without much
interest, and almost immediately knelt down beside the oilstove to make
the coffee.

They did not return to the subject until they had been in bed for half
an hour. The evening was just cool enough to make it worth while to pull
up the counterpane. From below came the familiar sound of singing and
the scrape of boots on the flagstones. The brawny red-armed woman whom
Winston had seen there on his first visit was almost a fixture in the
yard. There seemed to be no hour of daylight when she was not marching
to and fro between the washtub and the line, alternately gagging herself
with clothes pegs and breaking forth into lusty song. Julia had settled
down on her side and seemed to be already on the point of falling
asleep. He reached out for the book, which was lying on the floor, and
sat up against the bedhead.

``We must read it,'' he said. ``You too. All members of the Brotherhood
have to read it.''

``You read it,'' she said with her eyes shut. ``Read it aloud.
That\textquotesingle s the best way. Then you can explain it to me as
you go.''

The clock\textquotesingle s hands said six, meaning eighteen. They had
three or four hours ahead of them. He propped the book against his knees
and began reading:
\begin{center}
\emph{Chapter 1.}

IGNORANCE IS STRENGTH.
\end{center}

\begin{quotation}
Throughout recorded time, and probably since the end of the Neolithic
Age, there have been three kinds of people in the world, the High, the
Middle, and the Low. They have been subdivided in many ways, they have
borne countless different names, and their relative numbers, as well as
their attitude toward one another, have varied from age to age; but the
essential structure of society has never altered. Even after enormous
upheavals and seemingly irrevocable changes, the same pattern has always
reasserted itself, just as a gyroscope will always return to
equilibrium, however far it is pushed one way or the other.
\end{quotation}

``Julia, are you awake?'' said Winston.

``Yes, my love, I\textquotesingle m listening. Go on.
It\textquotesingle s marvelous.''

He continued reading:

\begin{quotation}
The aims of these three groups are entirely irreconcilable. The aim of
the High is to remain where they are. The aim of the Middle is to change
places with the High. The aim of the Low, when they have an aim---for it
is an abiding characteristic of the Low that they are too much crushed
by drudgery to be more than intermittently conscious of anything outside
their daily lives---is to abolish all distinctions and create a society
in which all men shall be equal. Thus throughout history a struggle
which is the same in its main outlines recurs over and over again. For
long periods the High seem to be securely in power, but sooner or later
there always comes a moment when they lose either their belief in
themselves, or their capacity to govern efficiently, or both. They are
then overthrown by the Middle, who enlist the Low on their side by
pretending to them that they are fighting for liberty and justice. As
soon as they have reached their objective, the Middle thrust the Low
back into their old position of servitude, and themselves become the
High. Presently a new Middle group splits off from one of the other
groups, or from both of them, and the struggle begins over again. Of the
three groups, only the Low are never even temporarily successful in
achieving their aims. It would be an exaggeration to say that throughout
history there has been no progress of a material kind. Even today, in a
period of decline, the average human being is physically better off than
he was a few centuries ago. But no advance in wealth, no softening of
manners, no reform or revolution has ever brought human equality a
millimeter nearer. From the point of view of the Low, no historic change
has ever meant much more than a change in the name of their masters.

By the late nineteenth century the recurrence of this pattern had become
obvious to many observers. There then arose schools of thinkers who
interpreted history as a cyclical process and claimed to show that
inequality was the unalterable law of human life. This doctrine, of
course, had always had its adherents, but in the manner in which it was
now put forward there was a significant change. In the past the need for
a hierarchical form of society had been the doctrine specifically of the
High. It had been preached by kings and aristocrats and by the priests,
lawyers, and the like who were parasitical upon them, and it had
generally been softened by promises of compensation in an imaginary
world beyond the grave. The Middle, so long as it was struggling for
power, had always made use of such terms as freedom, justice, and
fraternity. Now, however, the concept of human brotherhood began to be
assailed by people who were not yet in positions of command, but merely
hoped to be so before long. In the past the Middle had made revolutions
under the banner of equality, and then had established a fresh tyranny
as soon as the old one was overthrown. The new Middle groups in effect
proclaimed their tyranny beforehand. Socialism, a theory which appeared
in the early nineteenth century and was the last link in a chain of
thought stretching back to the slave rebellions of antiquity, was still
deeply infected by the Utopianism of past ages. But in each variant of
Socialism that appeared from about 1900 onwards the aim of establishing
liberty and equality was more and more openly abandoned. The new
movements which appeared in the middle years of the century, Ingsoc in
Oceania, Neo-Bolshevism in Eurasia, Death-worship, as it is commonly
called, in Eastasia, had the conscious aim of perpetuating unfreedom and
inequality. These new movements, of course, grew out of the old ones and
tended to keep their names and pay lip-service to their ideology. But
the purpose of all of them was to arrest progress and freeze history at
a chosen moment. The familiar pendulum swing was to happen once more,
and then stop. As usual, the High were to be turned out by the Middle,
who would then become the High; but this time, by conscious strategy,
the High would be able to maintain their position permanently.

The new doctrines arose partly because of the accumulation of historical
knowledge, and the growth of the historical sense, which had hardly
existed before the nineteenth century. The cyclical movement of history
was now intelligible, or appeared to be so; and if it was intelligible,
then it was alterable. But the principal, underlying cause was that, as
early as the beginning of the twentieth century, human equality had
become technically possible. It was still true that men were not equal
in their native talents and that functions had to be specialized in ways
that favored some individuals against others; but there was no longer
any real need for class distinctions or for large differences of wealth.
In earlier ages, class distinctions had been not only inevitable but
desirable. Inequality was the price of civilization. With the
development of machine production, however, the case was altered. Even
if it was still necessary for human beings to do different kinds of
work, it was no longer necessary for them to live at different social or
economic levels. Therefore, from the point of view of the new groups who
were on the point of seizing power, human equality was no longer an
ideal to be striven after, but a danger to be averted. In more primitive
ages, when a just and peaceful society was in fact not possible, it had
been fairly easy to believe in it. The idea of an earthly paradise in
which men should live together in a state of brotherhood, without laws
and without brute labor, had haunted the human imagination for thousands
of years. And this vision had had a certain hold even on the groups who
actually profited by each historic change. The heirs of the French,
English, and American revolutions had partly believed in their own
phrases about the rights of man, freedom of speech, equality before the
law, and the like, and had even allowed their conduct to be influenced
by them to some extent. But by the fourth decade of the twentieth
century all the main currents of political thought were authoritarian.
The earthly paradise had been discredited at exactly the moment when it
became realizable. Every new political theory, by whatever name it
called itself, led back to hierarchy and regimentation. And in the
general hardening of outlook that set in round about 1930, practices
which had been long abandoned, in some cases for hundreds of
years---imprisonment without trial, the use of war prisoners as slaves,
public executions, torture to extract confessions, the use of hostages
and the deportation of whole populations---not only became common again,
but were tolerated and even defended by people who considered themselves
enlightened and progressive.

It was only after a decade of national wars, civil wars, revolutions and
counterrevolutions in all parts of the world that Ingsoc and its rivals
emerged as fully worked-out political theories. But they had been
foreshadowed by the various systems, generally called totalitarian,
which had appeared earlier in the century, and the main outlines of the
world which would emerge from the prevailing chaos had long been
obvious. What kind of people would control this world had been equally
obvious. The new aristocracy was made up for the most part of
bureaucrats, scientists, technicians, trade-union organizers, publicity
experts, sociologists, teachers, journalists, and professional
politicians. These people, whose origins lay in the salaried middle
class and the upper grades of the working class, had been shaped and
brought together by the barren world of monopoly industry and
centralized government. As compared with their opposite numbers in past
ages, they were less avaricious, less tempted by luxury, hungrier for
pure power, and, above all, more conscious of what they were doing and
more intent on crushing opposition. This last difference was cardinal.
By comparison with that existing today, all the tyrannies of the past
were half-hearted and inefficient. The ruling groups were always
infected to some extent by liberal ideas, and were content to leave
loose ends everywhere, to regard only the overt act, and to be
uninterested in what their subjects were thinking. Even the Catholic
Church of the Middle Ages was tolerant by modern standards. Part of the
reason for this was that in the past no government had the power to keep
its citizens under constant surveillance. The invention of print,
however, made it easier to manipulate public opinion, and the film and
the radio carried the process further. With the development of
television, and the technical advance which made it possible to receive
and transmit simultaneously on the same instrument, private life came to
an end. Every citizen, or at least every citizen important enough to be
worth watching, could be kept for twenty-four hours a day under the eyes
of the police and in the sound of official propaganda, with all other
channels of communication closed. The possibility of enforcing not only
complete obedience to the will of the State, but complete uniformity of
opinion on all subjects, now existed for the first time.

After the revolutionary period of the Fifties and Sixties, society
regrouped itself, as always, into High, Middle, and Low. But the new
High group, unlike all its forerunners, did not act upon instinct but
knew what was needed to safeguard its position. It had long been
realized that the only secure basis for oligarchy is collectivism.
Wealth and privilege are most easily defended when they are possessed
jointly. The so-called ``abolition of private property'' which took place
in the middle years of the century meant, in effect, the concentration
of property in far fewer hands than before; but with this difference,
that the new owners were a group instead of a mass of individuals.
Individually, no member of the Party owns anything, except petty
personal belongings. Collectively, the Party owns everything in Oceania,
because it controls everything and disposes of the products as it thinks
fit. In the years following the Revolution it was able to step into this
commanding position almost unopposed, because the whole process was
represented as an act of collectivization. It had always been assumed
that if the capitalist class were expropriated, Socialism must follow;
and unquestionably the capitalists had been expropriated. Factories,
mines, land, houses, transport---everything had been taken away from
them; and since these things were no longer private property, it
followed that they must be public property. Ingsoc, which grew out of
the earlier Socialist movement and inherited its phraseology, has in
fact carried out the main item in the Socialist program, with the
result, foreseen and intended beforehand, that economic inequality has
been made permanent.

But the problems of perpetuating a hierarchical society go deeper than
this. There are only four ways in which a ruling group can fall from
power. Either it is conquered from without, or it governs so
inefficiently that the masses are stirred to revolt, or it allows a
strong and discontented Middle Group to come into being, or it loses its
own self-confidence and willingness to govern. These causes do not
operate singly, and as a rule all four of them are present in some
degree. A ruling class which could guard against all of them would
remain in power permanently. Ultimately the determining factor is the
mental attitude of the ruling class itself.

After the middle of the present century, the first danger had in reality
disappeared. Each of the three powers which now divide the world is in
fact unconquerable, and could only become conquerable through slow
demographic changes which a government with wide powers can easily
avert. The second danger, also, is only a theoretical one. The masses
never revolt of their own accord, and they never revolt merely because
they are oppressed. Indeed, so long as they are not permitted to have
standards of comparison, they never even become aware that they are
oppressed. The recurrent economic crises of past times were totally
unnecessary and are not now permitted to happen, but other and equally
large dislocations can and do happen without having political results,
because there is no way in which discontent can become articulate. As
for the problem of overproduction, which has been latent in our society
since the development of machine technique, it is solved by the device
of continuous warfare (\emph{see} Chapter 3), which is also useful in
keying up public morale to the necessary pitch. From the point of view
of our present rulers, therefore, the only genuine dangers are the
splitting-off of a new group of able, underemployed, power-hungry
people, and the growth of liberalism and skepticism in their own ranks.
The problem, that is to say, is educational. It is a problem of
continuously molding the consciousness both of the directing group and
of the larger executive group that lies immediately below it. The
consciousness of the masses needs only to be influenced in a negative
way.

Given this background, one could infer, if one did not know it already,
the general structure of Oceanic society. At the apex of the pyramid
comes Big Brother. Big Brother is infallible and all-powerful. Every
success, every achievement, every victory, every scientific discovery,
all knowledge, all wisdom, all happiness, all virtue, are held to issue
directly from his leadership and inspiration. Nobody has ever seen Big
Brother. He is a face on the hoardings, a voice on the telescreen. We
may be reasonably sure that he will never die, and there is already
considerable uncertainty as to when he was born. Big Brother is the
guise in which the Party chooses to exhibit itself to the world. His
function is to act as a focusing point for love, fear, and reverence,
emotions which are more easily felt toward an individual than toward an
organization. Below Big Brother comes the Inner Party, its numbers
limited to six millions, or something less than two per cent of the
population of Oceania. Below the Inner Party comes the Outer Party,
which, if the Inner Party is described as the brain of the State, may be
justly likened to the hands. Below that come the dumb masses whom we
habitually refer to as ``the proles,'' numbering perhaps eighty-five per
cent of the population. In the terms of our earlier classification, the
proles are the Low, for the slave populations of the equatorial lands,
who pass constantly from conqueror to conqueror, are not a permanent or
necessary part of the structure.

In principle, membership in these three groups is not hereditary. The
child of Inner Party parents is in theory not born into the Inner Party.
Admission to either branch of the Party is by examination, taken at the
age of sixteen. Nor is there any racial discrimination, or any marked
domination of one province by another. Jews, Negroes, South Americans of
pure Indian blood are to be found in the highest ranks of the Party, and
the administrators of any area are always drawn from the inhabitants of
that area. In no part of Oceania do the inhabitants have the feeling
that they are a colonial population ruled from a distant capital.
Oceania has no capital, and its titular head is a person whose
whereabouts nobody knows. Except that English is its chief lingua franca
and Newspeak its official language, it is not centralized in any way.
Its rulers are not held together by blood ties but by adherence to a
common doctrine. It is true that our society is stratified, and very
rigidly stratified, on what at first sight appear to be hereditary
lines. There is far less to-and-fro movement between the different
groups than happened under capitalism or even in the pre-industrial
ages. Between the two branches of the Party there is a certain amount of
interchange, but only so much as will ensure that weaklings are excluded
from the Inner Party and that ambitious members of the Outer Party are
made harmless by allowing them to rise. Proletarians, in practice, are
not allowed to graduate into the Party. The most gifted among them, who
might possibly become nuclei of discontent, are simply marked down by
the Thought Police and eliminated. But this state of affairs is not
necessarily permanent, nor is it a matter of principle. The Party is not
a class in the old sense of the word. It does not aim at transmitting
power to its own children, as such; and if there were no other way of
keeping the ablest people at the top, it would be perfectly prepared to
recruit an entire new generation from the ranks of the proletariat. In
the crucial years, the fact that the Party was not a hereditary body did
a great deal to neutralize opposition. The older kind of Socialist, who
had been trained to fight against something called ``class privilege,''
assumed that what is not hereditary cannot be permanent. He did not see
that the continuity of an oligarchy need not be physical, nor did he
pause to reflect that hereditary aristocracies have always been
short-lived, whereas adoptive organizations such as the Catholic Church
have sometimes lasted for hundreds or thousands of years. The essence of
oligarchical rule is not father-to-son inheritance, but the persistence
of a certain world-view and a certain way of life, imposed by the dead
upon the living. A ruling group is a ruling group so long as it can
nominate its successors. The Party is not concerned with perpetuating
its blood but with perpetuating itself. \emph{Who} wields power is not
important, provided that the hierarchical structure remains always the
same.

All the beliefs, habits, tastes, emotions, mental attitudes that
characterize our time are really designed to sustain the mystique of the
Party and prevent the true nature of present-day society from being
perceived. Physical rebellion, or any preliminary move toward rebellion,
is at present not possible. From the proletarians nothing is to be
feared. Left to themselves, they will continue from generation to
generation and from century to century, working, breeding, and dying,
not only without any impulse to rebel, but without the power of grasping
that the world could be other than it is. They could only become
dangerous if the advance of industrial technique made it necessary to
educate them more highly; but, since military and commercial rivalry are
no longer important, the level of popular education is actually
declining. What opinions the masses hold, or do not hold, is looked on
as a matter of indifference. They can be granted intellectual liberty
because they have no intellect. In a Party member, on the other hand,
not even the smallest deviation of opinion on the most unimportant
subject can be tolerated.

A Party member lives from birth to death under the eye of the Thought
Police. Even when he is alone he can never be sure that he is alone.
Wherever he may be, asleep or awake, working or resting, in his bath or
in bed, he can be inspected without warning and without knowing that he
is being inspected. Nothing that he does is indifferent. His
friendships, his relaxations, his behavior toward his wife and children,
the expression of his face when he is alone, the words he mutters in
sleep, even the characteristic movements of his body, are all jealously
scrutinized. Not only any actual misdemeanor, but any eccentricity,
however small, any change of habits, any nervous mannerism that could
possibly be the symptom of an inner struggle, is certain to be detected.
He has no freedom of choice in any direction whatever. On the other
hand, his actions are not regulated by law or by any clearly formulated
code of behavior. In Oceania there is no law. Thoughts and actions
which, when detected, mean certain death are not formally forbidden, and
the endless purges, arrests, tortures, imprisonments, and vaporizations
are not inflicted as punishment for crimes which have actually been
committed, but are merely the wiping-out of persons who might perhaps
commit a crime at some time in the future. A Party member is required to
have not only the right opinions, but the right instincts. Many of the
beliefs and attitudes demanded of him are never plainly stated, and
could not be stated without laying bare the contradictions inherent in
Ingsoc. If he is a person naturally orthodox (in Newspeak, a
\emph{goodthinker}), he will in all circumstances know, without taking
thought, what is the true belief or the desirable emotion. But in any
case an elaborate mental training, undergone in childhood and grouping
itself round the Newspeak words \emph{crimestop}, \emph{blackwhite}, and
\emph{doublethink}, makes him unwilling and unable to think too deeply
on any subject whatever.

A Party member is expected to have no private emotions and no respites
from enthusiasm. He is supposed to live in a continuous frenzy of hatred
of foreign enemies and internal traitors, triumph over victories, and
self-abasement before the power and wisdom of the Party. The discontents
produced by his bare, unsatisfying life are deliberately turned outwards
and dissipated by such devices as the Two Minutes Hate, and the
speculations which might possibly induce a skeptical or rebellious
attitude are killed in advance by his early acquired inner discipline.
The first and simplest stage in the discipline, which can be taught even
to young children, is called, in Newspeak, \emph{crimestop}.
\emph{Crimestop} means the faculty of stopping short, as though by
instinct, at the threshold of any dangerous thought. It includes the
power of not grasping analogies, of failing to perceive logical errors,
of misunderstanding the simplest arguments if they are inimical to
Ingsoc, and of being bored or repelled by any train of thought which is
capable of leading in a heretical direction. \emph{Crimestop}, in short,
means protective stupidity. But stupidity is not enough. On the
contrary, orthodoxy in the full sense demands a control over
one\textquotesingle s own mental processes as complete as that of a
contortionist over his body. Oceanic society rests ultimately on the
belief that Big Brother is omnipotent and that the Party is infallible.
But since in reality Big Brother is not omnipotent and the Party is not
infallible, there is need for an unwearying, moment-to-moment
flexibility in the treatment of facts. The key word here is
\emph{blackwhite}. Like so many Newspeak words, this word has two
mutually contradictory meanings. Applied to an opponent, it means the
habit of impudently claiming that black is white, in contradiction of
the plain facts. Applied to a Party member, it means a loyal willingness
to say that black is white when Party discipline demands this. But it
means also the ability to \emph{believe} that black is white, and more,
to \emph{know} that black is white, and to forget that one has ever
believed the contrary. This demands a continuous alteration of the past,
made possible by the system of thought which really embraces all the
rest, and which is known in Newspeak as \emph{doublethink}.

The alteration of the past is necessary for two reasons, one of which is
subsidiary and, so to speak, precautionary. The subsidiary reason is
that the Party member, like the proletarian, tolerates present-day
conditions partly because he has no standards of comparison. He must be
cut off from the past, just as he must be cut off from foreign
countries, because it is necessary for him to believe that he is better
off than his ancestors and that the average level of material comfort is
constantly rising. But by far the more important reason for the
readjustment of the past is the need to safeguard the infallibility of
the Party. It is not merely that speeches, statistics, and records of
every kind must be constantly brought up to date in order to show that
the predictions of the Party were in all cases right. It is also that no
change in doctrine or in political alignment can ever be admitted. For
to change one\textquotesingle s mind, or even one\textquotesingle s
policy, is a confession of weakness. If, for example, Eurasia or
Eastasia (whichever it may be) is the enemy today, then that country
must always have been the enemy. And if the facts say otherwise, then
the facts must be altered. Thus history is continuously rewritten. This
day-to-day falsification of the past, carried out by the Ministry of
Truth, is as necessary to the stability of the regime as the work of
repression and espionage carried out by the Ministry of Love.

The mutability of the past is the central tenet of Ingsoc. Past events,
it is argued, have no objective existence, but survive only in written
records and in human memories. The past is whatever the records and the
memories agree upon. And since the Party is in full control of all
records, and in equally full control of the minds of its members, it
follows that the past is whatever the Party chooses to make it. It also
follows that though the past is alterable, it never has been altered in
any specific instance. For when it has been recreated in whatever shape
is needed at the moment, then this new version \emph{is} the past, and
no different past can ever have existed. This holds good even when, as
often happens, the same event has to be altered out of recognition
several times in the course of a year. At all times the Party is in
possession of absolute truth, and clearly the absolute can never have
been different from what it is now. It will be seen that the control of
the past depends above all on the training of memory. To make sure that
all written records agree with the orthodoxy of the moment is merely a
mechanical act. But it is also necessary to \emph{remember} that events
happened in the desired manner. And if it is necessary to rearrange
one\textquotesingle s memories or to tamper with written records, then
it is necessary to \emph{forget} that one has done so. The trick of
doing this can be learned like any other mental technique. It \emph{is}
learned by the majority of Party members, and certainly by all who are
intelligent as well as orthodox. In Oldspeak it is called, quite
frankly, ``reality control.'' In Newspeak it is called \emph{doublethink},
though \emph{doublethink} comprises much else as well.

\emph{Doublethink} means the power of holding two contradictory beliefs
in one\textquotesingle s mind simultaneously, and accepting both of
them. The Party intellectual knows in which direction his memories must
be altered; he therefore knows that he is playing tricks with reality;
but by the exercise of \emph{doublethink} he also satisfies himself that
reality is not violated. The process has to be conscious, or it would
not be carried out with sufficient precision, but it also has to be
unconscious, or it would bring with it a feeling of falsity and hence of
guilt. \emph{Doublethink} lies at the very heart of Ingsoc, since the
essential act of the Party is to use conscious deception while retaining
the firmness of purpose that goes with complete honesty. To tell
deliberate lies while genuinely believing in them, to forget any fact
that has become inconvenient, and then, when it becomes necessary again,
to draw it back from oblivion for just so long as it is needed, to deny
the existence of objective reality and all the while to take account of
the reality which one denies---all this is indispensably necessary. Even
in using the word \emph{doublethink} it is necessary to exercise
\emph{doublethink}. For by using the word one admits that one is
tampering with reality; by a fresh act of \emph{doublethink} one erases
this knowledge; and so on indefinitely, with the lie always one leap
ahead of the truth. Ultimately it is by means of \emph{doublethink} that
the Party has been able---and may, for all we know, continue to be able
for thousands of years---to arrest the course of history.

All past oligarchies have fallen from power either because they ossified
or because they grew soft. Either they became stupid and arrogant,
failed to adjust themselves to changing circumstances, and were
overthrown, or they became liberal and cowardly, made concessions when
they should have used force, and once again were overthrown. They fell,
that is to say, either through consciousness or through unconsciousness.
It is the achievement of the Party to have produced a system of thought
in which both conditions can exist simultaneously. And upon no other
intellectual basis could the dominion of the Party be made permanent. If
one is to rule, and to continue ruling, one must be able to dislocate
the sense of reality. For the secret of rulership is to combine a belief
in one\textquotesingle s own infallibility with the power to learn from
past mistakes.

It need hardly be said that the subtlest practitioners of
\emph{doublethink} are those who invented \emph{doublethink} and know
that it is a vast system of mental cheating. In our society, those who
have the best knowledge of what is happening are also those who are
furthest from seeing the world as it is. In general, the greater the
understanding, the greater the delusion: the more intelligent, the less
sane. One clear illustration of this is the fact that war hysteria
increases in intensity as one rises in the social scale. Those whose
attitude toward the war is most nearly rational are the subject peoples
of the disputed territories. To these people the war is simply a
continuous calamity which sweeps to and fro over their bodies like a
tidal wave. Which side is winning is a matter of complete indifference
to them. They are aware that a change of overlordship means simply that
they will be doing the same work as before for new masters who treat
them in the same manner as the old ones. The slightly more favored
workers whom we call ``the proles'' are only intermittently conscious of
the war. When it is necessary they can be prodded into frenzies of fear
and hatred, but when left to themselves they are capable of forgetting
for long periods that the war is happening. It is in the ranks of the
Party, and above all of the Inner Party, that the true war enthusiasm is
found. World-conquest is believed in most firmly by those who know it to
be impossible. This peculiar linking-together of opposites---knowledge
with ignorance, cynicism with fanaticism---is one of the chief
distinguishing marks of Oceanic society. The official ideology abounds
with contradictions even where there is no practical reason for them.
Thus, the Party rejects and vilifies every principle for which the
Socialist movement originally stood, and it chooses to do this in the
name of Socialism. It preaches a contempt for the working class
unexampled for centuries past, and it dresses its members in a uniform
which was at one time peculiar to manual workers and was adopted for
that reason. It systematically undermines the solidarity of the family,
and it calls its leader by a name which is a direct appeal to the
sentiment of family loyalty. Even the names of the four Ministries by
which we are governed exhibit a sort of impudence in their deliberate
reversal of the facts. The Ministry of Peace concerns itself with war,
the Ministry of Truth with lies, the Ministry of Love with torture, and
the Ministry of Plenty with starvation. These contradictions are not
accidental, nor do they result from ordinary hypocrisy: they are
deliberate exercises in \emph{doublethink}. For it is only by
reconciling contradictions that power can be retained indefinitely. In
no other way could the ancient cycle be broken. If human equality is to
be forever averted---if the High, as we have called them, are to keep
their places permanently---then the prevailing mental condition must be
controlled insanity.

But there is one question which until this moment we have almost
ignored. It is: \emph{why} should human equality be averted? Supposing
that the mechanics of the process have been rightly described, what is
the motive for this huge, accurately planned effort to freeze history at
a particular moment of time?

Here we reach the central secret. As we have seen, the mystique of the
Party, and above all of the Inner Party, depends upon
\emph{doublethink}. But deeper than this lies the original motive, the
never-questioned instinct that first led to the seizure of power and
brought \emph{doublethink}, the Thought Police, continuous warfare, and
all the other necessary paraphernalia into existence afterwards. This
motive really consists ...
\end{quotation}

Winston became aware of silence, as one becomes aware of a new sound. It
seemed to him that Julia had been very still for some time past. She was
lying on her side, naked from the waist upwards, with her cheek pillowed
on her hand and one dark lock tumbling across her eyes. Her breast rose
and fell slowly and regularly.

``Julia.''

No answer.

``Julia, are you awake?''

No answer. She was asleep. He shut the book, put it carefully on the
floor, lay down, and pulled the coverlet over both of them.

He had still, he reflected, not learned the ultimate secret. He
understood \emph{how}; he did not understand \emph{why}. Chapter 1, like
Chapter 3, had not actually told him anything that he did not know; it
had merely systematized the knowledge that he possessed already. But
after reading it he knew better than before that he was not mad. Being
in a minority, even a minority of one, did not make you mad. There was
truth and there was untruth, and if you clung to the truth even against
the whole world, you were not mad. A yellow beam from the sinking sun
slanted in through the window and fell across the pillow. He shut his
eyes. The sun on his face and the girl\textquotesingle s smooth body
touching his own gave him a strong, sleepy, confident feeling. He was
safe, everything was all right. He fell asleep murmuring ``Sanity is not
statistical,'' with the feeling that this remark contained in it a
profound wisdom.


\section{X}\label{x}

When he woke it was with the sensation of having slept for a long time,
but a glance at the old-fashioned clock told him that it was only
twenty-thirty. He lay dozing for a little while; then the usual
deep-lunged singing struck up from the yard below:

\begin{quotation}
  ``It was only an \textquotesingle opeless fancy,\\
  It passed like an Ipril dye,\\
  But a look an\textquotesingle{} a word an\textquotesingle{} the
  dreams they stirred\\
  They \textquotesingle ave stolen my \textquotesingle eart awye!''
\end{quotation}

The driveling song seemed to have kept its popularity. You still heard
it all over the place. It had outlived the ``Hate Song.'' Julia woke at
the sound, stretched herself luxuriously, and got out of bed.

``I\textquotesingle m hungry,'' she said. ``Let\textquotesingle s make some
more coffee. Damn! The stove\textquotesingle s gone out and the
water\textquotesingle s cold.'' She picked the stove up and shook it.
``There\textquotesingle s no oil in it.''

``We can get some from old Charrington, I expect.''

``The funny thing is I made sure it was full. I\textquotesingle m going
to put my clothes on,'' she added. ``It seems to have got colder.''

Winston also got up and dressed himself. The indefatigable voice sang
on:

\begin{quotation}
  ``They sye that time \textquotesingle eals all things,\\
  They sye you can always forget;\\
  But the smiles an\textquotesingle{} the tears across the years\\
  They twist my \textquotesingle eartstrings yet!''
\end{quotation}

As he fastened the belt of his overalls he strolled across to the
window. The sun must have gone down behind the houses; it was not
shining into the yard any longer. The flagstones were wet as though they
had just been washed, and he had the feeling that the sky had been
washed too, so fresh and pale was the blue between the chimney pots.
Tirelessly the woman marched to and fro, corking and uncorking herself,
singing and falling silent, and pegging out more diapers, and more and
yet more. He wondered whether she took in washing for a living, or was
merely the slave of twenty or thirty grandchildren. Julia had come
across to his side; together they gazed down with a sort of fascination
at the sturdy figure below. As he looked at the woman in her
characteristic attitude, her thick arms reaching up for the line, her
powerful marelike buttocks protruded, it struck him for the first time
that she was beautiful. It had never before occurred to him that the
body of a woman of fifty, blown up to monstrous dimensions by
childbearing, then hardened, roughened by work till it was coarse in the
grain like an overripe turnip, could be beautiful. But it was so, and
after all, he thought, why not? The solid, contourless body, like a
block of granite, and the rasping red skin, bore the same relation to
the body of a girl as the rose-hip to the rose. Why should the fruit be
held inferior to the flower?

``She\textquotesingle s beautiful,'' he murmured.

``She\textquotesingle s a meter across the hips, easily,'' said Julia.

``That is her style of beauty,'' said Winston.

He held Julia\textquotesingle s supple waist easily encircled by his
arm. From the hip to the knee her flank was against his. Out of their
bodies no child would ever come. That was the one thing they could never
do. Only by word of mouth, from mind to mind, could they pass on the
secret. The woman down there had no mind, she had only strong arms, a
warm heart, and a fertile belly. He wondered how many children she had
given birth to. It might easily be fifteen. She had had her momentary
flowering, a year, perhaps, of wildrose beauty, and then she had
suddenly swollen like a fertilized fruit and grown hard and red and
coarse, and then her life had been laundering, scrubbing, darning,
cooking, sweeping, polishing, mending, scrubbing, laundering, first for
children, then for grandchildren, over thirty unbroken years. At the end
of it she was still singing. The mystical reverence that he felt for her
was somehow mixed up with the aspect of the pale, cloudless sky,
stretching away behind the chimney pots into interminable distances. It
was curious to think that the sky was the same for everybody, in Eurasia
or Eastasia as well as here. And the people under the sky were also very
much the same---everywhere, all over the world, hundreds or thousands of
millions of people just like this, people ignorant of one
another\textquotesingle s existence, held apart by walls of hatred and
lies, and yet almost exactly the same---people who had never learned to
think but who were storing up in their hearts and bellies and muscles
the power that would one day overturn the world. If there was hope, it
lay in the proles! Without having read to the end of \emph{the book}, he
knew that that must be Goldstein\textquotesingle s final message. The
future belonged to the proles. And could he be sure that when their time
came, the world they constructed would not be just as alien to him,
Winston Smith, as the world of the Party? Yes, because at the least it
would be a world of sanity. Where there is equality there can be sanity.
Sooner or later it would happen: strength would change into
consciousness. The proles were immortal; you could not doubt it when you
looked at that valiant figure in the yard. In the end their awakening
would come. And until that happened, though it might be a thousand
years, they would stay alive against all the odds, like birds, passing
on from body to body the vitality which the Party did not share and
could not kill.

``Do you remember,'' he said, ``the thrush that sang to us, that first day,
at the edge of the wood?''

``He wasn\textquotesingle t singing to us,'' said Julia. ``He was singing
to please himself. Not even that. He was just singing.''

The birds sang, the proles sang, the Party did not sing. All round the
world, in London and New York, in Africa and Brazil and in the
mysterious, forbidden lands beyond the frontiers, in the streets of
Paris and Berlin, in the villages of the endless Russian plain, in the
bazaars of China and Japan---everywhere stood the same solid
unconquerable figure, made monstrous by work and childbearing, toiling
from birth to death and still singing. Out of those mighty loins a race
of conscious beings must one day come. You were the dead; theirs was the
future. But you could share in that future if you kept alive the mind as
they kept alive the body, and passed on the secret doctrine that two
plus two make four.

``We are the dead,'' he said.

``We are the dead,'' echoed Julia dutifully.

``You are the dead,'' said an iron voice behind them.

They sprang apart. Winston\textquotesingle s entrails seemed to have
turned into ice. He could see the white all round the irises of
Julia\textquotesingle s eyes. Her face had turned a milky yellow. The
smear of rouge that was still on each cheekbone stood out sharply,
almost as though unconnected with the skin beneath.

``You are the dead,'' repeated the iron voice.

``It was behind the picture,'' breathed Julia.

``It was behind the picture,'' said the voice. ``Remain exactly where you
are. Make no movement until you are ordered.''

It was starting, it was starting at last! They could do nothing except
stand gazing into one another\textquotesingle s eyes. To run for life,
to get out of the house before it was too late---no such thought
occurred to them. Unthinkable to disobey the iron voice from the wall.
There was a snap as though a catch had been turned back, and a crash of
breaking glass. The picture had fallen to the floor, uncovering the
telescreen behind it.

``Now they can see us,'' said Julia.

``Now we can see you,'' said the voice. ``Stand out in the middle of the
room. Stand back to back. Clasp your hands behind your heads. Do not
touch one another.''

They were not touching, but it seemed to him that he could feel
Julia\textquotesingle s body shaking. Or perhaps it was merely the
shaking of his own. He could just stop his teeth from chattering, but
his knees were beyond his control. There was a sound of trampling boots
below, inside the house and outside. The yard seemed to be full of men.
Something was being dragged across the stones. The
woman\textquotesingle s singing had stopped abruptly. There was a long,
rolling clang, as though the washtub had been flung across the yard, and
then a confusion of angry shouts which ended in a yell of pain.

``The house is surrounded,'' said Winston.

``The house is surrounded,'' said the voice.

He heard Julia snap her teeth together. ``I suppose we may as well say
good-by,'' she said.

``You may as well say good-by,'' said the voice. And then another quite
different voice, a thin, cultivated voice which Winston had the
impression of having heard before, struck in: ``And by the way, while we
are on the subject, \emph{Here comes a candle to light you to bed, here
comes a chopper to chop off your head!}''

Something crashed on to the bed behind Winston\textquotesingle s back.
The head of a ladder had been thrust through the window and had burst in
the frame. Someone was climbing through the window. There was a stampede
of boots up the stairs. The room was full of solid men in black
uniforms, with iron-shod boots on their feet and truncheons in their
hands.

Winston was not trembling any longer. Even his eyes he barely moved. One
thing alone mattered: to keep still, to keep still and not give them an
excuse to hit you! A man with a smooth prizefighter\textquotesingle s
jowl in which the mouth was only a slit paused opposite him, balancing
his truncheon meditatively between thumb and forefinger. Winston met his
eyes. The feeling of nakedness, with one\textquotesingle s hands behind
one\textquotesingle s head and one\textquotesingle s face and body all
exposed, was almost unbearable. The man protruded the tip of a white
tongue, licked the place where his lips should have been, and then
passed on. There was another crash. Someone had picked up the glass
paperweight from the table and smashed it to pieces on the hearthstone.

The fragment of coral, a tiny crinkle of pink like a sugar rosebud from
a cake, rolled across the mat. How small, thought Winston, how small it
always was! There was a gasp and a thump behind him, and he received a
violent kick on the ankle which nearly flung him off his balance. One of
the men had smashed his fist into Julia\textquotesingle s solar plexus,
doubling her up like a pocket ruler. She was thrashing about on the
floor, fighting for breath. Winston dared not turn his head even by a
millimeter, but sometimes her livid, gasping face came within the angle
of his vision. Even in his terror it was as though he could feel the
pain in his own body, the deadly pain which nevertheless was less urgent
than the struggle to get back her breath. He knew what it was like: the
terrible, agonizing pain which was there all the while but could not be
suffered yet, because before all else it was necessary to be able to
breathe. Then two of the men hoisted her up by knees and shoulders and
carried her out of the room like a sack. Winston had a glimpse of her
face, upside down, yellow and contorted, with the eyes shut, and still
with a smear of rouge on either cheek; and that was the last he saw of
her.

He stood dead still. No one had hit him yet. Thoughts which came of
their own accord but seemed totally uninteresting began to flit through
his mind. He wondered whether they had got Mr. Charrington. He wondered
what they had done to the woman in the yard. He noticed that he badly
wanted to urinate, and felt a faint surprise, because he had done so
only two or three hours ago. He noticed that the clock on the
mantelpiece said nine, meaning twenty-one. But the light seemed too
strong. Would not the light be fading at twenty-one hours on an August
evening? He wondered whether after all he and Julia had mistaken the
time---had slept the clock round and thought it was twenty-thirty when
really it was nought eight-thirty on the following morning. But he did
not pursue the thought further. It was not interesting.

There was another, lighter step in the passage. Mr. Charrington came
into the room. The demeanor of the black-uniformed men suddenly became
more subdued. Something had also changed in Mr.
Charrington\textquotesingle s appearance. His eye fell on the fragments
of the glass paperweight.

``Pick up those pieces,'' he said sharply.

A man stooped to obey. The cockney accent had disappeared; Winston
suddenly realized whose voice it was that he had heard a few moments ago
on the telescreen. Mr. Charrington was still wearing his old velvet
jacket, but his hair, which had been almost white, had turned black.
Also he was not wearing his spectacles. He gave Winston a single sharp
glance, as though verifying his identity, and then paid no more
attention to him. He was still recognizable, but he was not the same
person any longer. His body had straightened, and seemed to have grown
bigger. His face had undergone only tiny changes that had nevertheless
worked a complete transformation. The black eyebrows were less bushy,
the wrinkles were gone, the whole lines of the face seemed to have
altered; even the nose seemed shorter. It was the alert, cold face of a
man of about five-and-thirty. It occurred to Winston that for the first
time in his life he was looking, with knowledge, at a member of the
Thought Police.

\clearpage
\part{THREE}\label{three}

\section{I}

He did not know where he was. Presumably he was in the Ministry of Love;
but there was no way of making certain.

He was in a high-ceilinged windowless cell with walls of glittering
white porcelain. Concealed lamps flooded it with cold light, and there
was a low, steady humming sound which he supposed had something to do
with the air supply. A bench, or shelf, just wide enough to sit on ran
round the wall, broken only by the door and, at the end opposite the
door, a lavatory pan with no wooden seat. There were four telescreens,
one in each wall.

There was a dull aching in his belly. It had been there ever since they
had bundled him into the closed van and driven him away. But he was also
hungry, with a gnawing, unwholesome kind of hunger. It might be
twenty-four hours since he had eaten, it might be thirty-six. He still
did not know, probably never would know, whether it had been morning or
evening when they arrested him. Since he was arrested he had not been
fed.

He sat as still as he could on the narrow bench, with his hands crossed
on his knee. He had already learned to sit still. If you made unexpected
movements they yelled at you from the telescreen. But the craving for
food was growing upon him. What he longed for above all was a piece of
bread. He had an idea that there were a few breadcrumbs in the pocket of
his overalls. It was even possible---he thought this because from time
to time something seemed to tickle his leg---that there might be a
sizable bit of crust there. In the end the temptation to find out
overcame his fear; he slipped a hand into his pocket.

``Smith!'' yelled a voice from the telescreen. ``6079 Smith W! Hands out of
pockets in the cells!''

He sat still again, his hands crossed on his knee. Before being brought
here he had been taken to another place which must have been an ordinary
prison or a temporary lock-up used by the patrols. He did not know how
long he had been there; some hours, at any rate; with no clocks and no
daylight it was hard to gauge the time. It was a noisy, evil-smelling
place. They had put him into a cell similar to the one he was now in,
but filthily dirty and at all times crowded by ten or fifteen people.
The majority of them were common criminals, but there were a few
political prisoners among them. He had sat silent against the wall,
jostled by dirty bodies, too preoccupied by fear and the pain in his
belly to take much interest in his surroundings, but still noticing the
astonishing difference in demeanor between the Party prisoners and the
others. The Party prisoners were always silent and terrified, but the
ordinary criminals seemed to care nothing for anybody. They yelled
insults at the guards, fought back fiercely when their belongings were
impounded, wrote obscene words on the floor, ate smuggled food which
they produced from mysterious hiding places in their clothes, and even
shouted down the telescreen when it tried to restore order. On the other
hand, some of them seemed to be on good terms with the guards, called
them by nicknames, and tried to wheedle cigarettes through the spy-hole
in the door. The guards, too, treated the common criminals with a
certain forbearance, even when they had to handle them roughly. There
was much talk about the forced-labor camps to which most of the
prisoners expected to be sent. It was ``all right'' in the camps, he
gathered, so long as you had good contacts and knew the ropes. There
were bribery, favoritism, and racketeering of every kind, there were
homosexuality and prostitution, there was even illicit alcohol distilled
from potatoes. The positions of trust were given only to the common
criminals, especially the gangsters and the murderers, who formed a sort
of aristocracy. All the dirty jobs were done by the politicals.

There was a constant come-and-go of prisoners of every description: drug
peddlers, thieves, bandits, black marketeers, drunks, prostitutes. Some
of the drunks were so violent that the other prisoners had to combine to
suppress them. An enormous wreck of a woman, aged about sixty, with
great tumbling breasts and thick coils of white hair which had come down
in her struggles, was carried in, kicking and shouting, by four guards
who had hold of her one at each corner. They wrenched off the boots with
which she had been trying to kick them, and dumped her down across
Winston\textquotesingle s lap, almost breaking his thighbones. The woman
hoisted herself upright and followed them out with a yell of ``F------
bastards!'' Then, noticing that she was sitting on something uneven, she
slid off Winston\textquotesingle s knees onto the bench.

``Beg pardon, dearie,'' she said. ``I wouldn\textquotesingle t
\textquotesingle a sat on you, only the buggers put me there. They dono
\textquotesingle ow to treat a lady, do they?'' She paused, patted her
breast, and belched. ``Pardon,'' she said, ``I ain\textquotesingle t
meself, quite.''

She leant forward and vomited copiously on the floor.

``Thass better,'' she said, leaning back with closed eyes. ``Never keep it
down, thass what I say. Get it up while it\textquotesingle s fresh on
your stomach, like.''

She revived, turned to have another look at Winston, and seemed
immediately to take a fancy to him. She put a vast arm round his
shoulder and drew him toward her, breathing beer and vomit into his
face.

``Wass your name, dearie?'' she said.

``Smith,'' said Winston.

``Smith?'' said the woman. ``Thass funny. My name\textquotesingle s Smith
too. Why,'' she added sentimentally, ``I might be your mother!''

She might, thought Winston, be his mother. She was about the right age
and physique, and it was probable that people changed somewhat after
twenty years in a forced-labor camp.

No one else had spoken to him. To a surprising extent the ordinary
criminals ignored the Party prisoners. ``The pol\emph{its},'' they called
them, with a sort of uninterested contempt. The Party prisoners seemed
terrified of speaking to anybody, and above all of speaking to one
another. Only once, when two Party members, both women, were pressed
close together on the bench, he overheard amid the din of voices a few
hurriedly whispered words; and in particular a reference to something
called ``room one-oh-one,'' which he did not understand.

It might be two or three hours ago that they had brought him here. The
dull pain in his belly never went away, but sometimes it grew better and
sometimes worse, and his thoughts expanded or contracted accordingly.
When it grew worse he thought only of the pain itself, and of his desire
for food. When it grew better, panic took hold of him. There were
moments when he foresaw the things that would happen to him with such
actuality that his heart galloped and his breath stopped. He felt the
smash of truncheons on his elbows and iron-shod boots on his shins; he
saw himself groveling on the floor, screaming for mercy through broken
teeth. He hardly thought of Julia. He could not fix his mind on her. He
loved her and would not betray her; but that was only a fact, known as
he knew the rules of arithmetic. He felt no love for her, and he hardly
even wondered what was happening to her. He thought oftener of
O\textquotesingle Brien, with a flickering hope. O\textquotesingle Brien
must know that he had been arrested. The Brotherhood, he had said, never
tried to save its members. But there was the razor blade; they would
send the razor blade if they could. There would be perhaps five seconds
before the guards could rush into the cell. The blade would bite into
him with a sort of burning coldness, and even the fingers that held it
would be cut to the bone. Everything came back to his sick body, which
shrank trembling from the smallest pain. He was not certain that he
would use the razor blade even if he got the chance. It was more natural
to exist from moment to moment, accepting another ten
minutes\textquotesingle{} life even with the certainty that there was
torture at the end of it.

Sometimes he tried to calculate the number of porcelain bricks in the
walls of the cell. It should have been easy, but he always lost count at
some point or another. More often he wondered where he was, and what
time of day it was. At one moment he felt certain that it was broad
daylight outside, and at the next equally certain that it was pitch
darkness. In this place, he knew instinctively, the lights would never
be turned out. It was the place with no darkness: he saw now why
O\textquotesingle Brien had seemed to recognize the allusion. In the
Ministry of Love there were no windows. His cell might be at the heart
of the building or against its outer wall; it might be ten floors below
ground, or thirty above it. He moved himself mentally from place to
place, and tried to determine by the feeling of his body whether he was
perched high in the air or buried deep underground.

There was a sound of marching boots outside. The steel door opened with
a clang. A young officer, a trim black-uniformed figure who seemed to
glitter all over with polished leather and whose pale, straight-featured
face was like a wax mask, stepped smartly through the doorway. He
motioned to the guards outside to bring in the prisoner they were
leading. The poet Ampleforth shambled into the cell. The door clanged
shut again.

Ampleforth made one or two uncertain movements from side to side, as
though having some idea that there was another door to go out of, and
then began to wander up and down the cell. He had not yet noticed
Winston\textquotesingle s presence. His troubled eyes were gazing at the
wall about a meter above the level of Winston\textquotesingle s head. He
was shoeless; large, dirty toes were sticking out of the holes in his
socks. He was also several days away from a shave. A scrubby beard
covered his face to the cheekbones, giving him an air of ruffianism that
went oddly with his large weak frame and nervous movements.

Winston roused himself a little from his lethargy. He must speak to
Ampleforth, and risk the yell from the telescreen. It was even
conceivable that Ampleforth was the bearer of the razor blade.

``Ampleforth,'' he said.

There was no yell from the telescreen. Ampleforth paused, mildly
startled. His eyes focused themselves slowly on Winston.

``Ah, Smith!'' he said. ``You, too!''

``What are you in for?''

``To tell you the truth---'' He sat down awkwardly on the bench opposite
Winston. ``There is only one offense, is there not?'' he said.

``And have you committed it?''

``Apparently I have.''

He put a hand to his forehead and pressed his temples for a moment, as
though trying to remember something.

``These things happen,'' he began vaguely. ``I have been able to recall one
instance---a possible instance. It was an indiscretion, undoubtedly. We were
producing a definitive edition of the poems of Kipling. I allowed the word
`God' to remain at the end of a line. I could not help it!'' he added almost
indignantly, raising his face to look at Winston. ``It was impossible to
change the line. The rhyme was `rod.' Do you realize that there are only
twelve rhymes to `rod' in the entire language? For days I had racked my
brains. There \emph{was} no other rhyme.''

The expression on his face changed. The annoyance passed out of it and
for a moment he looked almost pleased. A sort of intellectual warmth,
the joy of the pedant who has found out some useless fact, shone through
the dirt and scrubby hair.

``Has it ever occurred to you,'' he said, ``that the whole history of
English poetry has been determined by the fact that the English language
lacks rhymes?''

No, that particular thought had never occurred to Winston. Nor, in the
circumstances, did it strike him as very important or interesting.

``Do you know what time of day it is?'' he said.

Ampleforth looked startled again. ``I had hardly thought about it. They
arrested me---it could be two days ago---perhaps three.'' His eyes
flitted round the walls, as though he half expected to find a window
somewhere. ``There is no difference between night and day in this place.
I do not see how one can calculate the time.''

They talked desultorily for some minutes, then, without apparent reason,
a yell from the telescreen bade them be silent. Winston sat quietly, his
hands crossed. Ampleforth, too large to sit in comfort on the narrow
bench, fidgeted from side to side, clasping his lank hands first round
one knee, then round the other. The telescreen barked at him to keep
still. Time passed. Twenty minutes, an hour---it was difficult to judge.
Once more there was a sound of boots outside. Winston\textquotesingle s
entrails contracted. Soon, very soon, perhaps in five minutes, perhaps
now, the tramp of boots would mean that his own turn had come.

The door opened. The cold-faced young officer stepped into the cell.
With a brief movement of the hand he indicated Ampleforth.

``Room 101,'' he said.

Ampleforth marched clumsily out between the guards, his face vaguely
perturbed, but uncomprehending.

What seemed like a long time passed. The pain in
Winston\textquotesingle s belly had revived. His mind sagged round and
round on the same track, like a ball falling again and again into the
same series of slots. He had only six thoughts. The pain in his belly; a
piece of bread; the blood and the screaming; O\textquotesingle Brien;
Julia; the razor blade. There was another spasm in his entrails; the
heavy boots were approaching. As the door opened, the wave of air that
it created brought in a powerful smell of cold sweat. Parsons walked
into the cell. He was wearing khaki shorts and a sports shirt.

This time Winston was startled into self-forgetfulness.

``\emph{You} here!'' he said.

Parsons gave Winston a glance in which there was neither interest nor
surprise, but only misery. He began walking jerkily up and down,
evidently unable to keep still. Each time he straightened his pudgy
knees it was apparent that they were trembling. His eyes had a
wide-open, staring look, as though he could not prevent himself from
gazing at something in the middle distance.

``What are you in for?'' said Winston.

``Thoughtcrime!'' said Parsons, almost blubbering. The tone of his voice
implied at once a complete admission of his guilt and a sort of
incredulous horror that such a word could be applied to himself. He
paused opposite Winston and began eagerly appealing to him: ``You
don\textquotesingle t think they\textquotesingle ll shoot me, do you,
old chap? They don\textquotesingle t shoot you if you
haven\textquotesingle t actually done anything---only thoughts, which
you can\textquotesingle t help? I know they give you a fair hearing. Oh,
I trust them for that! They\textquotesingle ll know my record,
won\textquotesingle t they? \emph{You} know what kind of a chap I was.
Not a bad chap in my way. Not brainy, of course, but keen. I tried to do
my best for the Party, didn\textquotesingle t I? I\textquotesingle ll
get off with five years, don\textquotesingle t you think? Or even ten
years? A chap like me could make himself pretty useful in a labor camp.
They wouldn\textquotesingle t shoot me for going off the rails just
once?''

``Are you guilty?'' said Winston.

``Of course I\textquotesingle m guilty!'' cried Parsons with a servile
glance at the telescreen. ``You don\textquotesingle t think the Party
would arrest an innocent man, do you?'' His froglike face grew calmer,
and even took on a slightly sanctimonious expression. ``Thoughtcrime is a
dreadful thing, old man,'' he said sententiously. ``It\textquotesingle s
insidious. It can get hold of you without your even knowing it. Do you
know how it got hold of me? In my sleep! Yes, that\textquotesingle s a
fact. There I was, working away, trying to do my bit---never knew I had
any bad stuff in my mind at all. And then I started talking in my sleep.
Do you know what they heard me saying?''

He sank his voice, like someone who is obliged for medical reasons to
utter an obscenity.

```Down with Big Brother!' Yes, I said that! Said it over and over again, it
seems. Between you and me, old man, I\textquotesingle m glad they got me
before it went any further. Do you know what I\textquotesingle m going to
say to them when I go up before the tribunal? `Thank you,' I\textquotesingle
m going to say, `thank you for saving me before it was too
late.'{}''

``Who denounced you?'' said Winston.

``It was my little daughter,'' said Parsons with a sort of doleful pride.
``She listened at the keyhole. Heard what I was saying, and nipped off to
the patrols the very next day. Pretty smart for a nipper of seven, eh? I
don\textquotesingle t bear her any grudge for it. In fact
I\textquotesingle m proud of her. It shows I brought her up in the right
spirit, anyway.''

He made a few more jerky movements up and down, several times casting a
longing glance at the lavatory pan. Then he suddenly ripped down his
shorts.

``Excuse me, old man,'' he said. ``I can\textquotesingle t help it.
It\textquotesingle s the waiting.''

He plumped his large posteriors onto the lavatory pan. Winston covered
his face with his hands.

``Smith!'' yelled the voice from the telescreen. ``6079 Smith W! Uncover
your face. No faces covered in the cells.''

Winston uncovered his face. Parsons used the lavatory, loudly and
abundantly. It then turned out that the plug was defective, and the cell
stank abominably for hours afterwards.

Parsons was removed. More prisoners came and went mysteriously. One, a
woman, was consigned to ``Room 101,'' and, Winston noticed, seemed to
shrivel and turn a different color when she heard the words. A time came
when, if it had been morning when he was brought here, it would be
afternoon; or if it had been afternoon, then it would be midnight. There
were six prisoners in the cell, men and women. All sat very still.
Opposite Winston there sat a man with a chinless, toothy face exactly
like that of some large, harmless rodent. His fat, mottled cheeks were
so pouched at the bottom that it was difficult not to believe that he
had little stores of food tucked away there. His pale-gray eyes flitted
timorously from face to face, and turned quickly away again when he
caught anyone\textquotesingle s eye.

The door opened, and another prisoner was brought in whose appearance
sent a momentary chill through Winston. He was a commonplace,
mean-looking man who might have been an engineer or technician of some
kind. But what was startling was the emaciation of his face. It was like
a skull. Because of its thinness the mouth and eyes looked
disproportionately large, and the eyes seemed filled with a murderous,
unappeasable hatred of somebody or something.

The man sat down on the bench at a little distance from Winston. Winston
did not look at him again, but the tormented, skull-like face was as
vivid in his mind as though it had been straight in front of his eyes.
Suddenly he realized what was the matter. The man was dying of
starvation. The same thought seemed to occur almost simultaneously to
everyone in the cell. There was a very faint stirring all the way round
the bench. The eyes of the chinless man kept flitting toward the
skull-faced man, then turning guiltily away, then being dragged back by
an irresistible attraction. Presently he began to fidget on his seat. At
last he stood up, waddled clumsily across the cell, dug down into the
pocket of his overalls, and, with an abashed air, held out a grimy piece
of bread to the skull-faced man.

There was a furious, deafening roar from the telescreen. The chinless
man jumped in his tracks. The skull-faced man had quickly thrust his
hands behind his back, as though demonstrating to all the world that he
refused the gift.

``Bumstead!'' roared the voice. ``2713 Bumstead J! Let fall that piece of
bread.''

The chinless man dropped the piece of bread on the floor.

``Remain standing where you are,'' said the voice. ``Face the door. Make no
movement.''

The chinless man obeyed. His large pouchy cheeks were quivering
uncontrollably. The door clanged open. As the young officer entered and
stepped aside, there emerged from behind him a short stumpy guard with
enormous arms and shoulders. He took his stand opposite the chinless
man, and then, at a signal from the officer, let free a frightful blow,
with all the weight of his body behind it, full in the chinless
man\textquotesingle s mouth. The force of it seemed almost to knock him
clear of the floor. His body was flung across the cell and fetched up
against the base of the lavatory seat. For a moment he lay as though
stunned, with dark blood oozing from his mouth and nose. A very faint
whimpering or squeaking, which seemed unconscious, came out of him. Then
he rolled over and raised himself unsteadily on hands and knees. Amid a
stream of blood and saliva, the two halves of a dental plate fell out of
his mouth.

The prisoners sat very still, their hands crossed on their knees. The
chinless man climbed back into his place. Down one side of his face the
flesh was darkening. His mouth had swollen into a shapeless
cherry-colored mass with a black hole in the middle of it. From time to
time a little blood dripped onto the breast of his overalls. His gray
eyes still flitted from face to face, more guiltily than ever, as though
he were trying to discover how much the others despised him for his
humiliation.

The door opened. With a small gesture the officer indicated the
skull-faced man.

``Room 101,'' he said.

There was a gasp and a flurry at Winston\textquotesingle s side. The man
had actually flung himself on his knees on the floor, with his hands
clasped together.

``Comrade! Officer!'' he cried. ``You don\textquotesingle t have to take me
to that place! Haven\textquotesingle t I told you everything already?
What else is it you want to know? There\textquotesingle s nothing I
wouldn\textquotesingle t confess, nothing! Just tell me what it is and
I\textquotesingle ll confess it straight off. Write it down and
I\textquotesingle ll sign it---anything! Not room 101!''

``Room 101,'' said the officer.

The man\textquotesingle s face, already very pale, turned a color
Winston would not have believed possible. It was definitely,
unmistakably, a shade of green.

``Do anything to me!'' he yelled. ``You\textquotesingle ve been starving me
for weeks. Finish it off and let me die. Shoot me. Hang me. Sentence me
to twenty-five years. Is there somebody else you want me to give away?
Just say who it is and I\textquotesingle ll tell you anything you want.
I don\textquotesingle t care who it is or what you do to them.
I\textquotesingle ve got a wife and three children. The biggest of them
isn\textquotesingle t six years old. You can take the whole lot of them
and cut their throats in front of my eyes, and I\textquotesingle ll
stand by and watch it. But not room 101!''

``Room 101,'' said the officer.

The man looked frantically round at the other prisoners, as though with
some idea that he could put another victim in his own place. His eyes
settled on the smashed face of the chinless man. He flung out a lean
arm.

``That\textquotesingle s the one you ought to be taking, not me!'' he
shouted. ``You didn\textquotesingle t hear what he was saying after they
bashed his face. Give me a chance and I\textquotesingle ll tell you
every word of it. \emph{He\textquotesingle s} the one
that\textquotesingle s against the Party, not me.'' The guards stepped
forward. The man\textquotesingle s voice rose to a shriek. ``You
didn\textquotesingle t hear him!'' he repeated. ``Something went wrong
with the telescreen. \emph{He\textquotesingle s} the one you want. Take
him, not me!''

The two sturdy guards had stooped to take him by the arms. But just at
this moment he flung himself across the floor of the cell and grabbed
one of the iron legs that supported the bench. He had set up a wordless
howling, like an animal. The guards took hold of him to wrench him
loose, but he clung on with astonishing strength. For perhaps twenty
seconds they were hauling at him. The prisoners sat quiet, their hands
crossed on their knees, looking straight in front of them. The howling
stopped; the man had no breath left for anything except hanging on. Then
there was a different kind of cry. A kick from a guard\textquotesingle s
boot had broken the fingers of one of his hands. They dragged him to his
feet.

``Room 101,'' said the officer.

The man was led out, walking unsteadily, with head sunken, nursing his
crushed hand, all the fight gone out of him.

A long time passed. If it had been midnight when, the skull-faced man
was taken away, it was morning; if morning, it was afternoon. Winston
was alone, and had been alone for hours. The pain of sitting on the
narrow bench was such that often he got up and walked about, unreproved
by the telescreen. The piece of bread still lay where the chinless man
had dropped it. At the beginning it needed a hard effort not to look at
it, but presently hunger gave way to thirst. His mouth was sticky and
evil-tasting. The humming sound and the unvarying white light induced a
sort of faintness, an empty feeling inside his head. He would get up
because the ache in his bones was no longer bearable, and then would sit
down again almost at once because he was too dizzy to make sure of
staying on his feet. Whenever his physical sensations were a little
under control the terror returned. Sometimes with a fading hope he
thought of O\textquotesingle Brien and the razor blade. It was thinkable
that the razor blade might arrive concealed in his food, if he were ever
fed. More dimly he thought of Julia. Somewhere or other she was
suffering, perhaps far worse than he. She might be screaming with pain
at this moment. He thought: ``If I could save Julia by doubling my own
pain, would I do it? Yes, I would.'' But that was merely an intellectual
decision, taken because he knew that he ought to take it. He did not
feel it. In this place you could not feel anything, except pain and the
foreknowledge of pain. Besides, was it possible, when you were actually
suffering it, to wish for any reason whatever that your own pain should
increase? But that question was not answerable yet.

The boots were approaching again. The door opened.
O\textquotesingle Brien came in.

Winston started to his feet. The shock of the sight had driven all
caution out of him. For the first time in many years he forgot the
presence of the telescreen.

``They\textquotesingle ve got you too!'' he cried.

``They got me a long time ago,'' said O\textquotesingle Brien with a mild,
almost regretful irony. He stepped aside. From behind him there emerged
a broad-chested guard with a long black truncheon in his hand.

``You knew this, Winston,'' said O\textquotesingle Brien.
``Don\textquotesingle t deceive yourself. You did know it---you have
always known it.''

Yes, he saw now, he had always known it. But there was no time to think
of that. All he had eyes for was the truncheon in the
guard\textquotesingle s hand. It might fall anywhere: on the crown, on
the tip of the ear, on the upper arm, on the elbow---

The elbow! He had slumped to his knees, almost paralyzed, clasping the
stricken elbow with his other hand. Everything had exploded into yellow
light. Inconceivable, inconceivable that one blow could cause such pain!
The light cleared and he could see the other two looking down at him.
The guard was laughing at his contortions. One question at any rate was
answered. Never, for any reason on earth, could you wish for an increase
of pain. Of pain you could wish only one thing: that it should stop.
Nothing in the world was so bad as physical pain. In the face of pain
there are no heroes, no heroes, he thought over and over as he writhed
on the floor, clutching uselessly at his disabled left arm.


\section{II}\label{ii-2}

He was lying on something that felt like a camp bed, except that it was
higher off the ground and that he was fixed down in some way so that he
could not move. Light that seemed stronger than usual was falling on his
face. O\textquotesingle Brien was standing at his side, looking down at
him intently. At the other side of him stood a man in a white coat,
holding a hypodermic syringe.

Even after his eyes were open he took in his surroundings only
gradually. He had the impression of swimming up into this room from some
quite different world, a sort of underwater world far beneath it. How
long he had been down there he did not know. Since the moment when they
arrested him he had not seen darkness or daylight. Besides, his memories
were not continuous. There had been times when consciousness, even the
sort of consciousness that one has in sleep, had stopped dead and
started again after a blank interval. But whether the intervals were of
days or weeks or only seconds, there was no way of knowing.

With that first blow on the elbow the nightmare had started. Later he
was to realize that all that then happened was merely a preliminary, a
routine interrogation to which nearly all prisoners were subjected.
There was a long range of crimes---espionage, sabotage, and the
like---to which everyone had to confess as a matter of course. The
confession was a formality, though the torture was real. How many times
he had been beaten, how long the beatings had continued, he could not
remember. Always there were five or six men in black uniforms at him
simultaneously. Sometimes it was fists, sometimes it was truncheons,
sometimes it was steel rods, sometimes it was boots. There were times
when he rolled about the floor, as shameless as an animal, writhing his
body this way and that in an endless, hopeless effort to dodge the
kicks, and simply inviting more and yet more kicks, in his ribs, in his
belly, on his elbows, on his shins, in his groin, in his testicles, on
the bone at the base of his spine. There were times when it went on and
on until the cruel, wicked, unforgivable thing seemed to him not that
the guards continued to beat him but that he could not force himself
into losing consciousness. There were times when his nerve so forsook
him that he began shouting for mercy even before the beating began, when
the mere sight of a fist drawn back for a blow was enough to make him
pour forth a confession of real and imaginary crimes. There were other
times when he started out with the resolve of confessing nothing, when
every word had to be forced out of him between gasps of pain, and there
were times when he feebly tried to compromise, when he said to himself:
``I will confess, but not yet. I must hold out till the pain becomes
unbearable. Three more kicks, two more kicks, and then I will tell them
what they want.'' Sometimes he was beaten till he could hardly stand,
then flung like a sack of potatoes onto the stone floor of a cell, left
to recuperate for a few hours, and then taken out and beaten again.
There were also longer periods of recovery. He remembered them dimly,
because they were spent chiefly in sleep or stupor. He remembered a cell
with a plank bed, a sort of shelf sticking out from the wall, and a tin
washbasin, and meals of hot soup and bread and sometimes coffee. He
remembered a surly barber arriving to scrape his chin and crop his hair,
and businesslike, unsympathetic men in white coats feeling his pulse,
tapping his reflexes, turning up his eyelids, running harsh fingers over
him in search of broken bones, and shooting needles into his arm to make
him sleep.

The beatings grew less frequent, and became mainly a threat, a horror to
which he could be sent back at any moment when his answers were
unsatisfactory. His questioners now were not ruffians in black uniforms
but Party intellectuals, little rotund men with quick movements and
flashing spectacles, who worked on him in relays over periods which
lasted---he thought, he could not be sure---ten or twelve hours at a
stretch. These other questioners saw to it that he was in constant
slight pain, but it was not chiefly pain that they relied on. They
slapped his face, wrung his ears, pulled his hair, made him stand on one
leg, refused him leave to urinate, shone glaring lights in his face
until his eyes ran with water; but the aim of this was simply to
humiliate him and destroy his power of arguing and reasoning. Their real
weapon was the merciless questioning that went on and on hour after
hour, tripping him up, laying traps for him, twisting everything that he
said, convicting him at every step of lies and self-contradiction, until
he began weeping as much from shame as from nervous fatigue. Sometimes
he would weep half a dozen times in a single session. Most of the time
they screamed abuse at him and threatened at every hesitation to deliver
him over to the guards again; but sometimes they would suddenly change
their tune, call him comrade, appeal to him in the name of Ingsoc and
Big Brother, and ask him sorrowfully whether even now he had not enough
loyalty to the Party left to make him wish to undo the evil he had done.
When his nerves were in rags after hours of questioning, even this
appeal could reduce him to sniveling tears. In the end the nagging
voices broke him down more completely than the boots and fists of the
guards. He became simply a mouth that uttered, a hand that signed
whatever was demanded of him. His sole concern was to find out what they
wanted him to confess, and then confess it quickly, before the bullying
started anew. He confessed to the assassination of eminent Party
members, the distribution of seditious pamphlets, embezzlement of public
funds, sale of military secrets, sabotage of every kind. He confessed
that he had been a spy in the pay of the Eastasian government as far
back as 1968. He confessed that he was a religious believer, an admirer
of capitalism, and a sexual pervert. He confessed that he had murdered
his wife, although he knew, and his questioners must have known, that
his wife was still alive. He confessed that for years he had been in
personal touch with Goldstein and had been a member of an underground
organization which had included almost every human being he had ever
known. It was easier to confess everything and implicate everybody.
Besides, in a sense it was all true. It was true that he had been the
enemy of the Party, and in the eyes of the Party there was no
distinction between the thought and the deed.

There were also memories of another kind. They stood out in his mind
disconnectedly, like pictures with blackness all round them.

He was in a cell which might have been either dark or light, because he
could see nothing except a pair of eyes. Near at hand some kind of
instrument was ticking slowly and regularly. The eyes grew larger and
more luminous. Suddenly he floated out of his seat, dived into the eyes,
and was swallowed up.

He was strapped into a chair surrounded by dials, under dazzling lights.
A man in a white coat was reading the dials. There was a tramp of heavy
boots outside. The door clanged open. The waxen-faced officer marched
in, followed by two guards.

``Room 101,'' said the officer.

The man in the white coat did not turn round. He did not look at Winston
either; he was looking only at the dials.

He was rolling down a mighty corridor, a kilometer wide, full of
glorious, golden light, roaring with laughter and shouting out
confessions at the top of his voice. He was confessing everything, even
the things he had succeeded in holding back under the torture. He was
relating the entire history of his life to an audience who knew it
already. With him were the guards, the other questioners, the men in
white coats, O\textquotesingle Brien, Julia, Mr. Charrington, all
rolling down the corridor together and shouting with laughter. Some
dreadful thing which had lain embedded in the future had somehow been
skipped over and had not happened. Everything was all right, there was
no more pain, the last detail of his life was laid bare, understood,
forgiven.

He was starting up from the plank bed in the half-certainty that he had
heard O\textquotesingle Brien\textquotesingle s voice. All through his
interrogation, although he had never seen him, he had had the feeling
that O\textquotesingle Brien was at his elbow, just out of sight. It was
O\textquotesingle Brien who was directing everything. It was he who set
the guards onto Winston and who prevented them from killing him. It was
he who decided when Winston should scream with pain, when he should have
a respite, when he should be fed, when he should sleep, when the drugs
should be pumped into his arm. It was he who asked the questions and
suggested the answers. He was the tormentor, he was the protector, he
was the inquisitor, he was the friend. And once---Winston could not
remember whether it was in drugged sleep, or in normal sleep, or even in
a moment of wakefulness---a voice murmured in his ear:
``Don\textquotesingle t worry, Winston; you are in my keeping. For seven
years I have watched over you. Now the turning point has come. I shall
save you, I shall make you perfect.'' He was not sure whether it was
O\textquotesingle Brien\textquotesingle s voice; but it was the same
voice that had said to him, ``We shall meet in the place where there is
no darkness,'' in that other dream, seven years ago.

He did not remember any ending to his interrogation. There was a period
of blackness and then the cell, or room, in which he now was had
gradually materialized round him. He was almost flat on his back, and
unable to move. His body was held down at every essential point. Even
the back of his head was gripped in some manner. O\textquotesingle Brien
was looking down at him gravely and rather sadly. His face, seen from
below, looked coarse and worn, with pouches under the eyes and tired
lines from nose to chin. He was older than Winston had thought him; he
was perhaps forty-eight or fifty. Under his hand there was a dial with a
lever on top and figures running round the face.

``I told you,'' said O\textquotesingle Brien, ``that if we met again it
would be here.''

``Yes,'' said Winston.

Without any warning except a slight movement of
O\textquotesingle Brien\textquotesingle s hand, a wave of pain flooded
his body. It was a frightening pain, because he could not see what was
happening, and he had the feeling that some mortal injury was being done
to him. He did not know whether the thing was really happening, or
whether the effect was electrically produced; but his body was being
wrenched out of shape, the joints were being slowly torn apart. Although
the pain had brought the sweat out on his forehead, the worst of all was
the fear that his backbone was about to snap. He set his teeth and
breathed hard through his nose, trying to keep silent as long as
possible.

``You are afraid,'' said O\textquotesingle Brien, watching his face, ``that
in another moment something is going to break. Your especial fear is
that it will be your backbone. You have a vivid mental picture of the
vertebrae snapping apart and the spinal fluid dripping out of them. That
is what you are thinking, is it not, Winston?''

Winston did not answer. O\textquotesingle Brien drew back the lever on
the dial. The wave of pain receded almost as quickly as it had come.

``That was forty,'' said O\textquotesingle Brien. ``You can see that the
numbers on this dial run up to a hundred. Will you please remember,
throughout our conversation, that I have it in my power to inflict pain
on you at any moment and to whatever degree I choose. If you tell me any
lies, or attempt to prevaricate in any way, or even fall below your
usual level of intelligence, you will cry out with pain, instantly. Do
you understand that?''

``Yes,'' said Winston.

O\textquotesingle Brien\textquotesingle s manner became less severe. He
resettled his spectacles thoughtfully, and took a pace or two up and
down. When he spoke his voice was gentle and patient. He had the air of
a doctor, a teacher, even a priest, anxious to explain and persuade
rather than to punish.

``I am taking trouble with you, Winston,'' he said, ``because you are worth
trouble. You know perfectly well what is the matter with you. You have
known it for years, though you have fought against the knowledge. You
are mentally deranged. You suffer from a defective memory. You are
unable to remember real events, and you persuade yourself that you
remember other events which never happened. Fortunately it is curable.
You have never cured yourself of it, because you did not choose to.
There was a small effort of the will that you were not ready to make.
Even now, I am well aware, you are clinging to your disease under the
impression that it is a virtue. Now we will take an example. At this
moment, which power is Oceania at war with?''

``When I was arrested, Oceania was at war with Eastasia.''

``With Eastasia. Good. And Oceania has always been at war with Eastasia,
has it not?''

Winston drew in his breath. He opened his mouth to speak and then did
not speak. He could not take his eyes away from the dial.

``The truth, please, Winston. \emph{Your} truth. Tell me what you think
you remember.''

``I remember that until only a week before I was arrested, we were not at
war with Eastasia at all. We were in alliance with them. The war was
against Eurasia. That had lasted for four years. Before that---''

O\textquotesingle Brien stopped him with a movement of the hand.

``Another example,'' he said. ``Some years ago you had a very serious
delusion indeed. You believed that three men, three one-time Party
members named Jones, Aaronson, and Rutherford---men who were executed
for treachery and sabotage after making the fullest possible
confession---were not guilty of the crimes they were charged with. You
believed that you had seen unmistakable documentary evidence proving
that their confessions were false. There was a certain photograph about
which you had a hallucination. You believed that you had actually held
it in your hands. It was a photograph something like this.''

An oblong slip of newspaper had appeared between
O\textquotesingle Brien\textquotesingle s fingers. For perhaps five
seconds it was within the angle of Winston\textquotesingle s vision. It
was a photograph, and there was no question of its identity. It was
\emph{the} photograph. It was another copy of the photograph of Jones,
Aaronson, and Rutherford at the Party function in New York, which he had
chanced upon eleven years ago and promptly destroyed. For only an
instant it was before his eyes, then it was out of sight again. But he
had seen it, unquestionably he had seen it! He made a desperate,
agonizing effort to wrench the top half of his body free. It was
impossible to move so much as a centimeter in any direction. For the
moment he had even forgotten the dial. All he wanted was to hold the
photograph in his fingers again, or at least to see it.

``It exists!'' he cried.

``No,'' said O\textquotesingle Brien.

He stepped across the room. There was a memory hole in the opposite
wall. O\textquotesingle Brien lifted the grating. Unseen, the frail slip
of paper was whirling away on the current of warm air; it was vanishing
in a flash of flame. O\textquotesingle Brien turned away from the wall.

``Ashes,'' he said. ``Not even identifiable ashes. Dust. It does not exist.
It never existed.''

``But it did exist! It does exist! It exists in memory. I remember it.
You remember it.''

``I do not remember it,'' said O\textquotesingle Brien.

Winston\textquotesingle s heart sank. That was doublethink. He had a
feeling of deadly helplessness. If he could have been certain that
O\textquotesingle Brien was lying, it would not have seemed to matter.
But it was perfectly possible that O\textquotesingle Brien had really
forgotten the photograph. And if so, then already he would have
forgotten his denial of remembering it, and forgotten the act of
forgetting. How could one be sure that it was simply trickery? Perhaps
that lunatic dislocation in the mind could really happen: that was the
thought that defeated him.

O\textquotesingle Brien was looking down at him speculatively. More than
ever he had the air of a teacher taking pains with a wayward but
promising child.

``There is a Party slogan dealing with the control of the past,'' he said.
``Repeat it, if you please.''

``\textquotesingle Who controls the past controls the future; who
controls the present controls the past,\textquotesingle'' repeated
Winston obediently.

``\textquotesingle Who controls the present controls the
past,\textquotesingle'' said O\textquotesingle Brien, nodding his head
with slow approval. ``Is it your opinion, Winston, that the past has real
existence?''

Again the feeling of helplessness descended upon Winston. His eyes
flitted toward the dial. He not only did not know whether ``yes'' or ``no''
was the answer that would save him from pain; he did not even know which
answer he believed to be the true one.

O\textquotesingle Brien smiled faintly. ``You are no metaphysician,
Winston,'' he said. ``Until this moment you had never considered what is
meant by existence. I will put it more precisely. Does the past exist
concretely, in space? Is there somewhere or other a place, a world of
solid objects, where the past is still happening?''

``No.''

``Then where does the past exist, if at all?''

``In records. It is written down.''

``In records. And---?''

``In the mind. In human memories.''

``In memory. Very well, then. We, the Party, control all records, and we
control all memories. Then we control the past, do we not?''

``But how can you stop people remembering things?'' cried Winston, again
momentarily forgetting the dial. ``It is involuntary. It is outside
oneself. How can you control memory? You have not controlled mine!''

O\textquotesingle Brien\textquotesingle s manner grew stern again. He
laid his hand on the dial.

``On the contrary,'' he said, ``\emph{you} have not controlled it. That is
what has brought you here. You are here because you have failed in
humility, in self-discipline. You would not make the act of submission
which is the price of sanity. You preferred to be a lunatic, a minority
of one. Only the disciplined mind can see reality, Winston. You believe
that reality is something objective, external, existing in its own
right. You also believe that the nature of reality is self-evident. When
you delude yourself into thinking that you see something, you assume
that everyone else sees the same thing as you. But I tell you, Winston,
that reality is not external. Reality exists in the human mind, and
nowhere else. Not in the individual mind, which can make mistakes, and
in any case soon perishes; only in the mind of the Party, which is
collective and immortal. Whatever the Party holds to be truth \emph{is}
truth. It is impossible to see reality except by looking through the
eyes of the Party. That is the fact that you have got to relearn,
Winston. It needs an act of self-destruction, an effort of the will. You
must humble yourself before you can become sane.''

He paused for a few moments, as though to allow what he had been saying
to sink in.

``Do you remember,'' he went on, ``writing in your diary,
\textquotesingle Freedom is the freedom to say that two plus two make
four\textquotesingle?''

``Yes,'' said Winston.

O\textquotesingle Brien held up his left hand, its back toward Winston,
with the thumb hidden and the four fingers extended.

``How many fingers am I holding up, Winston?''

``Four.''

``And if the Party says that it is not four but five---then how many?''

``Four.''

The word ended in a gasp of pain. The needle of the dial had shot up to
fifty-five. The sweat had sprung out all over Winston\textquotesingle s
body. The air tore into his lungs and issued again in deep groans which
even by clenching his teeth he could not stop. O\textquotesingle Brien
watched him, the four fingers still extended. He drew back the lever.
This time the pain was only slightly eased.

``How many fingers, Winston?''

``Four.''

The needle went up to sixty.

``How many fingers, Winston?''

``Four! Four! What else can I say? Four!''

The needle must have risen again, but he did not look at it. The heavy,
stern face and the four fingers filled his vision. The fingers stood up
before his eyes like pillars, enormous, blurry, and seeming to vibrate,
but unmistakably four.

``How many fingers, Winston?''

``Four! Stop it, stop it! How can you go on? Four! Four!''

``How many fingers, Winston?''

``Five! Five! Five!''

``No, Winston, that is no use. You are lying. You still think there are
four. How many fingers, please?''

``Four! Five! Four! Anything you like. Only stop it, stop the pain!''

Abruptly he was sitting up with
O\textquotesingle Brien\textquotesingle s arm round his shoulders. He
had perhaps lost consciousness for a few seconds. The bonds that had
held his body down were loosened. He felt very cold, he was shaking
uncontrollably, his teeth were chattering, the tears were rolling down
his cheeks. For a moment he clung to O\textquotesingle Brien like a
baby, curiously comforted by the heavy arm round his shoulders. He had
the feeling that O\textquotesingle Brien was his protector, that the
pain was something that came from outside, from some other source, and
that it was O\textquotesingle Brien who would save him from it.

``You are a slow learner, Winston,'' said O\textquotesingle Brien gently.

``How can I help it?'' he blubbered. ``How can I help seeing what is in
front of my eyes? Two and two are four.''

``Sometimes, Winston. Sometimes they are five. Sometimes they are three.
Sometimes they are all of them at once. You must try harder. It is not
easy to become sane.''

He laid Winston down on the bed. The grip on his limbs tightened again,
but the pain had ebbed away and the trembling had stopped, leaving him
merely weak and cold. O\textquotesingle Brien motioned with his head to
the man in the white coat, who had stood immobile throughout the
proceedings. The man in the white coat bent down and looked closely into
Winston\textquotesingle s eyes, felt his pulse, laid an ear against his
chest, tapped here and there; then he nodded to O\textquotesingle Brien.

``Again,'' said O\textquotesingle Brien.

The pain flowed into Winston\textquotesingle s body. The needle must be
at seventy, seventy-five. He had shut his eyes this time. He knew that
the fingers were still there, and still four. All that mattered was
somehow to stay alive until the spasm was over. He had ceased to notice
whether he was crying out or not. The pain lessened again. He opened his
eyes. O\textquotesingle Brien had drawn back the lever.

``How many fingers, Winston?''

``Four. I suppose there are four. I would see five if I could. I am
trying to see five.''

``Which do you wish: to persuade me that you see five, or really to see
them?''

``Really to see them.''

``Again,'' said O\textquotesingle Brien.

Perhaps the needle was at eighty---ninety. Winston could only
intermittently remember why the pain was happening. Behind his
screwed-up eyelids a forest of fingers seemed to be moving in a sort of
dance, weaving in and out, disappearing behind one another and
reappearing again. He was trying to count them, he could not remember
why. He knew only that it was impossible to count them, and that this
was somehow due to the mysterious identity between five and four. The
pain died down again. When he opened his eyes it was to find that he was
still seeing the same thing. Innumerable fingers, like moving trees,
were still streaming past in either direction, crossing and recrossing.
He shut his eyes again.

``How many fingers am I holding up, Winston?''

``I don\textquotesingle t know. I don\textquotesingle t know. You will
kill me if you do that again. Four, five, six---in all honesty I
don\textquotesingle t know.''

``Better,'' said O\textquotesingle Brien.

A needle slid into Winston\textquotesingle s arm. Almost in the same
instant a blissful, healing warmth spread all through his body. The pain
was already half-forgotten. He opened his eyes and looked up gratefully
at O\textquotesingle Brien. At sight of the heavy, lined face, so ugly
and so intelligent, his heart seemed to turn over. If he could have
moved he would have stretched out a hand and laid it on
O\textquotesingle Brien\textquotesingle s arm. He had never loved him so
deeply as at this moment, and not merely because he had stopped the
pain. The old feeling, that at bottom it did not matter whether
O\textquotesingle Brien was a friend or an enemy, had come back.
O\textquotesingle Brien was a person who could be talked to. Perhaps one
did not want to be loved so much as to be understood.
O\textquotesingle Brien had tortured him to the edge of lunacy, and in a
little while, it was certain, he would send him to his death. It made no
difference. In some sense that went deeper than friendship, they were
intimates; somewhere or other, although the actual words might never be
spoken, there was a place where they could meet and talk.
O\textquotesingle Brien was looking down at him with an expression which
suggested that the same thought might be in his own mind. When he spoke
it was in an easy, conversational tone.

``Do you know where you are, Winston?'' he said.

``I don\textquotesingle t know. I can guess. In the Ministry of Love.''

``Do you know how long you have been here?''

``I don\textquotesingle t know. Days, weeks, months---I think it is
months.''

``And why do you imagine that we bring people to this place?''

``To make them confess.''

``No, that is not the reason. Try again.''

``To punish them.''

``No!'' exclaimed O\textquotesingle Brien. His voice had changed
extraordinarily, and his face had suddenly become both stern and
animated. ``No! Not merely to extract your confession, nor to punish you.
Shall I tell you why we have brought you here? To cure you! To make you
sane! Will you understand, Winston, that no one whom we bring to this
place ever leaves our hands uncured? We are not interested in those
stupid crimes that you have committed. The Party is not interested in
the overt act; the thought is all we care about. We do not merely
destroy our enemies; we change them. Do you understand what I mean by
that?''

He was bending over Winston. His face looked enormous because of its
nearness, and hideously ugly because it was seen from below. Moreover it
was filled with a sort of exaltation, a lunatic intensity. Again
Winston\textquotesingle s heart shrank. If it had been possible he would
have cowered deeper into the bed. He felt certain that
O\textquotesingle Brien was about to twist the dial out of sheer
wantonness. At this moment, however, O\textquotesingle Brien turned
away. He took a pace or two up and down. Then he continued less
vehemently:

``The first thing for you to understand is that in this place there are
no martyrdoms. You have read of the religious persecutions of the past.
In the Middle Ages there was the Inquisition. It was a failure. It set
out to eradicate heresy, and ended by perpetuating it. For every heretic
it burned at the stake, thousands of others rose up. Why was that?
Because the Inquisition killed its enemies in the open, and killed them
while they were still unrepentant; in fact, it killed them because they
were unrepentant. Men were dying because they would not abandon their
true beliefs. Naturally all the glory belonged to the victim and all the
shame to the Inquisitor who burned him. Later, in the twentieth century,
there were the totalitarians, as they were called. There were the German
Nazis and the Russian Communists. The Russians persecuted heresy more
cruelly than the Inquisition had done. And they imagined that they had
learned from the mistakes of the past; they knew, at any rate, that one
must not make martyrs. Before they exposed their victims to public
trial, they deliberately set themselves to destroy their dignity. They
wore them down by torture and solitude until they were despicable,
cringing wretches, confessing whatever was put into their mouths,
covering themselves with abuse, accusing and sheltering behind one
another, whimpering for mercy. And yet after only a few years the same
thing had happened over again. The dead men had become martyrs and their
degradation was forgotten. Once again, why was it? In the first place,
because the confessions that they had made were obviously extorted and
untrue. We do not make mistakes of that kind. All the confessions that
are uttered here are true. We make them true. And, above all, we do not
allow the dead to rise up against us. You must stop imagining that
posterity will vindicate you, Winston. Posterity will never hear of you.
You will be lifted clean out from the stream of history. We shall turn
you into gas and pour you into the stratosphere. Nothing will remain of
you: not a name in a register, not a memory in a living brain. You will
be annihilated in the past as well as in the future. You will never have
existed.''

Then why bother to torture me? thought Winston, with a momentary
bitterness. O\textquotesingle Brien checked his step as though Winston
had uttered the thought aloud. His large ugly face came nearer, with the
eyes a little narrowed.

``You are thinking,'' he said, ``that since we intend to destroy you
utterly, so that nothing that you say or do can make the smallest
difference---in that case, why do we go to the trouble of interrogating
you first? That is what you were thinking, was it not?''

``Yes,'' said Winston.

O\textquotesingle Brien smiled slightly. ``You are a flaw in the pattern,
Winston. You are a stain that must be wiped out. Did I not tell you just now
that we are different from the persecutors of the past? We are not content
with negative obedience, nor even with the most abject submission. When
finally you surrender to us, it must be of your own free will. We do not
destroy the heretic because he resists us; so long as he resists us we never
destroy him. We convert him, we capture his inner mind, we reshape him. We
burn all evil and all illusion out of him; we bring him over to our side,
not in appearance, but genuinely, heart and soul. We make him one of
ourselves before we kill him. It is intolerable to us that an erroneous
thought should exist anywhere in the world, however secret and powerless it
may be. Even in the instant of death we cannot permit any deviation. In the
old days the heretic walked to the stake still a heretic, proclaiming his
heresy, exulting in it. Even the victim of the Russian purges could carry
rebellion locked up in his skull as he walked down the passage waiting for
the bullet. But we make the brain perfect before we blow it out. The command
of the old despotisms was `Thou shalt not.' The command of the totalitarians
was `Thou shalt.' Our command is `\emph{Thou
  art}.' No one whom we bring to this place ever stands out
against us. Everyone is washed clean. Even those three miserable traitors in
whose innocence you once believed---Jones, Aaronson, and Rutherford---in the
end we broke them down. I took part in their interrogation myself. I saw
them gradually worn down, whimpering, groveling, weeping---and in the end it
was not with pain or fear, only with penitence. By the time we had finished
with them they were only the shells of men. There was nothing left in them
except sorrow for what they had done, and love of Big Brother. It was
touching to see how they loved him. They begged to be shot quickly, so that
they could die while their minds were still clean.''

His voice had grown almost dreamy. The exaltation, the lunatic
enthusiasm, was still in his face. He is not pretending, thought
Winston; he is not a hypocrite; he believes every word he says. What
most oppressed him was the consciousness of his own intellectual
inferiority. He watched the heavy yet graceful form strolling to and
fro, in and out of the range of his vision. O\textquotesingle Brien was
a being in all ways larger than himself. There was no idea that he had
ever had, or could have, that O\textquotesingle Brien had not long ago
known, examined, and rejected. His mind \emph{contained}
Winston\textquotesingle s mind. But in that case how could it be true
that O\textquotesingle Brien was mad? It must be he, Winston, who was
mad. O\textquotesingle Brien halted and looked down at him. His voice
had grown stern again.

``Do not imagine that you will save yourself, Winston, however completely
you surrender to us. No one who has once gone astray is ever spared. And
even if we chose to let you live out the natural term of your life,
still you would never escape from us. What happens to you here is
forever. Understand that in advance. We shall crush you down to the
point from which there is no coming back. Things will happen to you from
which you could not recover, if you lived a thousand years. Never again
will you be capable of ordinary human feeling. Everything will be dead
inside you. Never again will you be capable of love, or friendship, or
joy of living, or laughter, or curiosity, or courage, or integrity. You
will be hollow. We shall squeeze you empty, and then we shall fill you
with ourselves.''

He paused and signed to the man in the white coat. Winston was aware of
some heavy piece of apparatus being pushed into place behind his head.
O\textquotesingle Brien had sat down beside the bed, so that his face
was almost on a level with Winston\textquotesingle s.

``Three thousand,'' he said, speaking over Winston\textquotesingle s head
to the man in the white coat.

Two soft pads, which felt slightly moist, clamped themselves against
Winston\textquotesingle s temples. He quailed. There was pain coming, a
new kind of pain. O\textquotesingle Brien laid a hand reassuringly,
almost kindly, on his.

``This time it will not hurt,'' he said. ``Keep your eyes fixed on mine.''

At this moment there was a devastating explosion, or what seemed like an
explosion, though it was not certain whether there was any noise. There
was undoubtedly a blinding flash of light. Winston was not hurt, only
prostrated. Although he had already been lying on his back when the
thing happened, he had a curious feeling that he had been knocked into
that position. A terrific, painless blow had flattened him out. Also
something had happened inside his head. As his eyes regained their focus
he remembered who he was, and where he was, and recognized the face that
was gazing into his own; but somewhere or other there was a large patch
of emptiness, as though a piece had been taken out of his brain.

``It will not last,'' said O\textquotesingle Brien. ``Look me in the eyes.
What country is Oceania at war with?''

Winston thought. He knew what was meant by Oceania, and that he himself
was a citizen of Oceania. He also remembered Eurasia and Eastasia; but
who was at war with whom he did not know. In fact he had not been aware
that there was any war.

``I don\textquotesingle t remember.''

``Oceania is at war with Eastasia. Do you remember that now?''

``Yes.''

``Oceania has always been at war with Eastasia. Since the beginning of
your life, since the beginning of the Party, since the beginning of
history, the war has continued without a break, always the same war. Do
you remember that?''

``Yes.''

``Eleven years ago you created a legend about three men who had been
condemned to death for treachery. You pretended that you had seen a
piece of paper which proved them innocent. No such piece of paper ever
existed. You invented it, and later you grew to believe in it. You
remember now the very moment at which you first invented it. Do you
remember that?''

``Yes.''

``Just now I held up the fingers of my hand to you. You saw five fingers.
Do you remember that?''

``Yes.''

O\textquotesingle Brien held up the fingers of his left hand, with the
thumb concealed.

``There are five fingers there. Do you see five fingers?''

``Yes.''

And he did see them, for a fleeting instant, before the scenery of his
mind changed. He saw five fingers, and there was no deformity. Then
everything was normal again, and the old fear, the hatred, and the
bewilderment came crowding back again. But there had been a moment---he
did not know how long, thirty seconds, perhaps---of luminous certainty,
when each new suggestion of O\textquotesingle Brien\textquotesingle s
had filled up a patch of emptiness and become absolute truth, and when
two and two could have been three as easily as five, if that were what
was needed. It had faded out before O\textquotesingle Brien had dropped
his hand; but though he could not recapture it, he could remember it, as
one remembers a vivid experience at some remote period of
one\textquotesingle s life when one was in effect a different person.

``You see now,'' said O\textquotesingle Brien, ``that it is at any rate
possible.''

``Yes,'' said Winston.

O\textquotesingle Brien stood up with a satisfied air. Over to his left
Winston saw the man in the white coat break an ampoule and draw back the
plunger of a syringe. O\textquotesingle Brien turned to Winston with a
smile. In almost the old manner he resettled his spectacles on his nose.

``Do you remember writing in your diary,'' he said, ``that it did not
matter whether I was a friend or an enemy, since I was at least a person
who understood you and could be talked to? You were right. I enjoy
talking to you. Your mind appeals to me. It resembles my own mind except
that you happen to be insane. Before we bring the session to an end you
can ask me a few questions, if you choose.''

``Any question I like?''

``Anything.'' He saw that Winston\textquotesingle s eyes were upon the
dial. ``It is switched off. What is your first question?''

``What have you done with Julia?'' said Winston.

O\textquotesingle Brien smiled again. ``She betrayed you, Winston.
Immediately---unreservedly. I have seldom seen anyone come over to us so
promptly. You would hardly recognize her if you saw her. All her
rebelliousness, her deceit, her folly, her dirty-mindedness---everything
has been burned out of her. It was a perfect conversion, a textbook
case.''

``You tortured her.''

O\textquotesingle Brien left this unanswered. ``Next question,'' he said.

``Does Big Brother exist?''

``Of course he exists. The Party exists. Big Brother is the embodiment of
the Party.''

``Does he exist in the same way as I exist?''

``You do not exist,'' said O\textquotesingle Brien.

Once again the sense of helplessness assailed him. He knew, or he could
imagine, the arguments which proved his own nonexistence; but they were
nonsense, they were only a play on words. Did not the statement, ``You do
not exist,'' contain a logical absurdity? But what use was it to say so?
His mind shriveled as he thought of the unanswerable, mad arguments with
which O\textquotesingle Brien would demolish him.

``I think I exist,'' he said wearily. ``I am conscious of my own identity.
I was born, and I shall die. I have arms and legs. I occupy a particular
point in space. No other solid object can occupy the same point
simultaneously. In that sense, does Big Brother exist?''

``It is of no importance. He exists?''

``Will Big Brother ever die?''

``Of course not. How could he die? Next question.''

``Does the Brotherhood exist?''

``That, Winston, you will never know. If we choose to set you free when
we have finished with you, and if you live to be ninety years old, still
you will never learn whether the answer to that question is Yes or No.
As long as you live, it will be an unsolved riddle in your mind.''

Winston lay silent. His breast rose and fell a little faster. He still
had not asked the question that had come into his mind the first. He had
got to ask it, and yet it was as though his tongue would not utter it.
There was a trace of amusement in
O\textquotesingle Brien\textquotesingle s face. Even his spectacles
seemed to wear an ironical gleam. He knows, thought Winston suddenly, he
knows what I am going to ask! At the thought the words burst out of him:

``What is in Room 101?''

The expression on O\textquotesingle Brien\textquotesingle s face did not
change. He answered drily:

``You know what is in Room 101, Winston. Everyone knows what is in Room
101.''

He raised a finger to the man in the white coat. Evidently the session
was at an end. A needle jerked into Winston\textquotesingle s arm. He
sank almost instantly into deep sleep.


\section{III}\label{iii-2}

``There are three stages in your reintegration,'' said
O\textquotesingle Brien. ``There is learning, there is understanding, and
there is acceptance. It is time for you to enter upon the second stage.''

As always, Winston was lying flat on his back. But of late his bonds
were looser. They still held him to the bed, but he could move his knees
a little and could turn his head from side to side and raise his arms
from the elbow. The dial, also, had grown to be less of a terror. He
could evade its pangs if he was quick-witted enough; it was chiefly when
he showed stupidity that O\textquotesingle Brien pulled the lever.
Sometimes they got through a whole session without use of the dial. He
could not remember how many sessions there had been. The whole process
seemed to stretch out over a long, indefinite time---weeks,
possibly---and the intervals between the sessions might sometimes have
been days, sometimes only an hour or two.

``As you lie there,'' said O\textquotesingle Brien, ``you have often
wondered---you have even asked me---why the Ministry of Love should
expend so much time and trouble on you. And when you were free you were
puzzled by what was essentially the same question. You could grasp the
mechanics of the society you lived in, but not its underlying motives.
Do you remember writing in your diary, \textquotesingle I understand
\emph{how}; I do not understand \emph{why}\textquotesingle? It was when
you thought about `why' that you
doubted your own sanity. You have read \emph{the book},
Goldstein\textquotesingle s book, or parts of it, at least. Did it tell
you anything that you did not know already?''

``You have read it?'' said Winston.

``I wrote it. That is to say, I collaborated in writing it. No book is
produced individually, as you know.''

``Is it true, what it says?''

``As description, yes. The program it sets forth is nonsense. The secret
accumulation of knowledge---a gradual spread of
enlightenment---ultimately a proletarian rebellion---the overthrow of
the Party. You foresaw yourself that that was what it would say. It is
all nonsense. The proletarians will never revolt, not in a thousand
years or a million. They cannot. I do not have to tell you the reason;
you know it already. If you have ever cherished any dreams of violent
insurrection, you must abandon them. There is no way in which the Party
can be overthrown. The rule of the Party is forever. Make that the
starting point of your thoughts.''

He came closer to the bed. ``Forever!'' he repeated. ``And now let us get
back to the question of `how' and `why.' You understand well enough
\emph{how} the Party maintains itself in power. Now tell me \emph{why} we
cling to power. What is our motive? Why should we want power? Go on,
speak,'' he added as Winston remained silent.

Nevertheless Winston did not speak for another moment or two. A feeling
of weariness had overwhelmed him. The faint, mad gleam of enthusiasm had
come back into O\textquotesingle Brien\textquotesingle s face. He knew
in advance what O\textquotesingle Brien would say: that the Party did
not seek power for its own ends, but only for the good of the majority.
That it sought power because men in the mass were frail, cowardly
creatures who could not endure liberty or face the truth, and must be
ruled over and systematically deceived by others who were stronger than
themselves. That the choice for mankind lay between freedom and
happiness, and that, for the great bulk of mankind, happiness was
better. That the Party was the eternal guardian of the weak, a dedicated
sect doing evil that good might come, sacrificing its own happiness to
that of others. The terrible thing, thought Winston, the terrible thing
was that when O\textquotesingle Brien said this he would believe it. You
could see it in his face. O\textquotesingle Brien knew everything. A
thousand times better than Winston, he knew what the world was really
like, in what degradation the mass of human beings lived and by what
lies and barbarities the Party kept them there. He had understood it
all, weighed it all, and it made no difference: all was justified by the
ultimate purpose. What can you do, thought Winston, against the lunatic
who is more intelligent than yourself, who gives your arguments a fair
hearing and then simply persists in his lunacy?

``You are ruling over us for our own good,'' he said feebly. ``You believe
that human beings are not fit to govern themselves, and therefore---''

He started and almost cried out. A pang of pain had shot through his
body. O\textquotesingle Brien had pushed the lever of the dial up to
thirty-five.

``That was stupid, Winston, stupid!'' he said. ``You should know better
than to say a thing like that.''

He pulled the lever back and continued:

``Now I will tell you the answer to my question. It is this. The Party
seeks power entirely for its own sake. We are not interested in the good
of others; we are interested solely in power. Not wealth or luxury or
long life or happiness; only power, pure power. What pure power means
you will understand presently. We are different from all the oligarchies
of the past in that we know what we are doing. All the others, even
those who resembled ourselves, were cowards and hypocrites. The German
Nazis and the Russian Communists came very close to us in their methods,
but they never had the courage to recognize their own motives. They
pretended, perhaps they even believed, that they had seized power
unwillingly and for a limited time, and that just round the corner there
lay a paradise where human beings would be free and equal. We are not
like that. We know that no one ever seizes power with the intention of
relinquishing it. Power is not a means; it is an end. One does not
establish a dictatorship in order to safeguard a revolution; one makes
the revolution in order to establish the dictatorship. The object of
persecution is persecution. The object of torture is torture. The object
of power is power. Now do you begin to understand me?''

Winston was struck, as he had been struck before, by the tiredness of
O\textquotesingle Brien\textquotesingle s face. It was strong and fleshy
and brutal, it was full of intelligence and a sort of controlled passion
before which he felt himself helpless; but it was tired. There were
pouches under the eyes, the skin sagged from the cheekbones.
O\textquotesingle Brien leaned over him, deliberately bringing the worn
face nearer.

``You are thinking,'' he said, ``that my face is old and tired. You are
thinking that I talk of power, and yet I am not even able to prevent the
decay of my own body. Can you not understand, Winston, that the
individual is only a cell? The weariness of the cell is the vigor of the
organism. Do you die when you cut your fingernails?''

He turned away from the bed and began strolling up and down again, one
hand in his pocket.

``We are the priests of power,'' he said. ``God is power. But at present
power is only a word so far as you are concerned. It is time for you to
gather some idea of what power means. The first thing you must realize
is that power is collective. The individual only has power in so far as
he ceases to be an individual. You know the Party slogan:
`Freedom is Slavery.' Has it ever
occurred to you that it is reversible? Slavery is freedom.
Alone---free---the human being is always defeated. It must be so,
because every human being is doomed to die, which is the greatest of all
failures. But if he can make complete, utter submission, if he can
escape from his identity, if he can merge himself in the Party so that
he \emph{is} the Party, then he is all-powerful and immortal. The second
thing for you to realize is that power is power over human beings. Over
the body---but, above all, over the mind. Power over matter---external
reality, as you would call it---is not important. Already our control
over matter is absolute.''

For a moment Winston ignored the dial. He made a violent effort to raise
himself into a sitting position, and merely succeeded in wrenching his
body painfully.

``But how can you control matter?'' he burst out. ``You
don\textquotesingle t even control the climate or the law of gravity.
And there are disease, pain, death---''

O\textquotesingle Brien silenced him by a movement of the hand. ``We
control matter because we control the mind. Reality is inside the skull.
You will learn by degrees, Winston. There is nothing that we could not
do. Invisibility, levitation---anything. I could float off this floor
like a soap bubble if I wished to. I do not wish to, because the Party
does not wish it. You must get rid of those nineteenth-century ideas
about the laws of nature. We make the laws of nature.''

``But you do not! You are not even masters of this planet. What about
Eurasia and Eastasia? You have not conquered them yet.''

``Unimportant. We shall conquer them when it suits us. And if we did not,
what difference would it make? We can shut them out of existence.
Oceania is the world.''

``But the world itself is only a speck of dust. And man is
tiny---helpless! How long has he been in existence? For millions of
years the earth was uninhabited.''

``Nonsense. The earth is as old as we are, no older. How could it be
older? Nothing exists except through human consciousness.''

``But the rocks are full of the bones of extinct animals---mammoths and
mastodons and enormous reptiles which lived here long before man was
ever heard of.''

``Have you ever seen those bones, Winston? Of course not.
Nineteenth-century biologists invented them. Before man there was
nothing. After man, if he could come to an end, there would be nothing.
Outside man there is nothing.''

``But the whole universe is outside us. Look at the stars! Some of them
are a million light-years away. They are out of our reach forever.''

``What are the stars?'' said O\textquotesingle Brien indifferently. ``They
are bits of fire a few kilometers away. We could reach them if we wanted
to. Or we could blot them out. The earth is the center of the universe.
The sun and the stars go round it.''

Winston made another convulsive movement. This time he did not say
anything. O\textquotesingle Brien continued as though answering a spoken
objection:

``For certain purposes, of course, that is not true. When we navigate the
ocean, or when we predict an eclipse, we often find it convenient to
assume that the earth goes round the sun and that the stars are millions
upon millions of kilometers away. But what of it? Do you suppose it is
beyond us to produce a dual system of astronomy? The stars can be near
or distant, according as we need them. Do you suppose our mathematicians
are unequal to that? Have you forgotten doublethink?''

Winston shrank back upon the bed. Whatever he said, the swift answer
crushed him like a bludgeon. And yet he knew, he \emph{knew}, that he
was in the right. The belief that nothing exists outside your own
mind---surely there must be some way of demonstrating that it was false.
Had it not been exposed long ago as a fallacy? There was even a name for
it, which he had forgotten. A faint smile twitched the corners of
O\textquotesingle Brien\textquotesingle s mouth as he looked down at
him.

``I told you, Winston,'' he said, ``that metaphysics is not your strong
point. The word you are trying to think of is solipsism. But you are
mistaken. This is not solipsism. Collective solipsism, if you like. But
that is a different thing; in fact, the opposite thing. All this is a
digression,'' he added in a different tone. ``The real power, the power we
have to fight for night and day, is not power over things, but over
men.'' He paused, and for a moment assumed again his air of a
schoolmaster questioning a promising pupil: ``How does one man assert his
power over another, Winston?''

Winston thought. ``By making him suffer,'' he said.

``Exactly. By making him suffer. Obedience is not enough. Unless he is
suffering, how can you be sure that he is obeying your will and not his
own? Power is in inflicting pain and humiliation. Power is in tearing
human minds to pieces and putting them together again in new shapes of
your own choosing. Do you begin to see, then, what kind of world we are
creating? It is the exact opposite of the stupid hedonistic Utopias that
the old reformers imagined. A world of fear and treachery and torment, a
world of trampling and being trampled upon, a world which will grow not
less but \emph{more} merciless as it refines itself. Progress in our
world will be progress toward more pain. The old civilizations claimed
that they were founded on love or justice. Ours is founded upon hatred.
In our world there will be no emotions except fear, rage, triumph, and
self-abasement. Everything else we shall destroy---everything. Already
we are breaking down the habits of thought which have survived from
before the Revolution. We have cut the links between child and parent,
and between man and man, and between man and woman. No one dares trust a
wife or a child or a friend any longer. But in the future there will be
no wives and no friends. Children will be taken from their mothers at
birth, as one takes eggs from a hen. The sex instinct will be
eradicated. Procreation will be an annual formality like the renewal of
a ration card. We shall abolish the orgasm. Our neurologists are at work
upon it now. There will be no loyalty, except loyalty toward the Party.
There will be no love, except the love of Big Brother. There will be no
laughter, except the laugh of triumph over a defeated enemy. There will
be no art, no literature, no science. When we are omnipotent we shall
have no more need of science. There will be no distinction between
beauty and ugliness. There will be no curiosity, no enjoyment of the
process of life. All competing pleasures will be destroyed. But
always---do not forget this, Winston---always there will be the
intoxication of power, constantly increasing and constantly growing
subtler. Always, at every moment, there will be the thrill of victory,
the sensation of trampling on an enemy who is helpless. If you want a
picture of the future, imagine a boot stamping on a human
face---forever.''

He paused as though he expected Winston to speak. Winston had tried to
shrink back into the surface of the bed again. He could not say
anything. His heart seemed to be frozen. O\textquotesingle Brien went
on:

``And remember that it is forever. The face will always be there to be
stamped upon. The heretic, the enemy of society, will always be there,
so that he can be defeated and humiliated over again. Everything that
you have undergone since you have been in our hands---all that will
continue, and worse. The espionage, the betrayals, the arrests, the
tortures, the executions, the disappearances will never cease. It will
be a world of terror as much as a world of triumph. The more the Party
is powerful, the less it will be tolerant; the weaker the opposition,
the tighter the despotism. Goldstein and his heresies will live forever.
Every day, at every moment, they will be defeated, discredited,
ridiculed, spat upon---and yet they will always survive. This drama that
I have played out with you during seven years will be played out over
and over again, generation after generation, always in subtler forms.
Always we shall have the heretic here at our mercy, screaming with pain,
broken up, contemptible---and in the end utterly penitent, saved from
himself, crawling to our feet of his own accord. That is the world that
we are preparing, Winston. A world of victory after victory, triumph
after triumph after triumph: an endless pressing, pressing, pressing
upon the nerve of power. You are beginning, I can see, to realize what
that world will be like. But in the end you will do more than understand
it. You will accept it, welcome it, become part of it.''

Winston had recovered himself sufficiently to speak. ``You
can\textquotesingle t!'' he said weakly.

``What do you mean by that remark, Winston?''

``You could not create such a world as you have just described. It is a
dream. It is impossible.''

``Why?''

``It is impossible to found a civilization on fear and hatred and
cruelty. It would never endure.''

``Why not?''

``It would have no vitality. It would disintegrate. It would commit
suicide.''

``Nonsense. You are under the impression that hatred is more exhausting
than love. Why should it be? And if it were, what difference would that
make? Suppose that we choose to wear ourselves out faster. Suppose that
we quicken the tempo of human life till men are senile at thirty. Still
what difference would it make? Can you not understand that the death of
the individual is not death? The Party is immortal.''

As usual, the voice had battered Winston into helplessness. Moreover he
was in dread that if he persisted in his disagreement
O\textquotesingle Brien would twist the dial again. And yet he could not
keep silent. Feebly, without arguments, with nothing to support him
except his inarticulate horror of what O\textquotesingle Brien had said,
he returned to the attack.

``I don\textquotesingle t know---I don\textquotesingle t care. Somehow
you will fail. Something will defeat you. Life will defeat you.''

``We control life, Winston, at all its levels. You are imagining that
there is something called human nature which will be outraged by what we
do and will turn against us. But we create human nature. Men are
infinitely malleable. Or perhaps you have returned to your old idea that
the proletarians or the slaves will arise and overthrow us. Put it out
of your mind. They are helpless, like the animals. Humanity is the
Party. The others are outside---irrelevant.''

``I don\textquotesingle t care. In the end they will beat you. Sooner or
later they will see you for what you are, and then they will tear you to
pieces.''

``Do you see any evidence that that is happening? Or any reason why it
should?''

``No. I believe it. I \emph{know} that you will fail. There is something
in the universe---I don\textquotesingle t know, some spirit, some
principle---that you will never overcome.''

``Do you believe in God, Winston?''

``No.''

``Then what is it, this principle that will defeat us?''

``I don\textquotesingle t know. The spirit of Man.''

``And do you consider yourself a man?''

``Yes.''

``If you are a man, Winston, you are the last man. Your kind is extinct;
we are the inheritors. Do you understand that you are \emph{alone}? You
are outside history, you are nonexistent.'' His manner changed and he
said more harshly: ``And you consider yourself morally superior to us,
with our lies and our cruelty?''

``Yes, I consider myself superior.''

O\textquotesingle Brien did not speak. Two other voices were speaking.
After a moment Winston recognized one of them as his own. It was a sound
track of the conversation he had had with O\textquotesingle Brien, on
the night when he had enrolled himself in the Brotherhood. He heard
himself promising to lie, to steal, to forge, to murder, to encourage
drug taking and prostitution, to disseminate venereal diseases, to throw
vitriol in a child\textquotesingle s face. O\textquotesingle Brien made
a small impatient gesture, as though to say that the demonstration was
hardly worth making. Then he turned a switch and the voices stopped.

``Get up from that bed,'' he said.

The bonds had loosened themselves. Winston lowered himself to the floor
and stood up unsteadily.

``You are the last man,'' said O\textquotesingle Brien. ``You are the
guardian of the human spirit. You shall see yourself as you are. Take
off your clothes.''

Winston undid the bit of string that held his overalls together. The zip
fastener had long since been wrenched out of them. He could not remember
whether at any time since his arrest he had taken off all his clothes at
one time. Beneath the overalls his body was looped with filthy yellowish
rags, just recognizable as the remnants of underclothes. As he slid them
to the ground he saw that there was a three-sided mirror at the far end
of the room. He approached it, then stopped short. An involuntary cry
had broken out of him.

``Go on,'' said O\textquotesingle Brien. ``Stand between the wings of the
mirror. You shall see the side view as well.''

He had stopped because he was frightened. A bowed, gray-colored,
skeletonlike thing was coming toward him. Its actual appearance was
frightening, and not merely the fact that he knew it to be himself. He
moved closer to the glass. The creature\textquotesingle s face seemed to
be protruded, because of its bent carriage. A forlorn,
jailbird\textquotesingle s face with a nobby forehead running back into
a bald scalp, a crooked nose and battered-looking cheekbones above which
the eyes were fierce and watchful. The cheeks were seamed, the mouth had
a drawn-in look. Certainly it was his own face, but it seemed to him
that it had changed more than he had changed inside. The emotions it
registered would be different from the ones he felt. He had gone
partially bald. For the first moment he had thought that he had gone
gray as well, but it was only the scalp that was gray. Except for his
hands and a circle of his face, his body was gray all over with ancient,
ingrained dirt. Here and there under the dirt there were the red scars
of wounds, and near the ankle the varicose ulcer was an inflamed mass
with flakes of skin peeling off it. But the truly frightening thing was
the emaciation of his body. The barrel of the ribs was as narrow as that
of a skeleton; the legs had shrunk so that the knees were thicker than
the thighs. He saw now what O\textquotesingle Brien had meant about
seeing the side view. The curvature of the spine was astonishing. The
thin shoulders were hunched forward so as to make a cavity of the chest,
the scraggy neck seemed to be bending double under the weight of the
skull. At a guess he would have said that it was the body of a man of
sixty, suffering from some malignant disease.

``You have thought sometimes,'' said O\textquotesingle Brien, ``that my
face---the face of a member of the Inner Party---looks old and worn.
What do you think of your own face?''

He seized Winston\textquotesingle s shoulder and spun him round so that
he was facing him.

``Look at the condition you are in!'' he said. ``Look at this filthy grime
all over your body. Look at the dirt between your toes. Look at that
disgusting running sore on your leg. Do you know that you stink like a
goat? Probably you have ceased to notice it. Look at your emaciation. Do
you see? I can make my thumb and forefinger meet around your bicep. I
could snap your neck like a carrot. Do you know that you have lost
twenty-five kilograms since you have been in our hands? Even your hair
is coming out in handfuls. Look!'' He plucked at
Winston\textquotesingle s head and brought away a tuft of hair. ``Open
your mouth. Nine, ten, eleven teeth left. How many had you when you came
to us? And the few you have left are dropping out of your head. Look
here!''

He seized one of Winston\textquotesingle s remaining front teeth between
his powerful thumb and forefinger. A twinge of pain shot through
Winston\textquotesingle s jaw. O\textquotesingle Brien had wrenched the
loose tooth out by the roots. He tossed it across the cell.

``You are rotting away,'' he said; ``you are falling to pieces. What are
you? A bag of filth. Now turn round and look into that mirror again. Do
you see that thing facing you? That is the last man. If you are human,
that is humanity. Now put your clothes on again.''

Winston began to dress himself with slow stiff movements. Until now he
had not seemed to notice how thin and weak he was. Only one thought
stirred in his mind: that he must have been in this place longer than he
had imagined. Then suddenly as he fixed the miserable rags round himself
a feeling of pity for his ruined body overcame him. Before he knew what
he was doing he had collapsed onto a small stool that stood beside the
bed and burst into tears. He was aware of his ugliness, his
gracelessness, a bundle of bones in filthy underclothes sitting weeping
in the harsh white light; but he could not stop himself.
O\textquotesingle Brien laid a hand on his shoulder, almost kindly.

``It will not last forever,'' he said. ``You can escape from it whenever
you choose. Everything depends on yourself.''

``You did it!'' sobbed Winston. ``You reduced me to this state.''

``No, Winston, you reduced yourself to it. This is what you accepted when
you set yourself up against the Party. It was all contained in that
first act. Nothing has happened that you did not foresee.''

He paused, and then went on:

``We have beaten you, Winston. We have broken you up. You have seen what
your body is like. Your mind is in the same state. I do not think there
can be much pride left in you. You have been kicked and flogged and
insulted, you have screamed with pain, you have rolled on the floor in
your own blood and vomit. You have whimpered for mercy, you have
betrayed everybody and everything. Can you think of a single degradation
that has not happened to you?''

Winston had stopped weeping, though the tears were still oozing out of
his eyes. He looked up at O\textquotesingle Brien.

``I have not betrayed Julia,'' he said.

O\textquotesingle Brien looked down at him thoughtfully. ``No,'' he said,
``no; that is perfectly true. You have not betrayed Julia.''

The peculiar reverence for O\textquotesingle Brien, which nothing seemed
able to destroy, flooded Winston\textquotesingle s heart again. How
intelligent, he thought, how intelligent! Never did
O\textquotesingle Brien fail to understand what was said to him. Anyone
else on earth would have answered promptly that he \emph{had} betrayed
Julia. For what was there that they had not screwed out of him under the
torture? He had told them everything he knew about her, her habits, her
character, her past life; he had confessed in the most trivial detail
everything that had happened at their meetings, all that he had said to
her and she to him, their black-market meals, their adulteries, their
vague plottings against the Party---everything. And yet, in the sense in
which he intended the word, he had not betrayed her. He had not stopped
loving her; his feeling toward her had remained the same.
O\textquotesingle Brien had seen what he meant without the need for
explanation.

``Tell me,'' he said, ``how soon will they shoot me?''

``It might be a long time,'' said O\textquotesingle Brien. ``You are a
difficult case. But don\textquotesingle t give up hope. Everyone is
cured sooner or later. In the end we shall shoot you.''


\section{IV}\label{iv-2}

He was much better. He was growing fatter and stronger every day, if it
was proper to speak of days.

The white light and the humming sound were the same as ever, but the
cell was a little more comfortable than the others he had been in. There
were a pillow and a mattress on the plank bed, and a stool to sit on.
They had given him a bath, and they allowed him to wash himself fairly
frequently in a tin basin. They even gave him warm water to wash with.
They had given him new underclothes and a clean suit of overalls. They
had dressed his varicose ulcer with soothing ointment. They had pulled
out the remnants of his teeth and given him a new set of dentures.

Weeks or months must have passed. It would have been possible now to
keep count of the passage of time, if he had felt any interest in doing
so, since he was being fed at what appeared to be regular intervals. He
was getting, he judged, three meals in the twenty-four hours; sometimes
he wondered dimly whether he was getting them by night or by day. The
food was surprisingly good, with meat at every third meal. Once there
was even a packet of cigarettes. He had no matches, but the
never-speaking guard who brought his food would give him a light. The
first time he tried to smoke it made him sick, but he persevered, and
spun the packet out for a long time, smoking half a cigarette after each
meal.

They had given him a white slate with a stump of pencil tied to the
corner. At first he made no use of it. Even when he was awake he was
completely torpid. Often he would lie from one meal to the next almost
without stirring, sometimes asleep, sometimes waking into vague reveries
in which it was too much trouble to open his eyes. He had long grown
used to sleeping with a strong light on his face. It seemed to make no
difference, except that one\textquotesingle s dreams were more coherent.
He dreamed a great deal all through this time, and they were always
happy dreams. He was in the Golden Country, or he was sitting among
enormous, glorious, sunlit ruins, with his mother, with Julia, with
O\textquotesingle Brien---not doing anything, merely sitting in the sun,
talking of peaceful things. Such thoughts as he had when he was awake
were mostly about his dreams. He seemed to have lost the power of
intellectual effort, now that the stimulus of pain had been removed. He
was not bored; he had no desire for conversation or distraction. Merely
to be alone, not to be beaten or questioned, to have enough to eat, and
to be clean all over, was completely satisfying.

By degrees he came to spend less time in sleep, but he still felt no
impulse to get off the bed. All he cared for was to lie quiet and feel
the strength gathering in his body. He would finger himself here and
there, trying to make sure that it was not an illusion that his muscles
were growing rounder and his skin tauter. Finally it was established
beyond a doubt that he was growing fatter; his thighs were now
definitely thicker than his knees. After that, reluctantly at first, he
began exercising himself regularly. In a little while he could walk
three kilometers, measured by pacing the cell, and his bowed shoulders
were growing straighter. He attempted more elaborate exercises, and was
astonished and humiliated to find what things he could not do. He could
not move out of a walk, he could not hold his stool out at
arm\textquotesingle s length, he could not stand on one leg without
falling over. He squatted down on his heels, and found that with
agonizing pains in thigh and calf he could just lift himself to a
standing position. He lay flat on his belly and tried to lift his weight
by his hands. It was hopeless; he could not raise himself a centimeter.
But after a few more days---a few more mealtimes---even that feat was
accomplished. A time came when he could do it six times running. He
began to grow actually proud of his body, and to cherish an intermittent
belief that his face also was growing back to normal. Only when he
chanced to put his hand on his bald scalp did he remember the seamed,
ruined face that had looked back at him out of the mirror.

His mind grew more active. He sat down on the plank bed, his back
against the wall and the slate on his knees, and set to work
deliberately at the task of re-educating himself.

He had capitulated; that was agreed. In reality, as he saw now, he had
been ready to capitulate long before he had taken the decision. From the
moment when he was inside the Ministry of Love---and yes, even during
those minutes when he and Julia had stood helpless while the iron voice
from the telescreen told them what to do---he had grasped the frivolity,
the shallowness of his attempt to set himself up against the power of
the Party. He knew now that for seven years the Thought Police had
watched him like a beetle under a magnifying glass. There was no
physical act, no word spoken aloud, that they had not noticed, no train
of thought that they had not been able to infer. Even the speck of
whitish dust on the cover of his diary they had carefully replaced. They
had played sound tracks to him, shown him photographs. Some of them were
photographs of Julia and himself. Yes, even... He could not fight
against the Party any longer. Besides, the Party was in the right. It
must be so: how could the immortal, collective brain be mistaken? By
what external standard could you check its judgments? Sanity was
statistical. It was merely a question of learning to think as they
thought. Only---!

The pencil felt thick and awkward in his fingers. He began to write down
the thoughts that came into his head. He wrote first in large clumsy
capitals:

\headline{FREEDOM IS SLAVERY.}

Then almost without a pause he wrote beneath it:

\headline{TWO AND TWO MAKE FIVE.}

But then there came a sort of check. His mind, as though shying away
from something, seemed unable to concentrate. He knew that he knew what
came next, but for the moment he could not recall it. When he did recall
it, it was only by consciously reasoning out what it must be; it did not
come of its own accord. He wrote:

\headline{GOD IS POWER.}

He accepted everything. The past was alterable. The past never had been
altered. Oceania was at war with Eastasia. Oceania had always been at
war with Eastasia. Jones, Aaronson, and Rutherford were guilty of the
crimes they were charged with. He had never seen the photograph that
disproved their guilt. It had never existed; he had invented it. He
remembered remembering contrary things, but those were false memories,
products of self-deception. How easy it all was! Only surrender, and
everything else followed. It was like swimming against a current that
swept you backwards however hard you struggled, and then suddenly
deciding to turn round and go with the current instead of opposing it.
Nothing had changed except your own attitude; the predestined thing
happened in any case. He hardly knew why he had ever rebelled.
Everything was easy, except---!

Anything could be true. The so-called laws of nature were nonsense. The
law of gravity was nonsense. ``If I wished,'' O\textquotesingle Brien had
said, ``I could float off this floor like a soap bubble.'' Winston worked
it out. ``If he \emph{thinks} he floats off the floor, and if I
simultaneously \emph{think} I see him do it, then the thing happens.''
Suddenly, like a lump of submerged wreckage breaking the surface of
water, the thought burst into his mind: ``It doesn\textquotesingle t
really happen. We imagine it. It is hallucination.'' He pushed the
thought under instantly. The fallacy was obvious. It presupposed that
somewhere or other, outside oneself, there was a ``real'' world where
``real'' things happened. But how could there be such a world? What
knowledge have we of anything, save through our own minds? All
happenings are in the mind. Whatever happens in all minds, truly
happens.

He had no difficulty in disposing of the fallacy, and he was in no
danger of succumbing to it. He realized, nevertheless, that it ought
never to have occurred to him. The mind should develop a blind spot
whenever a dangerous thought presented itself. The process should be
automatic, instinctive. \emph{Crimestop}, they called it in Newspeak.

He set to work to exercise himself in crimestop. He presented himself
with propositions---``the Party says the earth is flat,'' ``the Party says
that ice is heavier than water''---and trained himself in not seeing or
not understanding the arguments that contradicted them. It was not easy.
It needed great powers of reasoning and improvisation. The arithmetical
problems raised, for instance, by such a statement as ``two and two make
five'' were beyond his intellectual grasp. It needed also a sort of
athleticism of mind, an ability at one moment to make the most delicate
use of logic and at the next to be unconscious of the crudest logical
errors. Stupidity was as necessary as intelligence, and as difficult to
attain.

All the while, with one part of his mind, he wondered how soon they
would shoot him. ``Everything depends on yourself,''
O\textquotesingle Brien had said; but he knew that there was no
conscious act by which he could bring it nearer. It might be ten minutes
hence, or ten years. They might keep him for years in solitary
confinement; they might send him to a labor camp; they might release him
for a while, as they sometimes did. It was perfectly possible that
before he was shot the whole drama of his arrest and interrogation would
be enacted all over again. The one certain thing was that death never
came at an expected moment. The tradition---the unspoken tradition:
somehow you knew it, though you never heard it said---was that they shot
you from behind, always in the back of the head, without warning, as you
walked down a corridor from cell to cell.

One day---but ``one day'' was not the right expression; just as probably
it was in the middle of the night: once---he fell into a strange,
blissful reverie. He was walking down the corridor, waiting for the
bullet. He knew that it was coming in another moment. Everything was
settled, smoothed out, reconciled. There were no more doubts, no more
arguments, no more pain, no more fear. His body was healthy and strong.
He walked easily, with a joy of movement and with a feeling of walking
in sunlight. He was not any longer in the narrow white corridors of the
Ministry of Love; he was in the enormous sunlit passage, a kilometer
wide, down which he had seemed to walk in the delirium induced by drugs.
He was in the Golden Country, following the foot track across the old
rabbit-cropped pasture. He could feel the short springy turf under his
feet and the gentle sunshine on his face. At the edge of the field were
the elm trees, faintly stirring, and somewhere beyond that was the
stream where the dace lay in the green pools under the willows.

Suddenly he started up with a shock of horror. The sweat broke out on
his backbone. He had heard himself cry aloud:

``Julia! Julia! Julia, my love! Julia!''

For a moment he had had an overwhelming hallucination of her presence.
She had seemed to be not merely with him, but inside him. It was as
though she had got into the texture of his skin. In that moment he had
loved her far more than he had ever done when they were together and
free. Also he knew that somewhere or other she was still alive and
needed his help.

He lay back on the bed and tried to compose himself. What had he done?
How many years had he added to his servitude by that moment of weakness?

In another moment he would hear the tramp of boots outside. They could
not let such an outburst go unpunished. They would know now, if they had
not known before, that he was breaking the agreement he had made with
them. He obeyed the Party, but he still hated the Party. In the old days
he had hidden a heretical mind beneath an appearance of conformity. Now
he had retreated a step further: in the mind he had surrendered, but he
had hoped to keep the inner heart inviolate. He knew that he was in the
wrong, but he preferred to be in the wrong. They would understand
that---O\textquotesingle Brien would understand it. It was all confessed
in that single foolish cry.

He would have to start all over again. It might take years. He ran a
hand over his face, trying to familiarize himself with the new shape.
There were deep furrows in the cheeks, the cheekbones felt sharp, the
nose flattened. Besides, since last seeing himself in the glass he had
been given a complete new set of teeth. It was not easy to preserve
inscrutability when you did not know what your face looked like. In any
case, mere control of the features was not enough. For the first time he
perceived that if you want to keep a secret you must also hide it from
yourself. You must know all the while that it is there, but until it is
needed you must never let it emerge into your consciousness in any shape
that could be given a name. From now onwards he must not only think
right; he must feel right, dream right. And all the while he must keep
his hatred locked up inside him like a ball of matter which was part of
himself and yet unconnected with the rest of him, a kind of cyst.

One day they would decide to shoot him. You could not tell when it would
happen, but a few seconds beforehand it should be possible to guess. It
was always from behind, walking down a corridor. Ten seconds would be
enough. In that time the world inside him could turn over. And then
suddenly, without a word uttered, without a check in his step, without
the changing of a line in his face---suddenly the camouflage would be
down and bang! would go the batteries of his hatred. Hatred would fill
him like an enormous roaring flame. And almost in the same instant bang!
would go the bullet, too late, or too early. They would have blown his
brain to pieces before they could reclaim it. The heretical thought
would be unpunished, unrepented, out of their reach forever. They would
have blown a hole in their own perfection. To die hating them, that was
freedom.

He shut his eyes. It was more difficult than accepting an intellectual
discipline. It was a question of degrading himself, mutilating himself.
He had got to plunge into the filthiest of filth. What was the most
horrible, sickening thing of all? He thought of Big Brother. The
enormous face (because of constantly seeing it on posters he always
thought of it as being a meter wide), with its heavy black mustache and
the eyes that followed you to and fro, seemed to float into his mind of
its own accord. What were his true feelings toward Big Brother?

There was a heavy tramp of boots in the passage. The steel door swung
open with a clang. O\textquotesingle Brien walked into the cell. Behind
him were the waxen-faced officer and the black-uniformed guards.

``Get up,'' said O\textquotesingle Brien. ``Come here.''

Winston stood opposite him. O\textquotesingle Brien took
Winston\textquotesingle s shoulders between his strong hands and looked
at him closely.

``You have had thoughts of deceiving me,'' he said. ``That was stupid.
Stand up straighter. Look me in the face.''

He paused, and went on in a gentler tone:

``You are improving. Intellectually there is very little wrong with you.
It is only emotionally that you have failed to make progress. Tell me,
Winston---and remember, no lies; you know that I am always able to
detect a lie---tell me, what are your true feelings toward Big Brother?''

``I hate him.''

``You hate him. Good. Then the time has come for you to take the last
step. You must love Big Brother. It is not enough to obey him; you must
love him.''

He released Winston with a little push toward the guards.

``Room 101,'' he said.


\section{V}\label{v-2}

At each stage of his imprisonment he had known, or seemed to know,
whereabouts he was in the windowless building. Possibly there were
slight differences in the air pressure. The cells where the guards had
beaten him were below ground level. The room where he had been
interrogated by O\textquotesingle Brien was high up near the roof. This
place was many meters underground, as deep down as it was possible to
go.

It was bigger than most of the cells he had been in. But he hardly
noticed his surroundings. All he noticed was that there were two small
tables straight in front of him, each covered with green baize. One was
only a meter or two from him; the other was further away, near the door.
He was strapped upright in a chair, so tightly that he could move
nothing, not even his head. A sort of pad gripped his head from behind,
forcing him to look straight in front of him.

For a moment he was alone, then the door opened and
O\textquotesingle Brien came in.

``You asked me once,'' said O\textquotesingle Brien, ``what was in Room
101. I told you that you knew the answer already. Everyone knows it. The
thing that is in Room 101 is the worst thing in the world.''

The door opened again. A guard came in, carrying something made of wire,
a box or basket of some kind. He set it down on the further table.
Because of the position in which O\textquotesingle Brien was standing,
Winston could not see what the thing was.

``The worst thing in the world,'' said O\textquotesingle Brien, ``varies
from individual to individual. It may be burial alive, or death by fire,
or by drowning, or by impalement, or fifty other deaths. There are cases
where it is some quite trivial thing, not even fatal.''

He had moved a little to one side, so that Winston had a better view of
the thing on the table. It was an oblong wire cage with a handle on top
for carrying it by. Fixed to the front of it was something that looked
like a fencing mask, with the concave side outwards. Although it was
three or four meters away from him, he could see that the cage was
divided lengthways into two compartments, and that there was some kind
of creature in each. They were rats.

``In your case,'' said O\textquotesingle Brien, ``the worst thing in the
world happens to be rats.''

A sort of premonitory tremor, a fear of he was not certain what, had
passed through Winston as soon as he caught his first glimpse of the
cage. But at this moment the meaning of the masklike attachment in front
of it suddenly sank into him. His bowels seemed to turn to water.

``You can\textquotesingle t do that!'' he cried out in a high cracked
voice. ``You couldn\textquotesingle t, you couldn\textquotesingle t!
It\textquotesingle s impossible.''

``Do you remember,'' said O\textquotesingle Brien, ``the moment of panic
that used to occur in your dreams? There was a wall of blackness in
front of you, and a roaring sound in your ears. There was something
terrible on the other side of the wall. You knew that you knew what it
was, but you dared not drag it into the open. It was the rats that were
on the other side of the wall.''

``O\textquotesingle Brien!'' said Winston, making an effort to control his
voice. ``You know this is not necessary. What is it that you want me to
do?''

O\textquotesingle Brien made no direct answer. When he spoke it was in
the schoolmasterish manner that he sometimes affected. He looked
thoughtfully into the distance, as though he were addressing an audience
somewhere behind Winston\textquotesingle s back.

``By itself,'' he said, ``pain is not always enough. There are occasions
when a human being will stand out against pain, even to the point of
death. But for everyone there is something unendurable---something that
cannot be contemplated. Courage and cowardice are not involved. If you
are falling from a height it is not cowardly to clutch at a rope. If you
have come up from deep water it is not cowardly to fill your lungs with
air. It is merely an instinct which cannot be disobeyed. It is the same
with the rats. For you, they are unendurable. They are a form of
pressure that you cannot withstand, even if you wished to. You will do
what is required of you.''

``But what is it, what is it? How can I do it if I don\textquotesingle t
know what it is?''

O\textquotesingle Brien picked up the cage and brought it across to the
nearer table. He set it down carefully on the baize cloth. Winston could
hear the blood singing in his ears. He had the feeling of sitting in
utter loneliness. He was in the middle of a great empty plain, a flat
desert drenched with sunlight, across which all sounds came to him out
of immense distances. Yet the cage with the rats was not two meters away
from him. They were enormous rats. They were at the age when a
rat\textquotesingle s muzzle grows blunt and fierce and his fur brown
instead of gray.

``The rat,'' said O\textquotesingle Brien, still addressing his invisible
audience, ``although a rodent, is carnivorous. You are aware of that. You
will have heard of the things that happen in the poor quarters of this
town. In some streets a woman dare not leave her baby alone in the
house, even for five minutes. The rats are certain to attack it. Within
quite a small time they will strip it to the bones. They also attack
sick or dying people. They show astonishing intelligence in knowing when
a human being is helpless.''

There was an outburst of squeals from the cage. It seemed to reach
Winston from far away. The rats were fighting; they were trying to get
at each other through the partition. He heard also a deep groan of
despair. That, too, seemed to come from outside himself.

O\textquotesingle Brien picked up the cage, and, as he did so, pressed
something in it. There was a sharp click. Winston made a frantic effort
to tear himself loose from the chair. It was hopeless: every part of
him, even his head, was held immovably. O\textquotesingle Brien moved
the cage nearer. It was less than a meter from Winston\textquotesingle s
face.

``I have pressed the first lever,'' said O\textquotesingle Brien. ``You
understand the construction of this cage. The mask will fit over your
head, leaving no exit. When I press this other lever, the door of the
cage will slide up. These starving brutes will shoot out of it like
bullets. Have you ever seen a rat leap through the air? They will leap
onto your face and bore straight into it. Sometimes they attack the eyes
first. Sometimes they burrow through the cheeks and devour the tongue.''

The cage was nearer; it was closing in. Winston heard a succession of
shrill cries which appeared to be occurring in the air above his head.
But he fought furiously against his panic. To think, to think, even with
a split second left---to think was the only hope. Suddenly the foul
musty odor of the brutes struck his nostrils. There was a violent
convulsion of nausea inside him, and he almost lost consciousness.
Everything had gone black. For an instant he was insane, a screaming
animal. Yet he came out of the blackness clutching an idea. There was
one and only one way to save himself. He must interpose another human
being, the \emph{body} of another human being, between himself and the
rats.

The circle of the mask was large enough now to shut out the vision of
anything else. The wire door was a couple of hand-spans from his face.
The rats knew what was coming now. One of them was leaping up and down;
the other, an old scaly grandfather of the sewers, stood up, with his
pink hands against the bars, and fiercely snuffed the air. Winston could
see the whiskers and the yellow teeth. Again the black panic took hold
of him. He was blind, helpless, mindless.

``It was a common punishment in Imperial China,'' said
O\textquotesingle Brien as didactically as ever.

The mask was closing on his face. The wire brushed his cheek. And
then---no, it was not relief, only hope, a tiny fragment of hope. Too
late, perhaps too late. But he had suddenly understood that in the whole
world there was just \emph{one} person to whom he could transfer his
punishment---\emph{one} body that he could thrust between himself and
the rats. And he was shouting frantically, over and over:

``Do it to Julia! Do it to Julia! Not me! Julia! I don\textquotesingle t
care what you do to her. Tear her face off, strip her to the bones. Not
me! Julia! Not me!''

He was falling backwards, into enormous depths, away from the rats. He
was still strapped in the chair, but he had fallen through the floor,
through the walls of the building, through the earth, through the
oceans, through the atmosphere, into outer space, into the gulfs between
the stars---always away, away, away from the rats. He was light-years
distant, but O\textquotesingle Brien was still standing at his side.
There was still the cold touch of a wire against his cheek. But through
the darkness that enveloped him he heard another metallic click, and
knew that the cage door had clicked shut and not open.


\section{VI}\label{vi-2}

The Chestnut Tree was almost empty. A ray of sunlight slanting through a
window fell yellow on dusty tabletops. It was the lonely hour of
fifteen. A tinny music trickled from the telescreens.

Winston sat in his usual corner, gazing into an empty glass. Now and
again he glanced up at a vast face which eyed him from the opposite
wall. \textsc{Big Brother Is Watching You}, the caption said. Unbidden, a
waiter came and filled his glass up with Victory Gin, shaking into it a
few drops from another bottle with a quill through the cork. It was
saccharine flavored with cloves, the speciality of the café.

Winston was listening to the telescreen. At present only music was
coming out of it, but there was a possibility that at any moment there
might be a special bulletin from the Ministry of Peace. The news from
the African front was disquieting in the extreme. On and off he had been
worrying about it all day. A Eurasian army (Oceania was at war with
Eurasia; Oceania had always been at war with Eurasia) was moving
southward at terrifying speed. The mid-day bulletin had not mentioned
any definite area, but it was probable that already the mouth of the
Congo was a battlefield. Brazzaville and Leopoldville were in danger.
One did not have to look at the map to see what it meant. It was not
merely a question of losing Central Africa; for the first time in the
whole war, the territory of Oceania itself was menaced.

A violent emotion, not fear exactly but a sort of undifferentiated
excitement, flared up in him, then faded again. He stopped thinking
about the war. In these days he could never fix his mind on any one
subject for more than a few moments at a time. He picked up his glass
and drained it at a gulp. As always, it made him shudder and even retch
slightly. The stuff was horrible. The cloves and saccharine, themselves
disgusting enough in their sickly way, could not disguise the flat oily
smell; and what was worst of all was that the smell of gin, which dwelt
with him night and day, was inextricably mixed up in his mind with the
smell of those---

He never named them, even in his thoughts, and so far as it was possible
he never visualized them. They were something that he was half aware of,
hovering close to his face, a smell that clung to his nostrils. As the
gin rose in him he belched through purple lips. He had grown fatter
since they released him, and had regained his old color---indeed, more
than regained it. His features had thickened, the skin on nose and
cheekbones was coarsely red, even the bald scalp was too deep a pink. A
waiter, again unbidden, brought the chessboard and the current issue of
the \emph{Times}, with the page turned down at the chess problem. Then,
seeing that Winston\textquotesingle s glass was empty, he brought the
gin bottle and filled it. There was no need to give orders. They knew
his habits. The chessboard was always waiting for him, his corner table
was always reserved; even when the place was full he had it to himself,
since nobody cared to be seen sitting too close to him. He never even
bothered to count his drinks. At irregular intervals they presented him
with a dirty slip of paper which they said was the bill, but he had the
impression that they always undercharged him. It would have made no
difference if it had been the other way about. He had always plenty of
money nowadays. He even had a job, a sinecure, more highly paid than his
old job had been.

The music from the telescreen stopped and a voice took over. Winston
raised his head to listen. No bulletin from the front, however. It was
merely a brief announcement from the Ministry of Plenty. In the
preceding quarter, it appeared, the Tenth Three-Year
Plan\textquotesingle s quota for bootlaces had been overfulfilled by
ninety-eight per cent.

He examined the chess problem and set out the pieces. It was a tricky
ending, involving a couple of knights. ``White to play and mate in two
moves.'' Winston looked up at the portrait of Big Brother. White always
mates, he thought with a sort of cloudy mysticism. Always, without
exception, it is so arranged. In no chess problem since the beginning of
the world has black ever won. Did it not symbolize the eternal,
unvarying triumph of Good over Evil? The huge face gazed back at him,
full of calm power. White always mates.

The voice from the telescreen paused and added in a different and much
graver tone: ``You are warned to stand by for an important announcement
at fifteen-thirty. Fifteen-thirty! This is news of the highest
importance. Take care not to miss it. Fifteen-thirty!'' The tinkling
music struck up again.

Winston\textquotesingle s heart stirred. That was the bulletin from the
front; instinct told him that it was bad news that was coming. All day,
with little spurts of excitement, the thought of a smashing defeat in
Africa had been in and out of his mind. He seemed actually to see the
Eurasian army swarming across the never-broken frontier and pouring down
into the tip of Africa like a column of ants. Why had it not been
possible to outflank them in some way? The outline of the West African
coast stood out vividly in his mind. He picked up the white knight and
moved it across the board. \emph{There} was the proper spot. Even while
he saw the black horde racing southward he saw another force,
mysteriously assembled, suddenly planted in their rear, cutting their
communications by land and sea. He felt that by willing it he was
bringing that other force into existence. But it was necessary to act
quickly. If they could get control of the whole of Africa, if they had
airfields and submarine bases at the Cape, it would cut Oceania in two.
It might mean anything: defeat, breakdown, the redivision of the world,
the destruction of the Party! He drew a deep breath. An extraordinary
medley of feelings---but it was not a medley, exactly; rather it was
successive layers of feeling, in which one could not say which layer was
undermost---struggled inside him.

The spasm passed. He put the white knight back in its place, but for the
moment he could not settle down to serious study of the chess problem.
His thoughts wandered again. Almost unconsciously he traced with his
finger in the dust on the table:

\headline{2 + 2 = 5.}

``They can\textquotesingle t get inside you,'' she had said. But they
could get inside you. ``What happens to you here is \emph{forever},''
O\textquotesingle Brien had said. That was a true word. There were
things, your own acts, from which you could not recover. Something was
killed in your breast; burnt out, cauterized out.

He had seen her; he had even spoken to her. There was no danger in it.
He knew as though instinctively that they now took almost no interest in
his doings. He could have arranged to meet her a second time if either
of them had wanted to. Actually it was by chance that they had met. It
was in the Park, on a vile, biting day in March, when the earth was like
iron and all the grass seemed dead and there was not a bud anywhere
except a few crocuses which had pushed themselves up to be dismembered
by the wind. He was hurrying along with frozen hands and watering eyes
when he saw her not ten meters away from him. It struck him at once that
she had changed in some ill-defined way. They almost passed one another
without a sign; then he turned and followed her, not very eagerly. He
knew that there was no danger, nobody would take any interest in them.
She did not speak. She walked obliquely away across the grass as though
trying to get rid of him, then seemed to resign herself to having him at
her side. Presently they were in among a clump of ragged leafless
shrubs, useless either for concealment or as protection from the wind.
They halted. It was vilely cold. The wind whistled through the twigs and
fretted the occasional, dirty-looking crocuses. He put his arm round her
waist.

There was no telescreen, but there must be hidden microphones; besides,
they could be seen. It did not matter, nothing mattered. They could have
lain down on the ground and done \emph{that} if they had wanted to. His
flesh froze with horror at the thought of it. She made no response
whatever to the clasp of his arm; she did not even try to disengage
herself. He knew now what had changed in her. Her face was sallower, and
there was a long scar, partly hidden by the hair, across her forehead
and temple; but that was not the change. It was that her waist had grown
thicker and, in a surprising way, had stiffened. He remembered how once,
after the explosion of a rocket bomb, he had helped to drag a corpse out
of some ruins, and had been astonished not only by the incredible weight
of the thing, but by its rigidity and awkwardness to handle, which made
it seem more like stone than flesh. Her body felt like that. It occurred
to him that the texture of her skin would be quite different from what
it had once been.

He did not attempt to kiss her, nor did they speak. As they walked back
across the grass she looked directly at him for the first time. It was
only a momentary glance, full of contempt and dislike. He wondered
whether it was a dislike that came purely out of the past or whether it
was inspired also by his bloated face and the water that the wind kept
squeezing from his eyes. They sat down on two iron chairs, side by side
but not too close together. He saw that she was about to speak. She
moved her clumsy shoe a few centimeters and deliberately crushed a twig.
Her feet seemed to have grown broader, he noticed.

``I betrayed you,'' she said baldly.

``I betrayed you,'' he said.

She gave him another quick look of dislike.

``Sometimes,'' she said, ``they threaten you with something---something you
can\textquotesingle t stand up to, can\textquotesingle t even think
about. And then you say, `Don\textquotesingle t do it to
me, do it to somebody else, do it to so-and-so.' And
perhaps you might pretend, afterwards, that it was only a trick and that
you just said it to make them stop and didn\textquotesingle t really
mean it. But that isn\textquotesingle t true. At the time when it
happens you do mean it. You think there\textquotesingle s no other way
of saving yourself, and you\textquotesingle re quite ready to save
yourself that way. You \emph{want} it to happen to the other person. You
don\textquotesingle t give a damn what they suffer. All you care about
is yourself.''

``All you care about is yourself,'' he echoed.

``And after that, you don\textquotesingle t feel the same toward the
other person any longer.''

``No,'' he said, ``you don\textquotesingle t feel the same.''

There did not seem to be anything more to say. The wind plastered their
thin overalls against their bodies. Almost at once it became
embarrassing to sit there in silence; besides, it was too cold to keep
still. She said something about catching her Tube and stood up to go.

``We must meet again,'' he said.

``Yes,'' she said, ``we must meet again.''

He followed irresolutely for a little distance, half a pace behind her.
They did not speak again. She did not actually try to shake him off, but
walked at just such a speed as to prevent his keeping abreast of her. He
had made up his mind that he would accompany her as far as the Tube
station, but suddenly this process of trailing along in the cold seemed
pointless and unbearable. He was overwhelmed by a desire not so much to
get away from Julia as to get back to the Chestnut Tree Café, which had
never seemed so attractive as at this moment. He had a nostalgic vision
of his corner table, with the newspaper and the chessboard and the
ever-flowing gin. Above all, it would be warm in there. The next moment,
not altogether by accident, he allowed himself to become separated from
her by a small knot of people. He made a half-hearted attempt to catch
up, then slowed down, turned and made off in the opposite direction.
When he had gone fifty meters he looked back. The street was not
crowded, but already he could not distinguish her. Any one of a dozen
hurrying figures might have been hers. Perhaps her thickened, stiffened
body was no longer recognizable from behind.

``At the time when it happens,'' she had said, ``you do mean it.'' He had
meant it. He had not merely said it, he had wished it. He had wished
that she and not he should be delivered over to the------

Something changed in the music that trickled from the telescreen. A
cracked and jeering note, a yellow note, came into it. And
then---perhaps it was not happening, perhaps it was only a memory taking
on the semblance of sound---a voice was singing:

\begin{quotation}
 \noindent ``Under the spreading chestnut tree\\
  I sold you and you sold me---''
\end{quotation}

The tears welled up in his eyes. A passing waiter noticed that his glass
was empty and came back with the gin bottle.

He took up his glass and sniffed at it. The stuff grew not less but more
horrible with every mouthful he drank. But it had become the element he
swam in. It was his life, his death, and his resurrection. It was gin
that sank him into stupor every night, and gin that revived him every
morning. When he woke, seldom before eleven hundred, with gummed-up
eyelids and fiery mouth and a back that seemed to be broken, it would
have been impossible even to rise from the horizontal if it had not been
for the bottle and teacup placed beside the bed overnight. Through the
mid-day hours he sat with glazed face, the bottle handy, listening to
the telescreen. From fifteen to closing time he was a fixture in the
Chestnut Tree. No one cared what he did any longer, no whistle woke him,
no telescreen admonished him. Occasionally, perhaps twice a week, he
went to a dusty, forgotten-looking office in the Ministry of Truth and
did a little work, or what was called work. He had been appointed to a
sub-committee of a sub-committee which had sprouted from one of the
innumerable committees dealing with minor difficulties that arose in the
compilation of the Eleventh Edition of the Newspeak dictionary. They
were engaged in producing something called an Interim Report, but what
it was that they were reporting on he had never definitely found out. It
was something to do with the question of whether commas should be placed
inside brackets, or outside. There were four others on the committee,
all of them persons similar to himself. There were days when they
assembled and then promptly dispersed again, frankly admitting to one
another that there was not really anything to be done. But there were
other days when they settled down to their work almost eagerly, making a
tremendous show of entering up their minutes and drafting long memoranda
which were never finished---when the argument as to what they were
supposedly arguing about grew extraordinarily involved and abstruse,
with subtle hagglings over definitions, enormous digressions,
quarrels---threats, even, to appeal to higher authority. And then
suddenly the life would go out of them and they would sit round the
table looking at one another with extinct eyes, like ghosts fading at
cock-crow.

The telescreen was silent for a moment. Winston raised his head again.
The bulletin! But no, they were merely changing the music. He had the
map of Africa behind his eyelids. The movement of the armies was a
diagram: a black arrow tearing vertically southward, and a white arrow
tearing horizontally eastward, across the tail of the first. As though
for reassurance he looked up at the imperturbable face in the portrait.
Was it conceivable that the second arrow did not even exist?

His interest flagged again. He drank another mouthful of gin, picked up
the white knight, and made a tentative move. Check. But it was evidently
not the right move, because---

Uncalled, a memory floated into his mind. He saw a candle-lit room with
a vast white-counterpaned bed, and himself, a boy of nine or ten,
sitting on the floor, shaking a dice box and laughing excitedly. His
mother was sitting opposite him and also laughing.

It must have been about a month before she disappeared. It was a moment
of reconciliation, when the nagging hunger in his belly was forgotten
and his earlier affection for her had temporarily revived. He remembered
the day well, a pelting, drenching day when the water streamed down the
window pane and the light indoors was too dull to read by. The boredom
of the two children in the dark, cramped bedroom became unbearable.
Winston whined and grizzled, made futile demands for food, fretted about
the room, pulling everything out of place and kicking the wainscoting
until the neighbors banged on the wall, while the younger child wailed
intermittently. In the end his mother had said, ``Now be good, and
I\textquotesingle ll buy you a toy. A lovely
toy---you\textquotesingle ll love it''; and then she had gone out in the
rain, to a little general shop which was still sporadically open near
by, and come back with a cardboard box containing an outfit of Snakes
and Ladders. He could still remember the smell of the damp cardboard. It
was a miserable outfit. The board was cracked and the tiny wooden dice
were so ill-cut that they would hardly lie on their sides. Winston
looked at the thing sulkily and without interest. But then his mother
lit a piece of candle and they sat down on the floor to play. Soon he
was wildly excited and shouting with laughter as the tiddlywinks climbed
hopefully up the ladders and then came slithering down the snakes again,
almost back to the starting point. They played eight games, winning four
each. His tiny sister, too young to understand what the game was about,
had sat propped up against a bolster, laughing because the others were
laughing. For a whole afternoon they had all been happy together, as in
his earlier childhood.

He pushed the picture out of his mind. It was a false memory. He was
troubled by false memories occasionally. They did not matter so long as
one knew them for what they were. Some things had happened, others had
not happened. He turned back to the chessboard and picked up the white
knight again. Almost in the same instant it dropped onto the board with
a clatter. He had started as though a pin had run into him.

A shrill trumpet call had pierced the air. It was the bulletin! Victory!
It always meant victory when a trumpet call preceded the news. A sort of
electric thrill ran through the café. Even the waiters had started and
pricked up their ears.

The trumpet call had let loose an enormous volume of noise. Already an
excited voice was gabbling from the telescreen, but even as it started
it was almost drowned by a roar of cheering from outside. The news had
run round the streets like magic. He could hear just enough of what was
issuing from the telescreen to realize that it had all happened as he
had foreseen: a vast seaborne armada secretly assembled, a sudden blow
in the enemy\textquotesingle s rear, the white arrow tearing across the
tail of the black. Fragments of triumphant phrases pushed themselves
through the din: ``Vast strategic maneuver---perfect
co-ordination---utter rout---half a million prisoners---complete
demoralization---control of the whole of Africa---bring the war within
measurable distance of its end---victory---greatest victory in human
history---victory, victory, victory!''

Under the table Winston\textquotesingle s feet made convulsive
movements. He had not stirred from his seat, but in his mind he was
running, swiftly running, he was with the crowds outside, cheering
himself deaf. He looked up again at the portrait of Big Brother. The
colossus that bestrode the world! The rock against which the hordes of
Asia dashed themselves in vain! He thought how ten minutes ago---yes,
only ten minutes---there had still been equivocation in his heart as he
wondered whether the news from the front would be of victory or defeat.
Ah, it was more than a Eurasian army that had perished! Much had changed
in him since that first day in the Ministry of Love, but the final,
indispensable, healing change had never happened, until this moment.

The voice from the telescreen was still pouring forth its tale of
prisoners and booty and slaughter, but the shouting outside had died
down a little. The waiters were turning back to their work. One of them
approached with the gin bottle. Winston, sitting in a blissful dream,
paid no attention as his glass was filled up. He was not running or
cheering any longer. He was back in the Ministry of Love, with
everything forgiven, his soul white as snow. He was in the public dock,
confessing everything, implicating everybody. He was walking down the
white-tiled corridor, with the feeling of walking in sunlight, and an
armed guard at his back. The long-hoped-for bullet was entering his
brain.

He gazed up at the enormous face. Forty years it had taken him to learn
what kind of smile was hidden beneath the dark mustache. O cruel,
needless misunderstanding! O stubborn, self-willed exile from the loving
breast! Two gin-scented tears trickled down the sides of his nose. But
it was all right, everything was all right, the struggle was finished.
He had won the victory over himself. He loved Big Brother.

\begin{center}
\textbf{THE END}
\end{center}


\clearpage
\part{APPENDIX}\label{appendix}

\section{THE PRINCIPLES OF NEWSPEAK}\label{the-principles-of-newspeak}

Newspeak was the official language of Oceania and had been devised to
meet the ideological needs of Ingsoc, or English Socialism. In the year
1984 there was not as yet anyone who used Newspeak as his sole means of
communication, either in speech or writing. The leading articles in the
\emph{Times} were written in it, but this was a tour de force which
could only be carried out by a specialist. It was expected that Newspeak
would have finally superseded Oldspeak (or Standard English, as we
should call it) by about the year 2050. Meanwhile it gained ground
steadily, all Party members tending to use Newspeak words and
grammatical constructions more and more in their everyday speech. The
version in use in 1984, and embodied in the Ninth and Tenth Editions of
the Newspeak dictionary, was a provisional one, and contained many
superfluous words and archaic formations which were due to be suppressed
later. It is with the final, perfected version, as embodied in the
Eleventh Edition of the dictionary, that we are concerned here.

The purpose of Newspeak was not only to provide a medium of expression
for the world-view and mental habits proper to the devotees of Ingsoc,
but to make all other modes of thought impossible. It was intended that
when Newspeak had been adopted once and for all and Oldspeak forgotten,
a heretical thought---that is, a thought diverging from the principles
of Ingsoc---should be literally unthinkable, at least so far as thought
is dependent on words. Its vocabulary was so constructed as to give
exact and often very subtle expression to every meaning that a Party
member could properly wish to express, while excluding all other
meanings and also the possibility of arriving at them by indirect
methods. This was done partly by the invention of new words, but chiefly
by eliminating undesirable words and by stripping such words as remained
of unorthodox meanings, and so far as possible of all secondary meanings
whatever. To give a single example. The word \emph{free} still existed
in Newspeak, but it could only be used in such statements as ``This dog
is free from lice'' or ``This field is free from weeds.'' It could not be
used in its old sense of ``politically free'' or ``intellectually free,''
since political and intellectual freedom no longer existed even as
concepts, and were therefore of necessity nameless. Quite apart from the
suppression of definitely heretical words, reduction of vocabulary was
regarded as an end in itself, and no word that could be dispensed with
was allowed to survive. Newspeak was designed not to extend but to
\emph{diminish} the range of thought, and this purpose was indirectly
assisted by cutting the choice of words down to a minimum.

Newspeak was founded on the English language as we now know it, though
many Newspeak sentences, even when not containing newly created words,
would be barely intelligible to an English-speaker of our own day.
Newspeak words were divided into three distinct classes, known as the A
vocabulary, the B vocabulary (also called compound words), and the C
vocabulary. It will be simpler to discuss each class separately, but the
grammatical peculiarities of the language can be dealt with in the
section devoted to the A vocabulary, since the same rules held good for
all three categories.

\sectionbreak

\emph{The A vocabulary.} The A vocabulary consisted of the words needed
for the business of everyday life---for such things as eating, drinking,
working, putting on one\textquotesingle s clothes, going up and down
stairs, riding in vehicles, gardening, cooking, and the like. It was
composed almost entirely of words that we already possess---words like
\emph{hit}, \emph{run}, \emph{dog}, \emph{tree}, \emph{sugar},
\emph{house}, \emph{field}---but in comparison with the present-day
English vocabulary, their number was extremely small, while their
meanings were far more rigidly defined. All ambiguities and shades of
meaning had been purged out of them. So far as it could be achieved, a
Newspeak word of this class was simply a staccato sound expressing
\emph{one} clearly understood concept. It would have been quite
impossible to use the A vocabulary for literary purposes or for
political or philosophical discussion. It was intended only to express
simple, purposive thoughts, usually involving concrete objects or
physical actions.

The grammar of Newspeak had two outstanding peculiarities. The first of
these was an almost complete interchangeability between different parts
of speech. Any word in the language (in principle this applied even to
very abstract words such as \emph{if} or \emph{when}) could be used
either as verb, noun, adjective, or adverb. Between the verb and the
noun form, when they were of the same root, there was never any
variation, this rule of itself involving the destruction of many archaic
forms. The word \emph{thought}, for example, did not exist in Newspeak.
Its place was taken by \emph{think}, which did duty for both noun and
verb. No etymological principle was followed here; in some cases it was
the original noun that was chosen for retention, in other cases the
verb. Even where a noun and verb of kindred meaning were not
etymologically connected, one or other of them was frequently
suppressed. There was, for example, no such word as \emph{cut}, its
meaning being sufficiently covered by the noun-verb \emph{knife}.
Adjectives were formed by adding the suffix \emph{-ful} to the
noun-verb, and adverbs by adding \emph{-wise}. Thus, for example,
\emph{speedful} meant ``rapid'' and \emph{speedwise} meant ``quickly.''
Certain of our present-day adjectives, such as \emph{good},
\emph{strong}, \emph{big}, \emph{black}, \emph{soft}, were retained, but
their total number was very small. There was little need for them, since
almost any adjectival meaning could be arrived at by adding \emph{-ful}
to a noun-verb. None of the now-existing adverbs was retained, except
for a very few already ending in \emph{-wise}; the \emph{-wise}
termination was invariable. The word \emph{well}, for example, was
replaced by \emph{goodwise}.

In addition, any word---this again applied in principle to every word in
the language---could be negatived by adding the affix \emph{un-}, or
could be strengthened by the affix \emph{plus-}, or, for still greater
emphasis, \emph{doubleplus-}. Thus, for example, \emph{uncold} meant
``warm,'' while \emph{pluscold} and \emph{doublepluscold} meant,
respectively, ``very cold'' and ``superlatively cold.'' It was also
possible, as in present-day English, to modify the meaning of almost any
word by prepositional affixes such as \emph{ante-}, \emph{post-},
\emph{up-}, \emph{down-}, etc. By such methods it was found possible to
bring about an enormous diminution of vocabulary. Given, for instance,
the word \emph{good}, there was no need for such a word as \emph{bad},
since the required meaning was equally well---indeed, better---expressed
by \emph{ungood}. All that was necessary, in any case where two words
formed a natural pair of opposites, was to decide which of them to
suppress. \emph{Dark}, for example, could be replaced by \emph{unlight},
or \emph{light} by \emph{undark}, according to preference.

The second distinguishing mark of Newspeak grammar was its regularity.
Subject to a few exceptions which are mentioned below, all inflections
followed the same rules. Thus, in all verbs the preterite and the past
participle were the same and ended in \emph{-ed}. The preterite of
\emph{steal} was \emph{stealed}, the preterite of \emph{think} was
\emph{thinked}, and so on throughout the language, all such forms as
\emph{swam}, \emph{gave}, \emph{brought}, \emph{spoke}, \emph{taken},
etc., being abolished. All plurals were made by adding \emph{-s} or
\emph{-es} as the case might be. The plurals of \emph{man}, \emph{ox},
\emph{life} were \emph{mans}, \emph{oxes}, \emph{lifes}. Comparison of
adjectives was invariably made by adding \emph{-er}, \emph{-est}
(\emph{good}, \emph{gooder}, \emph{goodest}), irregular forms and the
\emph{more}, \emph{most} formation being suppressed.

The only classes of words that were still allowed to inflect irregularly
were the pronouns, the relatives, the demonstrative adjectives, and the
auxiliary verbs. All of these followed their ancient usage, except that
\emph{whom} had been scrapped as unnecessary, and the \emph{shall},
\emph{should} tenses had been dropped, all their uses being covered by
\emph{will} and \emph{would}. There were also certain irregularities in
word-formation arising out of the need for rapid and easy speech. A word
which was difficult to utter, or was liable to be incorrectly heard, was
held to be ipso facto a bad word; occasionally therefore, for the sake
of euphony, extra letters were inserted into a word or an archaic
formation was retained. But this need made itself felt chiefly in
connection with the B vocabulary. \emph{Why} so great an importance was
attached to ease of pronunciation will be made clear later in this
essay.

\sectionbreak

\emph{The B vocabulary.} The B vocabulary consisted of words which had
been deliberately constructed for political purposes: words, that is to
say, which not only had in every case a political implication, but were
intended to impose a desirable mental attitude upon the person using
them. Without a full understanding of the principles of Ingsoc it was
difficult to use these words correctly. In some cases they could be
translated into Oldspeak, or even into words taken from the A
vocabulary, but this usually demanded a long paraphrase and always
involved the loss of certain overtones. The B words were a sort of
verbal short-hand, often packing whole ranges of ideas into a few
syllables, and at the same time more accurate and forcible than ordinary
language.

The B words were in all cases compound words.\footnote{Compound words, such
  as \emph{speakwrite}, were of course to be found in the A vocabulary, but
  these were merely convenient abbreviations and had no special ideological
  color.} They consisted of two or more words, or portions of words, welded
together in an easily pronounceable form. The resulting amalgam was always a
noun-verb, and inflected according to the ordinary rules. To take a single
example: the word \emph{goodthink}, meaning, very roughly, ``orthodoxy,''
or, if one chose to regard it as a verb, ``to think in an orthodox manner.''
This inflected as follows: noun-verb, \emph{goodthink}; past tense and past
participle, \emph{goodthinked}; present participle, \emph{goodthinking};
adjective, \emph{goodthinkful}; adverb, \emph{goodthinkwise}; verbal noun,
\emph{goodthinker}.

The B words were not constructed on any etymological plan. The words of
which they were made up could be any parts of speech, and could be
placed in any order and mutilated in any way which made them easy to
pronounce while indicating their derivation. In the word
\emph{crimethink} (thoughtcrime), for instance, the \emph{think} came
second, whereas in \emph{thinkpol} (Thought Police) it came first, and
in the latter word \emph{police} had lost its second syllable. Because
of the greater difficulty in securing euphony, irregular formations were
commoner in the B vocabulary than in the A vocabulary. For example, the
adjectival forms of \emph{Minitrue}, \emph{Minipax}, and \emph{Miniluv}
were, respectively, \emph{Minitruthful}, \emph{Minipeaceful}, and
\emph{Minilovely}, simply because \emph{-trueful}, \emph{-paxful}, and
\emph{-loveful} were slightly awkward to pronounce. In principle,
however, all B words could inflect, and all inflected in exactly the
same way.

Some of the B words had highly subtilized meanings, barely intelligible
to anyone who had not mastered the language as a whole. Consider, for
example, such a typical sentence from a \emph{Times} leading article as
\emph{Oldthinkers unbellyfeel Ingsoc}. The shortest rendering that one
could make of this in Oldspeak would be: ``Those whose ideas were formed
before the Revolution cannot have a full emotional understanding of the
principles of English Socialism.'' But this is not an adequate
translation. To begin with, in order to grasp the full meaning of the
Newspeak sentence quoted above, one would have to have a clear idea of
what is meant by \emph{Ingsoc}. And, in addition, only a person
thoroughly grounded in Ingsoc could appreciate the full force of the
word \emph{bellyfeel}, which implied a blind, enthusiastic acceptance
difficult to imagine today; or of the word \emph{oldthink}, which was
inextricably mixed up with the idea of wickedness and decadence. But the
special function of certain Newspeak words, of which \emph{oldthink} was
one, was not so much to express meanings as to destroy them. These
words, necessarily few in number, had had their meanings extended until
they contained within themselves whole batteries of words which, as they
were sufficiently covered by a single comprehensive term, could now be
scrapped and forgotten. The greatest difficulty facing the compilers of
the Newspeak dictionary was not to invent new words, but, having
invented them, to make sure what they meant: to make sure, that is to
say, what ranges of words they canceled by their existence.

As we have already seen in the case of the word \emph{free}, words which
had once borne a heretical meaning were sometimes retained for the sake
of convenience, but only with the undesirable meanings purged out of
them. Countless other words such as \emph{honor}, \emph{justice},
\emph{morality}, \emph{internationalism}, \emph{democracy},
\emph{science}, and \emph{religion} had simply ceased to exist. A few
blanket words covered them, and, in covering them, abolished them. All
words groupings themselves round the concepts of liberty and equality,
for instance, were contained in the single word \emph{crimethink}, while
all words grouping themselves round the concepts of objectivity and
rationalism were contained in the single word \emph{oldthink}. Greater
precision would have been dangerous. What was required in a Party member
was an outlook similar to that of the ancient Hebrew who knew, without
knowing much else, that all nations other than his own worshiped ``false
gods.'' He did not need to know that these gods were called Baal, Osiris,
Moloch, Ashtaroth, and the like; probably the less he knew about them
the better for his orthodoxy. He knew Jehovah and the commandments of
Jehovah; he knew, therefore, that all gods with other names or other
attributes were false gods. In somewhat the same way, the Party member
knew what constituted right conduct, and in exceedingly vague,
generalized terms he knew what kinds of departure from it were possible.
His sexual life, for example, was entirely regulated by the two Newspeak
words \emph{sexcrime} (sexual immorality) and \emph{goodsex} (chastity).
\emph{Sexcrime} covered all sexual misdeeds whatever. It covered
fornication, adultery, homosexuality, and other perversions, and, in
addition, normal intercourse practiced for its own sake. There was no
need to enumerate them separately, since they were all equally culpable,
and, in principle, all punishable by death. In the C vocabulary, which
consisted of scientific and technical words, it might be necessary to
give specialized names to certain sexual aberrations, but the ordinary
citizen had no need of them. He knew what was meant by
\emph{goodsex}---that is to say, normal intercourse between man and
wife, for the sole purpose of begetting children, and without physical
pleasure on the part of the woman; all else was \emph{sexcrime}. In
Newspeak it was seldom possible to follow a heretical thought further
than the perception that it \emph{was} heretical; beyond that point the
necessary words were nonexistent.

No word in the B vocabulary was ideologically neutral. A great many were
euphemisms. Such words, for instance, as \emph{joycamp} (forced-labor
camp) or \emph{Minipax} (Ministry of Peace, i.e., Ministry of War) meant
almost the exact opposite of what they appeared to mean. Some words, on
the other hand, displayed a frank and contemptuous understanding of the
real nature of Oceanic society. An example was \emph{prolefeed}, meaning
the rubbishy entertainment and spurious news which the Party handed out
to the masses. Other words, again, were ambivalent, having the
connotation ``good'' when applied to the Party and ``bad'' when applied to
its enemies. But in addition there were great numbers of words which at
first sight appeared to be mere abbreviations and which derived their
ideological color not from their meaning but from their structure.

So far as it could be contrived, everything that had or might have
political significance of any kind was fitted into the B vocabulary. The
name of every organization, or body of people, or doctrine, or country,
or institution, or public building, was invariably cut down into the
familiar shape; that is, a single easily pronounced word with the
smallest number of syllables that would preserve the original
derivation. In the Ministry of Truth, for example, the Records
Department, in which Winston Smith worked, was called \emph{Recdep}, the
Fiction Department was called \emph{Ficdep}, the Teleprograms Department
was called \emph{Teledep}, and so on. This was not done solely with the
object of saving time. Even in the early decades of the twentieth
century, telescoped words and phrases had been one of the characteristic
features of political language; and it had been noticed that the
tendency to use abbreviations of this kind was most marked in
totalitarian countries and totalitarian organizations. Examples were
such words as \emph{Nazi}, \emph{Gestapo}, \emph{Comintern},
\emph{Inprecorr}, \emph{Agitprop}. In the beginning the practice had
been adopted as it were instinctively, but in Newspeak it was used with
a conscious purpose. It was perceived that in thus abbreviating a name
one narrowed and subtly altered its meaning, by cutting out most of the
associations that would otherwise cling to it. The words \emph{Communist
International}, for instance, call up a composite picture of universal
human brotherhood, red flags, barricades, Karl Marx, and the Paris
Commune. The word Comintern, on the other hand, suggests merely a
tightly knit organization and a well-defined body of doctrine. It refers
to something almost as easily recognized, and as limited in purpose, as
a chair or a table. \emph{Comintern} is a word that can be uttered
almost without taking thought, whereas \emph{Communist International} is
a phrase over which one is obliged to linger at least momentarily. In
the same way, the associations called up by a word like \emph{Minitrue}
are fewer and more controllable than those called up by \emph{Ministry
of Truth}. This accounted not only for the habit of abbreviating
whenever possible, but also for the almost exaggerated care that was
taken to make every word easily pronounceable.

In Newspeak, euphony outweighed every consideration other than
exactitude of meaning. Regularity of grammar was always sacrificed to it
when it seemed necessary. And rightly so, since what was required, above
all for political purposes, were short clipped words of unmistakable
meaning which could be uttered rapidly and which roused the minimum of
echoes in the speaker\textquotesingle s mind. The words of the B
vocabulary even gained in force from the fact that nearly all of them
were very much alike. Almost invariably these words---\emph{goodthink},
\emph{Minipax}, \emph{prolefeed}, \emph{sexcrime}, \emph{joycamp},
\emph{Ingsoc}, \emph{bellyfeel}, \emph{thinkpol}, and countless
others---were words of two or three syllables, with the stress
distributed equally between the first syllable and the last. The use of
them encouraged a gabbling style of speech, at once staccato and
monotonous. And this was exactly what was aimed at. The intention was to
make speech, and especially speech on any subject not ideologically
neutral, as nearly as possible independent of consciousness. For the
purposes of everyday life it was no doubt necessary, or sometimes
necessary, to reflect before speaking, but a Party member called upon to
make a political or ethical judgment should be able to spray forth the
correct opinions as automatically as a machine gun spraying forth
bullets. His training fitted him to do this, the language gave him an
almost foolproof instrument, and the texture of the words, with their
harsh sound and a certain willful ugliness which was in accord with the
spirit of Ingsoc, assisted the process still further.

So did the fact of having very few words to choose from. Relative to our
own, the Newspeak vocabulary was tiny, and new ways of reducing it were
constantly being devised. Newspeak, indeed, differed from almost all
other languages in that its vocabulary grew smaller instead of larger
every year. Each reduction was a gain, since the smaller the area of
choice, the smaller the temptation to take thought. Ultimately it was
hoped to make articulate speech issue from the larynx without involving
the higher brain centers at all. This aim was frankly admitted in the
Newspeak word \emph{duckspeak}, meaning ``to quack like a duck.'' Like
various other words in the B vocabulary, \emph{duckspeak} was ambivalent
in meaning. Provided that the opinions which were quacked out were
orthodox ones, it implied nothing but praise, and when the \emph{Times}
referred to one of the orators of the Party as a \emph{doubleplusgood
duckspeaker} it was paying a warm and valued compliment.

\sectionbreak

\emph{The C vocabulary.} The C vocabulary was supplementary to the
others and consisted entirely of scientific and technical terms. These
resembled the scientific terms in use today, and were constructed from
the same roots, but the usual care was taken to define them rigidly and
strip them of undesirable meanings. They followed the same grammatical
rules as the words in the other two vocabularies. Very few of the C
words had any currency either in everyday speech or in political speech.
Any scientific worker or technician could find all the words he needed
in the list devoted to his own speciality, but he seldom had more than a
smattering of the words occurring in the other lists. Only a very few
words were common to all lists, and there was no vocabulary expressing
the function of Science as a habit of mind, or a method of thought,
irrespective of its particular branches. There was, indeed, no word for
``Science,'' any meaning that it could possibly bear being already
sufficiently covered by the word \emph{Ingsoc}.

\sectionbreak

From the foregoing account it will be seen that in Newspeak the
expression of unorthodox opinions, above a very low level, was well-nigh
impossible. It was of course possible to utter heresies of a very crude
kind, a species of blasphemy. It would have been possible, for example,
to say \emph{Big Brother is ungood}. But this statement, which to an
orthodox ear merely conveyed a self-evident absurdity, could not have
been sustained by reasoned argument, because the necessary words were
not available. Ideas inimical to Ingsoc could only be entertained in a
vague wordless form, and could only be named in very broad terms which
lumped together and condemned whole groups of heresies without defining
them in doing so. One could, in fact, only use Newspeak for unorthodox
purposes by illegitimately translating some of the words back into
Oldspeak. For example, \emph{All mans are equal} was a possible Newspeak
sentence, but only in the same sense in which \emph{All men are
redhaired} is a possible Oldspeak sentence. It did not contain a
grammatical error, but it expressed a palpable untruth, i.e., that all
men are of equal size, weight, or strength. The concept of political
equality no longer existed, and this secondary meaning had accordingly
been purged out of the word \emph{equal}. In 1984, when Oldspeak was
still the normal means of communication, the danger theoretically
existed that in using Newspeak words one might remember their original
meanings. In practice it was not difficult for any person well grounded
in \emph{doublethink} to avoid doing this, but within a couple of
generations even the possibility of such a lapse would have vanished. A
person growing up with Newspeak as his sole language would no more know
that \emph{equal} had once had the secondary meaning of ``politically
equal,'' or that \emph{free} had once meant ``intellectually free,'' than,
for instance, a person who had never heard of chess would be aware of
the secondary meanings attaching to \emph{queen} and \emph{rook}. There
would be many crimes and errors which it would be beyond his power to
commit, simply because they were nameless and therefore unimaginable.
And it was to be foreseen that with the passage of time the
distinguishing characteristics of Newspeak would become more and more
pronounced---its words growing fewer and fewer, their meanings more and
more rigid, and the chance of putting them to improper uses always
diminishing.

When Oldspeak had been once and for all superseded, the last link with
the past would have been severed. History had already been rewritten,
but fragments of the literature of the past survived here and there,
imperfectly censored, and so long as one retained one\textquotesingle s
knowledge of Oldspeak it was possible to read them. In the future such
fragments, even if they chanced to survive, would be unintelligible and
untranslatable. It was impossible to translate any passage of Oldspeak
into Newspeak unless it either referred to some technical process or
some very simple everyday action, or was already orthodox
(\emph{goodthinkful} would be the Newspeak expression) in tendency. In
practice this meant that no book written before approximately 1960 could
be translated as a whole. Prerevolutionary literature could only be
subjected to ideological translation---that is, alteration in sense as
well as language. Take for example the well-known passage from the
Declaration of Independence:

\begin{quotation}
We hold these truths to be self-evident, that all men are created
equal, that they are endowed by their Creator with certain inalienable
rights, that among these are life, liberty and the pursuit of happiness.
That to secure these rights, Governments are instituted among men,
deriving their powers from the consent of the governed. That whenever
any form of Government becomes destructive of those ends, it is the
right of the People to alter or abolish it, and to institute new
Government \ldots{}
\end{quotation}

It would have been quite impossible to render this into Newspeak while
keeping to the sense of the original. The nearest one could come to
doing so would be to swallow the whole passage up in the single word
\emph{crimethink}. A full translation could only be an ideological
translation, whereby Jefferson\textquotesingle s words would be changed
into a panegyric on absolute government.

A good deal of the literature of the past was, indeed, already being
transformed in this way. Considerations of prestige made it desirable to
preserve the memory of certain historical figures, while at the same
time bringing their achievements into line with the philosophy of
Ingsoc. Various writers, such as Shakespeare, Milton, Swift, Byron,
Dickens, and some others were therefore in process of translation; when
the task had been completed, their original writings, with all else that
survived of the literature of the past, would be destroyed. These
translations were a slow and difficult business, and it was not expected
that they would be finished before the first or second decade of the
twenty-first century. There were also large quantities of merely
utilitarian literature---indispensable technical manuals and the
like---that had to be treated in the same way. It was chiefly in order
to allow time for the preliminary work of translation that the final
adoption of Newspeak had been fixed for so late a date as 2050.
